\chapter*{王小波年谱简编}

\footnote{年谱简编原载《王小波文集》第四卷;中国青年出版社,1999年}

◇ 1952年5月13日出生

    王小波出生于北京一个干部家庭。此时正值“三反”运动期间,家庭境况突发变故,这一突变对王小波的人生产生极大影响。他的名字“小波”就是这一事件的记录。

  父亲王方名原籍四川省渠县,逻辑学家,中国人民大学教授。1935年参加中国共产党领导下的学生运动,不久赴延安,转战至山东。50年代初任国家教育部干部。1952年被错划为“阶级异己分子”,1979年平反恢复党籍。母亲宋华为国家教育部干部,原籍山东省牟平县。

  王小波在全家五个孩子中排行老四,在男孩中排行老二。他的许多小说中主人公取名“王二”,或许并非偶然。大姐王小芹,二姐王征,兄王小平,弟王晨光。

◇ 1957年五岁

  父亲就逻辑学发表的系列文章引起较大反响。4月11日与周谷城等人一起受到毛泽东的接见。这件事对王小波的家庭状况、成长环境有一定影响。

◇ 1958年六岁

  “大跃进”运动给王小波留下了深刻印象,这可以从他的一些杂文和小说中看到。

◇ 1959年七岁

  9月入北京市二龙路小学读书。

◇ 1964年十二岁

  小学五年级时一篇作文被选作范文,在学校中广播。王小波对于小学语文老师对他写作能力的欣赏印象颇深,这位老师可以说是他写作生涯中的第一位“伯乐”。

◇ 1965年十三岁

  9月入北京市二龙路中学读书。

◇ 1966年十四岁

  上初一时“文化革命”开始,作家对这一运动的印象可以在《似水流年》等小说中看到。

◇ 1968年十六岁

  在云南兵团劳动,并开始尝试写作。这段经历成为《黄金时代》的写作背景,也是处女作《地久天长》的灵感来源。

◇ 1971年十九岁

  在母亲老家山东省牟平县青虎山插队,后做民办教师。一些早期作品如《战福》等就是以这段生活经历为背景写作的。

◇ 1973年二十一岁

  在北京牛街教学仪器厂做工人。后在北京西城区半导体厂做工人。工人生活是《革命时期的爱情》等小说的写作背景。 1977年二十五岁

  与在《光明日报》做编辑的李银河相识并恋爱。当时在王小波朋友圈中传阅的小说手稿《绿毛水怪》是二人相识的契机。

◇ 1978年二十六岁

  参加高考,考取中国人民大学,就读于贸易经济系商品学专业。大学期间在《读书》杂志发表关于《老人与海》的书评。

◇ 1980年二十八岁

  1月21日与李银河结婚。同年在《丑小鸭》杂志发表处女作《地久天长》。

◇ 1982年三十岁

  大学毕业后,在中国人民大学一分校教书。教师生活是《三十而立》等小说的写作背景。开始写作历经十年才完成面世的成名作《黄金时代》。

◇ 1984年三十二岁

  赴妻子就读的美国匹兹堡大学,在东亚研究中心做研究生。1986年获硕士学位。开始写作以唐传奇为蓝本的仿古小说,继续修改《黄金时代》。其间得到他深为敬佩的老师许倬云的指点。在美留学期间,与妻子李银河驱车万里,游历了美国各地,并利用1986年暑假游历了西欧诸国,这段经历在一些杂文和小说中可以看到。留学期间,父亲去世。

◇ 1988年三十六岁

  与妻子一道回国,任北京大学社会学所讲师。

◇ 1989年三十七岁

  9月出版第一部小说集《唐人秘传故事》,山东文艺出版社出版,原拟名《唐人故事》,“秘传”二字为编辑擅自添加,未征得作者同意。小说集包括五篇小说:《立新街甲一号与昆仑奴》、《红线盗盒》、《红拂夜奔》、《夜行记》、《舅舅情人》。 1991年三十九岁

  任中国人民大学会计系讲师。小说《黄金时代》获第13届《联合报》文学奖中篇小说大奖,小说在《联合报》副刊连载,并在台湾出版发行。获奖感言《工作·使命·信心》发表于《联合报)1991年9月16日第24版。这次获奖对王小波的写作事业起了鼓励作用。

  10月5日,《人民日报》海外版第4版报道了《黄金时代》获奖的消息。

◇ 1992年四十岁

  1月,与李银河合著的《他们的世界——中国男同性恋群落透视》由香港天地图书公司出版。 

  3月,《王二风流史》由香港繁荣出版社出版。收人三篇小说:《黄金时代》、《三十而立》、《似水流年》。

  8月,《黄金年代》(由于编辑的疏忽,“时代”一词误印为“年代”)由台湾联经出版事业公司出版。

  9月,正式辞去教职,做自由撰稿人。此时至去世的近五年间,写作了他一生最主要的著作。

  11月,与李银河合著的《他们的世界——中国男同性恋群落透视》由山西人民出版社出版。

  12月,应导演张元之约,开始写作同性恋题材的电影剧本《东宫·西宫》。

◇ 1993年四十一岁

  写作完成并曾计划将《红拂夜奔》、《寻找无双》和《革命时期的爱情》合编成《怀疑三部曲》,寻找出版机会。

◇ 1994年四十二岁

  7月,《黄金时代》由华夏出版社出版。收入五篇小说:《黄金时代》、《三十而立》、《似水流年》、《革命时期的爱情》、《我的阴阳两界》。

  9月,王小波作品《黄金时代》研讨会在华夏出版社召开,著名文学评论家及记者近二十人与会。

1995年四十三岁

  5月,小说《未来世界》获第16届《联合报》文学奖中篇小说大奖。

  7月,《未来世界》由台湾联经出版事业公司出版。

◇ 1996年四十四岁

  10月,妻子赴英国剑桥大学做访问学者。

  11月,杂文集《思维的乐趣》由北岳文艺出版社出版。

◇ 1997年四十五岁

  4月11日,因心脏病突发辞世。

  4月,妻子李银河发表悼文《浪漫骑士·行吟诗人·自由思想者——悼小波》。

  4月,与张元合著的电影剧本《东宫·西宫》在阿根廷国际电影节上获得最佳编剧奖。同年,电影《东宫·西宫》人围嘎纳电影节。

  4月26日,王小波遗体告别仪式在北京八宝山公墓举行。

  5月,《黄金时代》、《白银时代》、《青铜时代》由花城出版社出版,5月13日首发式于北京中国现代文学馆举行。

  5月,杂文集《我的精神家园》由文化艺术出版社出版。

  10月,《沉默的大多数——王小波杂文随笔全编》由中国青年出版社出版。

  10月,《沉默的大多数》由香港明镜出版社出版。

◇ 1998年

  2月,《地久天长——王小波小说剧本集》、《黑铁时代——王小波早期作品及未竟稿集》由时代文艺出版社出版。

◇ 1999年

  2月,《黄金时代》(上、下)、《白银时代》、《青铜时代》(上、中、下)由台湾风云时代出版公司出版。

  4月,《王小波文存》由中国青年出版社出版。

\footnote{以下是我添加的}

◇ 2007年3月8号

    英文版小说选集《Wang in Love and Bondage--Three Novellas by Wang Xiaobo》由 State University of New York Press出版,ISBN: 9780791470657 译者:华裔美籍女作家张洪凌和芳邦大学英语系教授、诗人Jason Sommer用了近六年时间将它们译成英文。


%% Last Modified: 2007/12/27 11:48:16
