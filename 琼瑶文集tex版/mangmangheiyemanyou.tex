\chapter{茫茫黑夜漫游}

\section{一} 

  现在是夜里两点钟;是一天最黑暗的时刻。我在给电脑编程序;程序总是调不通——我怀念早期的PC机,还有DOS系统。在那上面我要风得风,要雨得雨。现在的机器是些可怕的东西,至于win95,这是一场浩劫。最主要的问题还不在于技术进步,而是我老了,头脑迟钝,记忆力减退,才看过的东西就忘掉,得写在手上才成——手才是多大的地方。人的手腕上应接长两面蒲扇,除了可以往上写字,还可以扇风——我觉得浑身燥热。写这些事没有人爱看。我来讲个故事吧—— 

  有个美国的杂志的编辑,年龄和我现在相仿,也曾是个有才华的文学青年,但大好年华都消磨在杂志的运作里,不由他不长吁短叹。忽然老板闻进他的办公室,说道;“我们的订户数在下降!下期专访准备登什么?”他呈上选题,老板看了大怒说道;“就登这种没滋没味的东西?你在毁我的生意!现在人心不古,世道浇漓,亏你们坐得住!”我要的题目是这个——你给我亲自去采访!说完摔下张报纸就走了。编辑拣起来一看,是分类广告。上面红笔圈起来的广告内容实在有点惊世骇俗。编辑大叫一声:Oh my goodness!常听美国人这么嚷嚷,声音大得像叫驴,我不知道是什么意思,但不知道意思的话我也能喊出口来…… 

  你听音乐吗?我现在正听着。不知何时何地,有人用萨克管吹着一支怪腔怪调的布鲁斯,现在正有一搭没一搭地进到我耳朵里来。我的故事也是这样,它和我们的处境毫无关系。我是写小说的。知道我的人会说,我已经出了一本小说。那只是写出的一小部分。更多的都压着呢。为此就要去求人。主编先生很耐心地提出大量的修改意见,改完了还是不给出。有人当面对我说,看来你很有写作才能,但有些题材对你是不合适的。你何不写点都市题材的小说?既好卖一又不招惹是非……我不明白什么叫做都市题材。于是就耐心请教。别人就举了个例子《曼哈顿的中国女人》。有没有搞错啊?我住在北京,是男人,不是女人。另一个例子就是某香港女作家的作品。我的脸登时变作猪肝色。王二脾气发作了。有个庸俗的富婆,坐在奔驰车后座上瞎划拉几笔,就想当我写作的楷模?啊呀呸!……如你所知,我四十多岁了。也不能老是王二呀,所以我忍着。等到出了门—一你知道吗,口外的良马关中驴,关中的驴子比别处的大上一号。我像条关中大叫驴一样大叫起来:Oh my goodness! 

  这些事就扯到这里,不能忘记我的故事—一在老板摔下的报纸上,有些女孩声称自己有独特的性取向,寻求伴侣。这是个人欲横流的社会,无奇不有——我说这些,是要证明我也会装孙子。小说出不了,编程不顺利,现在我写点杂文。杂文嘛,大家都知道,写个小故事,凑些典故,再发点小议论。故事我会编,典故我也知道一些。至于教育意义吗,我不傻,好歹能弄出一个来——想采访这种事,就得打进去。编辑先生按广告上的通讯地址寄信去,声称自己正是被寻求的人,回信多是复印的纸条,上面写着:我们还不认识呢,请寄二十五美元来,我给你寄张照片,咱们加深一下了解,岂不是更好些……二十五美元寄去,相片寄来,再去信就不回了。很快他就攒了一抽屉稀奇古怪的相片,自己都不好意思了,在抽屉上加了三把锁。这些通信地址全是邮局的信箱,找都没处找。我以为登这些广告的不是所谓的金发女郎,也可能是老头,也可能是老太太,甚至是彪形大汉,见面会吓你一跳的。总之,全是拉丁美洲的移民,照片是低价买来的,这件事是他们的家庭副业,但这么一解释就没什么教育意义了。这不是人欲横流,而是某些层次低的人骗点小钱来花,这种事咱们这里也有…… 

  编辑先生对此另有理解,他发现S/M是这样一种生意:M是卖照片的,S是卖照片的。他就这么写成专访,交了上去。然后就发生了我很熟悉的事:稿子被枪毙了……看来他非得找着一个不卖照片的。去亲身体验一下才成——这位兄弟为此满心的别扭,他是虔诚的夭主教徒,每礼拜都要望弥撒,而且古板得要命。他的处境比我还坏,想到这一点蛮开心的。我很困。要睡了。故事下回再说吧……。 

\section{二} 

  “茫茫黑夜漫游”,这是别人小说的题目,被我偷来了。我讲这个故事,也是从别人那里抄的,既然大家都是小说家,那就有点交情,所以不能叫偷,应该说是借——我除了会写小说,还会写程序。三年前,有个朋友到我家里来,看了我的本领后说:哥们儿。你别写小说了,跟我来骗棒槌吧。现在棒槌很多,随便拿 DBASE写两句,就能弄着钱啊!所谓棒槌,就是外行的别名,这称呼里没什么恶意。我喜欢棒槌。尤其是可爱的女棒槌。我会尽心尽力地帮助她一—但我正觉得写小说很好,没和他去骗棒槌。 

  就在前两天,我又巴巴地去找这位朋友,求他给我点事做。朋友面有难色——他说,这个行当现在不好做。棒槌依旧很多,钱却没了。企业都亏损,没钱,个人不在软件上花钱,我听了这话就叹起气来你也许不知道,这世界上最叫人本忍看的事不是西子棒心,而是王二失意——平日很疯狂的一个人,一下就蔫得不成样子。朋友不忍看,就说:好吧,我给你找活。你自己先操练一下,本领要过硬——现在不是三年前了。我现在就在操练。你猜我发现了什么?我自己就是一根棒槌……仅仅三年,电脑就变成了这种鬼样子——从Intel公司到比尔·盖茨,全是一伙疯子! 

  现在我是根电脑棒槌,但我不以为自己会成一根小说棒槌。现在不会,将来也不会。永远都不会。这是我的终身事业,我时刻努力。这件事就不说了,还是讲我的故事吧:希腊医神说得好:这个人的美酒佳肴,就是那个人的穿肠毒药。就说这故事里的编辑吧,面临一项采访任务。我估计有些人接到这样的任务会兴致勃勃,但他完全是捏着鼻子在做。他在老板的逼迫之下继续着,看了无数无聊的小报,浪费了很多信纸,写了很多肉麻的信,起了很多身鸡皮疙瘩……终于联系上了一个。这一位没让他买照片,也没让他寄照片。而是直截了当地要求见面。编辑先生也想快点见面来完成他的专访,但是他想,这件事还是应该按S/M的套路来进行才对。用通信的方式约好了见面的方式约好了见面干什么,他又在市中心匿名租了一所房子,作为见面的地点。 

  然后,这个故事真正到了开始的时节:这位先生穿着黑色的皮衣、皮裤、皮坎肩,戴上皮手套和皮护腕,坐在空房子里等人。穿上这些衣服,可以驾飞机飞上寒冷的高空,也可以到北极去探险。有件事我忘了说了,这故事发生在七月份的纽约。那里又热又闷,他租的房子又没空调,但他不能不穿这些衣服,否则就没有气氛 ——所以只好起痱子。这位先生是一个真正的绅士,所以今晚要做的事也不能让他开心:他要把一位陌生的lady叫作一条worm——中文太难听了,只能写英文。还要把她图娜婚港来打她的屁股。他想,下回仟悔可有得说了。他觉得没滋没味,没情没绪,恨不能一头撞死。这也是我此时的感觉——我刚刚看了自己写出的程序,乱糟糟的像一锅豆腐渣,转起来七颠八倒,还常常死机。像这样的源码别说拿给别人看,自己留着都是种耻辱,赶紧删了算了。但是朋友要看我操练的结果,有点破烂总比没有要强…… 

  且把故事放到一旁,谈谈医神的这句话:此人之肉,彼人之毒。这是我所知道的最重要的至理名言。在美国,S/M就是很好的例子。有些人很喜欢,有些人很不喜欢。但对更大多数的人来说,它是无穷无尽的笑料。在美国我讲这个故事,听见的人都笑。在中国讲这个故事,听见的都不笑。还有人直愣愣地看着我说:你这个故事意义在哪里?倒能把我逗笑了。《生活》的朋友说,他们有四万读者。我总不相信这四万读者全是傻得愣瞪着双眼等待受教育的人、就算是阳,我也能想出一个来。所以接着讲吧:那位编辑先生穿着—身皮农,坐在空房子里。对面有个穿衣镜,他在里面打量着自己,觉得像个潜水员,只是没戴水镜,也没背氧气瓶。说句老实话,潜水员在岸上也不是这样的打扮。就在这时,有人按门铃。出去开门时;他在身上罩了件风衣——这是必要的,万一是有人走错门了呢。门廓里站着个很清纯的姑娘,没有化妆,身上穿着一件米黄色的风衣,她紧张得透不过气来……故事先讲到这里,容我想想它的教育意义。 

\section{三} 

  我年轻的时候,喜欢科学、艺术,甚至还有哲学。上大一时,读着微积分,看着大三的实变函数论,晚上在宿舍里和人讨论理论物理,同时还写小说。虽然哪样也谈不上精通,但我觉得研究这些问题很过瘾。我觉得每种人类的事业都是我的事业,我要为每种事业而癫狂——这种想法不能说是正常的,但也不是前无古人。古希腊的人就是这么想问题。假设《生活》读者都是这样的人;就可以省去我提供意义的苦难:在为科学或者艺术疯狂之余,翻开“晚生杂谈”,听听我这不着调的布鲁斯,也是满不错的——我知道作这种假设既不合道理,又不合国情。我的风衣口袋里正揣着两块四四方方很坚硬的意义,等到故事讲得差不多,就掏出来给你一下,打得你迷迷糊 糊,觉得很过瘾——我保证。我的故事里,有一个穿风衣的姑娘站在门廓里—— 

  编辑先生不敢贸然打招呼,生伯闹误会了。虽然他也想到了,七月底的傍晚,除了有重大的原故,谁也不会穿风衣。他自己不但穿着风衣,还穿了一双高腰马靴,靴根上带着踢马刺;手上戴着黑皮手套——他当然也有重大的原故。据此认为他不怕热是不对的,他不仅伯热,而且汗手汗脚,手心和脚心,现在一共有四汪水。此时他暗自下定了决心说,不管发生什么情况,今晚决不脱靴子。让人家闻见这股味儿不好——当然,他早忘了,这里没有“人家”,只有一条worm……他把手夹在腋下,但靴子是隐藏不住的。女孩看清以后,就钻了进来,脱下风衣挂在衣钩上。里面是黑皮短衣,不仅短,而且古怪。她不尴不尬地转过身来,打招呼道;你好。那男的想好了该说什么后,答道:你好,worm——说时迟,那时快,女孩扬起手来要给他个嘴巴。假如打着了的话,这故事就发生了重大的转折—— 谁是S,谁是M都得倒过来——但她及时想明白了,把手收回来,摸摸鼻子说,你好,大老爷,奴家这厢有礼了——这几句倒是中规中式,不但合乎S/M的礼仪,也和我们民族的文化传统暗暗相通。可惜她马上就觉得不自在;翻口道:叫蛆太难听了!咱们改改吧,你可以叫我小耗子。可以理解,谁都不想做昆虫的幼虫,都想做哺乳动物,这个要求本不过分,但我们的编辑先生从小到大痛恨一切啮齿类,所以硬下心来说道:不行。我又没逼你,是你自己要做蛆的。那女孩想了想,叹口气说,是吗?那好吧。但是,叫你大老爷,是不是太肉麻些了?那男的马上想说:好,你就叫我比尔吧——但他立朗想到,叫比尔怎么成呢,气氛就没有了,专访怎么写?于是硬下心来答道:不行!怎么这么罗嗦呢?不要忘了,你是条蛆呀!与此同时,他在心里记下:下回埋头工作忏悔时别忘了说,我对人家女孩子发横。主啊,原谅我吧。我也是为了新闻事业——这个人的毛病是顾虑太多,一点都不干脆…… 

  我有些编辑朋友,他们说,你也不能老这么不酸不凉的。文章要让一般读者能看懂,还要有教育意义。具体到我讲这个故事,教育意义就是:资本主义社会太黑暗,让有才华的文学青年去做无聊的专访,逼良为娼——好吧,我把砖头掏出来了。拍过了这一下,就可以接着讲故事了。说句实在话,我讨厌这个男主人公。他粘粘糊糊,满心的顾虑。至于我,过去是干脆的,现在也变得顾虑重重。一位报纸编辑告诉我说:兄弟,你是个写稿的人,不是载运死刑犯的囚车啊。别老写些让我们老总见了就毙的东西,拜托了……这是个合理的要求。对于我讲的故事,也该加些批判进去,让我自己也显得乖些。那美国编辑说,他是为了新闻事业。什么事业?男盗女娼的事业——唉。我自己也是个小说家。假如我真看不出来这个故事是别人编来逗笑的,还要一本正经的批判一番,那就象个傻×了。傻×就傻×吧,我现在已经很随和了。你可以叫我傻×,还甚至可以说我是worm,我都没意见,虽然我也想做个啮齿类。程序调不通,稿子又不肯好好写,我算个什么人呢。做人应该本分,像老舍先生生前说过的那样,多配合……只有一点我不明白。像这样活着,到底是为了什么呢。 

\section{四} 

  我年轻时,觉得一切人类的事业都是我的事业,我要拥有一切……如果那时能编程序,一定快乐得要死。顺便说一句,想要拥有一切时,我正在云南挖坑,什么都没拥有。假如有个人什么都想吃,那他一定是饿得发了慌。在现代,什么都想干的人一定是不正常。不管怎么说吧,我怀念那个时代。那是我的黄金时代。 

  现在我也在编程序,但感觉很不好。这说明我正在变成另外—个人,那种嚣张的气焰全没有了。关汉卿先生曾说,他是蒸不熟煮不烂碾不扁磨不碎整吃整屙的— 颗铜豌豆。我很赞赏这种精神,但我也知道,这样的豆子是没有的。生活可以改变一切。我最终发现,我只拥有一项事业,那就是写小说。对—个人来说,拥有一项事业也就够了……所谓小说,是指卡尔维诺、尤瑟纳尔等人的作品,不是别的,这两位都不是中国人,总提外国人的名字不好,人家要说我是民族虚无主义者。所以,所谓小说。乃卡威奴,尤丝拿之事也。这么一说;似乎实在得多了。像这样闲扯下去真是不得了,且听我讲这个故事吧。 

  那位编辑和—个陌生的女孩在门厅,寒喧过后,就到后面卧室里去。那女孩一路上东张西望,不停地打听:你就住在这儿吗?长住短住?你什么职业?喂喂,除了叫大老爷,你还叫什么呢?编辑先生感到很大的不快,想道:他妈的!我要做专访;可这到底是谁访谁啊?但他没有说出口来。他只是板起脸来说道:不要叫我 “喂喂”,该叫我什么你知道。你是个什么也别忘了……那女孩吐吐舌头说,好吧,我记住。等会儿我当完了worm,你可要告诉我啊。这位编辑登时有种毛骨抹然的感觉。座山雕在威虎山见了杨子荣也有这种感觉,这个土匪头子是这么表达:你不是个溜子,是个空子!但编辑没说什么?他只是想着:上帝啊,保佑我的专访吧!让我有东西向老板交差!……我就不信专访有这么重要。所以,他说的“专访”,应该理解为“饭碗”才对。在饭碗的驱使之下,他把那女孩引到了卧室里;这问房子挂着黑布窗帘,点着一盏昏黄的灯。这里静得很,因为这所房子在小巷里。除此之外,编辑先生亲自动手,把窗缝都封上了。房子中央放着一张黑色的大铁床。到了这个地方,女孩变得羞答答的。而那个编辑也有点扭捏。他干咳了一声,从背后掏出一把手铐——这是一件道具。女孩的脸涨得通红,她盯着他说:喂喂!有必要吗?真的有必要吗?那个男人臊得要死,但还是硬下心来说:什么必要不必要的!我也不叫做“喂喂”!别忘了,你只是一条蛆!整个故事里就是这句话最重要。在生活里,也就是这句话我老也记不住。 

  塞利纳杜撰了一首瑞士卫队之歌; 

  我们生活在漫漫寒夜,   —人生好似长途旅行。    仰望天空寻找方向,   天际却无引路的明星! 

  我给文章起这么个名字,就是因为想起了这首歌;我讲的故事和我的心境之间有种牵强附会的联系,那就是:有人可以从屈服和顺从中得到快乐,但我不能。与此相反,在这种处境下,我感到非常不愉快。近几年认识了一些写影视剧本的作者,老听见他们嘀咕:怎么怎么一写,就能拍。还提到某某大腕,他写的东西都能拍。我不喜欢这样的嘀咕,但能体谅他们的苦衷,但这种嘀咕不能钻到我脑子里来。人家让我写点梁风仪式的东西,本是给我面子,但我感到异常的恼怒。话虽如此说,看到梁凤仪—捆捆地出书,自己的书总出不来,心里也不好受。那个写的东西全能拍的大腕。他是怎么想的呢……在我的故事里,那个女孩摸摸羞红的鼻子(现在不摸一会儿就模不到了),把手伸了出来,被铐到了床栏上;这是一种S/M套路。不要问我现在陷到什么套路里了,我不知道——我也想当个写什么都能拍或者登的大腕,但不愿把手伸出来,让别人铐住;其实我也是往自己脸上贴金:有谁稀罕铐我来呢。 

  在我的故事里,那个男编辑把牙齿咬得格格乱殉,猛然闭上限睛,挥起戴着黑手套的左手(这是因为位置的关系,他不是左撇子),劈里啪啦,连打了二十多下;必须给人类的善良天性以适当的评价——二十多下多数都打到床垫上了。在此说句题外之语,我也不喜欢拿教育意义去拍别人,打完以后睁眼一看,那女孩挣得满脑通红,趴在床上浑身颤抖。假如是在哭,那人必定会为此难受。实际上是在笑,所以他感觉更糟。他满身都是臭汗,皮衣底下很是枯稠。左手在抽筋,左臂又像脱了臼。所以他不管三七二十一,转身向酒柜扑去。首先,他练了特大号的杯子,往里面加满了冰炔,然后先灌满汽水,再加一小点杜松子酒,正准备一口全喝下去,忽听身后有响声。回头—看:那个女孩挣扎着跪在了床上,扭着脖子看他,眼睛瞪得比酒杯还大。两人这样对视了一会儿;那女孩说:别光顾你自己喝啊!那人想,她说得对,就把酒杯放下,问道:你喝什么?女孩说:苏格兰威士忌。黑牌的,加两块冰。他转身去拿酒——顺便说一句,这编辑是个会享受的人,酒柜里什么都不缺———面倒酒;他一面唠叨着;苏格兰酒。黑牌的。加两块冰。这可不像是一条蛆的要求呀…… 

  又到了夜里两点多钟,看来,电脑这个行当我是弄不下去了,Win3.1刚会弄,又出来了win95。BC4.5刚会写;又出来了5.0。像这样花样翻新,好像就是为了让我头晕;只有一件事不让我头晕,那就是小说。在此必须澄清—种误会:好像小说人人都能写,包括坐在奔驰车后座上的富婆……小说不是这样轻松的事业。要知道卡尔维诺从中年开始,一直在探讨小说艺术的无限可能。小说和计算机科学一样,确实有无限的可能。可惜我没有口才,也没有耐心说服我的主编先生。对我来说;只有一种生活是可取的,就是迷失在这无限的可能性里。这种生活可望而不可及。现在我的心情就像那曲时断时续,鬼腔鬼调的布鲁斯……但是,我说这些干什么呢?逗主编先生笑吗?“还小说艺术的无限可能呢你。你不就是那个王二吗?” 

  现在还是来讲这个故事吧。那个编辑端了酒,朝女孩走去。她挣扎着想接过这杯酒,但是不可能……于是,他很温柔地揽住她的肩头,把酒喂到她唇边——同时下意识地数落道:苏格兰酒。黑牌的。不多不少,两块冰。可你不是一条蛆吗?那女孩马上就喝呛着了。她浑身颤抖着说:你就别提这个字了……我说过的吧,这故事编出来;就是为了博人一笑。我的动机也是如此。我说自己兜里揣着两块教育意义,随时可以掏出来,这是吹牛皮。要真有这样的本领,我就不编程序了,不追求教育意义的读者一定已经猜到了故事的结局:那个男的掏出钥匙来,打开了手铐,打着哈哈说:对不起。我不是真的——我是个报纸的编辑,出来找写文章的材料。那女孩揉着手腕说:对不起。我也不是真的;我是个社会学家,做点社会调查。笑过了以后,两人换上凉快衣服,—起出门找凉快地方去喝咖啡。在我自己的故事里,出版社的总编给我打电话说,那天你在门外吼什么呀你?开个玩笑嘛,你怎么拔腿就跑了……快回来。稿子的事还没谈完呢。唉。我的故事要是真能这样讲,那就好了。 

  故事已经讲完了。还有一点需要补充的,这个故事拿S/M“搞笑”,但我对有这种嗜好的人不存偏见。可笑的是,既不是这种人,又不是这种事,还要这么搞。现在我揉揉眼睛,振奋起精神,退出写文章的程序。发了些牢骚,心情好多了。 

  我觉得我还是我,我要拥有一切——今天要是不把那段C++程序调通,老子就不睡了…… 

 
