\chapter{猫}

下午,我回家的时候,看到地下室窗口的栅栏上趴着一只洁白的猫。它好象病了。我朝它走去时,它背对着我,低低的伏在那里,肚子紧紧的贴着铁条。我还从来没有见到过猫会这么谨小慎微的趴着,爪子紧紧的扒在铁条上。它浑身都在颤抖,头轻微的摇动着,耳朵在不停的转动,好象在追踪着每一个声响。 

它听见我的脚步声,每次我的脚落地都引起它的一阵痉挛。猫怕的厉害,可是它不逃走,也不转过头来。风吹过时,它那柔软的毛打着旋。一只多么可爱的猫啊。 

我走到它的前面时,才发现有人把它的眼睛挖掉了。在猫咪的小脸上,有两道鲜红的窄缝,血还在流,它拼命的往地下缩,好象要把自己埋葬。也许它想自杀?总之,这只失去眼睛的猫,显得迟迟疑疑。它再也不敢向前迈出一步,也不敢向后迈出一步。它脸上那两道鲜红的窄缝,好象女人涂了口红的嘴巴。我看了一阵子就回家了。 

我回到家里,家里空无一人。没看到那只猫以前,我觉得很饿,心里老想着家里还有一盒点心,可是现在却一阵阵的泛恶心。此外,我还感到浑身麻木,脑袋里空空荡荡,什么念头也没有。 

外边的天空阴沉沉的,屋里很黑。但是通往阳台的门打开着,那儿比较明亮。我到阳台上去,往下一看,那只猫不知什么时候爬到了栅栏平台的边上,伸出前爪小心翼翼的往下试探。栅栏平台离地大约有20厘米,比猫的前腿长不了多少。它怎么也探不到底,于是它趴在那里久久的试探着,它的爪子就象一只打水的竹篮。我站在那儿,突然感到一种要从三楼上跳下去的欲望。我回屋去了。 

天快黑的时候,我又到阳台上去。在一片淡蓝色的朦胧之中,我看见那只猫还在那里,它的前爪还在虚空中试探。那道半尺高的平台在那只猫痛苦的感觉之中一定被当作了一道可怕的深渊。我不知道它为什么不肯放弃那个痛苦而无望的企图。后来它昂起头来,把那鲜血淋淋的空眼眶投向天空,张开嘴无声的惨叫起来,我明白它一定是在哀求猫们的好上帝来解救它。 

我小时候也象它一样,如果打碎了什么值两毛钱以上的东西,我害怕会挨一顿毒打,就会把它的碎片再三的捏在一起,在心里痛苦的惨叫,哀求它会自动长好,甚至还会把碎片用一张旧报纸包好,放在桌子上,远远的躲开不去看。我总希望有什么善神会在我不看的时候把它变成一个好的,但是没有一次成功。 

现在那只猫也和我小时侯一样的愚蠢。它那颗白色的小脑袋一上一下的摆动着。正是痛苦叫它无师自通的相信了上帝。 

夜里我睡不着觉,心砰砰直跳,屋里又黑的叫人害怕。我怎么也想不出人为什么要挖掉猫的眼睛。猫不会惨叫吗?血不会流吗?猫的眼睛不是清澈的吗?挖掉一只之后,不是会有一个血淋淋的窟窿吗?怎么能再挖掉另一只呢?因此,人要怎么才能挖掉猫的眼睛?想的我好几次干呕起来。我从床上爬起来,走到阳台上去。下边有一盏暗淡无光的路灯,照见平台上那只猫,它正沿着平台的水泥沿慢慢的爬,不停的伸出它的爪子去试探。它爬到墙边,小心的蹲起来,用一只前爪在墙上摸索,然后艰难万分的转过身去,象一只壁虎一样肚皮贴地地爬回去。它就这么不停的来回爬。我想这只猫的世界一定只包含一条窄窄的通道,两边是万丈深渊而两端是万丈悬崖,还有原来是眼睛的地方钉着两把火红的铁钎。 

凌晨三点钟,那只猫在窗前叫,叫的吓死人的可怕。我用被子包住了脑袋,那惨叫还是一声声传进了耳朵里来。 

早上我出去的时,那只猫还趴在那儿,不停的惨叫,它空眼窝上的血已经干了,显得不那么可怕,可是它凄厉的叫声把那点好处全抵消了。 

那一天我过的提心吊胆。只觉得天地昏沉,世界上有一道鲜红的伤口迸开了,正在不停的流血。人在光天化日之下干出了这件暴行,可是原因不明,而且连一个藉口都没有。 

我只知道有一种现成的藉口,那就是这是猫不是人,不过就是这么说了,也不能使这个伤口结上一层疤。 

下午下班回家的路上,我又想起几件令人毛骨悚然的事来,什么割喉管、活埋之类。干这些事情时,都有它的藉口,可是这些藉口全都文不对题,它不能解释这些暴行本身。 

走到那个平台时,我看到那只猫已经死了,它的尸体被丢到墙角里,显得比活的时候小的多。我长长的出了一口气,身上觉得轻松了许多。早上我穿了件厚厚的大棉袄,现在顿时觉得热得不堪。我一边脱棉袄一边上楼去,嘴里大声吹着口哨。我的未婚妻在家里等我,弄了好多菜,可是我还觉得不够,于是我就上街去买啤酒。 

我提着两瓶啤酒回来,路过那个平台时,看到那只猫的幻影趴在那儿,它的两只空眼眶里还在流着鲜血,可怜的哆嗦着。我感到心惊肉跳,扭开头蹑手蹑脚地跑过去。 

上楼梯的时候,我猛然想起有一点不对。死去的那只猫是白色的,可是我看见的那个幻影是只黄猫。走到家门口时,我才想到这又是一只猫被挖掉了眼珠,于是我的身体剧烈的抖动起来。 

我回到家里,浑身上下迅速地被冷汗浸透了。她问我是怎么回事。我没法向她解释,只能说我不舒服。于是她把我送上床去,加上三床被子,四件大衣。 她独自一人把满桌菜都吃了,还喝了两瓶啤酒。 

夜里那只猫在惨叫,吓的我魂不附体。我又想起明朝的时候,人们把犯人捆起来,把他的肉一片一片的割下来,割到没有血的时候,白骨上就流着黄水,而那犯人的眼睛还圆睁着。 

以后,那个平台上常常有一只猫,没有眼睛,鲜血淋漓。可我总也不能司空见惯。我不能明白这事。人们经过的时候只轻描淡写的说一声:“这孩子们,真淘气。”据说这些猫是他们从郊外捉来的。 我也曾经是个孩子,可我从来也没起过这种念头。在单位里我把这件事对大家说,他们听了以后也那么说。只有我觉得这件事分外的可怕。于是我就经常和别人说起这件事。他们渐渐的听腻了。有人对我说:“你这个人真没味儿。” 

昨天晚上,又有一只猫在平台上惨叫。我彻夜未眠,猛然想到这些事情都不是偶然的,这里边自有道理。 

当然了,一件这样频繁出现的事情肯定不是偶然的,必然有一条规律支配它的出现。人们不会出于一时的冲动就去挖掉猫的眼睛。支配他们的是一种力量。 

这种力量也不会单独的出现,它必然有它的渊源,我竟不知道这渊源在哪里,可是它必然存在。 可怕的是我居然不能感到这种力量的存在,而大多数人对它已经熟悉了。也许我不了解的不单单是一种力量,而是整整的一个新世界?我已经觉到它的存在,但是我却不能走进它的大门,因为在我和它之间隔了一道深渊。我就象那只平台上的瞎猫,远离人世。 

第二天早上,我出去时那一只猫已经死了。但平台上不会空很久的。我已经打定了主意。 

我背着书包,书包里放着一条绳子和一把小刀。我要到动物收购站去买一只猫来。当我把它的眼睛挖掉送上平台时,我就一切都明白了。 

到那个时候,我才真正跨入人世。

%% 我以为我明白了,真的么?
%% Last updated: 2008|05|17  20:19:47 by Van Tae
