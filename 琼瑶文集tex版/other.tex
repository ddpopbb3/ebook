\chapter{浪漫骑士}

作者:李银河 

本文又名:《浪漫骑士 行吟诗人 自由思想家——悼小波》 

日本人爱把人生喻为樱花,盛开了,很短暂,然后就凋谢了。小波的生命就像樱花,盛开了,很短暂,然后就溘然凋谢了。 

三岛由纪夫在《天人五衰》中写过一个轮回的生命,每到18岁就死去,投胎到另一个生命里。这样,人就永远活在他最美好的日子里。他不用等到牙齿掉了、头发白了、人变丑了,就悄然逝去。小波是这样,在他精神之美的巅峰期与世长辞。 

我只能这样想,才能压制我对他的哀思。 

在我心目中,小波是一位浪漫骑士,一位行吟诗人,一位自由思想者。 

小波这个人非常的浪漫。我认识他之初,他就爱自称为“愁容骑士”,这是唐吉诃德的别号。小波生性相当抑郁,抑郁即是他的性格,也是他的生存方式;而同时,他又非常非常的浪漫。 

我是在1977年初与他相识的。在见到他这个人之前,先从朋友那里看到了他手写的小说。小说写在一个很大的本子上。那时他的文笔还很稚嫩,但是一种掩不住的才气已经跳动在字里行间。我当时一读之下,就有一种心弦被拨动的感觉,心想:这个人和我早晚会有点什么关系。我想这大概就是中国人所说的缘分吧。 

我第一次和他单独见面是在《光明日报》社,那时我大学刚毕业,在那儿当个小编辑。我们聊了没多久,他突然问:你有朋友没有?我当时正好没朋友,就如实相告。他单刀直入地问了一句:“你看我怎么样?”我当时的震惊和意外可想而知。他就是这么浪漫,率情率性。 

后来我们就开始通信和交往。他把情书写在五线谱上,他的第一句话是这样写的:“作梦也想不到我会把信写在五线谱上吧。五线谱是偶然来的,你也是偶然来的。不过我给你的信值得写在五线谱里呢。但愿我和你,是一支唱不完的歌。”我不相信世界上有任何一个女人能够抵挡如此的诗意,如此的纯情。被爱已经是一个女人最大的幸福,而这种幸福与得到一种浪漫的骑士之爱相比又逊色许多。 

我们俩都不是什么美男美女,可是心灵和智力上有种难以言传的吸引力。我起初怀疑,一对不美的人的恋爱能是美的吗?后来的事实证明,两颗相爱的心在一起可以是美的。我们爱得那么深。他说过的一些话我总是忘不了。比如他说:“我和你就好像两个小孩子,围着一个神秘的果酱罐,一点一点地尝它,看看里面有多少甜。”这形象的天真无邪和纯真诗意令我感动不已。再如他有一次说:“我发现有的人是无价之宝。”他这个“无价之宝”让我感动极了。这不是一般的甜言蜜语。如果一个男人真的把你看作是无价之宝,你能不爱他吗? 

我有时常常自问,我究意有何德何能,上帝会给我小波这样一件美好的礼物呢?去年10月10日我去英国,在机场临分别时,我们虽然不敢太放肆,在公众场合接吻,但他用劲搂了我肩膀一下作为道别,那种真情流露是世间任何事都不可比拟的。我万万没有想到,这一别竟是永别。他转身向外走时,我看着他高大的背影,在那儿默默流了一会儿泪,没想到这就是他给我留下的最后一个背影。 

小波虽然不写诗,只写小说随笔,但是他喜欢把自己称为诗人,行吟诗人。其实他喜欢韵律,有学过诗的人说,他的小说你仔细看,好多地方有韵。我记忆中小波的小说中唯一写过的一行诗是在《三十而立》里:“走在寂静里,走在天上,而阴茎倒挂下来。”我认为写得很不错。这诗原来还有很多行,被他划掉了,只保留了发表的这一句,小波虽然以写小说和随笔为主,但在我心中他是一位真正的诗人。他的身上充满诗意,他的生命就是一首诗。 

恋爱时他告诉我,16岁时他在云南,常常在夜里爬起来,借着月光用蓝墨水笔在一面镜子上写呀写,写了涂,涂了写,直到整面镜子变蓝色。从那时起,那个充满诗意的消息、云南山寨中皎洁的月光和那面涂成蓝色的镜子,就深深地印在了我的脑海中。 

从我的鉴赏力看,小波的小说文学价值很高。他的《黄金时代》和《未来世界》两次获联合报文学大奖,他的唯一一部电影剧本《东宫西宫》获阿根廷国际电影节最佳编剧奖,并且该影片成为1997年戛纳国际电影节入围作品,使小波成为在国际电影节为中国拿到最佳编剧奖的第一人,这些可以算作对他的文学价值的客观评价。他的《黄金时代》在大陆出版后,很多人都极喜欢。有人甚至说:王小波是当今中国小说第一人,如果诺贝尔文学奖将来有中国人能得,小波就是一个有这种潜力的人。我不认为这是溢美之辞。虽然也许其中有我特别偏爱的成分。 

小波的文学眼光极高,他很少夸别人的东西。我听他夸过的人有马克·吐温和萧伯纳。这两位都以幽默睿智著称。他喜欢的作家还有法国的新小说派,杜拉斯,图尼埃尔,尤瑟纳尔,卡尔维诺和伯尔。他不喜欢托尔斯泰,大概觉得他的古典现实主义太乏味,尤其是受不了他的宗教说教。小波是个完全彻底的异教徒,他喜欢所有有趣的、飞扬的东西,他的文学就是想超越平淡乏味的现实生活。他特别反对车尔尼雪夫斯基的“真即是美”的文学理论,并且持完全相反的看法。他认为真实的不可能是美的,只是创造出来的东西和想象力的世界才可能是美的。他有很多文论都精辟之至,平常聊天时说出来,我一听老要接一句:不行,我得把你这个文论记下来。可是由于懒惰从来没真记下来过,这将是我终身的遗憾。 

小波的文字极有特色。就像帕瓦罗蒂一张嘴,不用报名,你就知道这是帕瓦罗蒂,胡里奥一唱你就知道是胡里奥一样,小波的文字也是这样,你一看就知道出自他的手笔。台湾李敖说过,他是中国白话文第一把手,不知道他看了王小波的文字还会不会这么说,真的,我就是这么想的。 

有人说,在我们这样的社会中,只出理论家、权威理论的阐释者和意识形态专家,不出思想家,而在我看来,小波是一个例外,他是一位自由思想家。自由人文主义的立场贯穿在他的整个人格和思想之中。读过他文章的人可能会发现,他特别爱引证罗素,这就是所谓气味相投吧。他特别崇尚宽容、理性和人的良知,反对一切霸道的、不讲理的、教条主义的东西。我对他的思路老有一种意外惊喜的感觉。这就是因为长这么大,满耳听的不是些陈词滥调,就是些蠢话傻话,而小波的思路却总是那么清新。这是一个他最让人感到神秘的地方。我分析这和儿时他的家庭受过挫折有关。这一遭遇使他从很小就学着用自己的判断力来找寻真理,他就找到了自由人文主义,并终身保持着对自由和理性的信念。 

小波在一篇小说里说:人就像一本书,你要挑一本好看的书来看。我觉得我生命中最大的收获和幸运就是,我挑了小波这本书来看。我从1977年认识他,到1997年与他永别,这20年间我看到了一本最美好、最有趣、最好看的书。作为他的妻子,我曾经是世界上最幸福的人;失去了他,我现在是世界上最痛苦的人。小波,你太残酷了,你潇洒地走了,把无尽的痛苦留给我们这些活着的人。虽然后面的篇章再也看不到了,但是我还会反反复复地看这20年。这20年永远活在我心里。我相信,小波也会通过他留下的作品活在许多人的心里。 

樱花虽然凋谢了,但它毕竟灿烂地盛开过。 我最最亲爱的小波,再见,我们来世再见。到那时我们就可以在一起一百年,一千年,一万年。再也不分开了。

\chapter{J的外地之行}


  我们\footnote{本节由王小波所写.}在写这篇书稿时, 接到一位朋友从外地打来的电话。这个朋友叫J,曾经给过我们很多帮助。这次他去了南方,访问了很多同性恋朋友。他听说我们在写书,就主动提出回来后要接受我们的访谈。以下是访谈记录。

J说这一次去了不少地方。 我走沿海下去,沿京广线回来,路上到处有。逛了两个多月,身上带了三千块钱,都花光了。 
 
   我觉得J的行为可说是摩顶放踵, 奔走天下。我和他相交久了,觉得他有一点古之大陕墨子的气质。 只不过墨子奔走四方是在实行非攻的主张,而J是在寻找肛交的对象。 除此之外,处处都像了。比如墨子主张兼爱无等差,J就是这样。别人说J有点疯,逮著谁就要和谁干。 
 
   我问:到处你都能找到吗? 
 
   也不是。往山东就没找到。要在一个陌生的地方找到同性恋接头地点,必须事先有个线索。比如,我先知道了一个点,找到了几个人聊,他们会告诉我哪里还有。假如一点头绪都没有,那就难了。但是假如住下去,早晚能找到。这回在济南,别人告诉我XX湖公园里有,我进去转了三圈找不到人,别的地方又不知道,就抓瞎了。 
 
   看来XX湖这个消息是错的? 
 
   也不一定。 当时XX湖公园正办荷花展,入场卷涨到了3块钱。进去转一圈,找个朋友, 还不知能不能找到呢,先掏3块,换了我是济南人,我也不干。何况在里面我发现了宣传画…… 
 
   什么叫宣传画? 
 
   就是画在厕所墙上的画。和一般的画不一样,不画女人的,一看就知道--所以这里过去肯定是个点,可能是被荷花展冲了。北京也有这种情况,原来某厕所是,后来改收费了,就没人去了。 
 
   我看收费厕所很干净,收点费也不多,不是挺好吗? 
 
   这你就外行了。第一,收费厕所门口有人,出来进去招人眼目,多有不便。第二,有些收费厕所有隔板,不像一般厕所大通铺式的茅坑,便于大家一见目成。我们找朋友,上下都要看。隔扳固然讨厌,抽水马桶更叫入难受--什么都看不见了。或外有在收费厕所里的,那是因为他们那里厕所全是那样。从隔板上面探头探脑。多不方便哪。我还发现一类地点大有前途,就是大饭店的厕所。水磨石地板反光,正好看要紧的部位…… 
 
   好啦,谢谢你,我已经明白了。还有一个问题。这类公共厕所气味不好,你发现了没有? 
 
   没什么不太好罢。 
 
   你这么说我恕难苟同!比方说,现在这种天气(时值仲夏),环卫部门在厕所里创收,放上大塑料罐收集尿,做尿激酶。30多度气温一蒸,简直要命…… 
 
   我知道,有点杀眼睛。你说怎麽办?戴上防毒面具?再说,接头的地点和玩的地点不在一处,一般是相邻的两个地点(所谓地点,是指厕所)。一个好找,是接头用的,一个僻静,是玩的。后一种地点门口两辆自行车,里面就有了。后一个地点的卫生往往好一点。 
 
   接下去我又去了青岛,也没找到。后来听说,当地的同性恋地点有季节性,我找的是冬季地点。又听说一个浴室里有,我在里面呆了半天,几乎中了署。大概也是冬季地点罢,没找到。这是七月底的事。八月一号到了上海,找到了。此后我总是问好了下一站的地点再动身,再也没有失去同性恋朋友的联系。 
 
   到上海那天是8月1日。听说某某报栏前有,下午去了,隔着马路看了一眼,没有见到。当时陪阗个朋友,不便过去看。晚上去外滩,在北京就听人说那里有。到那里一看,异性恋谈恋爱的、同性恋扎堆聊天的都有,和北京大不相同。 
 
   哪里不同呢? 
 
   都不是正经找伴,聚在一趣瞎聊,时髦青年目无余子,像我这样年龄大点穿着一般的,没人注意。说实在的,我也看不大上那些人。后来遇上一个中年男子,挺朴实的,我喜欢。迎头擦肩而过时他看我,过去后我回头他也回头,肯定是了。我就让和我在一起的朋友等一下,自己去和他聊…… 
 
   暂停!你说有个朋友和你在一起。他是谁? 
 
   忘了告诉你了。这回到南方去,一方面我要找朋友,另一方面有个德国老太太要去香港,我陪她到厦门,我玩我的,她逛她的,两不耽误。我找到伴,让她等等我,她也不问我干嘛去了。德国人嘛,最拘谨了。不是自己的事从不打听。 
 
   你为什么要带着她? 
 
   不为什么。老太太一个人到南方不方便,学学雷锋罢。我让老太太等等我,就去找那人聊天。 他说是山西X市来出差的,听说这里多,来看看。还说,这儿怎么都是这样的人,太叫人失望。我听了大觉投缘,就和他握握手。我们互留了地址,我告诉他,还有人等我,让他稍候。然后我送老太太到了饭店,又回来找他。一块到XX饭店楼上的厕所,做了爱。 
 
   你真忙呀! 
 
   那天就忙到这里。第二天,我叫老太太自己去玩,我又去那个报栏。仔细一看,果然有奥妙。有人在看报,有人在互相挤。我一站下,就有人来挤我。稍微一打量,那人不好看,赶紧躲开。站到了另一边。这回是两面夹击。我的妈,都这么爱我! 
 
   我提个问题。假如我正好去看报,他们也会来挤我吗? 
 
   你别把问题说到这么绝对。同性恋之间的试探是一步一步的。打个比方罢,在公共场所,有人踩了你一脚,你怎么办? 
 
   我说:"喂,硌不硌呀!"或者"哥们儿!该减肥了!"怎么样? 
 
   不怎么样。别人踩你一下,可能是个信号。你可以看他一眼,要是喜欢,就嫣然一笑,不喜欢就走你的,说这么多难听的干嘛?像你这么生性的人,也没人来踩你挤你。 
 
   那我就放心了。接着讲吧。 
 
   其实,同性恋者打招呼,并不踩脚。这太有侵略性。我这只是打个比方。言归正传,后来有人摸我屁股。我没有像你那样,所以那人胆就壮了。到一定火候,我走开,他跟上来。这回是个上海人。机关干部。也是个挺朴实的人。 
 
   J说, 他喜欢朴实的人,这大概是一种偏爱。其他人也有喜欢小白脸的,喜欢有名的人的。但是我又发现同性恋者中,喜欢民工,喜欢农民的倾向非常普遍。这些农民要么不是同性恋,要么根本没有任何性经历。去年冬天我访过一位中学生,才17岁,他就有和民工交往的经历,而且明确表示喜欢这样的人。据说这些人没有任何太大的毛病,只有纯净的急待发泄的性欲。 
 
   J还说, 有一些人贫嘴聊舌,实在讨厌。有一回在北京和一位小伙子做爱,情绪太激烈, 把他的胸口抓破了。完了事正要分手,那人说道 :你就这么走了吗?我说:真奇怪,我不走,你要请我吃饭吗?他说:你没看见,把我胸口都抓破了。我说:那你要怎么样?他说:你还不得意思意思。我说怎么意思。他说:给盒烟罢。我说:你看,我不抽烟,身上也没钱。他说,把你地址留下来。我把地址留给他了。当然,是假地址。事后好几天都觉得恶心。 
 
   J说, 他和那个上海人聊了一会儿,就把他带到饭店里做了爱。和朴实的人在一起,总是很愉快的。那人说:你在上海要小心。上海有一些人很不好,要钱,骗人等等。J说他在上海倒没遇上骗子,但是很不好的人可是遇到了。 
 
   8月4号临离开上海, 他又到报栏去。上午去时,衣着不好,没有 理睬。下午弄了几件好衣服再去,就有人理了。这回是几个小伙子。于是聊起来。对方问:你是哪里来的?回答说:北京的。对方就说:北京人好,北京人大方,豪爽,仗义疏财等等,说得他都有点不好意思了。 
 
   我说道,你看,咱们北京人名声在外,历代帝王之都,天下首善之区,到底不同寻常。J说:你以为他们是真心仰慕北京人?不是的。这是灌米汤。灌完了之后:你住哪家饭店?答曰:XX饭店。对方说:没听说过。房间里有彩电吗?热水可是全天供应?你住几人间?都问完了,说:走,咱们去你那儿玩玩。 
 
   我说道: 你这几个朋友,大概是旅游局的,所以问那么仔细。J说:第一,他们不是我的朋友;第二,他们问这些不是关心旅游事业的发展,而是要从我信的地方判断我有多少钱。只有上海人有这样的心眼。他们说要上我那里玩,我说:好,去罢。走了几步,有一个小伙子说:就这么去呀!我说,还怎么去。他说打个的。我说,没钱。他们说,住那么好的饭店,没钱坐taxi。我说可不是嘛。于是一路坐电车去。半路上电车停电,下来了。他们又说要打的。我说打什么,走罢,练练腿。那三个人就有点不想走的意思。这时有个小伙子去买冰棍,我拿起来就吃。他说,这冰棍是我买的。我说:谢谢。他说:光谢谢是不够的。呆会儿请吃饭时,你可得给我加个菜。我说:谁说要请你吃饭。他们三人一起说道:呀,你不请我们吃饭哪!那让我们到哪儿吃饭?我说:你们回家吃去!他们听了这话,立刻身后转,拔腿跑掉了。妈的,什么东西! 
 
   J说, 这几个都是二十一二岁,年纪轻轻就不学好。我以为他说的未必对。就我观察到的同性恋者,在感情方面,也有若干区别。有一种人,江湖气浓,颇有侠风,对待同伴有一种四海之内皆兄弟的意思。还有一种人,带有很深的自恋倾向,希望成为别人注意的中心,得到别人的宠爱,要别人意思意思,倒不一定是贪财。就如时下一些美丽的女郎,与男友出行,总要吃东喝西,也不是饿了渴了,而是给别人一些机会,来表示对她的重视。但是假如这两种人走到一起,就会发生误会。如果细分,同性恋社群里可以分出多种类型。可当我把这番道理讲给J听时,他说:去他们的罢,我不希罕! 
 
   8月5日,J到了苏州。白天陪德国老太太去虎丘玩。晚上六七点,出去找朋友。在上海听说苏州XX公园,XX报栏前都有。他去了其中一处,正在徘徊,马路上有一位骑车的中年男子, 带着孩子经过。一看J的样子,连忙下来,叫孩子一边玩,走过来说,你怎么还在这里?这里不安全了。亚运期间警察来捉过,现在大家都去XX路XX桥。你一个人在这里特别扎眼。J说:这个朋友真热心,让人感动! 
 
   我也想起过去在美国时,有一次开夜车,实在累了,在路边休息一会儿。当时停在一个商场的门前。商场守夜的老头跑出来,先问是不是要帮忙,然后请我们去喝咖啡。啊,漫漫人生路,就如不尽的寒夜,别人的一点好意,就如夜空中一点晨星!…… 
 
   J说, 我当时也极感动,但是感动中有一点内容与你不一样--我想和他做爱。通过了姓名地址,就一块儿走了,找到体育场的厕所,见四下没人,叫小孩在外面玩,我们就在里面玩。和他分手,我去了XX路XX桥。也是个厕所。进去一看,里面有几个人,出来时跟出来一个,像新疆人,自己也说是新疆人,其实不是,连乌鲁木齐在哪里都不知道。我问他到哪里玩,他带我去了一个机关大院里的厕所。蚊子多极了,真能把人咬死。和他分了手,又回原来那个地方。遇上了一个老先生。有五十多了。老先生里被动的多。这一回没玩,刚玩了两次,有点不成了。这老先生说是姐夫教的。那还是年轻时的事 
 
   这个姐夫太不像话!和小舅子干这种事,把太太置于何处?夫人不知道吗? 
 
   当然不知道。小地方的女人,像傻子一样。我这次做了一点调查,发现大城市的同性恋者有四五成结了婚,小城市要占到九成。一方面大城市里风气已开,不结婚没什么;另一方面也是小地方的女人缺少知识,好骗。遇上一个河南安阳的,结婚三年了,老婆还是处女,她也没怨言。结个婚,没什么麻烦,还多个洗衣服做饭的人,不是挺好嘛? 
 
   对你们好了,可对女人呢?像你说的这位安阳的妇女,这叫结的什么婚? 
 
   我不知道,你别问我。我认识的同性恋妻子,对丈夫倒挺满意。 
 
   为什么? 
 
   她们说,他正派,不乱搞,看都不多看别的女人一眼。 
 
   行了行了,接着说罢。 
 
   J和老先生谈到半夜, 说到武汉、南昌、郑州等地都有,各在什么地方。聊完了去XX街电影院的厕所,已经11点了,人很多。都蹲着说话。苏州话听不懂。也有人找他玩,他实在累了,回去睡觉了。 
 
   J对苏州非常的留恋,他说那儿的人非常朴实。初次见面的人,只要言语相投,就能捕到家里去过夜。家里的情况是隔着一层板壁就睡着妻子儿女,两个大男人赤条条睡在竹板床上。谈到了竹板床,J面露惊恐之色:那个东西老是格格地响,而且越是要命时时候,它响得越厉害!外面睡的就是人家家里的人--我的胆子都叫它响碎了!有过这种经历之年,他再也不肯到别人家去睡竹板床。 
 
   J说, 8月6日在苏州,白天陪老太太去了狮子林,拙政园,晚上去了另一个地点。来了一个小伙子,请他去家里住。一问,他家里是竹板床。刚尝过这种滋味,他不肯去。后来又来了两个小伙子,告诉他说,联防队员要来。把他骗到没人的地方问起来: 
 
   "你的家伙大吗?" 
 
   "大。" 
 
   "有多大?" 
 
   "要多大有多大。" 
 
   "可以看看吗?" 
 
   J说: 那地方的人都爱他,到哪里都有人跟着,因为他们喜欢北方人。混到夜里十一二点,还有人陆续到来,有个大个子要和他做爱,找不到地方。去访过好几个朋友,家里都不方便。最后俩人去了一个待拆的危楼,那里伸手不见五指,爬上摇摇晃晃的楼梯,脚下的楼板一踩就陷,好像席梦思床垫一样。他说,我随时都准备一脚踩空从几丈高的地方摔下去,但是没有摔。这一回可算是冒了险了。 
 
   我说:你就不怕那人在黑地里给你一闷棍,把你的钱包掏去? 
 
   他说:不怕。在苏州不会有这样的事。在要是在上海,借我个胆子我也不敢去,非被人大卸八块不可。 
 
   离开苏州去了福建,送走了老太太,自己又一路玩回来。到处都有朋友。其中最离奇的是在寺庙里住宿,和和尚也搞了起来。我们很怕把这种事写出来引起佛教徒的抗议, 只好把这段故事忍痛割爱。回来时J走了京广线。人家说,这条线上,凡带市字的地方全有。在他这次旅行中,离奇的遭遇就发一在京广线一个小城里。这座城市在北方,连市郊车在内,公交车不超过10路。但是同性恋者的活动很多。因为其中有一些事例过于具体, 所以不便指出城名。在此之前J到过武汉,被人骗得一塌糊涂,留下了上有九头鸟,下有湖北佬,名不虚传的印象。又到过南昌,留下了热得晕头转向,玩都玩不好的印象。到过杭州,见到一个公共厕所里才晚上九点钟居然有三对在玩的奇观。 最后他回北方来,已经到了8月下旬,北方的天气转凉,J的热情却高涨起来。 
 
   J说, 在这个城里他遇上了很多热情的朋友,也碰上了敲竹杠的朋友。圈子里的人要一点钱的事,以前也遇上过,但是双方都不欲声张,所以不出商量的范畴。圈子外的人来要钱, 十之八九是讹诈,这是不言喻的事。在此之前J也遇上过一些可疑的事。比如在苏州自己一人去了一个别人说有危险的地方,遇上了一个年轻人。在玩的过程中,可以发现对方不是同性恋者--太紧张,太不自然了。事情完了之后,两人之间有过很不自然的时刻。当时对方先请J和他一块儿去一个地方,J说,他还有事, 不能去。后来对方又说,自己有点困难,要借钱。J说,自己的差旅费也不富裕,没钱可借。说完了这些话,双方无言对恃了很久。当时是上午,天下着小雨,地点是公园的一角,四周没有人。对方是个穿旧军衣的小伙子,骑一辆旧自行车。过了一会儿,那人说,那好罢。你以后小心点。说完就走了。后来听说,在那个公园里,有好几个老头被人坑了钱去。 
 
   J和我说到这事时,还是满不在乎。他说:那小子吓不倒我。我可不是老头子!打我不怕,见官我也不怕--大不了一人五十大板,我就不信你也乐意蹲班房。不过,这是到那小城之前的事。到过那小城后,他的想法彻底改变了。 
 
   J说, 那座小城的同性恋地点都在一座体育场里。这个地方是铁栅栏围起的一片地方,里面有绿地,有一座带一面看台的体育场(田径场加足球场,看台下有几间房),乒乓馆,灯光排球场,几座住宅楼,还有三个厕所。这里虽然有围墙,但是永不锁门盛夏的夜晚,里面有很多人。呆在厕所附近的基本上是同性恋者。只有一个厕所附近没有人,那是因为它太靠近居民楼,可能有居民来上厕所。 
 
   除了厕所,栅栏门附近的人最多,都在聊天。这个体育场共有东、西、西南三个门,所以也有三群人。西南面的有群以老人居多,大家不说话,只顾看出入的人,大概急于找人发泄。其它两群各有十余人,正在高谈阔论。老远就听见"大姐大妈"一类的字眼。当时是晚上八九点钏。J马上受到吸引,投身其中。聊些什么呢?J一进人群,就有一河南人问: 
 
   你是哪里来的? 
 
   J:北京的。 
 
   河南人:哦!是北京来的大姐呀! 
 
   J:去你妈的!谁是你的大姐!再犯贱我揍你。 
 
   这位河南人见J如此凶恶,就不敢理他了,只顾自怨自艾: 
 
   咳。都说这里的人热情,我都来了一个多小时了,还没人理我,我成了没人要的了!我可是真心要和别人好的。上个男朋友吹了时,我真的自杀过!这里有我的同乡……小X咱们是同乡, 我没撒谎罢?他还说,很想到泰国去作人妖,似乎是个性倒错者, 但J说此人纯粹是发贱。那个小X倒是个很朴实的人。J一下就看上了。他叫他走,可那人不肯去,说这位同乡上了劲,要是没人理,恐怕要出事,所以要看着他。 J一个人走出去,到处看了看,到处有人跟着。也许是被那位犯贱的河南人弄倒了胃口,J那天不喜欢这样张狂的人。 
 
   J说, 那天晚上他避开了灯光下那些浮嚣的人群,走到了黑暗处,在排球场后面的阴影里, 看见了一位40多岁的中年人。J说,他当时正想找个沉稳的人谈谈,于是坐到了他的身边。那个人马上用颤抖的声音说起话来,说他爱人在外地工作,耐不住寂寞。又说自己有两个孩子,既为人父,不该有这样强烈的感情。他又说,他想玩, 但是不敢。他不敢玩,但是寂寞难熬。J说,我从来没听见过这样鬼哭狼嚎的声音和这种死阳活气的情调,敷衍了两句,马上就跑了。 
 
   最后J还是到了厕所里, 用上各地通用的方法。有一个农民模样、憨直可爱的中年人进来时,和他并肩站在小便池前--最后他把手伸了过去。这时那人说了一句话,把J的兴致全吓跑了: 
 
   你玩不玩女人? 
 
   J说:你是不是拉皮条的? 
 
   那人说,不是。干那种还叫人吗?J觉得怪极了,就和他出去说。那人让J跟他去,还说,干这事是帮他的忙。于是他就跟他去了,走在半路上,忽然见到一个很漂亮的小伙子走过。 J一见此人,马上就说,我还有点事,告辞了那个叫人去玩女人的人,跟上去了。 
 
   我问J, 那人是怎么回事。J说,当时他也不知道。J还说,那次他不是想去和女人睡觉, 而是想看看到底是怎么回事。总之,那件事暂时没了下梢。J看见一个漂亮小伙子就跟了下去,那个小伙子反穿着一件制服,看不清是军服还是警服。在农村和小城市,常有这么穿衣的人。他们说,这们穿衣服省。平时把衣服反穿着,遇到正式场合再翻个个儿,又是一件新衣服。当然,在大城市里这样穿衣就有被捉进精神病院的可能。 我们说过,J喜欢朴实的人,什么农民、民工之类,对他有特别的诱惑力。 他就跟了下去。一直跟到了黑的地方,开始攀谈。J说:你是我们一伙的吗?答曰:是。问:你怎么知道的?答曰:体育馆里的人告诉的。虽然说话不清不楚, 但是刚入道的人着不多都是这样,不好意思嘛。J和他动起手来,做了一番同性恋标准的forplay(准备动作) 。最后那孩子把领子一翻,露出个领花来,像个警察领花样子。随即左手从胯下掏出一把枪来说:老子就是干这个的!跟我走一趟罢! 
 
   J说: 那时我好心慌!俗话说的好,罐儿不离井上破,我总算碰上了。到了这步田地,只好认栽。也不知道要挨一顿打,还是蹲班房。在这种小地方,什么事都碰得上。 那人把枪从左手换到右手说:看看你的身份证!J把身份证呈了上去。那人看了以后收了起来,说: 
 
   你知道你在干什么吗? 
 
   J:不知道。 
 
   问:你知不知道,你这叫同性恋! 
 
   答:谢谢你告诉我,以前我真不知道。 
 
   问:你知不知道,同性恋是不容许的? 
 
   答:不知道。 
 
   问:你说这事怎么了罢? 
 
   J说:既然犯了错误,我就跟你走吧。 
 
   J对我说, 那个人拿枪像拿个滚烫的烤白薯,左手倒右手,右手倒左手。除此之外,那枪也有点不像真的。于是他就和那人走到体育场的大道上。说来也巧,遇上了曾和J在排球场聊天的中年人。那人只看清了J,没看见另一个是谁,还以为这两人有什么好事呢,就跟在后面,若即若离,有10米的样。就这样走到门口附近,警察倒有点怕。他站住了,掏出J的身份证说: 
 
   你想私了呢,还是官了? 
 
   J马上坚定地说: 
 
   官了! 
 
   J心说:狐狸尾巴露出来了。那人说:你是不见棺材不落泪呀!J说,可不是吗。把我身份证拿回来。劈手夺回身份证。那人厉声喝道:站在这里,不准动!说完拔腿说跑。 J紧追不舍,于是出现了徒手犯人追持枪警察的精彩场面。到底是警察厉害,跑了个无影无踪。 
 
   那天晚上的事是这么民结的: J到派出所报案,说遇上了持枪的截匪,那个截匪还穿了警服。 民警们觉得问题严重,来了四个人,拿了电棍,由J带路,来抓那穿警服的小伙子, 搜遍了整个体育馆,也没搜到。等民警走了,J找别人一打听,原来此人是在附近打工的民工。他敲诈同性恋不是第一次了,诈过其他人的手表和钱,还打过人。至于那把枪,别人都知道,那是一只打火机。 
 
   这件事到此还没有完。 第二天,J找到了那个小伙子,告诉他,他的事犯了,昨晚上带了四民警来抓他等等,直到把他吓到半死。J叫他把抢人的东西都交出来,他乖乖地照办了。 J说,交赃还不能算完,你还打了人家呢,写人检讨书!那人就写了个检讨书。 后来J把表和钱还了本主,但是那检讨书没有给挨打的人,而是带回了北京。其实那不是检讨书,而是保证书,全文如下: 
 
   保证书 
 
   保证人戴XX,19岁,河北省XX县XX乡人。今年在体育场上厕所,有人来摸我下部。一气知(之)下,留了他的手表,还打了他。我这样做是不对的。但是他是同性连(恋)。 
 
   我今后在(再)也不做犯发(法)的事了。 
 
   我今后在(再)也不做对不起人的事了。 
 
   年 月 日 
 
   我认为这孩子很有修辞才能。所叙事实简短有力,措辞巧妙(留了他的手表!),而且结尾写得非常哀婉动人--今后再也不做犯法的事了--今后再也不做对不起人的事了。我小时候淘气,检讨书写了不计其数,没一篇写得这么好。至于那几个白字,只是白玉微疵。我这么说了之后J气得吼起来: 
 
   怎么,你说这检讨写得好? 
 
   那当然了。你想候,这孩子顶多也就是初中毕业吧,写成这样,还能说坏吗? 
 
   混账。你看这句:"我这样做是不对的,可他们是同性恋!" 
 
   这叫什么话!难道我们是同性恋,就可以抢吗? 
 
   由这个话题我们谈到了同性恋权益的问题。我认为,在我们国家,目前不能谈同性恋合法化的问题。 但是正如J指出的,就算同性恋非法,也有权益问题,总不能让十几岁的小流氓来抢罢。我记得文革时,社会上有很多牛鬼蛇神,任凭别人来抢,结果是使很多当时十几岁的人成了流氓。 
 
   J说,那孩子写完检查后,就管他叫大哥,说要和他交朋友。最后的结果用J的话来说, 他们玩了。但是J说,这孩子一点也不无辜。虽然不是同性恋者,但是有很多同性恋经历。据他自己交待,每次都是他和别人玩了以后才翻脸--他很能欣赏同性恋行为。那孩子,坏着哪。 
 
   了结了这位冒充警察者后中, J又回到体育场里,把从他手里要回的手表和钱还给了受害者。那座城市很小,同性恋者没有地方可去,每天都要以这里来,所以找倒那些受害者并不难。J马上成了英雄。这时J发现,那个和他在排球场后面谈过话的中年人(那位黑夜里的呻吟者),居然也是受害者之一。他对该呻吟者说:你这么大的人,被个十几岁的孩子抢了,羞也不羞?大声吼一吼,就把他吓跑了! 
 
   那人说:你看我这个样子--我敢吗? 
 
   J对我说, 我们这些人都是这样,干着这种事,难免胆小,见人矮三分,所以会被外人敲诈。这回出去,像这样的事听说了不少。以前我听到别人被讹诈,总觉得问题出在被诈乾身上--谁让你那么胆小。我是满不在乎的。被诈了一次,总算明白了。 
 
   明白了什么? 
 
   明白了被诈的滋味。假如有个人说,自己是个警察,叫你跟他走,你心里马上想到:我完了……所有的人都会知道我是同性恋……身败名裂……家庭,单位,所有的人……那一瞬间,真想死掉。尤其是中年人,肯定会精神崩溃。 
 
   那你怎么没被吓住呢? 
 
   也算事有凑巧,我光棍一条。最近单位不如意,正在调动。崩是崩了崩,溃的不厉害。所以有胆子和他周旋。他碰上我,也算命里该着罢。第二天,那个戴XX就辞了工,回家去了。 
 
   J说, 他老惦记着那位让人去干他老婆的河南人。那人马上把他缠住,要他带他上北京。 此人简直是个花痴。J对该花痴没兴趣,对他的同伴有兴趣。所以他找到那个小X,两人做了爱,又聊起来。这位小X虽然不是本地人,对这里的事却全知道。问起有个男人请他去干女人的事,小X说: 
 
   这人我知道,是近郊的农民。他也是同性恋。请去干的女人,就是他老婆。 
 
   这件事的原委是这样的:那位农民大概在结婚时,还有能力满足老婆,后来越来越不成。该老婆知道了他为什么不成后大怒,打得天翻地覆。最后的结果是:你要出去搞同性恋,我在家里也不能闲着,你给我找人罢。这家已有两个孩子了,光景也不错, 所以不想离婚。J说:当然是混账胡来,但也不失为平等。据说那位老婆性欲强烈,该小X上次去时,几乎死在那里。 
 
   J说。第二天,他就由小X带路,去找那位农民,走到村口遇上了。当时是大白天,那位农民见两条大汉找上门来,就有点怕事,说今天丈母娘来了,改天罢。正在聊天, 该农民的老婆从村里撵了出来看见了J,眼睛里冒出鬼火一样的光芒,就要J到家里去。农民居中劝解,乱成了一团。J对女人的兴趣原本不大,这一趟本意是看看热闹。见到这种场面,居然心里一慌,拔腿跑掉了。 
 
   J的故事到这里就结束了。 当天下午他乘火车回到了北京,结束了一月有余的游历。我们还没有把他说的每件事写出来。有的事情虽然有趣,但不是社会学关心的事。 有一些事和我们已记述的很相似。从社会学的角度来看,J的外地之行在两方面值得重视:

    首先,它说明了在全国很多城市都存在同性恋社群,有一些人从事像采购员,推销员一类的职业,而同性恋者对外地来的人有很大的好奇心。所以同性恋的交流,不仅存在,而且有相当规模。

    其次,仅就他说到的情形,他在这一次旅行中,起码和四个非同性恋人士发生过性关系。这四个人全是农村来的民工。这些人都很年轻,大多数是在挣钱准备结婚。我们认为,在同性恋形成过程中,先天的因素和后天因素都起作用。随着农村婚姻支付的增长,有很多青年难以结婚,所以存在境遇性同性恋的土壤。又有很多青年流入城市做工,与同性恋社区接触。所以说,农村里的同性恋一旦出现,比城市还容易曼延。现在已经知道,有些住在小城市近郊的农民参与城里的同性恋活动。将来会不会在农村集市一类的地方出现同性恋社区, 或是现在已有这样的社区存在,还是未定之数。我们身为社会学工作者,常常深感内疚:一方面,有些同仁嘲笑我们,只能搞这类小题目,实在是鼠肚鸡肠。另一方面,我们因缺少财力人力,连这样的小题目也搞不周全。但是这类小题目,实在是社会学存在的基础。别人对实际存在的社会一无所知倒也罢了,连我们都不知道,所司何事?简直是失职。

\chapter{与安德烈的访谈}

    1996年,意大利独立纪录片制作人安德烈来到中国。他想拍摄一部表现中国先锋艺术的纪录片,他选择了崔健和王小波。其时,王小波从人民大学辞职,《黄金时代》刚刚获《联合文学》小说大奖。安德烈事后说,没有想到王小波能出名,当时王小波的读者很少,他的书无法进入主流图书市场,只在书摊中流转。 
    这部纪录片拍摄于1996年10月,其大部分素材已经丢失,以下对话为现存片断的记录。 

    ●受访人:王小波 

    ●采访人:安德烈 

    ●整理:严小额 

    安德烈:第一个问题是,你什么时候成为作家的?是什么原因让你成为作家? 

    王小波:1992年,那时候我已经40岁了。40岁突发其想要写小说,中国的作家里,没有这么晚开始写作的吧,主要原因可能是喜欢写小说吧,以前觉得没这种可能性,后来发现有这种可能性了。 

    安德烈:你觉得要成为一个作家得具备什么样的因素? 

    王小波:因素?主要看能不能靠它维生吧。维持生活,大概,我觉得当作家在中国挣的钱能够维持生活吧,所以就能当作家了。 

    安德烈:这是你个人的情况,还是总的现象? 

    王小波:总的不一定,我个人的情况是可以。 

    安德烈:你最喜欢的小说家是谁? 

    王小波:小说家?我不是奉承意大利人,我比较喜欢卡尔维诺。 

    安德烈:我听说,也不知道是不是真的,你写的书,在国内也卖得不是特别容易? 

    王小波:我也不太清楚,因为它都不在书店里卖,都在书摊儿上卖,书店好像不太愿意卖小说吧,也不光是我,我可能也有点个人麻烦,但是不说也好。 

    安德烈:“文革”之后你做过什么事情? 

    王小波:刚开始我是一个中学生,刚进中学一年,13岁,然后我又到云南插队去了。我干过的事太多了,经常换事,然后又回来了,做过乡村的小学教师,街道工厂里也干过,有一段时间又没有工作,无业游民。 

    安德烈:有没有被抓? 

    王小波:没有,我没有当过罪犯,因为我特别胆小。 

    安德烈:你是不是认为有两种思维方式,一种东方的,一种西方的,你自己能区分吗? 

    王小波:我觉得这种说法相当无聊。一般来说东方的思想方式跟西方的很不一样,这种说法本身就在掩饰着什么,其实,人嘛,都是同一个物种,怎么会思想的方法不同呢? 

    安德烈:我是说看世界的方法。 

    王小波:看世界的方法我不认为有多少区别,当然,我们对事物的趣味可能不一样,但我们做为一个人,所承担的责任和义务,怎么看待别人,怎么看待自己,这些基本的方面是一样的。就像所谓的东方主义里说的,东方社会更注重整体,更忽视个人的尊严啊什么什么的,我不相信这种东西,而且我很反对! 

    我觉得人活着必须要有尊严。 

    安德烈:什么是尊严? 

    王小波:尊严就是,你在任何地方都被当作一个人物来看待,不是一个东西来看待,一个人他过去可能在单位内,在自己家里的时候他有尊严,当他走在街上的时候他就没有尊严了,别人不认识他,就把他当作一个东西来看待,我希望在任何地方都被当作一个人物来看待,这就是我说的尊严。 

    安德烈:在你心目中“文化”是什么东西? 

    王小波:我自己心目中的文化,包括艺术和科学在某种意义上是等同的,都是给人带来幸福和快乐的东西。 

    安德烈:你觉得知识分子对文明社会有多重要。 

    王小波:我觉得一个社会的文明首先要存在在一个知识分子的心里,知识分子明白以后,整个社会才会进步,我不知道其他国家怎么样,中国肯定是要这样。中国是要依赖知识分子的。中国人,不论是有文化的人,还是没文化的人,都崇尚智慧,都喜欢学习。虽然知识分子并不都是干好事,但中国人崇尚智慧的心,始终没有变过。 

    安德烈:你的作品,包括过去写的,主要是谈什么? 

    王小波:恐怕很难很简单的概括出来吧,有段时间有怀疑的色彩,怀疑眼前发生的事情,到底是怎么回事,有个朋友说,我的小说老在追问自己,追问自己对一些事情的理解是不是对的, 

    安德烈:这种追问有什么背景吗? 

    王小波:可能跟过去的经历有关系吧,比如说有一个小说《黄金时代》,在反复地怀疑什么叫做性爱,它当然是在一个特殊的时间背景下,在“文化大革命”里头,然后不断的追问,那个时候的人,对性爱,会有怎么样的理解。然后一重一重的盘问下去, 

    安德烈:你到目前为止写了几部小说? 

    王小波:论书?还是论篇? 

    安德烈:书。 

    王小波:书就两本,因为中国的出版商他们觉得说一个书书就得比较厚,小册子他就不给出。 

    安德烈:这两本书在谈什么? 

    王小波:其中一个是在谈我们的生活,我所关心的生活背后的事情,生活的原因是什么,为什么会是这样,为什么会是那样。另一个纯粹是想象的。想象是一件快乐的事,你想象一个东西,自己写很快乐,你给别人看也很快乐。 

    安德烈:那么,你对生活发现了什么? 

    王小波:我起初发现生活有很多虚伪的成分,像一个洋葱头一样,一层一层的皮都是假的,我剥到最后,还没剥到心呢。 

    安德烈:你下一步要探索的是什么样的问题? 

    王小波:下一步我不想写那种自问自答的小说了,我不是说想写想象的小说、快乐的小说嘛! 

    安德烈:想象的内容什么呢? 

    王小波:等到我想到了才知道,因为在想象开始之前,那个地方是未知的,所以才有意思。 

    安德烈:我听有人说你很悲观? 

    王小波:我其实并不悲观,相反我很乐观,不知道别人怎么会这么说,我可能把一些东西否定得太厉害了,但我觉得,相信智慧是没有错的。

\chapter{田松和王小波的对话}

1995年2月,我在王小波的寓所与他有过一次两个多小时的谈话。这是一次非正式的采访。
谈话的主题涉及个人经历,社会现状与知识分子的处境,等等。根据录音,我把这些谈话包
括口误甚至人逐一变成文字,把它作为我们某一时刻的生存状态的切片,希望几年以后,就同样的问题再
做一个,讨论比较两个切片的差异,也许可以作为我们个人和社会演化的缩影。当时我与很
多人有过类似的谈话,但由于我的懒惰,只有3个对话整理成文。现在,两年过去了,大家
多多少少会有些变化。但对小波,我已不可能再做一个切片了。


这是我和小波唯一一次谋面,但谈话还算融洽,也无甚顾忌。当然免不了有误解,也有诡辩
,有抬杠。非正式采访,谈话便很意识流。昨天把小波今年编辑的自选集《我的精神家园》
翻了一阵,发现我们当年谈到的第一个问题,小波几乎都已有了清晰的表述。然而无论如何
,这份原生形态的文字记录是我们的思想者王小波生命中的一个片段。


    以下是那次对话的几个部分。对话本是连续的,小标题是整理时加上去的。


从反面看一看


田松:首先对你这些年的经历做一个大致的总结吧,然后谈谈对自己现状和社会现状的看法
,以及知识分子的处境。近年来北京有几个杂志在谈这个问题,比如人文精神的失落。我看
到你在《东方》上也在写一个社会伦理漫谈的栏目。


王小波:说是社会伦理,但我基本上都在谈知识分子的处境。我自己的情况和知识青年差不
多。中间失了学,后来又回来当知识分子,不过感觉还是不一样的。一个学工的人可能觉得
很痛苦,损失太大。对搞文的可能还有点好处。到底是什么好像也说不清楚。


我回顾底层生活的时候吧,可能跟别人不一样。我没看过《年轮》的小说,只看了电视剧。
觉得和梁晓声的想法不大一样。文化革命,插队,有一个机会,可以对社会呀,意识形态呀
,从反面看一看。《年轮》使我觉得有些人反面一课还没有上完。


田松:反面一课你具体指什么呢?


王小波:说出来可能比较严重。看到过社会实际上是怎么样的,回来之后再当知识分子就大
不一样。体验不一样。在美国的时候上了一个台湾有名的中央研究院院士的课,下了课老找
他聊,关于社会呀,政治呀,青年呀,他有个感触特别大,说你们和台湾同学比起来,知识
肯定不行了,比较可贵的是你们对社会的了解比台湾人要强多了。有机会从反面看一看还是
大不一样的。


田松:你的反面就是指王二写的论文?


王小波:可能有点吧,那个是开玩笑。咱们这个社会没有宗教,可意识形态就是宗教,有点
神圣的,金科玉律的那种劲头。你要是当过知青就可能不信了。这种东西算什么呢?算一种
统治的权力吧,当了知青以后,统治的权力就要破灭了,想信也不信了。倒有个机会重新学
习怎么生活吧。


田松:你指的是彻底不信了,对任何一种都不信了?


王小波:觉得意识形态挺虚伪的,不真实。


田松:阿尔都塞就把意识形态和个人的关系定义为想象的关系。


王小波:是吗?个人感觉我估计咱们俩还能一致。因为有机会学一学科学还是挺有好处的。
就跟罗素说的似的,近代人不再相信统治的权力,开始相信理性的权力。


田松:那你认为梁晓声他们那些知青还是相信统治的权力。


王小波:他不是相信,他是渴望一种统治的权力。你想一想。你要不信的话,对当前的那些
话,所有的事件,都会有个特别的感觉。你要是一个特理性的人,你看当前的事就有黑色幽
默的感觉;重述当年的事件,就有一种冷嘲的口吻。我觉得他们说我跟王朔相象,也可能有
那么一点。王朔的幽默感里也包含着不信的反虔诚的味道。


田松:我感觉你和王朔,都跳出来了,但是你们跳的方向不一样。可能你更客观一些,更超
然一些。


王小波:不信意识形态了,旧的价值也没有了,重新立足在什么地方呢?总有个立足吧。不
然不就变成彻底粗俗,浑浑噩噩的人吗?我觉得王朔不是。王朔我见过,朴的一个人。


田松:我认为王朔还是很真诚的。《王朔文集》四卷本我仔细研究过。


相信理性


王小波:王朔不是像有些人对他那种理解似的。我自己虽然不信意识形态,可能还是有立足
之地的。从科学训练里呀,总能获得一些东西吧。另外相信理性。


田松:说的具体点。理性是一个纲的东西。


王小波:是呀。你看《东方》上经常说三道四的,论到最后恐怕文科的老先生不欢迎。实际
上还是宣扬理性至上的观点。


田松:什么都不信,对知识分子来讲,可能没什么害处。信什么反而是个束缚。但是对老百
姓,恐怕还是得让他


们信点什么吧?


王小波:对呀。这是肯定的。知识分子和老百姓还是不一样的。大知识分子自己都有哲学,
还能弄个哲学给他信吗?我就是这样――以知识分子自居吧。不过我觉得对老百姓不能太强
求。真糊弄不了人的,你强迫人家信也不可能。实际上,我觉得信什么是价值问题,它跟理
论问题不一样,跟社会伦理就更不一样。其实很简单,不一定就非得说咱们大家都信一个教
,才能在这个社会中共存;都信了共产主义,这个社会才能安定。这是一种很荒唐的想法,
这恐怕是中世纪的想法。


田松:我觉得道德这东西是从内心深处生长出来的一种东西。我考虑过一个问题,就是我们
还剩下什么。比如农村一个儿子不孝,从前他自己是有禁忌的。他自己感到是一种罪恶,同
时,全村人也会谴责他。但是现在,他内心的禁忌已经没有了,周围的人对他也不大理会了
。再比如偷东西,当他伸出手的时候,他自己就认为这个事件是可耻的,他就会缩回去。这
是他内心深处的道德禁令,现在这个禁令就没有了。比如北大的学生偷自行车,根本就不当
回事。如果他心里没有,你用一个法律来约束他,那是管不胜管的,这时的法律是没有根的
。


王小波:可是你说用单一的道德去约束合理吗?


田松:这种道德应该是生长出来的,而不是外在的。


王小波:道德不是能灌输得进的东西。


田松:而我们现在宣扬的道德,恰恰是把我们生长出来的东西给砍掉了。把延续了几千年的
儒教的道德用另一个道德给替换了,而这个道德是外加的,它没有办法走到人们的内心深处
。


王小波:我觉得无所谓外加不外加。最大的问题就在于它从道义上垮了。它有很多不合理的
成分,不太成功,并不是灌输得不好。我觉得信仰问题当然很重要,但它跟社会问题是不能
等价的。不能说社会问题当因,信仰当果。我们不能根据社会问题来制造一种信仰,然后让
大家来信它。


田松:我的意思是说,比如动物也有本能的禁忌,某些行为动物自动地不会去做。人的道德
也应该从类似这种本能的东西中长出来。


王小波:不过这是个比较复杂的事情。比方说现在信奉的任何一种价值观,都有从远古传来
的某种东西,它在古代和近代是完全不同的。它在古代是一种统治的权威,你不信不行。在
近代,它变成了另外一种东西――理性解决不了的东西,它来解决。现在英美人信教,大体
就是这个信法。恐怕最后还是免不了变成一种现代形式。中国的传统伦理道德,你要强行恢
复,搞出宗法那一套,可能吗?


田松:那是肯定不可能的,也是不需要的。


王小波:那么为什么现代社会里要信奉同一种道德或同一种价值呢?


田松:它并不是同一种,但是它们有共同的东西。


王小波:共同的东西,对文明社会来说是都有的。比如说美国,信什么的都有,但是它整个
的社会伦理并不是宗教给出的。


田松:但是起码做人的准则?


王小波:做人的准则也不是哪一种宗教给出来的。社会伦理不一定以一种宗教的形式存在,
它还有宗教里没有的一些东西,像什么人人生来平等。所以我觉得作为现代社会的知识分子
,不必拘泥于一种意识形态或哪一种宗教,就是一般人也不一定要这样。


田松:那你如何看现在有些人提出的价值重建呢?


王小波:价值重建?我觉得与其去搞它还不如去搞一种更简单的伦理的重建,按现在的标准
,重建一种更合适一点的伦理原则。在现代社会,还以美国为例,它作为社会的基础首先是
人人生来平等,还有商业社会要求的人人必须诚实。我觉得它比那种赶紧让大家都信一种教
或赶紧读经以孝治天下都更合理一些。


田松:我问我们还剩下什么,这些东西我们已经没有了。


王小波:其实原来它就没有。


田松:现在最可怕的是某些地方政府。什么计划生育委员会,公路检查呀,已经赤裸裸地不
顾廉耻地搜刮民财了,可当年它还是顾忌的。


王小波:这当中有一个很严重的问题。当年最基本的道德和伦理都是意识形态制造出来的。
它本身不合理,现代社会谁也不会接受,可是一种合理的作法,现在还没有。一想到价值重
建,就马上想到在古代它是从一种什么宗教,什么哲学恢复出来的,就想赶紧恢复那种宗教
或哲学,它能合理吗?就说美国,它是新教国家,它有一部宪法和独立宣言。现在高唱道德
重建的,都说只缺了一种新教,没有人说缺一部宪法,或独立宣言似的东西。


田松:但是只建一个法还是没有大用的。我认为还应深入到人们心中去。


王小波:对,仅有宪法是不行的,还应有独立宣言。独立宣言里有一整套伦理的原则,如人
生来平等。


田松:实际上还是人心里缺这个。我相信它以前曾存在于人们心中。比如一个偏僻山沟中的
老百姓,你让他往酒里掺水,他也许还能做得来,你要让他造假药,他就会有一种禁忌。现
在人心中的禁忌没有了,这是最可怕的。心里的东西要重建就非常难了,恐怕一两代人也建
不起来。一个法是容易的,立一个,一表决通过就行了。


王小波:我指的不是一种法呀,而是一种伦理的原则。


田松:那还是要进入到内心中去。


王小波:伦理的原则也能进入心中。起码美国是这样。美国人《独立宣言》都是会背的,而
且也进入到心中了。哪个黑人要是被警察揍了,他准会大叫,我人权被侵犯了。他知道他有
哪些权力和义务。美国在第二次世界大战时,一征兵都去了。现在你在美国随便碰到一个
50岁的人,全都当过兵,没当过的那就是有病。可是你看中国抗战的时候,抓个壮丁多难。
现代国家要求的东西和古代国家很不一样。西方国家里,信什么的都有,各种各样的,可是
他们除了信仰宗教,对国家也是很尊重的。一打仗,呼啦一下全去了。黑的,白的,没有几
个人说装病什么的。所以这个东西也能深入人心,为什么不能呢?


田松:但总得有个途径吧?


王小波:途径啊,有各种各样的。教会是一种途径,祖宗祠堂也是一种途径。实际上《独立
宣言》是一个文件,但所有有理性的人都愿意相信它。罗素写中国问题的时候就指出,传统
道德有不适合现代社会的地方,而且是很糟糕的。他认为中国传统道德好的地方就是什么乐
天知命那一套,但是最坏的就是孝道。因为孝道是私德,但是社会缺少公德。每个人你到他
家里头很干净,门外就全是灰。清末民初就是这样,现在还这样。重新提倡传统伦理道德仍
然带不来一种公德。


田松:对孝道本身我也是一直持攻击态度的。有些东西不是孝的问题,而是作为人的基本的
东西。我们现在连包括孝在内的传统道德都没有了。群体性动摇了,但是个体性的东西还没
有生长起来。


王小波:很多知识分子都是自由主义者,或者说理性至上的人,他们不见得在行为上就很糟
糕吧?不见得就出门随地吐痰,然后随便偷阴沟盖。我觉得从理性基础上也可以做一个好人
。


田松:这种小坏事他是不会去干的。


王小波:要干就干大的?


田松:对,要干就干大的。


智者与坏事


王小波:没有听说过真正的一个智者会干很坏的事。

田松:智者?那当然了,智者的定义就是这样的。


王小波:不见得智者就是善,但是智者干大坏事的很少。从一个人彻底地相信理性的那一天
起,就应该能成为一个好人了。


田松:这可不好说,理性本身会成为一种力量。


王小波:为善为恶都可能吗?


田松:对。


王小波:我觉得不是这样。比如说你是一个智者,你又关心伦理问题的话,好坏还是知道的
。


田松:这恐怕是太理想化了。


王小波:太理想化了?


田松:对。再一个看你怎么定义智者。比如如果他做了大坏事,就不把他定义为智者。


王小波:不对吧?知识分子他有区别智慧的标志。比如说爱因斯坦是智者,他不是做好事情
做的。而是有一个什么事表现了他的智力,才成为智者。不见得他干了多少好事。而且据说
爱因斯坦私德还很成问题。


田松:最近听说有的传记是这么写的。不过这很难说,爱因斯坦咱们也不认识。


王小波:据说对妻子儿女都不好。


田松:这恐怕――


王小波:不过也没听说他对他们做很坏的事。


田松:他可能没时间对他们好,也没时间对他们坏。


王小波:不是说高斯就很糟糕吗?太太要死了。仆人说老爷,太太要跟你说句话,他说让她
等一会儿,我正忙着呢。不过大坏事干不了。


田松:关于这个好坏的标准也很难说清楚,当年许多科学家给希特勒研究原子弹。


王小波:那也可能受了强迫了。


田松:有些物理学家本身就是纳粹。他也是有信仰的。希特勒那么短时间就把德国的工业搞
上去了,也是很多人处于信仰投入进去了。


王小波:我感觉一种信仰吧,很空。你拿它怎么发挥都可以。新教也很空,搞成希特勒那个
样子也行。我就不相


信一种价值观真有很多很多的帮助。那不一定。


田松:可能这个事情得反过来看。通常从一个人群中,我们能够从里面归纳出一些这个人群
共同遵守的价值观。但是如果把这个价值观硬放到另一个人群中去,恐怕也很难。


王小波:我对特别的价值观能起多少作用表示怀疑。你觉得这个东西很重要,是吧?


田松:我觉得还是很重要的。现在的情况是这样,原来有的全被慢慢地打碎了。旧的不一定
是好的,但它是一个很稳定的东西。比如在一个山村里面,王二要干这种事肯定千夫所指了
。整个村子里都从内心深处有这样的观念。现在这种东西经过几次反传统,文革以后,几乎
什么都没有了。没有任何约束,没有任何禁忌,就什么事都可以干了。比如有一个地区的教
委主任,就可以公然勒索考生,他心里没有这种禁忌。我见过一个报道,说巴西的贪污官僚
几乎都短命,因为他们的高血压,脑血栓的几率特别高。


王小波:吃太好了。


田松:不是吃太好了。而是由于他们心里的自我谴责引起的,一种心理对生理的
作用。但中国的贪污官僚就不会,心里没有。


王小波:不过我总觉得,制止他干坏事有两个东西。一个是他的信仰,还有一个是他信奉的
一些伦理原则,就像美国的新教和独立宣言是两种东西一样。干了坏事,他又内心矛盾,再
给自己拖出一个精神病来,不是双重的害处吗?所以你说真正发自内心的信仰能解决什么问
题,我还真不信。


田松:不。发自内心的就没有这个矛盾了,矛盾恰恰因为超我是外加的。


王小波:怎么说呢,我总有点怀疑。比如陀斯妥耶夫斯基笔下的人物不也是实际存在吗?人
家挺虔诚的,因为痛苦甚至短命,但是坏事还是干。


纯洁文坛


(我感觉需要换电池了,就拿出两节充电电池。于是有了一段关于充电电池的谈话。之后,
我们便换了一个话题。)


田松:我曾有一个想法就是纯洁文坛。组织一些副刊联手,点评。


王小波:纯洁文坛,不纯洁吗?现在。


田松:现在文坛乱七八糟的,为什么写的都有。为钱的,为功利的,为职称的。

王小波:还有按照惯性写的。


田松:为钱写的有人写得特别恶心。好多家报纸都登过一个整版广告,写505的总裁,叫什
么《寻找生命的太阳》,我一看,把这人写成老佛爷了。


王小波:你看过柯云路的书吗?


田松:看过一些。


王小波:你觉得他是真信假信呀?


田松:柯云路是真信,他是走火入魔了。


王小波:五迷三道的,我就看过一本,都快气疯了。


田松:哪一本?


王小波:就那个什么《大气功师》。什么运起气来,半云半雾的从湖南长沙就朝北京飞过来
了。挺危险的。你说要是哪个守大桥的解放军看见了,可疑飞行物,给他一梭子怎么办呐?
竟瞎掰。


田松:按照他的说法,有功夫的人能看见,没功夫的人看不见。


王小波:假如守大桥的那个解放军用功看见了,你说他打不打?纯粹胡说八道。


田松:他马上又有一个假设,有功夫的人都是相互理解的。


王小波:是么?这就叫写小说了?


田松:即使他们不互相理解,他马上还有下一个假设。他是打不着的。


王小波:黑更半夜的满天乱飞不大好吧。还什么练起气功,大钢化玻璃一头就滚过去了。胡
说八道!什么文曲星之类的,不是瞎掰吗?八卦这类的,越说越玄。你说纯洁文坛,是不是
也包括这个在内?


田松:对。不过我首先要纯洁的还是那些报告文学呀,通讯呀,还有一些小情调的小说。


王小波:煽情的?


田松:对,畸趣。


王小波:不过我很少看小说。


田松:你没有必要看,但是我必须看。看那些来稿我挺生气的。而且关键的是这些人不知道
自己写的糟,还觉得自己挺美的。


王小波:上梁不正下梁弯吧。不管怎么说,现在为什么写作的都有。过去都是作为意识形态
工具来写的。


田松:那是另一回事。这种主流文化总是存在的。


王小波:不过我觉得,咱们要写不好还是写作动机不纯吧?


田松:有的人写作动机很纯,但是他不知道什么东西是好的,这都是中学教师把他们害了。



王小波:我也这么想。


写作风格


田松:那么你写《黄金时代》什么动机?


王小波:《黄金时代》就是想写而已,没有动机。我觉得和王朔感觉事情的方式不大一样,
虽然有人感觉文风很像,实际不一样。王朔主要是嘲笑,嘲笑多吧。可能我冷嘲多点,反讽
多点。跟嘲笑不一样。不过我觉得他写的还好吧。不知道为什么好多人攻击他。


田松:我一直认为王朔是一个很严肃的作家。


王小波:很严肃的。有些感觉不错。我就看过他一些很小的,很短的东西。长的可能就不那
么认真了。有个小说写他认识的一个女孩子怎么死了。死呀,有一种挺犀利的感觉,挺不错
的。接下来才使劲地耍贫嘴,恐怕也跟许多人指责他有关。长的,《顽主》和下面那个《一
点正经没有》还可以,再往下,《千万别把我当人》就感觉粗了,整体结构还可以。


田松:《黄金时代》之后你又写了什么?


王小波:还有一些吧。写一本书容易,出一本书难。


田松:你除了给《东方》写,还给哪写?


王小波:《读书》写了一些。原来不怎么太写杂文,写杂文不务正业,后来越写越多。


田松:有天我在《北京青年报》看到一篇文章,说《黄金时代》是自然主义。我觉得不像。



王小波:是,我也觉得不像。自然主义,15世纪的东西吧?左拉之类的。我的小说还是很现
代的。起码看了不少现代的小说受了影响才写成这样的。


(在另一段对话里,我问他这种风格是什么时候形成的,他答是40岁以后。)


田松:你比较喜欢哪些人的作品?


王小波:像什么杜拉斯呀,可能最早看过的现代小说是图 习 尔的。


田松:我不知道,哪国的?


王小波:法国的。近年来不怎么写了吧。当时看的时候还觉得印象挺深刻。现代小说跟过去
很不一样,进了很大一步。写得相当紧凑,直截了当地就写到主题了。反正一言半语也说不
清楚。还是很不一样。杜拉斯的小说你肯定看过?


田松:我看过《情人》


王小波:杜拉斯的小说跟陀斯妥耶夫斯基相比那区别就大了。


田松:陀斯妥耶夫斯基我一本也没读过。


王小波:那就跟托尔斯泰比一比。


田松:托尔斯泰也没读过。


王小波:是吗?跟旧的经典小说比一比。


田松:旧的经典小说我看的也非常少。


王小波:那可能直接看到现代了。


田松:那个小号手叫什么来着?昆德拉。


王小波:昆德拉的东西很现代呀。


田松:对,他的东西我很欣赏。


王小波:昆德拉是一种散文的风格。


田松:有些笔法我觉得你和他有相象之处。他也经常做一些理性分析,一二三之类的。


王小波:一二三就是开个玩笑而已。


田松:理工科的人都有这种癖好。


……


(由于换电池,这一段录音缺了一段。王小波大概是问我对《黄金时代》怎么看。)


田松:…感觉就是一个过来的人冷眼看过去的事,把过去的事编了一编,编的好玩一点。


王小波:不过还比较客观,可能还有点反讽的色彩,反讽比较多。


田松:但我觉得这里边夸张的程度比较大的,是吧?


王小波:也不大,从最严格的意义上讲,夸张的程度一点都不大。你今年多大了?


田松:30岁。


王小波:实际上我书里写的事基本不夸张,相当真实的。当时就是那样,现在看来好像胡编
,真实不是。其实就是那样。我写的细节都是非常真的,大结构可能是假的,好多细节都是
真的。年轻一点的人看了可能觉得是瞎编,像黑色幽默。


田松:你当时身处其中肯定没有那么洒脱。


王小波:当然不洒脱了。有时连命都顾不了,还能洒脱吗?不过小孩子可能胆大,不在乎。
比如插队什么的,都是挺真实的,可能跟有的地方不一样。我们插队的地方,基本上一去三
个月,所有的革命道德都荡然无存了。就开始偷鸡摸狗,甚至说反动话了。天高皇帝远呗。



田松:不像红卫兵兵团那样?


王小波:我们那儿也是兵团。后来谁也管不了,军队干部也管不了。


田松:但是我看老鬼写的《血色黄昏》。


王小波:他挺虔诚的。


田松:他写的兵团和你这个就完全不一样了。


王小波:可不是不一样吗。我们去的时候还是农场呐,没改呐。


马波啊!每个人复述当年的事情都不一样。这跟人的气质有关系。《血色黄昏》从始至终还
是很虔诚的。


我大概从17岁开始就不虔诚了,就已经很坏了。气质也不一样。我确实就是不虔诚的气质。
我可以想象马波从小肯定都是好孩子,小队长,中队长,大队长之类的。我从小到大当的最
大的一次官是管4个人的,还当了没几个月就给人撤职了。人和人就是有很大的不同。


田松:好人一直当不了。


王小波:也不是坏人,就是不怎么端正。


\chapter{王小波和编程}

喜欢读书的人,对王小波都不陌生,他是中国最富创造性的作家之一,他是中国近半世纪的苦难和荒谬所结晶出来的天才,他英年早逝。他的作品对我们生活 中所有的荒谬和苦难作出最彻底的反讽刺。他还做了从来没有人想做和做也没才力做到的事:他唾弃中国现代文学那种“软”以及伤感和谄媚的传统,而秉承罗素、 伊塔洛·卡尔维诺他们的批判、思考的精神,同时把这个传统和中国古代小说的游戏精神作了一个创造性的衔接。

对于王小波也就读过一本《一只特立独行的猪》,让我对王小波产生兴趣的是在读到《Mac Talk》这本书里写到王小波除了作家的身份外,还是一名程序员,并且是一名很牛的程序员。以下是一些王小波和程序相关的故事。

 

多数人知道王小波是小说家,部分人分不清财经作家吴晓波和小说家王小波是不是一回事儿。却很少有人知道王小波可以算的上中国早期的程序员,在 90 年代初的时候因为国内应用软件缺乏,爱捣鼓东西的王小波利用闲暇时间学习了汇编和C语言,编了中文编辑器和输入法。中文编辑器和输入法任何一个都是大牛级 的 GEEK 才会去尝试的东西,比如求伯君。王小波通过卖软件还挣了些钱,当时很多中观村的老板要拉他入伙,当然写代码这种来钱快的活对屌丝王小波还是有吸引力的,所 幸他还是拒绝了人家。

王小波一个写小说的为什么沦落(/升级)成了程序员?这还得慢慢说。王小波大学在人大学的是贸易,毕业后在人大当了几年老师。后来去了美国匹茨堡大 学读经济的研究生,到那老师跟他说你就是一写作的奇才,以后必能称霸话语文坛。老师又说你在我这什么都不用干了,好吃好住,毕业证照发,抓紧时间写小说。 学成归国,王小波接着又回到人大做统计学的讲师。

因为做统计,各种分析工具是必不可少的,以前人手工计算,有了计算机当然最好使用计算机,基本理工科的都知道用 MATLAB 做个毕设和作业是多么重要。所以当时计算机对王小波的工作是相当重要的,但是 90 年的时候,软件相当稀缺,电脑又相当不智能。王小波不得不自己写软件,当然开发软件也不是那么简单,所以他先是学会了 FORTRAN,汇编,C等各种语言,接着又要学习数据结构,算法的相关知识,还有编译原理。

在做出中文编辑器和输入法,解决了自己的需求之后,王小波对写软件的兴趣已经没有多少了。因为他觉得写软件可以赚钱,写小说同样也可以赚钱。处于一 个程序员的修养,王小波还是不断地从数据结构和算法来优化这两个软件。93 年的时候,王小波买了一台 286,这在当时是一台奢华无比的机子了,他自己也是这些认为的,高兴得一塌糊涂。不过这台顶配机子还是满足不了王小波的要求,后来他把时间花在了不断地 去优化这台机子上面。

王小波可以算的上是个 GEEK。大学学的贸易,后来又学化学,再后来转到了数学系。他的同学形容他解数学题就像杀猪一样,特别来劲,一道一道解决,那感觉可能就像打 CS 拿人头一样爽。不过解多了也觉得没意思。

王小波小说里的男主角基本都是理工男,除了天天意淫还有一些奇怪的想法。其中一些还有自己的发明,比如《未来世界》里的王二是个工程师,整天想着和 单位一起研制的永动机,还有《红拂夜奔》里的李靖发明过一个巨大的开跟号机器,在战场上,这台机器运转起来挥舞着杠杆边开跟攻击敌人,有的人死在根号 3 下,有的人倒在了根号 5 下。这些都只有 GEEK 才会想得出来。

王小波干过很多事情,下过乡,考过大学,出过国,学过经济,打过代码,成了小说家,去世的那一年完成了心愿做了编剧。总之不管他干了什么,他身上让我们尊重的还是独立之精神,自由之思想。

以下内容是从王小波和朋友的书信里收集了他所做软件的各种信息,汇集起来,可以了解小波在软件方面的造诣。顺序按照原文的书信顺序,应该也就是时间顺序。
  1988 年 12 月。

回来之前我曾往人大一分校计算机站写过一封信,问他们可要带什么软件,主管的工程师回了封信,我没收到。回来之后人家还提到此事。现在国内软件一面 混乱,又逐渐有形成市场之势。首先以年兄学统计这一事实来看,回来做事非有会用的软件不可。Macintosh 根本就没打进中国市场,你非带几个可用的 IBM 微机软件回来不可。至于什么机器上能使倒不必太担心。我这个狗屁计算机室,IBMPS/2 就有二台。AT 机也不少。

SASSPSSStatistx 都有,可代表国内上等一般统计微机房的水平,可就是少了一种宜于作统计的语言。年兄如有 APL (Aprogramminglanguage)之 IBM 微机本,可给我寄 copy 来。我在美还有一个户头,连 manual 复印费一并写支票给你们。Glim 我也没有,如年兄有便人可捎来。邮寄太贵,能省就省吧。
1990 年 1 月。

我现在正给北大社会学所做统计,手上除 SPSS 没有可用的软件,国内这方面很差。我现在会用 FORTRAN,编统计程序不方便。闻兄谈起你们用S语言,不知是否好用。工具书也不知好找不。不管好歹,烦兄找个拷贝给我,要就算了。照我看只要能解决 各种矩阵运算就够:当然也要有各种分布函数。反正也是瞎胡混,我就算努把力,少混点吧。
  1990 年 5 月。

晓阳到底也加入了 IBM 的行列。照我看,苹果机还是买不得。因为 IBM-PC 的兼容机队伍庞大。INTEL 又总能推出新一代 CPU,将来还有大发展。买微机钱的投资是大事,时间、精力投资更为巨大,买 386 兼容机是明智之举。 我最近可能调入人大,投奔班长。最近胡思乱想想出了个理论来,还没认真推导,大抵是设立多个 Dummy (两分变量)构成的联合分布,其合计样本点构成一球面,点到点的距离以总误差计算。所以一样本点的对点就是它的否,误差最大。其余的正在想。
  1991 年 2 月。

兄谈及 IBM 中文软件不可用,估计是图像版有问题,可至有 Colormonitor 之机器上一试。Mac 机国内亦有,唯不及美国多也。
1991 年 3 月。

你寄来的严氏 2.0A 我也收到,还没用。因为一者是 3 盘要倒,二者我自己写的 WK 也有重大进展。我也自做了词组功能,是棵B树,我觉得自写的软件自用,感觉是最好的。词组用处不是很大,主要用于定义人地名等专有名词,但是严氏软件对我 还是有重大启示,拼音加四声是个极好的主意,写起东西来声韵铿锵,与其他软件大不一样。自写一遍,从分页到编辑键分配,都能合乎自家习惯,不是存心狗尾续 貂也。如能见到严氏,可代为致意。
  1991 年 5 月。

前次寄来软件,上机一试发现非有浮点处理机不能运转。因为缺少软件,国内机器一般不装协处理机,冷不丁撞出您这一件来,搞得不大有办法。
  1991 年 5 月。

闲着没事搞了个发明。原有中文软件是用线扫描方式出汉字。我做了一个用调整字模发生器方法出汉字的系统,自以为很优越,可惜还未找到用户。用此法可以很容易地在西文软件上出中文窗口,还在 SPSS 上加了几句骂娘的话。
  1991 年 9 月。

晓阳托人带来软件,周转很多日才到手里,软盘有些污损,坏一片烂一套,不可用矣。但是十分感念晓阳的好意。去年托人带来的中文软件(严氏 By×),我用着尚好,而且又用C语言仿编了一个,程序是我的,拼音字典是人家的,执此招摇撞骗,骗了一些钱。干这个事,熟悉了C语言,都是拜小阳所赐。
 1991 年 9 月。

不过现在我对微机已无兴趣,因为发现写小说也可赚到钱。
 1992 年 1 月。

编译程序一盘(有说明书,见 shou),源程序一盘。我的音典与严氏同名内容不同。功能上与严氏的近似,但是多了改进拼音字典的功能。按 F4 后可以把拼音重定义。也可加字,在拼音拣字时,按 enter,就进入国标拣字,拣到的字加入字典。这个软件由五个c语言(另有两个头文件)和一个汇编语言文件组成,可用 turboc 编译,但是汇编部分不必重汇了,可以把汇编文件写成的部分形成的 obj (我的磁盘上叫 wk5.obj)放到硬盘上,与其它c语言文件分开,用 turboc 的 commandline 编译器编一下,命令如下:tcc-mc-ewka:wk*.ca:wk5.objgraphics.lib 形成 wk.exe,但是必须有 yindian,cclib,egavga.bgi 三文件支持才工作。*.bgi 是图象板参数表,可以包括到*.exe 内的。但是要改改程序。你的机器好。我还用个老掉牙的 XT 机,简直落伍了。

turbo.c 你一定能找到。假如你用过其它c软件,有一点要提醒你,turbo.c 有一种极讨厌的特性,就是你在一个函数内 alloc 的内存,退出该函数时不会自动释放;还有一点也很糟,就是模型问题,在大模型下写的程序,到了小模型上一概不能用,我的程序是在 compact 模型下写的,就不能用 small 来编译,这两条是可以气死人的。

据说可以用 far,near 之类的前缀说明指针,其实是屁用不管。我干了一年多c,得到的结论是微机c还不能使人快乐,有时叫人怀念汇编。

f1 是提示键。我的打印机有汉卡,F5 你恐不能用。这个打印机是人家借给我的,性能非常之好,(美国 amt-525);但是不知能用多久。这个程序我还在修改中。与严氏的软件比,在硬的方面的优点是达到了很好的紧凑性,现在编译后是 55k,扩展余地大。缺点是图象更新没他的快(在我的老爷机上可以看出区别),不知他是怎么搞的,我很佩服。我的图像部分也是汇编写的,反复优化,也达不 到他的水平,不得不承认技不如人。另外,磁盘文件的处理,我也写不好。还要请阳公指教。
  1992 年 7 月。

小阳的信又用 MAC 机,看来你的机器不少,可喜可贺。我这一台老 PC/XT,用了六年换不下来,太惭愧,近来老想狠狠心,花几百块买个 286 主板换上,老婆又不同意,真是要命了。 我自编软件又有进展,把一部分程序递归化,出现了很新奇的特征。等我换了 286,就需要能写虚址方式的C语言了,未知晓阳能否找到?
1992 年 9 月。

你给我寄的软件因为是三寸盘,在这里很不通用,所以我也没用。盘上有什么,至今不知。我用C编的软件已经用熟,并做出了各种写小说的工具,别人的软件已不用了。现在主要是写书赚钱。从今年初开始写长篇,首先做了写长篇的专用软件,现在基本调通,开始写了。
1992 年 9 月。

递归论我没学过。我哥哥大概懂一些。我和你说的大概是计算机内的递归算法。我在美国读的书都是关于机器的。有关算法、数据结构等等,全在国内看的, 也不知英文叫什么。在C语言里是指在一个函数(相当于其他语言的 subprocedure)内调用同一个函数。一般程序书里都能查到。

所谓保护方式,是指 286protectedmode。因为一般的 IBM 机器,不管是 386,486,只要是 dos 操作系统,实际能操作的内存只是 640K,相当于一个较快的 PC 机。想要用到 640K 以外的e×tendmemory,只有用 pretectmode 才能用上,我打算换 286,还是想用多于 640K 的内存。这就要有比现在的C更好的编程工具。

当然,我也不一定要用到保护方式,有各种 EMS 软件,不过我还是想往多里捞摸,多留一手。MSC 我只见过 5.0 版,7.0 版的性能还不知道。不过我猜现在流行的C应该有这些手段,到这时候了。 我有一段时间很关心 personalcomputer 的发展,属于想玩玩不到过干瘾的那种。这路东西的发展都是由处理机片芯的发展开始。从 8088 到 286,386 看文献就知道快了很多。

从实用的角度来看,286 多了虚存保护,386 又多了很多用户多任务手段。486 据说把 386,387,和 64K 的高速缓存集成到一个片子上,不但整数运算快,浮点也快多了。不过不做科学运算,意义就小了。586 还不知道是个什么东西,想必有惊人之处。不过到了我手上用作文字处理,也是暴殄天物。我有个 286 用用就算行了。太好的东西我也使不出来。 听说美国微机多媒体搞得甚火,微机接电视,音响,vedio 等等。这玩艺听上去倒是蛮有意思的。
  1992 年 10 月。

当时不知道你为什么这么干,原来是机器坏了。居然叫人敲去 150dollars,老兄真是有钱。这种事叫我遇上,肯定自己修了。现在的微机修理都是换线换板,机器一坏,先找块表量量是不是电源坏了。只要不是电源 坏,估摸是哪部分不好,就去买块版换上。送出去也是这么修,还要敲你手工钱。我看 150 什么板子都能买来。万一自己故障没找对,就说人家的板子不好,退给他。我的机器里什么牌子的板子都有了,都像你那样挨人敲,还玩得起吗?这么弄,还能有点 乐趣。比方说,你爱 486,就去买块 486 主机板,把自己的主机板换下来,这种搞法不怕杂牌水货,坏了再换,反正便宜。我的问题在于这么搞都搞不起。
  1992 年 10 月。

我现在从我哥哥那里弄了一套 TurboC++,软件方面暂时没有问题了。只是 286 还没买,因为听说中国要加入关贸总协定,这类东西要掉价;有钱先买家具。无论如何,我是用不到 486 的,因为要 286 也不是为了追求高速度,主要是要解决内存问题。我现在软件写得出神入化,大概 8088 上能做到的一切,我都做到了。自己觉得该告一段落,去写小说了。
  1992 年 11 月。

先有朋友把你寄来的软盘倒了一下,看看是数字,就没动它。记得原来有一套干这种事的软件,是你给的,但是盘坏过,再也找不到了。今天写了个小程序倒出来,拜读了大作,甚是有趣。
 1993 年 3 月。

我终于下决心买了一台 286,这些日子在改造软件,作了不少汇编工作。其核心是它在虚拟保护方式(virtualaddressprotected)下工作,以便利扩展内存 (expandedmemory)。现在终于完全成功,我的软件现在可以编辑 400K 长的文件,可以把一部长篇小说全部调到内存里编写了。只可惜我这个机器还是低级,只有 1MRAM,并且没有硬盘,所以也就到此为止了。这个程序的缺点是太低级,有大量对端口(port)的操作,虽然效率是高的,兼容性不会太好。我的 XT 机给山妻用了,算起来我用那台机器已经七年,就如一位老友,骤然割爱,如心头割肉。
  年份不详。

我们的 pc 机还没有和 Internet 连上。本来中国有几个国内网发展得很快,现在又出了问题,谁要上 Internet,必须到有关部门去登记,留个案底,以备当局监控,很有一点监狱的气味。我还不想找这份麻烦,再说,通过 Chinanet 联网,每月也要交七八百的月费,我也没有这么多的钱。既然×反对信息时代,我们就不进这个时代罢,有什么法子。所以还是写信好了。
