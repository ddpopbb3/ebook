\chapter{绿毛水怪}


\section{一、人妖}
 
   “我与那个杨素瑶的相识还要上溯到十二年以前”,老陈从嘴上取下烟斗,在一团朦胧的烟雾里看着我。 
 
   这时候我们正一同坐在公园的长椅上:“我可以把这段经历完全告诉你,因为你是我唯一的朋友,除了那个现在在太平洋海底的她。我敢凭良心保证,这是真的;当然了,信不信还是由你。”老陈在我的脸上发现了一个怀疑的微笑,就这样添上一句说。 
 
   十二年前,我是一个五年级的小学生。我可以毫不吹牛的说,我在当初是被认为是超人的聪明,因为可以毫不费力看出同班同学都在想什么,就是心底最细微的思想。因此,我经常惹得那班孩子笑。我经常把老师最宠爱的学生心里那些不好见人的小小的虚荣、嫉妒统统揭发出来,弄得他们求死不得,因此老师们很恨我。就是老师们的念头也常常被我发现,可是我蠢得很,从不给他们留面子,都告诉了别人,可是别人就把我出卖了,所以老师都说我“复杂”,这真是一个可怕的形容词!在一般同学之中,我也不得人心。你看看我这副尊容,当年在小学生中间这张脸也很个别,所以我在同学中有一外号叫“怪物”。 
 
   好,在小学的一班学生之中,有了一个“怪物”就够了吧,但是事情偏不如此。班上还有个女生,也是一样的精灵古怪,因为她太精,她妈管她叫“人妖”。这个称呼就被同学当作她的外号了。当然了,一般来说,叫一个女生的外号是很下流的。因此她的外号就变成了一个不算难听的昵称“妖妖”。这样就被叫开了,她自己也不很反感。喂,你不要笑,我知道你现在一定猜出了她就是那个水怪杨素瑶。你千万不要以为我会给你讲一个杜撰的故事,说她天天夜里骑着笤帚上天。这样事情是不会有的,而我给你讲的是一件真事呢。 
 
   我记得有那么一天,班上来了一位新老师,原来我们的班主任孙老师升了教导主任了,我们都在感谢上苍:老天有眼,把我们从一位阎王爷手底下救出来了。我真想带头三呼万岁!孙老师长了一副晦气脸,四年级刚到我们班来上课时,大家都认为他是特务!也有人说他过去一定当过汉奸。这就是电影和小人书教给我们评判好赖人的方法,凭相貌取人。后来知道,他虽然并非特务和汉奸,却是一位地地道道的土匪,粗野得要命。“你没完成作业?为什么没完成!”照你肚子就捅上一指头!他还敢损你、骂你,就是骂你不骂你们家,免得家里人来找。你哭了吗?把你带到办公室让你洗了脸再走,免得到家泪痕让人看见。他还敢揪女生的小辫往外拽。谁都怕他,包括家长在内。他也会笼络人,也有一群好学生当他的爪牙。好家伙,简直建立了一个班级地狱! 
 
   可是他终于离开我们班了。我们当时是小孩,否则真要酌酒庆贺。新来了一位刘老师,第一天上课大家都断定她一定是个好人,又和气,相貌又温柔。美中不足就是她和孙主任(现在升主任了)太亲热,简直不同一般。同学们欢庆自己走了大运,结果那堂课就不免上得非常之坏。大家在互相说话,谁也没想提高嗓门,但渐渐的不提高嗓门对方就听不见了。于是大家就渐渐感觉到胸口痛,嗓子痛,耳朵里面嗡嗡嗡。至于刘老师说了些什么,大家全都没有印象。到了最后下课疗响了,我们才发现:刘老师已经哭得满脸通红。 
 
   于是第二节课大家先是安静了一会儿,然后课堂里又乱起来。可是我再也没有跟着乱,可以说是很遵守课堂纪律。我觉得同学们都很卑鄙,软的欺侮,硬的怕。至于我吗,我是个男子汉大丈夫,我不干那些卑鄙的勾当。 
 
   下了课,我看见刘老师到教导处去了。我感到很好奇,就走到教导处门口去偷听。我听见孙主任在问:“小刘,这节课怎么样?”“不行,主任。还是乱哄哄的,根本没法上。” 
 
   “那你就不上,先把纪律整顿好再说!”“不行啊,我怎么说他们也不听!”“你揪两个到前面去!” 
 
   “我一到跟前他们就老实了。哎呀,这个课那么难教……” 
 
   “别怕,哎呀,你哭什么,用不着哭,我下节课到窗口听听,找几个替你治一治。谁闹得最厉害?谁听课比较好?”“都闹得厉害!就是陈辉和杨素瑶还没有跟着起哄。” 
 
   “啊,你别叫他们骗了,那两个最复杂!估计背地里捣鬼的就是他们!你别怕……今天晚上我有两张体育馆的球票,你去吗?……我听得怒火中烧,姓孙的,你平白无故地污蔑老子!好,你等着瞧! 
 
   好,第三节课又乱了堂。我根本就没听,眼睛直盯着窗外。不一会就看见窗台上露出一个脑瓢,一圈头发。孙主任来了。他偷听了半天,猛地把头从窗户里伸上来,大叫:“刘小军!张明!陈辉!杨素瑶!到教导处去!” 
 
   刘小军和张明吓得面如土色。可是我坦然地站起来。看看妖妖,她从铅笔盒里还抓了两根铅笔,拿了小刀。我们一起来到办公室。孙主任先把刘小军和张明叫上前一顿臭骂,外加一顿小动作:“啊,骨头就是那么贱?就是要欺负新老师吗?啊,我问你呢……”然后他俩抹着泪走了。孙主任又叫我们:“陈辉,杨素瑶!你到这儿来削铅笔来了吗?你知道我为什么叫你来?” 
 
   妖妖收起铅笔,严肃地说:“知道,孙主任,因为我们两个复杂!” 
 
   “哈哈!知道就好。小学生那么复杂干什么?你们在课堂里起什么好作用了吗?啊!!” 
 
   “没有,”妖妖很坦然地说。我又加上一句:“不过也没起什么坏作用。” 
 
   “啊,说你们复杂你们就是复杂,在这里还一唱一和的哪……”我气疯了。孙主任真是个恶棍,他知道怎么最能伤儿童的心。我看见刘老师进来了,更是火上添油,就是为了你孙魔鬼才找上我!我猛地冒了一句:“没你复杂!”“什么,你说什么!说清楚点!!”“没你复杂,拉着新老师上体育馆!” 
 
   “呃!”孙主任差点儿噎死,“完啦,你这人完啦!你脑子盛的些什么?道德、品质问题!走走走,小刘,咱们去吃饭,让这两个在这里考虑考虑!” 
 
   孙主任和刘老师走了,还把门上了锁,把我们关在屋里。妖妖撅着嘴坐在桌子上削铅笔,好好的铅笔被削去多半截。我站在那儿发呆,直到两腿发麻,心说这个漏子捅大了,姓孙的一定去找我妈。我听着挂钟“咯噔咯噔”地响,肚子里也咕噜咕噜地叫。哎呀,早上就没吃饱,饿死啦!忽然妖妖对我说:你顶他干嘛!白吃苦。好,他们吃饭去了,把咱们俩关在这里挨饿!” 
 
   我很抱歉:“你饿吗?”“哼!你就不饿么?” 
 
   “我还好。”“别装啦。你饿得前心贴后心!你刚才理他干嘛?” 
 
   “啊,你受不了吗?你刚才为什么不说‘孙主任,我错了’!” 
 
   “你怎么说这个!你你你!!”她气得眼圈发红。我很惭愧。但是也很佩服妖妖。她比我还“复杂”。 
 
   我朝她低下头,默默地认了错。我们两个就好一阵没有再说话。 
 
   过了一会,肚子饿得难受,妖妖禁不住又开口了:“哎呀,孙主任还不回来!” 
 
   “你放心,他们才不着急回来呢。就是回来,也得训你到一点半。”我真不枉了被叫做怪物,对他们的坏心思猜得一点不错。 
 
   妖妖点点头承认了我的判断。然后说:“哎呀,十二点四十五了!要是开着门,我早就溜了!我才不在这里挨饿呢!” 
 
   我忽然饿急生智,说:“听着,妖妖。他们成心饿我们,咱们为什么不跑?”“怎么跑哇?能跑我早跑了。”“从窗户哇,拔开插销就出去了。外面一个人也没有。” 
 
   说的好。我们爬上了窗户,踏着孙主任桌子上的书拔开了插销,跳下去,一直溜出校门口没碰上人,可是心跳得厉害,真有一种做贼的甜蜜。可是在街碰上一大群老师从街道食堂回来,有校长,孙主任,刘老师,还有别的一大群老师。 
 
   孙主任一看见我们就瞪大了眼睛说:“谁把你们放出来的?”我上前一步说:“孙主任,我们跳窗户跑的。我饿着呢。都一点了,早上也没吃饱。”妖妖说:“等我们吃饱了您再训我们吧。” 
 
   老师们都笑得前仰后合。校长上来问:“孙主任为什么留你们?”“不为什么。班上上刘老师的课很乱,可是我们可没闹,但是孙老师说我们‘复杂’,让我们考虑考虑。”老师们又笑了个半死。校长忍不住笑说:“就为这个么?你们一点错也没有?” 
 
   妖妖说:“还有就是陈辉说孙主任和刘老师比我们还复杂。”“哈!哈!哈!”校长差点笑死了,孙主任和刘老师脸都紫了。校长说:“好了好了,你们回去吃饭吧,下午到校长室来一下。” 
 
   我们就是这样成了朋友,在此之前可说是从来没说过话呢。 
 
   我鼓了两掌说:“好,老陈,你编得好。再编下去!”老陈猛地对我瞪起眼睛,大声斥道:“喂,老王,你再这么说我就跟你翻脸!我给你讲的是我一生最大的隐秘和痛苦,你还要讥笑我!哎,我为什么要跟你讲这个,真见鬼!心灵不想沉默下去,可是又对谁诉说!你要答应闭嘴,我就把这件事情原原本本地告诉你。”.你听着,当天中午我回到家里,门已经锁上了。妈妈大概是认为我在外面玩疯了,决心要饿我一顿。 
 
   她锁了门去上班,连钥匙也没给我留下,我在门前犹豫了一下,然后坚决地走开了。我才不象那些平庸的孩子似的。在门口站着,好象饿狗看着空盘一样,我敢说像我这般年纪,十个孩子遇上这种事,九个会站在门口发傻。 
 
   好啦,我空着肚子在街上走。哎呀,肚子饿得真难受。在孩子的肚子里,饥饿的感觉要痛切得多。我现在还能记得哪,好象有多少个无形的牙齿在咬啮我的胃。我看见街上有几个小饭馆,兜里也有几毛钱。可是那年头,没有粮票光有钱,只能饿死。 
 
   我正饥肠碌碌在街上走,猛然听见有人在身边问我:“你这么快就吃完饭了吗?”我把头抬起来一看,正是妖妖。她满心快活的样子,正说明她不唯没把中午挨了一顿训放在心上,而且刚刚吃了一顿称心如意的午饭。我说:“吃了,吃了一顿闭门羹!”你别笑,老王。我从四年级开始,说起话来有些同学就听不懂了。经常一句话出来,“其中有不解语”,然后就解释,大家依然不懂,最后我自己也糊涂了。就是这样。 
 
   然后妖妖就问我:“那么你没吃中午饭吧?啊,肚子里有什么感觉?”老王,你想想,哪儿见过这么卑鄙的人?她还是个五年纪小学生呢!我气坏了:“啊啊,肚子里的感觉就是我想把你吃了!”可是她哈哈大笑,说:“你别生气,我是想叫你到我家吃饭呢。” 
 
   我一听慌了,坚决拒绝说:“不去不去,我等着晚上吃吧。” 
 
   “你别怕,我们家里没有人。”“不不不!!那也不成!”“哎,你不饿吗?我家真的一个人也没有呢。” 
 
   我有点动心了。肚子实在太饿了,到晚饭时还有六个钟头呢。尤其是晚饭前准得训我,饿着肚子挨训那可太难受啦。当然我那时很不习惯吃人家东西,可是到了这步田地也只好接受了。 
 
   我跟着她走进了一个院子,拐了几个弯之后,终于到了后院,原来她家住在一座楼里。我站在黑洞洞的楼道里听着她哗啦啦地掏钥匙真是羡慕,因为我没有钥匙,我妈不在家都进不了门。好,她开了门,还对我说了声“请进”。 
 
   可是她们家里多干净啊。一般来说,小学生刚到别人家里是很拘谨的,好象桌椅板凳都会咬他一口。 
 
   可是她家里就很让我放心。没有那种古老的红木立柜,阴沉沉的硬木桌椅,那些古旧的东西是最让小学生骇然的。它们好象老是板着脸,好象对我们发出无声的喝斥:“小崽子,你给我老实点!” 
 
   可是她家里没有那种倚老卖老的东西。甚至新家具也不多。两间大房间空旷的很。大窗户采光很多,四壁白墙在发着光。天花板也离我们很远。 
 
   她领我走进里间屋,替我拉开一张折叠椅子,让我在小圆桌前坐下。她铺开桌布,啊啊,没有桌布;老王,你笑什么!!!然后从一个小得不得了的碗橱往外拿饭,拿菜,一碟一碟,老王,你又笑! 
 
   她们家是上海人!十一粒花生米也盛了一碟;我当时数了,一个碟子就是只有十一粒花生米。其它像两块咸鱼,几块豆腐干,几根炒青菜之类,浩浩荡荡地摆了一桌子,其实用一个大盘子就能把全部内容盛下。然后她又从一个广口保温瓶里倒出一大碗菜汤,最后给我盛了一碗冷米饭。她说:“饭凉了,不过我想汤还是热的。”“对对,很热很热”,我口齿不清地回答,因为嘴里塞了很多东西。 
 
   她看见我没命的朝嘴里塞东西就不逗我说话了,坐在床上玩弄辫子。后来干脆躺下了,抄起一本书在那里看。 
 
   过了不到三分钟,我把米饭吃光了,又喝了大半碗汤。她抬起头一看就叫起来:“陈辉,你快再喝一碗汤,不然你会肚子痛的!” 
 
   我说:“没事儿,我平时吃饭就是这么快。”“不行,你还是喝一碗吧。啊,汤凉了,那你就喝开水!”她十万火急地跳起来给我倒开水。我一面说没事,一面还是拿起碗来接开水,因为肚子已经在发痛了。 
 
   在我慢慢喝开水的时候,她就坐在床上跟我胡聊起来。我们甚至说自己的父母凶不凶,你知道,就是在小孩子中间,这也是最隐秘,最少谈到的话题。忽然我看到窗户跟前有个闹钟,吓得一下跳起来:“哎呀,快三点了!” 
 
   可是妖妖毫不惊慌地说:“你慌什么?等会咱们直接去校长室,就说是回家家里现作的饭。” 
 
   “那他还会说我们的!”“不会了,你这人好笨哪!孙主任留咱们到一点多对吗?学校理亏呢。校长准不敢再提这个事。” 
 
   我一想就又放下心来:真的,没什么。孙主任中午留我们到一点多真的理亏呢。可是我就没想到。不过还是该早点去。我说:“咱们现在快去吧。” 
 
   妖妖无可奈何地站起来:“其实根本不用怕。陈辉,你怕校长找你吗?”“我不怕。我觉得,怎么也不会比孙主任更厉害。”“我也不怕,我觉得,咱们根本没犯什么错。咱们有理。”我心里说真对呀,咱们有理。后来我们一起出来上学校。走在路上,妖妖忽然很神秘地说:“喂,陈辉,我告诉你一句话。” 
 
   “什么呀?”喂,老王,你这家伙简直不是人!你听着,她说:“我觉得大人都很坏,可是净在小孩面前装好人。他们都板着脸,训你呀,骂你呀。你觉得小孩都比大人坏吗?”我说我决不这样以为。 
 
   “对了。小孩比大人好的多。你看孙主任说咱们复杂,咱们有他复杂吗?你揪过女孩的小辫子吗? 
 
   他要是看见你饿了,他会难受吗?哼,我说是不会。” 
 
   我说:“不过,咱们班同学欺负刘老师也很不好,干嘛软的欺负硬的怕呢?” 
 
   “咱们班的同学,哼!都挺没出息的,不过还是比孙主任好。刘老师也不是好人,孙主任把咱们俩关起来,她说不对了吗?”我不得不承认刘老师也算不上一个好人。 
 
   “对了,他们都是那样,刘老师为了让班上不乱,孙主任揍你她也不难受。我跟你说,世界上就是小孩好。真的,还不如我永远不长大呢。” 
 
   她最后那句话我永远不会忘记。啊,那时我们都那么稚气,想起真让人心痛!.老陈用手紧紧地压着左胸,好象真的沉湎于往事之中了。我也很受感动,简直说不上是佩服他的想象天才呢,还是为这颗真正的童年时代的泪珠所沉醉。说真的,我听到这儿,对这故事的真实性,简直不太怀疑了。 
 
   老陈感慨了一阵又讲下去:“后来我们一直就很好。哎呀,童年时期,回想起来就像整整一生似的。 
 
   一切都那么清晰,新鲜,毫不褪色,如同昨日!”我说:“你快讲呀!编不下去了么?” 
 
   “编,什么话!你真是个木头人。大概你的童年是在猪圈里度过的,没有一宗真正的感情。” 
 
   后来我发现了一个新大陆。那是五年级下学期的事情。这个新大陆就是中国书店的旧书门市部。老王,你知道我们那条街上商场旁边有个旧书铺吧?有一天我放了学,不知怎么就走到那里去了。真是个好地方!屋子里暗得像地下室,点了几盏日光灯。烟雾腾腾!死一样的寂静!偶尔有人咳嗽几声,整整三大间屋子里就没几个人。满架子书皮发黄的旧书,什么都有,而且可以白看,根本没人来打搅你。净是些好书,不比学校图书馆里净是些哄没牙孩子的东西。安徒生的无画的画册,谜一样的威尼斯,日光下面的神话境界!马克·吐温的哈克贝利·芬,妙不可言!我跟你说,我能从头到尾背下来。还有无数的好书、书名美妙封面美好的书,它们真能在我幼小时的心灵里唤起无穷的幻想。我要是有钱的话,非把这铺子盘下来不可。可是我当时真没有几个大子儿,而且这几个大子儿也是不合法的,就是说被我妈发现一定要没收的。我看看这一本,又看看那一本,都是好书,价钱凭良心说也真公道。可是不想买。我总共有七毛钱,可以买一本厚的,也可以买两本薄的。我尽情先看了一通,翻了有八九本,然后挑了一本《无画的画册》,大概不到一毛钱吧,然后又挑了一本《马尔夏斯的芦笛》,我咒写那本破书的阿尔巴尼亚人不得好死!这本破书花了我四毛钱,可是写了一些狗屁不如的东西在上面。我当时不知道辨认作者的方法,就被那个该死的书名骗了,要知道我正看马克吐温的哈克贝利看得上瘾,就因为那本书卖六毛钱放弃了它!我到收款处把带着体温的,沾着手汗的钱交了上去,心里很为我的没气派害羞。可是过了一会,我就兴高采烈地走了出去,小心眼地用手捂着书包里那两本心爱的书。我想,我就是被车压死,人们也会发现我书包里放着两本好书的,心里很为书和我骄傲。后来仔细看了一遍马尔夏斯的芦笛,真为这个念头羞愧。幸亏那天没被车压死,否则要因为看这种可耻的书遗臭万年的。不过这是后话了,不是当天的事。 
 
   我为这幸福付出了代价。因为回家晚挨了一顿好打。不过我死不悔改,晚上睡觉时还想着我发现了一个无穷无尽的快乐的源泉。第二天我上课时完全心不在焉。不过不要紧,我不听课也能得五分。 
 
   好容易忍到下午放学,我找到妖妖对她说:“喂,妖妖,我发现一个好地方!” 
 
   “什么好地方?”“旧书店,里面有无尽其数的好书!!” 
 
   “书?看书有什么意思?不过是小白兔,大萝卜之类。我每天放学之后都去游泳,你看我把游泳衣都带着呢。你陪我去吧?”“小白兔,大萝卜根本就不是书。你跟我上一次旧书店吧。包你满意。” 
 
   她不大愿意去,不过看我那么兴致勃勃,也不愿扫我的兴。哎呀,那么小的时候我们就学会了誐惜友谊…… 
 
   “老陈,少说废话,否则我叫你傻瓜了!”“傻瓜?你才是傻瓜!你懂得什么叫终生不渝的友谊吗? 
 
   我领着她钻进那个阴暗的书店。我看见“哈克贝利·芬”还在书架上,高兴极了,立刻把她抽下来给妖妖,说:“你看看这本书,担保你喜欢!”我其实就是为了这本书来的,可是为了收买她的兴致把它出卖了。我又在书架翻了一通,找着了一本卡达耶夫的《雾海孤帆》,马上就看入了迷。 
 
   可是我看了一会,还不忘看看妖妖。呵,她简直要钻到书里去了。我真高兴!如果,一个人有什么幸福不要别人来分享,那一定是守财奴在数钱。可是我又发现一点小小的悲哀,就是她把我给她的哈克贝利·芬放到一边去了,捧着看的是另一本。被她从书架上取下来放在一边的书真是不少,足足有五六本:《短剑》、《牛虻》,还有几本。后来我们长大了,这些书看起来就大不足道了。可是当时! 
 
   我看看书店的电钟,六点钟了。昨天被揪过的耳朵还有点痛呢!我说:“妖妖,回家吧!”“急什么,再看一会。”“算了吧! 
 
   明天还能看的。”妖妖抬起头看着我说:“你急什么呀?”“六点了。”妖妖说:“不要紧,到七点再回家。” 
 
   我也真想再看一会,但是揪耳朵的滋味不想在尝了,我坚决地说:“妖妖,我非得回家不可了。” 
 
   “你怎么啦?”我什么也不瞒她。我说:“我妈要揍我。你看我今天早上左耳朵是不是大一点?噢,现在还肿着哪!” 
 
   妖妖伸手轻轻地摸着我的耳朵,声音有点发抖:“痛吗?” 
 
   “废话,不痛我也不着急走了。”“好,咱们走吧。” 
 
   我看看《雾海孤帆》的标价,又把它放下了。其实不贵,只要四毛钱。可是我就剩两毛钱了。妖妖问我:“这书不好吗?”“不,挺有意思。”“那干嘛不把它买回去看?” 
 
   我不瞒她,告诉她我没钱了。她说:“我有钱哪。明天我管我妈要一块钱。她准会给的。我还攒了一些钱,把它拿着吧。” 
 
   她选了好几本,连哈克贝利·芬也在内,交了钱之后书包都塞不下了。她跟我说:“你替我拿几本吧,看完了还我。” 
 
   可是我不敢拿,怕拿回家叫家里人看见。褥子底下放一两本书还可以,多了必然被发现。如果被我妈看见了,那书背后还打着中国书店的戳哪!要是一下翻出四五本来,准说是偷钱去买的,就是说借妖妖的她也不信。所以我就只拿了《雾海孤帆》回家。 
 
   第二天我完全叫《雾海孤帆》迷住了:敖德萨喧闹的街市!阳光!大海!工人的木棚!彼加和巴甫立克的友谊!我看完之后郑重地推荐给妖妖,她也很喜欢。后来她又买了一本《草原上的田庄》,我们也很喜欢:因为这里又可以遇见彼加和巴普立克,而且还那么神妙地写了威尼斯、那波里和瑞士。不过我们一致认为比《雾海孤帆》差多了。 
 
   后来我们又看了无数的书,每一本到现在我都差不多能背下来。《小癞子》、《在人间》,世界上的好书真多哇! 
 
   有一天,下课以后我被孙主任叫去了。原因是我在上课看《在人间》。他恐怕根本不知道高尔基是谁。刘老师也不知道。我到教导处时他们两个狗男女正在看那本书哪。我不知他们在书里看出什么,反正他们对我说话时口气凶得要命:“陈辉,你知道你思想堕落到什么地步了吗?你看黄色书籍!” 
 
   我当时对高尔基是个什么人已经了解一点,所以不很怕他们的威吓。我说:“什么叫黄色书籍呀?” 
 
   “就是这种书!你看这种书,就快当小流氓了!” 
 
   我猛然想起书里是有一点我不懂的暧昧的地方,看起来让人觉得有点心跳。可是我对小流氓这个称呼坚决反对。我甚至哭了。我说:“你瞎说!高尔基不是流氓!他和列宁都是朋友!”孙主任听了一楞,马上跳起来大发雷霆:“你说谁胡说?你强词夺理!你还敢骗人!这个流氓会和列宁是朋友?你知道列宁是谁吗?你污蔑革命领袖!”这时候校长走了近来,问:“怎么啦?啊,是陈辉!你怎么又不遵守纪律呀?” 
 
   孙主任气呼呼的说:“这问题严重了,非得找家长不可!看黄色小说!校长,这孩子复杂得很,说这个‘割尔基’和列宁是朋友,真会撒谎!” 
 
   校长看了看书皮,笑了:“高尔基,老孙。我告诉你,高尔基是俄国伟大的无产阶级作家,列宁很关心他的写作。这孩子看这书是早了点。你千万别找陈辉的家长,他爸爸是教育局的呢。你让他知道一个教导主任连高尔基是谁都不知道,那可太丢人了。”   我哭着说:“孙主任说我是流氓,我非告诉我爸爸不可。他还说高尔基是流氓作家!他大概根本也不知道列宁是哪国人!” 
 
   孙主任脸都吓白了。校长和刘老师赶紧上来哄我:“你也别太狂了!大人不比你强?你看过几本书? 
 
   你现在不该看这种书,我们是为你好。你上课看小说就对吗?好啦,拿着书走吧,回家别乱说,啊?” 
 
   我拿回了《在人间》,真比老虎嘴里抢下了一头牛还高兴,赶紧就跑。我根本不敢回家去说,家里知道和老师顶了嘴准要揍我。我赶快跑去找妖妖,可是妖妖已经走了。我又想去书店,可是已经晚了。于是我就回家了。 
 
   老王,你看学校就是这么对付我们:看见谁稍微有点与众不同,就要把他扼杀,摧残,直到和别人一样简单不可,否则就是复杂! 
 
   好了,我要告诉你,我们不是天天上书店的:买来的书先得看个烂熟。而且还要两个人凑够七八毛钱时才去。我经常两分、五分的凑给妖妖存着。她也从来不吃冰棍了,连上天然游泳场两分钱的存衣钱也舍不得花。我和她到钓鱼台游了几次泳,都是把衣服放在河边。那一天我被孙主任叫去训的时候,她一个人上书店了,后来我看见她拿了一本薄薄的书在看。过了几天她把那本书拿给我说:“陈辉,这本书好极了!我们以前看过的都没这本好!你放了学不能回家到我家去看吧,别在教室里看。” 
 
   我一看书名:《涅朵奇卡·涅茨瓦诺娃》。我看了这本书,而且终生记住了前半部。 
 
   我到现在还认为这是一本最好的书,顶得上大部头的名著。我觉得人们应该为了它永远纪念陀思妥耶夫司基。 
 
   我永远也忘不了叶菲莫夫的遭遇,它使我日夜不安。并且我灵魂里好象从此有了一个恶魔,它不停地对我说:人生不可空过,伙计!可是人生,尤其是我的人生就要空过了,简直让人发狂。还不如让我和以前一样心安理得地过日子。 
 
   不过这也是后话,不是当时的事情。当时我最感动的是卡加郡主和涅朵奇卡的友谊真让我神醉魂消! 
 
   不过你别咧嘴,我们当时还是小孩呢。喂,你别装伪君子好不好!我当然是坚决的认为妖妖就是──卡加郡主,我的最亲密的朋友。唯一的遗憾是她不是个小男孩。我跟妖妖说了,她反而抱怨我不是个小女孩。可是结果是我们认为我们是朋友,并且永远是朋友。 
 
   不过这样的热情可没维持多长,到了毕业的时候,我们还是很好,但是各考了一个学校。我考了一个男校,妖妖考上了女校五百八十九中。从此就不大见面了。因为妖妖住校。有时在街上走我也不好意思答理她,因为有同学在旁边呢。我也不愿到她家去。为什么呢?因为我们大了,知道害羞了。并且也会把感情深藏起来,生怕人家看到。不过我从来没有忘记她,后来有一段时间根本没有看见她。中学里很热闹,我有很多事情干呢,甚至不常想起她来。 
 
   可是后来女五百八十九中解散了,分了一部分到我们学校来插班,我们学校从此就成了男女合校。那是初二的事情。妖妖正好分在我们班! 

\section{二、人妖(续)}
 
   那天下午,老师叫我们在教室里等着欢迎新同学。当然了,大家都很不感兴趣,纷纷溜走,只剩下班干部和几个老实分子。我一听说是五百八十九中,就有点心怀鬼胎,坐在那里不走。 
 
   我听见走廊里人声喧哗,好象有一大群女生走了进来,她们一边走一边说,细心听去,好象在谈论校舍如何如何。忽然门砰的一声开了,班主任走进来说:“欢迎新同学,大家鼓掌!嗯,人都跑到哪儿去了?” 
 
   没人鼓掌,大家都不好意思。她们也不好意思进来,在门口探头探脑。终于有两个大胆的进来了,其余的人也就跟进。我突然看见走在后面的是杨素瑶! 
 
   啊,她长高了,脸也长成了大人的模样:虽然消瘦,但很清秀。身材也很秀气,但是瘦得惊人,不知为什么那么瘦。梳着两条长辫子,不过那是很自然的。长辫子对她瘦长的身材很合适。 
 
   我细细地看她的举止,哎呀,变得多了。她的眼睛在睫毛底下专注地看人,可是有时又机警得像只猫:闪电般地转过身去,目光在搜索,眉毛微微有一点紧皱;然后又放松了,好象一切都明白了。我记得她过去就不是很爱说话的。现在就更显得深沉,嘴唇紧紧地闭着。可是她现在又把脸转向我,微微地一笑,嘴角嘲弄人似的往上一翘。 
 
   后来她们都坐下了,开了个欢迎的班会,然后就散了伙。我出了校门,看见她沿着街道朝东走去。我看看没人注意我,也就尾随而去。可是她走得那么坚决,一路上连头也没回。我不好在街上喊她,更不好意思气喘吁吁地追上去。我看见她拐了个弯,就猛地加快了脚步。可是转过街角往前再也看不见她了。我正在失望,忽然听见她在背后叫:“陈辉!” 
 
   我像个傻子一样地转过身去,看见她站在拐角处的阴凉里,满脸堆笑。她说:“我就知道你得来找我。 
 
   喂,你近来好吗?”我说:“我很好。可是你为什么那么瘦?要不要我每天早上带个馒头给你?” 
 
   她说:“去你的吧!你那么希望人人胖得像猪吗?” 
 
   我想我绝对不希望任何一个人胖得像猪,但是她可以胖一点吧?不对!她还是这个样子好。虽然瘦,但是我想她瘦得很妙。于是我又和她并肩的走。我问:“你上哪里去?” 
 
   “我回家,你不知道我家搬了吗?你上哪儿去?”“我?我上街去买东西。你朝哪儿走?” 
 
   “我上十路汽车站。” 
 
   “对对,我要买盒银翘解毒丸。你知道松鹤年堂吗?就在双支邮局旁边。咱们顺路呢!” 
 
   我和她一起在街上走,胡扯着一些过去的事情。我们又想起了那个旧书店,约好以后去逛逛。又谈起看过的书,好象每一本都妙不可言。我忽然提到:“当然了,最好的书是……”“最好的书是——-“涅……!!!”我突然在她的眼神里看出了制止的神色,就把话吞了下去,噎了个半死。不能再提起那本书了。我再也不是涅朵奇卡,她也不是卡加郡主了。那是孩子时候的事情。忽然她停下来,对我说:“陈辉,这不是松鹤年堂吗?”我抬头一看,说:“呀,我还得到街上去买点东西呢,回来再买药吧。” 
 
   我送她到街口,然后就说:“好,你去上车吧。”可是她朝我狡猾地一笑,扬扬手,走开了。我径直往家走,什么药也没有买。 
 
   可是我感到失望,感到我们好象疏远了。我们现在不是卡加郡主和涅朵奇卡了,也不是彼加和巴普立克了。老王,你挤眉弄眼地干什么!我们现在想要亲近,但是不由自主地亲近不起来。很多话不能说,很多话不敢说。我再不能对她说:妖妖,你最好变成男的。她也不敢说:我家没有男孩子,我要跟我爸爸说,收你当我弟弟。这些话想起来都不好意思,好象小时侯说的蠢话一样,甚至都怕想起来。可是想起那时侯我们那么亲密,又很难舍。我甚至有一个很没有男子气概的念头。对了,妖妖说得真不错,还不如我们永远不长大呢。 
 
   可是第二天,妖妖下了课之后,又在那条街的拐角那儿等我,我也照旧尾随她而去。她笑着问我:“你上哪儿呀?”我又编了个借口:“我上商场买东西,顺便上旧书店看看。你不想上旧书店看看吗?” 
 
   她二话没说,跟我一起钻进了旧书店。 
 
   哎,旧书店呀旧书店,我站在你的书架前,真好比马克·吐温站在了没有汽船的码头上!往日那些无穷无尽的好书哪儿去了呢?书架上净是些《南方来信》和《艳阳天》之类的是书。呵……欠!!   我想,我们在旧书店里如鱼得水的时候,,正是这些宝贝在新书店里撑场面的时候。现在这一流的书也退了下来,到旧书店里来争一席位置,可见…… 
 
   纯粹是为了怀旧,我们选了两本书:《铁流》和《毁灭》。我想起了童年时候的积习,顺手把兜里仅有的两毛钱掏给她。可是她一下就皱起眉头来,把我的手推开。后来大概是想起来这是童年时的习惯,朝我笑了笑,自己去交钱了。 
 
   出了书店,我们一起在街上走。她上车站,我在送她。奇怪的是我今天没有编个口实。她忽然对我说:“陈辉,记得我们一起买了多少书吗?二百五十八本!现在都存在我那儿呢。我算了算总价钱,一百二十一块七毛五。我们整整攒了一年半!不吃零食,游泳走着去,那是多大的毅力呀!对了对了,我应该把那些书给你拿来,你整整两年没看到那些书了。” 
 
   我说:“不用,都放在你那儿吧。”“为什么呢?”“你知道吗?到我手里几天就得丢光!这个来借一本,那个来借一本,谁也不还。” 
 
   那一天我们就没再说别的。我一直送她上汽车,她在汽车上还朝我挥手。 
 
   后来我就经常去送她,开始还找点借口,说是上大街买东西。后来渐渐地连借口也不找了。她每天都在那个拐角等我,然后就一起去汽车站。 
 
   我可以自豪的说,从初二到初三,两年一百零四个星期,不管刮风下雨,我总是要把她送到汽车站再回家。至于学校的活动,我是再也没参加过。 
 
   可是我们在路上谈些什么呢?哎呀,说起来都很不光彩。有时甚至什么也不说,就是默默地送她上了汽车,茫然地看着汽车远去的背影,然后回家。 
 
   有一天我们在街上走,她忽然问我:“陈辉,你喜欢诗吗?” 
 
   那时我正读莱蒙托夫的诗选读得上瘾,就说:“啊,非常喜欢。”后来我们就经常谈诗。她喜欢普希金朴素的长诗,连童话诗都喜欢。可是我喜欢的是莱蒙托夫那种不朽的抒情短诗。我们甚至为了这两种诗的优劣争执起来。为了说服我,她给我背诵了青铜骑士的楔子,我简直没法形容她是怎么念出:我爱你,彼得建造的大城……她不知不觉在离车站十几米的报亭边停住了,直到她把诗背完。 
 
   可是我也给她念了:《我爱这连绵不断的青山》和《遥远的星星是明亮的》。那一天我们很晚才分手。 
 
   有一天学校开大会,我们出来的时候已经很晚了。那是五月间的事情。白天下了一场雨。可是晚上又很冷。没有风。结果是起了雨雾。天黑得很早。沿街楼房的窗户上喷着一团团白色的光。大街上,水银灯在在半天织起了冲天的白雾。人、汽车隐隐约约地出现和消失。我们走到十路汽车站旁。几盏昏暗的路灯下,人们就像在水底一样。我们无言地走着,妖妖忽然问我:“你看这个夜雾,我们怎么形容它呢?”我鬼使神差地做起诗来,并且马上念出来。要知道我过去根本不认为自己有一点作诗的天分。 
 
   我说:“妖妖,你看那水银灯的灯光像什么?大团的蒲公英浮在街道的河流口,吞吐着柔软的针一样的光。”妖妖说:“好,那么我们在人行道上走呢?这昏吰的路灯呢?” 
 
   我抬头看看路灯,它把昏黄的灯光隔着蒙蒙的雾气一直投向地面。 
 
   我说:“我们好象在池塘的水底。从一个月亮走向另一个月亮。” 
 
   妖妖忽然大惊小怪地叫起来:“陈辉,你是诗人呢!”我说:“我是诗人?不错,当然我是诗人。” 
 
   “你怎么啦?我说真的呢!你很可以做一个不坏的诗人。你有真正的诗人气质!” 
 
   “你别拿我开心了。你倒可以做个诗人,真的!” 
 
   “我做不成。我是女的,要做也只能成个蓝袜子。哎呀,蓝袜子写的东西真可怕。” 
 
   “你什么时候看到过蓝袜子写的东西?” 
 
   “你怎么那么糊涂?我说蓝袜子,就是泛指那些没才能的女作家。比方说乔治·爱略特之流。女的要是没本事,写起东西来比之男的更是十倍的要不得。”“具体一点说呢?” 
 
   “空虚,就是空虚。陈辉,我不是跟你开玩笑,你一定可以当个诗人!退一万步说,你也可以当个散文家。莱蒙托夫你不能比,你怎么也比田间强吧?高尔基你不能比,怎么也比杨朔、朱自清强吧?” 
 
   我叫了起来:“田间、朱自清、杨朔!!!妖妖,你叫我干什么?你干脆用钢笔尖扎死我吧!我要是站在阎王爷面前,他老爷子要我在作狗和杨朔一流作家中选一样,我一定毫不犹豫的选了作狗,哪怕作一只赖皮狗!” 
 
   妖妖哈哈大笑起来,笑了又笑,连连说:“我要笑死了,我活不了啦……哈哈,陈辉,你真有了不得的幽默感!哎呀,我得回家了不过你不要以为我在和你开玩笑,你可以作个诗人!” 
 
   她走了。可是我心里像开了锅一样蒸汽腾腾,摸不着头脑。她多么坚决地相信自己的话!也许,我真的可以作个诗人?可是我实际上根本没当什么诗人。老王,你看我现在坐在你身旁,可怜的像个没毛的鹌鹑,心里痛苦。思想正在听样板戏,哪里谈得上什么诗人!” 
 
   我说:“老陈,你别不要脸了。你简直酸得像串青葡萄!” 
 
   你听着!你要是遇见过这种事,你就不会这么不是东西了。这以后,我就没有和妖妖独自在一起呆过了。我还能记得起她是什么样子吗?最后见到她已经是七年前的事情了。啊!我能记得起的!她是──她是瘦小的身材,消瘦的脸,眼睛真大啊。可爱的双眼皮,棕色的眼睛!对着我的时候这眼睛永远微笑而那么有光彩。光洁的小额头,孩子气的眉毛,既不太浓,也不太疏,长的那么恰好,稍微有点弯。端立的鼻子,坚决的小嘴,消瘦的小脸,那么秀气!柔软的棕色发辫。脖子也那么瘦:微微的动一下就可以看见肌肉在活动。小姑娘似的身材,少女的特征只能看出那么一点。喂,你的小手多瘦哇,你的手腕多细哇,我都不敢握你的手。你怎么光笑不说话?妖妖,我到处找你,找了你七年!我没忘记你!我真的一刻也不敢忘记你,妖妖!” 
 
   老陈站起来,歇斯底里朝前俯着身子,眼睛发直,好象瞎了一样,弄得过路人都在看他。我吓坏了,一把把他扯坐下来,咬着耳朵对他说:“你疯了!想进安定医院哪!” 
 
   老陈呆呆的坐了一会,然后茫然地擦了擦头上的汗。 
 
   “我刚才看见她了,就像七年前一样。我讲到哪儿了?” 
 
   “讲到她说你是个诗人,”对对,后来过了几天,就开始文化大革命了。后来就是大串联!我走遍了全国各地。逛了两年!我和着了魔一样!后来我回到北京,我又想起了妖妖。我想再和她见面,就回到学校。可是她再也没来过学校。我在学校里等了她一年!我不知道她家住在哪儿,我也没有地方去打听!后来我就去陕西了。 
 
   我在陕西非常苦闷!我渐渐开始想念她,非常非常想念她!我明白了,圣经里说亚当说夏娃是他骨中的骨,肉中的肉,对,就是这么一回事!她是我骨中的骨,肉中的肉。可是到哪里去找她? 
 
   后来我又回到北京,可是并不快乐。可是有一天,我在家里坐着,眼睛突然看见书架上有一本熟悉的书,精装的《雾海孤帆》。那是我童年读过的一本,虽然旧了,但是决不会认错的。老王,假如你真正爱过书的话,你就会明白,一本在你手中呆过很长时间的好书就像一张熟悉的面孔一样,永远不会忘记。那就是我和她在旧书店买的那一本!可是我记得它在妖妖那儿呀!我简直不能想象出它是在哪儿冒出来的。还认为是我记错了,我看起它,无心去看,但是翻了一翻,还想重温一个童年的旧梦。忽然里头翻出个纸条来,上面的话我一字不漏地记得:陈辉:我家住在建国路永安东里九楼431号,来找我吧。杨素瑶1969年4月7日那正是我到陕西去的第三天!我拿着书去问我妈,这书是谁送来的。我妈很没害臊的说:“是个大姑娘,长得可漂亮了。大概是两年前送来的吧。” 
 
   我骑上车子就跑!找到永安东里九楼的时候,我连上楼的力气都没有了。腿软得很。心跳得要命,好象得了心律过速。我敲了敲她家的门,有人来开门了!我想把她一把抱住,可是抱住了一个摇头晃脑的老太太。老太太可怕得要命!眼皮干枯,满头白发,还有摇头疯,活象一个鬼! 
 
   我问:“杨素瑶在家吗?”老太太一下愣住了:“你是谁?”“我,我是她的同学,我叫陈辉。” 
 
   “你是陈辉!进来吧,快进来。哎呀……(老太太哭了,没命地摇头)小瑶,小瑶已经死啦!” 
 
   我发了蒙,一切好象在九重雾里。我记得老太太哭哭啼啼地说她回老家去插队,有一次在海边游泳,游到深海就没回来。她哭着说:孩子,我就这么一个女儿呀!我为什么让她回老家呢?我为什么要让她到海边去呢?呜呜! 
 
   我听老太太告诉我,说妖妖在信中经常提到说:如果陈辉来找她就赶快写信告诉她。我陪老太太坐到天黑,也流了不少眼泪。这是平生唯一的一次!等到我离开她家的时候,在楼梯上又被一个姑娘拦住了。 
 
   她说:“你叫陈辉吧?”我木然答道:“是,我是陈辉。” 
 
   “我的邻居杨素瑶叫我把这封信交给你,可惜你来的太晚了。” 
 
   我到家拆开了这封信,这封信我也背得上来:陈辉:你好!我在北京等了你一年,可是你没有来。 
 
   你现在好吗?你还记得你童年的朋友吗?如果你有更亲密的朋友,我也没有理由埋怨你。你和我好好地说一声再见吧。我感谢你曾经送过我两千五百里路,就是你从学校到汽车站再回家的六百二十四个来回中走过的路。如果你还没有,请你到山东来找我吧。我是你永远不变的忠实的朋友杨素瑶。 
 
   我要去的地方是山东海阳县葫芦公社地瓜蛋子大队。 
 
   老陈讲到这里,掏出手绢擦擦眼睛。我深受感动。站起身来准备走了。可是老陈又叫住了我。他:“你上 哪儿去?我还没讲完呢!。后来我和她又见了一面。” 
 
   “胡说!你又要用什么显魂之类的无稽之谈来骗我了吧?” 
 
   “你才是胡说!你这个笨蛋。这件事情你一定要怀疑不是真的,可是我愿用生命担保它的真实性。 
 
   要不是亲身经历过,我也不相信这是真的。你听着!” 
 
   他又继续讲下去。如果他刚才讲过的东西因为感情真挚使我相信有这么一回事的话,这一回老陈可就使我完全怀疑他的全部故事的真实性了。不是怀疑,他毫无疑问是在胡说!下面就是他讲的故事── 

\section{三、绿毛水怪} 

 
   后来我在北京呆不下去了,也回了山东老家。至于老家嘛,简直没有什么可说的。闭塞得很,人也很无知。我所爱的是那个大海。我在海边一个公社当广播员兼电工。生活空虚透了,真像爱略特的小说!唯一的安慰是在海边上!海是一个永远不讨厌的朋友!你懂吗?也许是气势磅礴的朝岸边推涌,好象要把陆地吞下去;也许不尽是朝沙滩发出的浪,也许是死一样静,连一丝波纹也兴不起来。但是浩瀚无际,广大的蔚蓝色一片,直到和天空的蔚蓝联合在一起,却永远不会改!我看着它,我的朋友葬身大海,想着它多大呀,无穷无尽地大;多深哪,我经常假想站在海底看着头上茫茫的一片波浪,像银子一样。 
 
   我甚至有一点高兴,妖妖倒找到一个不错的葬身之所!我还有些非非之想,觉得她若有灵魂的话,在海底一定是幸福的了。 
 
   可是在海中远远的有一片礁石,退潮的时候就是黑黑的一大捧,你可以把它想象成很多东西,一片新大陆,圣海伦岛,之类之类。涨潮的时候就是可笑的一点点,好象在引诱你去那里领受大海的嬉戏。如果是夏天,我每天傍晚到大海里游泳,直到筋疲力尽时,就爬到那里去休息一下。真是个好地方!离岸足有三里地呢。在那里往前看,大海好象才真正把它宽广地显示给你…… 
 
   有一天傍晚时分我又来到了海滨。那一天海真像一面镜子!只有在沙滩尽边上,才有海水最不引人注意地在抽溅…… 
 
   我把衣服藏在一块石头底下,朝大海里走去。夕阳的余辉正在西边消逝,整个天空好象被红蓝铅笔各涂了一半。。海水浸到了我的腰际,心里又是一阵隐痛……你知道,我听说她死已经是一年前的事情,是一件已经无法挽回的事情了。这种痛苦对于我已经转入了慢性期,偶尔发作一下。我朝大海扑去,游了起来。我朝着那丛礁石游,看着它渐渐大起来,我来了一阵矫健的自由式,直冲到那两片礁石上。你要知道那是一大片犬牙交错的怪石,其实在水下是其大无比的一块,足有二亩地大。 
 
   一个个小型的石峰耸出水面,高的有一人多高,矮的刚刚露出水面一点儿。在那些乱石之间水很浅,可是水底下非常的崎岖不平。我想,若千万年前,这里大概是一个石头的孤岛,后来被波涛的威力所摧平。 
 
   我爬到最高的一块礁石上。这一块礁石约有两米高,形状是酷似一颗巨大的臼齿。我就躺在凹槽里,听着海水在这片礁石之间的轰鸣。天渐渐暗下来。我从礁石后面看去!黑暗首先在波浪间出现。海水有点发黑了。 
 
   “该回去了。不然就要看不见岸了。”我在心里请清楚楚地说。找不着岸,那可就糟了。只有等着星星出来才敢往回游,要是天气变坏,就得在石头上过一夜,非把我冷出病来不可!我可没那么大瘾! 
 
   我站起身来,眼睛无意间朝礁石中一扫:嗬,把我吓出了一身冷汗!我看见,在礁石中间,有一个好像人的东西在朝一块礁石上爬! 
 
   我一下把身子蹲下,从石头后面小心地看去,那个怪物背对着我。它全身墨绿,就像深潭里的青苔,南方的水蚂蝗,在动物身上这是最让人憎恶的颜色了。可是它又非常地像一个人,宽阔的背部,发达的肌肉和人一般无异。我可以认为它是一个绿种人,但是它又比人多了一样东西,就其形状来讲,就和蝙蝠的翅膀是一样的,只是有一米多长,也是墨绿色的,完全展开了,紧紧地附在岩石上。蝙蝠的翅膀靠趾骨来支撑。在这怪物的翅膀中,也长了根趾骨,也有个爪子伸出薄膜之外紧紧地抓住岩石。 
 
   它用爪子抓住岩石,加上一只手的帮助,缓缓地朝上爬,而一只手抓着一杆三箘叉,齿锋锐利,闪闪有光,无疑是一件人类智慧的产物。可是我并不因为这个怪物有人间兵器而产生什么生理上的好感:因为它有翅膀又有手,尽管像人,比两个头的怪物还可怕。你知道,就连鱼也只有一对前鳍,有两对前肢的东西,只有昆虫类里才有。 
 
   它慢慢把身体抬出水面。不管怎么说,他无疑很像一个成年的男子,体形还很健美,下肢唯一与人不同的地方就是因为水下生活腿好象很柔软,而且手是圆形的,好象并在一起就可以成为很好的流线体。脚上五趾的形象还在,可是上面长了一层很长很宽的蹼,长出足尖足有半尺。头顶上戴了一顶尖尖的铜盔。我是古希腊人的话,一定不感到奇怪。可我是一个现代人哪。我又发现他腰间拴了一条大皮带,皮带上带了一把大得可怕的短剑:根本没有鞘,只是拴着剑把挂在那里。 
 
   我不大想和他打交道。他装备得太齐全了,体格太强壮了。可是我又那么骨瘦如柴。我想再看一会,但是不想惊动他。因为如果他有什么歹意,我绝对不是个。 
 
   我必须先看好一条逃路,要能够不被他发信地溜到海里去,并且要让人在相当长的距离里看不见我,再远一点,因为天黑,在波浪里一个人头都和根木头看起来差不多了。我回头朝后看看地势,猛然吓出了一身冷汗:原来身后的礁石上也爬上来好几个同样的怪物,还有女的。女的看起来样子很俊美,一头长长的绿头发,一直披到腰际。可就是头发看起来很粗,湿淋淋地像一把水藻。 
 
   他们都把翅膀伸开钩住岩石,赤裸的皮肤很有光泽。至于装扮和第一个差不多。头上都有铜盔,手里也都拿着长茅或钢叉,离我非常之近!最远的不过十米,可是居然谁也没发现我。可是我现在真是无路可逃了。我找不到一个地方可以躲出他们的交叉视线之外,如果一头跳下去,那更是没指望。这班家伙在水里追上我是毫无问题的;在水里搞掉我更比在礁石上容易。 
 
   我下了一个勇士的决心,坚决地站了起来,把手交叉在胸前,傲慢地看着他们。第一个上岸的水怪发现我了,他拄着钢叉站了起来,朝我一笑,着一笑在我看来是不怀好意。他一笑我还看见了他的牙齿:雪白雪白,可是犬齿十分发达。我认为自己完了。这无疑是十分不善良的生物,对我又怀有十分不善良的用心!我在一瞬间慌忙地回顾了一下自己的一生:有很多后悔的地方。可是到这步田地,也没有什么太可留恋、叫我伤心得流泪的东西。我仔细一想,我决不向他乞怜,那不是男子汉的作为。相反的,我唯一要做到的就是死得漂亮一些。我迎上几步对他说:“喂,伙计,听得懂人的话吗?我不想逃跑了。逃不过你们,抵抗又没意思,你把刀递过来吧,不用你们笨手笨脚地动手!”他摇摇头,好象是不同意,又好象不理解。然后伸手招我过去。 
 
   我说:“啊,想吃活的,新鲜!那也由你!”我绝不会容他们生吞活剥的。我要麻痹他的警惕性,然后夺下叉子,拼个痛快!可是我耳边突然响起了一阵笑声。那水怪大声笑着对我说:“你把我们当成什么了?食人生番?哈哈!” 
 
   其他的水怪也随着他一起大笑。我非常吃惊。因为他说的一口美妙的普通话,就口音来说毫无疑问是中国人。我问:“那么您是什么……人呢?”“什么人?绿种人!海洋的公民!懂吗?”“不懂!” 
 
   “告诉你吧。我过去和你恐怕还是同乡呢!我,还有我们这些伙计,都是吃了一种药变成这个样子的。我门现在在大海里生活。” 
 
   “大海里?吃生鱼?(他点点头)成天在海水里泡着?喂,伙计,你不想再吃一种药变回来吗?” 
 
   “还没有发明这种药。但是变不回来很好。我们在海里过得很称心如意。” 
 
   “恐怕未必吧。海里有鲨鱼,逆戟鲸,还有一些十分可怕的东西。大海里大概也不能生火,只能捉些小鱼生吃。恐怕你们也不会给鱼开膛,连肠子一起生嚼,还觉得很美。晚上呢?爬到礁石上露宿。” 
 
   像游魂一样地在海里漂泊!终日提心吊胆!我看你们可以向渔业公司去报到。这样你们就可以一半时间在岸上舒服的房间里过。我想你们对他们很有用。 
 
   “哈哈,渔业公司!小伙子,你的胆量大起来了,刚才你还以为我们要吃你当晚饭!你把我们估计得太简单了。鲨鱼肉很臊,不然我们准要天天吃它的肉。告诉你,海里我们是霸王!鲨鱼无非有几颗大牙,你看看我们的钢叉!海里除了剑鱼什么也及不上我们的速度。我们吃的东西吗,当然是生鱼为主。无可否认,吃的方面我们不大讲究。但是也有一些东西是你们享用不到的。你知道鲜海蛰的滋味吗?龙虾螃蟹,牡蛎海参”…… 
 
   我大叫一声:“你快别说了,我要吐了。我一辈子也不吃海里的玩意!” 
 
   “是吗?那也不要紧,慢慢会习惯的。小伙子,我看你还有点种。参加我们的队伍吧!吃的当然比不上路易十四,可是我看你也不是爱吃的人,不然你就不会这么瘦了。跟我们一起去吧。海里世界大得很呢。它有无数的高山竣岭,平原大川,辽阔得不可想象!还有太平洋的珊瑚礁:真是一座重重叠叠的宝石山!我可以告诉你,海是一个美妙的地方,一切都笼罩着一层蓝色的宝石光!我们可以像飞快的鱼雷一样穿过鱼群,像你早上穿过一群蝴蝶一样。傍晚的时候我们就乘风飞起,看看月光照临的环行湖。我们也常常深入陆地,美国的五大淡水湖我们去过,刚果河,亚马逊我们差一点游到了源头。半夜时分,我们飞到威尼斯的铅房顶上。我们看见过海底喷发的火山,地中海神秘的废墟。 
 
   “海底有无数的沉船是我们的宝库”……“不过你们还是一群动物,和海豚没什么两样。” 
 
   “是吗?你如果这么认为就大错特错了。我们中间有学者。我在海中碰上过四个剑桥的大学生,五个牛津的。有一个家伙还邀我们去看他的实验室:设在一个珊瑚礁的山洞里。哈哈,我们中间真有一些好家伙!迟早我们海中人能建立一个强国,让你们望而生畏;不过还得我们愿意。总的来说,我们是不愿意欺负人的,不过,现在我们不想和你们打交道,甚至你们都不知道海里有我们。可是你们要是把海也想的乌烟瘴气的话,我们满可以和你们干一仗的。” 
 
   “啊!我是不是在和海洋共和国外交部长说话?” 
 
   “不是,哈!哪有什么海洋共和国!只不过我们在海底碰上的同类都有这样的意见。” 
 
   “哈哈,这么说,所谓海底强国的公民,现在正三五成群地在大海里漫游,和过去的蒙古人一样?” 
 
   “笑什么?当然在某种程度上是这样,可也有人在海底某处定居,搞搞科研,甚至有相当规模的工业,相当规模的城市,有人制造水下猎枪,有人回冷锻盖房子的铅板,有人给水下城市制造街灯。还可以告诉你,有人在研究和陆地打一场核战争的计划,作为一种有备无患的考虑。” 
 
   “真的么?哎呀,这个世界更住不得了。” 
 
   “你不信吗?你可以去看看!只要你加入我们的行列,你就知道我说得不假了。陆地上的对海洋知道什么?海大得很!海底什么没有啊!…… 
 
   告诉你,我们可不是食人生番。今天晚上我们要到济州岛东面的岩洞音乐厅去听水下音乐会。水下音乐!岸上的音乐真可怜哪。我们有的是诗人和其他艺术家,在海底,象征派艺术正在流行。得啦,告诉你的不少了,你来不来?”“不来!我从小就不能吃鱼,闻见腥味就要吐,哎呀,你身上真腥!”…… 
 
   “你不来就算了,为什么要侮辱人?你不怕我吃你?刚才你还全身发抖,现在就这么张狂!好啦,回去不要跟别人说你碰见水怪了。不过你说也无妨,反正不会有人相信。”我点点头。这时天已经很暗了,周围成了黑白两色的世界,而且是黑色的居多。只有最近的东西才能辨出颜色。最后的天光在波浪上跳跃。我看看远处模糊的海岸,真想和海怪们告辞了。可是我忽然听见有人在背后叫:“陈辉!” 
 
   我回头一看:有一个女水怪,半截身子还在水里,伏在礁石上,一顶头盔放在礁石上,长长的头发披下来遮掩住了她的身躯。可是她朝我伸出一条手臂低低地叫着:“陈辉!” 
 
   声音是陌生的低沉,她又是那么丰满而柔软,像一只海豹。但是我认出了她的面容,她独一无二的笑容,我在天涯海角也能认出来,她是我的妖妖! 
 
   我打了个寒噤,但是一个箭步就到她跟前,在礁石上跪下对她俯下身子,把头靠在她的头发上。 
 
   她伸出手臂,抱住我的脖子。哎呀,她的胳膊那么凉,好象一条鱼!我老实跟你说,当时把我吓了一大跳,不由自主地想把它拿下来。 
 
   我们静默了一会,忽然其它的水怪大笑起来。和我说话的那一个大笑着说:“哈哈!他就是陈辉! 
 
   在这儿碰上了!伙计们,咱们走吧!” 
 
   他们一齐跳下水去。强健的两腿在身后泛起一片浪花,把上身抬出水面,右手高举钢叉,在水面上排成一排,疾驰而去,好象是海神波赛顿的仪仗。 
 
   等到他们在远处消失,妖妖就把双手紧紧地抱住我的脖子。我打了一个寒噤,猛一下挣开了,不由自主地说:“妖妖,你像一个死人一样凉!”她从石头上撑起身子看看我,猛然双眼噙满了泪,大发雷霆:“对了对了,我像死人一样凉,你还要说我像鱼一样腥吧?可是你有良心吗?一去四五年,连个影子也不见。现在还来说风凉话!你怎么会有良心?我怎么瞎了眼,问你有没有良心?你当然不会有什么良心!你根本不记得有我!” 
 
   我吃了一惊:“你怎么了?你为什么要说这种话?我到处找你!我怎么会知道你当了……海里的人?” 
 
   “啐!你直说当了水怪好了。我怎么知道还会遇上你?啊?我等了你四年,最后终于死了心。然后没办法才当了水怪。我以为当水怪会痛快一些,谁知你又冒了出来?可是我怎么变回去呢?我们离开海水二十四个小时就会干死!” 
 
   “妖妖,你当水怪当得野了,不识人了。你怎么知道我不愿意和你一起当水怪了呢?” 
 
   “啊?真的吗?我刚才还听见你说死也不当水怪呢!”“此一时彼一时也。你把你们的药拿来吧。” 
 
   “可是你怎么不早说呢?药都由刚才和你说话的人带着,他们现在起码游出十五海里了!” 
 
   我觉得头里轰的一声响,眼前金星乱冒,愣在那里像个傻瓜。我听见妖妖带着哭声说:“怎么啦陈辉,你别急呀,你怎么了?别那么瞪着眼,我害怕呀!喂!我可以找他们去要点药来,明天你就可以永远和我在一块了!” 
 
   我猛然从麻木中惊醒:“真的吗?对了,你可以找他们去要的,我怎么那么傻,居然没有想到?哈哈,我真是个傻瓜!你快去吧,我在这里等你。半个小时能回来吗?” 
 
   “半个小时!陈辉,你不懂我们的事情。他们走了半个多钟头了。大概离这儿三十五里。我们用最快的速度去追,啊,大概七个小时能追上他们。然后再回来,如果不迷失方向,明天中午可以到。 
 
   我们这些人根本就不会慢慢遛哒,在海里总是高速行驶,谁要是晚走一天就得拼命地赶一个月。我大概不能在途中追上他们,得到济州岛去找他们了。” 
 
   “那好,我就在这儿等你,明天中午你还上这儿找我吧。” 
 
   “你就在这礁石上过夜吗?我的天,你要冻病的!一会要涨潮了,你要泡在水里的!后半夜估计还有大风,你会丧命的!我送你上岸吧!” 
 
   “你怎么送我上岸?背着我吗?我的天,真是笑话!你快走吧,我自己游得回去。星星快出来了,我能找着岸。明天中午我在这里等你,你快走吧!” 
 
   这时候整个天空已经暗下来,只有西面天边的几片云彩的边缘上还闪着光。海面上起了一片片黑色的波涛,沉重地打在脚下。不知什么时候起了风,现在已经很大了。水不知不觉已经涨到了脚下,又把溅起的飞沫吹到身上。我觉得很冷。尽力忍着,不让上下牙打架。 
 
   妖妖抬起头,仔细地看了我一眼,然后“嗵”地一声跃入海里。等到我把脸上的水抹掉,她已经游出很远了。我看到她迎着波涛冲去,黑色的身躯两侧泛起白色的浪花。她朝着广阔无垠的大海──无穷无尽的波涛,昏暗无光之下的一片黑色的、广漠浩瀚的大海游去了。我看见,她在离我大约半里地的地方停下了,在汹涌的海面上把头高高抬出海面在朝我了望。我站起来朝她挥手。她也挥了挥手,然后转身,明显加快了速度,像一颗鱼雷一样穿过波浪,猛然间,她跃出水面,张开背上的翅膀在水面上滑翔了一会,然后像蝙蝠一样扑动翅膀,飞上了天空。转瞬之间就变成了一个天上的小黑点。 
 
   我尽力注视着她,可是不知在那一瞬间,那个黑点忽然看不见了。我看看北面天上,北斗七星已经能看见了,也就跳下海去。 
 
   那一夜正好刮北风,浪直把我朝岸上送。不过尽管如此,到了岸上。不过尽管如此,到了岸上,天已经黑得可怕。一爬出水来,风一吹,浑身皮肉乱颤。我已经摸不清在哪儿上的岸,衣服也找不到了。幸亏公社的会议室灯火通明,怕上一个小山就看见了,我就摸着黑朝它走去。 
 
   我到现在也不知那一夜我走的是些什么路,只觉得脚下时而是土埂,时而是水沟,七上八下的,栽了无数的跟头。黑暗里真是什么也看不见。不一会,我就觉得身上发烧,头也晕沉沉的。我栽倒了又爬起来,然后又栽倒,真恨不得在地上爬!看起来,好象路不远,可是天知道我走了多久! 
 
   后来总算到了。我摸回宿舍,连脚也没洗,赶快上床,拉条被子捂上:因为我自己觉得已经不妙了,身上软得要命。我当时还以为是感冒,可是过一会,身上燥热不堪,头脑晕沉,思想再也集中不起来,后来意识就模糊了。 
 
   半夜时分,我记得电灯亮了一次,有人摸我的额头。然后又有两个人在我床头说话。我模模糊糊听见他们的话:“大叶肺炎……热度挺高……不要紧他体质很好……” 
 
   然后有人给我打了一针。我当时虽然头脑昏乱,但是还是想:“坏了,明天不知能不能好?还能去吗?可是一定要去!”然后就昏昏睡去。 
 
   等我醒来,只觉得头痛得厉害,可是意识清醒多了。屋里一个人也没有,但是天已经大亮。我看看闹钟,吓了一跳:已经两点半了。我拼命挣扎起来,穿上拖鞋,刚一起立,脑袋就嗡嗡作响,勉强走到门口,一握门把,全身就坠在地上。我在地上躺了一会,等到地上的凉气把身上的冰得好过一点,又拼命站起来。我尽力不打晃,在心里坚定地喊着:一!二!一!振作起精神,开步走到院里,眼睛死盯着院门,走过去。 
 
   忽然有人一把捉住我的手。我一回头,脑袋一转,头又晕了。我看见一张大脸,模模糊糊只觉得上面一张大嘴。后来看清是同住的小马。他朝我拼命地喊着什么,可是我一点也听不见。猛然我勃然大怒觉得他很无礼,就拼命挥起一拳把他打倒。然后转身刚走了一步,腿一软也倒下了,随即失去了知觉。 
 
   以后我就什么也看不见了:眼前一片黄雾,只偶然能听见一点。我再朦胧中听见有人说:“反应性精神病……因高烧所致”,我就大喊,“放屁!你爷爷什么病也没有!快把我送到海边,有人再那等我!(然后又胡喊了一阵,)妖妖!快把药拿来呀!拿来救我的命呀!……” 
 
   后来我在公社医院里醒来了,连手带脚都被人捆在床上。我明白,这回不能是使蛮的了。如果再说要到海边去,就得被人加上几根绳索。我嬉皮笑脸地对护士说:“大姐,你把我放了吧。我都好了,捆我干什么?”护士报告医生,医生说等烧退了才能放。我再三哀求也不管用。 
 
   过了半天,医生终于许可放开我了。一等护士离开,我就从窗户里跳了出去,赤着脚奔到海边。可是等我游到礁石上,看见了什么呢?空无一物!在我遇到妖妖的那块石头上,有一片刀刻的字迹:陈辉,祝你在岸上过得好,永别了。但是你不该骗我的。杨素瑶。 
 
   老陈猛一下停了下来,双手抱住头。停一会抬起头的时候,我看见他眼里噙满了泪。他大概看见我满脸奸笑,霍的一下坐直了:“老王,我真是对牛弹琴了!”我说:“怎么,你以为我会信以为真么?” 
 
   “你可以不信,”……“我为什么要信,”“但是我怎么会瞎了眼,把你当成个知音!再见老王,你是个混蛋!”“再见,老陈,绿毛水怪的朋友先生,候补绿毛水怪先生!” 
 
   忽然老陈眼里冒出火来,他猛地朝我扑来。所以到分手的时候,我带着两个青眼窝回家。 
 
   可是你们见过这样的人吗?编了一个弥天大谎,却硬要别人相信?甚至动手打人!可是我挨了打,我打不过他,被他骑着揍了一顿……世上还有天理吗? 
 
