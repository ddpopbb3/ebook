\chapter{东宫·西宫}

  1 

  小史家房间——内——日 

  小史拉开抽屉,里面放的东西之中,有一把剪子,起初,他想把信封扔进去,但是又改变了主意,把剪子拿了出来。在剪开信封之前,小史回头看了看空空的房间。然后操剪刀剪 开信封,里边是一本紫色的书。他摩挲着书的封面,叉开手指,手微屈,用指尖轻触扉页上赫然写着的:献给我的爱人。 

  他又把书合上,放进抽屉,上了锁,然后站起身来在室内缓缓地走动。 

  他突然停下,他的耳边响起了悠长的无歌词的昆曲之音。 

  2 

  公园——外——傍晚 

  昆曲继续在回荡着。 

  深秋的一个细雨蒙蒙的傍晚。这里有湖有山,还有一段城墙作围墙。三三两两的人散布在城墙下、城墙附近的绿地及假 

  山上。这些人站在小雨里,打着伞,穿着雨衣,或干脆冒着雨。 

  他们小声交谈着,雨水浸湿了他们的鞋,打湿他们的衣服,雨水顺着他们的头发流到脸颊上。我们看不到他们的面孔,听不清 

  他们的谈话。已是傍晚时分,高大的城墙爬满了深绿色的叶子,厚厚的,湿漉漉的。 

  阿兰在阴雨中出现。 

  阿兰的画外音(很低沉):“我们这里有很多雨,烟雨漉漉,冷冰冰的。所以,我的身体总被水汽包围。到处都是这种软绵绵, 

  弥散着的水。这世界上如果还有雨以外的东西,就是我了…… 

  仿佛在天地之间,我是唯一的肉体。有时候,我真想融化在雨里。 

  3 

  同上,但天色更黑。场上的人也挨得更近。 

  小史穿着警服出现,走向派出所。 

  小史的画外音:“我们这个公园。老有一些男的在这儿腻歪。 

  这些孙子有毛病。我们也不想管他们的事,但是不管又不成。” 

  4 

  公园树林里——外——夜 

  突然,传来一阵嘈杂声,几个手电筒的光柱在不远处不断闪 

  动。只听见几个人在高喊:“都别跑了!站住!”听到的更多的是许多人在泥地里的乱跑声,以及挨打时发出的痛苦声音。顿时, 

  树林中,一片慌乱,可是没人喊叫。有的人动作麻利,夺路就逃,在逃走的人中间,阿兰从容不迫地走着。高大英俊的警察小勇和两个联防队员手中拿着手电筒冲了过来。 

  5 

  城墙边——外——夜 

  警察小史:“手扶在墙上,站成一排!不许乱动!”挨个地拧脑袋用手电照,看是谁。 

  阿兰非常的顺从,又带有几分潇洒。他面朝墙站着,头也不回,仿佛对身后的事漠不关心——其他人都禁不住要回头的。 

  然后警察小史伸手把阿兰的脸掰转过来,阿兰表情平静。 

  小史旁白:“那天晚上我逮住他时,他就是这样的满不在乎。假如不是看他眼熟,我会以为逮错人了呢。” 

  6 

  公园的树丛里——外——夜 

  警察小史一只手打着手电,一只手抓住阿兰的胳膊,推着他往前走。阿兰顺从,很自然地半倚着小史。走着走着,就像对一个老朋友一样,把手放在小史背上,很狎熟、很随意地抚摸小史,一直摸到屁股上。小史震惊而不适,但很奇怪,他一直没有发作 

  打阿兰,一直是想发作又发作不了的样子。后来,他放开了阿兰,自己朝前走。片刻后回头,阿兰站在原地,似在目送他。又过了片刻,小史下了决心,转过身来,想要去逮住阿兰,但是阿兰在从容地走开。现在似乎没有理由再去逮他了。 

  小史旁白:“那天晚上,我真不知自己是怎么了。” 

  7 

  公园里的林荫道——外——日 

  小史的画外音:“我该打丫的一顿。我知道,这孙子是个作家,叫阿兰。他老上这儿来。听说他很贱。好哇,犯贱犯到我身 

  上来了……没关系,跑得了初一,跑不了十五。早晚我还会逮着他。” 

  阿兰长时间地坐在林荫道边的长椅上,表情慵懒,把右手放在长椅上,轻轻地摩挲着长椅的板条(手好像做着下意识的动 

  作)。看着过往的行人。这时他有三十多岁了,但仍然很漂亮,他的脸似乎还化了一些淡妆。东张西望的样子很突出。 

  8 

  路旁树荫下——外——日 

  一个搔首弄姿、步态蹒跚的人走过,阿兰久久地盯住,直到看不见时为止。 

  阿兰看到一个长得漂亮的人,他站起来盯梢,在公园里转了好几圈,被盯的人也时时停下来看他是不是跟着。这一切就像 

  特务接头一样,双方都很谨慎。直到那个人站下来和他攀谈。阿兰的画外音:“我每天都出来,最近也这样。这个朋友告 

  诉我说,不要出来,正抓得厉害。”虽然如此,他也出来了。阿兰很是懒散,但对方则免不了东张西望。 

  两人勾肩搭背,并肩行去。 

  9 

  公园里的厕所——内——日 

  阿兰(画外,懒洋洋):“下午我回家时差点出了事。” 

  马路边上的这个厕所又小又脏。阿兰进去之前,看到了墙上新刷的标语,坚决打击厕所里的各种流氓活动。里面墙上有同性恋的“宣传画”。有个男人站在小便池前,正在摆弄自己的那个东西。阿兰站到他边上,侧着头看。看了一会,觉得不对,就离去了。 

  10 

  公园里的厕所——外——日 

  阿兰走到厕所外面,走向自己的自行车。 

  那人追了出来,喝道:“站住!” 

  走到阿兰前面,把卷起的袖子往下一放,里面有个红袖标,然后就把自行车的车把按住。 

  阿兰:“什么事?” 

  那人:“你干什么了自己知道!” 

  盘问的场面,阿兰从容不迫,说话慢条斯理,对方无计可施。那人时时做个捻钞票的手势,但阿兰视而不见。周围逐渐聚起了围观的人。 

  阿兰(画外):“他问我看他干什么,我说我没看他,还问他干什么了,怕我看到。他说我有流氓活动,我问他什么是流氓活动,还说,也不知谁在搞流氓活动。后来他把人群撵开,放我走了。分手时他小声对我说:‘哥儿们,你丫真是舍命不舍财呀。’” 

  11 

  公园里的厕所——内——傍晚 

  阿兰(画外):“傍晚时,又有一次很危险。” 

  这一次阿兰在一个很干净的厕所里。灯光如昼。 

  阿兰(画外):“平常,这里的人很多,今天一个都没有,大概是因为抓得厉害吧。” 

  阿兰小便,进来一个警察,仔细地打量他。阿兰想往外走,被警察叫住了。 

  阿兰(画外):“他把我问了一溜够,家住哪里,上班在哪里,为什么上这儿来。” 

  最后,警察问道:“外面那辆车是你的吗?” 

  阿兰:“是。” 

  警察:“带执照了吗?” 

  阿兰:“带了。” 

  阿兰掏自行车执照给他看。警察看了一眼,还给他。说:“我就问你这个。” 

  阿兰(画外):“总是这样,我都有点烦了。这个借口不好——有在厕所里查自行车执照的吗?” 

  12 

  公园里的假山——外——夜 

  阿兰(画外,微微有一点兴奋):“就如落叶归根,我终于进去了。那天晚上公园里大抄。” 

  晚上在公园里,在一团漆黑中,警察悄悄地走来,忽然电光一闪,照到了正在缠绵的野鸳鸯。手电光死死盯住了女方,照着她低着头整理衣裙,然后朝光柱走来。但是光柱又晃到了别处。今天夜里警察不抓野鸳鸯。 

  阿兰和朋友呆在假山后面的石凳上。 

  阿兰(画外):“这个地方平常查不到,但是那天不一样了。” 

  手电光一闪,照到阿兰正坐在一个男人身上。他站起来,低着头朝光柱走来。 

  警察小史假装诧异地说:“嘿!林子大了,什么鸟都有啊!和我们走一趟吧。” 

  小史一把抓住了阿兰的手。原来被坐着的那个男人趁机逃掉了。 

  小史押着他去派出所,把他推得远远的,似乎提防着他伸手。 

  小史:“你是不是老上这儿来?” 

  阿兰不语。 

  小史:“你外号是不是叫阿兰?” 

  阿兰又不语。但继续从容自若。 

  小史加重语气:“前几天的晚上,咱们是不是在这儿见过一面?” 

  阿兰不语,但面带微笑。 

  小史有点恼羞成怒,小声嘀咕:“你丫还笑!有你哭的时候!” 

  13 

  派出所审问室——内——夜 

  小史把阿兰推到墙边,压他蹲下,说: 

  “老实蹲着啊。” 

  自己走到桌子后面坐下,把警棍放到了桌上,然后看报纸。 

  14 

  派出所里——内——夜 

  阿兰蹲不住,坐在了地上。 

  警察小史头也不抬地说:“我没让你坐着。” 

  阿兰又蹲了起来。过一会,又想直直腰。 

  小史:“也没叫你站着啊。” 

  阿兰又蹲下。 

  15 

  派出所——内——夜 

  警察小史放下报纸,给自己泡方便面,打量阿兰。 

  警察小史一边吃面,一边对阿兰说:“我找你好几天了。你躲哪儿去了。” 

  阿兰不语。 

  警察小史吃完了面条,给自己泡了一杯茶,然后伸了一个懒腰,这才看了阿兰一眼。 

  警察小史:“你知道我打你干啥?” 

  阿兰呆着不答。 

  警察小史:“嘿!我和你说话呢!” 

  阿兰答道:“不知道。” 

  小史:“不知道什么?” 

  阿兰:“不知道您为什么找我。” 

  小史笑,摇头:“不知道?好。”他又看报。 

  阿兰蹲不住,又要坐下。小史咳嗽一声。阿兰又蹲起。 

  小史:“对。让干吗再干吗。” 

  16 

  派出所——内——夜 

  亮着灯。似乎过了不少时候。阿兰低着头,弓着腰,看自己的膝盖。因为很累,所以相当狼狈:腰弯得后襟缩上去,脊梁露了出来。 

  小史收起报纸。“现在知道了没有?” 

  阿兰抬头看小史,摇摇头。 

  小史摇头,轻笑,轻轻说:“好,等你知道咱们再说。蹲着慢慢想吧。” 

  他把腿跷上桌子,瞪了阿兰一眼:“看我干吗?” 

  阿兰又低下头去。 

17 

  同上 

  小史展开报纸,继续看报,似乎漫不经心地问:“想好了吗?” 

  少顷又加一句:“你要愿意蹲一夜,就蹲一夜。反正我值班。” 

  在开始回答之前,阿兰看小史。小史很帅。 

  阿兰舔舔嘴唇:“我是同性恋。” 

  小史把目光从报纸上移开,看阿兰。 

  然后停了一会儿。 

  阿兰的画外音,平缓而从容不迫:“我告诉他说,我是同性恋者,常在公园里接头。” 

  18 

  公园外的小巷——外——日 

  阿兰尾随一男子行去。走向一所未完工的楼房。 

  阿兰的画外音:“我有很多朋友,叫做大洋马、业余华侨、小百合等等。名字无关紧要,反正不是真的。'我们在公园里相识,到外面的僻静角落里做爱……” 

  小史咳嗽。 

  19 

  派出所——内——夜 

  小史:“我没问你这个。” 

  阿兰停了一会儿,又说:“我到医院里看过。” 

  阿兰(幽幽地):“我试过行为疗法……还有一种药,服下去可以抑制性欲。不过,都没什么效果。再说,也不是我自己想去看,是别人送我去的。” 

  小史加重了口气:“我也没问你这个。” 

  阿兰(低沉):“我结了婚,我知道这是不好的。对不起太太。(声音低至不清)……再说,在圈子里,人家知道了我结过婚,也看不起我……” 

  小史近乎恼怒:“我没问你这个!” 

  阿兰不解地抬起头来。 

  小史:“我问你有什么毛病!” 

  静场。阿兰把手贴在自己脸上,喃喃自语似的:“我的毛病很多……” 

  小史厉声喝道:“你丫贱!你丫欠揍!知道吗?” 

  阿兰低下头去。 

  过了一会他抬起头来,脸上是既屈辱又宽慰的样子,说道:“是。知道了。我从小就是这样的。” 

  但语调低沉,甚至哽噎了一下。 

  20 

  很久以前,阿兰小时候住过的房子——外——日 

  阿兰(起初平缓,无感情):“小时候,我家在一个工厂宿舍区。三层楼的砖楼房,背面有砖砌的走廊。走廊上堆满了乱七八糟的东西。楼与楼之间搭满了伤风败俗的油毡棚子。” 

  顺着乱糟糟的走廊前进,进到一间房子里。打蜡的水泥地板,一台缝纫机。角落里有一堆积木。当转向积木时,响起了脚踏缝纫机的声音。 

  “我坐在地上玩积木,我母亲在我身边摇缝纫机。我们家里穷,她给别人做衣服来贴补家用。” 

  21 

  派出所——内——夜 

  灯下,警察小史收起报纸,对阿兰说:“好了,你可以起来了。” 

  阿兰站起来,艰难地走动。但依旧从容不迫。到桌前的圆凳上坐下,又疑虑地站了起来。 

  警察小史:“坐下吧。” 

  22 

  很久以前,阿兰小时候住过的房子——内——日 

  阿兰的声音:“除了缝纫机的声音,这房子里只能听到柜子上一架旧座钟走动的声音。每隔一段时间,我就停下手来,呆呆 

  地看着钟面,等着它敲响。我从来没问过,钟为什么要响,钟响又意味着什么。我只记下了钟的样子和钟面上的罗马字。我还记得那水泥地面上打了蜡,擦得一尘不染。我老是坐在上面,也不觉得它冷。这个景象在我心里,就如刷在衣服上的油漆,混在肉里的沙子一样,也许要等到我死后,才能分离出去。”钟鸣声。 

  “自鸣钟响了,母亲招手叫我过去。那时,我已经很高了。 

  母亲用一只手把我揽在怀里,解开衣襟给我喂奶,我站在地上,嘴里叼着奶头,她把手从我脑后拿开,去摇缝纫机。这个样子当然非常的难看。母亲的奶是一种滑腻的液体,顺着牙齿之间一个柔软、模糊不清的塞子,变成一两道温热的细线,刺着嗓子,慢慢地灌进我肚子里。” 

  打了蜡的水泥地面,陈旧的积木。阿兰的声音渐渐带有感情。 

  “有时候,我蓄意用牙咬住她,让她感到疼痛,然后她就会揪我的耳朵,拧我,打我,让我放开。” 

  “然后,我就坐在冰冷的地面上。这地面给人冰冷、滑腻的感觉,积木也是这样。与此同时,在我的肚子里,母亲的奶冰冷、滑腻、沉重,一点都没消化,就像水泥地面一样平铺着。时间好像是停住了。” 

  23 

  派出所——内——夜 

  阿兰犹豫、试探地看小史。小史在听。 

  小史的画外音:“听他说话真费劲……不过那天晚上我下了决心听他说。这不光是因为他对我动手动脚……听他们说,阿兰毛病很大。我倒要看看他有什么毛病。” 

  阿兰看过小史后,又重新开始了。 

  24 

  阿兰小时候住过的房子——内——日 

  门敞开了。外面很亮。 

  阿兰的画外音: 

  “我从没想过房子外面是什么。但是有一天,走到房子外面去了。我长大了,必须去上学。我没上过小学、所以,我到学校里时,已经很大了。” 

  25 

  学校外面的路——外——日 

  “那座学校纪律荡然无存,一副破烂相。学校旁边是法院, 

很是整齐、威严,仿佛是种象征。法院的广告牌,上面打着红钩。” 

  布告栏。打着红钩的布告。 

  “上学路上,我经常在布告栏前驻足。布告上判决了各种犯人。‘强奸’这两个字,使我由心底里恐惧。我知道,这是男人侵 

  犯了女人。这是世界上最不可想象的事情。还有一个字眼叫做‘奸淫’,我把它和厕所墙上的淫画联系在一起——男人和女人 

  在一起了,而且马上就会被别人发现。然后被抓住,被押走。对于这一类的事,我从来没有羞耻感,只有恐惧。随着这些恐惧,我的一生开始了……说明了这些,别的都容易解释了。” 

  26 

  学校的教室——内——日 

  阿兰画外音,从平缓开始:“我长大了。上了中学。” 

  教室里坐满了学生。 

  “班上有个女同学,因为家里没有别的人了,所以常由派出所的警察或者居委会的老太太押到班上来,坐在全班前面一个隔离的座位上。她有个外号叫公共汽车,是谁爱上谁上的意思。” 

  公共汽车坐在隔离的座位上。 

  “她长得漂亮,发育得也早。穿着白汗衫,黑布鞋。上课时,我常常久久地打量她。” 

  “她和我们不同,我们都是孩子,但她已经是女人了。一个女人出现在教室里,大家都吓坏了。课间休息时,教室分成了两半,男的在一边,女的在另一边。只有公共汽车留在原来的地方。”公共汽车的体态。 

  “我看到她,就想到那些可怕的字眼:强奸、奸淫。与其说是她的曲线叫我心动,不如说那些字眼叫我恐慌。每天晚上入睡之前,我勃起经久不衰;恐怖也经久不衰——这件事告诉我,就像女性不见容于社会一样,男性也不见容于社会。” 

  “放学以后,所有的人都往外走,她还在座位上。低着头,看自己的手。” 

  镜头逐渐推近公共汽车。阿兰带有感情。 

  “这时我在门外,或者后排,偷偷地看她。逐渐地,我和她合为一体。我也能感到那些背后射来的目光,透过了那件白衬衫,冷冰冰地贴在背上……在我胸前,是那对招来羞辱,隆起的乳房……我的目光,顺着双肩的辫梢流下去,顺着衣襟,落到了膝上的小手上。那双手手心朝上地放在黑裤子上,好像要接住什么。也许,是要接住没有流出来的眼泪吧。” 

  27 

  派出所——内——夜 

  阿兰抬头看小史。 

  小史的画外音:“听了他这些话,我觉得他在炫耀他那点事儿,很臭美,故意把话说得让人听着费劲,显摆他是作家。我很 

  想叫他知道知道自己是个什么东西……不过,这不用着急。” 

  稍顿,又加一句:“不过,这孙子真的很特别。” 

  小史:“接着谈,谈你有什么毛病。少说点废话。” 

  小史有点烦的样子了。 

  阿兰重新开始:“我的第一个同性爱人,是同班的一个男同学,他很漂亮,强壮,在学校里保护我。那一次是在他家里,议论过班上的女同学——尤其是公共汽车以后,就动了手。我说,我是女的,我是公共汽车。而且我觉得,我真的就是公共汽车。” 

  28 

  男同学的家——内——日 

  在单人床上,两人赤裸相拥着。 

  阿兰:“我马上就感到自己是属于他的了。我像狗一样跟着他。他可以打我、骂我、对我做任何事——只要是他对我做的事,我都喜欢。我也喜欢他的味道——他是咸的。睡在铺草席的棕绷床上,他脊背上印上了花纹,我久久地注视这些花纹,直到它们模糊不清——我觉得在他身边总能有我呆的地方,不管多么小,只要能容身,我就满足了。我可以钻到任何窄小的地方,壁柜里、箱子里。我可以蜷成一团,甚至可以折叠起来,随身携带……但是,后来他有了女朋友,对我的态度就变了。” 

  29 

  男同学家窗外——外——夜 

  阿兰:“他家住在一座花园式的洋房里。有一天,已经黑了。我找他,站在花坛上往窗户里看。他正在灯下练大字。我看了好久,然后敲窗户。他放下笔,走到窗前,我们隔窗对视。我打手势让他开窗,他却无动于衷地摇头。他要走开时,我又敲窗户。最后他关上了灯。”阿兰坐在窗外。颓唐地把头倚在墙上。“我在黑夜里直坐到天明。夜很长,很慢。整整一夜,没有人经过,也没有人看到我。开头还盼他开窗户来看我一眼,后来也不盼了。他肯定睡得很熟。而我不过是放在他窗外的一件东西罢了。我真正绝望——觉得自己不存在了。忽然一下,外面的路灯都灭了。这时我想哭,也哭不出来。天快亮时起了雾,很冷,树林里忽然来了很多鸟,叽叽喳喳叫个不停。这时候我猛然想到,我是活着的!” 

  30 

  派出所——内——夜 

  阿兰抬头看小史,小史仔细看阿兰,面露厌恶之态,但马上又把这表情收了起来。 

  [建议:在小史面前,朦朦胧胧出现了一扇窗户,在他的灯影中,有一个人在外面敲窗、做手势,要求进去。然后又推、拨,想要开那扇窗户。后来他力竭,退后了半步,往里看。]阿兰接着说下去:“后来,我继续关注公共汽车。”
