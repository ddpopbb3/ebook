\chapter{我的阴阳两界}

\section{第一章}



再过一百年,人们会这样描述现在的北京城:那是一大片灰雾笼罩下的楼房,冬天里,灰雾好象冻结在天上。每天早上,人们骑着铁条轮子的自行车去上班。将来的北京人,也许对这样的车子嗤之以鼻,也可能对此不胜仰慕,具体怎样谁也说不准。将来这样的车子可能都进了博物馆,但也可能还在使用,具体会怎样谁也说不准。将来的人也许会这样看我们:他们每天早上在车座上磨屁股,穿过漫天的尘雾,到了一座楼房面前,把那个洋铁皮做的破烂玩艺锁起来,然后跑上楼去,扫扫地,打一壶开水,泡一壶茶,然后就坐下来看小报,打呵欠,聊大天,打瞌睡,直到天黑。但是我不包括在这些人之内。每天早上我不用骑车上班,因为我住在班上。我也不用往楼上跑,因为我住在地下室,上班也在地下室,而且我从来不扫地。我也不打开水,从来是喝凉水。每天早上我从床上起来,坐到工作台前,就算上了班。这时候我往往放两个响屁,标志着我也开始工作了。我呆的地方一天到晚总是只有一个人,所以放响屁也不怕别人听见。 

我住的地方是医院的地下室。这里的大多数房间是堆放杂物的,门上上着锁,并且都贴一张纸,写着:骨科,妇产科,内科一,内科二,等等。我搬进来以后,找了一支黑腊笔,在每张纸上都添了“的破烂”,使那些纸上写的是骨科的破烂,妇产科的破烂,等等。这样门上的招牌就和里面的内容一致了。但是没有人为此感谢我,反而说,小神经的毛病又犯了。他们对我说,我不该在门上写破烂二字。破烂二字不能写上墙。假如我要写,可以写储物室,写成骨科储物室,妇产科储物室。但是我说,你们玩去罢。他们听了这话,转身就逃了出去。地下室对他们来说,可不是个好地方。 

除了这些堆破烂的房子,就是我住的房子了,门上写着仪修组王工程师的字样。我的左边隔壁是破烂,右面隔壁也是破烂。但是除了破烂,这里还有一些别的东西。走廊上,每隔不远就有一个龛,龛里放着标本缸。缸里泡了一些七零八碎的死人。其中一个就在我的对门,和我同一性别,但是既没有脑袋,也没有四肢。我闲下来就去看他,照我看,他死掉时,大概还没有我大。他的腰板挺的板直,一副昂首阔步的样子,只可惜他既没了首,也迈不开步了。人家在他肚子上开了一扇门,在内脏上栓了好多麻线,每根麻线上栓了一个标签,写着大肠小肠之类的字样。假如这位仁兄活过来,一低头就能看见,自己的哪一部分叫什么。除此之外,他还会发现人家把他的阴茎切掉了,但是把阴囊和睾丸都留着,所以那些东西泡在缸里,就象半头蒜的样子。不知道他会不会觉得好看。还有一些龛放着一些玻璃柜,放的是骨头架子。那些东西自己不能够站立,所以柜底下安着一根木杆子,杆顶上有个铁夹子,夹在项骨上。把死人弄成这个样子,可是一种艺术。一般的人,你就是给他最好的死尸,他也作不出好的标本。因为这个原因,我住的地方就象一个艺术馆。我对这个住处很是满意。 

我住的地方就是这样。我就是门上写的那位王工程师。小神经也是我。他们叫我小神经,是因为我有点二百五。过了一百年,也许人们不知道什么叫二百五。这句话的意思是说,因为我只呆了二百五十天就从娘胎里爬了出来,所以行为怪诞。其实我在娘胎里呆足了三百天,但是因为我行为怪诞,大家就说我只呆了二百五十天。这种因果倒置是因为我们有幽默感。其实我行为怪诞,是因为我有阳痿病。因为我有阳痿病,所以和前妻离了婚。我现在四十多岁,还在独身,而且离群索居,沉默寡言。 

我不得不离群索居,沉默寡言,因为无论我到了哪里,总有人在我背后交头结耳,说我是个阳痿病人。这就使我很不好意思见人,虽然我已经阳痿了十年,对此已不再感到羞愧,但是我还是不乐意人家这样说我。我不愿他们把我看成了太监一类的东西,虽然实际上我的确和太监差不多。这件事的教训是不要找本单位的人结婚,除非你能确信自己没有阳痿病。我前妻原来是本院的护士,现在调走了。但是在调走以前,她已经把我不行这件事传的满城风雨。现在除了躲在地下室,我也采取了积极措施,到康复科去看病。康复科的马大夫和我关系很好,别人看病要钱(公费医疗不报销康复科),他不管我要钱。 

马大夫治我的阳痿病,开头是用内科疗法,给我开了很多药,并且让我多吃巧克力。他说巧克力壮阳。但是巧克力吃多了食欲全无,我还长了口疮。后来又换了外科疗法,住了一段时间院,躺在床上打牵引。这就是说,在那玩艺上挂上十公斤铅锤,往外拉。牵引了两周,那玩艺拉到了一尺多长(后来不牵引,慢慢又缩回去了),但是似乎比以前还软了。他又建议我动手术,移一节肋骨进去。我觉得这样不好,因为肋骨移进去,就会永远硬挺挺,这样很不雅。他对我的病真是尽心尽力,认为我的病老不好,是对他医术的挑战。最后他建议我做变性手术,当不了男人当个女人好了。但是我坚决不答应,因为我身高一米八五,体重九十公斤,头大如斗,手大脚大,当了女人也不好看。最后他说我不肯合作,就再不给我看病了。但是我们俩关系还是很好,他经常跑到我的工作室来和我聊天。这家伙有六十岁了,养得又白又胖,因为不正经,在头头脑脑面前很没人缘,和一些小大夫小护士倒满亲热的。就是他有一天跑到我这里来,说要给我介绍女朋友。我觉得他脑子有问题:头几天还要叫我作变性手术,现在又要给我介绍女人,一点逻辑都没有。我就这样和他说了。正说时,有个女孩子从外边闯了进来,说道:马老师,您出去,我自己和他说!然后她就自己介绍说:我是妇科的,我姓孙。其实我在食堂里见过她,就是不知道她是妇科的,也不知道她姓孙。 

小孙那一天来找我,起头情形就是这样的。马大夫走了以后,她一五一十地对我说:她马上就需要个男朋友,必须是人高马大,膀阔腰圆,能带得出去的那一种,来帮她解眼前的燃眉之急。这是因为她的前男朋友要结婚,今天晚上就要举行婚礼,她已经收到了邀请,想和一个大个子男人一块去。我想了想,说道:要是这样的话,我能帮上忙。别的事情我就帮不上忙了。这个姓孙的小鼻子小眼,娇小玲珑,一副小孩样,其实已经二十七岁了。到了晚上,我就和她一块去了。婚宴上全是些青年男女,大概都是她的同学,新娘子也是她的同学。我发现,医学院大概只招南方人,所以那一屋子男女全是小个子南方人,白面书生,个个戴着眼镜。我在其中象个巨人。认识我的人都说,我的脸相极凶,还说我吃相难看。我在席上喝了一瓶啤酒,就打了一个大嗝,声震屋宇。然后我讲了一个下流笑话,弄得四座皆惊。其实我没想去捣乱,只是在地下室里呆了很多年,很少有人请我来参加聚会,心里很高兴。但是已经把新郎吓坏了,把小孙叫到一边说了好半天。然后我们就提前退席了。回来的路上小孙说,王工,你把他们都镇了!你帮了我的大忙,我不会让你白帮的。我一定也帮你一个忙。 



后来小孙对我说,作为我给她出气的报答,她要把我的病治好。据她自己说,她读过Masters和Johnson的书,治我的病十拿九稳。我也看过那些书,所以我想这孩子真是个怪人。她梳了个齐耳短发,长得白白净净,还是满漂亮的。不管怎么说,也能嫁得出去,干嘛要来给我治阳痿?女孩子只要嫁得出去,就不必理睬不想嫁的男人。我对她说,你没搞错罢?那都是夫妇双修的办法。她说知道,所以我要和你结婚。先结婚,后治病。 

我和小孙要结婚的起因就是这样。开头我想,这个孩子还要给我治病,我看她自己就该找人治一下,是不是精神病。后来想到她起初找我那一回的情况,我怀疑她吃了别人的亏。既然她都要嫁我了,问一问也没什么。我就问道:你大概不是处女罢。她说当然不是。你要不要看看?我说看什么?她说我可以对她作个妇科检查。我对此是一没有经验,二没有兴趣,而且也没有必要。只有混充处女的,没有混充非处女的。所以我就说:结婚可是你自己要干的,将来可别埋怨我。她说绝不会。她说这些话时,一点也不脸红。 

再过一百年,人们可以在现在留下的相片里想象我:我和大家一样,目光呆滞,脸色灰暗,模样儿傻的厉害。现在你到美术馆去看看十六世纪的肖象画,就会发现上面的人头戴假发,长一张大屁股脸,个个都是傻模样。过去的人穿燕尾服,瘦腿裤,显得头大身子小,所以很难看。但这样的装束在当时,一定是了不起的好穿着。以此类推,现在的人不论穿什么,将来也会傻的厉害。基于这种心理,所以我根本不打扮,经常不理发,不刮脸。当然,小孙是女孩子,不能和我一样。她经常打扮得干净漂亮,因为留着齐耳短发,下面的头发茬每逃诩要推一推。因为这些原因,我们俩在一起不够班配。但是我们俩经常一道去逛大街,表示我们在恋爱。这是计划的一部分,首先做出了恋爱的姿态,将来请求结婚就不至于显得突兀。 

将来的人谈到我们结婚前的到处奔走,一定会感到奇怪。我根本就没有逛大街的欲望,我常年呆在地下室里,很少走动,所以腿上的肌肉都退化了,白天走了路,晚上就腿疼。天寒地冻,不能去公园。我们总是在商业区里逛,但也没有要买的东西,更没有买东西的钱。过去我一个人在城里逛,老是低着头,看看地上有没有掉的钱,这是我几十年的积习。现在我也和小孙在北京城里闲逛,我倒是不低头,但是对一切都视而不见。倒是小孙时常有所见,走着走着就会忽然捏我一把,说道:看见了没有,刚才那个人盯着我看。听了这话,我就会猛然转过头去,大声说道:哪一个?她把我拉回来说,别这样,你要把别人吓死了。走到街上,我有时也会注意到她忽然把小嘴一扁,小脸一扬,脸上似笑非笑的模样。要不然就是忽然抓住我的胳臂,把全身挂在我身上。这大概是因为又有人看她了。但是到底是些什么人在看她,我一个也看不见。 

星期天小孙把我带到王府井一家理发馆门前,让我往橱窗里看。我看了好半天,才认出橱窗里有一张相片是她。那是一辐黑白上色的相片,再过一百年,人们就会根据相片上的水彩,断言拍照时彩色摄影尚未发明。相片上的小孙涂了个红脸蛋,和她本人一点也不象。那相片就象现在看到的玛丽莲·梦露,或者猫王的相片那种五官不清,色彩斑斓的样子,露出五十年代那种村气土气;但是再过一百年,人家看到一个女孩子站在橱窗里自己的相片前流连忘返,也会露出会心的微笑。我对她说,快走罢,呆会人家会出来说:小姐,是不是想把相片要回去。她就勃然大怒道:你说什么呀你! 

小孙说,她在大街上走时,经常迎上这样的目光:先是盯上了脸,然后一路向下搜索,在胸部久久的停留。然后久久端详她细长的腿。她对自己的腿很是骄傲。这种景象我从没看见过。我想人家也许是在看她那条石磨蓝的牛仔裤,那条裤子值我一个月的工资。她对这种说法十分愤怒,说我在蓄意贬低她。其实我没有这样的意思。我早就注意到她的头发细密茂盛,柔软光滑,就象一只长毛猫的毛一样,每次从外面回去,走到医院门口时,她都要把手伸给我,让我拉着它。那只手非常小,柔若无骨,又凉又滑。我们拉着手从门口进去,她还要去问传达室的老头:有我的信没有?然后和每一个见到的人打招呼。我和小孙谈恋爱的情形就是这样的。 

我和小孙每天下了班就到王府井喝咖啡。后来我对咖啡上了瘾,每天必须喝五大杯,否则就呵欠连天,而咖啡太贵了,比外国烟还贵。据马大夫说,我这叫作咖啡因依赖。他又要给我治这种病,但是我拒绝了。我怕他用咖啡搀上大粪给我喝,据说他就是这样给人戒烟。我只是向他打听外界对我和小孙恋爱的反应。他告诉我说,情况不容乐观,人家说,小孙是面子下不来。这句话的意思是说,她借用我在她前男友结婚那一天去给她撑过场面之后,如果现在就不理我,则显得太冷酷,太薄情。因此她必须和我假恋爱一段,然后再把我甩掉。这就是说,一个女孩子,应该表现得温柔多情,尽管她其实不是那么温柔多情,也要假装成这样。这也就是说,小孙借用我去参加婚宴的事现在已经是尽人皆知了。这件事起初只有三个人知道:一个是我,一个是小孙,还有一个就是马大夫。我们每个人都有把这件事泄露给别人的嫌疑。马大夫主动告诉我说:这件事我可没对任何人说过,也不知别人怎么就知道了。 

假如马大夫没有把这件事告诉别人,小孙也不告诉别人(这事对她名声有损),剩下只有我最可疑。但是我成天呆在地下室,从来不和外人接触;最后的结论就是我们谁也没告诉别人,这事就自己传出去了。由此得到一个推论,我们医院里现在安装了一台可怕的仪器,可以窃听全院每一个角落。这台仪器由一个长舌妇操作,她听到了我们在地下室里的谈话,然后就告诉了医院里每一个人。但是这件事非常的不可能,因为他们安这仪器时,必定要找我。我是全院唯一的电气工程师。连我都不知道医院里有这台仪器,那就必定是没有。 

根据医院里现在的传闻,小孙是个极好面子的姑娘。她不乐意在前男朋友结婚那一天显得孤独无伴,所以借用了我。这是很正确的。根据同上传闻,她的小算盘又极精,找一个阳痿的男人来撑场面,将来不会有任何损失;有损失的是我,因为我被女人耍了。但是实际情况不是这样,实际情况是小孙正在献身于科学,准备在我身上探索一条治疗阳痿的新路。我和她是医生与病人的关系。当然这一点是秘密的。在开始治疗前,她必须嫁给我,然后治疗才合法,治好以后,才好写报告,拿出去发表。为此必须叫大家相信我们在恋爱。小孙说,我们俩必须在人前再亲密一点。她建议我们中午时到门厅里去接吻,但是我觉得过于肉麻。于是她建议我们从外面回到医院里时,显得再亲热一点。这就是说,在经过大门时,她要骑在我脖子上。我问了她的体重,体检时什么也不穿是四十三公斤,现在着了冬装,顶多也就是四十八公斤,这不算重;更何况她说,把你治好了以后,骑我的时候还多着哪;所以我实在没有理由不答应她。 

3 

在小孙骑我脖子之前,发生过很多事。首先是小孙说,她要扮演我未婚妻的角色,就要处处管着我。自从我成了小神经以后,已经习惯了别人对我耳提面命。在这些人里,女人尤多,多一个小孙也没什么。比方说,我去领工资,会计一定要再三关照我说:你数数,这是一百三十元。其实没有什么好数的,总共是一张一百元的大票,三张十元小票,完全可以一目了然;更何况数也数不多。因此我拿了钱总是看都不看就往兜里一揣。但是那个二十三岁的小会计一定从柜台后面赶出来,把我兜里的钱掏出来,当着我的面数一遍,然后再塞到我口袋里去。我到食堂里去买饭票,管理员大妈也会把饭票对我一五一十的交待:这种红的是菜票,那种绿的是饭票,千万别搞混了。其实我只是阳痿而已,并不色盲,更不是低智人。但是因为我阳痿,就不能阻止别人象关心低智人一样关心我。 

人家总要把男人的大脑袋和小脑袋联系起来看,小脑袋不行的大脑袋一定不行----这成了一种成见了。我也无心去纠正这种成见,因为既然是成见,就无法纠正。我只管我行我素,呆在地下室里不出来。这样省了好多的事:因为大家都觉得我是个傻子,所以什么开会、学习等等都不叫我去了;这样省了我和大家一起磨屁股。后世的人,对我们要开那么多的会一定惊诧不已,因为到了那时候,只有总经理、部长、总统才须开那么多的会。所以那时的人一定会以为我们都是些很重要的人物。其实我们不过是些电工、技师等等,开会讨论过马路要走人行横道而已。而且要开这样的会,必须有一条坚硬的鸡巴,软的不行。过去我除了领工资和买饭票,从来不到楼上去,现在发现连领工资都不必去,因为工资是小孙领去了。饭票也不必去买,因为饭票是小孙代我买了。别人还说,现在好了,王二的事都可以交待给小孙,省了多少麻烦。说完了总要哈哈大笑一通。 

小孙和我谈恋爱,结果是我们俩都变成了一种气体,叫作什么一氧化二氮,或者说,叫作笑气,人家一见到我们在一起就要笑。但是我们既然是气体,当然就没有自觉性。我和小孙一道出门去,走过楼道时,小孙一定要叫我站住,给我掖好围脖。其实我根本就不需要围脖,因为我长得相当肥胖,一点也不怕冷。但是小孙一定要这样做,她说这是在大庭广众下和我亲热的唯一机会。掖围脖的时候,过路的护士就会站下来,说道:“小两口出门去呀?”等等。小孙伶牙俐齿地答道:到王府井买点东西,等等。说完了我们一同向前走去。走不了几步,一阵大笑就会在脑后炸开。这时我们转过身去,就会看到那些护士聚成一堆,个个个脸色涨红。很显然,她们是在嘲笑我们。我就想转回去,把她们教训一顿。但是小孙把我拉住,叫我沉住气。她说这种情况会改变的。然后她就挽住我的手臂,把全身都挂在我身上。因为我壮的象个狗熊,而她长的娇小玲珑,所以这么挂着还算好看。假如双方的身坯换过来,那就象蚂蚁举着一片饼干渣,一点也不好看了。但是尽管她使了很大的力气往我身上贴,别人也不相信她真的要和我谈恋爱,更不要说真心嫁给我了。 

再过一百年,人们会这样形容我们的医院:这是一座四四方方的院子,四周围着栅栏。院子里全是一些古旧的灰砖房,有一些是两层的,有一些是三层的。他们想象起这些房子,就像现在我们想象地下的墓葬一样。那时候的房子大概都是一百层的大厦,底下五十层放汽车,上面五十层住人。在这些墓葬里,有一些人穿着白大褂来来去去,还有人穿着淡蓝色的睡衣睡裤来来去去。在这些灰砖楼之间,有几片草坪,几颗半死的树作为装点。但是我既不穿白大褂,也不穿蓝睡衣,穿一件粗蓝茄克衫,在这座古墓里显得很扎眼。但是我根本就很少到上面去,所以也就很少叫人看见。 

小孙那天骑着我脖子走进医院时,是星期天下午五多钟,门诊下了班,天气又很冷,所以到处都看不见很多人。我驼着她,两个人连在一起有两米五十左右,只能小心翼翼从拱门正中通过。两米五十的庞然大物从医院的正门走进去,可算是惊世骇俗之举。这个举动总算是引起了注意,第二天妇科主任就去找小孙谈话,叫她注意影响。但是这个举动也是非常费力的。假如你到过草原,见过人家骑骆驼,就会理解了。骑马骑驴都可以飞身而上,但是骑骆驼时这样干就绝对不可以,因为骆驼高了。你必须使骆驼倒下来,然后才能骑上去。但是骆驼一般是很不乐意倒下来的,赶骆驼的人要拿个装铁尖的小棍子,围着骆驼转上半天,敲敲前腿,敲敲后腿,磨上一两个小时的嘴皮子,骆驼才肯倒下去。那天下午,我就是那只骆驼,小孙就是赶骆驼的人,但是她手里没有赶骆驼的棍。她只是一遍又一遍的说:你坑谧下来呀! 

我在蹲下之前,先把医院门前的街道打量了很多遍。那条街不算宽,扫的干干净净。星期天下午,没有很多行人。然后我又把小孙的脸打量了很多遍:那是一张白白净净的娃娃脸,留着刘海,嘴巴很大。那时我想的是:记住了,就是这娘们要在大庭广众下骑我的脖子,叫我名声扫地。最后我就打量她的下半身:就是这东西要骑上我的脖子。洗得干净净的牛仔裤,又白又亮的护士鞋。最后我毅然决然地蹲了下来。她一把就揭下了我头上的帽子(那是一顶剪绒皮底的帽子,和二号的钢种锅一样大),然后哈哈笑了起来,说道:王二,你小时候头上几个旋?我知道自己是三个旋,因为一旋宁,二旋愣,三旋打架不要命。但是她说:你现在只剩一个旋了。她妈的,我怎么会不知道自己几个旋?我爸爸不到四十就秃了头,根据遗传,我现在本该一个旋都没有。 

后来我就看见两条细细的小腿搭上了我的肩膀。在我站起身之前,那双小手还在我脸上摸了老半天。这倒不是在调情,而是在找可以抓的地方。最后她抱住了我的下巴,说一声起。我就站了起来,脖子后面热烘烘,想起了一句歇后语:大姑娘骑瘦驴,严丝合缝。虽然我不是瘦驴,但是体会到了严丝合缝的感觉。这感觉非常的不好。尤其是她在我脖子上上下磨擦了几下后说:王二,这感觉非常古怪!好象是我把你生了出来!这时我往左一看,看到一条裹在洗白了的粗布里的大腿,往右一看,也是一条这样的大腿;这是我一生未曾见过的景象。这两条腿一齐夹紧,夹得我眼冒金星;我的感觉就更坏了。这时我想起了小时候看过的《天方夜谭》其中水手辛巴达的故事,那位辛巴达也被海老人骑过;但是海老人是个男人,所以辛巴达也没有被人如此严丝合缝的骑过。有史以来,有这种经历的,我是第一人。我就这样走进大门去,影影绰绰地发现有好多人在楼上的窗口看热闹。 

小孙初次骑我脖子的事就是这样的。有关这件事,还可以补充如下:开头我是不乐意让她骑的,但是她把我说服了。她说,就她个人而言,对我的脖子是很尊重的--我比她早毕业好几年,所以这是老学长的脖子;我比她大了十五六岁,所以这又是一位大叔的脖子。无论从哪方面说,骑这个脖子都是大不敬。但是为了事业,非骑不可。虽然这些说法相当牵强附会,但是我也无法批驳。而正式骑上去了之后,她就毫无崇敬之心。走过大门时,她把身体挺直,去够门顶上的灯泡。走过楼门时,她又蜷成一团,把我的脑袋整个包住。从大门口,到地下室门口,她总共在我头上盘踞了十分钟,在这十分钟里,她还给我讲了一个故事。其实这个故事我早就知道,典出纪晓岚《阅微草堂笔记》(假如你在那书里查不到这件事,你不要和我计较,我是小神经)。这故事说,某阁老家盖房子。按照中国的传统,盖房子时对梁柱之类都很崇敬,柱上要贴“擎天金柱”,梁上要贴“架海银梁”等等的红纸,安柱架梁时还要放鞭炮。当然了,这是生殖器崇拜的遗风,除了梁柱,祖宗还崇拜大炮,高塔以及一切又粗又长的东西。该阁老家放过了鞭炮,正要吊梁,发现一个丫环正骑在梁上。按照中国的传统,有一个东西是最肮脏,最不洁的;那东西却紧紧贴在了圣洁的架海银梁上。大家看了无比愤怒,有喊打的,有破口大骂的。但是那丫环却拍拍那东西答道:你们瞎嚷嚷什么?帝王将相,皆出于此也! 

这个故事我讲起来是这样的,小孙讲起来就不是这样。首先,她把出处记错了,说是《聊斋》;其次,她也不记得骑的是什么,只记得是骑个很神圣的东西。结尾倒是记住了:帝王将相,皆出于此也。讲完了以后,她还问我有何感想。我只谈了一点感受:你给我下去!从大门骑到这里,还没骑够哇! 

除此之外还有一点感想,就是她的裤子很干净,是用有香味的洗衣粉洗的,另带一点漂白粉的味道,这些气味很好闻,但是我没有说出来,我只是说这故事她完全讲错了。但是我丝毫也没有贬低她的意思,因为很少有女孩子会去看纪晓岚的书,所以就是看得不仔细也属难能可贵。谁知她根本就没看过纪晓岚的书,这个故事是她从老师那里听来的。原来她们在大学四年级分到了妇科实习,眼看后半辈子就要专门看这个东西,所以大家情绪沮丧。带实习的老师就讲了这个故事来鼓舞士气。这故事的寓意就是要让她们记住,眼前这个东西其实是很伟大的:帝王将相,皆从此出也! 

小孙给我讲这个故事,也是想鼓舞我的士气。她还说,她有一个完整的计划,给我治阳痿只是其中的一环。这个计划包括将来写一篇医学论文,一本书(记实文学类的),《我治好了阳痿的丈夫》,以及心理学、社会学方面的研究报告。干完了这件事,她就可以一举成名。要做这样的研究,和我结婚是必不可少的;否则就会受到社会方面的指责。考虑到这个研究惊世骇俗的性质,现在必须好好演出恋爱一幕,免得叫人看出漏洞来。这孩子是四川人,四川人就是有一点疯,而且她看侦探小说看多了,处处透着诡异的模样。她还怕我不乐意,答应将来把全部稿费都给我。为了这一切都能顺利实现,我也要付出些努力,其中就包括让她骑我的脖子,并且不要忘了,抵住我后脑那个东西,帝王将相,皆从此出也。 

4 

小孙骑过了我的脖子以后,我觉得丢尽了面子,更不肯上楼去了。这更合了她的意思,每顿饭都是她给我打来,可以向食堂里的人表示,我们的关系又进了一步。这就使她需要一架小计算器,以便每天晚上和我清帐:早餐的油饼是多少钱,中午的肉片又是多少钱。这些都要从我的饭票帐上支出。后来我从会计科送来修理的仪器里找到了一台,是精工牌的,上面带有一架打纸条的打印机,不但能算帐,还可以打印收据,花了五分钟修好了给她用。在找到那台计算器之前,一切都要从她的小脑袋瓜子里算出来。这时她躺在我房里的空床上,搜索枯肠,挖空心思,再加上搔首弄姿,看上去真叫人于心不忍。我自己也是医学院毕业的,所以真不能相信医学院能把人教得不识数。我们俩不但都是医学院毕业,而且是同一所医学院毕业,唯一的区别就是我学医疗仪器,她学临床医学,但是这一点区别就使她时时问我十二减九等于几。但是她算帐的模样还是满好看的,从她拖在地下的两条腿来看,你该相信她是仰卧在床上,但是从她的上半身来看,你又该相信她是俯卧在床上。假如是我在做这个姿势,下半生就要卧床不起了。那时候正是下午五点钟左右,一抹残阳从窗口照进来,正照在那块空床板上。她穿着一件牛仔上衣,脖子后面镶了一块三角形的皮革,一头柔软的短发都被她搔乱了。算到心力交瘁时,她就专心地去闻那只圆珠笔。这些表现一点也不象个人,倒象一只猫咪。这叫我觉得让她来给我治阳痿,实在不好意思。假如是个胖大女人,再长一点胡子,那就好意思了。 

这个小家伙每天还要给我讲一课,对着“帝王将相”的图谱,给我上女性的生理解剖学。有件事已经讲了不下十次了,就是一到了我能在帝王将相里站住了脚,我们俩必须立即离婚。就其本心来说,她一点也不想嫁给我,到时候一定要离婚,绝对不准赖的。我当然同意了,但是有另一个问题要提出来的,就是假如治疗没有效果,我老也进不到帝王将相里面去,那该如何是好。她说那是绝对不会有的事。人家Masters和Johnson作了那么多例实验,应该是很有把握。实在治不了,也只好离婚算了。反正双方都没有损失。为了避免将来离婚时闹纠纷,现在就该把帐算清。凡是共同开支,一律用二去除,精确到小数点后一位,然后再四舍五入。 

就我的本心来说,也一点不想娶她当老婆。我一点也不想娶任何人当老婆,但是很想把阳痿病看好,省得大家拿我当个怪物。所以我们俩在这方面一拍即合。为此就需要在某个时间,某个地点,取得性交的许可。我们俩正为此作出努力。下个礼拜天,我们又出去转了一天,晚上她又是骑着我的脖子回来的,这一回引来了更多的人来看。 

这一回我觉得她的裤子凉飕飕的,气息芬芳,不是洗衣粉的气味,也不是香水的气味,很可能来自帝王将相。那个东西,我虽然结过婚,却没有见过,现在每天看图谱,渐渐感到十分亲切。经过了一段时间训练,她认为可以了,我们就打报告请求结婚。谁知道居然出了意外,人家不批准。 后来我觉得这整个事情象一个谜。不知道为什么,小孙想和我结婚,也不知为什么,我会同意和她结婚。从表面上看,她是想给我治阳痿,做一项医学试验,其实这样的理由根本就不可信。从表面上看,我是想让她给我治好这种病,以便从此作个正常的男人,但是这个理由也一点不可信。其实我并不渴望从此做个正常的男人,小孙也不渴望做成这个医学试验。这件事从始至终都可疑得很。唯一可能的解释就是我觉得她是自己人,她也觉得我是自己人。用她自己的话来说,我们俩有缘份。

\section{第二章}

二十年前,有一冬天的早上,我骑车去找一个人。当时北京的上空飘着一层混了煤烟的脏雾,好象一口粘痰;我的自行车喀喀做响,好象一只铁皮玩具鸭子;我穿了一件油腻腻的棉袄,头上戴了一顶旧毡帽。当时的情形就是这样的。 

北京城的中心是紫禁城,绕着紫禁城有一些街道名和紫禁城有些关系,比方说,太仆寺街,光禄寺街,内务府街等等。有条胡同叫饽饽房,大概那里过去是专给皇宫大内蒸饽饽的;有条胡同叫奶子府,过去大概住了一些为大内服务的奶妈。那些胡同里的房子都不怎么样。七三年到七四年,我经常到那一带去,对那一带的情形知之甚详。当时那一带的胡同里都铺了柏油,但是胡同还是那么窄。有些破房子拆掉了,但是没有好好翻盖。新盖的房子都是用烧得很次的红砖砌的,背面甚至是空心的煤渣砖。没有翻盖的房子都是又矮又破的四合院,和过去完全一样。和过去不一样的还有每条胡同里都多了一间灰渣砖砌的小房子,那就是公共厕所。过去这种房子也有,但是不那么多,这是因为院里的茅房都被填死了,大家都得上公共厕所。自从有了这种小房子,每一条街都臭得厉害。冬天里我骑一辆自行车,从那些胡同里经过,路两边都结了薄冰。我看到那些房子上都喷上了青灰,好象死了爹又死了娘的模样。过去北京城里,只有煤铺墙上才喷青灰。但是尼克松来北京时,到处都喷了青灰,象煤铺一样。大概觉得这样比较美。我小的时候就没看出煤铺怎么美。我是清晨路过那些胡同的。北京城里当时有一层薄雾,所以没有风。天气很冷,但是并没有冷到冻鼻子的程度。那时候除了上早班的人,都还没起来。在胡同口碰见一位少妇,正在倒尿盆。她的头发还能看出一点理发馆的模样,身上裹了一件缎子的(或者是线绨的,这两种东西我分不清楚)的丝绵小棉袄,下面穿一件粉红的棉毛裤,脚下踩着两个毛窝(就是那种毡面松紧口的棉鞋),睡眼惺松,手提一个搪瓷痰桶迎面走来。棉袄和痰桶都是崭新的,这些迹象表明,她结婚还不到一个礼拜。当时我正盯着她领口看,因为她的脖子和胸口象雪一样白。我记得她是很漂亮的,但是现在想不起她的模样。就我当时的年龄来说,记性本不该这么坏。这是因为她走到了下水道口上,就把痰桶一倒。不仅是哗啦一声,里面还滚出两节屎来。所以我就没记住她的模样,只记住了屎的模样,那屎橛子无比之粗,无比之壮。那东西就冻在了铁蓖子上,大概要冻一冬天。在那上面还要冻上剩面条,剩米饭,好象一块奇形怪状的萨其玛。这件事情好象马路上冻结的一口粘痰,冻进了我的脑子里,大概要到我死后,才会释放罢。 

时隔二十年,我又想起了那天早上的事。那天我到奶子府去,是要找李先生。不知道现在李先生上哪里去了。现在他大概不会是过去那个模样。但是假如你在七三年看到他,就会说他是个狗头猫脸的玩意儿。狗头是指他的脸形,象个哈叭狗的模样,猫脸是指他的眼睛有点黄,瞳孔也有点窄长,他的头当时就泻了一半顶,现在大概全泻光了。此人身材不高,但是身上还算有肉。有一点鸡胸,又有一点驼背。我不但认识他的脸,还认识他的屁股,这是因为我那一天早上把他叫起来后,他只好当着我的面穿裤子。他的内裤太破了,就背朝着我。但是后面更破,和没有是一样的。那时我坐下来,一面欣赏他的屁股,一面找到了他的烟叶子,给自己卷一支烟当时我看见他的屁股,就象个风干的苹果,皱皱巴巴的,还有无数小的黑痣,息肉等等,我想任何狗急跳墙的同性恋者见了都不会动情。李先生背着脸说:给我也卷一根。这个笨蛋,穷到了抽烟叶的地步,却不会卷烟。于是他只好用烟斗来抽,那味道就象狗屁一样。抽到嘴里像狗屁,别人闻着也象狗屁。 

有关烟叶子也有很多学问,现在眼看要失传。这种东西二两一包,外观象简装洗衣粉。有一种是白纸上印红字,那是晒烟,抽起还可以,假如是特级,就是关东烟,比香烟还好。还有一种是绿字,那是烤烟,抽起来就象狗屁。但是狗屁也分级,二级以下烟叶里有草棍,席箔,秫桔杆,不是纯狗屁。李先生的烟叶子是五级的,抽到一半,烟头里掉出一个黑球来,经仔细辩认,是个烧糊了的死苍蝇。为此我还恶心了好半天。 

我还能想起不少有关李先生的事情。李先生出门时骑一辆自行车,那辆车可不是一般的自行车,而是一辆匈牙利的倒轮闸。这种非常少见,甚至比日本鬼子留下的老富士还少见,因为它是五二年匈牙利在北京开博览会时送来的样品。自从到了李先生手里,他就再没有修理过,任凭车上的零件一样样脱落下来。据说有一次车座不见了,李先生就在座管上骑了一段时间,其状就如在受桩刑:疼得呲牙咧嘴,手舞足蹈。后来他痔疮大发,才不得不买了一个旧车座。李先生上车的样子也是十分奇特,他总是推着车向前奔跑,奔跑中弯下腰,把脚蹬子转到一个特定的角度,然后踏着脚蹬骑上自行车。那种奔跑中矮身转脚蹬的身法,酷似狗撒尿。 

李先生和我一样,专干些不能干的事。我干的事是想写小说,经常往刊物投稿,但是总是被退回来,并且不是退给我本人,而是退到党委办公室,附有一封公函,建议对投稿人加强思想教育。但很少有人真来教育我,因为我是小神经。李先生干的事倒不是写有维多利亚时期风格的小说,而是要研究西夏文。这件事并没有思想意识方面的问题,但他本职工作是个俄文翻译,一研究起西夏文就不进俄文了。而且他在研究西夏文时,你就是在他眼前放鞭炮他也听不见,这个样子完全不能上班。因此他早早退了职,靠偶尔翻些稿子为生。谁知后来碰见了文化革命,取消了稿费,差一点就把他饿死了。李先生因此气急败坏,说过好多大逆不道的话。我听见了这样的话,就这样安慰他:其实这件事也是满公平的----为什么只许老天不下雨,饿死非洲的游牧民,就不许中国搞文化革命,饿死你这搞翻译的游牧民?何况从现在的情形来看,你到底饿得死饿不死还不一定。但是他还是要继续说些反动话:要是天不下雨,饿死我认了。现在的事是,我又没招了谁惹了谁,有人非要逼我跳火坑。李先生的情形就是这样,我到今天还记得。人活在世界上就象一海绵,生活在海底。海底还飘荡着各种各样的事件,遇上了就被吸到海绵里,因此我会记得各种事情。 

2 

那一年我正在山西插队。现在我长得人高马大,相貌凶恶,过去就不是这样。小时候我长得文静瘦弱,还爱和女同学跳猴皮筋。以我到山西插队时,我妈就睡不着觉。她以为我连窝头都不会蒸,一定要饿死,假如没饿死,也会被人欺负死。但是只过了一年,我就长了一嘴络腮胡子,活象一个老土匪,而且满子诩是操你妈。这说明环境可以改变一个人,只要一年就能变得连他的亲妈都认不出来。在乡下时我很少吃窝头,倒常常吃鸡。老乡们说,母鸡见了我就两腿发软,晕倒在地,连被提走了都不叫一声。这当然是过甚其辞。当时我虽然极具男性魅力,却未必能迷倒雌性鸟类。 

那一年冬天我原准备在乡下过冬,但是当地正好刮着很厉害的白毛风,烧炕的柴又不够。我们五六个人挤在一个被窝里,身上盖上了所有的大衣。第二天早上起来,发现所有的大衣都从被顶上滚下来,掉到了尿尿的脸盆里,冻成了铁板一块。我们中间没有一个人有勇气不穿大衣就到外面去生火,就在屋里点火把那盆尿煮开,大衣拿下来。那气味实在是可怕,把我的两只眼都熏坏了。出了这件事以后,大家都不好意思了;谁见了谁都是羞答答,因为六个堂堂的男子汉煮了一锅尿,实在是丢人。这说明我们虽然长得象土匪,脸还是很嫩。约定了谁敢把此事传出去就宰了谁后,我们就各奔东西。我跑回北京来,住在原来住过的地方。那地方原来是一所大学,里面有很多人。当时叫作"留守处",里面只住了很少几个人。很大的院子里到处是荒草,人们都下干校了。李先生原来也住在这个地方,后来才搬走了。这地方原来每个人都认识李先生。 

现在应该说说那天我去找李先生的原委。我从山西跑回来,住在留守处,那院里当时只有大崔一家住。这位大崔原来也是我们的邻居。除此之外,他还是我爸爸的同事,李先生的老同学,长得人高马大,笑口长开,一团和气。大家去下干校,家里还有些东西,是得找个大家都放心的人看着。大崔实在是最合适的人选。他老婆也是我们院的人,所以一起留下来。刚回来我去找他借房子,管他叫崔叔叔,管他老婆叫阿姨。借到了以后就改了口,管他叫大崔,管他老婆叫大嫂。当然这房子不能白住,我也得帮人家干点事,跑跑腿。所以大崔要找李先生,用不着自己去,告诉我一声就得。当时我非常年轻,也没有阳痿病。 

我从小就认识李先生。李先生从我小时候就在搞西夏文,而且我们两家过去是邻居,也记不清我第一次见到西夏文时是几岁。所以我后来见到西夏文,也不觉得有什么古怪。那种东西看上去很象汉字,笔划多得叫人头晕,很象是疯子写的,据说除了李先生,世界上没人能够读懂。因为只有李先生能读懂西夏文,所以他有大学问。但是他依然穷困潦倒,这是因为只有他能读懂西夏文,所以他的学问就得不到承认。假如别人能先读懂了西夏文,或许他的学问就有人承认,但是那又不是他的学问了。除此之外,还因为当时在文化革命中,北京城八百年的城墙被人拆掉了都没人说个不字,还谁关心西夏文。除了西夏文,我还记得隔壁李先生那间房子老是烟雾弥漫,李先生的脸色老是那么黄,好象得了黄疸病;李先生对我很凶。后来我才知道,过去李先生最烦有人不打招呼就到他那里串门。但是后来我专到他那里去串门,因为他反正没胆子把我吃了。所谓串门,就是没有事,跑到别人家里去坐着。但是那一天我去找李先生可不是没事,而是要告诉他,有人请他翻译些文件。没有稿,只有千字三毛钱的烟茶钱。李先生听了很高兴,马上就跑去了。在大天白日下骑着他那辆古怪车子,身穿着一件再生毛料的古怪衣服(那种料子和麻袋片是一样的),闯到那个原来是大学,当时叫留守处,而且人人认识他的地方去,并不是李先生的一惯作风。这是因为那个院子里现在没有几个人。人多时,李先生总是天黑后才去的。这说明李先生虽然穷困潦倒,依然很面嫩。 

我和李先生熟,除了过去在一个院里住过几年邻居,还因为不住邻居后,他还是老找我给他修收音机。李先生有一台里加牌的收音机,那收音机有小柜那么大,非常气派。这说明李先生并不是一惯穷困潦倒,还有过有能买起收音机的时候。这家伙晚上睡不着觉,想听听俄语台,但是听不清,就鼓捣他的收音机,胡乱修改线路。直到那收音机惨叫几声再也不响了,他才安心睡觉。李先生会那一点三脚猫的无线电,正好能把响的收音机修到不响。我去给他修收音机时,先要把他自己加上的放大全拆掉。同时还告诫他说,别只想着加放大,这不解决问题。还要想到有干扰:国家留着你的收音机,可不是让你听那些乌七八糟的东西。李先生说,是,是。我不听那些乌七八糟的东西,我只听外语。但是国家不相信李先生只听外语,还以为他要听乌七八糟的东西,所以还是要给他干扰掉。李先生又不相信收音机听不清是因为有干扰,老以为是灵敏度不够,就老往里面加放大。他的手还没有我的脚灵巧,一加就把收音机加死了。然后他就找我来修。这件事循环往复,周而复始。直到邻居揭发李先生偷听敌台,居委会把他的收音机拿走了方才告结束。我去找他那回,他刚刚失去了收音机。李先生见了我就说这件事,同时愁眉苦脸。我就安慰他说:这也好,省得再找我修。我这样安慰过以后,他好象更伤心了。这件事证明了一个道理:萨特先生说得很对,他人是你的地狱。我是李先生的地狱。李先生也是我的地狱:被他捅过的收音机就象个马蜂窝,焊过的线头就象些包锡纸的巧克力球。修完了他那个鬼东西,感觉就象吃了忆苦饭,不单肠胃受,而且拉不出屎。 

李先生走了以后,我在他那间小房子里还呆了好久,把他那一罐狗屁烟倒到了桌面上,把里面的死苍蝇、扫帚苗都挑了出来,然后又装了回去。我看了半天李先生的西夏文抄本,挨个数那些字的笔划。后来我从上面撕了一条纸,卷了一根烟,就替他锁上门,回来了。时隔二十年,我还清清楚楚的记得,我干了哪些事。但是我再也想不起来自己为什么要干那些事。大概这就叫手贱。 

3 

奶子府六号院里有一棵大槐树,盛夏时节,树上会掉下来数不清的槐蚕,弄得地上好象长满了会爬的草。那些草还会往家里爬。我对那儿的印象很好,因为那里一向邻近大内,街道上都立着禁止鸣笛的牌子,傍晚时分院里静极了。傍晚时分往往是阴天,云彩的颜色有点黄。黑暗凝集在古旧的窗棂上,附着在暗色的树皮上。在院里看天空,就象在水塘的水底,隔着厚厚的透明的水看水面。那院里还有一个个子高高的姑娘,傍晚时分穿一件床单布的大裤衩,赤着脚走来走去。我的视线久久的附着在她身上。朦胧中她是白蒙蒙的一团。久而久之,我的目光就和她的肌肤混为一体了。那是一种冷飕飕的感觉,好象早上的水汽一样。这种感觉真好,可惜过去了。 

我们医院旁边有个农贸市场,我常到那儿去买水果。后来那儿的人都认识我了。有人想和我拉近乎,就说,老师傅,你有五十了罢。我听了大怒,强忍着没发作。另一个说,老师傅,你的孩子都上小学了罢?气得我几乎动手打他。照他们看来,人要是活到了五十,又有了上小学的孩子,就算有成就。象我这样没到五十,还没结婚就阳痿的就是nothing了。虽然他们是想要我拍我马屁,我也不高兴。从那天以后,我再也不去那儿买桃了。从这件事你就可以想象当年别人对李先生的态度,和李先生对别人的态度。当年李先生虽然没有阳痿,但也没老婆。除此之外,他还没工作。大家当然以为他是矮人一等的家伙。平心而论,奶子府六号的街坊对李先生挺好的;又给他介绍工作,又给他介绍老婆。虽然那些工作不过是临时在副食店卖卖咸鱼,那些老婆都是残疾人,但是别人怎能知道李先生读通了西夏文,并且自视甚高呢。大家都觉得给他找个瘸子就是帮了他的大忙了。就是揭发他偷听敌台,也是怕他给街坊上招事,并无恶意。但是李先生对奶子府六号和街坊都深恶痛绝,老想搬出去。大崔找他翻译东西,他就借机搬到我们院,住进了我屋里。这件事当然有官冕堂皇的理由,(要翻的是一些内部文件,带来带去的不好,等等),那间房子又是大崔借给我的;他能借给我,当然也能借给别人,但我仍然很不高兴。这件事证明我一无所有,连睡觉的地方都是借来的。 

我现在依然一无所有,连睡觉的地方也不是我自己的。除此之外,又多了一个阳痿。现在马大夫要用心理疗法来给我治阳痿。所谓心理疗法,就是他反反复复对我说:兄弟,你想开点罢。人活在世界上,就是这一点享受哇。这话不错,但是不是我想不开,是它想不开。不知它听见了没有。 

现在该讲讲我们院的情况。我们院是一片房子,除了一些老房子,都是不加演饰的四方体,甭提有多难看。将来的人看到了这些房子,一定以为我们长着方鼻子,方眼睛。当时院里没人,长满了荒草。还有很多野猫,到了春天就嗷嗷叫。我和李先生,大嫂和大崔住在大门口一排平房里,就算看住了大门,可是别人从后面进来,把楼房的门窗都拆走了。我对那里的印象原来也很好,李先生来了才坏起来。李先生白天翻译文件,晚上也不睡觉,接着搞西夏文。我对此很不满,就坐在桌子对面,对西夏文发表自己的意见。我认为谁使用这种有这么多笔划的文字,就一定是笨蛋。这些笨蛋死了好几百年之后,还有人想把这种文字读出来,一定也是笨蛋。李生听了一声不吭。然后我又喝李先生的茶。李先生不知从哪里搞来了一些茶砖,都发了霉;喝过以后嗓子疼。我又告诉他,这茶的味道象墨水,真叫难喝。他听了以后还是一声不吭。说你已经把西夏文读通了,还看这玩意干嘛。他说,不看这玩意,还有什么可看的吗? 

和李先生同屋时,他告诉我说,他读通的不止是西夏文,还有契丹文,女真文;总之,他读通了一切看上去象是汉字又没人认识的古文字。这些文字有好多苏联人,法国人和中国人想读都没读懂。他认为这件事证明了他比大家都聪明,我认为这件事证明了他有毛病。对于这一点我还给出了证明如下:李先生干出了一件大家都干不出的事,这一点没有问题。这证明了他和大家不一样,这一点也没有问题。但是这种不一样是聪明还是有毛病,还没有定论。既然如此,就应该少数服从多数。大家说你聪明,你就是聪明,大家觉得你有毛病,你就是有毛病。很显然,认为他有毛病的人将是大多数。李先生听了为之语塞。后来他就不和我说什么了。 

现在别人也都以为我有毛病,所以很浅显的道理,都要告诉我。但是我也不觉得讨厌,因为我可以举一反三。比方说,马大夫以为我直不起来,是不知道人生在世就是这么一点享受,好比每年冬天只能买三十斤好的冬贮大白菜。他和老婆干事的心境与排队买大白菜时的心境相同。其实我知道一年冬天只有三十斤大白菜,但是我还是直不起来。因为我不是兔子,不那么爱吃大白菜。 

李先生住到我房子里以后,大崔就经常来了。他和李先生聊聊天,聊来聊去,总是当年在学校里的那点事,以至我到现在还能记得那些事:他们的学校叫做哈尔滨外专,四八年就成立了。五十年代初期是专门培养高级外语人才的,授课的全是专家,还雇了些老白俄来擦地板。在学校里不准讲中国话,讲一句做二十个俯卧撑。除此之外,还不准吃中国饭,只准吃红菜汤,刚来的吃不习惯,肠胃作起怪来,放起屁来抑扬顿错,每个屁都在一分钟以上。可惜他们也就美了那么一阵子。后来中苏交恶,这帮家伙全坐了冷板凳。其实李先生还会德文,法文,英文等等,但是咱们当时和那些国家也交恶。李先生说,假如加把油的话,他还能学会柬埔寨文,但是这种文字里有美国炸弹的味道,学会了也不是好饭碗。看起来他们两个老同学很是亲热,其实不是的。李先生背地里告诉我说,大崔真讨厌,尽耽误他的时间。大崔也说过,李先生真讨厌。有一阵子我不明白大崔在搞什么鬼:既然不喜欢李先生,还把他招来干嘛。后来才想明白了,这不关大崔的事。招李先生来的,另有其人。 

现在我很少到我们院去,因为它不再是"我的院"了。现在那里有好多的人,总数在两万六千以上。而在二十年前,若大的院子里只住了我们四个人,简直就象一座鬼城。我记得那片荒草离离的院子,草棵下面的石子儿和碎玻璃。马路上有好多风吹下来的枯枝,所有房子的门窗都用木条钉死了。住在附近的人有时溜进来发点洋财,倒也不敢偷什么东西。见到哪个厕所没钉死,就进去把三合板都拆走。我常常一个人在院子里漫步,看着风吹来的砂子和碎石若有所思。后来我就在闲逛中碰上了李先生给大崔带绿帽子。总的来说,这件事很难看。就和在草地上看见两条蛇绕在一起一样。在这种情况下我总是把两条蛇都打死。 

4 

我现在经常想起李先生,想起我们俩一起逛破烂市,买几毛钱一公斤的废纸边,五分钱一大把的锈笔尖。北京过去有好多破烂市,全称叫做废旧物资门市部,现在没有了。我到那种地方去买便宜电子管和废电容,李先生到那种地方去买散打的过期墨水。墨水这种东西也会腐败,坏了以后比大粪臭好几倍。和李先生住过一个屋以后,北京最脏的公共厕所我也进得去了。 

那一年李先生在我们院住了三个月,后来他又回奶子府去住了。其实他是被撵出去的,而且是我和大崔合力才把他撵走。这件事的详情不是我不肯讲,是我现在怎么也想不起来了。也可能推了他,也可能搡了他,甚至打了他,这些都记不得。只记得当时很有正义感。我这一辈子只有那一回有正义感,以后再也找不到那种感觉了。记得雨果说过,凡不可挽回的东西,都不属于人,属于上帝。所以正义感也不属于我,属于上帝。后来街道上把李先生的收音机还给他,等收音机坏了,他还来找我修。混到了那步田地,李先生不大要脸面。 

雨果先生还说过:凡人份内所没有的东西都属于上帝。所以象我这样的阳痿病人想娶小孙这样的漂亮姑娘为妻就是冒犯了上帝。上帝他老人家够狠的,把我们管得这么紧。 

我和前妻离婚时,听到了一种议论:阳痿根本就是一种思想病。换言之,上面的思想端正了,下面也会端正。人家还说,我一定是面对自己的老婆时想入非非,所以才阳痿。这话不是一点道理都没有的,当年面对我前妻的大裤衩时,我是有过一点古怪想法。如前所述,我自以为有写小说的才能,这种自信不是空穴来风。我的想象力极为丰富,以致我怎么也不敢相信自己的脑袋只有五号钢种锅那么大。在我该对我前妻行周公大礼时,脑子里忽然浮现出二十年前那个冬日骑车去找李先生时所见的情形:那个新婚少妇手提痰桶向我走来,把屎倒在铁蓖子上,那个少妇的模样不知为什么,活脱脱就是我前妻。这件事对我penis的物理性质大概是有一定的影响,但是要说那就是我阳痿的主因还难定论,因为当时我还在害胃疼。我在山西吃过好几年的土豆和连皮碾的谷子面,那些都是标准的健康食品。但是要是纯吃它们就很伤胃了。结婚那天,我虽然出席了好几个婚宴,但是什么都没吃到,所以到了晚上胃就疼得翻江倒海。在这种情况下,就该和我前妻取个商量。但是她早早的脱了大半衣服上了床,闭着眼睛直挺挺的躺着,脸色潮红,一句话都不肯讲。看到这种情形,我只好关了灯,在她身边躺下睡了。然后的事情我已经说过,她哭起来了。从此后,我的生活就进入了软的时期。 

后来我想起当年的事,觉得我前妻不会因为性欲没得到满足就哭了起来。她只是觉得在新婚之夜被弄破处女膜,是她份内当有的东西。只要是份内该有的东西还没拿到,就会引起一种急不可耐的情绪。至于弄破了疼不疼,她就不管了。 

李先生有一套二十卷本的汤恩比的历史哲学,我叫他教我英文,他就拿那书来教我,教得我七颠八倒,认识好几万单词,却一点语法都不会。我怀疑他对我破了他的好事怀恨在心,用这个法子来害我。汤先生说:人类的历史分作阴阳两个时期,阴时期的人类散居在世界各地,过着吃了就睡,睡足了再吃,浑浑噩噩的生活。后来人类又到一些河谷平原聚群居住,有了文明,一切烦恼就由此而起。与此相似,我的生活也有硬软两个时期,浑如阴阳两界。软了以后,回想起过去是如此的硬,简直不敢相信我也会有软的时候。 我性情冷漠,不善与人交往,一辈子不认识几个人。也许就因为这个原因,我很怀念那位搞西夏文的李先生。现在他也许还活着,也许死掉了,这都无关紧要。紧要的是我现在终于知道了他为什么撇开了好好的工作不要,去搞西夏文。这还是因为我已经软掉了。假如还在硬着的话,就只能想自己是多么的硬,想不到这类事情。在山西时听过一种地方戏,它发出一种极凄厉的,酷似挨刀断气的声音。听时阴囊兜紧,全部神经都在极大的痛苦中。可是大家都走十几里山路去听它。还有我那位前妻,用不着多么达练人情就能看出,将来她准是个母夜叉。可我过去为之颠三倒四。这种感觉就叫作硬。硬的时候我们急着去要自己份内的那点东西,丝毫不想它是不是自己想要的。等到有了一点自己想要的东西,不管是它是署了自己名字的小说,还是西夏文,就已经活到了另一界了。

\section{第三章}

我和小孙恋爱了一阵,就向领导上交了请求结婚的报告。从那时开始,大家就不再善意的对待我们。首先是登记结婚的证明老也开不来,总是说:这件事你是不是再考虑一下?我们再讨论讨论。实在逼急了,就说:介绍信找不到了,公章找不了。其次就是开始听到各种闲话。其实应该说,人们开始不再善意的对待小孙。这件事完全是她在办。我说“我们”,不过是表示自己没有完全置身事外。虽然我呆在地下室里不出来,但我已经在请求结婚的报告上签了名,并且认真听取了小孙的各种抱怨,就算尽到了责任,别的事我就帮不了忙了。我可以不参加政治学习,不去开会,不去看上级组织的乏味电影,可以尽情胡说八道;这些好处当然是有代价的。这个代价就是我说话别人可以不理会。因此我被叫作小神经。 

人家规劝小孙说,你千万不要和王二结婚。他这个人有点说不清。办公室的老太太还对别人说,他们俩的事拖一百年也不怕,反正不会造成人工流产。别人都说,不知我们结婚是要干什么。并且老有人把她叫到僻静处说:孙大夫,你真的要嫁他?你可真把自己看得一钱不值了。小孙说,她感到非常的不好意思,只好摆出一副瘦驴屙硬屎的架式说:我就是爱他嘛。但是晚上却对我说:我爱你个狗屁!除此之外,几乎每个人都要给她介绍对象,包括刚刚从护校毕业的不满二十岁的小护士。因为热心的人太多了,显得她简直象个花痴。假如不马上给她找个男人的话,她就要去和公牛睡觉,生下一个米诺牛来。对于这件事,她没有精神准备,感到惊慌失措。原先她以为结婚象在学校打报告申请实验动物一样轻松,写个报告交上去,然后拎着兔子耳朵到试验室,既可以把细菌打到它耳朵里,也可以把它炖了吃。现在我这九十公斤的公兔子就坐在对面,人家却不给她,可把她气坏了。 

小孙告诉我这些事时,都是在晚上。我的小屋里黑洞洞的,所有的灯都没有开,只靠一台示波器的绿光照亮。我不喜欢光亮。她在屋里走来走去,双手插在上衣口袋里。走了几趟以后,忽然对准我的耳朵大叫一声:都怪你!!!我耸耸肩说:阳痿还没治好呢,你别先把我耳朵治聋了。你怪我什么?她想了想说:算了,谁也不怪。不过这件事实在是真他妈的。而且她对我也起了疑心(这都是因为别人说我复杂),老是问:王二,你这人可靠吗?你能肯定自己没有偷过东西,或者趴过女厕所窗户吗? 

关于结婚的事,有一点开头我不明白。虽然我有阳痿病,但我还是个男人,起码户口本上是这样写的。群众怎样议论是另一回事,领导上决定问题,总要有个说头罢。这个谜后来马大夫给揭开了。他说他是康复科的主任,可以参加院务会,会上听见大家说,我有二十年工龄,十年院龄,加上中级职称;小孙又是本院的人。我们俩一结了婚,就是本院的双职工夫妇。其结果是婚后必须分给我们房子,这不是太便宜我们了?房子必须分给真正要结婚的人,而真正要结婚的人就是不管给不给房子都会结婚。他对我说这些话时,显出一付自己人的样子。但是我也不是傻瓜,一听就知道是上面有人叫他来传话。别看平日称兄道弟,但他不是自己人。所以我对马大夫说话用上了对领导说话的口吻:既然我们是为房子结婚,就别分我们房子了。他说,那是不可能的事。够了条件怎能不分哪。于是我就说,那就分我们房子罢。他又说,这也不成。你们想要房子就有房子,岂不是太便宜你了。想要房子的不能让他得房子,没想要的倒会得房子,这才符合辩证法。假如批了你们结婚,领导上会落入违反了辩证法的困境。唯一的办法就是不批准。我对马大夫说,其实我们真的不想要房子。您可以把我们俩都绑起来上电刑。假如我们在严刑拷打下说了是要房子,就别批准我们结婚。他说你又来了。到精神科去看看罢。说完就走了。 

有关分房子的事,我还有一点补充。我们医院只要分一套房子,全院都要搬家。这是因为院长分到了一间四室一厅搬进去,剩下三室的给科主任。科主任搬进去,两间一套让给主治医师;余类推,一直推到看门的老大爷。因此很多人的箱笼捆上以后就不打开了,一心一意等待搬家和再搬家,十冬腊月宁可穿着毛衣硬抗,也不开箱子找大衣;所以我们医院结了婚的少妇比没结婚的姑娘显得漂亮,冬天在室外只穿一件毛衣,一个个是那么苗条可爱。但是现在小神经和小孙要从主治医的层次插进去,打乱搬家的路线,就激起了公愤。 

那天下了班之后小孙到我这里来,眼睛都哭红了。原来领导也找她谈了,让她端正态度。她说道:为房子结婚,我是这样的人吗?王二,我不想和你结婚了。但是我还是要给你治阳痿病。我对小孙的想法一点也不理解。为房子结婚不是挺光明正大的吗?总比为性交结婚好听多了。但是我没有说这话,只是说,那就算了。你也别给我治什么病了。回去睡你的觉罢。她说,不行,听你的说法,我倒象个卑鄙小人了。我要陪你坐会儿。我说,你爱坐就坐罢。这时候我想起我表哥说过的话:人活在世界上,假如你想要什么,就没有什么。这就叫辩证法。所以假如你真想要什么的话,就别去想它。他说,他当年考不上大学,就是因为太想考上大学了。假如早懂了辩证法,就不会遇到这种不幸。我在大学里虽然学过辩证法,回回都是补考才及格的。而且那些任课教师总是这样讲:让你及格,我是昧了良心的。 

2 

晚上我一个人呆着时,总喜欢头戴立体声耳机。这样我虽然一个人呆在角落里,却与外面的世界取上了联系,可以听见各种声音,人家却听不见我;好象我从地下室往外看,看到了各种各样的人的脚,他们却看不见我一样。现在屋里有一个人,再也不能这样干了。为此我宁愿终身阳痿下去,也不愿有个人在我眼前转。这是因为她在我面前走动的样子,就象养貂场到了喂食的时间,铁笼子里那些貂一样。从人的角度来看,貂除了打盹的时候,都是神经病发作。假如人的行为象一条貂,那就更象神经病了。所幸她也有走累了的时候,那时候她也要坐下来歇歇腿。 

那天晚上我和小孙并排坐在一张床上,头上戴着立体声耳机。我开始反省我们俩之间的事,我知道,我们之间的关系就要完了,以后她也不会来看我,不会给我打饭,也不会趴在对面的木板床上算账了。这让我感到伤心,我真的很想要她,想把她留在我身边。这也许是因为,我以为她是一个自己人吧。现在自己人是越来越少了。由于有了这样的想法,就违背了辩证法。 

当年李先生说,自从创世之初,世界上就有两种人存在,一种是我们这种人,还有一种不是我们这种人。现在世界上仍然有这两种人,将来还是要有这两种人。这真是至理明言。这两种人活在同一个世界上,就是为了互相带来灾难。过去我老觉得小孙是自己人,现在我才发现,她最起码不是个坚定的自己人,甚至将来变成不是我们这种人也不一定。但是我不想说惹她生气的话,就闭上眼睛听广播。广播里正在劝女孩子们不要戴无纺布衬里的尼龙乳罩,因为无纺布的衬里会渗到她们乳房的导管里去,将来生了孩子没有奶。以前我不知道女孩子的乳房是象锅炉一样的设备,里面有很多管子,并且容易堵塞。于是我问小孙:你带什么样的乳罩?她回答说:尼龙的,无纺布衬里,将来没有奶。这不要紧,反正牛奶很便宜。原来她和我一样,正在听广播,并且听着一个台。后来我又有口无心的问道:你穿什么样的裤衩?她又说道:尼龙绸的。想看看吗?我说不了。后来她猛地跳了起来,一把从我耳朵上摘掉了耳机,对我大叫道:王二,你的毛病我找到了。你是淫物狂!这叫我很不高兴。不把事情问明白了就大呼小叫,简直是讨厌! 

有关裤衩的事是这样的:以前我结过一次婚,新婚之夜,我一看见我前妻那条皱皱巴巴的大裤衩,就不行了。这件事本不是没有挽回的余地,但是我前妻却大哭起来。引得丈母娘、大姨子都跑来了,问我:你什么意思罢。我妹妹可是个黄花闺女。叫她们这么一吵,我当然是越来越不行。最后终于离了婚。离婚之前我前妻还在医院哭闹了好几场,让大家都知道我不行,搞得我灰头土脸。但是对此我很能理解。她必须让大家都知道是我不行,而不是她有什么不好。小孙听了大笑说:我不穿大裤衩。咱们来试试罢。我苦笑一下说:还是别试为好。这件事现在对我已经很严重了。 

晚上我翻书时,耳朵上老架着耳机。耳机里有很多人说话,多数是女的。这些声音很不一样。有的声音很干脆,很紧凑。顺着那声音看去,可以看到一张小巧,湿润的嘴,紧凑高耸的胸膛和平坦的肚子。因为是和这些紧凑的东西共振,所以声音也紧凑。再往下看,就看到一条黑色尼龙绸的内裤。这也是一件紧凑的东西。但是顺着某些故作甜蜜的声音看去,就看到了肥大的鼻甲,身上的零件也松答答。再往下看,就是一条床单布的大裤衩,这东西也松答答。共振起来也就松松垮垮。除了这些区别,还有一些主观上的东西。有些广播员尽力让声音紧凑,所以说话有一点艰涩。另一些人讲话松松垮垮,一张嘴就是一大串,全是傻话。声音里传来的性有两种,一种讨人喜欢,还有一种叫人讨厌。以前我不懂这一点,所以结了一次婚。结果是使我只能欣赏广播里的性了。 

3 

后来我再想起小孙决定不和我结婚的事,也能够理解了。因为自从她和我表演了恋爱以后,软和硬这两个字就不再是物理名词,而归她专有了。工会分柿子,别人就这样对她说:小孙,来一点罢。软的。或者说,这个你准不喜欢,太硬。其实我们都决定要吹了,但是小孙还是老往我这里跑。别人也看不出我们要吹,还是说那些没咸淡的话。我告诉她说,讲这些话的都是些工友,是很朴实的人,别和人家当真,但她还是耿耿于怀。终于有一天,她在食堂里拿豆腐泼了大师傅一脸,然后哭着跑到地下室来,说道:快跟我走,什么也别问。呆会我叫你揍谁,你就揍谁。我跟着她跑上去,到了食堂里,见到一大群人。保卫科的人全来了,这也吓不倒谁。我可以直取目标,扭住他的领子。不管付多大的代价,都要把他的脸打烂。问题就在于找不到目标。过了一会,院长书记都来了,叫我们到办公室去解决问题。原来肇事的大师傅觉得在哪里都不能保证安全,已经跑到党委办公室去了。听说他事后对别人说:我真是晕了头啦,怎么就忘了地下室还有一个小神经! 

那天的事我们大获全胜,给讨厌鬼以沉重打击。大师傅被泼了一脸油汤,还要写检查。其实他不过说了一句:孙大夫,来一点豆腐罢。软的。这些话并不过份,不过是拾别人的牙彗,没有一点自己的发明。但是小孙已经火透了,就如一只骆驼,驮了好几百公斤,最后因为再加一根草的份量倒下了。 

这样处理领导上并非情愿,但是该大师傅很怕我,主动提出要写检查(后来他说,我要是被小神经打了,那还不是白打)。所以院长决定说我们几句:你们两个同志也真是的。都受过高等教育,是知识分子嘛,怎么也干这种哗众取宠的事情?他这些屁话还没说完,我的目光就如两道冷电在他脸上扫了一下,把他后半截的话扫回去了。书记来打圆场说:其实你们俩要结婚的事并不是没商量的,你们不要做不理智的事情。我就叫起来:谁说我们要结婚?他们听了都说,不结婚就对了。其实我们不是不准你们结婚,一套房子也能给得起。我们只不过是希望你们多考虑。小孙马上又叫道:谁说我们不要结婚?院长就说:今天就谈到这里,你们回去冷静一下罢。 

出来以后我问小孙:咱们不是说好了不结婚的吗?何不借此机会当众宣布一下?她说,咱们俩是说好了,但是没必要告诉他们。他妈的,结婚是咱俩的事,别人管得着吗?回到地下室里,想起没吃午饭,豆腐也泼了,赶紧在电炉上下挂面。吃完了,坐在光板床上晒太阳。吵了这么一架之后,吃饱了再一晒,就困了。小孙说,王二,你的胸围怎么这么大。我告诉她说是拉拉力器拉的。她说以后她也要拉健身器了。然后她打个呵欠说,太困了。我枕着它睡一觉,你没意见罢。说完她就枕着我的胸口睡着了。 

那天下午小孙枕着我胸口睡觉的事是这样结束的:她一觉睡到了快天黑,双手还圈住了我的腰,使我一动也不能动。我只剩了一只左手能动,就用左手掏出烟来吸。还有一件事使我感觉不便:她的头发又轻又软,经常跑到我嘴里来,我又要不停地把它吹开。所幸后来她终于醒了,爬起来伸了个懒腰说,真舒服呀!好多天没睡好觉了。做了好多的梦,全和工地有关。每个梦里都有打桩机。醒来才知道,是你的心在跳。你这里太好了。我要搬下来住。我听了没言声,因为她不是个自己人。我不欢迎她来住。过了一秒钟她又说,我干嘛不搬下来住呢?这就去搬! 

后来她真去把铺盖搬下来了,这件事连我都觉得象发疯。但是她说自己一点也没有疯,不过是想气气她们。于是她占领了对面的木板床,还带来了无数的毛巾,半干的小衣服,挂得满逃诩是。现在我在屋里走动,就要在三角裤底下经过了,这肯定要给我带来晦气。但是我一声也没吭。她要怎么干就怎么干罢,谈了小半年的恋爱,也该有这点交情。我不能象讨厌鬼那样小气。 

晚上睡觉前,我们又聊了一会天,谈到今天和大师傅打架。她说,从早上起就开始窝火了。早上她到病房时,看见有几个护士在交头结耳,传递某东西。她就走过去问:发什么好东西哪,不给我。那些护士一起笑得打跌道:东西倒是好东西,但和你没关系,你用不着。假如世界上没有王二其人,她马就能能想到,这是已婚的护士们在分发避孕工具。那样她就会红脸走开,或者说一句:臭美什么?恶心死了。但是世界上有我这个人,所以老有人在她背后窃窃私语,她就气昏了头,劈手就抢(这孩子手快极了,她说她在大学里打过垒球,是接球手,)。结果抢到手一大把避孕套。那些护士就说:抢什么?告诉你了,你用不着。小孙一瞪眼说:你怎么知道我用不着?再给我一把,要大号的! 

睡觉以前小孙说了一声:王二,往这边看。我抬头一看,发现她只穿了胸罩和裤衩站在地下,皮肤很白,胳臂腿很细,胸罩和裤衩都是黑色尼龙绸的。等我看完了以后,她就钻进了被窝,就着台灯看一本书。但是我还不能睡。我还要拉一百下拉力器,做一百个俯卧撑。这是因为我已经很胖了,如果不锻炼,很可能会死于高血压和心脏病。小孙说,我练得不对,这样只会越练越肥。但是我没理她。在这些事情上,我有我的一定之规。她就这样在我房间里住下了。 

4 

第二天一早我就起来拉拉力器,把弹簧撞的当当响。小孙在床上迷迷糊糊地说:你别这么抽疯好不好,让别人也睡个懒觉。但是我不理她。谁让你到我这里来住的?于是她就揉起眼睛来,那架势活象是猫洗脸;然后坐起来,在被窝里穿上衬衣,又伸出腿来,穿上袜子,就光着腿下地,拿了脸盆去打水。出了门又鬼叫一声被吓了回来,大概是看到了门口那个标本缸,觉得陌生罢。就这么折腾了一早上,我始终没有理她。后来她对我说:王二,你好象不高兴了。我说我总是这样的。她又说,不结婚的事你别往心里去。我是说着玩的。我始终是意志坚定的要嫁给你。我就说,我可真的有阳痿病。她又说,有关治阳痿的那些话你也别往心里去。我闹着玩哪。我说,那我就不知道你要嫁我干什么了。她说:我知道你好多事,要不要我一一讲出来?我把拉力器扔下说:不用了。咱们一块去吃早饭吧。这时我再不以为小孙是小娃娃,以为她是个自己人了。 

我十七岁时参加过北京市的数学竞赛,在复赛里得了八十来分。这件事本来是有点好处的,可以保送上什么大学数学系,但是后来我什么也没落着。小孙知道这件事。我告诉她,少提这件事。我现在对数学没有兴趣,而且连数都快不识了。我现在干的事是翻译“StoryofO”,已经译到第三遍了。有些地方拿不准,就托人找老外问。有一次问到一个法国lady头上,她向我赌咒说,从来也没听说过这本书。没听说过就没听说过罢,赌咒干嘛?虽然如此,我还是字斟句琢地译着。我干这件事,是因为我相信作者有极大的才气;还因为这本书不可能出版。假如一本书有可能出版,那么奸党也会去译,并且会争到打破头;因为有稿费。但是假如一本书既没有稿费,也不可能出版,我们不译谁译。小孙看了我的译稿,说道:王二,你要是去干翻译,准是一把好笔。但是你干嘛要翻这种书?连我这妇科大夫看了都要脸红,人家能给你出吗?我说,我根本就不想出。她说,不想出译它干嘛。我没接她的茬,因为这不是我们的逻辑。再说下去就是灾难。但我也不能说,你在给我带来灾难。这样说她就会给我带来更大的灾难。 

好多年前,我也说过这样的言论。那是在李先生的小屋里,抽着李先生的狗屁烟,喝着李先生的狗尿茶(那是用过期发的茶砖泡的),我在给李先生修他的狗屎收音机,一边修一边数落他。他听了不好意思,就埋头去看西夏文了。就在这时候我说,李先生,你看这玩艺干嘛?能当饭吃吗?他听了没理我。再问时就说,不能当饭吃。我又问:那你搞它干嘛?有人请你搞它吗?他再没吭声,就和没听见一样。对无聊的问题是否充耳不闻,这是我们和另一种人的分水岭。我听了小孙的话一声不吭,去拉了二十下拉力器,然后坐下来继续翻书。自从她搬进来以后,我的胸部越来越象两块门板了。小孙看着我拉拉力器,伸出一只手指抹抹鼻子,然后问:我说了什么错话了吗?我答道:没有。她听了要哭了:王二,你有什么话说哇。这么闷着干嘛。我就说:一本书,你看看它写得好不好,译得好不好就得了。害臊干什么。听了这话,她开始为自己的卑鄙言论惭愧了,就说:刚才那句话算我没讲好不好?拜托了。 

小孙住到我房里半个多月了,我对她秋毫无犯。虽然如此,我对她的行止也略有所知。她象只猫一样,喜欢钻被窝。一进了被窝就要把乳罩摘下来,挂在床头上,于是它就挂在那里晃晃当当,活象一付大号太阳镜,这使我很受刺激。她对我解释说,这东西就象缰绳一样,然后就把被子拉到下巴上看书,灯光把她的侧影照亮,我看了也很受刺激。她睡着了灯也不关,而我是有一点亮也睡不着----以前并不是这样的,所以经常半夜里起来去关灯。夜里经过她的床头,听见她轻轻的鼻息,也很受刺激。对此我很不满,和她说过一次。她回答道:你也抽烟哪,我也没有抱怨你,不是吗?一边说,一边瞪着眼睛看我,看了这个样子,我也很受刺激。我要是说,这是我的房子,那就是卑鄙的言论。所以我只好拉了一条线,把她的开关装到了我这边。要是看到她睡了不关灯,我就给她关上。此后半夜里经常听见她自言自语地说:这王二真讨厌,这不是逼着我犯错误吗!然后她就下了床,到我这边开灯来了。感到了她赤裸胸膛上传来的热气,我也很受刺激,只好紧闭着眼睛。现在我不但阳痿,还多了个失眠的毛病。我经常打呵欠,说晚上睡不好。我一打呵欠,她也跟着打呵欠,并且说:你以为我就睡得好吗?这件事证明了一点,在我和小孙之间,性的感觉等价于咖啡因,它的作用就是让人睡不着觉。 我和小孙之间,有好多话还没说。我翻译StoryOfO,不是因为它能让妇科大夫脸红,而是因为它是好的。这世界上好的东西岂只是不多,简直是没有。所以不管它是什么,我都情愿为之牺牲性命。我不知这话她是不是爱听。但是我知道还有一句话她肯定爱听,就是我觉得她也是好的。但是我没办法告诉她。人家不问我,我就讲不出话。所以我是小神经。

\section{第四章}

春天来到时,我把“StoryofO”又译了一遍,仔细校对了一遍,觉得译的很好,看不出任何败笔,就把它收了起来。干完了这件事,暂时又找不到别的事可干,就和小孙出去玩。在城里逛了一天,又在小饭馆里吃了晚饭,回来时天完全黑了。走进地下室的走廊里。她忽然悉悉索索地脱起衣服来,在一片黑暗中,我看到一个白色的模模糊糊的影子,然后又闻到了越来越浓烈的香水味。夜里四外的楼上都开着灯,所以眼前的走廊里有很多的白方块,就象是白漆涂成。小孙走到那些方块里去,马上就变得混身闪闪发光,而对面的标本柜上就会出现一个白色的影子。她就这样从一个个方块里走过去,在标本柜上留下了一个又一个影子。与此同时,门口的地下留下了蝉蜕似的影子。那些衣服扔在地下杂乱无章,好象是肢解了的人形。我把那些衣服检起来,小心翼翼地跟在她后面,避开窗口照进来的灯光。仿佛我一贯是这样作的似的。 

在每一块灯光里,小孙都回过头来朝我笑笑。那些人造月光照得她混身惨白。这种感觉好想在作梦一样。有时候她象是要伸个懒腰一样,把手向上伸起来,但又不完全是伸懒腰,因为她把身体弯向一侧,笑得很开心。我觉得这不象真的,所以不打算把它当真。但是我也感到一种冲动,要把鼻子伸入捧着的衣服里。那些衣服散发着香味,尚有余温。这种冲动就象狗想闻东西一样。 

走到房间里以后,小孙就径直钻进了被窝,一会就睡着了。我把她的衣服放在床头,回到自己床上,好久都没睡着。第二天早上起来以后,她不提起这件事,好象这件事只是她一时冲动,或者昨天晚上她在梦游一样。我也不便提起这件事。全当它没有发生。我想女人都有一种冲动,要把自己脱光。 

中午小孙告诉我说,她们科主任找她谈话,问她为什么要到我房间里住。小孙就反问一句道,你们为什么不准我们结婚?那老太太就期期艾艾答不上来。于是小孙提高了嗓子高叫起来:既然我们俩结婚是有其名,无其实,纯粹是为了骗房子;现在住到一起,又无名,又无实,又不要房子,你管这个干嘛。这一嚷嚷闹得全科都能听到。那老太太着了慌,委委屈屈地说:孙大夫,我求求你,不要这样。我这个科主任也不是我自己乐意当的。那口气好象是说,自己受了强奸一样。干完了这件事,小孙觉得兴高彩烈,得到了很大的满足,跑下来告诉我说,她又打了个大胜仗,并且要和我接吻以示庆祝。这孩子嘴里有薄菏味,大概是常嚼口香糖。她还把舌头伸到我嘴里来了。吻完以后,她打了个榧子道:French kiss!就扬长而去,回去上班了。但是我整个下午都不得安生,想着她裹在白色牛仔裤里的屁股,细长的两条腿和白色的护士鞋。除了屁股圆和腿长,她还有不少好处,包括给我打饭,和在熄灯以后陪我聊天,没得聊时就说和我阳痿有关的事。我们在一起,经常玩两种游戏,一种是情人的游戏,一种是医生和病人的游戏。到了前一种玩不下去时,就玩后一种。 

晚上我和小孙聊天时,她从被窝里钻出来,盘腿坐在被子上。这时候她背倚着被灯光照亮的墙。我看她十分清楚,那一头齐耳短发,宽宽的肩膀,细细的腰,锁骨下的一颗黑痣,小巧精致的乳房。乳头象两颗嫩樱桃一样。我也坐起来,点上一根烟,她眼睛里就燃起了两颗火星。我们俩近在咫尺,但是仿佛隔了一个世纪,有了这种感觉,什么话都可以说了。她问我,她长得好看吗?我说:很好看,她就说:真的呀。 

我和小孙谈这些事时,她的床在窗口射入的灯光中,我的床在阴影里,我们住的地方就象阴阳两界。这叫我想起了我自己的生活,它也有阴阳两界。在硬的时期我生活在灯光中,软了以后生活在阴影里。在这一点上,我很象过去的李先生。只是我不知道李先生是不是也阳痿过。 

2 

当年我问李先生,西夏文有什么用,他只是一声也不吭。后来他告诉我说,他根本不想它有什么用,也不想读懂了以后怎么发表成果。他之所以要读这个东西,只是因为没有人能够读懂西夏文。假如他能读懂西夏文,他就会很快乐。读不懂最后死了也就算了。后来他的晚景很悲惨,因为他终于把西夏文读通了,到处找地方发表,人家却不理他。因为他不是在组织的人,是个社会闲散人员。还因为当时对西夏文已经有了五六种读法,都读得通。李先生说,他的读法最优越,但是没人理他。后来他就把自己保留多年的西夏文拓片,抄本等等都烧掉了,到处去找工作,终于当上了一个中学教员。再以后就得了老年痴呆症。我算了算,李先生那会也有五十六七,到了该得这种病的年龄了。最后一次我见到他,他已经不认识我了。 

在我的硬时期,总有一个女人是我的意淫对像。有一年冬天我的意淫对像就是大嫂,她当时是个大个子中年女人,两条大辫子,在那个时期,她那个年龄的女人留辫子,可有卖俏的嫌疑。大嫂的脸也很长,下巴稍有点翘。当时我觉得下巴翘一点好,比较俏皮。脸白白净净的,有点浅麻子。一天到晚老在笑,好象缺心眼的样子。做为意淫的对像,她的屁股太大,腰也比较粗,这都是美中不足的地方。但是她老是笑嘻嘻的,弥补了体形的不足。我想象她作爱时也是这样笑嘻嘻,这会让我激动不已。 

小孙说,我简直是个下流坯。她希望我永远阳痿下去。但是说了些话之后,她又承认这样说不对。她说她是医生,我是病人,医生不该说病人是个下流坯。现在我们又玩起了那种医生和病人的游戏。她问我那个大嫂是谁,我告诉她说,是我们院大崔的太太。她又问,什么院,什么大崔。这个话说起来就长了。我从小住在一所大学里,因为我的父母都是该大学的教师。大崔和大嫂是比我父母小十几岁的另一对教师,是我们的老邻居。而且大崔和大嫂都认识李先生,他们是老同学。这件事的背景就是这样。 

我给小孙讲过:那一年冬天我去找李先生,其实就是奉了大嫂之命。大嫂和我说起这件事前,她正蹲在水管前面洗带鱼。而和我说这事时,她站了起来,身上穿了一件红色的套头毛衣,里面衬了一件蓝格子的浅色衬衣。我看到她脖子上有了几道皱纹,下巴也有一点两层的意思,但是大嫂还是满好看的。她对我说,让我去找李先生,让他来一下,有件事情可以照顾到他。我听着这些话,眼睛却在她胸口上看。在毛衣底下,她乳房的样子还是满好看,只是略微有点下垂了。就在这时候,她用洗鱼的手在我脸上抹了一把,说道:看什么看!快干你的事去。她这种满不在乎的口吻很使我turnon。 

小孙对我说,她也是很不在乎的。这种口吻很难说是医生对病人的口吻。这种口吻使我很紧张。好在她马上换了一种口吻说,好啦,讲你的大嫂罢。那天她叫你去找李先生,到底是为了什么? 

其实那件事没有什么重要性。大嫂让我告诉李先生,有一批材料要翻译。没有稿费,但是有一点烟茶费,每千字三毛钱。这就是说,你翻译了一千个字,可以抽一支好香烟,或者喝一杯好茶。就是不抽好烟,这笔钱也是太少了。但是李先生答应了干这个活儿。不但如此,他还以取稿子方便为名,搬到了我们院,住到了我的房间里。这件事我已经讲过了。现在我怀疑,每千字三毛钱,就是对李先生也太少了。当年李先生接下这个活,动机根本就不纯。 

比这还糟糕的是,大嫂和李先生开始在我眼皮底下幽会起来。见了面就接吻,手还不老实,李先生那对前蹄老从大嫂的毛衣底下伸进去。我一看见这种景象,就咳嗽不止。大嫂听见了,就说:小陈,你好不好回避一下?我们俩玩哪。当时我真是恨得牙根痒痒。大嫂孩子都老大的了,还这么不自觉,老要玩。而且李先生又老又难看,和他有什么好玩?要玩可以和我玩嘛。除了这些讨厌之处,李先生还得了不睡觉的毛病,白天和大嫂鬼混,翻译稿子,夜里还不忘看他的西夏文,二十四小时连轴转。象他那么大岁数的人怎么会有这么大的鬼精神? 

有关大嫂的情形,还有不少可以补充的地方。据说她一贯搞破鞋,年轻时就因为和苏联专家有不正当的关系,被开除了团籍。结了婚以后,还是乱七八糟。大崔也管不了她,只能要求她对丈夫好,对孩子好,在饭菜里别下耗子药。李先生在院里时,大崔气得要命,要打她。她也是满不在乎:要打你就打,只别打脸,打哪儿都成。可以用赶面杖,不准用火钩子----动铁为凶! 

大嫂对我说,她爱上李先生了,甘愿为他牺牲性命。我以为大崔要和她离婚了,但是大崔没提这个事。他告诉我说,大嫂经常会爱上谁,甘愿牺牲性命也有有好几回了,但是她到现在还活着哪。 

只要我肯耐心等待,没准大嫂也会爱上我,甘愿为我牺牲性命。但是我最缺的就是耐性。我绝对不会象李先生那样搞了二十多年西夏文,最后变成一个白痴。我搞什么事都是要么不干,要么立竿见影。 

3 

我和小孙聊天,经常聊到一半,她就说:今天聊到这里罢。再晚睡明早上查房起不来了。然后就钻进被子睡着了。当个住院医师实在辛苦,有时候白班,有时候夜班,睡觉的时间老是不够。小孙的眼窝常常发青,她问过我是不是该涂眼晕。我说你想涂就涂好了,我没什么意见。她说岂有此理,涂眼晕就是涂给你看,你居然没了意见!看到别人忙忙叨叨,我经常感到惭愧,因为我老觉得可干的事情太少。翻完了“StoryofO”,就再也找不到象这样的书了。但是我也不能象那种人一样,去干没意思的事情。我们的人在这种时候,往往是去证明一个定理,或者发明一个体系。比方说,费尔马和爱因斯坦干的事就是这样。但是去证明一个定理往往会掉进陷井里----有些定理可能没有证,遇上了一辈子都会陷在里面。而发明一个体系则谈何容易。想来想去,只有写小说比较有把握。但是自打认识了小孙,我就一个字也没写过。我写的小说,她每一页都要看,这就破坏了我的写作情绪。想想罢,昨天刚写出来的东西,今天就成了谈资,那是多么叫人厌烦。剩下只有一件事可干,那就是睡觉。 

后来我又想把李先生和大嫂的事讲给小孙听,但是她不肯听,说道:我知道,大嫂爱上了李先生,这就结了罢?讲点别的吧。其实那个故事还长得很。用大嫂的话来说,一次爱情就象吃一个巧克力壳的冰棍。开头是巧克力,后来是奶油冰激凌。最后嘴里剩下一个干木棍。我所讲的李先生,连巧克力壳都没化呢。但是小孙不肯听。她说与其听你这些胡说八道,不如到外面去看死人。说完她真的从床上爬了起来,拿了手电,到走廊上去了。 

我想给小孙讲的事,包括夜里李先生和大嫂在一块坐着念俄文诗,几几嘎嘎,听得人好不心烦。那时候我躺在灯影里,大棉被也挡不住那些卷舌音。这时候我只好想象自己是土耳其苏丹,带了队伍征讨俄罗斯草原。逮住了讲这这种话的人,就让他们脑袋瓜子朝上,屁眼朝下,坐在削尖的木棍上。还有他们俩唱一个俄文歌,叫作嘎嘎林。一边嘎嘎,一边亲嘴,就象斗鸡一样;听了叫人头大如斗。后来他们听我咳得那么厉害,也有点不好意思,到外面去找地方了。但是那已经是开了春后的事。在此之前,他们一直是在我面前表演。开了春以后,我们院子里就开始闹猫,天一傍了黑,它们就开始哀号。我总怀疑里面也有李先生和大嫂的一份。据说母猫的那玩艺里长了倒刺,公猫插进去,就象插进了蝎子窝一样,疼得拼命嚷嚷。不知李先生和大嫂是不是这样。 

我想给小孙讲的事还包括,那一年春天特别暖,晚上外面刮着黑色温暖的风,那种风就象一条深不可测的暖水河,叫人见到它就想脱光了衣服跳下去。用不着别人告诉我我就知道,这条河就是未实现的性欲。现在我心里就流着一条这样的暖水河。我要干的事不过是把这件事说一说。 

小孙刚出去时,我很上火。因为我想让她听我讲话,但是她却跑了,把我扔在突然到来的寂寞里。我在地下室里住了十年,原本最能忍受寂寞,现在却受不了啦。 

寂寞是我的选择,正如在地下室里离群索居是我的选择一样。在我看来,寂寞就是可以做一切事的自由,这是因为你做什么都没人知道,或者知道了也不理会。所以我能够翻译“StoryofO”,李先生能够读西夏文。自从我割断了对女人的单恋,寂寞就真正归我所有。寂寞纯黑如夜,甜蜜如糖,醇如酒。 

但是现在我却受不了寂寞了,因为它不再是过去那个样子,既不黑,也不甜了;而是惨烈如白昼。 

我坐在床上发了一会愣,忽然想起小孙出去半天了,我该去看看她。一推门看见门口堆了一堆衣服,原来现在她身上什么都没穿。我赶紧回去拿了件大衣,顺着灯光赶了去,看见她正趴在标本柜上,高举手电,正往死人眼窝里看哪。我叫道:你疯了,要冻死呀!她却头也不回地说:你别管我。 

后来我把她裹在大衣里,抱回屋里去,一直抱到了我床上。在黑暗里摸到了大衣前襟上是湿的,又赶紧去拿手巾给她擦脸,还用那种眼泪鼻涕一块擦的手法。然后我又给她揉揉脚。她带着哭声说:别的地方也得揉揉。于是我就往上揉去。从膝盖往上开始有鸡皮疙瘩,她混身都冷透了。我赶紧哄她几句: 

算了,我不讲那些无聊故事了。 

她说:和故事无关。你得爱我! 

我说:我爱我爱。这时正好揉到腰上,她趁势就钻了过来抱住我。我拿大衣把她包上,放在腿上,好像个大包裹。我和小孙恋爱就是这样的。 

4 

我和小孙之间带有性意味的接触是这样开始的:我的手从大衣前襟里伸进去,把她那两个小小的冷冰冰的乳房摸了一遍;与此同时,她的手也从衣襟里出来,揪住了我的耳朵,定好了位,来和我接吻。这两件事干好了,我又把大衣裹好,把她裹成个铺盖卷,放在膝盖上,又拿被子给她搭上腿。她在这个铺盖卷里宣布说,她现在很幸福,可以听我讲李先生和大嫂的事了。她还说,刚才不幸福,那件事就不能听,因为它属于幸福的范畴。我告诉她说,李先生现在是个大傻子,一天到晚只会摇头。大嫂是个老太太,头发掉了多一半。她说她不管这个。反正我最后也要变成老年痴呆,她也要变成老太太,这些都没什么,这些都能受得住。受不住的事是现在想要幸福却不能幸福。原来她的幸福就是被摸上一遍,再打成个铺盖卷,我既有手,又有打铺盖卷的材料,就可以给她幸福。这件事听了让人放心。我接着给她讲有关李先生的事,一讲到猫儿叫春,她就喵喵的叫唤。但是一点不象猫儿叫春,倒和一般的猫叫很象。小孙的行为通常就象一只猫,这里就包括了喜欢钻被窝,喜欢被包裹起来。但是猫就不会长雪白的小屁股和圆嘟嘟的乳房。 

后来我又给他讲李先生的故事。我们院子有一片待拆的危楼,我常到那里去转转,看看有什么可拆的,结果就碰上了他们两个给大崔带绿帽子。但是不是当面撞见,是在对面一座门窗都没了的破楼里。李先生他们呆的也是一座破楼,也没有门和窗子,他们所在的地方比我呆的地方矮半层。我看到的时候,大嫂的衣服都躺在地下了,摆得倒象个人似的。她只穿了皱巴巴的针织背心和床单布的大裤衩,跪在地下铺报纸。李先生的样子更难看,他脱得精赤条条,正在摆弄自己的那玩艺。那玩艺更难看,半直不直的样子,完全看不得。 

但是小孙却说,这也没什么看不得,人家相爱嘛,什么东西都能拿出来摆布。象这类的话,她早就听说了。前些日子她申请结婚时,有一些护士大姐吓唬她,什么话都说出来了。比方说,女孩子结婚时都要过一关,就象猪要挨杀一样。要是快刀子热水,死了也就完了。就怕碰上了钝刀子,软刀子,想死都死不了,那才叫难受哪。还有人说,遇上丈夫不成,就得拿手给他弄,后来就象摆布了死人,洗八遍手也去不了那股恶心劲。小孙说,那些话一点也吓不倒她,因为她是大夫,死人都敢摆布。她又说,让我摆布一下你好罢?也许能把你的阳痿治好呢。我说:算了,不好意思。她说有什么不好意思的?我都让你摆布了。这时候我闭上眼睛,小孙那双小手就出现在眼前。指甲老是剪得那么短,并且洗得老是那么白。这双手拿东西有个特别的样子,比方说,转个旋钮,从来不去抓,而是用侧握的姿式。拿个东西也是很用力,很仔细的样子。把自己交到这样的手里,大可以放心。所以我想了半天终于下定了决心,说道:好罢。呆会可别瞒怨我。她说,绝不会的。咱是这样的人吗? 

我想,假如女人都象小孙那样好说话,世界上就不会有阳痿的人了。但是我前妻就不是这样,她心情激动,满脸通红,上了新床就躺倒了象个死人。全身绷得甚紧,以致我把自己想象成一支打井队,要在地层上钻眼。但是我作这种对比,丝毫没有挖苦前妻的意思。不管怎么说,是我阳痿嘛。小孙说,你别紧张,就当咱们俩在一块吃个桃。这是因为咱们好嘛。她还帮我脱衣服。然后我平躺下,她一只手握住了我的把把说:王二,家伙很大呀。我告诉她说,这是马大夫用铅锤拉的,原来没这么大。等到她伸手兜了我几下,那东西就膨涨起来。于是她又说:你这就叫阳痿呀!我说平常我是阳痿的,今天也不知怎么了。她说,你说这话就叫没良心了。什么叫“也不知怎么了”?这是因为我呀! 

干这事时,小孙骑在我身上。也不知是为什么,开头很艰难。她一面从牙缝里吸凉气,一面说:刚才哭过,影响了情绪,里面很干。我觉得也是很干,就说,要不算了罢。她说:哪能算了。你不懂,老实躺着罢。于是我就闭上了双眼,一动也不动。后来就湿了,也进去了。从这时开始,我就不算是个阳痿病人。她向前俯下身子,我伸出手来抚摸她。我摸她的脸,那张白白净净的小脸就出现在我眼前。我甚至看到了她脸上有几粒雀斑,是我以前没看见的。象我这样的人,一点也不怕变成瞎子。睁着眼能看见的,闭上眼我都能看见。后来我又把手放到她肩上,大姆指和食指触到了她的脖子。她脑后那些乌黑的发根就进入我脑海里了。我最爱雪白皮肤上那些乌青的发根了。今后我可以尽情的亲近那些乌青的发根,这是一个很美好的前景。我的手还可以伸到这个小小的身体的任何地方,但是我不想那么做,我就想停留在现在这个地方。 

后来她把身体俯得更低了,这时我能感到她呼出的热气。等到事情完了,她在我身边躺下时说道:咱们俩同时达到了性高潮。这很重要。我问为什么重要?她说这样我也不必为你服务,你也不必为我服务,性生活谐调,好呗。我想,要是能搂着她睡一觉,那就更谐调了。谁知她是那样的不老实,睡了没有五分钟,就撩开被子坐起来,说道:你等我一会,就从我身上跨过去跑掉了。 

5 

那天晚上,我和小孙做完爱,她跑到自己床上去了。过了一会,她拿了一面小镜子回来,坐在我身上,拿了手电,往自己胯下照。然后她又转过身来,跨住了我的上半身,用手电照着说:你看。我抬头一看,看见她的帝王将相。和图谱上画的有点不同,是一副血肉模糊的惨状。我吃了一惊,说道:怎么了?她从我身上下来,钻进被窝说:你干的好事呗。 

后来小孙把头贴在我胸口上,我都快睡着了;猛然想起她说过自己不是处女,禁不住说出了口:不对呀。她马上就扬起头来说:什么不对什么不对,口气相当凶。我说我想起一本小说。她又问什么小说什么小说。我说,法国中尉的女人,那里面有个莎拉,干过你这种事。她就说,你真混。我想这样说是揭了她的疮疤,就不说了。正要睡着,她又把我推醒,说道:告诉你,以前我干过一回,谁知他干得这么不彻底。我说噢。然后我又问:你告诉我这个干嘛?她说:我告诉你这个,免得你太臭美! 

但是那天晚上我们到此还没有睡。她又跳起来说,等我一会。然后她又往腿上套裤子。我问她要干什么,她说上楼去,找人看看。我说这么厉害?我陪你去。她愣了一会儿,然后说道:那太好了!你也不能一点良心都没有,是吧? 

后来我陪她到了妇科病房,把值班大夫叫了起来。但是我没敢到放着妇科椅子的房间里去,呆在外面,听见她在里面说:王工那个家伙,一只手都握不住!真是疼死我了!等到出来以后,我问她:既然如此之疼,你怎么不告诉我呀?她又说,没那么疼,骗她们呢。这我就不懂了,好好的骗人家干嘛。她说:笨蛋。申请结婚,要房子呀。有房子不要,便宜他们吗? 

果然到了第二天中午,马大夫就来找我传话说,让我们到楼上去拿介绍信,领导上批准我们结婚了。他又对我谈了一阵辩证法,但是我没听。我知道领导上的打算:因为涉及到了房子,所以要控制已婚人数,原则上不批准结婚。但是假如不批准就要引起非法的性交,那就批准,因为两害相衡取其轻。马大夫还说,想调小孙去康复科搞科研,治疗阳痿。因为她居然能把我的顽症治好,显然是很有办法。后来小孙真的调过去了。科研工作比门诊,病房都轻松多了。她到康复科去给阳痿病人的妻子办学习班,讲Masters和Johnson那套方法,只不过是用中国式的术语---什么握,捏,捺,按,抹,勾,挑,弹八法,听上去就非常难懂了。 

后来我和小孙结了婚,住在两间一套的房子里。开头每逃诩干,后来每三天干一次,现在是每礼拜干一次,因为我毕竟是四十三岁了。小孙扬眉吐气,走到院子里都趾高气扬。因为她自以为无比性感,连阳痿病人见了她都不阳痿了。 

从此以后,寂寞再不归我所有。这有好处,也有不好处。走进了寂寞里,你就变成了黑夜里的巨灵神,想干啥就干啥,效率非常之高。你可以夜以继日的干任何事,不怕别人打断,直到事情干成。但是寂寞中也有让人不能忍受的时刻,那就是想说话时没有人听。 

现在我不再拥有寂寞了。我的事非常之多。我既然不阳痿,也就没有理由神经。没有了这两项毛病,就得上楼去开会。除此之外,我又成了中年业务骨干,什么仪器都得修了。除此之外,还得念念英文,准备到美国去接仪器。院长对我说,咱们医院懂电子的人太少了,你的病好了,就得多干点。还听说他对别人说:这套房子给得不亏!除此之外,我现在已经混迹于奸党之中了,说话作事都得特别小心。除此之外,回家还要应付小孙。除了背熟她身上的全部性敏感带,还要背熟她感情上的敏感带,才能讨到她的欢心。 

我和小孙结婚的事就是这样的。现在我们还住在一套房子里,有时还干那件事,但是已经谈到过离婚的事。我们医院不批准我们离婚,并且说:早就识破了我们想再骗一套房子的狼子野心。所以我们还在一起住。但是小孙说:她不能白给我做饭,我得给她洗裤衩。 

我现在和小孙做爱时,岂止是温存,简直是恭敬得很。我还告诉她说,我觉得她是好的,这世界上好的东西不多,我情愿为之牺牲性命。她说她很爱听这句话。但是她又说,我休想因为这句话逃掉洗裤衩的家务劳动。她还说:吾爱王二,吾更爱有人洗裤衩。这话是从柏拉图的名言"我爱苏格拉底,我更爱真理"变化而来,但就是柏拉图,也绝不肯给苏格拉底洗裤衩。 

小孙告诉我说,她是个女权主义者。所以用不着我告诉她,她就知道自己是好的。当时她到地下室去找我,就是向我证明这个。她所以要和我离婚,倒不是不喜欢我,而是要和我分清楚一点。这个小家伙现在又给我上课,不过不是讲纪晓岚,而是讲薄加丘(!),"从前有个教士告诉一个木匠说,他骑的母马,晚上就会变成女人和他睡觉……",一听就叫人脑仁疼。这是<十日谭>里那个装马尾巴的故事,不过又被她讲了个七颠八倒。 现在你买一本<十日谭>,里面就没有那个故事了。这肯定是因为这个故事比其它故事编得都好。小孙说,这个故事说明了"你们男人一个好东西都没有",因为我们想的是让她们白天变成马去干活,晚上变成女人陪我们睡觉。我就是这样倒霉,前半辈子阳痿,后半辈子又娶了女权主义者为妻。但是我没有再次阳痿的打算。我认命了。
