\chapter{似水柔情}

这件事发生在南方一个小城市里,市中心有个小公园,公园里有个派出所。有一天早上,有一位所里的小警察来上班,走进这间很大的办公室。在他走进办公室之前,听到里面的欢声笑语,走进去之后,就遇到了针对他的寂静。在一片寂静之中,几经传递之后,一个大大的黄信封支到了他的手里。给他这个信封的警察还说:小史,这些邮票归我了。小史看到这个大信封上的笔迹和花花绿绿的香港邮票,就知道它是谁寄来的。在这个屋子里,在这些人目光的注视之下,当然以暂时不打开信封为好。但是他忍耐不住,还是打开了。信封里除了一本薄薄的书,别无他物,甚至书里也没有一封夹带的信,扉页上也没有一行手写的字。小史在翻过了这本书之后,感到失望。就在这时,他看到扉页上印着:“献给我的爱人”看到了这行字,他长长地出了一口气,好像有一块石头落了地。他甚至还用手指仔细擦了一下这行字,然后把它锁在了抽屉里,出门去了。 

有关这本书,我们需要补充说,它是阿兰寄来的。信封上写了阿兰的名字,书上也印了他的名字,这本书就是阿兰写的。这间房子里的每个人都看到了,小史收到了一本阿兰寄来的书,看到了他如何急匆匆地搜索这本书,他如何急迫地注视扉页上的题字,又如何抚摸这行字——这一切都在静悄悄的众目睽睽之下。这屋里的人发现了小史很动情、很肉麻,绝大多数的人看到了这些就可以满意了。假如有一个人认为这还不够,需要打开小史的抽屉,把这本书拿出来给大家传看,她肯定是小史的老婆点子。她真的这样做了,拿出那本书,仔细地搜索,终于找到了扉页上的题字,让所有的人都看到小史这不可告人的一面。当然,这样做是不理智的。然而,点子远不是个理智的人。 

小史收到了阿兰寄来的书,心情非常的兴奋。他的心脏为之狂跳,脸为之涨红,手也为之颤抖;他不愿呆在办公室里让别人看,所以跑了出来。这种心境我们称之为爱情。他先去上厕所,而那个厕所是同性恋集会的场所,他在那里碰见了几个圈子里的人,那些人对他的神色十分注意,他也不想被这些人所注意,所以赶紧跑了出来,在公园里漫步,而在公园里见到的每一个人都注意地看着他。他觉得所有这些注意都不怀好意。他仔细回避这些目光,走到公园的一个角落里。这里有一把长椅,一年之前,阿兰就坐在这个椅子上。此时此刻,小史也坐在这个长椅上,拿手遮住自己的脸。阿兰离开他已经有一段时间了,他看不到他,摸不到他的身体,嗅不到他的气味,但是他寄来的一本书却能使他如受电击。这种感觉从未有过。小史自己也说:这就是爱情吧。 

二 

与此同时,阿兰生活在遥远的地方,在一间白色的房间里。这间房子很是空旷,只是在窗前地上放了一个床垫子。天气炎热,他赤身棵体,只在胯下盖了一条白色的毛巾被。在床垫上,放着他写的书,和寄给小史的那本一模一样。在他面前放了一个大可乐瓶子,还有一个空杯子。对他来说,那个小公园,公园里的人等等,都成为过去了。但是他当然记得这些人,还有绝望。这就如孤身经过一个站满了人的长廊,站在你面前的人一声不吭地闪开了,一切议论都来自身后。这就如赤身睡在底下爬满了臭虫的被单上。这是来自身后的绝望。来自身前的绝望则是一个张牙舞爪的小警察,羞辱他,苛待他,但是阿兰爱他。这个小警察就是小史。 

有关这位小警察,我们需要补充说,他容貌出众,衣着整洁,气质潇洒,正如你会在某个副食店里见到一位容貌出众的姑娘,并且为她在这里而纳闷,这个公园派出所里也有这么一个小警察。这个公园是同性恋聚集的场所,他们议论起男人时,就和议论女人一样,所以这个小警察就是公园里的大众情人。当然,这一点他自己并不知道。当他到公厕里去时——他当然也要到那里去,因为那个公园里只有一个厕所,而且大众情人也要上厕所,所有的隔板后面都伸出人头来看他。很难想像谁会追踪一个异性的大众情人到厕所里,看他在抽水马桶(更不要说是蹲坑)上的形象,但是同性恋是会的。 

三 

有关这位小警察,我们知道,每次他值夜班时,都要到公园里逮一个同性恋来做伴。有一天晚上,他在公园里的长椅上逮住阿兰。当时阿兰正坐在别人身上,和那个人卿卿我我,忽然被手电光照亮了,一副目瞪口呆的模样。小警察在灯光后面说道:嘿,你们俩,真新鲜哪。这时阿兰站了起来,而另外那个人则跑掉了。小警察走上前来,一把抓住他的手腕,说道:你别也跑了。阿兰并不经常被逮住,所以当时他感到如雷轰顶,目瞪口呆。小警察用手电在他脸上晃了一下,说道:挺面熟嘛。你是不是老来?而阿兰因为过于惊慌,答不上来。 

小警察说道:和我走一趟吧。他拿出一副手铐,说道:用不用给你戴上?阿兰结结巴巴地问道:什么?小警察说道:你想不想跑?阿兰答道:不……不。小警察说:那就用不着了。就该是这样,跑得了和尚跑不了庙嘛。他把手铐别在腰里,拉着阿兰走了。时隔很久,当时的恐惧早已散去之后,阿兰说:那天晚上开始时是多么美好啊。小史的一握使他怦然心动,而小史要给他戴上手铐,又使他很是兴奋。这些感觉使他张皇失措了。 

四 

小警察拉着阿兰走在林荫追上,一面走一面教育阿兰。有趣的是,这场教育开始的时候,竟是劝阿兰不要太害怕,不要这么哆哆嗦嗦。他是犯了错误,但是这个错误并不大,“既不是抢银行,又不是拦路强奸”,所以,小史也不想把阿兰怎么样。我们知道,他抓阿兰是要消遣他一场(这件事将会在后面谈到),假如阿兰吓得像一团烂泥,就会没意思了。 

时隔很久以后,阿兰回味那个夜晚,觉得小史拉着他走路,就像一个大人拉着一个捣蛋孩子一样。这就是说,前者竖着走,后者横着走。不过,他更愿意把这想像成一个漂亮男孩拉着他的捣蛋女朋友,这当然是出于他自己的嗜好。 

小警察这样说到阿兰所犯的错误:“你们的事我都知道……十个扁儿不如一个圆,是吧。差不多得了,那么讲究千吗。扁就扁点吧,现在是社会主义初级阶段,咱们别来外国人的高级玩艺儿。”这倒使阿兰吃了一惊,说:“这不是扁和圆的问题……”然后小警察粗暴地打断他说:甭跟我说这个,我不想听。时隔很久之后,阿兰回味这些话,觉得小警察的这些粗暴、无知的话不仅是有趣,而且是非常的可爱。 

五 

那天晚上在公园里,小警察拉着阿兰走,阿兰偷偷把手伸到他的后面,摸他的屁股。可能哪个捣蛋女朋友也会摸自己的漂亮男孩,但是他摸得过分了一点。阿兰的手极富表现力,并且变化多端。小警察渐渐走不动了。走到路灯下,小警察放开了他的手,阿兰放慢了脚步,逐渐和警察分开。最后他在路灯下站住,小警察单独行去,越走越远,直到在夜幕里消失,都没有回头。那天晚上,阿兰就这样逃掉了。而后来,他想起这件事,却感到无限的追悔。显然,他该和小警察到派出所里去,聆听他的训斥,陪他度过一夜。除此之外,伸手去摸小警察的屁股,是个粗俗无比的举动。而逃跑这件事又实在有违他的本心。阿兰把这件事归咎于粗俗男子的劣根性。是他自己把那一晚的浪漫情调破坏了。 

阿兰以为,爱情的美丽不是取决于爱人,而是取决于自己:取决于自己的温文柔顺。因此,就算有最可爱的爱人,但是自己不温文不柔顺,也不算是美好的爱情。因为这个原故,后来,阿兰又坐到了小史的面前,这完全是有意为之。而这一次小史不但毛躁,而且有点要算旧帐的情绪。这一点完全在阿兰的意料之中。 

六 

晚上,小史回到派出所的办公室里来,打开台灯,在灯下翻看那本书。他希望这本书里会谈到他们之间的爱情,但这却是一本历史小说,这使小史大失所望。不管怎么说,他还要读这本书,因为这是阿兰写的。但是他会抱着失望的心情来读这本书。现在阻碍他真正阅读这本书的,就是阿兰本人,或者说,是有关阿兰的种种回忆。一年之前,阿兰坐在公园里的椅子上。他穿了一件丝绸的衣服,是紫色的,在公园里很是显眼。在小史看来,他的样子过于花哨,除此之外,他还觉得阿兰看他的样子相当古怪。 

想起那天阿兰的举动,小史的心里升起报复的愿望,就把他抓到派出所里去。 

小史命阿兰蹲在墙根下。蹲在他左面的是一个教艺术的教授,蹲在他右面的是一个搞建筑的民工,一共是三个人。左面的教授有口臭,右面的民工有汗臭,气味不比厕所里好。这里的规矩是要他们用最低的蹲法,也就是说,像屙屎一样的蹲着,双手伸在膝盖上,脑袋朝前耷拉着,阿兰觉得这种姿式不雅,总要把重心——说准确了,是臀部,升起来,放在小腿上,但被警察喝止。人家要求他们这样蹲着想想自己的错误,而正常的人这样蹲着时只会想到屙屎,这样就给他们的错误定了性--这种错误十分的肮脏,而另外的蹲法就不那么肮脏,因而背离了他们错误的性质,所以被禁止。阿兰就这样蹲在墙下了。 

阿兰进去之前,在一种绝望的心境之中。蹲了一会之后,就摆脱了这种心境,因为他感到屁股疼,大腿疼,渴望能站起来,这样就不绝望了。蹲在他旁边的教授年纪较大,很快就吃不消了,发出了一种若有若无的哼哼。而那位民工则感觉较好,因为他比较习惯蹲着,而且也有事干,不觉得无聊。这件事就是从肋上往下搓泥球。他们蹲在一位女警察(该女警察就是点子)座位后面,使她感到干扰。她特别反感民工搓泥,所以拿了一张纸,让他搓在上面,然而这样做了以后,她还觉得恶心,就跑了出去,把那位小警察找了回来,让他把这些人弄走,“省得蹲在这里恶心”。她说话时用的是命令的口吻。说完这些话她就走开了,并且要求回来时这里没有讨她厌的东西。这些东西就包括阿兰在内。所以小警察就尊旨而行,把民工叫起来,打了他两个嘴巴,罚了他的款,让他走了。把教授叫了起来,教育了一顿,也让他走了。以上两位都是同性恋,都是有“行为”被看见了,民工还有敲诈的行为,这些在小警察的话语里有所流露(小警察说:你都干什么了?什么都没干我会逮你们吗?少废话,罚款……等等。他对民工说话,就不用教训孩子的口吻)。 

小警察在言谈中,特地提出了教授的年纪和地位,以此来激发后者的羞耻之心。但是他没有理阿兰。然后他请自己的太太回来坐,而后者不满意他说:怎么还剩了一个。对于请她凑合的要求,她的回答是:我不!结果是她在小警察的位子上坐,小警察出去了。然后出入的警察们问起墙角蹲的是谁,她就说,是小史的朋友。听说叫做阿兰。那些人说,阿兰,听说过。他们还说到,小史值夜班。看来小史要把阿兰留到夜班时谈谈。人们还说,小史可别扣阿兰搞了起来,阿兰可不一般——人家说阿兰很性感(当然是开玩笑)。女警察挺起了胸膛,很自信他说:他敢干! 

这些谈话在阿兰眼前进行,但大家都视阿兰如无物,否则不会把这些荤段子讲了出来。这些使阿兰又忘掉了屁股疼,回到了绝望的心境——这就是说,他又十分颓唐地蹲下了。 

七 

从异性恋,尤其是从警察的角度来看,被逮住的同性恋者就如一些笼子里的猴子。小史也是这样的看阿兰。天快黑时,那位小警察——小史给自己泡了一碗方便面,与此同时,阿兰坐在了地上,小警察连看都没看他,就说道:没让你坐下。阿兰又蹲了起来。过了一会儿,阿兰又弓着腰站了起来。小警察说:我也没有让你站起来啊。阿兰又蹲下去,屙屎的姿势。这时小史用托儿所阿姨的口吻,说道:唉(读ei)叫干吗再干吗。小警察吃完了面条,给自己泡了一杯茶,然后伸了一个懒腰,这才看了阿兰一眼,说道:你可以站起来了。此时阿兰站起来,揉自己的膝盖。然后,小警察坐在办公桌后面,半躺在椅子上,舒舒服眼地伸开了腿,说道:过来吧。等阿兰开始走时,他又说:自己拿个凳子过来。阿兰拿了凳子,走到屋子中间放下,坐在上面,两个人开始对视。这漫长的一夜就此开始了。 

在那漫长的一夜开始的时候,小史对阿兰说:你丫说点什么。后者就说:我是同性恋。他还补充说:每个人的生活都有一个主题,而他的主题就是同性恋。小史那时的主题是反对同性恋,但是也很能欣赏这种直言不讳。但是当小史问他是怎样一种同性恋法时,他却一声不吭了。时隔一年之久,小史坐在办公桌前,手里拿着阿兰的书,他当然能够明白,阿兰之所以不回答自己是怎样一种同性恋法,是因为他爱他。他就是这样一种同性恋法。小史翻开阿兰的书,浏览目——他希望在这本书里提到他们之间的爱情,但这却是一本历史小说。当然,他还要看这本书,因为它是阿兰写的。他怀着极其复杂的心情看这本书,因为这本书和他本人没有关系。时间就停在他将读未读的时候了。 

八 

阿兰说,那漫长的一夜是这么开始的: 

在一片寂静之中,阿兰低声说(声几不可闻):扁儿是社会主义,圆儿是资本主义。 

小警察不相信自己的耳朵:大声点,我没听见。 

阿兰:扁儿是无产阶级,圆儿是资产阶级。 

小警察强忍着笑,说道:再大点声。 

阿兰大声说道:扁儿是社会主义,圆儿是资本主义;扁儿是无产阶级,圆儿是资产阶级! 

小警察笑着招他过去,仿佛是要说什么悄悄话,但给了他个大耳光。 

阿兰挨了嘴巴倒在地上。小警察恢复了镇定,说:起来吧。阿兰起来后,他又说:坐下吧。阿兰坐下之后,他清清喉咙,说: 

“咱们说的不是扁和圆的问题。” 

阿兰笑了。 

然后,经过了长久的对峙之后,小警察忽然笑了,说道:咱们俩扯平了。这么干坐着有什么劲,你丫说点什么吧。此时他就不再像个警察,而像个通常的顽劣少年。阿兰后来坐在床垫上,对着小史的相片说,我想到这些,不是为了记住你的坏处,而是要说明,我是怎样爱上你的,我为什么要爱你。 

九 

那一夜里主要的事是:阿兰向小史交待自己的事情。这是因为天太热,前半夜睡不成觉,还因为派出所里蚊子很多,总之,小史在值夜班时总要逮个同性恋来审一审,让他们交待自己的“活动”,以此消闲解闷。那一夜逮住的是阿兰,他交待的不只是“活动”,所以那一夜也不止是消闲解闷。 

阿兰从地下站起来时,两腿好像不存在了,过了一会儿,它们又变得又疼又麻。但是他尽量不去想这些煞风景的事。现在小史就坐在他面前,他是他的梦中情人,又是他的奴隶总管……稍微犹豫了一会,阿兰就开始说。他想的是:要把一切都说出来。 

在那漫长的一夜里,阿兰这样交待自己:“我小的时候,一直呆在一间房子里。这间房子有白色的墙壁和灰色的水泥地面,我总是坐在地下玩一副颜色灰暗、油腻腻的积木,而我母亲总是在一边摇着缝纫机。除了缝纫机的声音,这房子里只能听到柜子上一架旧座钟走动的声音。每隔一段时间,我就停下手来,呆呆地看着钟面,等着它敲响。我从来没问过,钟为什么要响,钟响又意味着什么。我只记下了钟的样子和钟面上的罗马字。我还记得那水泥地面上打了蜡,擦得一尘不染。我老是坐在上面,也不觉得它冷。这个景象在我心里,就如刷在衣服上的油漆,混在肉里的砂子一样,也许要到我死后,才能从这里分离出去。我从没想过要走出这间房子,但这是不可避免的事。”“有时候,我母亲把我招到身边去,一只手摇着缝纫机,另一只手解开衣襟,让我吃她的奶。那时候我已经很大了,站在地下就能够到她的乳房,至今我还感到它含在我嘴里,那个软塌塌的东西,但是奶的味道已经忘掉了。到现在我不喝牛奶,也不吃奶制品。我母亲在喂我之前,喂我之后,和喂我的时候,始终专注于缝纫。她对我无动于衷。当然,我还有父亲,但是他对我更是无动于衷。我小时候的情况就是这样的。” 

十 

阿兰所交待的另一件事情是这样的:“我走出那所房子时,已经到了上中学的年龄。” 

“上学路上,我经常在布告栏前驻足。布告上判决了各种犯人,‘强奸’这两个字,使我由心底里恐惧。我知道,这是男人侵犯了女人。这是世界上最不可想像的事情。还有一个字眼叫做‘奸淫’,我把它和厕所墙壁上的淫画联系在一起——男人和女人在一起了,而且马上就会被别人发现。对于这一类的事,我从来没有羞耻感,只有恐惧。说明了这些,别的都容易解释了。” 

“班上有个女同学,因为家里没有别的人了,所以常由派出所的警察或者居委会的老太太押到班上来,坐在全班前面一个隔离的座位上。她有个外号叫公共汽车,是谁爱上谁上的意思。” 

她长得漂亮,发育得也早。穿着白汗衫、黑布鞋。上课时,阿兰久久地打量她。下课以后,男生和女生分成两边,公共汽车被剩在了中间。“我看到她,就想到那些可怕的字眼:强奸、奸淫。与其说是她的曲线叫我心动,不如说那些字眼叫我恐慌。每天晚上人睡之前,我勃起经久不衰;恐怖也经久不衰。” 

“公共汽车告诉我说,她跟谁都没干过。她只不过是不喜欢来上学吧了。这就是说,对于那种可怕的罪孽,她完全是清白的;但是没有人肯相信她。另一方面,她承认自己和社会上的男人有来往,于是等于承认了自己有流氓鬼混的行径。因此就在批判会上被押上台去斗争。… 

“我至今记得她在台上和别的流氓学生站在一起的样子。那是个古怪的年代,有时学生斗老师,有时老师斗学生。不管谁斗谁,被押上台去的都是流氓。” 

“我在梦里也常常见到这个景象,不是她,而是我,长着小小的乳房、柔弱的肩膀,被押上台去斗争,而且心花怒放。” 

“在梦里,我和公共汽车合为一体了。” 

十一 

那天夜里,阿兰就是这么交待自己,当然,小史一句也没有听到,因为他根本就没有讲出来,只是在心里对他交待着。或者他听到了没有往心里去。不管怎么说,小史当时不是同性恋者。他想听到的不过是些惊世骇俗的下贱之事。因为这个原故,所以双方对那一夜的回忆不尽相同。说实在的,小史对于同性恋者的行径知之甚详,他们在厕所里鬼混,肛交,口淫等等。这些故事他早已经听得不想再听。他只是想要听听阿兰怎么吃“双棒”,并且想要知道他怎么双手带电。但是阿兰说:这些事是瞎编的,或者是别人的事,以讹传讹传到了他身上。这使小史很不开心,要求他一定要说点什么。阿兰就没情没绪他说起他的初次同性恋经历:和高中一个姓马的男同学的事。这件事在非同性恋者听来索然无味,他在姓马的男同学家里,先是互相动了手,然后又用嘴。阿兰尝出了该男同学的味道——他是咸的。这件事使他体会到性的本意,那就是见到一个漂亮的棵体男子,在你面前面红耳赤,青筋凸显,快乐的呻吟。同时品尝到生命本来的味道。当时他想道,自己是这样的温顺,这样的善解人意,因而心花怒放。这些话使小史很是反感,觉得阿兰很贱,甚至想要马上就揍他一顿。 

时隔很久之后,小史对这件事有了新的体验。他很想听阿兰的“事”,在听之前很是兴奋;听到了以后,又觉得阿兰很贱。与其说他憎恶阿兰曾经获得的快感,不如说他憎恶这种快感与己无关。这就是是说,他身上早就有同性恋的种子,或者是他早就是同性恋而不自知。要不然就不会每次值夜班都要听同性恋的故事。 

十二 

时隔很久之后,小史坐在灯下,手里拿着阿兰的书,想明白了阿兰当时为什么不想谈到自己的同性恋经历和同性恋恋人,而喜欢谈不相干的事,这谜底就是:阿兰爱他,而他要求阿兰谈这些,是因为当时他不爱他。他终于打开了阿兰的书。阿兰的书里第一个故事是这样的:在古代的什么时候,有一位军官,或者衙役,他是什么人无关紧要,重要的是,他长得身长九尺,紫髯重瞳,具体他有多高,长得什么样子,其实也不重要,重要的是他在高高的宫墙下巡逻时,逮住了一个女贼,把锁链扣在了她脖子上。这个女人修肩丰臀,像龙女一样漂亮。他可以把她送到监狱里去,让她饱受牢狱之苦,然后被处死;也可以把锁链打开,放她走。在前一种情况下,他把她交了出去;在后一种情况下,他把她还给了她自己。实际上还有第三种选择,他用铁链把她拉走了,这就是说,他把她据为己有。其实,这也是女贼自己的期望。 

阿兰在书里写道:正是阳春三月,嫩柳如烟的时节,那位衙役把她带到柳树林里,推倒在乌黑的残雪堆上,把她强奸了。然后,她把自己裹在被污损了的白衣下,和他回家去。阿兰说:铁链的寒冷、残雪的污损,构成了惨遭奸污的感觉。她觉得这样的感觉真是好极了。小史想到这件事的始未,觉得阿兰简直是有病了。阿兰的书,阿兰在那一夜里对他讲到的一切、还有阿兰对他的爱情,这三件事混在一起,好像一个万花筒。而这三件事在阿兰那里就变得很清楚。这就是,在阿兰写到这段文字之前,他想到了自己在那一夜坐在派出所里,看着小史狰狞的面孔,感受了他对他的轻蔑。这些感觉就幻化成了那个女贼在树林里惨遭蹂躏,她白衣如雪,躺在一堆残雪之上。这个女贼就是阿兰。虽然如此,假如不把阿兰对小史的爱考虑在内,这个场面还是脉络不清。 

十三 

阿兰说,有些事情当时虽然想到了,但是不能写在这本书里。他坐在床垫上,回味着自己的书。这本书并不完整——书不能是完整的想像,想像也不能是完整的书。其实,阿兰的想像还包括了那个衙役的性器,坚硬如铁,残忍如铁,寒冷也如铁,正向他(她)的体内穿刺过来。这是刑讯,也是性。但是,这个想像就在他的书里失去了。阿兰想到,也许他还要写另外一本书,直言不讳地谈到这些感觉。 

阿兰说,这本书当然产生于他对小史的爱情,甚至可以说,完全产生于他和小史在派出所里度过的漫长的一夜,虽然已经失去了很多,但还是原来的样子,只要想到这本书,就能把那一夜全部收拢在胸。而把那一夜完全收拢在胸的同时,他就勃起如坚铁。阿兰把毛巾被撩起了一点,看看自己的那个东西,又把它盖上。这东西好像是爱情的晴雨表。阿兰觉得它并不是很必要,因为他是这样的柔顺,供污辱,供摧残;而那个张牙舞爪的器官,和他很不合拍。 

阿兰的中学时代就要结束的时候,公共汽车被逮走了,送去劳教,当时的情景他远远地看到了。她用盆套提了脸盆和其他的一堆东西,走到警察同志面前,放下那些东西,然后很仔细地逐个把手腕送给了一副手铐。这个情景看起来好像在市场上做个交易一样。然后,她抬起并在一起的两只手,拢了一下头发,拿起放在地上的东西,和他们走了。这个情景让阿兰不胜羡慕——在这个平静的表面发生的一切,使阿兰感同身受,心花怒放。 

十四 

在阿兰的书里,还有这样的一段:那位衙役用锁链扣住了女贼的脖子,锁住了她的双手,就这样拉着她走,远离了闹市,走到了河岸上。此时正是冬去春来的时候,所以,河就是一片光秃秃的河床,河堤上是成行的柳树,树条嫩黄,在河堤下面背阴的地方,还有残雪和冰凌。这个景象使女贼感到铁链格外的凉。这个女贼不知道衙役要把她带到哪里去,只是跟着走。 

实际情况却是大不相同:公共汽车那一行人走到学校门口,围上了很多的学生。他们就在人群里走去,她双手提着自己的东西,那些东西显得很沉重,所以她在绕着走——除了走路之外,她想不到别的了。后来,当她钻进警车时,才有机会回头环顾了一下,看到了人群里的阿兰。因为看到了他,她微笑了一下,弹动几根手指,作为告别。 

阿兰说,他觉得公共汽车是因为她的美丽、温婉和顺从才被逮走的。因此,在他的心目里,被逮走就成了美丽、温婉和顺从的同义语。当然,小史逮他,不是因为他有这些品行,而是因为传闻他手上有电,吃过双棒,等等。但阿兰愿意这样来理解。也就是说,他愿意相信自己是因为美丽、温婉和顺从被小史逮了起来;虽然他自己也知道,这未必对。 

十五 

阿兰说,公共汽车对自己会被逮走这一点早有预感。她对阿兰说过,我现在贱得很,早晚要被人逮走。而后来阿兰感觉自己也很贱,这是中学毕业以后。 

阿兰到农场去了(也不一定是农场,可以是其他性质的工作,但这个工作不在城里面)。他这个人落落寡欢的不爱埋人,这种气质反而被领导看上了,上级以为他很老实,就让他当了司务长,给大伙办伙食,因此就常去粮库买粮食。以后,他在粮库遇上了邻队的司务长。那个人也显得郁郁寡欢,不爱埋人。出于一种幼稚的想像,阿兰就去和他攀谈,爱上了他。这个故事发展得很快,过不了多久,在一个节日的晚上,阿兰在邻队的一间房子里,和这位司务长做起爱来。做了一半,准确他说,做完了阿兰对他的那一半,还没有做他对阿兰的那一半,忽然就跳出一伙人来,把阿兰臭揍了一顿,搜走了他的钱,就把他撵出队去。然后他在郊区的马路上走了一夜,数着路边上被刷白了的树干,这些树干在黑暗里分外显眼。像一切吃了亏的年轻人一样,他想着要报复,而事实上,他决无报复的可能性。谁也不会为他出头,除非乐意承认他自己是个同性恋。到天明时他走进了城,在别人看他的眼神中(阿兰当时相当狼狈),发现了自己是多么的贱,他甚至觉得,自己是世界上最贱的人了。从那时开始,他才把自己认同于公共汽车。 

十六 

阿兰说道:初到这个公园时,每天晚上华灯初上的时节,他都感觉有很多身材颀长的女人,穿着拖地的黑色长裙、在灯光下走动,他也该是其中的一个,而到了午夜时分,他就开始渴望肉体接触,仿佛现在没有就会太晚了。夜幕降临,华灯初上,使他感觉受到催促,急于为别人所爱。小史皱眉道:你扯这些干什么,还是说说你自己的事吧。阿兰因此微笑起来,因为这是要他坦白自己的爱情。一种爱情假如全无理由的话,就会受惩罚;假如有理由的话,也许会被原谅;这是派出所里的逻辑。公园里却不是这样,那里所有的爱情都没有理由,而且总是被原谅,因而也就不成其为爱情。这正是阿兰绝望的原因。他开始讲起这些事,比方说,在公园里追随一个人,经过长久的盯梢之后,到未完工的楼房或高层建筑的顶楼上去做爱,或者在公共浴池的水下,相互手淫。他说自己并不喜欢这些事,因为在这些事里,人都变成了流出精液的自来水龙头了。然而小史却以为阿兰是喜欢这些事,否则为什么要讲出来。作为一个警察,他以为人们不会主动地对他说什么,假如是主动他说,那就必有特别的用意。总之,他表情严肃,说道:你丫严肃一点!并且反问道:你以为我也是个自来水管子吗?阿兰没有回答,这个问题就这样被岔开了去。他只是简单他说,爱情应当受惩罚,全无惩罚,就不是爱情了。 

十七 

小史对阿兰做出了这样的论断:你丫就是贱。没有想到,阿兰对这样的评价也泰然处之。他说,有一个女孩子就这样告诉他:贱是天生的。这个女孩就是公共汽车。在公共汽车家里,阿兰和她坐在一个小圆桌前嗑瓜子。她说:我这个人生来就最贱不过。这大概是因为她没有搞过破鞋就被人称作是破鞋,没有干过坏事就被人送上台去斗争,等等。后来她说,来看看我到底有多贱吧,然后她就把衣服全部脱去,坐下来低着头继续嗑瓜子,头发溜到她嘴里去,她甩甩头,把发丝弄出来,然后她看到阿兰没有往她身上看,就说:你看吧,没关系。于是阿兰就抬起头来看,面红耳赤。但她平静如初,把一粒瓜子皮喷走了以后,又说:摸摸吧。阿兰把颤抖的手伸了出去,选择了她的乳房。当指尖触及她的皮肤时,阿兰像触电一样颤了一下,但是她似乎毫无感觉。后来,她把手臂放在桌面上,把头发披散在肩头,把自己的身体和阿兰触摸她的手都隐藏在桌下,平静他说,你觉得怎么样啊。忽然,她看到一只苍蝇飞过,就抓起手边的苍蝇拍,起身去打苍蝇。此时,公共汽车似乎一点都不贱,她也不像平日所见的那个人。因为她有一个颀长而白亮的身躯,乳房和小腹的隆起也饶有兴趣。只有穿上了衣衫,把自己遮掩起来时,她才显得贱。 

公共汽车对阿兰说过,每个人的贱都是天生的,永远不可改变。你越想掩饰自己的贱,就会更贱。唯一逃脱的办法就是承认自己贱,并且设法喜欢这一点。阿兰小的时候,坐在水泥地面上玩积木时,常常不自觉的摸索自己的生殖器,这时候他母亲就会扑过来,说他在耍流氓,威胁说要把它割了去,等等。后来她又说,要叫警察叔叔来,把他带走,关到监狱里去。在劝说无效时,她就把他绑起来,让他背着手坐在水泥地上。阿兰就这样背着手坐着,感到自己正在勃起,并且兴奋异常。他一直在等待警察叔叔来,把他带到监狱里。从那时开始,一个戴大檐帽,腰里挂着手铐的警察叔叔,就是他真正的梦中情人了。一个这样的警察叔叔就坐在他面前,不过,小史比他小了十岁左右。他承认自己贱,就是指这一点而言。 

阿兰想到公共汽车在自己面前裸露出身体的情形,想到她像缎子一细密的皮肤,就想说,这一切也该属于小史。他想把自己的一切都奉献出来——但是他没有说。首先,公共汽车已经没有了十七岁的身躯;其次,这种奉献也太过惊世骇俗。于是,这个念头就如一缕青烟,在他脑海里飘散了。阿兰说,刚从农场回来时,他曾想戒掉同性恋,也就是说,不要这样贱。所以他就到医院里去看。那里有个穿白大褂的大夫,坐在桌边用手拔鼻毛,并且给他两沓画片,一沓是男性的,另一沓是女性的;又给了他杯白色的液体,一杯是牛奶,另一杯是催吐剂,让他看女人的画面时喝一口牛奶,看男人的画片时喝一口催吐剂,就离去了。阿兰就开始呕吐起来。但是这里的环境和他正在做的事使他感到自己更贱了。 

阿兰浏览了整套画片,那些画片制作粗劣,人物粗俗,使他十分反感。他并不是特别讨厌女性,他也不是特别喜欢男性。他只是讨厌丑恶的东西,喜欢美丽的东西。后来,阿兰放下了画片,坐在水池边,把那一杯催吐剂一口一口地喝了下去。他呕吐的时候,尽量做到姿势优雅(照着水池上的镜子)!他甚至喜欢起呕吐来了。 

小史对阿兰说、没见过像你这样的——这就是说,没有人承认自己贱。所以,这就叫真贱。在大发宏论的同时,他没有注意到阿兰容光焕发,并且朝他抛过了一个媚眼,也就是说,小史没有注意到、阿兰爱他。他只注意到了表面的东西:在这间屋子里,有警察和犯了事的人,有好人和贱人,有人在训人。有人在挨训;没有注意到事情的另一面。 

十八 

阿兰坐在派出所里,感到自己是一个白衣女人,被五花大绑,押上了一辆牛车,载到霏霏细雨里去。在这种绝望的处境之中,她就爱上了车上的刽子手。刽子手庄严、凝重,毫无表情(像个傻东西),所以阿兰爱上他,本不无奸邪之意。但是在这个故事里,在这一袭白衣之下,一切奸邪、淫荡,都被遗忘了,只剩下了纯洁、楚楚可怜等等。在一袭白衣之下,她在体会她自己,并且在脖子上预感到刀锋的锐利。 

阿兰谈到了自己的感觉,他常常无来由地感到委屈,想把自己交出去,交给一个人。此时他和想像中的那位白衣女贼合为一体了。那辆牛车颠簸到了山坡上,在草地上站住了,她和刽子手从车上下来,在草地上走,这好似是一场漫步,但这是一生里最后一次漫步。而刽子手把手握在了她被皮条紧绑住的手腕上,并且如影随形,这种感觉真是好极了。她被紧紧地握住,这种感觉也是好极了。她就这样被紧握着,一直到山坡上一个土坑面前才释放。这个坑很浅,而她也不喜欢一个很深的坑。这时候她投身到刽子手的怀里,并且在这一瞬间把她自己交了出去。但是阿兰没有把这个感觉写进他的书里。一本书不能把一切都容纳进去。 

后来,阿兰讲的这个爱情故事是这样的:几年前,他还十分年轻,英俊异常,当时在圈里名声甚大。有一天,他和几个朋友,或者叫做仰慕者,在街上走着的时候,有一个男孩子远远地看着他,怯生生地不敢过来搭话。后来当然还是认识了,这孩子是个农村来的小学教师。他仅仅知道城里有个阿兰,就爱上了他,走到他面前,说:我爱你。并且又说,你对我做什么都成。这是一种绝对的爱情,也是一种绝望的奉献,你不可以不接受。但是这种绝望比阿兰的绝望容易理解,因为它是贫穷。阿兰到他家里去过,看到了一间满是裂缝的黄泥巴房子,一个木板床支在四个玻璃瓶子上,还有两个被贫困和劳作折磨傻了的老人。在那间破房子里,阿兰像一位雍容华贵的贵妇一样爱上了这位小学教师,并且在那张木床上,请他使用他。他觉得这种感觉真是好极了。 

阿兰还想说:那个男孩穷到了家徒四壁的程度,身上却穿了一套时髦的牛仔裤,骑了一辆昂贵的赛车。他像一切乡下来的人一样要面子,但他走过来对阿兰说:我爱你,我只属于你。他让阿兰看到的不但是他漂亮的外表,还有他破破烂烂的家,他走投无路的窘态——也就是说,提示了一切线索,告诉阿兰怎样地去爱他。但是阿兰的决定完全出乎他的意外,他要像爱一位百万富翁。爱一位帝王一样爱他。所以阿兰想说:自身生而美丽是多么的好哇——就像一个神祗一样,可以在人间制造种种的意外。 

可能,阿兰还讲过他和这个男孩之间别的事,比方说,他和他在河边上张网捕鸟,但是逮到的却是一些不值钱的老家贼。或者,他们长途贩运服装,结果是赔了钱。这些故事的结局都是一样的,在那间破泥巴房子里,阿兰摊开了身躯,要求那男孩爱他,并且把心中的绝望宣泄在他身后。那间房子里总是亮着一盏赤裸裸的灯泡,而布满了裂缝的墙上,总是爬着几只面目狰狞的大蟑螂。午夜里,雾气飘到房间里来了,在床边上,堆着那些旧书籍、旧报纸——穷困的人连一张纸条都舍不得扔——能被绝望的人爱,是最好的。但是小史对这个故事一点都不理解,他说,你丫讲的,就叫爱情了?阿兰只好把这个故事草草讲完,后来那个小学教师想让阿兰娶他妹妹,这样他们三个人就可以在一起过了。阿兰对此感到厌恶,就拒绝了。他可以爱他,但不想被拖到这种生活里去。现在再也不会有人怯生生地看着他,或者因为绝望走过来说:我爱你。年轻、漂亮、性感,有时候也是一种希望。但是这些东西阿兰已经没有了。 

阿兰的样子现在看起来还是可以的。不过他已经开始化妆了,眉毛是纹过的,脸上也涂了薄薄的一层冷霜。最主要的是他的皮肤已经发暗,关节上皮肤已经开始打堆。他想拥有一个又白又亮的修长的美少年的身躯。小史以为,他这是变态,但他自己不以为是变态。这样的身躯在男性和女性都是一样的,都可以称之为美。 

十九 

那天晚上在派出所里,阿兰还谈到公园里有一个易装癖。这个人穿着黑裙子,戴一个黑墨镜,看起来很像一个女人,假如不看他手背上的青筋,谁也看不出他竟是一个男人。这个人就在公园里走来走去,谁也不理。他也许只想展示自己。也许别人不容易注意到他是个男人,但同性恋者马上就看出来了。阿兰对他很是同情,曾经想和他攀谈一下,但是被他拒绝了。这是因为他拒绝承认自己是男人,哪怕是承认自己是一个同性恋者。这使阿兰感到,他的绝望比自己还要深。 

这个人的事小警察也知道,他拉开抽屉,里面有此人的全套作案工具。这件事是这样发生的:此人身上的曲线是布条绕出来的,除此之外,他也要上厕所。有一天,他在女厕所里解布条子,被一位女士看见。可以想见,后者发出了一阵尖叫,这个家伙就被逮住了。在派出所里,小史自告奋勇地给他解开了布条,并且兴高采烈地告诉他,你丫长痱子了。他们就这样缴获了此人的头套,连衣裙,还有很多沁满了汗水的纱布,足够缠好几个木乃伊。小史谈起这件事,依然是兴高采烈,但这使阿兰感到一点伤感,因为那一天他也在派出所外面,看到此人穿了几件破衣烂衫狼狈地离去,在涂了眼晕的眼睛里,流出了两溜黑色的泪水。这件事有顺埋成章的一面,因为此人是如此的贱,如此的绝望,理应受到羞辱;但也有残忍的一面,因为这种羞辱是如此的肮脏,如此的世俗。就连杀人犯都能得到一个公判大会,一个执行的仪式。羞辱和嘲弄不是一回事。这就是说,卑贱的人也想得到尊重。 

无须说,小史听到这些话大大地吃了一惊,他没有想到这些贱人也想要得到尊重,就有哭笑不得之感。因为听到了这么多闻所未闻的事,不管怎么说,阿兰好像很有学问,虽然是肮脏的学问。他也想要尊重阿兰,很客气地和阿兰重新认识,互相介绍,并且把他叫做阿兰老师。虽然这样做时不无调侃之意,但是阿兰也接受了,这是因为被叫做老师,和这种受凌辱、受摧残的气氛并不矛盾。 

二十 

在那本书里,阿兰写道:那位衙役用锁链把白衣女贼牵到自己家里,把她锁在房子中间的柱子上。这样,他就犯了重大的贪污罪。在这个地方,美丽的女犯是一种公共财产,必须放在光天化日之下凌辱、摧残,一直到死。他把她带回家里来,就是犯了贪污罪。 

而那一夜实际发生的事情是:午夜过后下了一场暴雨,空气因而变得凉爽。小史因而感到瞌睡,他打个呵欠说,可以睡一会了。他自己准备在办公桌上睡觉,至于阿兰,可以在墙边的椅子上歪一歪。有一件事使他犹豫再三,后来他下了决心,拿出一副手铐来,说道:阿兰老师,不好意思,这是规定。他不但是这样说,而且是真的感到不好意思。但是阿兰很平静地把右手递给了他,等到阿兰再把左手递过来时,他说:不是这样。转过身来。他把阿兰反铐起来,又扶他坐下。他铐起阿兰时,有点内疚,所以多少有点温文的表示——问他热不热,给他翻开了领子。然后他回到办公桌后坐下,看到阿兰的脸是赤红色的,带着期待的神情,没有一点想睡的意思。这就使他想要睡觉也不可能。 

二十一 

小史和阿兰对视,感到十分的尴尬,因为他很少单独面对一个被自己铐起来的人——他只是个顽劣少年,涉世不深。这个人他还称他为老师。此人承认自己贱,但这使他感到更加不好意思。他觉得这件事是不妥当的,但也不能把手铐给阿兰摘下来——如果摘下手铐,说明他了解到、并且害怕阿兰的受虐倾向——在这种情况下,最好的办法就是装傻。 

阿兰正在讲自己的一次恋情,这人很少到公园里来,来的时候穿一件风衣,戴着墨镜,站在公园的角落望……他是一位画家,自己住在一套公寓里,家里陈设简单,故而显得空旷。他喜欢干的事情之一,就是在家里摆上一只矮几,在几上铺上蜡染布(或者白布),摆上一两件瓷盘。瓷瓶,插上花或者摆上几个果实,然后把用皮索反绑着的阿兰推到几上伏下,干他或者用笔在他身上做画。在后一种情况下,他还要从身后给阿兰照像。更多的时候是先画完再干。阿兰觉得快门的声音冷酷而凛冽,渐渐他开始把相机和性器等量齐观。他对小史说,现在,有时他见到黑色的相机,就有下身发热的情形……他喜欢相机那种黑色无光的浑圆外形,还喜欢一切这样外形的东西。直到有一天,阿兰到画家家里去,叫了半天的门门才开开,然后又在屋里发现了女人。画家说,你晚上再来吧。当然,阿兰再也没有去过。但是他也不很恨他。他对这件事只有一句话的说明:“这件事结束了。”以后,在公园里再见到这位画家,阿兰就远远地打个招呼,或者只是远远地看着他。这就是说,他觉得自己已经被使用过了。这叫小史大为诧异,一再问他是什么意思,然后对他下了一个结论道:你丫真贱。这又使阿兰低下头去。后来他又抬起头来,说道:贱这个字眼,在英文里就是easy。他就是这样的,招之即来,挥之即去。他为自己是如此的easy感到幸福。这使小史膛目结舌,找不到话来批判他。 

二十二 

小史细心地用小指在书页上画了一道,取过一个小书签把它夹在书里。他合上那本书,让时光在那里停住。让他困惑的是:到此为止,他并没有爱上阿兰,也看不出有任何要爱他的迹象;而那一夜已经过去大半了。 

阿兰在单位里也很贱。我们说他是个作家,这就是说,他原来在一个文化馆里工作,有时写点小稿子之类的。因为他的同性恋早就暴露了,所以他早就受到这样的对待。他每天很早就到那个文化馆里去,拖地板,打开水,刷洗厕所,以这种方式寻找自己的地位,我们可以说,是寻找最贱的地位。但他找不到自己的地位。因为“贱”就是没有地位。 

阿兰还说,每次他走到外面去,也就是说,穿上了四个兜的灰色制服,提了人造革的皮包,到文化馆去上班;或者融入自行车的洪流;或者是坐在大家中间,半闭着眼睛开会时;就觉得浑浑噩噩,走头无路,因为这是掩饰自己的贱。每次上班之后,他都不能掩饰这种冲动,要到画家家里去,在那里被捆绑,被涂、被画、被使用。这种时候,他觉得自己的形象和所做的事才符合事实,也就是说,符合他与生俱来的品行。他说:因为穿这样的衣服、提这样的包。开这样的会的人有千千万万,这怎么可能不贱呢。 

二十三 

对于阿兰来说,最大的不幸就在于,他真的很爱公共汽车。也许我们该说他是个双性恋。公共汽车现在是他老婆,他们俩住在阿兰小时候住的那间房子里。这种现状使他处于矛盾之中,因为想爱和想被爱是矛盾的。每天他回到家里时,都会看到她衣帽整齐地站在他面前,很有礼貌他说:您回来了。在家里,公共汽车总是穿着出门的衣服:筒裙套装,长筒丝袜,化着妆。甚至坐在椅子上时,上身都挺得笔直,姿仪万方。阿兰非常无端地朝她逼过去,抓住肩头,把她往床上推。这时公共汽车会放低了声音说:能不能让我把门关上?阿兰把她推倒在床上,解开她的扣子,松掉她的乳罩,把它推上去——此时公共汽车看上去像一条被开了膛的鱼。阿兰爱抚她,和她做爱时,公共汽车用小拇指的指甲划着壁纸,若有所思。直到这件事做完,她才放下手来,问阿兰:感觉好吗?好像在问一件一般的事。此时她的神情像个处女。公共汽车对阿兰总是温婉而文静,但只对阿兰是这样。 

等到阿兰离开公共汽车的身体,她已经乱糟糟的像个破烂摊。回顾做爱以前的模样,使人相信,她是供凌辱、供摧残。她悄悄地爬起来,把那些揉皱了的衣服脱掉,叠起来,然后穿上破烂衣服,仔细地卸了妆,出门去买菜。只有在要出门时,她才仔细地卸装,穿上破烂衣服。当她服饰整齐,盛装以待之时,就是在等待性爱;当她披头散发,蓬头垢面之时,就是拒绝性爱。这一点和别人截然相反。从这一点上来看,她就像那位把内衣穿在外面的玛多娜一样的奇特。 

二十四 

那天下午、阿兰被小警察逮去时,因为那个城市不大,所以这件事马上就传到他太太耳朵里了。阿兰的老婆(公共汽车)在市场上买菜,有人告诉她阿兰进去了,她说了一声:“该!”然后就问进到哪里去了。一般来说,进去就是进去了,但对于同性恋老来说,可以进到正宫,也可以进后宫,正宫并不严重。这位女士问清了情况,并不着急,她回到家里做家务事。尽量保持平静的心情。她还算年轻,但显得有点憔悴;还算漂亮,但正在变丑。此人的模样就是这样。 

天快黑的时候,阿兰的太太做了饭,自己吃了之后,还给阿兰留了一些,然后她就从家里出来,到楼下给女友打投币电话,所说的第一句话就是:阿兰这混球又进去了。我想,对方不知道阿兰是为什么进去的,但是知道阿兰是经常进去的,所以就把他想像成一个一般的流氓。对方问她准备怎么办,她说,要是他今晚上不回来,就让他在里面呆着,要是明天不回来,就到派出所去领他——还能怎么办。我们知道,假如一位同性恋者被扣了起来,太太来接,警察是乐于把该男士交出去的,这是因为他们以为,他在太太手里会更受罪。警察做的一切,都以让他们多受些罪为原则。对方想听到的并不是这句话,我们可以听到她在耳机里劝她甩掉阿兰,“干吗这么从一而终哪。”然而,阿兰的太太并不想讨论这些操作性的事,她只是痛哭流涕,并且说,她已经烦透了。后来,她擦掉了眼泪,对对方说,对不起,打搅你了,就挂下电话一回家去了。阿兰虽然没有看到这些,但是一切都在他的想像之中。 

二十五 

阿兰的书里写道:那位衙役把女贼关在一间青白色的房间里,这所房子是石块砌成的,墙壁刷得雪白,而靠墙的地面上铺着干草。这里有一种马厩的气氛,适合那些生来就贱的人所居。他把她带到墙边,让她坐下来,把她项上的锁链锁在墙上的铁环上,然后取来一副木扭。看到女贼惊恐的神色,他在她脚前俯下身来说,因为她的脚是美丽的,所以必须把它钉死在木扭里。于是,女贼把自己的脚腕放进了木头上半圆形的凹槽,让衙役用另一半盖上它,用钉子钉起来。她看着对方做这件事,心里快乐异常。 

后来,那位衙役又拿来了一副木枷,告诉她说,她的脖子和手也是美的,必须把它们钉起来。于是女贼的项上就多了一副木伽。然后,那位衙役就把铁链从她脖子上取了下来,走出门去,用这副铁链把木栅栏门锁上了。等到他走了以后,这个女贼长时间地打量这所石头房子——她站了起来,像一副张开的圆规一样在室内走动。走到门口,看到外面是一个粉红色的房间。 

晚上阿兰太太一个人在家,她早早地睡了。她辗转反侧,不能入睡,后来就和自己做爱。这件事做完以后,她又开始啜泣。此种情况说明,她依然爱阿兰,对阿兰所做的事情不能无动于衷。但是在阿兰的书里,没有一个地方可以让人想到阿兰的太太。他不愿意让公共汽车知道,他是爱她的。 

午夜时分,外面下了一场大雨,公共汽车起来关窗户,她穿了一件白色的针织汗衫,这间房子是青白色的。阿兰后来住的房子也是这样。她把窗户关好,就躺下来睡了。公共汽车睡着时,把两手放在胸上,好像死了一样。 

那天晚上下雨时,小史的太太点子在酣睡。他们的房子是粉红色的,亮着的台灯有一个粉红色的罩子。点子穿着大红色的内衣,对准双人床上小史的空位,做出一个张牙舞爪的姿势。 

二十六 

小史也承认,每当他看到国营商店里或者合资饭店里的漂亮小姐对同胞的傲慢之态,就想把她们抓起来,让她们蹲在派出所的大墙底下。他还说,有时候大墙下面会蹲了一些野鸡(另一个说法叫做卖淫人员),那些女孩子蹲在那里会有一种特殊困难,因为她们往往穿了很窄的裙子。在这种情况下,她们只好把大腿紧并在一起,把双手按在上面,因而姿仪万方。他认为,这个样子比坐得笔直好看。当她们被戴上手铐押走时,会把头发披散下来,遮住半边脸。这个样子也比那些小姐拨开头发,板着脸要好看。所以,在小史心目中,性对象最好看最性感的样子也是:供羞辱、供摧残。于是,他和阿兰就有了共同之点。但也有不同之点:他属于羞辱的那一面,阿兰属于被羞辱那一面。他属于摧残,阿兰属于被摧残。明白这些,使小史感到窘迫——此时,到了应该划清界限的时候了。 

二十七 

小史往窗外看,东边天上微微露出了白色。这使他感到松懈,就伸了个懒腰道:谢天谢地,这一夜总算是完了。他还说,从来值夜班没有这么累过。而阿兰却有了一种紧迫感。小史呵欠连天,拿了钥匙走到阿兰面前,说道:转过身来,我下班了。阿兰迟疑不动时,小史说:你喜欢带这个东西,自己买一个去,这个是公物。阿兰侧过身来,当小史懒懒散散地给他开铐时,阿兰在他耳边低声说道:我爱你。这使小史发了一会愣。他听见了,不敢相信;或者自以为没听清。反正他也不想再打听。他直起腰来,说道:我看还是铐着你的好;然后走开了。但是小史面上绯红,这已经是无法掩饰的了。 

二十八 

阿兰对小史说,他温婉、善解人意。他从内心感觉到自己是个女人,甚至不仅于此。来到一个英俊性感的男子面前,他就感到柔情似水。就像那种长途跋涉之后,忽然出现在面前的一泓清凉的水。他也可以很美丽,因为美丽不仅是女性所专有。他特别提到了那位画家把他放倒在短几上时,那房间满是镜子。从镜子望看到了自己的后半身:紧凑的双腿,窄窄的臀部,还有从两腿之间看到的部分阴囊。他认为,说只有女性才美丽,这是一个绝大的错误。最大的美丽就是:活在世界上,供羞辱,供摧残。 

在阿兰的书里,这一段是这样的:那个女贼跪在那个粉红色的房间里,一伸一屈地在擦地板。她颈上的长枷已经卸去了,手上戴着手扭,双足分得很开钉在木头里,在她身前,有一个盛水的小木桶,她手里拿着板刷。她像尺蠖一样,向前一伸一屈。那个衙役坐在一边看着,后来,他站起身来,走到女贼的背后,撩起她的白衣,从后面使用她……而她继续在擦地板。 

阿兰说到这些话时,非常的女气,而且柔媚。这使小史感到毛骨悚然。但是阿兰讲这番话时反背着手,跷着腿,就如一位淑女,这样子又有些诱人之处。所以他皱着眉头说道:你丫到底是男的还是女的。阿兰说:这不重要。当你想爱的时候,你就是男的,当你想要承受爱的时候,你就是女的。没有比这更不重要的事情了。 

二十九 

阿兰举出和那位不知名的小学教师的爱情作为例证。如前所述,那天夜里,在乡下的黄泥巴房子里,小学教师说道:你对我做什么都成之后,阿兰就热吻他,请他平躺在床上,吻他的胸口,肘窝,颈下;爱抚他,使他平静;在不知不觉之中,把做爱的主动权归还给他了。他自己说,那天晚上,开头的时候他想要爱,但忽然感到柔情似水,就转为承受了。你既可以爱,又可以被爱,这是世界上最美好的事情。 

三十 

在阿兰的故事里,那个女贼擦过了地板之后,手里拿着一个盛着香草的小篮子。她继续像尺蠖一样一伸一屈,仔细地把香草洒匀,她专注于此,除此之外,好像什么都不关心。与此同时,邵个衙役坐在那里监视她。阿兰暗自想到,这种监视是很重要的。假如没有这种监视,一切劳作都是没有意义的了。 

而阿兰自己(此时他坐在床垫上)回想到的事和小史想到的大相径庭。那天晚上,他对小史说,他既可以爱,又可以承受爱,就温柔地低下头去说:我爱你。这就是说,他准备被小史羞辱、摧残。于是小史就把他拖了出去,放在自来水管子底下冲了一顿,然后,又把他拖了回来,放在凳子上,抽了一顿嘴巴。此时阿兰依然是被反铐着双手,心里快乐异常。等到这一切都过去之后,小史忽然惊慌地愣住了。这时,阿兰趁机去吻他的手心,并且说:美丽是招之即来的东西。这时,小史打开了他的手铐。阿兰还把自己扮成女人的相片拿给小史看,从照片上,完全看不出是阿兰。它认表面上看,只是一幅裸体女人的相片,假如你知道它的底蕴,就会更加体会到一种邪恶的美丽。小史就这样被他的邪恶所征服——因为这些原故,阿兰才觉得那一夜分外的值得珍视。 

在阿兰的书里,女贼做好了应该做的一切,就回到了她自己的房间门口。当然,也许应该叫作她的牢房门口,跪坐在地下,把手扭伸给衙役,等待卸下手扭,换上长枷。她全心全意地专注于此事,仿佛除此之外,再没有值得重视的事了。 

三十一 

阿兰在他的书里写道:有时候,那个衙役也把那个女贼的枷锁卸掉,从那间青白色的房子里带出来,带到粉红色的房子里,锁在一张化妆台上,然后就离去了。这时候,这个女贼就给自己化妆,仔细地描眉画目,让自己更美丽——也就是说,看起来更贱一点。 

阿兰在派出所里对小警察说,在那位画家那里,他曾经多次化妆成一个女人,作为裸体模特儿,被画入油画,或者被摄入照片。他说,只要你渴望被爱,美丽是招之即来的。对他来说,做模特儿,就是被爱。除此之外,每次画家画毕,都要和他做爱。画家说,如果不做爱,作品就不完全。对画家来说,爱情是一种艺术。而阿兰却说,艺术是一种爱情。小史就记住了这句话。他抚摸着阿兰的书,觉得这本书就是爱情。他取出一张相片夹到书里,而这张相片上就是女装的阿兰。 

后来,小警察拉开了抽屉,就离开了这间屋子。在那个抽屉里放着那位易装癖的全部行头,有衣裙,缠身体的布条,头套,还有他的化妆品。阿兰坐在案前,开始把自己化妆成一个女人。他像在做画一样画着自己的脸,这是艺术,用他自己的话来说,艺术就是一种爱情。而爱情就是——供羞辱,供摧残。小警察回到派出所的门前,隔着门上的玻璃,看到自己的案前坐了一位绝代佳人。他被这种美丽所震撼,好久都没有推门进去。 

三十二 

阿兰所化妆的女人穿着黑色的连衣裙。这种颜色阿兰也喜欢。等到小警察终于走进办公室里来的时候,阿兰站了起来,顾盼生姿、雍容华贵地走到他面前,稍微躬身收拾了一下裙角,就从容地跪下了。他拉开了小警察的拉锁,同时还用舌头抿了一下自己的嘴唇……小史俯身看到的景象,使他难以相信。他把自己的手臂举在半空,好像一位外科医生在手术室里……终于,他把手放下去,按住阿兰的头。与此同时,抬头向天,欲仙欲死。 

此时,阿兰坐在床垫上,抿着嘴唇,撩开了毛巾被,把手伸了进去……他同样的欲仙欲死。这仅仅是因为小史曾经欲仙欲死,面他则回味了这件事。在每次爱情里做的一切,都有可供回味的意义。 

三十三 

早上,光亮首先来到那间青白色的房子里。那个女贼坐在铺草上,项上套着长枷,足上上着木扭。好像这一夜什么都没有发生一样。但是她头发凌乱,脸上还带有残妆。在阿兰家里那个青白色的房间里,当曙光出现时,公共汽车也起床了。她着意打扮,穿上了最好的衣服,就在桌前坐下,双手放在桌子上,前面是一个闹钟。她在等时光过去,好去接阿兰。 

那天早上,阿兰的太太去接他,因为是绝早,所以整个城市像是死了一样。她在街上看到阿兰迎面走来,神色疲惫,脸上有黑色的污渍。看到他以后,她就在街上站住,等他走过来。等到阿兰走到了身边,她转过身去,和他并肩走去。对于这一夜发生了什么,她没有问。后来阿兰伸手给她,她就握住他的手腕——就如在夜里握住他的性器官。能握住的东西是一种实实在在的保证,一松手,就会失去了。阿兰的太太什么都不会问,只是会在没人的地方流上一两滴眼泪,等到重新出现时,又是那么温婉顺从。但是这些对阿兰一点用都没有,阿兰是个男人,这一点并不重要,在骨干里,也是和她一样的人。从某种意义上说,他们之间的事,才是真正的同性恋。 

那天夜里,阿兰曾经扮作一个女人,这一点从他脸上的残妆可以看出来。但是公共汽车没有问,回到家里之后,她只是从暖瓶里给他倒水,让他洗去脸上的污渍;然后问阿兰:吃不吃饭。阿兰说,要吃一点。但是他吃的不止一点,他很饿。然后,公共汽车说:你睡一会吧,我去买菜。但就在这时,阿兰拉住了她的手。这是一种表示。公共汽车禁不住叫了起来:“你干吗?你要干吗?”带一点惊恐之急。阿兰虽然低着头,但可以看到他的表情,他虽然羞愧,但也有点没皮没脸。一言以蔽之,阿兰像个儿奸母的小坏蛋。看清了这一点之后,公共汽车就叹了一口气,说道:好吧。她走到床边去,面朝着墙,开始脱衣服。后来,她在床上,身上盖着被单,用手背遮着眼睛。阿兰走过来,撩起了被单,开始猛烈地干她。对于这件事,我们可以解释说,在这一夜里,阿兰并没有发泄过,他只是被发泄,当然,这是只就体液而言。在阿兰势如奔马的时候,公共汽车哭了,并且一再说:你不爱我。但是等阿兰干完了时,公共汽车也哭完了,伸手拿了手绢来擦脸,表情平静。这时阿兰在她身边躺下,说道:我是想要爱你的。至于公共汽车对此满不满意,我们就不知道了。 

三十四 

光亮来到那间粉红色的房子里时,那个衙役在酣睡,他赤身裸体,在铺上睡成个大字形……点子也在熟睡。她的样子和衙役大不相同——她在双人床上睡成了一条斜道,并且把脸淹没在了枕头里。 

与此同时,小史走到了窗前,从窗子里往外看。在他面前的是空无一人的公园,阿兰早就消失在晨雾了。他觉得,阿兰把选择权交到他手里了。他可以回味这一夜,也可不回味;他可以招阿兰回来,也可以不这样做。这件事的意义就在于,使他明白了自己也是个同性恋者。 

三十五 

小史和阿兰在一起时,还是觉得他贱,甚至在做爱完毕时,也是这样。他们总是在防空洞一类的地方干这种事,那里有个烂垫子,点着蜡烛。那件事干完了之后,他总是有意无意他说上一句:你丫真贱。而阿兰则总是不接这个茬,只是说:抱抱你,可以吗?于是,小史懒洋洋地翻过身去,把脊背对着他,恩赐式他说:抱吧。这件事说明,当时小史并没有爱上阿兰,爱上他是以后的事了。 

小史又打开了那本书。那个故事是这么结束的:有一天,那个女贼早上醒来的时候,走到那木栅门前往外看,那间粉红色的房间里空无一人,连那条锁住门的铁链都不见了。她用木枷的顶端去触那扇门,门就开了。然后,她就走进了那个粉红色的房子里,缓缓地绕过绢制的屏风,后面是那张床一床上空无一人,只剩下了粗糙的木板。东歪西倒的家具似乎说明,主人再也不会回来了。她缓慢地移到了门口,用长枷的棱角拨开了门,不胜惊讶地发现,这座房子居然是在一个果园望。此时正值阳春三月,满园都是茂盛的花朵。 

后来,阿兰离开了本市,迁到别处去了。当时,小史到车站去送他。在火车站上出现了令人发窘的场面,在这两个女人的监视下,两个男人都不尴不尬。小警察管公共汽车叫嫂子,面红耳赤。而公共汽车的目光有如寒冰,但等她看到点子的时候,目光就温暖了。这一对女人马上就走到了一起,而小警察和阿兰走到了一起,其状有如两对同性恋在交谈。但是,小史和阿兰实质上是在女人的押解之下。 

在火车就要开走时,小史感到了一种无名的冲动,他开始从骨头里往外爱阿兰。在两个女人的注视下,他总禁不住伸出手来,要触摸他。在这时做这样的事,显然是不可以的。越是不可以的事,越想要去做,这种事情人人都遇到过吧——他就是在这时爱上了阿兰。这就是说,他不但承认了自己也是个同性恋者,并且承认了自己和阿兰一样的贱。 

三十六 

阿兰现在生活在一个灯红酒绿的地方,从他住的房间往下看,就是一条大街。他在房间里走动时,在腰上缠上了白色的布,看上去像个甘地。这个甘地和真甘地不同的地方,在于他的嘴唇,湿润而艳丽,好像用了化妆品。在他床头的矮柜上,放了一个镜框,里面有小史的相片。时至今日,他还像小史爱他一样地爱着他。不过,如今他一看到这张相片,就想到小史是如何的风风火火,尤其是在做爱之前。你必须告诉他:把上衣脱了吧,他才会想起要脱上衣;你还要说:把手表摘了吧,划人,他才会摘掉手表。这种时候,小史是个对眼。这种脸相,大概连他太太都没有见过。现在他对着小史的相片,想到这些事情,可以发出会心的微笑,但是在当时却不能——因为他正忙于承受小史的爱。所以,阿兰以为,爱情最美好之处,是它可以永远回味。现在他在回味这些的时候,并不觉得自己是贱的。 

晚上,阿兰坐在床垫上,听到了门外的脚步声,又听到钥匙在门里转动。他赶紧把小史的照片收藏起来,自己躺到床垫上闭上眼睛。然后,公共汽车走了进来。她踢掉了高跟鞋,走到卫生间里。然后,她穿着白色的睡袍走了出来,在阿兰的身边悄悄地躺了下来,用手背和手指拂动他们之间的被单,仿佛要划定一个无形的界限。她还是那么温文、顺从,但是谁也不知道,她还是不是继续爱着阿兰。因此,这间房子像一座古墓一样了。 

三十七 

后来,那个女贼又回到了衙役当初捕获她的地方——高高的宫墙下,披挂着她的全部枷锁,在那里徘徊,注意看每个行人。而小警察也在公园里徘徊着,有时走近成帮打伙的同性恋者。但是,他没有勇气和他们攀谈。在他心目里,阿兰仍是不可替代的。在我们的社会里,同性恋者就如大海里的冰山,有时遇上,有时分手,完全不能自主。从这个意义上看,小史只是个刚刚开始漂流的冰山。生为冰山,就该淡淡地爱海流、爱风,并且在偶然接触时,全心全意地爱另一块冰山。但是这些小史还不能适应。 

小史合上了阿兰写的书。 

小史开始体验自己的贱:他环顾这间黑洞洞的屋子。白天,在这间房子里,没有一个人肯和他面对面他说话。处此之外,喝水的杯子最能说明问题。派出所里有一大批瓷杯子,本来是大家随便拿着喝的,现在他喝水的杯子被人挑了出来。假如有人发善心给大家去刷杯子的话,他用的杯子必然会被单独自挑出来;而假如是他发善心去刷杯子的话,那些杯子必然会被别人另刷一遍。这些情况提醒他,他已经是这间房子里最贱的人了。 

三十八 

天已经很晚了,另一个警察从外面进来,说:还没走啊。小史告诉他说,他值夜班。对方则说:所长说了,以后不让你值夜班了。小史说:为什么?对方说:你别问为什么了。不值夜班还不好吗。说着用椅子开始拼一张床。小史说:干吗不让我值夜班哪。对方说:你老婆和所长说的(这就是说,告诉单位了)。他还说:两口子在一个派出所多好啊,女的不值夜班,男的也不值夜班。说话之间,床已经搭到半成。那个警察走到小史面前说:劳你驾,把椅子给我用用。说着把他臀下的椅子也抽走了。小史立着说道:怎么也不跟我说一声。那个警察答道:不知道。少顷又说:还用和你说吗。后来他(这位警察因为值了额外的夜班,有点不快)说:别不落忍。反正你就要调走了。同事一场,替你值几宿也没啥。小史听了又是一惊说:我去哪儿?那个警察说:不知道。反正这公园派出所对你不适合。听说想派你去劳改农场,让你管男队,你老婆不答应,可也不能让你去管女队啊。算了,不瞎扯。我什么都不知道。从这些话里,我们知道了同性恋者为什么不堪信任:既不能把他们当男人来用,又不能把他们当女人来用——或者,既不能用他们管男人,也不能用他们管女人。 

小史把阿兰的书锁进了抽屉,走了出去,走到公园门口站住了。他不知道该到哪里去。他不想回家,但是不回家也没处可去。眼前是茫茫的黑夜。曾经笼罩住阿兰的绝望,也笼罩到了他的身上。
