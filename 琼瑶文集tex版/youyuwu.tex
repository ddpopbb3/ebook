\chapter{有与无}

 
我靠写作为生。有人对我说:像你这样写作是不行的啊,你没有生活!我虽然长相一般,加上烟抽得多,觉睡得少,脸色也不大好看。但若说我已是个死尸,总觉得有点言过其实。人既然没死,怎么就没生活了呢?笔者过着知识分子的生活,如果说这种生活就叫做“没有”,则带有过时的意识形态气味——要知道,现在知识分子也有幸成为劳动人民之一。当然,我也可以不这样咬文嚼字,这样就可以泛泛地谈到什么样的生活叫做“有”,什么样的生活叫做“无”;换句话说,哪种生活是生活,哪种生活不是生活。众所周知,有些作家要跑到边远、偏僻的地方去“体验生活”——这话从字面上看,好像是说有些死人经常诈尸——我老婆也做过这样的事,因为她是社会学家,所以不叫体验生活或诈尸,而是叫实地调查——field  work。她当然有充分的理由做这事,我却没有。


有一次,我老婆到一个南方小山村调查,因为村子不大,每个人都在别人眼皮地下生活。随便哪个人,都能把全村每个人数个遍,别人的家庭关系如何、经济状况如何,无不在别人的视野之中,岁数大的人还能记得你几岁出的麻疹。每个人都在数落别人,每个人也都受数落,这种现象形成了一条非常粗的纽带,把所有人捆在一起。婚嫁丧娶,无不要看别人的脸色,个人不可能做出自己的决定。她去调查时,当地人正给自己修坟,无论老少、健康状况如何,每个人都在修。把附近的山头都修满了椅子坟。因为这种坟异常的难看,当地的景色也异常的难看,好像一颗癞痢头。但当地人陷在这个套里,也就丧失了审美观。村里人觉得她不错,就劝她也修一座——当然要她出些钱。但她没修,堂堂一个社会学家,下去一个月,就在村里修了一座椅子坟,这会是个大丑闻。这个村里的“文化”,或者叫“规范”,是有些特异性的。从总体来说,可以说存在一种集体的“生活”。但若说到属于个人的生活,可以说是没有的。这是因为村里每个成年人惦记的都是一模一样的事情:给自己修座椅子坟就是其中比较有趣的一件。至于为什么要这样生活,他们也说不出来。


笔者曾在社会学研究所工作,知道有种东西叫“norm”,可以译作“规范”,是指那些约定俗成,大家必须遵守的东西。它在不同的地方时不一样的,当然能起一些好作用,但有时也相当丑恶。人应该遵从所在社会的“norm”,这是不言而喻的。但除了遵从norm,还该不该干点别的,这就是问题。如果一个社会的norm很坏,就如纳粹德国或者“文革”初的中国,人在其中循规蹈矩地过了一世,谁都知道不可取。但也存在了这样的可能,就是经过某些人的努力,建立了无懈可击的norm,人是不是只剩遵从一件事可干了呢?假如回答是肯定的,就难免让我联想到笼养的鸡和圈养的猪。我想如何一个农场主都会觉得自己猪场里的norm对猪来说是最好的——每只猪除了吃什么都不做,把自己养肥。这种最好的norm当然也括这些不幸的动物必须在屠场里结束生命……但我猜想有些猪会觉得自己活得很没劲。


我老婆又在城里做一项研究,调查妇女的感情与性。有些女同志除了自己曾遵守norm就说不出什么,仿佛自己的婚姻是一片虚无,但也有些妇女完全不是这样,她们有自己的故事——爱情中的每个事件,在这些故事里都有特别的意义。这主要是因为,这些姐妹有属于自己的生活和属于自己的价值观。“到了岁数,找合适的对象结婚,过正常的性生活”和“爱上某人”,是截然不同的事情。当然,假如你说,性爱只是生活的一隅,不是全体,我无条件地同意。但我还想指出,到岁数了,找合适的人,过正常的生活,这些都是从norm的角度来判断的——属于个人的,只是一片虚无。我总觉得,把不是生活的事叫成“生活”,这是巧言掩饰。


现在可以说到我自己。我从小就想写小说,最后在将近四十岁时,终于开始写作——我做这件事,纯粹是因为,这是我爱的事业。是我要做,不是我必须做——这是一种本质的区别。我个人认为,做爱做的事才叫“有”,做自己不知道为什么要做的事则是“无”。因为这个缘故,我的生活看似平淡,但也不能说是“无”。有一种说法是这样的:人在年轻时,心气总是很高的,最后总要向现实投降。我刚过了四十四的生日,在这个年龄上给自己做结论似乎还为时过早。但我总觉得,我这一生绝不会向虚无投降。我会一直战斗到死。



