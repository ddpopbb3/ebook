\chapter{2015}

1 

从很小时开始,我就想当艺术家。艺术家穿着灯芯绒的外套,留着长头发,蹲在派出所的墙下──李家口派出所里有一堵磨砖对缝的墙,颜色灰暗;我小舅经常蹲在这堵墙下,鼓起了双腮。有些时候,他身上穿的灯芯绒外套也会鼓起来,就如渡黄河的羊皮筏子,此时他比平时要胖。这件事留给我一个印象,艺术家是一些口袋似的东西。他和口袋的区别是:口袋绊脚,你要用手把它挪开;艺术家绊脚时,你踢他一下,他就自己挪开了。在我记忆之中,一个灰而透亮的垂直平面(这是那堵墙的样子)之下放了一个黄色(这是灯芯绒的颜色)的球,这就是小舅了。 

在派出所里能见到小舅。派出所是一个灰砖白墙的院子,门口有一盏红灯,天黑以后才点亮。那里的人一见到我就喊:“啊!大画家的外甥来了!”有种到了家的气氛。正午时分,警察在门边的小房间里煮切面,面汤的气味使人倍感亲切。附近的一座大地咖啡馆里也能见到小舅,里面总是黑咚咚的,不点电灯,却点腊烛,所以充满了呛人的石腊味。在咖啡馆里看人,只能看到脸的下半截,而且这些脸都是红扑扑的,像些烤乳猪。他常在那里和人交易,也常在那里被人逮住,罪名是无照卖画。小舅常犯这种错误,因为他是个画家,却没有画家应有的证件。被逮住以后,就需要人领了。 

派出所周围有一大片商店,是上世纪五十年代建造的大顶子瓦房。人行道上还有两行小银杏树,有人在树下生火烤羊肉串,烤得树叶焦黄,景色总像是秋天;后来那些树就死掉了。他住的地方离那里不远,在一座高层建筑里有一间一套的房子──那座楼房方头方脑,甚是难看,楼道里也很脏。不管你什么时候去找──我舅舅总不在家,但他不一定真的不在家。 

我舅舅是个无照画家,和别人不同的是,他总在忙些正事。有时他在作画;有时他卖画,并且因此蹲在派出所里。他作画时把房门锁上,再戴上个防震耳罩,别人来敲门听不见,打电话也不接,独自一人面对画架,如痴如狂。因为他住在十四层楼上,谁也不能趴窗户往里看,所以没人见过他作画,除了一个贼。这个贼从十三楼的阳台爬上来,打算偷点东西,进了我舅舅的客厅,看到他的画大吃一惊,走过来碰碰他说:哥们儿,你丫这是干嘛呢?我舅舅正画得入迷,呜呜地叫着说:别讨厌!老子在画画!那个贼走到一边蹲下看了一会儿,又忍不住走过来,揭掉小舅左边的耳罩说:喂!画可不是这种画法!我舅舅狠狠地搡了他一把,把他推倒在地,继续作画。那人在地上蹲了很久,想和我舅舅谈谈怎样作画的问题,但始终不得机会,就打开大门走掉了,带走了我舅舅的录相机和几千块钱,却留下了一张条子,郑重告诫我舅舅说:再这样画下去是要犯错误的,他自己虽然偷东西,却不忍见到小舅误入歧途。作为一个善良的贼,他对失主的道德修养一直很关心。我舅舅说,这条子写得很煽情──他的意思是说,这条子让他感动了。 

后来有一天,我舅舅在派出所里遇上了那个偷他东西的贼:他们俩并排蹲在墙下。据我舅舅说,那个贼穿了一双灯芯绒懒汉鞋,鞋上布满了小窟窿。此君的另一个特徵是有一头乱蓬蓬的头发,上面全是碎木屑。原来他是一个工地上的民工,有时做木工的活,这时候头发上进了木屑;有时候做焊工的活,这时脚上的鞋被火花烫出了很多洞;有时候做贼,这时候被逮住进了派出所。我舅舅看他面熟,但已不记得他是谁。 

那个贼很亲热地打起了招呼:哥们儿,你也进来了?我舅舅发起愣来,以为是个美术界的同行,就含混地乱答应着。后来贼提醒他道:不记得了?上回我到你家偷东西?我舅舅才想了起来:啊!原来是你!Good morning!两人很亲切地聊了起来,但越聊越不亲切,最后打了起来;原因是那个贼说我舅舅满脑子都是带颜色的豆腐渣。假如不是警察敲了我舅舅的后脑勺,小舅能把那个贼掐死;因为他还敢说我舅舅眼睛有毛病。实际上我舅舅眼睛是有外斜视的毛病,所以老羞成怒了。警察对贼在艺术上的见解很赞成,假如不是他屡次溜门撬锁,就要把他从宽释放。后来,他们用我舅舅兜里的钱给贼买了一份冰激凌,让他坐在椅子上吃;让我舅舅蹲在地下看。当时天很热,我舅舅看着贼吃冷食,馋得很。 

我常上派出所去领小舅,也常在派出所碰上那个贼。此人是唐山一带的农民,在京打工已经十年了。他是个很好的木工、管子工、瓦匠,假如不偷东西,还是个很好的人。据说他溜进每套房子,都要把全屋收拾乾净,把漏水的龙头修好,把厨房里的油泥擦乾净,把垃圾倒掉;然后才翻箱倒柜。偷到的钱多,他会给检查机关写检举信,揭发失主有贪污的嫌疑,偷到的钱少,他给失主单位写表扬信,表扬此人廉洁奉公。 

他还备有大量的格言、人生哲理,偷一家、送一家。假如这家有录相带,他都要看一看,见到淫秽的就带走,以免屋主受毒害。有些人家录相带太多,他都要一一看过,结果屋主人回家来把他逮住了。从派出所到居委会,都认为他是个好贼,舍不得送他进监狱,只可惜他偷得太多,最后只好把他枪毙掉,这使派出所的警察和居委会的老大妈一齐掉眼泪。这个贼临死还留下遗嘱,把尸体捐给医院了。我有个同学考上了医科大学,常在福尔马林槽里看到他。他说,那位贼兄的家伙特别大,躺在水槽里仪表堂堂,丝毫也看不出是个贼,虽然后脑勺上挨了一枪,但不翻身也看不出来。每回上解剖课,女生都要为争他而打架。 

我舅舅犯的只是轻罪,但特别的招人恨。这是因为他的画谁也看不懂,五彩缤纷,谁也不知画了些什么。有一次我看到一位警察大叔手拿着他的画,对他厉声喝斥道:小子──站起来说话──这是什么?你要是能告诉我,我替你蹲着!我舅舅侧过头来看看自己的作品,又蹲下去说:我也不知这是什么,我还是自己蹲着好了。在我看来,他画了一个大旋涡,又像个松鼠尾巴。当然,哪只松鼠长出了这样的尾巴,也实属可恨。我舅舅原来是有执照的,就是因为画这样的画被吊销了。在吊销他执照之前,有关部门想做到仁至义尽,打出了一个名单,上面写着:作品1号,“海马”;作品2号,“袋鼠”;作品三号,“田螺”;等等。所谓作品,就是小舅的作品。引号里是上级给这些画起的名字。冠之以这些名目,这些画就可懂。当然,那些海马、袋鼠和田螺全都很古怪,像是发了疯。只要他能同意这些名称,就可以不吊销他的执照。但小舅不肯同意,他说他没画海马和袋鼠。人家说:你不画海马、袋鼠也可以,但总得画点什么;我舅舅听了不吭气也罢了,他还和人家吵架,说人家是傻逼。所以他就被从画家队伍里开除掉了。 

如你所知,我的职业是写小说。有一次,我写了一个我大舅舅的故事,说他是个小说家、数学家,有种种奇遇;就给自己招来了麻烦。有人查了我家的户口存根,发现我只有一个舅舅。这个舅舅七岁上小学,十三岁上中学,美术学院油画系毕业,现在是无业游民。人家还查到他从小学到中学,数学最好成绩就是三分,如果他当了数学家,无疑是给我国数学界抹黑。为此领导上找我谈,交给我一个故事梗概,大意是:我舅舅出世时,是一对双胞胎。因为家贫难养,就把大的送给了别人。这个大的有数学才能,也能编会写,和小舅很不同,所以他和小舅是异卵双胞胎。有关这一点,梗概里还解释道,我过世的姥姥是山东莱西人,当地的水有特殊成份,喝了以后卵子特别多。就因为是莱西人,我姥姥像一条母黄花鱼。领导上的意思是让我按这个梗概把小说改写一下,但我不同意──我姥姥带过我,我和她感情极深。我还以为,作为小说家,我想有多少舅舅,就有多少舅舅,别人管不着。我因此犯了个错误,被吊销了执照──这件事已经写过,不再赘述了。 

我去领小舅的年代,我妈也在世。我舅舅有外斜视的毛病,双眼同时往两边看,但比胖头鱼的情况还要好一些。我妈的眼睛也是这样。照起镜子时,我妈觉得自己各方面都漂亮,只有这双眼睛例外,她抱怨自己受了小舅的拖累。因为她比小舅先生出来,以谁受谁拖累还不一定。她在学校里教书,所习专业和艺术隔得很远,但作为小舅的姐姐,我妈觉得自己应该对他多些理解,有一次说,把你的画拿来我们看看。小舅却说:算了吧,看了你也不懂。我妈最恨人说这世界上还有她不懂的事,就把盘子往桌子上一摔说:好,你请我看也不看了!你最好也小心一些,别出了事再让我去领你!小舅沉默了一会儿,从我家里走出去,以后再也不来。去派出所领小舅原是我妈的义务,以后她就拒绝履行。但是小舅还照样要出事,出了事以后放在派出所里,就如邮局里有我们的邮件,逾期不领要罚我们的钱。所以只好由我去了。 

从很小的时候我就渴望爱情。我的第一个爱人是小舅。直到现在,我还为此而难为情。我舅舅年轻时很有魅力,他头发乌油油的,又浓又密,身上的皮很薄──他很瘦,又很结实,皮肤有光泽;光着身子站着时,像一匹良种马,肩宽臀窄,生殖器虽大,但很紧凑──这最后一点我并不真知道。我是男的,而且不是同性恋。所以你该去问小舅妈。 

小时候我长得细胳臂细腿,膝盖可以往后弯,肘关节也可以往后弯;尖嘴猴腮,而且是包茎。这最后一点藏在内裤里面看不见。我把小舅从派出所里领了出来,天气很热,我们都出了一身臭汗。小舅站在马路边上截“面的”,要带我去游泳。这使我非常高兴;甚至浮想连翩。忽然之间,膝盖后面就挨了他一脚。小舅说:站直了!这说明我的膝盖正朝前弯去,所以我在矮下去。据说膝盖一弯,我会矮整整十公分。又过了一会儿,我又挨了小舅一脚。这说明我又矮下去了。我不明白自己矮点关他什么事,就瞪眼看着他。小舅恶狠狠地说道:你这个样子真是讨厌!我确实爱小舅。但是这个坏蛋对我不好,这很伤我的心。 

我舅舅外斜视,我觉得他眼中的世界就如一场宽银幕电影,这对他的事业想来是有好处的。从科学的角度来说,眼睛隔得远,就会有更好的立体感,并且能够更好地估计距离。二十世纪前期,激光和雷达都未发明,人们就用这个原理来测距,用一根横杆装上两个镜头,相距十几米。因为人的眼珠不可能相距这么远,靠外斜视来提高视觉效果总是有限。 

后来车来了,我和小舅去了玉渊潭。那里的水有股泥土的腥味,小舅还说,每年冬天把水放干净,都能在泥里找到几个只剩骨头的死人。这使我感到在我身下的湖底里,有些死尸正像胖大海一样发开,身体正溶解在着墨绿色的水里;因此不敢把头埋进水面。把我吓够了以后,小舅自己游开,去看岸上女孩子的身材。据我所见,身材一般,真有一流身材的人也不到湖里来游水。不管有多少不快,那一天我总算看到了小舅的身体。他的家伙确实大。从水里出来以后,龟头泡得像蘑菇一样惨白。后来,这惨白的龟头就印在了我脑海里,晚上做梦,梦见小舅吻了我,醒来擦嘴唇──当然,这是个恶梦。我觉得这个惨白的龟头对世界是一种威胁。从水里出来以后,小舅的嘴唇乌紫,眼睛里布满了血丝。他给我十块钱,叫我自己打车回去,自己摇晃着身躯走开了。我收起那十块钱,小心翼翼地跟着他,走向大地咖啡馆,走向危险。因为我爱他,我不能让他一人去冒险。 

我舅舅常去大地咖啡馆,我也常去。它是座上世纪中叶建造的大屋顶瓦房,三面都是带铁栅栏的木窗。据说这里原来是个副食商场,改作咖啡馆以后,所有的窗子都用窗帘蒙住了。黑红两色的布窗帘,外红里黑,所以房子里很黑。在里面睡着了,醒来以后就不知是白天还是黑夜。除非坐在墙边的车厢座上,撩起了窗帘,才会看到外面的天光和满窗台的尘土。所有的小桌上都点着廉价的白色腊烛,冒着黑烟,散发着石腊的臭气,在里面呆久了,鼻孔里就会有一层黑。假如有一个桌子上点着无烟无臭的黄色腊烛,那必是小舅──他像我一样受不了石腊烟,所以总是自带腊烛。据说这种腊是他自己做的,里面掺有蜂蜡。他总是叫杯咖啡,但总是不喝。有位小姐和他很熟,甚至是有感情,每次他来,都给他上真正的巴西咖啡,却只收速溶咖啡的钱。但小舅还是不喝,她很伤心,躲到黑地里哭了起来。 

我希望自己能看到小舅卖画的情形,下功夫盯住了他,在大地咖啡馆的黑地上爬,把上衣的袖子和裤子全爬破了。服务小姐端咖啡过来,手里打着手电筒,我也爬着躲开她们。偶尔没爬开,绊到了她们的脚上,她们摔了盘子高叫一声:闹鬼啊!然后小舅起身过来,把我揪出去,指着回家的路,说出一个字:“滚”。我假装走开,一会儿又溜回来,继续在黑地上爬。在黑暗中,我感觉那个咖啡馆里有蟑螂、有耗子,还有别的一些动物;其中有一个毛茸茸,好像是只黄鼠狼。它咬了我一口,留下一片牙印,比猫咬的小,比老鼠咬的大。这个混帐东西的牙比锥子还要快。我忍不住叫了一声“他妈的!”又被小舅逮住了。然后被他揪到外面去,然后我又回来。这种事一下午总要发生几回,连我都烦了。 

后来,我舅舅终于等到了要等的人,那人身材粗壮,头顶秃光光,不住地朝他鞠躬,大概为不守时而道歉罢。我觉得他是个日本人,或者是久居日本的中国人。他们开始窃窃私语,我舅舅还拿出彩色照片给对方看。我认为,此时他正在谈交易,但既没看到画,也没看到钱。当然,这两样东西我也很想看一看,这样才算看清了艺术家的行径。他们从咖啡馆里出来后,我继续跟踪。不幸的是,我总在这时被我舅舅逮住。 

他藏在咖啡馆门边,或者小商亭后面,一把揪住我的脖领子,把我臭揍一顿──这家伙警觉得很。他们要去交割画和钱,这是可以被人赃并获的危险阶段,所以总是往身后看。在跟踪小舅时,必须把他眼睛的位置像胖头鱼考虑在内。他的视野比常人开阔,不用回头就能看到身后的事。一件事我始终没搞清楚:警察是怎么逮住他的。大概他们比我还要警醒吧。 

有一天,我在街上遇上那个日本人,他穿着条纹西装,挎着一个身材高挑的女郎。这位女郎穿着绿色的丝质旗袍,身材挺拔,步履矫健,但皮肤粗糙,看上去有点老我往她脸上看了一下,发现她两眼间的距离很宽,就心里一动,跟在后面。她蹲下整理高跟鞋,等我从身边走过时,一把揪住我,发出小舅的声音说:混蛋,你怎么又跟来了!除此之外,她还散发着小舅特有的体臭。开头我就怀疑是她是小舅,现在肯定了。我说:你怎么干起了这种事?他说:别胡扯!我在卖画。你再跟着,我就掐死你!说着,小舅捏着我肩膀的指头就如两道钢钩,嵌进了我的肉。要是换个人,准会放声大哭。但我忍得住。我说:好吧,我不跟着你,但你千万别这样叫人逮住!等他放开手,我又建议他戴个墨镜──他这个样子实在叫人不放心。说实在的,干这种事时把我带上,起码可以望望风。但是小舅不想把我扯进去,宁可自己去冒险。假如被人逮到,就不仅是非法交易,还是性变态。我还听说,有一次小舅在身上挂了四块硬纸板,蹲在街上,装做一个邮筒,那个日本人则装成邮递员去和他交易。但这件事我没见到,是警察说的。还有一次他装成中学生,到麦当劳去扫地,把画藏在麦当劳的垃圾桶里;那个日本人装成垃圾工来把画收走。这些事被人逮到了,我所以才能知道。 

但小舅不会次次被人逮到,那样的话他没有收入,只好去喝西北风。有一次我到百花山去玩,看到有些当地人带着小驴在路边,请游客骑驴游山,就忽发奇想,觉得小舅可能会扮成一条驴,让那个日本人骑上,一边游山,一边谈交易。所以我见到驴就打它一下──我是这样想的:假如驴是我舅舅,他绝不会容我打他,必然会人立起来,和我对打──驴倒没什么大反应,看来它们都不是小舅。驴主却要和我拼命,说道:这孩子,手怎么这样贱呢!看来小舅还没有想到这一出──这很好,我可不愿让舅舅被人骑。我没跟他们说我在找舅舅,因为说了他们也不信。这是我游百花山的情形。 

有一阵子我总想向小舅表白:你不必躲我,我是爱你的。但我始终没这样说,我怕小舅揍我。除此之外,我也觉得这话太惊世骇俗。小舅的双眼隔得远,目光朦胧,这让人感觉他离得很近。当然,这只有常受他暗算的人才能体会到。我常常觉得自己在危险的距离之外,却被他一脚踢到。据说二十世纪的功夫大师李小龙也有这种本领,但不知他是否也是外斜视。 

警察叔叔说,小舅也有一点好处,那就是被“抄”着以后从来不跑,而是迎着手电光走过来说:又被你们逮住了。他们说:小舅不愧是艺术家,不小气,很大气。这个“抄”字是警察的术语,指有多人参加的搜捕行动。我理解它是从用网袋从水里抄鱼的“抄”字化出来的。在这种情况下,鱼总是扑扑腾腾地乱跳,所以很小气。假如它们在袋底一动不动地躺着,那就是很大气的鱼。可惜此种水生脊椎动物小气的居多,所以层次很低。我舅舅这条大 气的鱼口袋里总是揣着一些卖画得来的钱,就被没收了。 

假如这件事就此结束,对双方都很方便。但这样做是犯错误。正确的作法是没收了赃款以后,还要把小舅带到派出所里进行教育。小舅既然很大气,就老老实实地跟他们去了。我总觉得小舅在这时跑掉,警察叔叔未必会追──因为小舅身上没有钱了。我舅舅觉得我说得也有道理,但他还是不肯跑。他觉得自己是个有身份的人,不是小毛贼,跑掉没有出息。有出息的人进了派出所,常常受到很坏的对待。真正没出息的小毛贼,在那里才会如鱼得水。 

警察叔叔说,骑辆自行车都有执照,何况是画画。他听了一声不吭,只顾鼓起双腮,往肚子里咽空气,很快就像个气球一样胀起来了。把自己吹胀是他的特殊本领,其中隐含着很深的含意。我们知道,过去人们杀死了一口猪,总是先把它吹胀,然后用原始的工艺给他褪毛。有一句俗话叫作死猪不怕开水烫,表示在逆境中的达观态度。 

我舅舅把自己吹胀,意在表示自己是个不怕烫的死猪。此后他鼓着肚子蹲在墙下,等家属签字领人。这本是我妈的任务,但她不肯来,只好由我来了。我是个小孩子,走过上世纪尘土飞扬的街道,到派出所领我舅舅;而且心里在想,快点走,迟了小舅会把自己吹炸掉,那样肠子肚子都崩出来很不好看。其实,我是瞎操心:胀到了一定程度,内部的压力太大,小舅也会自动泄气。那时“扑”的一声,整个派出所里的纸张都会被吹上天,在强烈的气流冲击之下,小舅的声带也会发出挨刀断气的声音。此后他当然瘪下去了,摊在地面上,像一张煎饼;警察想要踢他都踢不到,只能用脚去踩;一面踩一面说:你们这些艺术家,真叫贱。我不仅喜欢艺术家,也喜欢警察。我总觉得,这两种人里少了一种,艺术就会不存在了。 

小时候,我家住在圆明园附近。圆明园里面有个黑市,在靠围墙的一片杨树林里。傍着一片半乾涸的水面,水边还有一片乾枯的芦苇。夏天的傍晚,因为树叶茂盛,林子里总是黑得快;秋天时树叶总是像大雨一样地飘落。进公园是要门票的,但可以跳墙进去,这样就省了门票钱。树林里的地面被人脚踩得很磁实,像陶器的表面一样发着亮;树和树之间拉上了一些白布,上面写了一些红字,算作招牌。这里有股农村的气味。有一些农民模样的人在那里出售假古董,但假如你识货,也能买到刚从坟里刨出来的真货:一想到有人在卖死人的东西,我心里就发麻。在那些骗子中间,也有几个穿灯芯绒外套的人坐在马扎上,两眼直勾勾盯着自己的画,从早坐到晚,无人问津,所以神情忧郁。有些人经过时,丢下几张毛票,他不动,也不说谢。再过一会儿,那些零钱就不见了。有一阵子我常到那里去看那些人:我喜欢这种情调;而且断定,那些呆坐着的人都是像凡高一样伟大的艺术家──这种孤独和寂寞让我嫉妒得要发狂。 

我希望小舅也坐在这些人中间,因为他气质抑郁,这样坐着一定很好看,何况他正对着一洼阴郁的死水。一到春天,水面就要长水华,好像个浓绿色的垃圾场。湖水因此变得粘稠,不管多大的风吹来,都不会起波浪。我觉得他坐在这里特别合适,不仅好看,而且可以拣点毛票。但我忽略了他本人乐意不乐意。 

我把小舅领出来,我们俩走在街上时,他让我走到前面,这不是个好意思。就在这样走着时,我对他提起我家附近的艺术品黑市,卖各种假古董,字画,还有一些流浪艺术家在那里摆地摊。圆明园派出所离我家甚近,领起他来也方便,但我没有把那个“领”字说出来,怕他听了会不高兴。他听了一声不吭,又走了一会儿,他忽然给我下了一个绊儿,让我摔在水泥地上,把膝盖和手肘全都摔破了;然后又假惺惺地来搀我,说道:贤甥,走路要小心啊。从此之后,我就知道圆明园的黑市层次很低,我舅舅觉得把自己的画拿到那里卖辱没了身分。我舅舅总是一声不吭,像眼镜蛇一样的阴险;但是我喜欢他,也许是因为我们俩像吧。 

由小孩子去领犯事的人有不少好处,其中最大的一种是可以减少罗嗦。警察看到听众是这样的年幼,说话的欲望就会减少很多。开头时,我骑着山地车,管警察叫大叔,满嘴甜言蜜语,直到我舅舅出来;后来就穿着灯芯绒外套,坐在接待室里沉默不语,直到我舅舅出来;我到了这个年龄,想要说话的警察总算是等到了机会,但我沉默的态度叫他不知该说点什么;实在没办法,只好说说粮食要涨价,以及万安公墓出产的蛐蛐因为吃过死人肉,比较善斗。当然,蛐蛐再善斗,也不如耗子。警察说:斗耗子是犯法的,因为可以传染鼠疫。既然斗耗子犯法,我就不言不语。开头我舅舅出来时,拍拍我的头,给我一点钱做贿赂;后来我们俩都一言不发,各自东西──到那时,我已经不需要他的钱,也被他摔怕了。这段时间前后有五六年,我长了三十公分,让他再也拍不到我的头──除非他踮起脚尖来。本来我以为自己到了七八十岁还要拄着拐棍到派出所去领舅舅,但事情后来有了极好的转机──人家把他送进了习艺所。那里的学制是三年,此后起码有三年不用我领了。 

习艺所是给流浪艺术家们开设的。在那里,他们可以学成工程师或者农艺师,这样少了一个祸害,多了一个有益的人,社会可以得到双重的效益。我听说,在养猪场里,假如种猪太多,就阉掉一些,改作肉猪,这当然是个不伦不类的类比。我还听说现在中国人里性比失衡,男多女少,有人呼吁用变性手术把一部份男人改作女人。这也是个不伦不类的类比。艺术家太多的确是个麻烦,应该减少一些,但减少到我舅舅头上,肯定是个误会。种猪多了,我们阉掉一些,但也要留些作种;男人多了,我们做掉一些,但总要留下一些。假如通通做掉靠无性繁殖来延续种族,整个社会就会退化到真菌的程度。对于艺术来说,我舅舅无疑是一个种。把他做掉是不对的。

2 

我舅舅进习艺所之前,有众多的情人。这一点我知之甚详,因为我常溜进他的屋子,躲在壁柜里偷看。我有他房门的钥匙,但不要问我是怎么来的。小舅的客厅里挂满了自己的作品,但是不能看,看久了会头晕。这也是他犯错误的原因之一。领导上教训他说:好的作品应该让人看了心情舒畅,不该让人头晕。小舅顶嘴道:那么开塞露就是好作品?这当然是乱扳杠,领导上说的是心情,又不是肛门。不过小舅扳杠的本领很大,再高明的领导遇上也会头疼。 

每次我在小舅家里,都能等到一个不认识的姑娘。那女孩子进到小舅的客厅里,四下巡视一下,就尖叫一声,站不住了。小舅为这些来客备有特制的眼镜:平光镜上糊了一层黑纸,中央有个小洞。戴上这种眼镜后,来宾站住了脚,问道:你画的是什么呀?小舅的回答是:自己看嘛。那女孩就仔细看起来,看着看着又站不住了。小舅为这种情况备有另一种特制眼镜:平光镜上糊一层黑纸,纸上有更小的一个洞。透过这种眼镜看一会儿,又会站不住,直到戴上最后一种眼镜,这种眼镜只是一层黑纸,没有窟窿,戴上以后什么都看不见了,但是照样头晕;哪怕闭上眼,那些令人头晕的图案继续在眼前浮动。那些女孩晕晕糊糊地全都爱上了小舅,就和他做起爱来。我在壁柜里透过窄缝偷看,看到女孩脱到最后三点,就按照中学生守则的要求,自觉地闭上眼睛不看。只听见在娇喘声声中,那女孩还在问:你画的到底是什么呀。我舅舅的答案照旧是:自己看。我猜想有些女孩子可能是处女,她们最后问道:我都是你的人了,快告诉我你画的是什么。小舅就说:和你说实话罢,我也不知道。然后那女孩就抽他一个嘴巴。然后小舅说,你打我我也不知道。然后小舅又挨了一个嘴巴。这说明他的确是不知道自己画了一些什么。等到嘴巴声起时,我觉得可以睁眼看了。看到那些女孩子的模样都差不多:细胳膊细腿,身材苗条。她们都穿两件一套的针织内衣,上身是半截背心,下身是三角裤,区别只在内衣的花纹。有人的内衣是白底红点,有的是黑底绿竖纹,还有的是绿底白横纹。不管穿什么,我对她们都没有好感──既不是艺术家,也不是警察,想作我的舅妈,你配吗?我舅舅进习艺所时,我也高中毕业了。我想当艺术家,不想考大学。但我妈说,假如我像小舅一样不三不四,她就要杀掉我。为了证明自己的决心,她托人从河北农村买来了六把杀猪刀,磨得雪亮,插在厨房里,每天早上都叫我到厨房里去看那些刀。 

假如刀上长了黄锈,她再把它磨得雪亮,还时常买只活鸡来杀,试试刀子。杀过之后,再把那只鸡的尸体煮熟,让我吃下去。如此常备不懈,直到高考完毕。我妈是女中豪杰,从来是说到做到。我被她吓得魂不守舍,浑浑噩噩地考完了试,最后上了北大物理系。这件事的教训是:假如你怕杀,就当不了艺术家,只能当物理学家。如你所知,我现在是个小说家,也属艺术家之列。但这不是因为我不怕杀──我母亲已经去世,没人来杀我了。 

十年前,我送小舅去习艺所,替他扛着行李卷,我舅舅自己提着个大网兜──这种东西又叫作盆套,除了盛脸盆,还能盛毛巾、口杯、牙刷牙膏和几卷卫生纸,我们一起走到那个大铁门面前。那一天天气阴沉。我不记得那天在路上和舅舅说了些什么,大概对他能进去表示了羡慕罢。那座大门的背后,是一座水泥墙的大院,铁门紧关着,只开着一扇小门,每个人都要躬着腰才能进去,门前站了一大群学员,听唱名鱼贯而入。顺便说一句,我可不是自愿来送我舅舅,如果是这样,非被小舅摔散了架不可。 

领导上要求每个学员都要有亲属来送,否则不肯接受。轮到我们时,发生了一件事,可以说明我舅舅当年的品行。我们舅甥俩年龄相差十几岁,这不算很多,除此之外,我们俩都穿着灯芯绒外套──在十年前,穿这种布料的都是以艺术家自居的人──我也留着长头发,而且我又长得像他。总而言之,走到那个小铁门门口时,我舅舅忽然在我背上推了一把,把我推到里面去了。等我想要回头时,里面的人早已揪住了我的领子,使出拽犟牛的力气往里拉。人家拽我时,我本能地往后挣,结果是在门口僵住了。我外衣的腋下和背后在嘶嘶地开线,与此同时,我也在声嘶力竭地申辩,但里面根本不听。必须说明,人家是把我当小舅揪住的,这说明喜欢小舅的不止我一人。 

那个习艺所在北京西郊某个地方,我这样一说,你就该明白,它的地址是保密的。在它旁边,有一圈铁丝网,里面有几个鱼塘。冬末春初,鱼塘里没有水,只有乾裂的泥巴,到处是塘泥半干半湿的气味。鱼塘边上站了一个穿蓝布衣服的人,看到来了这么一大群人,就张大了嘴巴来看,也不怕扁桃腺着凉──那地方就是这样的。我在门口陷住了,整个上衣都被人拽了上去,露出了长长的脊梁,从肋骨往下到腰带,都长满了鸡皮疙瘩。至于好看不好看,我完全顾不上了。 

我和小舅虽像,从全身来看还有些区别。但陷在一个小铁门里,只露出了上半身,这些区别就不显著了。我在那个铁门里争辩说,我不是小舅;对方就松了一下,让人拿照片来对,对完以后说道:好哇,还敢说你不是你!然后又加了把劲来拽我。这一拽的结果使我上半身的衣服顿呈土崩瓦解之态。与此同时,我在心里犯起了嘀咕:什么叫“还敢说你不是你?”这句话的古怪之处在于极难反驳。我既可以争辩说:“我是我,但我是另一个人”,又可以争辩说:“我不是我,我是另一个人”,更可以争辩说:“我不是另一个人,我是我!”和“我不是另一个人,我不是我!”不管怎么争辩,都难于取信于人,而且显得欠揍。 

在习艺所门前,我被人揪住了脖领,这是一种非同小可的经历,不但心促气短,面红耳赤,而且完全勃起了。此种经历完全可以和性经历相比,但是我还是不想进去。主要的原因是:我觉得我还不配。我还年轻,缺少成就,谦逊是我的美德,这些话我都对里面的人说过了,但是她们不信。除此之外,我也想到:假如有一个地方如此急迫地欢迎你,最好还是别进去。说起来你也许不信,习艺所里面站着一条人的甬道,全是穿制服的女孩子,叽叽喳喳地说道:拿警棍敲一下──别,打傻了──就一下,打不傻,等等。你当然能想到,她们争论的对象是我的脑袋瓜。听了这样的对话,我的头皮一炸一炸的。揪我脖子的胖姑娘还对我说:王二,你怎么这样不开窍呢?里面好啊。她说话时,暖暖的气息吹到我脸上,有股酸酸的气味,我嗅出她刚吃过一块水果糖。但我呼吸困难,没有回答她的话。有关这位胖姑娘,还要补充说,因为隔得近,我看到她头上有头皮屑。假如没有头皮屑,也许我就松松劲,让她拽进去算了。 

后来,这位胖姑娘多次出现在我的梦境里,头大如斗,头皮屑飞扬,好像拆枕头抖荞麦皮。在梦里我和她做爱,记得我还不大乐意。当时我年轻力壮,经常梦遗。我长到那么大,还没有女人揪过我脖子哪。不过现在已是常事。我老婆想要对我示爱,径直就会来揪我脖领子。在家里我穿件牛仔服,脖子后面钉着小牛皮,很经拽。 

我小舅叫作王二,这名字当然不是我姥爷起的。有好多人劝他改改名字,但他贪图笔划少,就是不改。至于我,绝不会贪图笔划少,就让名字这样不雅。我想,被人揪住了脖子,又顶了这么个名字,可算是双重不幸了。后来还是我舅舅喝道:放开吧,我是正主儿,人家才放开我。就是这片刻的争执,已经把我的外套完全撕破。它披挂下来,好像我背上背了几面小旗。我舅舅这个混蛋冷笑着从我背上接过铺盖卷,整整我的衣服,拍拍我的肩膀,说道,对不起啊,外甥。然后他往四下里看了看,看到这个大门两面各有一个水泥门柱,这柱子四四方方,上面有个水泥塑的大灯球,他就从牙缝里吐口唾沫说:真他妈的难看。然后躬躬腰钻了进去。里面的人不仅不揪他,反而给他让出道儿来──大概是揪我揪累了。我独自走回家去,挂着衣服片儿,四肢和脖子上的肌肉酸痛,但也有如释重负之感。回到家里就和我妈说:我把那个瘟神送走了。我妈说:好!你立了一大功!无须乎说,瘟神指的是小舅。进习艺所之前,他浑身都是瘟病。 

我把小舅送进习艺所之后,心里有种古怪的想法:不管怎么说罢,此后他是习艺所的人了,用不着我来挂念他。与此同时,就想到了那个揪我脖子的胖姑娘。心里醋溜溜的。后来听说,她常找男的搬运工扳腕子,结过两次婚,现在无配偶,常给日本的相扑力士写求爱信。相扑力士很强壮,挣钱也多──她对小舅毫无兴趣,是我多心。 

习艺所里还有一位教员,身高一米四,骨瘦如柴、皮肤苍白,尖鼻子、尖下巴,内眼角上常有眼屎,稀疏的头发梳成两条辫子。她对小舅也没有兴趣。这位老师已经五十二岁,是个老处女,早就下了决心把一生献给祖国的特殊教育事业。在这两者之间,还有各种各样的女教员,但她们对小舅都无兴趣。小舅沉默寡言,性情古怪,很不讨人喜欢。在我舅舅的犯罪档案里,有他作品的照片。应该说,这些照片小,也比原画好看,但同样使人头晕。根据这些照片大家都得出了结论:我舅舅十分讨厌。看起来没有人喜欢小舅,是我多心了。 

在习艺所里,有各种各样的新潮艺术家;有诗人、小说家、电影艺术家,当然,还有画家。每天早上的德育课上,都要朗诵学员的诗文──假如这些诗文不可朗诵,就放幻灯。然后请作者本人来解释这段作品是什么意思。毫无疑问,这些人当然嘴很硬:这是艺术,不是外人所能懂的。但是这里有办法让他嘴不硬──比方说,在他头上敲两棍。嘴不硬了以后,作者就开始大汗淋漓,陷于被动;然后他就会变得虚心一些,承认自己在哗众取宠,以博得虚名。然后又放映学员拍的电影。电影也乌七八糟,而且叫人感到恶心。不用教员问,这位学员就感到羞愧,主动伸出头来要挨一棍。他说他拍这些东西送到境外去放映,是想骗外国人的钱。不幸的是,这一招对小舅毫无用处。放过他作品的幻灯片后,不等别人来问,他就坦然承认:画的是些什么,我自己也不懂。正因为自己不懂,才画出来叫人欣赏。此后怎样让他陷于被动,让所有的教员头疼。大家都觉得他画里肯定画了些什么,想逼他说出来。他也同意这画是有某种意义的,但又说:我不懂。我太笨。按所领导的意思,学员都是些自作聪明的傻瓜。因为小舅不肯自作聪明,所领导就认为,他根本不是傻瓜,而是精得很。 

我常到习艺所去看小舅,所里领导叫我劝劝他,不要装傻,还说,和我们装傻是没有好处的。我和我舅舅是一头的,就说:小舅没有装傻,他天生就是这么笨。但是所领导说:你不要和我们耍狡猾,耍狡猾对你舅舅是没有好处的。 

除了舅舅,我唯一的亲戚是个远房的表哥。他比小舅还要大,我十岁他就有四十多岁了,人中比朴克牌还宽,裤裆上有很大的窟窿,连阴毛带睾丸全露在外面,还长了一张鸟形的脸。他住在沙河镇上,常在盛夏时节穿一双四面开花的棉鞋,挥舞着止血带做的弹弓,笑容可掬地邀请过路的小孩子和他一道去打马蜂砣子──所谓马蜂砣子,就是莲蓬状的马蜂窝,一般是长在树上。表哥说起话来一口诚恳的男低音。他在镇上人缘甚好,常在派出所、居委会等地出出进进,你要叫他去推垃圾车、倒脏土,他绝不会不答应。有一次我把他也请了来,两人一道去看小舅;顺便让所领导看看,我们家里也有这样的人物。谁知所领导看了就笑,还指着我的鼻子说:你这个小子,滑头到家了!表哥却说:谁滑头?我打他!嗓音嗡嗡的。表哥进了习艺所,精神抖擞,先去推垃圾车、倒脏土,然后把所有的马蜂砣子全都打掉,弄得马蜂飞舞,谁也出不了门,自己也被螫得像个大木桶。虽然打了马蜂砣子,习艺所里的人都挺喜欢他。回去以后不久,他就被过路的运煤车撞死了,大家都很伤心,从此痛恨山西人,因为山西那地方出煤。给他办丧事时,镇上邀请我妈作为死者家属出席,她只微感不快,但没有拒绝。假如死掉的是小舅,我妈去不去还不一定。这件事我也告诉了小舅。小舅发了一阵愣,想不起他是谁;然后忽然恍然大悟道:看我这记性!他还来打过马蜂砣子哪。小舅还说,很想参加表哥的追悼会。但是已经晚了。表哥已经被烧掉了。 

德育课后,我舅舅去上专业课。据我从窗口所见,教室顶上装了一些蓝荧荧的日光灯管,还有一些长条的桌椅,看起来和我们学校里的阶梯教室没什么两样,只是墙上贴的标语特别多些,还有一种区别,就是这里的窗户上有铁栅栏、铁窗纱,上面有个带闪电符号的牌子,表示有电。这倒是不假,时常能看到一只壁虎在窗上爬着,忽然冒起了青烟,变成一块焦炭。还有时一只蝴蝶落在上面,“丝”地一声之后,就只剩下一双翅膀在天上飞。我舅舅对每个问题都积极抢答,但只是为了告诉教员他不会。 

后来所方就给他穿上一件紧身衣,让他可以做笔记,但举不起手来,不能扰乱课堂秩序。虽然不能举手,但他还是多嘴多舌,所以又给他嘴上贴上一只膏药,下课才揭下来。这样贴贴揭揭,把他满嘴的胡子全数拔光,好像个太监。我在窗外看到过他的这种怪相:左手系在右边腋下,右手系在左边腋下,整个上半身像个帆布口袋;只是两只眼睛瞪得很大,几乎要胀出眶来。每听到教员提问,就从鼻子里很激动地乱哼哼。哼得厉害时,教员就走过去,拿警棍在他头上敲一下。敲过了以后,他就躺倒打瞌睡了。有时他想起了蹲派出所时的积习,就把自己吹胀,但是紧身衣是帆布做的,很难胀裂,所以把他箍成了纺锤形──此时他面似猪肝。然后这些气使他很难受,他只好再把气放掉──贴住嘴的橡皮膏上有个圆洞,专供放气之用──这时坐在前面的人就会回过头来,在他头顶上敲一下说:你丫嘴真臭。 

所方对学员的关心无微不至,预先给每个学员配了一副深度近视镜,让他们提前戴上;给每个人做了一套棕色毛涤纶的西服做为校服,还发给每人一个大皮包,要求他们不准提在手里,要抱在怀里,这样看起来比较诚恳。学校里功课很紧,每天八节课,晚上还有自习。为了防止学生淘气,自习室的桌子上都带有锁颈枷,可以强使学生躬腰面对桌面。经过一段时间的学习,学生个个呈现出学富五车的模样──也就是说,个个躬腰缩颈,穿棕色西服,怀抱大皮包,眼镜像是瓶子底,头顶亮光光,苍蝇落上去也要滑倒──只可惜有名无实,不但没有学问,还要顺嘴角流哈喇子。我舅舅是其中流得最多的一位,简直是哗哗地流。就算习艺所里伙食不好,馋馒头,馋肉,也到不了这个程度。大家都认为,他是存心在流口水,而且是给所里的伙食抹黑。为了制止他流口水,就不给他喝水,还给他吃干辣椒。但我舅舅还是照样流口水,只是口水呈焦黄色,好像上火的人撒出的尿。 

像我舅舅这样的无照画家,让他们学作工程师是很自然的想法。可以想见,他们在制图方面会有些天赋;只可惜送去的人多,学成的少。每个无照画家都以为自己是像毕加索那样的绘画天才,设想自己除了作画还能干别的事,哪怕是在收费厕所里分发手纸,都是一种极大的污辱,更别说去作工程师。因为这个原故,所以当他们被枷在绘图桌上时,全都不肯画机械图。有些人画小猫小狗,有些人画小鸡小鸭,还有个人在画些什么,连自己都不清楚,这个人就是小舅。后来这些图纸就被用作钞票的图案;因为这些图案有不可复制的性质。我们国家的钞票过去是由有照的画家来画,这些画随便哪个画过几天年画的农民都能仿制。而习艺所学员的画全都怪诞万分,而且杂有一团一团的晕迹,谁都不能模仿;除非也像他们一样连手带头地被枷在绘图桌上。 

至于那些晕迹,是他们流下的哈喇子,和嘴唇、腮腺的状态相关,更难模仿。我舅舅的画线条少、污渍多,和小孩子的尿布相仿,被冒充齐白石画的水墨荷叶,用在五百元的钞票上。顺便说一句,我舅舅作这幅画时,头和双手向前探着,腰和下半身落在后面,就像动画片的老狼定了格。制图课的老师从后面走过时,用警棍在他头上敲上一下,说道:王犯(那地方就兴这种称呼)!别像水管子一样!老师嫌他口水流得太多了。因为口水流得太多,我舅舅总是要口渴,所以他不停地喝水。后来,他变得像巴甫洛夫的狗一样,一听到上课铃响,口水就忍不住了。 

我听说,在习艺所里,就数机械班的学员(也就是那些无照画家)最不老实。众所周知,人人都会写字,写成了行就是诗,写成了片就是小说,写成了对话的样子就是戏剧。所以诗人、小说家、剧本作家很容易就承认自己没什么了不起。画家就不同了,给外行一些颜色,你都不知怎么来弄。何况他们有自己的偶像:上上世纪末上世纪初的一帮法国印象派画家。你说他是二流子,他就说:过去人们就是这样说凡高的!我国和法国还有邦交,不便把凡高也批倒批臭。所里另有办法治这些人:把他们在制图课上的作品制成了幻灯片,拿到德育课上放,同时说道:某犯,你画的是什么?该犯答道:报告管教!这是猫。于是就放一张猫的照片。下一句话就能让该犯羞愧得无地自容:大家都看看,猫是什么样子的!经过这样的教育,那个人就会傲气全消,好好地画起机械图来。但是这种方法对我舅舅没有用。放到我舅舅的水墨荷叶,我舅舅就站起来说:报告管教!我也不知自己在画什么!教员只好问道:那这花里胡哨的是什么?小舅答道:这是干了的哈喇子。教员又问:哈喇子是这样的吗?小舅就说:请教管教!哈喇子应该是怎样的?教员找不到干哈喇子的照片,没有别的办法,只好用橡皮膏把他的嘴再贴上了。 

我舅舅进习艺所一个月以后,所里给他们测智商。受试时被捆在特制的测试器上,这种测试器又是一台电刑机。测出的可以说是IQ,也可以说是受试者的熬刑能力。那东西是两个大铁箱子,一上一下,中间用钢架支撑,中间有张轻便的担架床,可以在滑轨上移动。床框上有些皮带,受试者上去时,先要把这张床拉出来,用皮带把他的四肢捆住,呈“十”字形;然后再把他推进去──我们学校食堂用蒸箱蒸馒头,那个蒸箱一屉一屉的,和这个机器有点像──假如不把他捆住,智商就测不准。为了把学员的智商测准,所里先开了一个会,讨论他们的智商是多少才符合实际。教员们以为,这批学员实在桀傲难驯,假如让他们的智商太高,不利于他们的思想改造。但我舅舅是个特例,他总在装傻,假如让他智商太低,也不利于他的思想改造。 

我舅舅后来说,他绕着测智商的仪器转了好几圈,想找它的铭牌,看它是哪个工厂出产的,但是没找到;只看到了粗糙的钣金活,可以证明这东西是国货。他的结论是:原来有铭牌,后来抠掉了,因为还有铭牌的印子;拆掉的原因大概是怕学员出去以后会把那个工厂炸掉。那机器上有一对电极,要安到受测人的身上。假如安得位置偏低,就会把阴毛烧掉;安高了则把头顶的毛烧掉。总而言之,要烧掉一些毛,食堂里遇到毛没有退净的猪头猪肘子,也会送来测测智商,测得的结果是猪头的智商比艺术家高,猪肘的智商比他们低些。总而言之,这机器工作起来总有一股燎猪毛的味道。假如还有别的味儿,那就是忘了那条标语:“受试前先如厕”,标语后面还有一个箭头,指着厕所的方向。厕所的门和银行的金库一样,装了定时锁,进去以后就要关你半小时。里面还装了个音箱,放着创作歌曲──这种音乐有催屎催尿的作用。 

受测时,学员都是这样要求的:我们还要会女人,请给我留下底下的毛。有时候操作仪器的教员却说:我想要留下上边的毛。这是因为习艺所的教员全是纯真的女孩子,有些人和学员有了感情,所以留下他的头发,让他好看一点;烧掉他的阴毛,省得他沾花惹草。除此之外,她还和他隔着仪器商量道:你就少答对几道题罢,别电傻了呀!坦白地说,这种因素不一定能降低学员的智商,因为他很可能瘦驴屙硬屎,硬充男子汉。宁可挨电,也不把题答错。等到测试完成,学员往往瘫成一团,于是就时常发生教员哭哭啼啼地把学员往外背的动人情景。 

测智商的场面非常的刺激。房顶上挂了一盏白炽灯,灯泡很小,但灯罩却大,看起来像个高音喇叭。这盏灯使房间的下半截很亮,却看不到天花板。教员把学员带到这里,哗啦一声拉出放人的抽屉,说道:脱衣服,躺上去;然后转身穿上白大褂,戴上橡皮手套。那屋里非常冷,脱掉了衣服就起鸡皮疙瘩。有些人在此时和教员说几句笑话,但我舅舅是个沉默的人,他一声都不吭。抽屉里有皮带,教员动手把学员绑紧,绑得像十字架上的基督──两手平伸,两腿并紧,左脚垫在右脚下。贫嘴的学员说:绑这么紧干嘛,又不是猪。教员说:要是猪也好,我们省心多了。多数学员被绑上以后,都是直撅撅的。教员就说:这时候还不老实?而学员回答:没有不老实!平时它就是这么大嘛。教员说:别吹牛了,就轰地一声把他推进去。我舅舅躺在抽屉里时也是直撅撅,但人家问他话时,他一声不吭。教员在他肚子上一拍,说:喂!王犯!和你说话呢!你平时也是这么大吗?他却闭上眼睛,说道:平时比这要小。快点吧。于是也轰隆一声被推了进去。他们说,这抽屉下面的轮子很好使,人被推进去时,感觉自己是一个自由落体,完全没有了重量;然后就“通”地一声巨响,头顶撞在机器的后壁上,有点发麻。我对这一幕有极坏的印象──我很不喜欢被捆进去。当然,假如我是教员,身穿白大褂,把一些美丽的姑娘捆进抽屉,那就大不一样。 

人家说,在那个抽屉的顶壁上,有一个彩色电视屏幕,问题就在这里显示。假如教员和学员有交情,在开始测试之前,会招待他先看一段轻松的录相,然后再下手把他电到半死,就如一位仁慈的牙医,在下手拔牙前先给病人一块糖吃。但轮到我舅舅,就没有录相看。教员不出题,先把他电得一声惨叫。每一个学员被推进去之前,都是一段冰冷的肉体,只在口鼻之间有口气,胯间有个东西像旗杆一样挺着;但拉出来时就会热汽蒸腾,好像已经熟透了。但是这种热气里一点好味都没有,好像蒸了一块臭肉。假如他头上有头发,就会卷起来,好像拉力弹簧,至于那挺着的东西,当然已经倒下去了。但我舅舅不同,他出来时直橛橛的,比进去时长了两三倍,简直叫人不敢看。有些人哼哼着,就如有只牛蜂或者屎克螂在屋里飞,有些人却一声不吭。而我舅舅出来时,却像个疯子一样狂呼滥喊道:好啊!很好啊!很煽情!如前所述,此时要由教员把学员背走,背法很特别。她们把学员放开,把他的脚拽在肩上,吆喝一声,就大头朝下地背走了──据说在屠宰场里背死猪就是这样一种背法。但是没人肯来背我舅舅。她们说:王犯,别装死,起来走!别人都是死猪,而我舅舅不是。我舅舅真的扶着墙晃晃悠悠的站起来,走掉了。 

现在该谈谈他们的智商是多少。大多数学员的智商都在110-100之间,有个人得了最高分,是115。他还说自己想得个120非难事。但他怕得了这个120,此后就会变得很笨,因为电是能把人打傻了的。至于我舅舅,他的IQ居然是零蛋──他一道题也没答对。这就让所领导很是气愤:就是一根木头棍子,IQ也不能为零。于是他们又调整了电压,叫小舅进去补测。再测的结果小舅也没超过50分。当然,还可以提高一些,但有可能把我舅舅电死。有件事不说你也知道,别人是答对了要挨电,我舅舅是答错了要挨电。有经验的教员说,不怕学员调皮捣蛋,就怕学员像我舅舅这样耍死狗。 

测过智商以后,我舅舅满脸腊黄地躺在床上,好像得了甲型肝炎。这时候我问他感觉如何,他愣了一阵,然后脸上露出了鬼一样的微笑说:很好。他还说自己在那个匣子里精液狂喷,射得满处都是,好像摔了几碟子肉冻,又像个用过的避孕套;以致下一个被推进去的人在里面狂叫道:我操你妈,王二!你丫积点德好不好!大概是嫌那个匣子被我舅舅弄得不大卫生。据说,有公德的人在上测试器之前,除了屙和尿,还要手淫几次,用他们的话来说,叫做捋乾净了再进去,这是因为在里面人会失控。 

但我舅舅不肯这样做,他说,被电打很煽情,捋乾净了就不煽情。我觉得小舅是对的:他是个艺术家,真正的艺术家都是些不管不顾的家伙。但我搞不清什么很煽情:是测试器上显示的那些问题(他还记住了一个问题:“八加七等于几?”)很煽情,还是电流很煽情,还是自己在匣子里喷了一些肉冻很煽情。但我舅舅不肯回答,只是闭上了眼睛。测过智商的第二天,早上出操时,小舅躺在床上没有动;别人叫他他也不答应。等到中午吃完饭回来,他还是躺着没动。同宿舍的人去报告教员,教员说:甭理他,也别给他吃饭,看他能挺多久。于是大家就去上课。等到晚上回来时,满宿舍都是苍蝇。这时才发现,小舅不仅死掉了,而且还有点发绿。揭开被子,气味实在是难闻。 

于是他们就叫了一辆车,把小舅送往医院的太平间。然后就讨论小舅是怎么死的,该不该通知家属,怎样通知等等。经过慎重研究,得出的结论是我舅舅发了心脏病。死前住了医院,抢救了三天三夜,花了几万元医药费。但是我们可以放心,习艺所学员有公费医疗,可以报销──这就是社会主义的优越性。与此同时,习艺所派专人前往医院,把这些情况通知院方,以备我们去查问。等到所有的谎话都编好,准备通知我们时,李家口派出所来电话说,小舅在大地咖啡馆里无证卖画,又被他们逮住了,叫习艺所去领。这一下叫习艺所里的人全都摸不着头脑了。他们谁都不敢去领人,因为可能有三种情形:其一,李家口逮住了个像小舅的人。在这种情况下去领,好像连小舅死了所里都不知道,显得所里很笨;其二,李家口派出所在开玩笑,在这种情况下去领,也是显得很笨。其三,李家口派出所逮住了小舅的阴魂。在这种情况下去领,助长了封建迷信。后来也不知是哪位天才想起来到医院的太平间里看看死小舅,这才发现他是猪肉、黄豆和面粉做的。这下子活小舅可算惹出大漏子了。 

我的舅舅是位伟大的画家,这位伟大的画家有个毛病,就是喜欢画票证。从很小的时候,就会画电影票、洗澡票,就是不画钱,他也知道画钱犯法;只是偶尔画几张珍稀邮票。等到执照被吊销了以后,他又画过假执照。但是现在的证件上都有计算机号码,画出来也不管用。他还会做各种假东西,最擅长的一手就是到朋友家作客时,用洗衣肥皂做出一泡栩栩如生的大粪放在沙发上,把女主人吓晕过去。这家伙要溜出习艺所,但又要给所里一个交待,他叫我给他找几十斤猪,扛在麻袋里,偷带进习艺所。但我不知道他是做死人。假如知?的话,一定劝他用肥皂来做。把半扇瘟猪放到宿舍里太讨人厌了。 

认真分析小舅前半生的得失,发现他有不少失策之处。首先,他不该画些让人看不懂的画。但是如他后来所说,不画这些画就成不了画家。其次,他应该把那些画叫作海马、松鼠和田螺。但如小舅所说,假如画得是海马、松鼠和田螺,就不叫真正的画家。再其次,他不该在习艺所里装傻。但正如小舅所说,不装傻就太过肉麻,难以忍受了。然后是不该逃走、不该在床上放块死猪肉。但小舅也有的说,不跑等着挨电?不做假死尸,等着人家来找我?所以这些失策也都是有情可原。最后有一条,千不该、万不该,不该一跑出来就作画、卖画。再过几天,习艺所通知我们小舅死了,那就天下太平。那时候李家口派出所通知他们逮住了小舅,他们只能说:此人已死,你们逮错了。我以为小舅还要给自己找些借口,说什么自己技痒难熬,等等。谁知他却发起愣来,愣了好久,才给自己额上重重一掌道:真的!我真笨!

3 

生活里有各种情况,我有不止一个小舅妈,但在此提到的这个却是真的小舅妈。 

我很喜欢小舅,希望他和各种女人结婚;想来想去,一直想到玛丽莲·梦露身上。此人已经死掉多年,尸骨成灰,但听说她活着的时候胸围大得很。如前所述,我舅舅有外斜视的毛病,所以小舅妈的胸围一定要大,否则部份胸部游离于视野之外,视觉效果太差。事实上,我是瞎操心,真的小舅妈只用了一晚上,就把小舅的外斜视治好了。 

小舅妈身材硕长,皮肤白晰,腰肢柔软,无论坐在床上,还是坐沙发,总爱歪着,用一头乌溜溜的短发对着人。除此之外,她总呈现出憋不住笑的模样。她老对我说一句话:有事吗?这是她在我假装无心闯到她住的房间里去看她时说的,此时她就是这个模样。这种事有过很多次。不过都是以前的事。这件事开头时是这样的:我小的时候家住在一楼,后来搬到了六楼上,而且没有电梯。这些楼房有一些赤裸裸的混凝土楼梯,满是尘土、粉皮剥落的楼道,顺着墙角散着垃圾,等等。准确地说,垃圾是些葱皮、鸡蛋皮、还有各种塑料袋子,气味难闻。谁都想扫扫,但谁都觉得自己扫是吃亏。有一天,这个楼梯上响起了沉重的脚步声;然后有个女声在门外说:王犯,就是这儿吗?一个男声答道:是。我听了对我妈说:坏了,是小舅。我妈还不信,说小舅离出来的日子还远着呢。但我是信的,因为对我舅舅的道德品质,我比我妈了解得多。 

等打开门一看,果然是他,还带来了一个穿制服的女孩子,她就是小舅妈,但她不肯明说。我舅舅介绍我妈说:这是我大姐。小舅妈摘了帽子,叫道:大姐。我舅舅介绍我道:这是我外甥。她说:是嘛。然后就哈哈大笑道:王犯,你这个外甥很像你呀!我最不喜欢别人说我像小舅,但是那一次却例外。我觉得小舅妈很迷人。早知道进了习艺所会有这种艳遇,还不如我替我舅舅去哪。 

现在我要承认,我对小舅的女朋友都无好感。但小舅妈是个特例。她第一次出现时,身上穿着制服,头上戴着大檐帽,束着宽宽的皮带,腰里还别了一把小手枪,雄纠纠、气昂昂。我被她的装束给迷住了。而我舅舅出现时,手上带着一副不锈钢铐子;并且端在胸前,好像狗熊作揖一样。就像猫和耗子有区别一样,囚犯和管教也该有些区别,所以有人戴铐子,有人带枪。一进了我们家,小舅妈就把小舅的铐子开了一半。 

这使我以为她给他带手铐是做做样子。谁知她顺手又把开了的一半锁到了暖气管上,然后说:大姐,用用卫生间,就钻进去了。我舅舅在那里站不直蹲不下,半蹲半站,羞羞答答,这就使我犯起疑惑,不知发生了什么事。过一会儿小舅妈出来,又把我舅舅和她铐在了一起,并排坐在沙发上。我觉得他们好像在玩什么性游戏。总的来说,生活里某些事,必须有些幽默感才能理解。但我妈没有幽默感,她什么都不理解,所以气得要死。我有幽默感,我觉得正因为如此,小舅妈才格外的迷人。 

我一见到小舅妈,就知道她很辣,够我舅舅一呛。但不管怎么说,她总是个女的,比男的好吧。在阳台上我祝贺我舅舅,说小舅妈比他以前泡过的哪个妞都漂亮。我舅舅不说话,却向我要了一支烟抽。根据我的经验,我舅舅不说话时,千万别招惹他,否则他会暗算你。除此之外,他那天好像很不高兴。我和他铐在一起,假如他翻了脸打我,我躲都没处躲。我舅舅吸完了那支烟,对我说:这件事是福是祸还不一定;然后又说:回去吧。于是我们回到卧室里,请小舅妈开手铐。小舅妈打量了我们一通,说道:王犯,这小坏蛋长得真像你,大概和你一样坏罢──舅妈和外甥讲话,很少用这种口气。除此之外,我舅舅把那支烟吸得干净无比,连烟屁股都抽掉了。这说明他很需要尼古丁。因为他很能混人缘,所以到了任何地方都不会缺烟吸。如今猛抽起烟屁来,是个很不寻常的景象。总之,自我认识小舅,没见过他如此的低调。 

现在必须承认,年轻时我的觉悟很低,还不如公共汽车上一个小女孩。这个女孩子身上很干净,只穿了个小裤衩,连裙子都没穿。不穿裙子因为她母亲以为她的腿还不足以引起男人的邪念,穿裤衩是因为腿上面的部位足以引起男人的邪念。小舅妈押着我舅舅坐公共汽车,天很晚了,车上只有六七个人。这个小女孩跑到我舅舅面前来,看看他戴着的手铐,去问小舅妈道:阿姨!叔叔这是怎么了?小舅妈解释道:叔叔犯错误了。这孩子爱憎分明,同时又看出,我舅舅是铐着的,行动不便,就朝小舅妈要警棍,要把我舅舅揍一顿。小舅妈解释道,就是犯了错误的叔叔,也不是谁都能打的;那孩子眨着眼睛,好像没听懂。小舅妈又解释道:这个叔叔犯的错误只有阿姨才能打。这回那孩子听懂了,对着小舅妈高叫了一声:讨厌!你很没意思!就跑开了。 

说到觉悟,最低的当然是小舅。其次是我,我总站在他一边想问题。其次是我妈,她看到小舅妈铐着我舅舅就不顺眼。再其次是小舅妈,她对小舅保持了警惕。但是觉悟最高的是那个小女孩。见到觉悟低的人想揍他一顿,就是觉悟高了。 

我舅舅的错误千条万绪,归根结蒂就是一句话,画出画来没人懂。仅此而已还不要紧,那些画看上去还像是可以懂的,这就让人起疑,觉得他包藏了祸心。我现在写他的故事,似乎也在犯着同样的错误──这个故事可懂又没有人能懂。但罪不在我,罪在我舅舅,他就是这么个人。我妈对小舅舅有成见,认为小舅既不像大舅,也不像她,她以为是在产房里搞错了。我长得很像小舅,她就说,我也是搞错了。但我认为不能总搞错,总得有些搞对的时候才成。不管怎么说吧,她总以为只有我能懂得和小舅有关的事──其实这是一个误会,小舅自己都不知自己是怎么回事──所以把我叫到厨房里说:你们是一事的,给我说说看,这是怎么回事?我说:没什么。小舅又泡上了一个妞,是个女警察。他快出来了。我妈就操起心来,但不是为我舅舅操心,是为小舅妈操心。照她看来,小舅妈是好女孩,我舅舅配不上她──我妈总是注意这种配不配的问题,好像她在配种站任职。但是到了晚上她就不再为小舅妈操心,因为他们开始做爱──虽然是在另一间房子里,而且关上了门,我们还是知道他们在做爱,因为两人都在嚷嚷,高一声低一声,终夜不可断绝,闹得全楼都能听见。这使我妈很愤怒,摔门而去,去住招待所,把我也揪走了。最使我妈愤怒的是:原来以为我舅舅在习艺所里表现好,受到了提前毕业(或称释放)的处理,谁知却是相反:我舅舅在习艺所表现很坏,要被送去受惩诫,小舅妈就是押送人员。他们俩正在前往劳改场所途中,忙里偷闲到这里鬼混。为此我妈恶狠狠地对我说:你再说说看,这是怎么一回事?这回连我也不知是怎么回事,可见我和小舅不是一事的。 

等到领略了小舅妈的高觉悟之后,我对她的行为充满了疑问:既然你觉得我舅舅是坏人,干嘛还要和他做爱?她的回答是:不干白不干──你舅舅虽然是个坏蛋,可是个不坏的男人。这叫废物利用嘛。但是那天晚她没有这么说,说了以后我会告诉小舅,小舅会警觉起来──这是很后来的事了。 

小舅和小舅妈做爱的现场,是在我室的小沙发上。我对这一点很有把握,因为头天晚上我离开时,那沙发还硬挺挺的有个模样,等我回来时,它就变得像个发面团。除此之外,在沙发背后的墙壁上,还粘了三块嚼过的口香糖。我把其中一块取下来,尝了一下味道,发现起码嚼了一小时。因此可以推断出当时的景象:我舅舅坐在沙发上,小舅妈骑在小舅身上,嚼着口香糖。想明白了这些,我觉得这景象非常之好,就欢呼一声,扑倒在自己床上。这是屋里唯一的床,但一点睡过的痕迹都没有。但我没想到小舅妈手里拿着枪,枪口对准了我舅舅。知道了这一点,还欢不欢呼,实在很难讲。 

顺便说一句,小舅妈很喜欢和小舅做爱,每回都兴奋异常,大声嚷嚷。这时候她左手总和小舅铐在了一起,右手拿着小手枪,开头是真枪,后来不当管教了,就用玩具枪,比着我舅舅的脑袋。等到能透过气的时候,就说道:说!王犯,你是爱我,还是想利用我?凭良心说,我舅舅以为对国家机关的女职员,首先是利用,然后才能说到爱。但是在枪口对脑袋的时候,他自然不敢把实话说出来。除此之外,在这种状态下做爱,有多少快乐,也真的很难说。 

小舅妈和小舅不是一头儿的。不是一头儿的人做爱也只能这样。在我家里和小舅妈做爱时,我舅舅盯着那个钢铁的小玩意,心里老在想:妈的,这种东西有没有保险机?保险机在哪里?到底什么样子保险才算是合上的?本来他可以提醒一下小舅妈,但他们认识不久,不好意思说。等到熟识以后才知道,那枪里没有子弹;可把我舅舅气坏了;他宁愿被枪走火打死,也不愿这样白耽心。不过,这支枪把他眼睛的毛病治好了。原来他是东一只眼西一只眼,盯枪口的时间太长,就纠正了过来。只可惜矫枉过正,成了斗鸡眼了。 

小舅妈把小舅搞成了斗鸡眼后,开头很得意,后来也后悔了。她在小报上登了一则求医广告,收到这样一个偏方:牛眼珠一对,水黄牛不限,但须原生于同一牛身上者。蜜渍后,留下一只,将另一只寄往南京。估计寄到时,服下留在北京的一只,赶往南京去服另一只。小舅妈想让小舅试试,但小舅一听要吃牛眼珠,就说:毋宁死。因为没服这个偏方,小舅的两只眼隔得还是那么近。但若小舅服了偏方,眼睛变得和死牛眼睛那样一南一北,又不知会是什么样子。 

第二天早上,我妈对小舅妈说:你有病,应该到医院去看看。这是指她做爱时快感如潮而言。小舅妈镇定如常地磕着瓜子说,要是病的话,这可是好病哇,治它干嘛?从这句话来看,小舅妈头脑清楚,逻辑完备。我看她不像有病的样子。说完了这些话,她又做出更加古怪的事:小舅妈站了起来,束上了武装带,拿出铐子,“飕”一下把我舅舅铐了起来;并且说:走,王犯,去劳改,别误了时辰。我舅舅耍起赖皮,想要再玩几天,但小舅妈横眉立目,说道:少费话!她还说,恋爱归恋爱,工作归工作,她立场站得很稳,决不和犯人同流合污──就这样把我舅舅押走了。这件事把我妈气得要发疯,后来她英年早逝,小舅妈要负责任。


4 

上个世纪渤海边上有个大碱厂,生产红三角牌纯碱,因而赫赫有名。现在经过芦台一带,还能看到海边有一大片灰蒙蒙的厂房。因为氨碱法耗电太多,电力又不足,碱厂已经停了工,所需的碱现在要从盐碱地上刨来。这项工作十分艰苦,好在还有一些犯了错误的人需要改造思想,可以让他们去干。除此之外,还需要有些没犯错误的人押送他们,这就是这个故事的前因。我舅舅现在还活着,会有什么样的后果还很难说。总而言之,我舅舅在盐碱地上刨碱,小舅妈押着他。刨碱的地方离芦台不很远。 

每次我路过芦台,都能看到碱厂青白的空壳子厂房。无数海鸟从门窗留下的大洞里飞进飞出,遮天盖地。废了的碱厂成了个大鸟窝,还有些剃秃瓢拴脚镣的人在窝里出入,带着铲子和手推车。这说明艰苦的工作不仅是刨碱,还有铲鸟粪。听说鸟粪除了做肥料,还能做食品的添加剂。当然,要经过加工,直接吃可不行。 

每次我到碱场去,都乘那辆蓝壳子交通车。“厂”和“场”只是一字之差,但不是一个地方。交通车开起来咚咚地响,还个细长的铁烟囱,驶在荒废的铁道上,一路崩崩地冒着黑烟。假如路上抛了锚,就要下来推;乘客在下面推车走,司机在车上修机器。运气不好时,要一直推到目的地。这一路上经过了很多荒废的车站,很多荒废了的道岔,所有的铁轨都生了锈。生了锈的铁很难看。那些车站的墙上写满了标语:“保护铁路一切设施”、“严厉打击盗窃铁路财产的行为”,等等,但是所有的门窗都被偷光,只剩下房屋的壳子,像些骷髅头。空房子里住着蝙蝠、野兔子,还有刺猬。刺猬灰溜溜的,长了两双罗圈腿。我对刺猬的生活很羡慕:它很闲散,在觅食,同时又在晒太阳,但不要遇上它的天敌黄鼠狼。去过一回碱场,袜子都会被铁锈染红,真不知铁锈是怎么进去的。 

我到碱场去看小舅时,心里总有点别扭。小舅妈和小舅是一对,不管我去看谁,都有点不正经。假如两个一齐看,就显得我很贱。假如两个都不看,那我去看谁?唯一能安慰我的是:我和我舅舅都是艺术家。艺术家外甥看艺术家舅舅,总可以罢。但这种说法有一个最大的问题,那就是我既不知什么是艺术,也不知什么是艺术家。在这种情况下,认定了我们舅甥二人全是艺术家,未免有点不能服人。 

碱场里有一条铁路,一直通到帐蓬中间。在那些帐蓬外面围着铁丝网,还有两座木头搭的了望塔。帐蓬之间有一片土场子,除了黄土,还有些石块,让人想起了冰川漂砾。正午时分,那些石头上闪着光。交通车一直开到场中。场子中央有个木头台子,乍看起来不知派什么用场。我舅舅一到了那里,人家就请他到台子前面躺下来,把腿伸到台子上,取出一副大脚镣,往他腿上钉。等到钉好以后,你就知道台子是派什么用场的了。脚镣的主要部份是一根好几十公斤重、好几米长的铁链子。我舅舅躺在地上,看着那条大铁链子,觉得有点小题大作,还觉得铁链子冰人,就说:报告管教!这又何必呢?我不就是画了两幅画吗?小舅妈说,你别急,我去打听一下。过了一会儿,她回来说:万分遗憾,王犯。没有再小的镣子了,你说自己只画了两幅画,这儿还有只写了一首诗的呢。听了这样的话,我舅舅再无话可说。后来人家又把我舅舅极为珍视的长发剃掉,刮了一个亮闪闪的头。有关这头长发,需要补充说,前面虽然秃了,后面还很茂盛,使我舅舅像个前清的遗老,看上去别有风韵;等到剃光了,他变得朴实无华。我舅舅在绝望中呼救道:管教!管教!他们在刮我!小舅妈答道:安静一点,王犯!不刮你,难道来刮我吗?我舅舅只好不言语了。以我舅舅的智慧,到了此时应该明白事情很不对劲。但到了这个地步,小舅也只有一件事可做:一口咬定他爱小舅妈。换了我也要这样,打死也不能改口。 

我舅舅在碱场劳改时,每天都要去砸碱。据他后来说,当时的情形是这样的:他穿了一件蓝大衣,里面填了再生毛,拖着那副大脚镣,肩上扛了十字镐,在白花花的碱滩上走。那地方的风很是厉害,太阳光也很厉害,假如不戴个墨镜,就会得雪盲,碱层和雪一样反光。如前所述,我舅舅没有墨镜,就闭着眼睛走。小舅妈跟在后面,身穿呢子制服,足蹬高统皮靴,腰束武装带,显得很是英勇。她把大檐帽的带子放下来,扣在下巴上。走了一阵子,她说:站住,王犯!这儿没人了,把脚镣开了罢。我舅舅蹲下去拧脚镣,并且说:报告管教,拧不动,螺丝锈住了!小舅妈说:笨蛋!我舅舅说:这能怪我吗?又是盐又是碱的。他的意思是说,又是盐又是碱,铁器很快就会锈。小舅妈说:往上撒尿,湿了好拧。我舅舅说他没有尿。其实他是有洁癖,不想拧尿湿的罗丝。小舅妈犹豫了一阵说:其实我倒有尿棗算了,往前走。我舅舅站起身来,扛住十字镐,接着走。在雪白的碱滩上,除了稀疏的枯黄芦苇什么都没有。走着走着小舅妈又叫我舅舅站住,她解下武装带挂在我舅舅脖子上,走向一丛芦苇,在那里蹲下来尿尿。然后他们又继续往前走,此时我舅舅不但扛着镐头,脖子上还有一条武装带、一支手枪、一根警棍,走起路来东歪西倒,完全是一副怪模样。后来,我舅舅找到了一片碱厚的地方,把蓝大衣脱掉铺在地上,把武装带放在旁边,就走开,挥动十字镐砸碱。小舅妈绕着他嘎吱嘎吱地走了很多圈,手里掂着那根警棍。然后她站住,从左边衣袋里掏出一条红丝巾,束在脖子上,从右衣袋里掏出一副墨镜戴上,走到蓝大衣旁边,脱掉所有的衣服,躺在蓝大衣上面,摊开白晰的身体,开始日光浴。 

过了不久,那个白晰的身体就变得红扑扑的了。与此同时,我舅舅迎着冷风,流着清水鼻涕,挥着十字镐,在砸碱。有时小舅妈懒洋洋地喊一声:王犯!他就扔下十字镐,希里哗啦地奔过去说:报告管教,犯人到。但小舅妈又没什么正经事,只是要他看看她。我舅舅就弓下腰去,流着清水鼻涕,在冷风里眯着眼,看了老半天。然后小舅妈问他怎么样,我舅舅拿袖子擦着鼻涕,用低沉的嗓音含混不清地说:好看,好看!小舅妈很是满意,就说:好啦,看够了吧?去干活吧。我舅舅又希里哗啦地走了回去,心里嘀咕道:什么叫“看够了吧”?又不是我要看的!这么奔来跑去,还不如带个望远镜哪。说到用望远镜看女人,我舅舅是有传统的。他家里有各种望远镜棗蔡司牌的、奥林巴司的,还有一架从前苏联买回来的炮队镜。他经常伏在镜前,一看就是半小时,那架式就像苏军元帅朱可夫。有人说,被人盯着看就会心惊胆战,六神无主。他家附近的女孩子经常走着走着犯起迷糊,一下撞上了电线杆;后来她们出门总打着阳伞,这样我舅舅从楼上就看不到了。现在小舅妈躺在那里让他看,又没打伞,他还不想看,真叫作身在福中不知福。 

我舅舅在碱场时垂头丧气,小舅妈却不是这样。她晒够了太阳,就穿上靴子站了起来,走进冷风,来到我舅舅身边说:王犯,你也去晒晒太阳,我来砸一会,说完就抢过十字镐抡了起来,而我舅舅则走到蓝大衣上躺下。这时假如有拉碱的拖拉机从远处驶过,上面的人就会对小舅妈发出叫喊,乱打唿哨。这是因为小舅妈除了脖子上系的红丝巾鼻梁上的墨镜和鸡皮疙瘩,浑身上下一无所有。碱场有好几台拖拉机,冒着黑烟在荒原上跑来跑去,就像十九世纪的火轮船。那个地方天蓝得发紫,风冷得像水,碱又白又亮,空气乾燥得使皮肤发涩。我舅舅闭上了眼睛,想要在太阳底下做个梦。失意的人总是喜欢做梦。他在碱场时三十八岁,四肢摊开地躺在碱地上睡着了。后来,小舅妈踢了他一脚说:起来,王犯!你这不叫晒太阳,叫作捂痱子。这是指我舅舅穿着衣服在太阳底下睡觉而言。考虑到当时是在户外,气温在零下,这种说法有不尽不实之处。小舅妈俯下身去,把他的裤子从腿上拽了下来,一直拽到脚镣上。 

假如说我舅舅有过身长八米的时刻,就指那一回。然后她又俯下身去,用暴烈的动作解开他破棉袄上的四个扣子,把衣襟敞开。我舅舅睁开眼睛,看到一个红彤彤的女人骑在他身上,颈上的红丝巾和头发就如野马的鬃毛一样飞扬。他又把眼睛闭上。这些动作虽有性的意味,但也可以看作管教对犯人的关心。要知道农场伙食不好,晒他一晒,可以补充维生素D,防止缺钙。做完了这件事,小舅妈离开了我舅舅的身体,在他身边坐下,从自己的制服口袋里掏出一盒香烟,取出一支放在嘴上,又拿出一个防风打火机,正要给自己点火,又改变了主意。她用手掌和打火机在我舅舅胸前一拍,说道:起来,王犯!一点规矩都不懂吗?我舅舅应声而起,偎依在她身边,给她点燃了香烟。以后小舅妈每次叼上烟,我舅舅伸手来要打火机,并且说:报告管教!我懂规矩啦!后来,我舅舅在碱滩上躺成一个大字,风把刨碎的碱屑吹过来,落在皮肤上,就如火花一样的烫。白色的碱末在他身体上消失了,变成一个个小红点。小舅妈把吸剩的半支烟插进他嘴里,他就接着吸起来。然后,她就爬到他身上和他做爱,头发和红丝巾一起飘动。而我小舅舅一吸一呼,鼻子嘴巴一起冒出烟来。后来他抬起头来往下面看去,并且说:报告管教!要不要戴套?小舅妈则说:你躺好了,少操这份心!他就躺下来,看天上一些零零散散的云。后来小舅妈在他脸上拍了一下,他又转回头来看小舅妈,并且说道:报告管教!你拍我干什么?我舅舅原来是个轻浮的人,经过碱场的生活之后就稳重了。这和故事发生的地点有一定的关系。那地方是一片大碱滩,碱滩的中间有个黑糊糊的凹地,用蛇形铁丝网围着,里面有几十个帐蓬,帐蓬中间有一条水沟,水沟的尽头是一排水管子。日暮时分,我舅舅和一群人混在一起刷饭盒。 

水管里流出的水带有碱性,所以饭盒也很好刷。在此之前,我舅舅和舅妈在帐蓬里吃饭。那个帐蓬是厚帆布做的,中间挂了一个电灯泡。小舅妈岔开双腿,雄踞在铺盖卷上抬头吃着饭,她的饭盒里是白米饭、白菜心,还有几片香肠。小舅双腿并拢,坐在一个马扎上低头吃饭,他的饭盒里是陈仓黄米、白菜帮子,没有香肠。小舅妈哼了一声:“哞”,我舅舅把碗递了过去。小舅妈把香肠给了他。我舅又把饭盒拿了回去,接着吃。此时小舅妈对他怒目而视,并且赶紧把自己嘴里的饭咽了下去,说道:王犯!连个谢谢也不说吗?我舅舅应声答道:是!谢谢!小舅妈又说:谢谢什么?我舅舅犹豫了一下,答道:谢谢大姐!小舅妈就沉吟起来,沉吟的原故是我舅舅比她大十五岁。等到饭都吃完,她才敲了一下饭盒说:王犯!我觉得你还是叫我管教比较好。我舅舅答应了一声,就拿了饭盒出去刷。小舅妈又沉吟了一阵,感觉非常之好,就开始捧腹大笑。她觉得我舅舅很逗,自己也很逗,这种生活非常之好。我舅舅觉得自己一点也不逗,小舅妈也不逗。这种生活非常的不好。尽管如此,他还是爱小舅妈,因为他别无选择啦。 

我舅舅的故事是这么结束的:他到水沟边刷好了碗回来,这时天已经黑了,并且起了风。我舅舅把两个饭盒都装在碗套里,挂在墙上,然后把门拴上。所谓的门,不过是个帆布帘子,边上有很多带子,可以系在帆布上。我舅舅把每个带子都系好,转过身来。他看到小舅妈的制服零七乱八地扔在地下,就把它们收起来,一一叠好,放在角落里的一块木板上,然后在帐蓬中间立正站好。此时小舅妈已经钻进了被窝,面朝里,就着一盏小台灯看书。过了一会儿,帐蓬中间的电灯闪了几下灭了,可小舅妈那盏灯还亮着,那盏灯是用电池的。小舅妈说:王犯,准备就寝。我舅舅把衣服都脱掉,包括脚镣。那东西白天锈住了,但我舅舅找到了一把小扳手,就是为卸脚镣用的。 

然后他精赤条条的立正站着,冷得发抖,整个帐蓬在风里东摇西晃。等到他鼻子里开始流鼻涕,才忍不住报告说:管教!我准备好了。小舅妈头也不回地说:准备好了就进来,废什么话!我舅舅蹑手蹑脚钻到被里去,钻到小舅妈身后,那帐蓬里只有一副铺盖。因为小舅妈什么都没穿,所以我舅舅一触到她,她就从牙缝里吸气。这使我舅舅尽量想离她远一点。但她说:贴紧点,笨蛋!最后,小舅妈终于看完了一段,折好了书页,关上灯,转过身来,把乳房小腹阴毛等等一齐对准我舅舅,说道:王犯,抱住我。你有什么要说的?我舅舅想,黑灯瞎火的,就乱说吧,免得她再把我铐进厕所,就说:管教,我爱你。她说:很好。还有呢?我舅舅就吻她。两个身体在黑暗里纠缠不休。小舅妈说起这些事来很是开心,但我听起来心事重重:在小舅妈的控制下,我舅舅还能不能出来,几时出来,等等,我都在操心。假如最终能出来,我舅舅学点规矩也不坏。但是小舅妈说:“不把他爱我这件事说清楚,他永辈子出不来。”

5 

现在可以这样说,小舅为作画吃官司,吃了一场冤枉官司。因为他的画没有人懂,所以被归入了叵测一类。前清有个诗人写道:“清风不识字,何事乱翻书”,让人觉得叵测,就被押往刑场,杀成了碎片。上世纪有个作家米兰·昆德拉说:人类一思考,上帝就发笑。这上帝就很叵测。我引昆德拉这句话,被领导听见了,他就说:一定要把该上帝批倒批臭。后来他说,他以为我在说一个姓尚的人。总而言之,我舅舅的罪状就是叵测,假如不叵测,他就没事了。 

在碱场里,小舅妈扣住了小舅不放,也都是因为小舅叵测之故。她告诉我说,她初次见到小舅,是在自己的数学课上。我舅舅测过了智商后就开始掉头发,而且他还没有发现有什么办法可以从这里早日出去,为这两件事,他心情很不好,脑后的毛都直着,像一只豪猪。上课时他两眼圆睁、咬牙切齿,经常把铅笔一口咬断,然后就把半截铅笔像吃糖棍一样吃了下去,然后用手擦擦嘴角上的铅渣,把整个嘴都抹成黑色的了。一节课发他七支铅笔,他都吃个精光。小舅妈见他的样子,觉得有点渗人,就时时提醒他道:王犯,你的执照可不是我吊销的,这么盯着我干嘛?我舅舅如梦方醒,站起来答道:对不起,管教。你很漂亮。我爱你。这后一句话是他顺嘴加上去的,此人一惯贫嘴聊舌,进了习艺所也改不了。我告诉小舅妈说:她是很漂亮。她说:是啊是啊。然后又笑起来:我漂亮,也轮不到他来说啊!后来她说,她虽然年轻,但已是老油子了。在习艺所里,学员说教员漂亮,肯定是没安好心。至于他说爱她,就是该打了。我没见过小舅妈亲手打过小舅,从他们俩的神情来看,大概是打过的。 

小舅妈还说,在习艺所里,常有些无聊的学员对她贫嘴聊舌。听了那些话她就揍他们一顿。但是小舅和他们不同,他和她有缘份。缘份的证明是小舅的画,她看了那些画,感到叵测,然后就性欲勃发。此时我们一家三口:舅舅、外甥和舅妈都在碱滩上。小舅妈趴在一块塑料布上晒日光浴,我舅舅衣着整齐,睡在地上像一具死尸,两只眼睛盯着自己的鼻子。小舅妈的裸体很美,但我不敢看,怕小舅吃醋。小舅的样子很可怕,我想安慰他几句,但又不敢,怕小舅妈说我们串供。我把自己扯到这样的处境里,想一想就觉得稀奇。 

小舅妈还说,她喜欢我舅舅的画。这些画习艺所里有一些,是李家口派出所转来的。搁在那里占地方,所里要把它丢进垃圾堆。小舅妈把它都要下来,放在宿舍里,到没人的时候拿出来看。小舅事发进碱场,小舅妈来押送,并非偶然。用句俗话来说,不怕贼偷,就怕贼惦记。小舅早就被舅妈惦记上了。这是我的结论,小舅妈的结论有所不同。她说:我们是艺术之神阿波罗做媒。说到这里,她捻了小舅一把,问道:艺术之神是阿波罗吧?小舅应声答道:不知道是谁。嗓音低沉,听上去好像死掉的表哥又活过来了。 

我常到碱场去,每次都要告诉小舅妈,我舅舅是爱她的。小舅妈听了以后,眼睛就会变成金黄色,应声说道:他爱我,这很好啊!而且还要狂笑不止。这就让我怀疑她是不是真的觉得很好。真觉得好不该像岔了气那样笑。换个女人,感觉好不好还无关紧要。小舅的小命根握在小舅妈手里,一定要让她感觉好。于是我就换了一种说法:假如小舅不是真爱你,你会觉得怎样?小舅妈就说:他不是真爱我?哪也很好啊!然后又哈哈大笑。我听着像在狞笑。在这个问题上我们进退两难,就该试试别的门道。 

那次我去看小舅,带去了各种剪报──那个日本人把他的画运到巴黎去办画展,引起了很大的轰动。这个画展叫作“2010──W2”,没有透露作者的身份,这也是轰动的原因之一。各报一致认为,这批画的视觉效果惊人,至于说是伟大的作品,这么说的人还很少。展览会入口处,摆了一幅状似疯驴的画,就是平衡器官健全的人假如连看五秒钟也会头晕;可巧有个观众有美尼尔综合征,看了以后,马上觉得天地向右旋转,与此同时,他向左倾倒,用千斤顶都支不住。后来只好给他看另一幅状似疯马的画,他又觉得天地在向左旋转,但倒站直了。然后他就向后转,回家去,整整三天只敢喝点冰水,一点东西也没吃。大厅正中有幅画,所有的人看了都感到“嗡”地一声,全身的血都往头上涌。不管男女老幼,大家的头发都会直立起来,要是梳板寸的男人倒也无碍,那些长发披肩的金发美女立时变得像带尖顶帽的小丑。与此同时,观众眼睛上翻,三面露白,有位动脉硬化者立刻中了风。还有一幅画让人看了感觉五脏六腑往下坠,身材挺拔的小伙子都驼了背,疝气患者坠得裤裆里像有一个暖水袋。大家对这位叫作“W2”的作者有种种猜测,但有些宗教领袖已经判定他是渎神者,魔鬼的同谋,下了决杀令。他们杀了一些威廉、威廉姆斯、韦伯、威利斯,现在正杀世界卫生组织(WHO)里会画画的人,并杀得西点军校改了名,但还没人想到要杀姓王的中国人。我们姓王的有一亿人,相当于一个大国,谅他们也得罪不起。我把这些剪报给小舅妈看,意在证明小舅是伟大的艺术家,让她好好地对待他。小舅妈就说:伟大!伟大!不伟大能犯在我手里吗?后来临走时,小舅抽冷子踢了我一脚。他用这种方式通知我:对小舅妈宣扬他的伟大之处,对他本人并无好处。这是他最后一次踢我,以后他就病秧秧的,踢不动了。 

当在我沉迷于思索怎样救小舅时,他在碱场里日渐憔悴,而且变得尖嘴猴腮。小舅妈也很焦急,让我从城里带些罐头来,特别指定要五公斤装的午餐肉,我用塑料网兜盛住挂在脖子上,一边一个,样子很傻。坐在去碱场的交通车里,有人说我是猪八戒挎腰刀,邋遢兵一个。这种罐头是餐馆里用的,切成小片来配冷盘,如果大块吃,因为很油腻,就难以下咽。小舅妈在帐蓬里开罐头时,小舅躺在一边,开始乾呕。然后她舀起一块来,塞到小舅嘴里,立刻把勺子扔掉,一手按住小舅的嘴,另一手掐着他的脖子,盯住了他的眼睛说:一、二、三!往下咽!塞完了小舅,小舅妈满头大汗,一面擦手,一面对我说:小子,去打听一下,哪儿有卖填鸭子的机器。此时小舅嘴唇都被捏肿,和鸭子真的很像了在碱场里吃得不好,心情又抑闷,小舅患上了阳痿症。不过小舅妈自有她的办法。 

我舅舅的这些逸事是他自己羞羞答答地讲出来的,但小舅妈也有很多补充:在碱滩上躺着时,他的那话儿软塌塌地倒着,像个蒸熟的小芋头。你必须对它喊一声:立正!它才会立起来,像草原上的旱獭,伸头向四下张望。当然,你是不会喊的,除非你是小舅妈。这东西很听指挥,不但能听懂立正、稍息,还能向左右转,齐步走等等。在响应口令方面,我舅舅是有毛病的,他左右不分,叫他向左转,他准转到右面;齐步走时会拉顺。而这些毛病它一样都没有。小舅妈讲起这件事就笑,说它比我舅舅智商高。假如我舅舅IQ50,它就有150,是我舅舅的三倍。作为一个生殖器,这个数字实属难能可贵。小舅妈教它数学,但它还没学会,到现在为止,只知道听到一加一点两下头,但小舅妈对它的数学才能很有信心。她决心教会它微积分。这门学问她一直在教小舅,但他没有学会。她还详细地描写了立正令下后,那东西怎样蹒跚起身,从一个问号变成惊叹号,颜色从灰暗变到赤红发亮,像个美国出产的苹果。她说,作为一个女人,看到这个景象就会觉得触目惊心。但我以为男人看到这种景象也会触目惊心。 

小舅妈还说:到底是艺术家,连家伙都与众不同──别的男人肯定没有这种本领。我舅舅听到这里就会面红耳赤,说道:报告管教!请不要羞辱我!士可杀不可辱!而小舅妈却耸耸肩,轻描淡写地说:别瞎扯!我杀你干嘛。来,亲一下。此后小舅只好收起他的满腔怒火,去吻小舅妈。吻完以后,他就把自己受羞辱的事忘了。照我看来,小舅不再有往日的锐气,变得有点二皮脸,起码在舅妈面前是这样的。据说,假如小舅妈对舅舅大喝一声立正!我舅舅总要傻呵呵地问:谁立正?小舅妈说:稍息!我舅舅也要问谁稍息。在帐蓬里,小舅妈会低声说道:同志,你走错了路……我舅舅就会一愣,反问道:是说我吗?我犯什么错误了吗?小舅妈就骂道,人说话,狗搭茬!有时候她和我舅舅说话,他又不理,需要在脸上拍一把才有反应:对不起,管教!不知道你在和我说话。讨厌的是,我舅舅和他的那个东西都叫作王二。小舅妈也觉得有点混乱,就说:你们两个简直是要气死我。久而久之,我舅舅也不知自己是几个了。 

我舅舅和小舅妈在碱场里陷入了僵局,当时我以为有两个原因:其一是小舅妈不懂得艺术;所以她就知道拿艺术家寻开心。假如我懂得什么是艺术,能用三言两语对她解释清楚,她就会把小舅放出来。但我没有这个能耐。所以小舅也出不来。 

刚上大学时,我老在想什么是艺术的真谛,想着想着就忘了东西南北,所以就有人看到我在操场上绕圈子,他在一边给我数圈数,数着数着就乱了,只好走开;想着想着,我又忘掉了日出日落,所以就有人看到我在半夜里坐在房顶上抽烟,把烟蒂一个一个地往下扔;这件事的不可思议之处在于我有恐高症。因为这个缘故,有些女孩子爱上了我,还说我像维特根斯坦,但我总说:维特根斯坦算什么。听了这话,她们就更爱我了。但我忙于解开这个难题,一个女孩都没爱上,听任她们一个个从我身边飞走了,现在想起来未免后悔,因为在她们中间,有一些人很聪明,有一些人很漂亮;还有一些既聪明,又漂亮,那就更为难得。所谓艺术的真谛,就是人为什么要画画、写诗、写小说。我想作艺术家,所以就要把这件事先想想清楚。不幸的是,到了今天我也没有想清楚。 

现在我还在怀念上大学一年级的时期,那时候我写着一篇物理论文;还在准备投考历史系的研究生;时时去看望我舅舅;不断思辨艺术的真谛;参加京城里所有新潮思想的讨论会;还忙里偷闲,去追求生物系一个皮肤白晰的姑娘。盛夏时节,她把长发束成了马尾辫,穿着白色的T恤衫和一条有纵条纹的裙裤,脖子和耳后总有一些细碎的汗珠。我在校园里遇上她,就邀她到松树林里去坐。等到她在乾松针上细心地铺好手绢,坐在上面,脱下脚上的皮凉鞋,再把脚上穿的短丝袜脱下来放在两边时,我已经开始心不在焉,需要提醒,才能开始在她领口上的皮肤上寻找那种酸酸的汗味。 

据说,我的鼻子冬暖夏凉,很是可爱;所以她也不反对撩起马尾辫,让我嗅嗅项后发际的软发。从这个方向嗅起来,这个女孩整个就像一块乳酪。可惜的是,我经常想起还有别的事情要干,就匆匆收起鼻子来走了。我记得有一回,我在她乳下嗅到一股沉掂掂的半球形的味道,还没来得及仔细分辨,忽然想起要赶去看我舅舅的交通车;就这样走掉了。等下次见到她时,她露出一副要哭的样子,用手里端着的东西泼了我一脸。那些东西是半份炒蒜苗、半份烩豆腐,还有二两米饭。蒜苗的火候太过,变得软塌塌的。豆腐里放了变质的五香粉,有点发苦。至于米饭,是在不锈钢的托盘里蒸成,然后再切成四方块。我最反对这样来做米饭。经过这件事以后,我认为她的脾气太坏,还有别的缺点,从此以后不再想念她了;只是偶而想到:她可能还在想念我。 

在碱滩上,我想营救小舅时,忽然想到,艺术的真谛就是叵测。不过这个答案和没有差不多。世界上没有人知道什么是“叵测”,假如有人知道,它就不是叵测。 

我舅舅陷在碱场里的另一个原因是他不擅长爱情。假如他长于此道,就能让小舅妈把他放出来。在我看来,爱情似乎是种竞技体育;有人在十秒钟里能跑一百米,有人需要二十秒钟才能跑完一百米。和小舅同时进习艺所的人,有人已经出来了,挎着习艺所的前教员逛大街;看来是比小舅长于此道。竞技体育的诀窍在于练习。我开始练习这件事,不是为了救我舅舅,而是为了将来救我自己。 

最近,我在同学聚会时遇到一个女人,她说她记得我,并对这些记忆做了一番诗意的描绘。首先,她记得世纪初那些风,风里夹杂着很多的黄土。在这些黄土的下面,树叶就份外的绿。在黄土和绿叶之间,有一个男孩子,裹在一身灰土色的灯芯绒里,病病歪歪地穿过了操场──此人大概就是我罢──在大学期间我没生过病,不知她为什么要说我病歪歪。但由她所述的情形来看,那就是在我去碱场之前的事。 

这个女人是我们的同行,现在住在海外;闻起来就如开了瓶的冰醋酸,简直是颗酸味的炸弹。在她诗意的回忆里,那些黄沙漫天的日子里,最值得记忆的是那些青翠欲滴的绿叶;这些叶子是性的象征。然后她又说到一间小屋子,一个窗户。这个窗户和一个表达式联系在一起──这个表达式是2x2,说明这窗户上有四片玻璃,而且是正方形的──被一块有黑红两色图案的布罩住,风把这块印花布鼓成了一块大气包。气包的下面是一张皱巴巴的窄床;上面铺了一条蓝色腊染布的单子。她自己裸体躺在那张单子上,竭力伸展身躯,换言之,让头部和脚尖的距离尽可能的远;于是腹部就深凹下去,与床单齐。这时候,在她的腿上,闪着灰色的光泽。在这个怪诞的景象中,充满了一种气味,带有碱性的腥味;换言之,新鲜精液的气味。假如说这股气味和我有什么关系,我实在感到意外。但那间房子就是我上大二时的宿舍,里面只住了我一个人。至于说我在里面干了什么,我一点都记不得。这个女人涂了很重的眼晕,把头发染成了龌龊的黄色,现在大概有三百磅。要把她和我过去认识的任何一个女孩联系起来,很是困难。然而人家既知道我的房间,又知道我的气味,对这件事我也不能否认。她还说,当时我一声不响,脸皮紧绷,好像心事重重──忽然间精液狂喷,热烘烘的好像尿了一样。因为我是这样的一个心不在焉的尿炕者,她一直在想念我。但我不记得自己是这样的爱尿炕;而且,如果说这就是爱情,我一定要予以否认。 

在学校里,有一阵子我像疯了一样的选课,一学期选了二十门。这么多课听不过来,我请同学带台对讲机去,自己坐在宿舍里,用不同的耳机监听。我那间房子里像电话交换台一样,而我自己脸色青里透白。系里的老师怀疑我吸海洛因,抓我去验血。等到知道了我没有毒瘾后,就劝诫我说:何必急着毕业?重要的是做个好学生。但我忙着到处去考试,然后又忙着到处去补考。补到最后一门医用拉丁文,教授看我像个死人,连问都没问,就放我Pass了。然后我就一头栽倒,进了校医院。我之所以这样的疯狂,是因为一想到小舅的处境,就如有百爪挠心,方寸大乱。 

在寒假里,我听说化学系有个女生修了二十一门课,比我还要多一门。我因此爱上了她,每天在女生宿舍门口等她,手里拿了一束花。这是一个小四眼,眼镜的度数极深,在镜片后面,眼睛极大,并且盘旋着两条阿基米德螺线。她脸色苍白,身材瘦小,双手像鸟爪子,还有点驼背。后来才发现,她的乳房紧贴着胸壁,只是一对乳头而已,而且好像还没有我的大;肩膀和我十三岁时一样单薄。总而言之,肚脐以上和膝盖以下,她完全是个男孩子,对男女之间的事有种学究式的兴趣,总问:为什么是这样呢?我告诉她说:我爱她,这辈子再也不想爱别人。她扶扶眼镜说:为什么你要爱我?为什么这辈子不想爱别人?我无言以对,就提议做爱来证明这一点。但正如她事后所说,做爱并不能解决这个问题。假如我真的爱她,就该是无缘无故的。但无缘无故的事总让人怀疑。由此得出一个结论,不管谁说爱她都可疑。经她这样一说,我觉得自己并不爱她。她听了扶扶眼镜说:为什么你又不爱我了呢?我听了又不假思索地马上又爱上了她。我和她的感情就这样拉起锯来。又过了一个学期,她猛然开始发育,还配了隐形眼镜,就此变成个婷婷玉立的美女,而且变得极傻。此时她有不少追求者,我对她也没了兴趣。

6 

那一回和小舅、小舅妈在碱摊上晒太阳,直到天色向晚。天色向晚时,小舅妈站起身来,往四下看看。夕阳照在她的身体上,红白两色,她好像一个女神。如果详加描写,应该说到,她的肩头像镜子一样反光,胸前留下了乳房的阴影。在平坦的小腹上,有一蓬毛,像个松鼠尾巴──我怀疑身为外甥这样描写舅妈是不对的──然后她躬下身来穿裤子,我也该回学校了。这是我唯一一次看到小舅妈的裸体,以后再也没机会。早知如此,当初真该好好看看。 

说过了小舅妈,就该说到小舅。小舅的案子后来平了反,法院宣布他无罪,习艺所宣布他是个好学员。油画协会恢复他的会员资格,重新发给他执照,还想选他当美协的理事。谁知小舅不去领执照,也不想入油协。于是有关部门决定以给脸不要脸的罪名开除小舅,吊销他的画家执照。但是小舅妈不同意他们这样干,要和他们打官司,理由是小舅既然没有重入美协,也没有去领执照,如何谈得上开除和吊销。但是小舅妈败诉了。法院判决说,油画协会作为美术界的权力机关,可以开除一切人的会员资格,也可以吊销一切人的画家执照,不管他是不是会员,是不是画家。判决以后,美协开会,郑重开除了小舅妈。从此之后,她写字还可以,画画就犯法了。现在小舅没有执照,小舅妈也没有照。但是小舅继续作画,卖给那个日本人。但是价钱比以前低了不少。日本人说,现在世界经济不够景气,画不好脱手。其实这是一句假话。真话是小舅名声不如以前──他有点过气了。 

说过了我舅舅以后,也就该说到我舅舅画的日本人──此人老了很多,长了一嘴白胡子茬──在十字路口等红灯,他会大模大样地从人行横道上走过来,拉开车门说:王样,画!就把画取走了。顺便说一句,我大舅叫王大,我小舅叫王二。我妈那么厉害,我自己想不姓王也不行。这些画是我舅舅放在我这里的。假如红灯时间长,他还要和我聊几句,他说他想念我舅舅,很想见到他。我骗他说,我舅舅出家当了尼姑,要守清规,不能出来,你不要想他了;他纠正我说:和尚,你是说,和尚!然后替我关上车门,朝我鞠上一躬,就走了。其实他也知道我在撒谎。假如他和我舅舅没有联系,能找到我吗?反过来说,我也知道那个日本人在说谎。我们大家都在说谎,谁都不信任谁。 

有人说,这个日本人其实是个巴西人,巴西那地方日裔很多。他有个黑人老婆,像墨一样黑,有一次带到中国来,穿着绿旗袍和他在街上遛弯,就在这时发生了误会,人家把她当小舅逮去了。在派出所里,他们拿毛巾蘸了水、汽油、丙酮,使劲地擦,没有擦下黑油彩,倒把血擦出来了。等到巴西使馆的人闻讯赶来时,派出所换了一个牌子,改成了保育站,所有的警察都穿上了白大褂,假装在给黑女人洗脸。那女人身高1米98,像根电线杆,说是走失的小孩子勉强了一点。那日本人又有个白人情妇,像雪一样白。有一次和他在街上走,又发生了误会。人家把她逮进去,第一句话就问:好啊,王二,装得倒像!用多少漂白粉漂的?然后就去捏她的鼻子,看是不是石膏贴的,捏得人家泪下如雨;并且乱拔她的头发,怀疑这是个头套,一头金发很快就像马蜂窝一样了。等到使馆的人赶来,那派出所又换了一块牌子,“美容院”。但把鼻子捏得像酒渣鼻、把头发揪成水雷来美容,也有点怪。后来所有的外国女人和这日本人一起上街前,都在身上挂个牌子,上书“我不是王二”。 

还有一天他们逮住了我,一把揪住我的领带,把我拽得离了地,兴高彩烈地说:好啊王二!你居然连装都不装了!我很沉着地说道:大叔啊,你搞错了。我不是王二。我是王二的外甥。他愣住,把我放下地来,先是啐了一口,啐在我的皮鞋上;想了一会儿,又给我整整领带,擦擦皮鞋,朝我敬了一个礼,然后假装走开了。其实他没有走开,而是偷偷地跟着我,每隔十几分钟就猛冲到我面前,号我的脉搏,看我慌不慌。我始终不慌,他也没敢再揪我。幸亏他没把我揪到派出所,假如揪了去,我们单位的人来找时,他们又得换块牌子:柔道馆。之所以发生这些事,是因为他们知道我舅舅还在偷偷卖画,很想把他逮住,但总也逮不到他。这一点无关紧要。重要的是他揪我时,我感到很兴奋,甚至勃起了。这说明我有小舅的特徵。我是有艺术家的天赋,这大概是没有疑问的了。 

现在我提到了所有的人,就剩下我了。小时候我的志向是要当艺术家,等到看过小舅的遭遇之后,我就变了主意,开始尝试别的选择,其中包括看守公厕。我看守的的那座公厕是个墨绿色的建筑,看上去是琉璃砖砌的,实际上是水泥铸造的,表面上贴了一层不干胶的贴面纸,来混充琉璃。下一场大雨它就会片片剥落,像一只得了皮肤病的乌龟。房子里面有很多窄长的镜子,朝镜子里看时,感觉好像是在笼子里。房间里有一股苦杏仁味,那是一种消毒水。我在门口分发手纸,每隔一段时间,就用消防水龙冲洗一次里面,把坐在马桶上的人冲得像落汤鸡。还有一件事我总不会忘记,就是索要小费,如果顾客忘了给,我就揪住他衣服不放,连他的衣兜都扯掉。闹到了这个地步,也就没人敢再不给小费。因为工作过于积极,我很快就被开除掉。 

还有一段时间,我在火车站门前摆摊,修手表、打火机。像所有的修表摊一样,我的那个摊子是座玻璃匣子,可以推着走因为温室效应,坐在里面很热,汗出得很多,然后就想喝水。经我修过的手表就不能看时间,只能用来点烟;我修过的打火机倒有报时的功能,但又打不着火了,顾客对我不大满意。还有一段时间我戴着黑眼镜,假装是瞎子,在街上卖唱。但很少有人施舍。作为一个瞎子,我的衣服还不够脏。他们还说我唱得太难听,可以催小孩子的尿。后来我又当过看小孩子的保姆,唱歌给小孩子听,他们听了反而尿不出;见到雇主回家,就说:妈妈,叔叔唱!然后放声大哭。 

我做过各种各样的职业,拖延了很多时间,来逃避我的命运。我终于长大了,在写作部里工作;我舅舅也从碱场出来了,和小舅妈结了婚。他还当他的画家。小舅妈倒是改了行,在一家大公司里当公关秘书。这说明我舅舅除了画画,我除了会信口胡编,都别无所长,小舅妈倒是多才多艺。有时候她深更半夜给我打电话,说我舅舅的坏话。说他就知道神秘兮兮捣鬼,江郎才尽,再也画不出令人头晕的画了;还说他身体的那一部份功能还是老样子,她每天要给它发号令,还要假装很喜欢的样子,真是烦死了。这些话的意思好像是说,她嫁给小舅嫁亏了。但是每次通话结束时,她总要加上一句,这些话不准告诉你舅舅。只要你敢透半句口风,我就杀掉你!至于我,每天都在写小说。说句实在话,我不知道自己写的到底是什么。 

今天我们所面对的一切,都是我一手促成的。那一天我从碱场回来,心情烦闷,就去捣鼓电脑,想从交互网上找个游戏来玩。找来找去,没找到游戏,倒找到一份电子杂志,《今日物理》。我虽是物理系的学生,但绝不看物理方面的文献──教科书例外。那天又找到了一个例外,就是那本杂志。它的通栏标题是:谁是达利以后最伟大的画家──W2还是486?W2是我舅舅的化名,486是上世纪末一种个人电脑,已经完全过时,一块钱能买五六台。那篇文章还有张插图,上面有台486微机,屏幕上显示着我舅舅那幅让人犯疝气的画。当然,它已是画中画,看上去就不犯疝气,只使人有点想屙屎。 

等你把这篇文章看完,连屎都不想屙。它提到上个世纪末开始,有人开始研究从无序到有序的物理过程,这种东西又叫作“混沌”,用计算机模拟出来,显示在屏幕上很好看。其中最有名的是曼德勃罗集,放大了像海马尾巴,我想大家都是知道的。顺便说一句,曼德勃罗集不会使人头晕,和小舅的画没有一点相似之处。但是该文作者发明了一种名为依呀阿拉的算法,用老掉牙的486作图,让人看了以后晕得更加厉害。简单地说,用一行公式加上比一盒火柴还便宜的破烂电脑,就能作出小舅的画。任何人知道了这件事,看小舅的画就不会头晕,也不会犯疝气。很显然,小舅妈知道了这件事后再看小舅的画,也不会性欲勃发。这篇文章使我对小舅、小舅妈、艺术、爱情,还有整个世界产生了一种感觉,那就叫“掰开屁眼放屁,没了劲了”。假如我不到交互网上找游戏,一切就会是老样子,小舅照样是那么叵测,小舅妈还对他着迷。我也老大不小的啦,怎么还玩游戏呢?我看了这篇文章以后,犹豫了好久,终于下定了决心,把它打印了一百份,附上一封要求给小舅平反的信,寄往一切有关部门──不管怎么说,我舅舅在受苦,我不能不救他呀。有关部门马上作出了反应:小舅不是居心叵测,他画的是依呀啊拉集嘛,关他干嘛──放出来吧。有了这句话,我就驰往碱场,把一切都告诉小舅和小舅妈。 

小舅妈听了长叹一声,说道:原来是这样!对不起,王犯,让你吃了不少苦。回所给你要点补助吧。你也不用犟着说你爱我了。小舅听了我的话,变得像个死人,瘫软在地上。听到小舅妈最后一句话,他倒来了精神,从地上爬起来说:报告管教!我真的爱你!我从来没想利用你!等等。小舅妈听了,眼睛变成金黄色,对我狞笑着说:你听到了吧?咱俩快把这个死要面子活受罪的家伙揍上一顿!但还没等动手,她又变了主意,长叹一声道:算了。别打了。看来他是真的爱上我了。这似乎是说,假如小舅继续叵测,他就不可能真的爱上小舅妈,为此要狠狠地揍他,但和他做爱也非常的过瘾;假如他不再叵测,就可以爱上小舅妈,此后就不能打他,但和他做爱也是很烦人的了。小舅妈和小舅从碱场出去,结婚、过日子,一切都变得平淡无奇了。 今年是2015年,我是一个作家。我还在思考艺术的真谛。它到底是什么呢。
