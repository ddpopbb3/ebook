\chapter{三十而立}

王二生在北京城,我就是王二。夏天的早上,我骑车子去上班,经过学校门口时,看着学校庄严的大门,看着宽阔的操场和操场后面高耸的烟囱,我忽然觉得:无论如何,我也不能相信。 
 
 仿佛在不久之前,我还是初一的学生。放学时在校门口和同学们打书包仗。我的书包打在人身上一声闷响,把人家摔出一米多远。原来我的书包里不光有书,还有一整块板砖。那时节全班动了公愤,呐喊一声在我背后追赶。我奔过操场,逃向那根灰色的烟囱。后来校长出来走动,只见我高高爬在脚手梯上,迎着万里东风,敞开年轻的胸怀,高叫着:×你妈!谁敢上来我就一脚踹他下去!这好像是刚刚发生的事情。 
 
 转眼之间我就长大了很多,身高一米九十,体重八十多公斤。无论如何,一帮初一的男孩子不能把这样一条大汉撵得爬上烟囱,所以我绝不相信。 
 
 不知不觉我从自行车上下来,推车立在路旁。学校里静悄悄好像一个人也没有,这叫我心头一凛。多少次我在静悄悄的时候到校,穿过静悄悄的走廊,来到热悉的教室,推开门时几十张脸一齐转向我——我总是迟到。假如教室里有表扬批评的黑板报,批评一栏里我总是赫然有名。下课以后班长、班干部、中队长、小队长争先恐后来找我谈话,然后再去向班主任、辅导员表功。像拾金不昧、帮助盲人老大爷回家之类的好事不是每天都能碰到,而我是一个稳定的好事来源。只要找我谈谈话,一件好事就已诞生:“帮助了后进生王二!”我能够健康地成长,没有杀死校长老师,没有放火和在教室里撒尿,全是这些帮助的功劳。 
 
 二十年前谁都不会相信——校长不相信,教师不相信,同学们不相信,我自己也不相信,王二能够赶前四十分钟到校,但是这件事已经发生。如今王二是一名大学教师,在上实验课之前先到实验室看看。按说实验课有实验员许由负责,但是我对他不放心。 
 
 如今轮到我为别人操心,这真叫人难以置信。我和许由有三十年的交情,我们在幼儿园里合谋毒杀阿姨,好像是昨天发生的事情。我清清楚楚地记得自己在大班里凶悍异常,把小朋友都打通。我还记得阿姨揪住我的耳朵把它们朝刘备的方向改造。我永远也不会忘记那天午睡过后,阿姨带我们去大便。所有的孩子排成长龙,蹲在九曲十八回的长沟上排粪,阿姨躲在玻璃门外监视。她应该在大家屙完之后回来给大家擦屁股,可是那天她打毛衣出了神,我们蹲得简直要把肠子全屙出来,她也不闻不问。那个气味也真不好闻。我站起来,自己拿手纸擦了屁股,穿上裤子,然后又给别人接屁股。全班小朋友排成一排,由我排头擦去,真有说不出的得意。有多少今日的窈窕淑女,竟被我捷足先登,光顾了屁股,真是罪过!忽然间阿姨揪住了耳朵,她把我尽情羞辱了一番。 
 
 我气得鼓鼓的。星期天回家以后,我带了一瓶家里洗桃子的高锰酸钾水来。我妈说这种药水有毒,我想拿它毒死阿姨。吾友许由见了我的红色药水,问清用途,深表赞同。他还有一秘方可以加强药力,那就是石灰,许由抓住什么都往下吞,有一回吞石灰,被叔叔掐住了脖子,说石灰能把肠子烧穿。后来我们又在药水里加入了脚丫泥、尿、癞蛤蟆背上的浆汁等等,以致药水变得五彩缤纷。后来这瓶药水没来得及撒入阿姨的饭盒,就已被人揭发,这就是轰动幼儿园的王二毒杀案。根据以上事实,无论如何我也不能相信,如果不是为了毒死校长,我能为一个实验如此操心。 
 
 事实如此,不论我信与不信。八三年七月初的某个早上,我从本质上已经是个好人、好教师、好公民、好丈夫。事实证明,社会是个大熔炉,可以改造各种各样的人,甚至王二。现在我不但是某大学农业系的微生物讲师,还兼着基础部生物室的主任。我不但要管好自己,还要管好别人(如“后进生许由”之流,因为这家伙是我在校长那儿拍了胸脯才调进来的)。所以我在车棚里放下车子,就往实验室狂奔。推开门一看,果然不出我之所料。实验台上放着一锅剩面条,地上横七竖八几个啤酒瓶子。上回校长到(实验)室视察,看见实验台上放着吃剩的香肠,问我“这是什么?”我说是实验样品。他咆哮起来:“什么实验?造大粪的实验!”叫我心里好一阵发麻。我把这些东西收拾了,又闻见一股很奇怪的味:又像死猫死狗,又像是什么东西发了酵。找了半天,没找到味源。赶紧到里屋把许由揪起来。他睡眼惶松地说;“王二,你干什么?正梦见找到老婆……”“呸!七点四十了。快起来!我问你,屋里什么味?” 
 
 “别打岔。我这个梦非比一般,比哪回梦见的都好看。正要……” 
 
 我一把揪住他耳朵:“我问你,屋里什么东西这么臭?” 
 
 “这有什么可大惊小怪的?死耗子呗。我下了耗子药。” 
 
 “不是那种味!是你身上的味!” 
 
 “我哪知道。”他坐起来。这个东西就是这么不要脸,光屁股睡觉。“嘿,我鞋呢?王二,别开这种玩笑!” 
 
 “你死了吧!谁给你看着鞋!” 
 
 “呀!王二,我想起来了。我把球鞋放到烘箱里烤,忘了拿出来!” 
 
 我冲到烤箱前,打开门——我主!几乎熏死。急忙打开通风机,戴上防毒面具,套上胶皮手套,把他的臭球鞋用报纸包起来,扔进了厕所。回来一看,上午的实验许由根本就没准备,再过十五分钟学生就要来了,桌面上光秃秃的。我翻箱倒柜,把各种器具往外拿,折腾得汗都下来了。回头一看许由,这家伙穿着工作服,消消停停坐在显微镜前,全神贯注地往里看。见了这副景象,我不禁心头火起,大吼一声: 
 
 “许由!我要用胶布。给我上医务室拿点来。” 
 
 “不要慌。等一会儿。” 
 
 “什么时候了?火燎雀子毛了!快去!” 
 
 “别急。我还要穿几件衣服。” 
 
 “你穿得够整齐了。” 
 
 他风度翩翩地一撩衣服下摆。天,怎么不使雷劈了他!这家伙还光着屁股。他连做几个芭蕾动作,把三大件舞得像钟摆一样,进屋去穿衣服。过一会儿又舞出来,上医务室了。我把实验准备好,他还没回来,这不要紧,他不能死在那儿。擦擦汗,掸去身上的土,我又恢复了常态。学生还得一会儿来,我先看看许由刚才看什么。 
 
 显微镜里白花花的,满视野全是活的微生物,细长细长,像一盒活大头针。这是什么?许由能搞来什么稀罕玩艺?我要叫它难住,枉自教了微生物。这东西很眼熟,可就是想不起来了。 
 
 忽然许由揪住了我的后领,“王二,你是科班出身,说说这是什么?” 
 
 “胶布拿来了?每个实验台分一块。” 
 
 “别想混过去。你说!说呀!” 
 
 我直起身来,无可奈何地收起部主任的面孔,换上王二的嘴脸朝他奸笑一声。 
 
 “你以为能难倒我?我查查书,马上就能告诉你。可是你呀,连革兰氏染色都不会。” 
 
 “是是是。我承认你学问大。你今年还发过两篇论文,对不对?这些暂且不提。你就说说这镜下是什么?” 
 
 “我对你说实话,不知道。一时忘了,提笔忘字,常有的事。” 
 
 “这个态度是好的。告诉你吧,这是我的……” 
 
 我心里“格登”一声,往显微镜里一看——可不是吗,他的精虫像大尾巴蛆一样爬。“你把它收拾了!快!” 
 
 “别这么假正经!我还不知你是谁吗?” 
 
 “小声点,学生来了,看见这东西,我们就完了!” 
 
 “完什么?完不了。让他们看看人的精液,也长长见识。” 
 
 “他们要问,哪儿来的这东西?大天白日的,这儿又不是医院的门诊!怎么回答?” 
 
 “当然是你的了。你为科学,拿自己做了贡献,这种精神与自愿献血同等高尚。学校该给你营养补助。像你这种结了婚,入不敷出的同志能做到这一步,尤为难能可贵。” 
 
 我正急了眼要骂,学生来了,几个女孩子走过来说:“王老师早。你干什么呢?” 
 
 “早。都到自己实验台上去,看看短不短东西。缺东西向许老师要。” 
 
 “老师,你看什么片子?我们也看看!” 
 
 我赶紧俯身占住镜筒,可是这帮学生很赖皮。有人硬拿脸来挤我,长头发灌了我一脖子。大有伤风化! 
 
 我只好让开。这帮丫头就围上去,一边看一边叽叽喳喳:“活的哎!”“还爬呢!”“老师,这是什么呀?” 
 
 “噢,这是我的工作,不干你事。回位子去。” 
 
 “我们想知道!我们一定要知道!” 
 
 我叫起来:“班长!科代表!都上哪儿去了,谁不回位子,这节课我给你们零分!” 
 
 “老师,你怎么啦?”“吔!装个老头样。”“告诉一下何妨?” 
 
 “跟你们女孩子说这个不妥。还要听?好,告诉你们。这是荷兰进口的种猪精液。我要看看精子活力如何。” 
 
 这节课上得我头都大了。百分之七十的时间在回答有关配种的问题,女生兴趣尤大。她们从人工授精问到人造母猪的构造,净是我不了然的问题,弄得我火气越来越大。快下课时,校长进来,狠狠白了我一眼,还叫我下课去一下。 
 
 我去见校长,在校长室门口转了几圈才进去。不瞒你说,一见到师长之类的人物,就会激发我灵魂深处的劣根性,使我不像个好人。我进门时,校长正在浇花,他转过身来装个笑脸:“小王,你看我的花怎么样?” 
 
 “报告校长,这是蔷薇科蔷薇属,学名不知道。因为放在别的地方不长,只在驴棚里长,老百姓叫它毛驴花。” 
 
 “那么我就是毛驴了?你的嘴真无可救药。坐,近来工作如何?” 
 
 “报告,进展顺利。学生上实验课闹的事,已和他们班主任谈过,叫他做工作,再不行打电话叫刑警。许由在实验室做饭,我已对他提出最严重警告,再不听就往他锅里下泻药。实验室耗子成灾,我也有解决的方法,去买几只猫来。” 
 
 “全是胡说,只有养猫防鼠还不太离谱。可是你想了没有,我就在你隔壁。晚上我这儿开会,你的猫闹起来了怎么办?” 
 
 “我有措施。我把它阉了,它就不会闹。我会阉各种动物,大至大象,小到黄花鱼,我全有把握。” 
 
 “哈哈。我叫你来,还不是谈实验室约束。反正我也要搬走,随你闹去,我眼不见心不烦。谈谈你的事。你多大了!” 
 
 “三十有二。” 
 
 “三十而立嘛。你是大人了,别老像个孩子,星期天带爱人到我家玩。你爱人叫什么名字?” 
 
 “张小霞,小名二妞子。报告校长,此人是一名悍妇,常常侵犯我的公民权利。如果您能教育感化她,那才叫功德无量。” 
 
 “好,胡扯到此为止。告诉你一件事,你不要有情绪。你要借调出国,党委讨论过了,不能同意啊。” 
 
 “这干他们什么事?为什么不同意?吃错药了?” 
 
 “不要这样。我们新建的学校,缺教师这是事实。再说,你也太不成体统。大家说,放你这样的人出去,给学校丢人。同志们对你有偏见,我是尽力说服了的。你还是要以此事为动力,改改你的毛病……” 
 
 校长不酸不凉把我一顿数落,我全没听进去。这两年我和矿院吕教授合作搞项目,凭良心说,我干了百分之九十的工作。白天在学校上课,晚上到他那儿做试验。受累不说,还冒了被炸成肉末儿的危险。因为做的是炸药。我这么玩命。所为何事?就因为吕教授手下有出国名额。只要项目搞成,他就得把我借到他手下,出国走一圈,到外边看看洋妞儿有多漂亮。这本是讲好了的事,如今这项目得了国家科技一等奖,吕教授名利双收,可这点小事他都没给我办成。忽然听见校长喊我;“喂喂,出神儿啦?” 
 
 “报告校长,我在认真听。你说什么来着?” 
 
 “我在问你,还有什么意见?” 
 
 我当然有意见!不过和他说不着。“没有!我要找老吕,把他数落数落。”
 
 “你不用去了,吕教授已经走了。他说名额废了太可惜,你既然不能去,他就替你作主,凭良心说,他也尽了力。一晚上给我打七次电话,害得我也睡不着。我是从矿院调来的,你是矿院的子弟,咱们也不能搞得太过分。最主要的问题是:这件事你事先向组织上汇报了吗?下次再有这种事,希望你能让我挺起腰杆为你说话。首先要把许由管管,其次自己也别那么疯。人家说,凡听过你课的班,学生都疯疯癫癫的。” 
 
 “报告校长,这不怪我。这个年级的学生全是三年困难时坐的胎。那年头人人挨饿,造他们时也难免偷工减料。我看过一个材料,犹太孩子特别聪明、守规矩,全是因为犹太人在这种事上一丝不苟。事实证明,少摸一把都会铸成大错……” 
 
 “闭嘴,看你哪像大学教师的样子?我都为你脸红。回去好好想想,就谈到这里吧。” 
 
 我从校长室出来,怒发冲冠,想拿许由出气。一进实验室的门,看见许由在实验台上吃饭,就拼命尖叫起来:“又在实验室吃饭!!!你这猪……”吼到没了气停下来喘,只见他双手护耳。这时听见校长在隔壁敲墙。走到许由面前,一看他在吃香椿拌豆腐,弄了那么一大盆,我接着教训他: 
 
 “你这不是塌我的台吗?这东西产气,吃到你肚子里还了得?每次我在前边讲,你就在后面出怪声,好像吹喇叭。然后学生就炸了窝!” 
 
 “得了,王二,假正经干嘛。你看我拌的豆腐比你老婆弄得不差。” 
 
 “里面吃去。许由,你净给我找麻烦!” 
 
 “嘿嘿,你别拿这模样对我,我知道为什么。你出国没出成。王二,人生不如意之事,十有八九,别放在心上。人没出国,还有机会,我还有什么机会?老婆还不知上哪儿去找哩。” 
 
 说到这个事,我心里一凉。也许他不是这个意思,是我多心。我和许由三十年的交情,从来都是我出主意他干。从小学到中学,我们干尽了愉鸡摸狗的勾当,没捅过大漏子。千不该万不该,“文化革命”里我叫他和我一块到没人的实验室里造炸药玩,惹出一场大祸来。现在许由的脸比得过十次天花还要麻,都是我弄出来的。 
 
 他的脸里崩进了好几根试管,现在有时洗脸时还会把手割破,这全怪我在实验台上挥了一根雷管。没人乐意和大麻壳结婚,所以他找不着老婆。我们俩从来没谈过那场事故的原因,不过我想大家心里都有数。我对他说: 
 
 “你用不着拿话刺我!” 
 
 “王二,我刺你什么了?” 
 
 “是我把你炸伤的!我记着呢!” 
 
 “王二,你他妈的吃枪药了,你这叫狗眼看人低。嘿,在校长那儿吃了屁,拿我出气。我不理你,你自己想想吧!” 
 
 他气冲冲走开了。 
 
 和许由吵过之后,我心里乱纷纷的。这是我第一次和许由吵架,这说明我很不正常。我听说有些人出国黄了,或者评不上讲师就撒癔症,骂孩子打老婆搅得鸡犬不宁。难道我也委琐如斯?这倒是件新闻。 
 
 我在实验室里踱步,忽然觉得生活很无趣,它好像是西藏的一种酷刑:把人用湿牛皮裹起来,放在阳光下曝晒。等牛皮干硬收缩,就把人箍得乌珠迸出。生活也如是:你一天天老下去,牛皮一天天紧起来。这张牛皮就是生活的规律:上班下班、吃饭排粪,连做爱也是其中的一环,一切按照时间表进行,躺在牛皮里还有一点小小的奢望:出国,提副教授。一旦希望破灭,就撒起癔症。真他妈的扯淡:真他妈的扯淡得很! 
 
 不知不觉我在实验室的高脚凳上坐下来,双手支着下巴,透过试管架,看那块黑板。黑板上画了些煤球。我画煤球干什么!想了半天才想起是我画的酵母。有些委琐的念头,鬼鬼祟祟从心底冒出来。比方说我出国占矿院的名额,学校干嘛卡我?还有我是个怎样的人干你们球事等等。后来又想:我何必想这些屁事。这根本不该是我的事情。 
 
 我看着那试管架,那些试管挺然翘然,引起我的沉思。培养基的气味发臭,叫我闻到南国沼泽的气味,生命的气味也如是。新生的味道与腐烂的味道相混,加上水的气味。南方的太阳又白又亮,在天顶膨胀,平原上草木葱笼,水边的草根下沁出一片片油膜。这是一个梦,一个故事,要慢慢参透。 
 
 从前有一伙人,从帝都流放到南方荒蛮之地。有一天,其中一位理学大师,要找个地方洗一洗,没找到河边,倒陷进一个臭水塘里来了。他急忙把衣服的下摆撩起。乌黑的淤泥印在雪白的大腿上。太阳晒得他发晕,还有刺鼻的草木气味。四下空无一人,忽然他那话儿无端勃起,来得十分强烈,这叫他惊恐万分。他解开衣服,只见那家伙红得像熟透的大虾,摸上去烫手,没法解释为什么,他也没想到女人。水汽蒸蒸,这里有一个原始的欲望,早在男女之先。忽然一阵笑声打破了大师的惶惑——一对土人男女骑在壮硕的水牛上经过。人家赤身棵体,搂在一起,看大师的窘状。 
 
 有人对我说话,抬头一看,是个毛头小子,戴着红校徽,大概是刚留校的,我不认识他。他好像在说一楼下水道堵了,叫我去看下,这倒奇了,“你去找总务长,找我干什么?” 
 
 “师傅,总务处下班了。麻烦你看一下,反正你闲着。” 
 
 “真的吗!我闲着,你很忙是吗?” 
 
 “不是这回事,我是教师,你是锅炉房的。” 
 
 “谁是锅炉房的?喂喂,下水道堵了,干你什么事!” 
 
 “学校卫生,人人有责嘛。你们锅炉房不能不负责任!” 
 
 “×你妈:你才是锅炉房!你给我滚出去!” 
 
 骂走这家伙,我才想起为什么人家说我是锅炉房的。这是因为我常在锅炉房里呆着。而且我的衣着举止的确也不像个教师。也许就是因为这个,我才出不了国。这没什么。我原本是个管工,到什么时候都不能忘本。要不是他说我“闲着”,我也可能去跟他捅下水道,你怎么能对一个工人说“反正你闲着”? 
 
 太阳从西窗照进来,到下班的时候了,我还不想走。愤懑在心里淤积起来,想找个人说一说。许由进来,问我在不在学校吃饭。许由真是个好朋友,我想和他说说我的苦闷。但是他不会懂,他也没耐心听。 
 
 我想起拉封丹的一个寓言:有两个朋友住在一个城里,其中一个深夜去找另一个。那人连忙爬起来,披上铠甲,右手执剑,左手执钱袋,叫他的朋友进来说:“朋友,你深夜来访,必有重大的原因。如果你欠了债,这儿有钱。如果你遭人侮辱,我立刻去为你报仇。如果你是清夜无聊,这儿有美丽的女奴供你排遣。” 
 
 许由就是这样的朋友,但是现在他对我没用处。我心里的一片沉闷,只能向一个女人诉说,真想不出她是谁。 
 
 二 
 
 我骑上车出了校门,可是不想回家,在街上乱逛。我老婆见我烦闷时,只会对我喋喋不休,叫我烦上加烦。我心里一股苦味,这是我的本色。 
 
 好多年前,我在京郊插队时,常常在秋天走路回家,路长得走不完。我心里紧绷绷,不知道走到哪里去,也不知走完了路以后干什么。路边全是高高的杨树,风过处无数落叶就如一场黄金雨从天顶飘落。风声呼啸,时紧时松。风把道沟里的落叶吹出来,像金色的潮水涌过路面。我一个人走着,前后不见一个人。忽然之间,我的心里开始松动。走着走着,觉得要头朝下坠入蓝天,两边纷纷的落叶好像天国金色的大门。我心里一荡,一些诗句涌上心头。就在这一瞬间,我解脱了一切苦恼,回到存在本身。 
 
 我看到天蓝得像染过一样。薄暮时分,有一个人从小路上走来,走得飞快,踢土扬尘的姿势多熟悉呀!我追上去在她肩上一拍,她一看是我,就欢呼起来:“是他妈的你!是他妈的你!”这是我插队时的女友小转铃。 
 
 我们迎着风走回去,我给她念了刚刚想到的诗,其中有这样的句子: 
 
 走在寂静里,走在天上, 
 
 而阴茎倒挂下来。 
 
 虽然她身上没有什么可以倒挂下来,但是她说可以想象。小转铃真是个难得的朋友,她什么都能想象。 
 
 我应该回劲松去,可是转到右安门外去了,小转铃就住在附近。我也不知道自己为什么走到这儿来,我绝没有找她的意思,可是偏偏碰上了。 
 
 她穿浅黄色的上衣,红裙子,在路边上站着,嘴唇直哆嗦,一副要哭的样子,看样子早就看见我了。我赶紧从车上下来。打个招呼说: 
 
 “铃子,你好吗?” 
 
 她说:“王二,你他妈的……”然后就哭了,我觉得这件事不妙——我们俩最好永远别见面。 
 
 小转铃叫我陪她去吃饭。走进新开的得月楼,一看菜单,我差点骂出口来:像这种没名的馆子竟敢这么要钱,简直是不要脸。这个东我做不起,可要她请我又不好意思。过去我可以说:铃子,我有二十块钱。你有多少钱!现在不成了。我是别人的丈夫,她是别人的妻子。所以我支支吾吾,东张西望,小转铃见我这个洋子,先是撅嘴,后来就火了。 
 
 “王二,你要是急着回家,就滚!要是你我还有在一块吃饭的交情,就好好坐着。别像狗把心叼走了一样。” 
 
 “你这是怎么了,我在想,这年头吃馆子多少钱,等付帐时闹个大红脸就不好了。” 
 
 “这用你说吗!我要是没钱,早开口了!王二,你真叫我伤心,你一定被你那个二妞子管得不善!” 
 
 “你别这么说。我就不会说这种话。” 
 
 小转铃的脸红了。她说:“我就是想说这个。好吧,不谈这种话,你好吗?最近还写东西吗?” 
 
 我说顾不上了。近来忙着造炸药。她听了直撇嘴。正说着,服务员来叫点菜。她像怄气一样点了很多。我不习惯在桌面上剩东西,所以她可能是要撑死我。 
 
 十年前,我常和小转铃去喝酒。我喝过酒以后,总是很难受,但每次都是我要喝。而小转铃体质特异,喝白酒如饮凉水,喝多少也没反应。天生一个酒漏。夏天在沙河镇上,我们喝了一种青梅酒,这东西喝起来味道尚可,事后却头疼得像是脑浆子都从耳朵眼里流出来。酒馆里只有一种下酒菜,乃是猪脑子。铃子说看着都恶心。我还是要了一盘,尝了一口,腥得要命。她不敢看那个东西,把它推到桌角,我们找个题目开始讨论。 
 
 所谓讨论,无非是没事扯淡罢了。那天谈的是历史哲学。据说克莉奥佩屈拉的鼻子决定了罗马帝国的兴衰,由此类推,一切巨大的后果莫不为细小的前因所注定。而且早在亿万斯年之前,甚至在创世之初,就有一个最微小的机缘,决定了今日今时,有一个王二和小转铃,决定了他们在此喝酒,还决定了下酒菜是猪脑子,小转铃不肯吃。你也可以说这是规律使然,也可以说是命中注定。小转铃说,倘若真的如此,她简直不想活了。为了证明此说不成立,她硬着头皮吃了一口猪脑子。这东西一进了嘴,她就要吐,我也劝她把它吐了,可是她硬把它吞了下去,眼见它像只活青蛙,一跳一跳进了她的胃。小转铃就是这么倔! 
 
 小转铃对什么都认真,而我总是半真不假。坐在她面前,我不无内疚之感,抓起啤酒瓶往肚子里灌,脸立刻就红了。 
 
 铃子说:“王二,我今天难得高兴。请你把着点量,别灌到烂醉如泥。记得吗?那次在沙河镇上,你出了大洋相!” 
 
 那天晚上我出的什么洋相已经记不清了,只记得是她把我扛回去的,很难想象她能扛得起我。但她要是硬要扛,好像也没什么扛不动的东西。我站起来到柜台上买了一瓶白兰地。回来后铃子问我要干什么。我说我今晚上不想回家,想和她上公园里坐一宿,这瓶酒到后半夜就用得着了。小转铃大喜: 
 
 “王二,你要让我高兴,总能想出办法。不必去公园,上我家去,近得很。” 
 
 “不好吧?你丈夫准和我打起来。” 
 
 “我早离婚了。” 
 
 “为什么?” 
 
 “不为什么!” 
 
 我说离婚可不容易,尤其是通过法院判离,她说可不是?她们报社就派了一位副主编来做工作,叫她别离婚。“假正经!完全是假正经!” 
 
 “你怎么和他说?” 
 
 “我说,有的人配操我的×,有的人就不配!老先生当场晕倒,以后再没人找茬!” 
 
 “你别故做惊人之语啦,没这话吧。” 
 
 “我说过!我什么时候对你说过假话?我可不像你,说句真话就脸红。你的论文还在我这儿呢!我常看,获益极多!” 
 
 提起那篇论文,我的心往下一沉,好似万丈高楼一脚蹬空。我早己忘了除了爆炸物化学和微生物,好多年前还写过一篇哲学论文。这种事怎么会忘记?我有点怀疑自己是存心忘记的,这是件很奇怪的事。 
 
 我在知青点最后一个冬天,别人都回城去了,男生宿舍里只有我一个。我叫铃子搬过来,我们俩形同夫妇。我从城里搬来很多书,看到那么多漂完的书堆在炕上,真叫人心花怒放! 
 
 那一年城里中国书店开了一家机关服务部,供应外文旧书。我拿了我妈搞来的介绍信和我爸爸的钱混进去,发现里面应有尽有。有好多过去的书全在扉页上题了字、盖了印章。其中很多人已经死了,还有好多人不知去向。站在高高的书架下面。我觉得自己像盗墓贼一样。我记得有几千本书上盖着“志摩藏书”的字样——曾几何时,有过很多徐志摩那样的人,在荒漠上用这些书筑起孤城。如今城已破,人已亡,真叫人有不胜唏嘘之情! 
 
 我在知青点看了一冬天的书。躺在热坑上,看到头疼时,就看看窗玻璃上的冰花。这时小转铃就凑上来说;王二,讲讲呀!她翻着字典慢慢看,一天也看不了几页。 
 
 我从小受家传的二手洋奴教育,英文相当不赖,所以能有阅读的乐趣,但是我只颠三倒四乱讲几句,又埋头读书。天黑以后,像狗一样趴在坑上,煤油灯炙黄了头发。到头皮发紧,眼皮发沉时,我才说;“铃子,咱们得睡了。”但是自己还在看书,影影绰绰觉得小转铃在身边忙碌,收拾东西,还从我身上剥衣服。最后她吹熄了灯,我才发觉自己精赤条条躺在被窝里。 
 
 我在黑暗里给小转铃讲自己刚看的书,因为兴奋和疲惫,虚火上升。小转铃对我做了必要的措施,嘴里还催促着:“讲。后来呢?” 
 
 等到开始干时她不说话了,刚刚结束,她又说:“后来呢?” 
 
 这真叫岂有此理!我说:“喂,你这么讲像话吗?” 
 
 “对不起,对不起,可是后来呢?” 
 
 “后来还没看到。我还得点起灯来再看!” 
 
 “你别看了!你现在虚得很,我能觉出来,好好睡一觉吧。” 
 
 有一天晚上我总是睡不着,想到笛卡尔的著名思辩“我思,故我在”。我不诧异笛卡尔能想出东西来,我只奇怪自己为什么不是笛卡尔。我好使缺少点什么,这么一想思绪不宁。我爬起来,抽了两支姻,又点起煤油灯,以笛卡尔等辈曾达到的境界来看,我们不但是思维混乱,而且有一种精神病。 
 
 小转铃醒来,问我要干什么,我说要做笛卡尔式的思辩。这一番推论不知推出个什么来。她大喜,说;“王二。推!快推!”以后就有了那篇论文。 
 
 我不乐意想到自己写下的东西,就对小转铃说:“铃子,我们有过好时光!那一冬读书的日子,以后还会有吗?” 
 
 她放下酒杯说;“看书没有看你的论文带劲。” 
 
 又提到那篇论文!这就如澡塘里一池热水,真不想跳下去。我不得不想起来,我那篇论文是这么开头的:假若笛卡尔是王二,他不会思辩。假若堂吉柯德是王二,他不会与风车搏斗。王二就算到了罗得岛,也不会跳跃。因为王二不存在。不但王二不存在,大多数的人也不存在,这就是问题症结所在。 
 
 发了这个怪论以后,我又试图加以证明。如果说王二存在,那么他一定不能不存在。但是王二所在的世界里没有这种明晰性,故此他难以存在。有如下例子为证: 
 
 凡人都要死。皇帝是人,皇帝万岁。 
 
 还有: 
 
 人都要死,皇帝是人,皇帝也会死。 
 
 这两种说法王二都接受,你看他还有救吗!很明显,这个世界里存在着两个体系。一个来自生存的必要,一个来自存在本身,于是乎对每一个问题同时存在两个答案。这就叫虚伪,我那篇论文题目就叫《虚伪论》。 
 
 我写那篇东西时太年轻,发了很多过激议论。只有一点还算明白:我没有批判虚伪本身。不独如此,我认为虚伪是伟大的文明。小转铃对此十分不满,要求把这段删去,而我拿出吕不韦作春秋的气概说:一字干金不易。现在想,当时好像有精神病。 
 
 想着这件事,不知不觉喝了很多酒。天已经晚了,饭厅里只剩了几桌客人。有一个服务员双手叉腰站在厨房门口,好像孙二娘在看包子馅。我在恍惚之间被她拖进了厨房,倒挂在铁架上。大师傅说:“这牛子筋多肉少,肉又骚得紧。调馅时须是要放些胡椒。” 
 
 那母夜叉说道:“索性留下给我做个面首,牛子你意下如何?” 
 
 她上唇留一撮胡须,胸前悬着两个暖水袋。我说道:“毋宁死。”她踢了我一脚说:“不识抬举。牛子,忍着些。过一个时辰来给你放血。”于是就走了。厨房里静悄悄的,忽然一只狮子猫,其毛白如雪,像梦一样飘进来,蹲在我面前。 
 
 铃子对我说:“王二!醉啦?出什么神?” 
 
 其实我还没醉,还差得远。我坐端正,又想起自己写过的论文。不错,我是写过,虚伪还不是终结。从这一点出发后,每个人都会进化。 
 
 所谓虚伪,打个比方来说,不过是脑子里装个开关罢了。无论遇到任何问题,必须做出判断:事关功利或者逻辑,然后就把开关拨动。扳到功利一边,咱就喊皇帝万岁万万岁,扳到逻辑一边,咱就从大前题、小前题,得到必死的结论。由于这一重负担,虚伪的人显得迟钝,有时候弄不利索,还要犯大错误。 
 
 人们可以往复杂的方向进化:在逻辑和功利之间构筑中间理论。通过学习和思想斗争,最后达到这样的境界:可以无比真诚地说出皇帝万岁和皇帝必死,并且认为,这两点之间不存在矛盾。也不知道为什么,这条光荣的道路一点也不叫我动心。我想的是退化而返朴归真。 
 
 在我看来,存在本身有无穷的魅力,为此值得把虚名浮利全部放弃。我不想去骗别人,受逼迫时又当别论。如此说来,我得不到什么好处。但是,假如我不存在,好处又有什么用? 
 
 当时我还写道,以后我要真诚地做一切事情,我要像笛卡尔一样思辩,像堂吉河德一样攻击风车。无论写诗还是做爱,都要以极大的真诚完成。眼前就是罗得岛,我就在这里跳跃——我这么做什么都不为,这就是存在本身。 
 
 在我看来,春天里一棵小草生长,它没有什么目的。风起时一匹公马发情,它也没有什么目的。草长马发情,绝非表演给什么人看的,这就是存在本身。 
 
 我要抱着草长马发情的伟大真诚去做一切事,而不是在人前差羞答答的表演。在我看来,人都是为了要表演,失去了自己的存在。我说了很多,可一样也没照办。这就是我不肯想起那篇论文的原因。 
 
 服务员拿了把笤帚扫地。与其说是扫地,不如说是扬场。虽然离饭店关门还有半个钟头,我们不得不站起来,恋恋不舍地到外面去。那年冬天我和铃子也是这么恋恋不舍地离开集体户。 
 
 我和小转铃在集体户住了二十多天,把一切都吃得精光,把柴火也烧得精光。最后离开时,林子里传来了鞭炮声。原来已经是大年三十,天上飘着好大的雪,天地皆白,汽车停开,行人绝迹。我们俩在一片寂静中走回城去。 
 
 如今我和铃子上她家去,走过一条田间的土路,这条路我从来没走过,也不知道通到哪里去。我有点怕到小转铃那里去,这也许是因为她对生活的态度,还像往日一样强硬。 
 
 我和小转铃走过茫茫大雪回城去,除了飞转的雪片和沙沙的落雪声,看不见一个影子,听不见一点声音。冷风治好了持续了好几天的头疼。忽然之间心底涌起强烈的渴望,前所未有:我要爱,要生活,把眼前的一世当做一百世一样。这里的道理很明白;我思故我在,既然我存在,就不能装作不存在。无论如何,我要对自己负起责任。 
 
 到了小转铃家,弄水洗了脸,我们坐在院子里继续喝酒。不知为什么,这回越喝越清醒,平时要喝这么多早醉了。小转铃坐在我对面的躺椅里,一声也不吭。我看着她,不觉怦然心动。 
 
 那一年我们踏雪回家,走到白雾深处,我看着她也怦然心动。那时候四面一片混沌,也不知天地在哪里,我看见她艰难地走过没膝的深雪,很想把她抱起来。她的小脸冻得通红,呵出的白气像喷泉一样。那时候天地茫茫,世界上好像再没有别的人。我想保护她,得到她,把她据为已有。 
 
 没人能得到小转铃,她是她自己的。这个女人勇捍绝伦,比我还疯狂。我和她初次做爱时,她流了不少血,涂在我们俩的腿上。不过片刻她就跳起来,嬉笑着对我说;王二,不要脸!这么大的东西就往这里杵! 
 
 我和她是上大学时分手的。在此之前同居了很长时间。性生活不算和谐,但是也习惯了。小转铃是性冷淡,要用润滑剂,但是她从没拒绝过,也没有过怨言。我也习惯了静静躺在身下的娇小身躯。但是最后还是吹了,我总觉得是命中注定。 
 
 小转铃就坐在面前,上身戴个虎纹乳罩,下身穿了条短裙,在月光下显得很漂亮。我还发现她穿了耳朵眼,不过这没有用。她的鞋尖还是一场糊涂,这说明她走路时还是要踢石子。这就叫江山易改,本性难移。 
 
 我知道,如果小转铃说:“王二,我需要你”,结果会难以想象。小转铃也知道,我经不起诱惑。但是她什么都没有说,只是放下了酒杯又抽烟。其实她很想说,但是她不肯。 
 
 小转铃说过,她需要我这个朋友,她要和我形影不离,为此她不惜给我当老婆。和一个朋友在一起过一辈子可够累的。所以我这么和她说:也许咱们缘分不够,也许你能碰上一个人,不是不惜给他当老婆,而是原本就是他老婆。不管怎么说,小转铃是王二的朋友,这一点水远不会变。说完了这些话,我就和她分手了。 
 
 假如今天小转铃肯说:“王二,我是你老婆”,这事情就不妙得很。二妞子可不容人和她打离婚。但是这件事没有发生。我们直坐到月亮西斜,我才说:“铃子,我要回去了。” 
 
 有一瞬间小转铃嘴唇抖动,又像是要哭的样子,但是马上又恢复了平静。她说:“你走吧,有空常来看我。”我赶紧住家赶,可了不得了,已经是夜里两点钟! 
 
 三 
 
 我蹑手蹑脚出了院门,骑车回家去。把车扛上楼锁在扶手上,轻轻开门进去,屋里一团漆黑。脱下鞋小心翼翼往床上一躺,却从床上掉下来。然后灯亮了,我老婆端坐在床上。刚才准是她一脚把我从床上踹下来,她面色赤红,头发都竖了起来。 
 
 “你上哪儿去了?我以为你死了哩!学校、矿院,到处都打了电话,还去了派出所。原来你去喝酒!和谁混了一夜?” 
 
 我虽然很会撒谎,可是不会骗老婆。和某些人只说实话,和某些人只说假话,这是我的原则。于是我期期艾艾地说:“和小转铃碰上了,喝了一点儿。” 
 
 她尖叫一声,拿被子蒙上头,就在床上游仰泳。现在和她说什么都没用,我去厕所洗了脚回来,关上灯又往床上一躺。忽然脖子被勒住,憋得我眼冒金星,二妞子在我耳边咬牙切齿地说:“叫你知道我的厉害!” 
 
 这个泼妇是练柔道的,胳膊真有劲。平时她也常向我挑衅,但是我不怕她。不管她对我下什么绊儿,我只把她拎起来往床上一扔。她是四十七公斤级的,我是九十公斤级的,差了四十多公斤。现在在床上被她勒住了脖子,这就有点棘手。这女人成天练这个名堂,叫做什么“寝技”。我翻了两下没翻起来,太阳穴上青筋乱蹦。最后我奋起神威,炸雷也似大喝一声(行话叫喊威),往起一挣,只听天崩地裂一声巨响,床塌了。我在地上滚了几滚,又撞倒了茶几,稀哩哗啦。我终于摔开她,爬起来去开灯,只见她坐在地上哭,这时候应该先发制人。 
 
 “夜里三点啦!你疯什么?诈尸呀!” 
 
 我是如此理直气壮,她倒吃一谅,半天才觉过味来:“你混蛋!离婚!” 
 
 “明天早上陪你去,今晚上先睡觉。” 
 
 “我找你妈告状去!” 
 
 “你去吧,不过我告诉你,你没理。” 
 
 “我怎么会没理?” 
 
 “事情是这样的:不管怎么说,我和小转铃是多年的老朋友了,见面哪能不理呢?陪她吃顿饭,喝一点,完全应该。” 
 
 “一点儿?一点是多少!” 
 
 “也就是半斤吧。不是白干,是白兰地。” 
 
 “好混蛋,喝了这么多。在哪儿吃的饭?” 
 
 “齐家河得月楼。莱糟得一塌糊涂,小转铃开的钱。” 
 
 “混蛋!显她有钱。明天咱们去新侨,敢不去阉了你。菜!一样一样说。” 
 
 这还有完吗?深更半夜的,我又害头疼。“炒猪屄!” 
 
 二扭子气得又哭又笑。扯完了淡,已经是四点钟。刚要合眼,二妞子又叫我把自行车搬进来,结果还是迟了一步。前后胎的气都被人放光。还算客气,没把气门嘴拔去。这是邻居对我们刚才武斗的抗议。 
 
 那一夜我根本没睡。二妞子在我身边翻来覆去闹个不休。天快亮时,我才迷糊了一会儿,一双纤纤小手又握住了我的要命处,她要我证明自己没二心。这一证明不要紧,睡不成了。第二天早上教师会,校长布置工作。不到一刻钟的工夫,我往地下出溜了三回。校长大喝一声:“王二,你站起来!” 
 
 “报告校长,我已经站起来了!” 
 
 “你就这么站着醒醒!以前开会你打磕睡,我没说你。你是加夜班做实验,还得了奖嘛,可以原谅。如今不加夜班了,你晚上干什么去了?” 
 
 不提这事犹可,一提我气不打一处来。难道该着我加夜班?一屋子幸灾乐祸的嘴脸,一屋子假正经!不要忙,待我撒泼给你们看:“报告校长,老婆打我。” 
 
 全场哄然。后排校工座上有人鼓掌。 
 
 “报告校长,我为了学校荣誉,奋起抗暴,大打出手,大败我老婆,没给学校丢脸!” 
 
 后排的哥儿们全站起来,掌声雷动。校长气得面皮发紫,大吼一声:“出去!到校长室等我!” 
 
 到了校长室,我又有点后悔。太给校长下不来台。校长拿我当他的人百般庇护,他提我当生物室主任,虽然只管许由一个宝贝,好多人还是反对。人事处长拿了我档案去说:王二历史上有问题,他和许由犯过爆炸案。这两个家伙可别把办公楼炸了,最好让我当副主任,调食堂胖三姑当正主任。校长哈哈大笑说:两个小屁孩,“文化革命”里闹着玩,有什么问题。倒是食堂的胖三姑好贪小便宜,放到实验室里是个祸害。最近我和吕教授项目搞成,到手二千元奖金,他拿大头,给我三百。这钱到了学校会计科,科长就要全部没收。理由是王二拿了学校的工资,夜里给外单位于活。白天上课打呵欠,坐第一排的学生能看见我的扁桃腺,校长又为我说话,说王二加班搞项目,功在国家,于学校也有光彩。国家奖下来的钱,你们克扣不是佛面刮金吗?结果这钱全到了我手,比吕教授到自己手的还多。 
 
 想到这些事,我心里发软。我不想被人看成个不知好歹的人。但是转念一想,心里又硬起来,×你妈,谁说我是你的人?老子是自己的人。正在想着,校长进来了。他坐下沉默了两分钟,凝重地说:“小王,我要处分你。” 
 
 “报告校长,我早该处分!” 
 
 “你不要有情绪。出国的事,你不满意,可以理解。但不能在会场上这么闹!我不处分你,就不能服众。” 
 
 “报告,我没情绪。我对组织一贯说实话。二妞子是打了我。你看我脖子上这一溜紫印……也就是我,换上别人早被掐死了。” 
 
 校长一看我脖子,简直哭笑不得:“你这小子!夫妇打架也要有分寸!” 
 
 “校长,你不知道。这可不是夫妇打闹!我老婆是真打我。她是柔道队的!上次把我肘关节扭掉了环,贴了好多虎骨膏,现在还贴着呢。” 
 
 校长沉吟了半晌,走了出去。我心里暗笑:看你怎么处理我。过一会儿他把工会主席和人事处长叫进来,这两人是我的大对头。校长很激动地说: 
 
 “你们看看,这成什么体统!把人打成这个样子!男同志打老婆单位要管,女同志打老公,我们能不管吗?不要笑!这情况特殊!得给体委打电话,叫他们管教一下运动员!工会人事要出面。伤成这个样子,影响工作。小王呀,要是不行就回家休息。最好坚持一下,把会开完。” 
 
 鬼才给他坚持。出了校门我就拍着肚皮大笑:世界上居然有这样的校长!回家睡了一大觉,起来已然三点钟。我老婆留条叫我四点钟去新侨,还把西装取出来放在桌上。我打扮起来照照镜子,怎么看怎么不像那么回事。我这个人根本就没体面。出了门我怕熟人看见我,就溜着墙根走。到了新侨门口,老远就看见我老婆。她穿了一件鲜红的缎子旗袍,有加一床缎子被。她还擦了烟脂抹了粉,活脱脱一个女妖精:我走过去挽住她的手,手心里全是汗。只听她娇叹一声: 
 
 “我要死了!” 
 
 “别怕,往前走,打断我骨头的劲儿上哪儿去了?别看地,没钱,有钱我比你先看见。抬头!挺胸!” 
 
 “我怕人家看见我抹了粉!” 
 
 “怕什么?你蛮漂亮的嘛。抹了粉也比没鼻子的人好看。要像模特儿那么走。晃肩膀,扔屁股!” 
 
 她这么一走,好似发了自发功,骨节都响起来。我老婆穿得随便一点,走到街上还蛮有人看的,现在别人都把头扭到一边去,走进饭店在桌前坐下,她都要哭了。 
 
 这顿饭吃得很不舒服,我觉得我们俩是在饭店里耍了一场活宝。回家以后,我有好一阵若有所思,似乎有所领悟。第二天早上到班,我就比平时更像个恶棍。 
 
 我一到学校,就先与许由汇合。出国出不成,我已经想通了,反正没我的份。前天和许由闹了一架,彼此不痛快,现在应该聊一聊。从小到大,他一直是我的保镖,我不能叫他和我也生分了。正聊得高兴,墙壁响了,这是校长的信号,召我去听训。 
 
 进了校长室,只见他气色不正。桌子上放着我上报的实验室预算。只听他长叹一声: 
 
 “王二呀王二,你的行为用四个字便可包括!” 
 
 “我知道,克己奉公。” 
 
 “不。少年无行!你瞧你给总务处的预算。什么叫‘二百立升冰箱三台,给胖三姑放牛奶’?” 
 
 “她老往我冰箱里放牛奶,说是冰箱空着也是白费电。冰箱是我故菌种的,她把菌种放到外边,全坏了。现在人家又怀上了,不准备下来行吗?” 
 
 “这意见应该提,可是不要在报告里乱写。再说,为什么写三台?有人说,你是借题发挥,有意破坏团结。” 
 
 “校长,三姑生的是第二胎。第一始是生肚子,生不多。第二胎生十个八个是常有的事。真要是老母猪,人家有那么多个奶。三姑只有两个,咱们要为第二代着想。这道理报告里写了。” 
 
 “胡扯!本来有理的事,现在把柄落在人家手里。你坐下,咱们推心置腹地谈谈。你知道咱们学校处境不好吗?” 
 
 “报告校长,我看报了。现在新建的大学太多,整顿合并是党中央的英明决策。就说咱们学校,师资校舍一样没有,关了也罢。” 
 
 “你这叫胡说八道!咱们学校从无到有,在很艰苦的条件下给国家培养了几千名毕业生,成绩明摆着。现在有了几百教职员工,这么多校舍设备。怎么能关了也罢?学校关了你去哪儿!” 
 
 “我去矿院。老吕调我好几回了,都是您给压着。你再看看我,是不是放我走了更适合?” 
 
 “你别做梦了。学校有困难,请调的一大批。放了你我怎么挡别人?党委讨论了,一个都不放。谁敢辞职,先给个处分,叫他背一辈子。另一方面,我们也要大胆提拔年轻人。能干的我们也往国外送,提教授。就说你吧,几乎无恶不作,我们还提你当生物室主任,学校有什么地方对不住你?” 
 
 “对不起我的地方太多了。就说住房吧。我同学分到农委,才毕业就是一套房。我呢?打了半天报告,分我一间地下室。又湿又黑,养蘑菇正合适。就说我落后,也没落后到这个份上。蘑菇是菌藻植物门担子菌纲。我呢,起码是动物,灵长目,人科人属,东亚亚种,和您一样。您看我哪一点像蘑菇?” 
 
 “当然!谁也不是蘑菇!我们要关心人。房子会有的。你不要哭穷。你住得比我宽敞!” 
 
 “那可是体委的房。我老婆说,我占了她的便宜,要任打任骑。要说打,打得过她,可是咱们理亏。咱们七尺大汉,就因为进了这个学校,被老婆打得死去活来,还不敢打离婚——离婚没房子住。要不就得和许由挤实验室。许由的脚有多臭,你知道吗?” 
 
 “所以休想把学校闹得七颠八倒。明白和你说了吧,这学校里也不是我一个人说了算。你和我耍贫嘴没用。就算你真调成了,也没个好儿。我把你的政治鉴定写好了,想不想听听!‘王二同志,品行恶劣。政治上思想反动,工作上吊儿郎当,生活上品行恶劣。’这东西塞在你档案里,叫你背一辈子。怎么样?想不想拿着它走?” 
 
 校长对我狞笑起来,笑得我毛骨悚然。我只好低声下气地求他: 
 
 “校长,您老人家怎么能这么对待我。我是真想学好,天分低一点,学得不像。好吧,这报告我拿回去重写。许由我也要管好,你还要我干什么?有话明说,别玩阴的。” 
 
 “你要真想学好,先把嘴改改。刚才说话的态度,像教员和校长说话的态度吗?” 
 
 “知道了。下次上您这儿来,就像和遗体告别。还有呢?” 
 
 “政治学习要参加!你是农三乙的班主任,知道吗?” 
 
 “什么叫农三乙,简直像农药名字。好,我知道了。星期三下午去和学生谈话。做到这些你给我什么好处!放我出国?” 
 
 “你想得倒美!政治部反映上来,你有反动言论。上次批精神污染的教师会上,你说什么来着?” 
 
 “那一回会上念一篇文章,太下流了,说什么牛仔裤穿不得。批精神污染是个严肃的事儿,不能庸俗化。说什么牛仔裤不通风,裹住了女孩子的生殖器,要发霉。试问,谁发霉了?你是怎么看见的?中国人穿了这几天就发霉,美国那些牛仔岂不要长蘑菇?” 
 
 “你的思想方法太片面,要全面地看问题。外国那些乱七八糟的东西进来,非抵制不可。再说那牛仔裤好在哪?我看不出。” 
 
 “您穿三尺的裤腰,穿上像大萝卜,当然穿不得。腰细的人穿上就是好看——好了,不争这个了。就说穿它发霉。咱们可以改进,在裤档上安上个小风机,用电池带动。这要是好主意,咱们出口赚大钱。要是卖不出去,那个写文章的包陪损失,准让他胡扯,我就发了这么个言。” 
 
 “这就不对!文章是我让念的。当时咱们学校也有女教师穿那个东西,我是要提醒大家注意:现在又说不准穿衣服的问题,再穿我也不管了。当然,发霉不发霉你是专家,但是不要乱讲。你明白了吗?” 
 
 “有一点不明白。你这么盯着我干嘛?” 
 
 “这话怪了。我是关心你,爱护你。” 
 
 “你关心我干嘛!” 
 
 “好吧,咱们说几句不上纲的话。学校现在是创业阶段,需要创业的人。大家对你有看法,但是我是这么看:不管你王二有多少毛病,反正你是既能干,又肯干。只要有这两条,哪怕你青面镣牙我也要——现在的年轻人,有几个肯干活的?这是从我这方面来看。从你这方面来看,我对你怎么样?古人还讲个知遇之恩哩!你到校外给老吕干活,他给你什么好处了?出国都不对你说一声。可我在校务会上说了你多少好话:老吕对你许了多少愿,他办成了吗?不负责任。我把这话放在这里:只要你表现好,什么机会我都优先你。其他年轻人比你会巴结的多的是,我都不考虑。因为我觉得你是个人材。这么说你懂了吗?” 
 
 这么说我就懂了。我说世界上怎么还有这样的校长!原来是这样。原来我是个人材!承他看得起,我也要拿出点良心来。矿院我决心不去了。 
 
 那天上午我带着学生去参观,大家精神抖擞地等着我。我把这帮人带到传达室等车,自己给接待单位中心配种站打电话。那儿有我一个同学当主任。 
 
 “配种站吗?我找郭主任。不!我什么都不送……我自己也没兴趣……我们公的母的都有。郭二,我们要去了。现在不是节气,只能看看样子了。刚才接电话的是谁?” 
 
 “我这儿没正经人。王二你来吧。不到季节,咱们可以人工催情哪。我这儿的牲口全打了针,全要造反呀!我设计了一头人造母猪,用上了电子技术,公猪们上去都不乐意下来!” 
 
 “人造的不要太多。我们是基础课,没那么专门。” 
 
 “天然的也有。我有云南来的一头小公驴,和狗一样大,阳具却大过了关中驴,看到的没有不笑的。你快来!” 
 
 “别这么嚷嚷,我这儿一大群学生,你吼的大伙全听见了。” 
 
 “嘿,你也正经起来了,骗谁呀。我还要和你切磋技术呢!” 
 
 “你越扯越下道了!同学们,把耳朵堵上。好了,不多说。半小时以后见。” 
 
 放下电话,心里犯嘀咕。我不该带学生去配种站,这样显得我没正经。等了半天,汽车还不来。正要派人去催,农学系主任刘老先生来了。他把嘴撅得像嘬了奶嘴一样: 
 
 “对不起王老师,对不起同学们,咱们的用车计划取消了。请回教室上课。参观下周去。” 
 
 “刘主任,你也是个农学家,这叫开的什么玩笑!这个季节配种要人工催情,忽而去忽而不去,叫人家怎么向种驴交代!好好,您来我也不说什么。我给配种站打电话。” 
 
 电话打通,郭二听说我们下星期去就叫:“放屁放屁,下星期不接待,我这配种站是给你开的?”说完啪一下挂上了。我对刘先生说:“您听听,人家怎么说我!配种站给我开的。我成什么了。同学们,咱们去不成了。再下周咱们考试。” 
 
 学生鼓噪起来,有人喊罢课。这么拦着校门起哄谁也吃不消,我赶紧说:“去去!咱们走着去。女同学和伤病员就别去了,下了公共汽车还要走六七里路呢。我们拍幻灯片给你们看。” 
 
 这么说也通不过。班上有个校队的,打球伤了腿,今天拄着拐来了,就是为了看配种。学生要抬着他去,这是胡闹。我对刘先生说:“您看,是不是派辆小车?起码得把伤兵带上。” 
 
 “王老师,不是我不派车!我们系里不像有些人那么不懂事——学农的不看配种站,那不是笑话吗?总务处说没车有啥办法。这些人可真浑,也不先打个招呼。” 
 
 “真的?我不信。您看我的。”抓起电话叫司机班,“你是谁?小马?给我把大轿车开出来。我带学生参观。” 
 
 “王二,车是你要的?我们处长瞎眼了。这么着,咱们坐驾驶楼,好不好?” 
 
 “不行!让别人坐卡车,我要大轿车。” 
 
 “我们处长叫把大轿车藏起来,别叫人看见。他要用。咱们给他留个面子,好吧?” 
 
 “那么我的面子呢?你以为谁的面子重要?” 
 
 “当然是王二了。王二是大哥嘛!车马上到。” 
 
 刘先生不走,看样子不信车能来。过一会儿车真从外边开进来了,学生欢呼着往上冲。刘老头气得险通红,手抖成七八只。我赶紧给他圆面子:“老先生,小马送我们想着风险呢。有人准给他穿小鞋。这可是为了咱们系的事……” 
 
 老头马上吼起来:“你放心,绝不让马师傅吃亏,我去找校长。问问他有车藏起来是什么作风!” 
 
 参观回来,学生全变了样,三五成群窃窃私语。我们拍了好几盒胶卷。我把班长叫来,关照几句: 
 
 “你把这片子送去制幻灯片,先放你这儿保存。谁借也别给,记住啦?除了农三乙,他们参观植物园,可能不满意。你要是把幻灯片借给外班看,下回我再不带你们出去。” 
 
 “老师,我们班对你最忠心。乙班人老说你坏话,我们班绝没这样人。这幻灯片我说不借,就说曝光了。” 
 
 “好,就依你。他们说我什么了?” 
 
 那些坏话无非是说我上课时衣冠不整,讲到得意忘形时还满嘴撒村。他不说我也知道,但是还想听一听,回到了学校,校长又叫我去一趟。怎么这么多麻烦?我简直有点儿烦了。 
 
 校长问我总务长藏车的事——其实他知道的比我还多。总务长想用大轿车送外单位的人去八达岭游玩,被我搅了。校长对此击节赞赏,对我大大鼓励了一番。但是我打不起兴致:我不过是个教员罢了,不想参与上层的事情。下午带同学去植物园,这班人对我有意见: 
 
 “老师,甲班人说配种站里有头驴,看上去有五条腿,中间一条比其它的长五倍。他们吹牛吧?” 
 
 “别听他们胡扯。这是科学,不是看玩艺儿。不过那驴是有点个别。” 
 
 “老师你偏心!我们也要去配种站参观!” 
 
 “别闹了。它们需要休息。现在是什么季节?人家是打了针才能表演的。” 
 
 “再打针!多打几针!” 
 
 “呸!这又不是机器。有血有肉,和人是一样的。打你几针试试!你们少说几句坏话,我让甲班把幻灯片拿给你们看。” 
 
 “老师,别听他们跳拔离间!二军子说你坏话,我们开了三次班会批他。他们班唐小丽说你上课吃东西,还说了许老师许多坏话。说许老师等于是说你。你以为他们班好,上大当了!” 
 
 这种话我已经听腻了。所以我这样想:说我坏话就是爱我,说得越多的越甚。到了植物园,我把学生交给带参观的副研究员,自己溜出去看花草。这一溜不要紧,碰上我师傅刘二了。 
 
 我师傅是个奇人,长得一对牛蛋(公牛的蛋)也似大眼,面黑如锅底,疙疙瘩瘩不甚平整。他什么活都会干,但是七五年我进厂给他当徒弟时,他什么活都不肯干。他本是育婴堂带大的孤儿,讨了农村老婆,在乡下喂了几口猪,心思全在猪身上。嘴上说绝不干活,车间主任、班组长逼急了也练几下子,那时节他哼一支小调,曲是东北红高梁的调子,词是自编的。我在一边给他帮腔,唱完一节他叫我一声:“我说我的大娘呀!”我应一声“哎”。我们俩全跑调儿,听的人没有不笑的。 
 
 刘二之歌有多少节我说不清,反正一回有一回的词儿。一唱就从小唱起,说自己是那还用说婊子养的,不走运。接下来唱到进工厂走错了门。我们厂是五八年街道上老娘们组织起来的,建厂时他十五岁,进来当了个徒工。然后唱到街道厂不长工资,拿了十几年的二十六块五。然后唱到老婆找不到。谁也不跟街道厂工人,除了瘸子拐子,要找个全须全羽的万不可能。没奈何去找农村的,讨了个老婆是懒虫。说是嫁汉嫁汉,穿衣吃饭,躺在坑上不起来不说,一顿要吃半斤猪头肉。然后唱到我的两位世兄,前奔儿后勺,鼠眉之极,见了馒头就目光炯炯。这两个儿子吃得他走投无路,要挣钱没路子,干什么都是资本主义(这会儿有人喝止,说他反动了——那是七五年),只剩了一条路养猪。从这儿往后,全唱猪。猪是他的衣食父母。一个是他的爹,长得如何如何,从鬃毛唱到蹄子,他是如何的爱它,可是要卖钱,只好把它阉了。另一个是他娘,长得如何美丽,正怀了他一窝小兄弟,不能亏了它的嘴。否则他弟弟生出来嘴不够大没人买。于是乎要找东西给猪吃,这一段要是没人打断可以唱一百年。刘二唱他打草如何如何,捡菜帮子如何如何,一百多个历险记。唱了好久才唱到他爹娘也不能光吃菜,这不是孝养爹娘的做法,他要去淘人家的泔水。那几年农业学大寨,家家发一口缸,把泔水苦起来支农。天一热臭气冲天,白花花的蛆满地爬,北京城里无人不骂。我师傅也骂,他不是骂泔水缸,而是骂这政策绝了他爹娘的粮草。于是乎唱到半夜去偷泔水。他和我(我有时帮他的忙)带着作案工具(漏勺和水桶),潜近一个目标,听的人无不屏住了呼吸,我师傅忽然不见了。他老人家躲在工作台下边,叫我别做声。这时你再听,有个人从厂门外一路骂进来,是个老娘们儿。另一路骂法,也是有板有眼,一路骂到车间门口。这是泔水站的周大娘,骂的是刘二。她双手叉腰,卡着门口一站,厉声喝道:“王二,你师傅呢?叫他出来!”我说师傅犯了猪瘟,正在家养病,她就骂起来,骂一段数落一段,大意是居民们恨他们,怪他们带来了泔水缸。他们如此受气,其实一个月只挣二十五块钱。三九天蹬平板喝西北风。泔水冻了,要砸冰,这是多么可怕的工程。热天忙不过来,泔水长了蛆,居民们指着鼻子骂。总之,他们已经是气堵了心了。接下来用咏叹调的形式表示诧异:世界上居然还有刘二这种动物,去偷泔水。偷泔水他们还求之不得呢,可这刘二把泔水捞定了还怕人看出来,往水缸里投入巨石泥土等等,让他们淘时费了很多力量。别人欺负他们也罢了,刘二还拿他们寻开心,这不是丧尽天良又是什么。继而有个花腔的华彩乐段,请求老天爷发下雷霆,把刘二劈了。车间主任奔出来,请她去办公室谈,她不去,骂着走了。我师傅从工作台下钻出来,黑脸臊得发紫,可是装得若无其事,继续干活儿。 
 
 我常常劝我师傅别去偷泔水,可以去要,就是偷了也别在缸里下石头。他不听,据说是要讲点体面。当时我不明白,怎么偷还要体面?现在想明白了:泔水这东西只能偷,不能要,否则就比猪还不要脸。 
 
 我师傅为人豁达,我和他相识多年,只见过他要这么点体面。这回我见他的样子,我说了你也不信。他穿一身格子西服,手指上戴好粗一个金戒指,见面敬我一根希尔顿。原来他从厂里留职停薪出来,当了个包工头。现在他正领着一班农村来的施工队给植物园造温室。他见了我有点发窘,不尴不尬地问我认不认识甲方单位(即植物园)的人。 
 
 我说认识一个,恐怕顶不了用。说着说着我也害起臊来,偷泔水叫人逮住也没这样。问候了师娘和两位世兄,简直找不出话来谈,看见我师傅穿着雪白的衬衫,越看越不顺眼,我猜他穿上这套衣服也不舒服。 
 
 我猜我师傅也是这么看我。嘿,王二这小子居然也当了教师,人模狗样的带学生来参观!其实我不喜欢现在的角色,一点也不喜欢。 
 
 四 
 
 晚上到家时,我情绪很坏,下了班以后,校长又叫我去开教务会。与会的乃是各系主任、教务长等等,把我一个室主任叫去实属勉强,再说了,我从来也不承认自己是室主任。全校人都知道我是什么玩艺儿!在会场上的感觉,就如睾丸叫人捏住了一样。 
 
 洗过澡以后,我赤条条走到阳台上去。满天都是星星,好像一场冻结了的大雨。这是媚人的星空。我和铃子好时,也常常晚上出去,在星空下走。那时候我们一无所有,也没有什么能妨碍我们享受静夜。 
 
 我和铃子出去时,她背着书包。里面放着几件可怜的用具:麻袋片,火柴,香烟(我做完爱喜欢抽一支烟),一小瓶油,还有避孕套。东西齐全了,有一种充实感,不过常常不齐全。自从有一次误用了辣椒油,每次我带来的油她都要尝尝才让抹,别提多影响情绪了。 
 
 尽管如此,每次去钻高梁地还是一种伟大的幸福。坐在麻袋上,解开铃子的衣服,就像走进另外的世界。我念着我的诗:前严整后零乱,最后的章节像星星一样遥远。铃子在我身下听见最后的章节,大叫一声把我掀翻。她赤条条伏在地上,就着星光把我的诗记在小本子上。 
 
 我开始辨认星座。有一句诗说:像筛子筛麦粉,星星的眼泪在洒落。在没有月亮的静夜,星星的眼泪洒在铃子身上,就像荧光粉。我想到,用不着写诗给别人看,如果一个人来享受静夜,我的诗对他毫无用处。别人念了它,只会妨碍他享受自己的静夜诗。如果一个人不会唱,那么全世界的歌对他毫无用处;如果他会唱,那他一定要唱自己的歌。这就是说,诗人这个行当应该取消,每个人都要做自己的诗人。 
 
 我一步步走进星星的万花筒。没有人能告诉我我在何处,没人能告诉我我是什么人,直到入睡,我心里还带着一丝迷惘。 
 
 五 
 
 没有课的日子我也得到学校里去,这全是因为我是生物室主任。坐在空荡荡的实验室里打磕睡,我开始恨校长和他的知遇之恩。假如他像我爸爸和我以前的师长一样,把我看成不堪造就之辈,那我该是多么幸福!忽然我妈打电话来,叫我去吃午饭。这是必须要去的。不然她生我这儿子干嘛?我立刻就上路。 
 
 三十三年前,发生了一件决定我终身的大事。那天下午,我妈在协和医院值了个十二小时的长夜班,走回家去,关于那个家,我还有一点印象,是在皇城根一条小胡同里,一间半大明朝兴建的小瓦房。前面房子太高,那房子里完全暗无天日,我妈妈穿着印花布的旗袍,足蹬高跟鞋,小心翼翼地绕过小巷里的污水坑。她买了一小点肉,那分量不够喂猫,但是可以做一顿炸酱面。她和我爸爸吃完了那顿炸酱面,就做出了那件事情。 
 
 我最不爱吃炸酱面,因为我正是炸酱面造出来的。那天晚上,他们用的那个避孕套(还是日本时期的旧货,经过很多次清洗、晾干扑上滑石粉)破了,把我漏了出来。事后拿凉水冲洗了一番,以为没事了,可是才过了一个月,我妈就吐得脸青。 
 
 也许就是因为灌过凉水,我做路梦时老梦见发大水;也许就是因为灌过凉水,我还早产了两个月,我出世时软塌塌、毛茸茸,像个在泔水桶里淹死的耗子。我妈妈见了就哭,长叹一声道:“我的妈!生出了个什么东西!” 
 
 我到东来顺三楼上等我妈,这是约定的老地方。我不能到医院去。因为王二的事迹在那儿脍炙人口。我在那儿的早产儿保温箱里趴了好几个月。当时的条件很差,用的是一种洋铁皮做成的东西,需要定时添加热水。有一回不慎灌入了一桶滚水,王二差点成了涮羊肉。我到医院时,连那些乳臭未干的实习医生都敢叫我“烫不死的小老鼠”! 
 
 我妈定期要和我说一阵悄悄话,这是她二十年来的积习。这事要追溯到二十多年前我上小学三年级的时候。我和我爸爸住在那个小院里,我妈妈住在医院的单身宿舍。我归我爸爸教育,他的方针是严刑拷打,鸡毛掸子一买一打。一方面是因为我太淘气,另一方面因为我是走火造出来的,他老不相信我是个正经东西。 
 
 为了破坏课桌的事,老师写了一封信,叫我带回家。那信被我全吃了,连信皮在内,好像吃果丹皮一样。第二天老师管我要回信,我说我爸爸没写,她知道我撒谎,又派班长再带一封信去,我领了一帮小坏蛋在胡同口拦截,追杀了五里方回。最后老师自己来了。她刚走,我爸爸就拎着耳朵把我一顿狠抽,打断了鸡毛掸,正要拿另一根,妈正好回来。她看见我爸爸揪着耳朵把我拎离了地(我的耳朵久经磨练,坚固异常),立刻惨呼一声,扑过来把我抢下来。接着她把我爹一顿臭骂。我爸爸说这样做是因为“这孩子像土行孙,一放下地就没影儿”,我妈不听,她把我救走了。 
 
 我妈救我到医院,先送我到耳科,看看耳朵坏了没有。大夫对我的耳朵叹为观止,认为这不是耳朵,乃是起重机的吊钩。然后她到房产科要了一张单人床,把我安顿在她房间里。发我一把钥匙,和我约法三章:一是可以不上学,她管开病假条,但是考试要得九十分以上。第二是如果不上学,不准出去玩,以防被人看见。第三是钱在抽屉里,可以自由取用,不过要报帐,用途必须正当。如果没有意见,这就一言为定。违反约定,就把我交给我爸爸管教。我立刻指天为誓道:倘若王二有违反以上三条的行为,情愿下地狱或者和爸爸一块过。我妈大笑,说她真是糊涂,有这么大一个儿子,自己还一个人过。 
 
 我住下来,在女宿舍二楼称王称霸。好多年轻的阿姨给我买零食,听我讲聊斋。白天我经常不在,和夜班护士上动物园了。如此过了一个冬天,觉得女儿国里的生活也无趣,要鼓捣点什么。我妈马上给我找了好几个家庭教师,今天学书法,明天鼓捣无线电,后天学象棋。晚上我妈看医书,我在地上鼓捣玩艺儿。累了大家聊一会儿,我把每位教师的毛病都拿来取笑。我妈听了高兴,把我的脸贴在她乳房上,冬天隔了毛衣犹可,夏天太刺激,我把她推开,她挑起眉毛叫道:“哟!摆架子了!你忘了你叼着这儿嘬了。跟你爸爸学的假正经。好好,不跟你玩了,看会儿书!” 
 
 我的象棋没学成,原因是我师傅不喜欢我的棋风。他老人家是北京棋界的前辈。擅长开局、布局、排局,可惜年老了、血气两衰,敌不过我那恶毒凌厉的棋风。所以他来和我妈说,这孩子天分没得说,可是涵养不够,杀气太盛。让他再长两年,我再给他介绍别的老师。他一走,我妈就问我,是不是在人家家里捣蛋了,这老先生涵养极好,怎么容不下我。我告诉她,我看出老头有个毛病:他见不得凶险的棋局,一碰上手指就打颤。所以我和他对局时专门制造险恶气氛,居然创下了十二局全胜的纪录,我妈妈听了大笑,说我一肚子全是鬼!每次我干了这样的葛事告诉她,她都打个榧子,说:“嘿,这儿子,怎么生的!” 
 
 我在我妈那儿住了三年,头两年还爱把我干的事儿告诉她,听她喝彩,后来就不乐意了。我长大了,生理上发生了变化。最后一个夏天,我看到女宿舍里那些阿姨穿着短裤背心,背上就起鸡皮疙瘩。我也不乐意我妈在屋里脱那么光。有时候她不戴乳罩,我就抗议:“妈!你穿上点儿!”那时候我妈大腿纤长,乳胸饱满,如二十许人,我实在不乐意和她住在一起。我开始要有自己的隐私,上中学时考了个住宿的学校搬了出去。 
 
 从那以后,我们俩之间爆发了长达二十年的间谍战。她想方设法来探我的隐私,我想方设法去骗她。我不记得什么时候在她面前说过实话。 
 
 我妈妈现在也老了,明眸皓齿变成了老眼昏花和一口假牙,丰满的乳房干瘪下去,修长的双腿步态蹒跚。我妈妈超脱了肉体,变成一个漂亮老太大。我爱我妈,我要用我的爱还报她对我三十二年的厚爱,不过我还是要骗她。 
 
 我妈问我为什么星期天不回家,我说是忙。她说再忙也得回家,因为家里那套四室一厅的住宅是以四个人的名义要下来的,现在里面只住了老两口,别人知道了要有意见。这简直不成个理由。我说忙得回不了家也不是理由,其实是我要躲我爸爸的痰气。夫子曰:人之惠在于好为人师——到底不愧是夫子,好大的学问!我搞我的化学,我爸爸搞他的数学,井水不犯河水,他非要问我数学学得怎样。我要说不会,他就发火,说是不学数学能成个什么气候?我要说会呢,那更不得了。他要出题给我做。忙了一星期,回家去做题!这叫什么家,简直是地狱。我妈也知道是这么回事,就说:“你躲你爸爸,可别连我也躲呀!再说你爸爸关心你,你这么计较就不对了。” 
 
 “我没计较。妈,爸爸是虐待狂。他就喜欢看我做不出题出冷汗。其实不是我做不出,是他编的题目不通。我都不好意思说。我要是胡编几道题,他也做不出。让他尝尝这拉不出屎的滋味,你看了一定不忍心。” 
 
 “算了算了,就当陪他玩玩,你何必当真?他这人这样干了一辈子,我都改造不了,别说你了。” 
 
 “他老想证明我一文不值。我说我真一文不值,他还是不干,真不知怎么才能让他满意。他想证明我不如他的一根鸡巴毛。这有什么!三十几年的我还是他射出的一个精虫哩……” 
 
 我妈笑了:“别胡扯!和你妈说这个,是不是太过分?和你说正经事儿。你什么时候生孩子?我想抱孙子。” 
 
 这是个老问题。“妈,我一定生,现在忙,要做大学问,当教授。现在教授香,一分就分一大套房。可是小助教呢?惨啦。我—个同学分到清华,孩子都九岁了,三口人挤一间小房子。三十几岁的人,性欲正强烈,结果孩子到学校里去说:爸爸妈妈夜里又对×了,腆得人家了不得,现在在办公室,趁大家去吃午饭,锁上门急急忙忙脱裤子。办公桌多硬呀!能干好吗?” 
 
 “你跟我说这个干什么!咱们家又不是没地方!” 
 
 “是呀。可房子是爸爸的,又不是我的。那房子多好!水磨石地镶铜条,我看着眼红,也想挣一套。等房子到手,就生儿子!” 
 
 “别胡扯。等你把房子挣下来,我都老死了。” 
 
 “说真的,我看我也不像个当爹的科。瞧你把我生的,没心没肺。再说了,人家没出世就被你灌了凉水,现在做梦老梦见发大水……生个儿子没准是傻子!” 
 
 “别拿这个打掩护,我是干什么的!生孩子我是专家。生吧!好算我的。” 
 
 “我还要造炸药,当了大教授,哪有功夫养孩子?爸爸对我是一种刺激。我非混出个人样儿不可!” 
 
 我妈妈忽然狡黠地一笑,说道:“你别想糊弄我,你的事情我全知道。你呀,要真像所说的那样倒也奇了!” 
 
 我妈说得我心里抨抨直跳:她又知道了我什么事情?自打我上了初中,她无时不在侦察我,我爸爸分了房子,我妈每周到矿院度周末。我自已有个小房间,门上加了三道锁。我妈居然都能捅开,而且捅过一点儿也不坏,简直是妙手空空。我知道她有这种手段,就把一切都藏起来,戒掉了写日记的习惯,重要的东西都留在学校里,可还是挡不住她的搜索。 
 
 那时候,星期六回家简直是受罪,回去要编谎骗我妈,还要和我爸爸抬杠,只要我妈不在家,他就跃跃欲试地要揍我。后来我长了老大的个子,又有飞檐走壁之能,他揍我不着了,就改为对我现身说法。我爸爸有一段光荣历史,从小学到中学从来都考第一名,又以第一名考进了清华。要不是得了一场大病,准头一名考上官费去留洋。按我妈的话来说,我爸爸是一部伟大的机器,专门解各种习题。 
 
 我爸爸还说,他现在混得也不错,住的房子只有前辈教授才住得上。在矿院提起他的大名,不要说教授学生,连校工都双挑大指。他说:“你妈老埋怨我打你,你只要及上我的百分之一,我绝不动你一指头!” 
 
 我爸爸自吹白擂时,我妈坐在一边冷笑。吃完饭我回自己屋去,我妈就来说悄悄话:“别听你爸爸的,他那个人没劲透了;你自己爱干啥就干啥,首先要当个正直的人,其次要当个快乐的人。什么走正路、争头名,咱们不干这事,你是我的儿子!” 
 
 光说这些没什么,她还要扯到不相干的事上去,每次都把我说个大红脸。“我给你洗裤衩,发现一点问题。你感觉怎么样?” 
 
 我立刻气急败坏地喊起来:“谁让你给我洗裤衩?裤衩我会洗!” 
 
 “别这样,妈是大夫,男孩子都有这个阶段,是正常的。要是旧社会,你就该娶媳妇了。” 
 
 “呸!我要媳妇干什么?她算是什么东西!” 
 
 星期一早上我去上学,我妈去上班。我骑自行车,她也骑上一辆匈牙利倒轮闸和我一路走。那还是奥匈帝国时期的旧货,老要掉链子,骑到医院肯定是两手黑油。可她非要骑车上班不可,为的是路上继续盘问我,可是我把话扯到别的地方去。 
 
 “妈,你为什么不和爸爸离婚?” 
 
 “干嘛要离婚?” 
 
 “你要是早和他离了,我也少挨几下打。” 
 
 她笑得从车上跳下去。到了“文化革命”里,她终于知道了我的事情:我和许由玩炸药的事败露了,我被公安局拘了进去。这验证了我爸爸对我的判断;我是个孽子,早晚要连累全家。 
 
 我妈妈始终爱我。她对小转铃说,人生是一条寂寞的路,要有一本有趣的书来消磨旅途。我爸爸这本书无聊之极,叫她懊悔当初怎么挑了这么一本书看。她羡慕铃子有了一本好书,这种书只有拿性爱做钥匙才能打得开。我和小转铃好的事知道的人很少,她居然能打探出来,足见手段高明。我妈妈喜欢小转铃,她说铃子“真是个好女孩”;可是我最后还是搞上了二妞子。这个事里多少有点和我妈抬杠的意思。 
 
 我认为无论是二妞子还是小转铃都不会背叛我,所以很自信地说:“妈,你知道我什么了?” 
 
 “你和你爸爸到底不一样。你是我生的嘛!” 
 
 “怎么啦?” 
 
 “写诗呀,你的诗文我全看过,写得真他妈的带劲。你还说,活着就是要证道,精彩。你还不知道道是什么,告诉你,道就是你妈,是你妈把你生成这样的!” 
 
 她啪一声打个榧子,转瞬之间,年轻时倾国倾城的神采又回到脸上来。我觉得全身的血都往头上涌,差一点中了风。写诗乃是我的大秘密,这种经历与性爱相仿:灵感来临时就如高潮,写在纸上就如射精,只有和我有性关系的女人才能看,怎么能叫我妈见到!我顿时觉得自己成了褪毛的鸡。连个遮屁眼的东西都没有了。桌子上火柴、香烟、筷子劈里啪啦落了一地,我急红了脸吼出来: 
 
 “小转铃这坏蛋!下次见面宰了她。妈,她把我稿子给你了?还给我吧!” 
 
 “稿子还在她那儿,我复印留了底。你想要,拿钱来换,影印费三百元!” 
 
 “太贵了,半价怎么样?算了算了,反正看进你眼里也拔不出来了。你再别提我写的东西,那不是给人看的,行不行?尤其不能给爸爸看,你给他看了我就自杀。” 
 
 “好,不给他看,真怪了,这又不是什么坏事情,你躲我干嘛。你还写了什么?拿来给我看看。” 
 
 从我妈那儿回来,我下了一个大决心,从今以后再不写诗,也不干没要紧的事,我也要像我爸那样定正路,争头名。我的确是我妈生的,这一点毫无问题,我也爱我妈,甚至比爱老婆还甚。但是我一定要证明,我和她期望的有所不同。 
 
 六 
 
 第二天轮到生物室卫生值周。以前卫生值周我是不理睬的,任凭厕所手纸成山。如今不同了。我不能叫人挑了眼去。我提前到校,叫起许由来,手持笤帚开始工作。 
 
 这楼里大小三十个单位,每单位轮一次卫生值周。轮到校长室。校长亲自去刷洗厕所。这是因为学校里人心浮动,校长想收买人心。如今王二想走正路,说不得也要来一回。扫完了厕所,到化学实验室讨了几瓶废酸,把厕所的便器洗得光可鉴人。后来一想,光刷了厕所不成,人家不知是谁干的。我弄来几幅红纸写了大幅的标语,厕所门上贴一张: 
 
 “欢迎您来上厕所!生物室宣。” 
 
 小便池上方贴的是“请上前一步——生物室郑重邀请。” 
 
 厕所门背后是:“再见。我们知道您留恋这优美的环境,可现在是工作时间。何日君再来?生物界同人恭送。” 
 
 隔间里的标语各有特色。男厕所里写着:“大珠小珠落玉盘”,“一片冰心在玉壶“。女厕所里写着:“花径不曾缘客扫,蓬门今始为君开”。还有额匣,“暗香亭”。要说王二的书法,那是没说的。我写碑就写过几十斤纸,眼见厕所像个书法比赛的会场,谁知道校长一来就闯进生物室板着脸喝道: 
 
 “厕所里的字是你写的?” 
 
 “是呀。您看这书法够不够评奖?” 
 
 “评个屁!高教局来人检查工作,限你十分钟,把这些字全刷了!” 
 
 贴时容易洗时难。还没刮洗完,高教局的人就来了,看着标语哈哈大笑,校长急得头上青筋乱蹦。等那帮人走了,校长叫我去,我对他说: 
 
 “校长,不管怎么着,厕所我是洗了。总得表扬几句吧?” 
 
 “表扬什么?下回开会点名批评。” 
 
 “这他妈的怎么整的!您去看看,厕所刷的有多白!算了,我也不装孙子了。以前怎么着还怎么着吧。” 
 
 “不准去!坐下。刷厕所是好事,写标语就不对了。将来校务会上一提到你,大家又会想起今天的事,说你是个捣蛋鬼!你呀,工作没少做,全被这些事抵消了。今后要注意形象。回去好好想想,不要头脑冲动!” 
 
 从校长室出来以后,我恨得牙根痒痒,让我们刷厕所,又不准有幽默感,真他娘的假正经。铃声一响,我扛着投影仪去上课。我想把形象补救过来,课上得格外卖命。这一节讲到微生物的镜下形态。讲到球菌,我蹲下去鼓起双腮;讲到杆菌,就做一个跳水准备姿势;讲到弧形菌,几乎扭了腰;讲到螺旋菌,我的两条腿编上了蒜辫子,学生不敢看;讲到有鞭毛的细菌可以移动,我翩翩起舞:讲到细菌分裂,正要把自己扯成两半儿,下课铃响了。满地是铅笔头,一滑一跤。我满嘴白沫地走回实验室,照照镜子,发现自己像只螃蟹,一拔头发,粉笔末就像大雪一样落下来。刚喘过气来,医务所张大夫又来看我。他说农学系有人给他打电话,说王老师在课上不正常。他来给我量体温,看看是不是发高烧。我把张大夫撵出去,许由又朝我冷笑,我把他也撵出去。自己一个人坐着,什么都不想。 
 
 我忽然觉得恶心,到校园里走走。我们的校舍是旧教堂改成。校园里有杂草丛生的花坛,铸铁的栏杆。教学校有高高的铁皮房顶。我记不清楼里有多少黑暗的走廊,全靠屋顶一块明瓦照亮;有多少阁楼,从窗户直通房顶。古旧的房子老是引起我的遐想,走着走着身边空无一人。这是一个故事,一个谜,要慢慢参透。 
 
 首先,房顶上不是生锈的铁皮,是灰色厚重的铅。有几个阉人,脸色苍白,身披黑袍,从角落里钻出来。校长长着长长的鹰勾鼻子,到处窥探,要保持人们心灵的纯洁。铸铁的栏杆是土耳其刑桩,还有血腥的气味,与此同时,有人在房顶上做爱。我见过的那只猫,皮毛如月光一样皎洁,在房顶上走过。 
 
 你能告诉我这只猫的意义吗?还有那墙头上的花饰?从一团杂乱中,一个轮廓慢慢走出来。然后我要找出一些响亮的句子,像月光一样干净……正在出神,一阵铃响吵得我要抽风。这个故事就俺小王二一样,埋在半夜里的高粱地里了。 
 
 我正好走在大电铃底下,铃声就在我头顶炸响。学生呐喊着从楼里冲出来,往食堂飞奔——这是中午的下班铃。我忽然下定决心:妈的,我回家去。中午饭也不吃了! 
 
 走上大街,看见有人在扫地,我猛然想起今天是爱国卫生日,全城动员,清扫门前三包地段。今天又是班主任与学生定期见面的日子。按学校的统一规定,我该去给学生讲一节德育课,然后带他们去扫地。这对我也是个紧要关头,如果现在溜回家去,以后再也别想当个正经人。 
 
 我犹豫了一会儿,还是回学校去。其实这不说明我有多大决心走正路、争头名,而是因为我觉得下了那么大决心,只坚持了一上午,未免不好意思。吃饱喝足又睡了一觉,我该到班上去。首先找到代理班主任团委书记小胡,问了一点情况,然后就去啦。 
 
 我教四门课,接触两个系八个班,农三乙我最不喜欢。这班学生专挑老师的毛病,教授去上课犹可,像我们这样的年轻教师去上课,十次有九次要倒霉。派我做这班的班主任,完全是个阴谋。但是这节德育课我还得讲呀! 
 
 一进教室我就头疼,上午说我发高烧的,就是这帮家伙。现在他们直勾勾地看着我,千夫所指,无疾而死,这节课下来不知要掉多少头发。我走上讲台,清清喉咙: 
 
 “同学们,男同学和女同学们,也就是男女同学们。我站在这里,看着大家的眼睛,就像看捷尔仁斯基同志的眼睛,我不敢看。不说笑话。从同学的眼睛里,我看出两个问题。第一,你们想问;王老师不是发高烧吗?怎么没死又来了?对不对?班长回答。” 
 
 班长板着脸说:“有同学向医务室打电话,说王老师有病,不代表全班意见,班委开会认为,王老师的课讲得比较活,不是什么问题。打电话的同学我们已经批评他了。” 
 
 “很好。老师的努力得到同学的肯定,别提多快乐。第二个问题,你们想问:这家伙现在来干什么?下节微生物是星期四,我要告诉你,我是你们的班主任。前一段忙,经上级批准,由胡老师代理。从今天开始,我正式接任,今天的题目是道德教育,……班长,什么问题?” 
 
 “老师,你备课了吗!” 
 
 我拼命咽下一句“去你妈的”,说出:“当然备了。虽然没拿教案,可我全背下来了,老师的记性你可以放心,请坐。今天第一次由我来上德育课,我觉得应该沟通沟通,同学们对我有什么意见请提出来。” 
 
 “老师,你是党员吗?” 
 
 “不是,正在争取。谢谢你提了这个问题。” 
 
 “老师,你是否研究生毕业?” 
 
 “不是,本科。年龄大了,不适合念研究生。按上级规定,本科毕业可以教基础课,还有什么?提具体点儿。” 
 
 “老师,你为什么说我们是冻猪肉?” 
 
 “我说过这话吗?我只说到了这个班就像进了冷库,你们见了我就像见了吊死鬼。好好,我收回冷库的话。还有什么?” 
 
 他们说不出什么来了,我把脸一板: 
 
 “同学们,我的缺点你们都看见了,你们是优秀班集体。实质怎么样?是不是捧出来的?考试作弊,我亲眼所见。班上丢了东西,用班费补上,不捉贼。歪风邪气够多了。我是你们的班主任,我宣布立即整风。先把贼捉出来,考试作弊也要大整。还有,你们对本系教师毕恭毕敬,专挑外系教师的眼。这叫什么呢?看人下菜碟!明天我就把外系任课老师召来开会,写个意见报校长。我知道有人指使你们,我怕他们也不敢支持学生整老师,我知道有的年轻女教师上了你们的课,回去就哭。教师描眉怎么啦?资产阶级?帽子不小啦。你们是学生还是政治局?这班四十多人要进政治局,也不知中央什么看法。……什么学生?公然调戏老师!哭什么,不准哭!” 
 
 我继续大骂,把恶气出足,然后宣布分组讨论。班干部上前开会,这几个人走过来,乖极了,净说好话。 
 
 “老师,我们怎么得罪你了?这么整我们?” 
 
 “谈不上得罪,为你们好。” 
 
 “老师,我们错了,你原谅我们吧!” 
 
 “原谅不敢当,班风还是要整!” 
 
 拿这种架子,真有一种飘飘欲仙的快感。等把那帮孩子整到又要哭出来,我才松了口。 
 
 “好吧,老师当然要原谅同学了。可是你们为什么要和老师作对!老实说出来!” 
 
 这事不问我也明白,无非是有人看我们这些外校调来的人不顺眼。可恨的是朝学生吹风,说我作风有问题,可能乱搞男女关系。我把脸板下来说: 
 
 “这是放狗屁。我自会找他们算帐。只要你们乖乖的,我绝不把你们扯进去,以后这种话听了要向我汇报,我是班主任。现在,少废话,上街扫地!” 
 
 我带学生上街,军容整齐,比别的班强了一大块。我亲自手持竹答帚在前开路。直扫得飞沙走石,尘头大起。扫了一气,我把扫帚交给班长,交待了几句,就去找校长汇报。一见面他就表扬我今天德育课上得不错,原来他就在门外听着。我把从学生那儿听来的话一说,他连连点头: 
 
 “好,这些人大不像话,拉帮结派,这事我要拿到校长办公会上去说。小王呀,这么工作就对了。像早上在厕所贴标语,纯属胡闹。” 
 
 “报告校长,说我作风有问题,这叫无风不起浪,老姚这老小子也得整整,他净给我造谣!” 
 
 “老姚的情况不同,这个同志是很忠诚、很勤奋的。他能力低一点,嘴上又没闸。学校里案子多,他破不了心急,乱说几句,你别往心里去。还有个事儿要和你商量:昨晚上他巡夜摔伤了,你知道吗?” 
 
 “不知道,要是知道了,还要喝两盅。这种人乃是造大粪的机器,还当什么保卫科长。你和我商量什么?” 
 
 “他伤得不轻,胯骨脱了臼,医院要求派人陪床。老姚爱人陪白天,咱们派人值夜。” 
 
 “这是医院的规矩,咱就派人吧。不过,这事和我有什么关系?” 
 
 “有关系。老姚是校部的,你们基础部也是校部的,校部的小青年都不肯陪老跳,你来带个头好不好?你一去,别人谁也不敢说不去。” 
 
 我叫起来:“别×你那亲爱的……”我本想说“×你妈”。又想到是校长,就改了口:“我的意思是说,我很尊敬您的妈。你说说看,凭什么叫我去看护他?” 
 
 “瞧你这张嘴!对我都这样,对别人还了得吗?我和你说,现在上面要学校报科研项目,咱们也不能没有。我们准备成立个研究所,把各系能提得起来的项目往一块凑凑。你搞炸药恐怕还得算主要的一个,先搭个架子,怎么样?” 
 
 “不怎么样,我能在这楼里造炸药吗?” 
 
 “谁让你在这儿做实验?实验还去矿院做,咱们只是要个名义,有了名义就可以请求科研经费。将来我们也要盖实验楼,买仪器设备,这都是进一步的设想了。所长的位子吗,只能空一阵子,副所长我准备让你当,因为只有你有提得起的项目。这可提了你好几级,将来评职称、出国进修你都优先。看你的样子好像不乐意,真不识抬举!” 
 
 “我没说不乐意呀!” 
 
 “可光我想提你不成。你想别人怎么看你!像你现在这样子。我提也白搭。从现在到讨论定所的领导班子,还要几个星期。你得有几样突出表现,才能扭转形象。眼前这老姚的事,简直是你的绝好机会。叫你去你还不去,你真笨哪!” 
 
 “照你这么说,我还真得去了。我爸爸病了,我要去陪,说用不着我。这老姚算个什么东西,居然要抢我爸爸上风!我还要给他擦屁股,真跌份儿!我什么时候去?” 
 
 “今晚上就找不着人,你去吧。明天我派许由。你们俩去了,别的坏小子也都肯去了。” 
 
 学好真不容易,除了和学生扯淡,还得给老姚擦屁股,而且我还要感谢老姚摔断了腿,给我创造了机会。回到实验室,我给老婆打电话,说我不回去了。她二话没说,咔嚓一下把话筒搁下。我又对许由说这事儿,他默默地看了我好半天,才冒出一句:“王二,你别寒碜我啦。”说完了晚饭,我就出发上医院。 
 
 七 
 
 老姚要是不给我造谣,就是个很可爱的老头儿。他长着红扑扑的脸儿,上面还有一层软软的茸毛,一副祖国花朵的嫩相,他有几根长短不齐的白胡子,长得满险都是。此人常年戴一顶布帽子,鼻梁上架上了个白边眼镜,在校园里悄悄地走来走去,打算捉贼。我们学校里贼多极了,可他就是捉不到。一般机关单位的保卫科也都很少能捉到贼,主要起个吓阻作用,可我们的老姚不但不能吓阻,自己还成了贼的目标。只要他一不注意,洗脸的毛巾就到浴室里成了公用的,大家都拿它擦脚。老姚把它找回来,稍微洗洗再用,结果脸上长了脚廯,偷他毛巾的就是他的助手王刚。王刚这小子太不像话,老姚摔伤了他也不去看着。说是丈母娘从外地来北京,他要去陪着,其实他丈母娘来了有半年了,他纯粹是找借口。 
 
 老姚自己捉不到贼,就发动群众帮他捉。无论是全校大会、各系的会,甚至于各科的会,他都要到会讲话,要求大家提高警惕,协助捉贼。他又是个废话篓子,一说就是一个钟头还没上正题,所以大家开会都躲着他。我们基础部开会,就常常躲到地下室,还派人在门口放哨,一见老姚来了,立刻宣布休会。他还做了十几个检举箱到处安放,谁也不往箱里投检举信,除了男厕所里那一个,有人做了仿古文章:“老姚一过厕所之坑,纸篓遂空。”简直是亵渎古人! 
 
 这些都是他的事,不是我的事。只可恨他捉不到成还顺嘴胡说。学校里一丢东西,他就怀疑是校工里小年轻的偷了。这也不能说没有道理,他有公安局公布的数字为证:去年全市刑事犯罪者百分之八十是青少年,青年工人又占到第二位,占第一位的青年农民我们学校里没有。他又进一步缩小怀疑圈,认为锅炉房那几位管子工年龄最小,平时又吊儿郎当不像好人。一丢东西,他就说他们几个偷的。人家怎肯吃这种哑巴亏?正好厕所下水道堵了,用竹片捅不开,管工弟兄们刨开地面,掏出一大团用过的避孕套,有几十个。这帮人就用竹杆挑着进了保卫科,往办公桌上一摔,摔得汁水四溅,还逼着他立即破案,否则下水道再堵了,就叫老姚去刨地。然后老姚就来破避孕套的案。他也不知怎么就想到学校里还有生物室,拿了那些东西来找我化验。正好一进门,听到许由和我开玩笑,说那些东西里有我一份。这可不得了,老姚当了真,到处去讲我作风有问题,谣言这东西是泼水难收,到现在我还背着黑锅。平时我恨不得掐死他,现在他住医院我去看护,你看我是不是吃错药了? 
 
 我到医院去,向门房打听老姚。人家说记录上无此人,可能已经拖走了。我知道这医院不怎么样,可是一下午就把老姚治死,也太快了点儿。再问时,人家问我什么时候送来的,我说早上送来的。他又问我们认不认识院长大夫,我说都不认识。他说那准是躺在急诊室里。要是不赶紧托人找关系,病人还要在急诊室里一直躺下去。我去找急诊室,顺着路标绕来绕去,一直走到后门边上,找到一间房子门上挂着急诊室的牌子,可是怎么看这房子都是太平间。看来原来的急诊室在翻盖,急诊病人向死人借位子。我在门前欲进又迟,心里狂跳不止,和第一次与铃子搭话时的心境相仿。 
 
 我第一次和铃子搭话,预先找过无数借口,可是都觉得不充分,不足以掩饰我要搞她的动机;那年头男女青年要不是为了这样的目的,可以一辈子不搭话。同理,今天我来看着老姚,也没法掩饰我要装好人、往上爬的动机。我和他非亲非故,平时还有些宿怨,我来干嘛? 
 
 从小学我就会挖苦先进的小同学,那些恶毒之辞现在不提也罢。现在我骑虎难下,前进一步,我骂人的话全成了骂自已,要是走了呢?呸!更不成个体统。 
 
 我开始编些借口。我要这么说;“姚大叔,校长叫我来照看你。这话就和旧社会新房里新郎说过的一样。他和个陌生女孩待在一起,不好意思了,就这么说:“父母之命,媒妁之言……”你看他多干净,其实过一会儿,他就要操人家。新郎倌的话是自欺欺人,我的话也是自欺欺人,我身后又没有两个武装警察押送,要是不乐意,可以不来呀! 
 
 我还可以说:“老姚,听说你病了没人照看,我心里不安。我们八十年代的青年,照顾有病的老人是我的本分,”这话很好,怎奈我不是这样的人,不合身分。还有一种说法比较合理,“老姚,咱们是同事,我又年轻,该着我来。”不过王刚怎么不来说这话?算了算了不想这么多。我先进去,到时候想起什么说什么。 
 
 一进急诊室,吓了我一跳。这是间有天窗的房子,天花板上一盏水银灯,灯光青紫,照得底下的人和诈尸的死人一般无二。有若干病人直挺挺躺在板床上,那床宽不过二尺,一头高一头低,板子薄得叫人担心。这床看着这么眼熟!小时候我住在医院里,经常钻地下室。有一次钻到太平间里,就看见了这样的床。 
 
 盛夏里我看见过一个年轻的女尸躺在这种床上,浑身每个毛孔都沁出一团融化的脂肪,那种黄色的油滴像才流出的松脂一样。现在躺在床上的人谁也不比她好看,尤其是屋子正中那一位。她是个胖者太大,好像一个吹胀的气球,盘踞在两张床拼起的平台上。她浑身的皮肤肿得透亮,眼皮像两个小水袋,上身穿医院的条子褂,下面光着屁股,端坐在扁平便器上,前面露出花白的阴毛,就如一团油棉丝。老太大不停地哼哼,就如开了的水壶。已经胀得要爆炸了,身上还描着管子打吊针,叫人看着腿软。幸亏她身下它在哗哗地响,也不知是屙是尿,反正别人听了有安全感。其他病人环肥燕瘦各有态,看架式全是活不长的。 
 
 这屋子里的味儿实在不好,可说是闻一鼻子管饱一辈子。屎尿、烂肉、馊苹果、烂桔子汇到一块儿,我敢保你不爱闻。声音也就不必细讲,除了几位倒气的声音,还有几个人在哼哼。顶难听的是排泄的声响。我向门口陪床的一个毛头小伙打听是否见过一个断了腿的红脸老头儿,他说在里面。我踮脚一看,果然,老姚和他老婆在里面墙角,那边气味一定更难闻。我先不忙着进去,先和脸前这小伙子聊一会。我敬他一支烟,他一看烟是重九牌的,眼睛就亮了。 
 
 “你在哪儿买的?” 
 
 “云南商店呗。您这是陪您的哪一位?” 
 
 “姥姥呗,喉癌,不行了,哥儿们,云南商店在哪儿呀?” 
 
 “大栅栏,去了一打听谁都知道。叼呀,这地方这么糟模,您还不如把她拉回去。” 
 
 “家里有女的,害怕死人。这一屋子差不多都是要死的,家里放不下,弄到医院又进不了病房,躺在这儿倒气儿。我们快了,空出地方来你们可以往这边搬,空气好多了。” 
 
 那位姥姥忽然睁开眼,双手乱比划。这个老太太浑身成了红砖色,嘴里呼出癌的恶臭,还流出暗红色的液体。她像鲶鱼一样张口闭口,从口形上看她在大呼要回家。那位毛头小伙低头和她说:“姥姥,您忍一忍,这儿有这玩艺(小伙子用手捏捏老太太鼻子上的氧气管),您插上舒服一点呀!” 
 
 老太大嘴乱动,意思是说你们的话我全听见了,她要还能发声,一定要把这不孝的外孙大骂一顿。可惜她只能怒视。她还用充满仇恨的目光扫了我一眼,吓得我赶紧走开。看看这一屋子人,都是叫那些怕见死人的女人轰出家门的,真叫人发指!女人呀女人,是她妈的毒蛇! 
 
 走到老姚面前,我正要搜索枯肠,编一句什么话,老姚的老婆倒把我的话头抢过去了。 
 
 “你就是学校派来陪床的吧?怎么不早来!老姚给你们学校守夜,摔断了腿,就这么对待他!老实告诉你,不成!赶紧把他送到病房里去!” 
 
 她这么咄咄逼人,把我气坏了:“姚大嫂,这话和我说不着,你去找我们校长好不好!” 
 
 “明天我就去,这叫怎么一回事?你们学校这么没起子?老姚一个党委委员,病了就往狗窝里送?” 
 
 这话很有道理。我要是病了,也要躺在这狗窝里,应该支持老姚老婆去找领导大打一架。我说:“你去闹吧,这年头撑死胆大的饿死胆小的。你去闹了以后,学校兴许能把老姚送到北大医院去。” 
 
 她走了,老姚睁开一只眼看看我,又闭上了。他和我没话可讲。我拍拍他的腿说:“要尿叫我一声啊!”就闭目养神。过了一会儿,只觉得气味和声音太可怕。一睁眼,正看见几个人把个病人往外送,是个老得皮包骨的老头子,已经死掉了。我想到外边走走,老姚一把扯住我,气如游丝地说: 
 
 “别走!我一个人躺着害怕!” 
 
 真他妈的倒霉,我又坐下,忽然想起李斯的名言:人之不肖如鼠也!这是他老人家当仓库保管员时的感慨。他是说,有两种耗子。粮库里的老鼠吃得大腹便便,官仓几年不开一次,耗子们过得好似在疗养,闲下来饮酒赋诗,好不快活。可是厕所里的老鼠吃的是屎,人上厕所就吓得哇哇叫,真是惨不忍睹。于是他就说:人和他妈的耗子一样。混得好就是仓房鼠,混得不好就是厕所鼠。这话讲很有勇气!基督徒说,人是天主的儿女;李斯说,人和耗子是一个道理。比起来还是我们的祖先会写文章,能说明问题。我一贯以得道高人自居,从来没在耗子的高度上考虑问题。可是面对这个急诊室,真得想一想了,说这里是茅坑一点也不过分。要是我到了垂危时,也挺在这么一个木板床上听胖老太大哗哗响,这是什么滋味?就算我是诗人,可以把它想象成屋檐滴水〔有这么一支吉它曲,美不胜收),可是隔一会就有山洪暴发之声,恶臭随之弥漫,想象力怕也无法将之美化。那时候每喘一口气就如吞个大铁球,头晕得好似乘船通上了八级风,还要听这种声音,闻这种气味,我这最后一口气怕也咽不下去。我的二妞子(她已经白发苍苍)俯在我身上泪如泉涌,看我这惨相,恨不得一刀捅死我,又下不了手,这种情景我不喜欢,还是换上一种。 
 
 再过五十年,王二成了某部的总工程师,再兼七八个学会的顾问,那时候挺在床上,准是在首都医院的高干病房里。我像僵尸一样,口不能盲,连指尖也不能动,沙发床周围是一种暗淡的绿光,枕头微微倾斜,我看见玻璃屏后的仪器。我的心在示波器上跳动。 
 
 一个女护士走进来,她化了妆,面目姣好,是那种肉多的女人。乳房像大山,手臂肉滚滚。她解开我的睡衣,把它从我身上拽出去。啊呀王二,你怎么成了这个样子!胸膛上的皮皱巴巴,肚皮深陷下去。腿呀腿,就如深山中的枯木,阴毛蓬蓬,没几根黑的。那活儿像根软软的面条。我不明白,一米九十的身高,老了怎么缩得这么短?女护士用一根手指把我掀翻过身来,在我背上按摩。这可是女人的手!王二老到八十五,也是个男人。可是就是反应不起来。她又把我翻起来,按摩我的胸前,手臂。心狂跳起来,可是身体其它部分木然不动。只有尿道发热,一滴液体流出来。她按摩完毕,忽然发现我身体的异常,“咳”了一声。嘻嘻,谁让你拨弄我?王二还没死。那女人拿出一个棉球,把我龟头擦干净。然后把它轻巧地弹入废纸篓。王二,你完了!脸也臊不红,实在是太老了。她给我穿上衣服,就出去了。我猛然觉得活够了,就想死,示波器上的心脏不跳了,警报声响成一片。白衣战士们冲进来,在我手上、腿上、胸上打针,扣上氧气面具,没用了!仪器上红灯亮了。一个时钟记下时间。几名穿毛料中山装的人进来,脱帽肃立。十二点五十七分二十七秒,伟大的科学家,社会活动家,中国科学界的巨星王二陨落了。然后干部们退出。护士们一齐动起手来,脱下睡衣,把我揿翻过去。掰开屁股,往直肠里塞入大团棉花。这感觉可真逗!然后又掀翻过来,往我身上狂喷香水,凉飕飕的,反正她们不怕我着凉。一个漂亮小护士把我那活儿理顺,箍上一条弹力护身,另有几个人在我肚皮上垫上泡沫塑料。然后把上身架起来,穿衬衣,路上套上西装裤。上身穿上上衣,打上领带。嘿!这领带怎么打的!拴牛吗?你给你丈夫打领带也这样!任凭我大声疾呼,她浑然无觉。又来了个提皮箱的中年人,先给我刮脸,又往我嘴里垫棉花,这可不舒服。快点!我要硬了!涂上口红,贴上假眉毛。棺材拾进来,几个人七手八脚把我往里拾,西式棺材就是好,躺着舒服。在胸袋里插上一朵花,胸前放上礼帽。再往手里放一支手杖,拿了到阴间打人。嘿嘿,王二这叫气派!同志们,这就叫服务!现在可以去出席追悼会了! 
 
 脑袋嘭一下撞在木板床上,我又醒过来。我困极了,恨不得把老姚从板床上揪下来,自己睡上去。起来看看周围的人,全都睡了,就连那个胖老太太也坐在便盆上睡了。就在我打磕睡这一会儿,屋里又少了好几个人。门口那个和我一块抽过烟的小伙子和他姥姥都不见了,那个女人现在在天国里。我再也坐不住了,到院子里走走。 
 
 夜黑到发紫,星星亮得像一些细小的白点。在京郊时我常和铃子钻高梁地,对夜比一般人熟悉很多。这是险恶的夜,夜空紧张得像鼓面,夜气森森,我不禁毛发直立。 
 
 在这种夜里,人不能不想到死,想到永恒。死的气氛逼人,就如无穷的黑暗要把人吞噬。我很渺小,无论作了什么,都是同样的渺小。但是只要我还在走动,就超越了死亡。现在我是诗人。虽然没发表过一行诗,但是正因为如此,我更伟大。我就像那些行吟诗人,在马上为自己吟诗,度过那些漫漫的寒夜。 
 
 我早就超越了老鼠,所以我也不向往仓房。如果我要死,我就选择一种血淋淋的光荣。我希望他们把我五花大绑,拴在铁战车上游衔示众。当他们把我施上断头台时,那些我选中的剑子手——面目娟秀的女孩,身穿紧绷绷的黑皮衣裙,就一齐向我拥来,献上花环和香吻。她们仔仔细细地把我捆在断头桩上,绕着台子走来走去,用杠刀棍儿把皮带上挂的牛耳尖刀一把把杠得飞快,只等炮声一响,她们走上前来,随着媚眼送上尖刀,我就在万众欢呼声中直升天国。 
 
 我又走回急诊室,坐在板凳上打盹。早上八点钟,老姚的老婆才来换我,我困得要死,回家太远了,就骑车上学校,打算在实验室里打个盹。 
 
 走在大街上,汇入滚滚的人流,我想到三十三年前,我从我爸爸那儿出来,身边也有这么许多人,那一回我急急忙忙奔向前去。在十亿同胞中抢了头名,这才从微生物长成一条大汉。今天我又上路,好像又要抢什么头名,到一个更宏观的世界里去长大几亿倍。假如从宏观角度来看,眼前这世界真是一个授精的场所,我这么做也许不无道理,但是我无法证明这一点。就算真是如此,能不能中选为下一次生长的种子和追名求利又有什么关系?事实上,我要做个正经人,无非是挣死后塞入直肠的那块棉花。 
 
 我根本用不着这么做,我也用不着那块棉花,就算它真这么必要,我可以趁着还有一口气,自己把它塞好,然后静待死亡。自己料理自己的事,是多么大的幸福:在许由那张臭烘烘的床上躺下时,我还在想:我真需要把这件事想明白,这要花很多时间,眼前没有功夫,也许要到我老了之后。总之,是在我死之前。

%% 通篇读了一遍,现在这个字体读着舒服,方方正正的字,Emacs显示汉字真是诡异。写这篇小说的背景,我猜是
%% 作者决定辞去教职工作做专业作家的时候,怀疑、还有那么一点坚定。还是想说,这个不适合小孩子看啦!
%% 顺便修改了一些明显不通的词语,该是手录造成的。
%% Last updated: 2008|05|17  03:04:56 by Van Tae
