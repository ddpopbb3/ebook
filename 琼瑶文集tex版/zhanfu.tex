\chapter{战福}

来吧,孩子,让我们一起升到高空,来看看脚下的大地吧。 

在金色的阳光照耀下,翠绿的山峦显出琉璃瓦的光泽,蓝色的大河在它们中间像一条条巨蟒般缓缓的爬动。偶而,群山中的湖泊猛然发出镜子般的闪光。 

在陆地的尽头,大海蔚蓝色的波涛中间,有一条狭长的陆地,好像大陆朝海洋的胸膛伸出去的一条手臂。这一块金黄色的土地呀,多少黄昏,多少夜晚,我就在那里独步徘徊,想念着你们。 

你看到了吗?那墨绿色的一丛,那里是一片高大的杨树和槐树。他们的叶片正在阳光下懒洋洋的耳语。在它的遮蔽下,有一个很大的村庄,我给你们讲的故事就从这里开始。 


战福 

在绿荫遮蔽下的石沟,有一条大路伸过村子,一头从村南的山岗上直泻下来,另一端从村北一座大石桥上爬过去,直指向远方。 

如果是逢集的日子,这条路上就挤得水泄不通。手推小车的人们嘴里怪叫着,让人们让开,有人手挎着篮子,走走停停地看着路旁的小摊,结果就被小车撞在屁股上。人来人往,都从道中的小车两旁挤过,就像海中的大浪躲避礁石,结果踏碎了放在地上的烟叶或者鸡蛋,摆摊的人就绝望地伸手去抓犯罪的脚,然后爆出一阵歇斯底里的尖叫。集市上有一种难以形容的喧哗,你绝不可能从中听出什么来。这地方聋子也不会什么也听不见,不聋的人也会变成聋子,什么也听不见。 

人们都拥挤在供销社和饭馆的门前,刚卖的几个钱就急着把它花了去。凡是赶集的人,都要走过这两个大门,都在柜台前拥挤过,可是都在这两个门之一的前面,看见过一个伤风败俗的家伙。不管什么时候,人们总是看见,他穿着一件对襟红绒衣,脏得就像在柏油里泡过一样。扣子全掉光了,他就用一块破布拦腰系住。再加上一只袖子全烂光了,露着乌黑的膀子,使他活像一个西藏农奴。由于又脏又乱的头发长过了耳朵,所以对于他的性别,谁也得不出明确的概念。一条露着膝盖的破裤子大概原来就是黑的,否则也要变黑。这条裤子所以还成为裤子,就因为它只是裤裆下后面开了花。如果前面也破得那么厉害,就要丧失一件裤子的主要作用了。他全身的皮肤上大概积有半厘米厚的污泥,手背和脚背上更厚一些。在摩尔人一样黑的脸上,浓重的眉毛下,一双呆滞的眼睛,看着人们上空大概十米的地方。 

这就是石沟村的战福,大概姓初。每隔五天,他准要站在那个地方,成为石沟逢集的一个重要标志,就像那一天集上会有很多的人,很多待买的东西一样,使人不能忘怀。所以有一天,在那个地方,站的不是战福,而变成了一条毛片斑驳的黑狗时,人们就感到吃惊,想要明白发生了一些什么事情。 

在弄明这件事情之前,我先要说明,战福是男的。 

当初,他爹在世的时候,他也曾经像个人样。也就是说,衣服常常比较干净,脚上比现在多了双鞋。夏天,他穿的是一件白布小褂,那条黑裤子比现在像样的多。头发经常理,隔三五天还洗洗脸。除此之外,其它的差别就不太多了。 

当初,他爹在世的时候,他也曾经像个人样。也就是说,衣服常常比较干净,脚上比现在多了双鞋。夏天,他穿的是一件白布小褂,那条黑裤子比现在像样的多。头发经常理,隔三五天还洗脸。除此之外,其它的差别就不太多了。 

他爹六一年死了,给他留下了两间摇摇晃晃的破草房,快空的粮囤和一个分遗产的哥哥。他妈死的很早。可是他不能埋怨他爹留下的东西太少,他有什么理由去埋怨一个因为要把饭留给儿子们吃,结果得了水肿病,躺在冷炕上的父亲呢?而且,就是在弥留之际,父亲还把头从战福手上的粥碗前扭开,说是不管用了,留着你们吃吧。对于这样一个父亲,战福除了后悔平日争吃的和哥打架之外,还能有什么呢? 

第二年光景好了,可是父亲已不可能再活。哥哥的岁数已经不小,必须盖几间新房子了。战福已经十六岁,在生产队也算一个六分劳动力。每天晚上下工之后,乘着天黑前一点微光,人们总能看见这哥俩在从山上往下推石头,给未来的房子打基础。盖一幢新房子要好多石头呢。如果需要到外村去推石头和砖瓦,永远是战福一人去。因为他在生产队里挣六分,其实干起活来,不比哥哥差多少。 

就因为这哥俩拼命的干活,所以家里乱成了一锅粥。战福的衣着那时就和现在有点像了。他们有时早上不吃饭,有时中午不吃饭,有时一天只吃一顿饭。即使吃饭,也不刷锅。炕席破了,碎了,成了片片了。被子破了,黑了,成了球了。衣服破了,从来不补。哥哥为了漂亮,总是穿新的,战福以白的为满足。他倒很识大体,知道哥哥要讨媳妇了,不能穿得太糟糕。 

他们房子盖成了,就在旧房子的旁边,两幢房子合留一个院子。新房子石头砌到腰线,新式的门窗,青瓦的顶,在当时的胶东农村,真是不可多得的建筑。 

战福和他哥哥一起搬了进去。没用多久,这间房子就和过去的草房一样,弄得猪都不愿意进去。直到新嫂子过了门,家里乌七八糟的情况才好转。原来战福的哥哥二来子的老婆最爱整洁。可是战福仍然旧习不改。二来子的老婆就让二来子和战福分家,叫战福搬到小屋去住。终于,因为生活有人照顾而美得要命的战福,终于发现了嫂子经常给他脸色看,而且把他脱下的脏衣服毫不客气地团起来扔到炕洞里。战福鲁钝得毫无觉悟,结果有一天嫂子毫不客气讲出来,让他搬出去!理由是她不能侍候两个人,再说战福已经大了,不能总住哥嫂家里。 

战福看着凶神恶煞般的嫂子和不敢置一辞的哥哥,惊得瞠目结舌,气得口眼歪斜。结果还是乖乖地搬了出去。 

据人们议论,二来子把战福撵出去,是为了免得将来战福要盖房子有很多麻烦和花销。据此我看,二来子不一定想把战福撵走,他们弟兄感情倒不坏。问题还在他老婆身上。不过二来子也不是什么好家伙,看着老婆把兄弟赶走不说话,分明也是怕给战福盖房。我觉得二来子毕竟还是有情可原:谁要是像他那么样在人家下工后没夜拉黑地推过石头,拉过石灰,就会同情拉车的牲口的苦处了。吃过那种苦头的人杀了他也不愿意再吃。 

从此,战福开始三天两头不出工,那身打扮也越来越不成样。言语和行为也开始慌悖起来,也绝少和人们来往。秋天不知道往家弄烧的,春天不知道往自留地里种菜,其实一个十七岁的孩子也不懂这些。他开始偷东西,于是又常挨打,结果越来越不像个人。 

就这么过了十年,他就成了现在这么个样子:三分人,七分鬼。最近三年他共出了二十天工,好在队里因为他是孤儿救济点,哥哥还有点良心,有时送点饭给他。不然,他早就饿死了。平时,他到处游手好闲。每逢赶集,他就像个傻子一样的站在那里。可是最糟糕的是他又不疯不傻,想想他过的日子,真叫别人也心里难受。 

有一天,西北来的狂风在大道上掀起滚滚的黄沙。风和路边的杨树在空中争夺树叶,金黄色的叶片像大雪一样飘落下来。一阵劲风吹过,一团落叶就像旋风前的纸钱灰一样跳起来狂舞,仿佛要把人撞倒。大路上空无一人,就连狗们也被飞沙赶回家去了。 

可是战福不愿意回家。那两间破败的小屋,那个破败的巢穴,就是战福也不愿意在里面呆着。他在供销社里走来走去,煞有介事地看着柜台里的商品,一只手在衬衣里捉拿那些成群地乱爬的虱子。石沟的供销社相当的不小,从东到西头足有三十多米,平时站在柜台后面的售货员也有十五六个。不过我要说,他们之中有几个很够枪毙的资格。上午九点钟上班,十一点他们就把当天的帐结清了,钱点好了,下午谁来买东西,他就有本事不卖给你。你叫他拿什么来看看,叫三遍,他把头转过去,再叫几遍,他又把头转过来,厚颜无耻对你瞪大眼睛,好像他是一头驴似的。其中有一个女的叫小苏,如果杀人不偿命,准有人来活剥她的皮。看起来,很朴实可爱的样子,让人有些好感,其实,是个最无耻的骚娘们。 

这一天,供销社总共也没有几人来光顾。天渐渐的黑了,柜台后面那些没人味的东西干干地坐了一天,无聊得要发疯。有人伸懒腰,有人双手扶着柜子,扭着腰,样子恶心得吓死人。小苏打呵欠,眼泪都流出来了,好像鼻孔里进了烟末子。她看看手表,又看看窗外,居然很盼着有人来买东西。因为他们这些人之间再也谈不出什么有意思的东西,如果有人来买东西,就是不是熟人,说不上话,也可以散散心。 

可是时间一分分地过去,没有什么人来。只有战福在屋子走来走去,好像一个鬼一样瞪着大眼到处看。 

小苏眼睛猛的一亮,看出战福可以拿来散心解闷,她叫:“战福,过来!” 

战福猛的站住了,身上莫名其妙地打了个寒噤。谁叫他?是小苏吗?怎么会是小苏?战福扭过头来,却看见小苏在对他招手,而且满脸堆着笑。 

战福小心翼翼地朝她走去,好像一条野狗走向手里拿着肉片的人。他不知小苏要和他说什么。也许他不知不觉中冒犯她了?总之,这类人对他总不会有什么好意的。但她脸上明明堆着笑。 

等他走到柜台前面,小苏就肉声说道:“战福,你为什么这么脏啊?” 

战福脸变紫了。并不是因为脸红的怎么厉害,也就是一般的红法。不过他脸上固有的污黑和红色一经混合,就是紫的。对了,他为什么这么脏呀? 

“真的,战福,你要是把脸洗干净,头发理一理,还是很飒利的呢!” 

供销社里响起了一片笑声。战福的脑子里也在嗡嗡地响。卖书本文具的小马(他也很够枪毙的资格)也走过来凑趣:“战福,回去把脸洗干净,头发理一理,打扮得漂漂亮亮地来。” 

小苏猛的像恶狗一样瞪起眼睛:“小马,你想放个什么屁?” 

“嗯?怎么是放屁?你心里想说的不好说,我替你说就是放屁?战福,你福气不小啊!我们这位苏小姐看上你啦!” 

“哈哈哈!!!”供销社里全体人中猪狗都笑得前仰后合。小苏老着脸皮说:“笑什么,人家也是个人!” 

“哈哈哈!!!”全部人中猪狗又一次狂笑。小马摸着肚子,揉着眼泪说:“对,对,是个人!战福,回家收拾收拾,苏小姐岁数不小了,也该出门子了!” 

那些家伙笑得几乎断气。小苏的脸也涨红了,但是还是恬不知耻地说:“怎么啦?你比人家强吗?”“呃呀,口气挺硬,你真要跟他?”“真跟他怎么样?”“我买一对暖瓶送你……们!”“哈!哈!”“我要笑死啦!”人中猪狗们说,“让我歇口气吧!” 

小马喘着气说:“哎呀,小苏,你真是'刮不知恬'!”供销社里又一次响起了笑声,可是笑的人少多了。这里有点文化的人毕竟太少。 

战福在笑声中逃离了供销社。那些突然的哄笑声像鞭子一样有力地抽打他。街道上的风用飞扬的沙土迎接他,飞舞的落叶又一直把他伴送到家里。他推开虚掩着的院门,一头钻进他那个破烂不堪的小屋里,躺在炕上,心里难过得要发狂。他想到在供销社里的无端羞辱,又想到自己这些狗一样的日子,就感到心像刀绞一样痛。这倒是不多见的。平时,战福的脑子总是麻木的,不欢喜,也不沮丧。没有热情,也没有追念往事火一样的懊悔。他不向命运抱怨什么,当然也不会为什么暗自庆幸。不分析,也不判断。没有幻想,也没有对往事甜蜜的沉缅。他的脑子是一片真空。 

战福脑海里的翻腾平息下来了。只有往事在头脑里无声上演。嫂子狰狞的面孔,然后是他的破狗窝。懒洋洋、无所作为的感觉。粮食缸空了。可是也不想吃。到人家菜园里偷菜。冬天夜里到人家柴火垛上偷柴。挨打…… 

街门咣当一声响,是上工回来的二来子。战福抬起头来,屋里黑了。肚子有点钝钝的痛,是一天没吃饭了。缸里队上才送了三十斤玉米来,可是要吃还得去磨。唉,再忍一顿吧!战福把破棉花球拉过来,抱在怀里,便昏然入梦了。 

清晨的凉气透过撕破的窗户纸,把战福从梦乡唤起。他从炕上坐起来,环顾着四周,第一次发现,这间屋子实在不像是人住的场所,而像是狗窝猪圈一类的东西。看吧,锅台上长起了青草,窗户上的灰尘也已经足有半寸。由于窗格上和窗户纸上灰土太厚,屋里也是灰蒙蒙的,更增加了灰暗破败的气象。当然了,如果是平常,战福一定是熟视无睹。可是在今天,不知是什么鬼附了体,战福“觉今是而昨非”,居然觉得以往的日子实在过得太恶心了。是什么力量促使他自新了呢?我说不上来,当时战福也说不上来。 

战福起身下炕,首先扫去了多年堆积在地下的灰土。然后扫了扫窗台,又把窗户纸通通撕下来。他铲去锅台上的青草,掏了掏锅底下的陈灰,然后又把缸里担满了清水。看一看屋里,仍然有破败的景象,于是把破棉被扔到了炕旮旯里。然后巡视一下屋里,觉得他的小草房真是一座意想不到的辉煌建筑。 

这时,他的脑子里开始迷惑不解地想:“我要干什么?难道是要像别人一样的生活吗?”其实那最后的半句话根本就没在他脑子里出现,是我加上的。战福想到一半就恐惧地停住了。因为他是这样的一种人,丝毫也不想振作起来,把衣服洗一洗,把锅刷一刷。至于跟大家到地里去干活,更是想都不敢想,一想就要头皮发炸。就是最勤劳的农民,就不过是靠了日复一日不断的劳作,把好安逸的念头磨掉了呢;就是牛,早上被拉出圈时,也是老大的不愿意。就那么日复一日地干活,除了吃和睡什么也不想,然后再死掉?难怪战福不乐意呢! 

不过,谁说什么也不想?这不是污蔑农民吗!就连战福也想过盖个房子,娶个老婆呢!只不过现在没了过分希望罢了。战福现在在炕上坐着,可真是什么也没想。猛然,他的脑子里一亮,似乎觉得置身于青堂瓦舍之中。好美的房子呀!雪白的顶棚,水泥的地。院子里,密密地长满了高大的杨树,枝叶茂盛,就是烈日当空的时候,院子里也只有清凉的、叶片的绿光。 

啊,美哉!战福理想的房屋!地面没有肮脏的泥土,只有雕琢后的条石砌成的地面,被夏日的暴雨冲刷得清清爽爽。 

清凉的泉水环绕着他的院落奔流。院子周围是高大的砖墙。这伟大的房子上空会有喧闹的噪音吗?绝没有!那会打扰了战福先生神圣的睡眠。 

吃什么?偷来的嫩南瓜?老玉米粒煮韭菜?胡说!他想吃罐头。长这么大还没尝过罐头味呢。罐头供销社的货架上就有。可是怎么能拿来?有人坐在前面看着那些罐头呢。吃不着了吗?看着罐头的是谁?坐在那里的人是小苏哇。小苏满面微笑,向他招手…… 

战福浑身发热,推开门就奔了出去,满脑子都是辉煌的房屋,罐头的美味,微笑的小苏,冷不妨一头撞在一个人身上。立刻,身边响起了一个无比可怕的声音:“瞎了?奔你娘的丧!” 

战福战战兢兢地抬头一看,他嫂子正双手叉腰,凶刹一般的瞪着大眼看他。战福今天发现,嫂子居然那么可憎;发黄的头发拉里拉塌地爬在头上,粗糙的面孔,黑里透灰。木桩一般的身段,半男不女。总的印象是:下贱,不值一文。 

战福平时就恨他嫂子,不过还有几分敬畏。可是他居然敢从牙缝里说出两个字:“丑相”,就他自己也很觉得惊奇。但是,他从这两个字里又发觉自己很英勇,伟大。于是,又盯着他嫂子多看了一眼。 

二来子嫂气得发了楞,马上又气势磅礴地反击回来:“王八蛋!你不要脸!你不看看你自己!全中国也没有你这样的第二个!死不了也活不成,丢中国人的脸!” 

战福被折服了,屁滚尿流地逃到街上去。二来子嫂念过小学呢。如今又常常去学习,胸中很有一点全局观念,骂起人来,学校的老师都害怕,何况战福。 

二来子嫂的大骂居然命中了战福的要害,使他像一条挨了打一样气馁自卑。他垂头丧气地走,不觉走到供销社里。 

供销社大概只有八九个顾客,售货员倒有十七八个。小马第一个看见了战福,发出一声欢呼来迎接他的到来:“啊呀!小苏的姑爷来了!”“哈哈哈!”猪狗们发出一片狂笑。 

顾客们大为惊奇:“怎么了,出了什么事?”猪狗们笑着把这件事情添油加醋地宣传出去,为了开心,为了显示自己多么有幽默感。其中小马的声音最响亮:“昨天,昨天下午(他笑得喘不过气来),战福到供销社来,我们的苏小姐一看,那个含情脉脉呀,我可学不来……” 

小苏慌了,昨天只不过是为了骚滴滴地开个玩笑,谁知道今天闹成这个样子;而且要在全公社传扬开了,就这可不好!她像狮子狗一样地跳了起来反击:“小马,你刮不知恬,你刮不知恬!” 

可是她的挖苦真是屁用没有。在场的大家都是喜欢猎取无聊新闻的人中猪狗,所以全都支棱起耳朵听小马的述说:……我要送一对暖壶给他们,小苏替战福嫌少!”“哈哈!”“哈哈!”“小马,你大概是撒谎吧?”全体售货员一起作证说:“是真的!” 

“哈哈哈!”公社副书记乐不可支地拍打自己的大肚子。“嘻嘻嘻”,文教助理员从牙缝里奸笑着。“哈哈,哈哈,哈哈”,学校的孙老师抬头看着天花板,嘴发出单调的傻笑,好像一头苯驴;其它人也在怪笑,都要在这稍纵即逝的一瞬间里,得到前所未有的欢乐。这个笑话对他们多宝贵呀!他们对遇到的一切人讲,然后又可以在笑声里大大地快乐。“哈哈,哈哈哈!嘻嘻嘻!” 

小苏已经瘫倒在柜台上了。人们看看她,又看看战福黑紫色的鬼脸,又是一场狂笑。小苏招招手,把战福叫过来,对他说,声音是意想不到的温柔。“战福,你这两天别到供销社来,啊?” 

别人也许会奇怪,小苏为什么对战福这么和气。原来是战福个儿很矮,脸又太黑,看不出是多少岁。所以,小苏就从他的个儿上来判断他只有十三四岁。因为她到石沟才一年,所以也没人告诉过她战福二十八了。所以她要哄着战福,要他别来。要是她知道战福岁数那么大,就绝不会干这种傻事。 

好,战福离开供销社回家去了,浑身发热,十年来第一次下定了决心,要好好干,把自己弄得像个人样,还要盖三间,不,四间大瓦房。为了他的幸福,为了吃不完的罐头。(说来可笑,他以为买罐头的人可以把罐头随便拿回家去。) 

晚上,人们收工回家的时候,看见有人在山上的石头坑里起石头。(石沟的石头很好打,用铁棍一撬就可以弄到大块的上好石料)。装在一辆破破烂烂的小车上。当人们走近的时候,十分吃惊地看见,那是战福! 

战福满头是汗,勉勉强强把三五百斤石头推到家的时候,已经天黑了。他做了一锅难吃无比的玉米面饼子,把肚子塞饱,就躺在他那破炕上。想着白天在供销社的情景,心头火热。他以为,小苏对他很有意思,但是当着那么多的人,不好意思。可是他就没想想,人家是个什么样的人,以及为什么会看上他等等。他躺在那里,“愈思而愈有味焉”。于是猛然从炕上跳起,找队里要盖房子的地皮去了。 

第二天早上,全村都传遍了战福找大队书记要盖房子的地基的新闻。这又是一个笑话。书记问战福,你怎么想起要盖房子了?他答之曰:要成家立业!何其可笑乃尔! 

这个新闻和小苏在供销社闹笑话的新闻一汇合,马上又产生了一种谣传。以致有人找到在山上打石头的战福问他是不是看上了供销社的小苏,问得战福心花怒放。他觉得村子都传开了,当然是好事将成,竟然直认不讳。 

好家伙,不等天黑战福下山,这个笑话轰动了全村的街头巷尾!供销社里的猪狗们逼着小苏买糖,二来子不巧这时去供销社打酱油,立刻被一片“小苏,你大伯子来啦”的喊声臊了出来。等到天黑,战福回来的时候,刚到门口,就被二来子拦住了。 

他们两人一起到战福的小屋里坐下。二来子问:“兄弟,你是要盖房子吗?”“是呀”。“盖房好哇。你这房子是好另盖了。当哥哥的能帮你点么?”“不用了哥呀。嫂子能同意吗?”“咳,不帮钱物也能帮把力呀。”“好哇哥。少不了去麻烦你。” 

二来子站起身来要走,猛然又回过头来:“战福,有个话不好问你。你是看上了供销社的苏了吗?” 

战福默然不语。不过显出一副洋洋得意的样子。“兄弟,不是当哥的给你泼凉水,你快死了这个心吧。人家是什么人,咱是什么人?给人家提鞋都嫌你手指头粗……”二来子絮叨了好一阵,看看兄弟没有悔悟的样子,叹着气走了。 

第三天早上,当战福推起小车要上山,刚出门就碰上了隔壁的大李子。大李子嬉皮笑脸地对战福说:“战福,你的福气到了!供销社的小苏叫你去呢!她在宿舍等你。”战福扔下小车楞住了。大李子又说:“哎,还不快去?北边第二排靠西第二个门!” 

战福撒腿就跑,一气直跑小苏门前,站在那里呆住了。他既不敢推开房门(小苏在他心目里虽不是高不可攀,也还有某种神圣的味道)也不敢走开一步。倒是凑巧,站了不到半个钟点门就开了。小苏好像要出门,一看见战福,就喝了一声:“进来!” 

战福像一只狗一样进了门,门就砰一声关上了,好像还插死了。他的心脏停止了跳动,脑子发木,扭头一看…… 

小苏呲牙咧嘴,脸色铁青,面上的肌肉狰狞地扭成可怕的一团,毛发倒竖,眉毛倒立着,好像一个鬼一样立在那里。战福的心头不再幸福地发痒了。可是脑子还是木着。 

小苏发出可怕的声音:“战福,我问你,你在外面胡说了一些什么?你胡呲乱冒!啊!你不要脸!你说什么!你妈个+的,你盖你的房,把我扯进去干吗?你说呀!”苏小姐看战福呆着,拿出一根针,一下子在他脸上扎进多半截。 

“战福,你哑巴了!喂!我告诉你(一针扎在胸膛上),不准你再去乱说,听见没有……” 

小苏开始训诫战福,一边说一边用针在他身上乱刺。战福既不答辩,也不回避,连一点反应也没有,完全像一块木头。在我看来,苏小姐这时的行为比较冒险。 

好了,过了两个钟点,苏小姐的训导结束了,战福脸上也有十来处冒出了血珠,身上更不用说。可是战福还是木着,也没有任何迹象证明他对苏小姐的训诫听进了一句。可是苏小姐已经疲倦,手也酸得厉害,于是开开门,把他推了出去。 

后来,有人看见他默默地走过街头,又有人看见他在村外的河边上走,一面撕着衣服,一边狗一样嘶叫着。再以后,就没有任何人看见他了。只有河边找到过他的破衣服,还有就是石沟村多了一条没主的黑狗,全身斑秃,瘦得皮包骨头。每逢赶集,就站在战福站过的地方。没有人看见它吃过东西,也没有人看见它天黑后在哪里。它从来也不走进供销社的大门。过了几个月,人们发现它死在二来子的院子里。 据说二来子因此哭了一场,打了一次老婆,以后关于这条狗,关于这个人,似乎再没有什么可讲的了。
