\chapter{白银时代}

\section*{一`}

大学二年级时有一节热力学课,老师在讲台上说道:“将来的世界是银子的。”我坐在第一排,左手支在桌面上托着下巴,眼睛看着窗外。那一天天色灰暗,空气里布满了水汽。窗外的山坡上,有一棵很粗的白皮松,树下铺满了枯黄的松针,在乾裂的松塔之间,有两只松鼠在嬉戏、做爱。 

松鼠背上有金色的条纹。教室里很黑,山坡则笼罩在青白色的光里。松鼠跳跳蹦蹦,忽然又凝神不动。天好像是要下雨,但始终没有下来。教室里点着三盏荧光灯,有一盏总是一明一灭。透过这一明一暗的快门,看到的是过去发生的事情。 

老师说,世界是银子的。然后是一片意味深长的沉默。这句话没头没尾,所以是一个谜。我把右手从腮下拿下来,平摊在桌子上。这只手非常大,有人叫它厄瓜多尔香蕉——当然,它不是一根,而是一排厄瓜多尔香蕉。这个谜好像是为我而出的,但我很不想进入这个谜底。在我身后,黑板像被水洗过,一片漆黑地印在墙上。老师从讲台上走下来,这位老师皮肤白晰,个子不高,留了一个娃娃头,穿着一件墨绿色的绸衫。那一天不热,但异常的闷,这间教室因此像一间地下室。老师向我走来时,我的脸上也感到一阵逐渐逼近的热力。据说,沙漠上的响尾蛇夜里用脸来看东西——这种爬虫天黑以后什么眼睛都看不见,但它的脸却可以感到红外线,假如有只耗子在冰冷的沙地上出现,它马上就能发现。我把头从窗口转回来,面对着走近来的老师。她身上墨绿的绸衫印着众多的热带水果,就如钞票上的水印隐约可见。据她说,这件衣服看上去感觉很凉快,我的感觉却是相反。绸衫质地紧密,就像一座不透风的黑牢,被关在里面一定是很热的;所以,从里面伸出来的裸露手臂带有一股渴望之意……老师在一片静止的沉默里等待着我的答案。 

天气冷时,老师穿一件黑色的皮衣,在校园里走来走去,在黑衣下面露出洁白的腿——这双腿特别吸引别人的注意。有人说,在皮衣下面她什么都没有穿,这是个下流的猜想。据我所知不是这样:虽然没穿别的东西,但内裤是穿了的。老师说,她喜欢用光腿去趟冰冷的皮衣。一年四季她都穿皮凉鞋,只是在最冷那几天才穿一双短短的皮靴,但从来就不穿袜子。这样她就既省衣服、又省鞋,还省了袜子。我就完全不是这样:我是个骇人听闻的庞然大物,既费衣服又费鞋。更费袜子——我的体重很大,袜子的后跟很快就破了。学校里功课很多,都没什么意思。热力学也没有意思,但我没有缺过课。下课以后,老师回到宿舍里,坐在床上,脱下脚上的靴子,看脚后跟上那块踩出来的红印,此时她只是个皮肤白晰、小腿健壮的小个子女郎。上课时我坐在她面前,穿着压皱的衣服,眼睛睁得很大,但总像刚睡醒的样子;在庞大的脸上,长着两道向下倾斜的八字眉。我的故事开始时,天气还不冷。这门课叫作“热力学二零一”,九月份开始。但还有“热力学二零二”,二月份开始;“热力学二零三”,六月份开始。不管叫二零几,都是同一个课。一年四季都能在课堂上遇到老师。 

我猛然想到:假如不是在那节热力学课上,假如我不回答那个问题,又当如何……我总是穿着压皱的土色灯芯绒外衣出现在教室的第一排——但出现只是为了去发愣。假如有条侏罗纪的蛇颈龙爬行到了现代,大概也是这样子。对它来说,现代太吵、太乾燥,又吃不到爱吃的蕨类植物,所以会蔫掉。人们会为这个珍稀动物修一个四季恒温的恐龙馆,像个蓝球队用的训练馆,或是闲置不用的车间,但也没有什么用处。它还是要蔫掉。从后面看它,会看到一条死气沉沉的灰色尾巴搁在地下。尾巴上肉很多,喜欢吃猪尾巴的人看了,会感到垂涎欲滴的。从前面去看,那条著名的脖子拍在地下,像条冬眠中的蛇,在脖子的顶端,小小的三角脑袋上,眼睛紧闭着——或者说,眼睛罩上了灰色的薄膜。大家都觉得蛇颈龙的脖子该是支着的,但你拿它又有何办法,总不能用吊车把它吊起来吧。用绳子套住它的脖子往上吊,它就要被勒死了。 

我就是那条蛇颈龙,摊倒在水泥地上,就如一瓣被拍过的蒜。透过灰色的薄膜,眼前的一切就如在雾里一般。忽然,在空荡荡的房子里响起了脚步声,就如有人在地上倒了一筐乒乓球。有个穿黑色皮衣的女人从我面前走过,灰色的薄膜升起了半边。随着雾气散去,我也从地下升起,摇摇晃晃,直达顶棚——这一瞬间的感觉,好像变成了一个氢气球。这样我和她的距离远了。于是我低下头来,这一瞬的感觉又好似乘飞机在俯冲——目标是老师的脖子。有位俄国诗人写过:上古的恐龙就是这样咀嚼偶尔落在嘴边的紫罗兰。这位诗人的名字叫作马雅可夫斯基。这朵紫罗兰就是老师。假如蛇颈龙爬行到了现代,它也需要受点教育,课程里可能会有热力学……不管怎么说罢,我不喜欢把自己架在蛇颈龙的脖子上,我有恐高症。老师转过身来,睁大了惊恐的双眼,然后笑了起来。蛇颈龙假如眼睛很大的话,其实是不难看的——但这个故事就不再是师生恋,而是人龙恋……上司知道我要这样修改这个故事,肯定要把我拍扁了才算。其实,在上大学时,我确有几分恐龙的模样:我经常把脸拍在课桌面上,一只手臂从课桌前沿垂下去,就如蛇颈龙的脖子。但你拿我也没有办法:绕到侧面一看,我的眼睛是睁着的。既然我醒着,就不用把我叫醒了——我一直在老师的阴影里生活,并且总是要回答那句谜语:世界是银子的。 


\section*{二} 

现在是2020年。早上,我驶入公司的停车场时,雾汽正浓。清晨雾汽稀薄,随着上午的临近,逐渐达到对面不见人的程度——现在正是对面不见人的时刻。停车场上的柏油地湿得好像刚被水洗过,又黑又亮。停车场上到处是参天巨树,叶子黑得像深秋的腐叶,树皮往下淌着水。在浓雾之中,树好像患了病。我停在自己的车位上,把手搭在腮下,就这样不动了。从大学时代开始,我就经常这个模样,有人叫我扬子鳄,有人叫我守宫——总之都是些爬虫。我自己还要补充一句,我像冬天的爬虫,不像夏天的爬虫。大夫说我有抑郁症。他还说,假如我的病治不好,就活不到毕业。他动员我住院,以便用电打我的脑袋,但我坚决不答应。他给我开了不少药,我拿回去喂我养的那只绿毛乌龟。乌龟吃了那些药,变得焦躁起来,在鱼缸里焦急地爬来爬去,听到音乐就人立起来跳迪斯科,一夜之间毛就变了色,变成了一只红毛乌龟——这些药真是厉害。我没吃那些药也活到了大学毕业。但这个诊断是正确的:我是有抑郁症。抑郁症暂时不会让我死去,它使我招人讨厌,在停车场上也是这样。 

在黑色的停车场正面,是一片连绵不绝的玻璃楼房。现在没有下雨,但停车场上却是一片雨景。车窗外面站了一个人,穿着橡胶雨衣,雨衣又黑又亮,像鲸鱼的皮——这是保安人员。我把车窗摇了下来,问道:你有什么问题?他愣了一下,脸上泛起了笑容,说道:这话应该是我问你才对。这话的意思是说,停车场不是发愣的地方。我无可奈何地耸耸肩,从车上下来,到办公室里去——假如我不走的话,他就会在我面前站下去,站下去的意思也是说:停车场不是发愣的地方。保安人员像英国绅士一样体面,脸上挂着意味深长的微笑。相比之下,我们倒像是些土匪。我狠狠地把车门摔上,背对着他时,偷偷放了个恶毒的臭屁——我猜他是闻到味了,然后他会在例行报告里说,我在停车场上的行为不端正——随他去好了。走进办公室,我在桌后坐下,坐了没一会儿,对面又站了一个人,这个人还是我的顶头上司。她站在这里的意思是说:办公室也不是发愣的地方。到处都不是发愣的地方。我把手从腮下拿出来,放在桌子上,伸直了脖子,正视着我的上司——早上我来上班时的情形就是这样。 

我一直在写作公司里写着一篇名为《师生恋》的小说。这篇小说我已经写了十几遍了,现在还要写新的版本,因为公司付了我薪水,而且不是每个人都有机会和老师恋爱的,所以这部小说总是有读者,我也总是要写下去。 

在黑色的皮衣下,老师是个杰出的性感动物。在椅子上坐久了,她起身时大腿的后面会留下红色的皮衣印迹——好像挨了打,触目惊心。那件衣服并不暖和,我不知道她为什么要穿这件皮衣。在夏季,老师总在不停地拽那件绸衫——她好像懒得熨衣服,那衣服皱了起来,显得小了。好在她还没懒得拽。拽来拽去,衣服也就够大了。 

这故事发生的时节,有时是严冬,有时是酷暑。在严冬,玻璃窗上满是霜窗花,教室的水泥地下满是鞋跟带进来的雪块。有些整块地陈列着,有些已经融化成了泥水——其实,我并不喜欢冷。在酷暑时节,从敞开的门到窗口,流动着乾热的风。除了老师授课声,还能听到几声脆响。那是构成门框、窗框或者桌椅的木料正在裂开。而这一次则是在潮湿的初秋季节。从本性来说,我讨厌潮湿。但我别无选择——因为这是我唯一能选择的东西。在潮湿的秋季,老师说:未来的世界是银子的……这是一道谜语。我写着的小说和眼前发生的一切,全靠这道谜语联系着。在班上,我总对着桌上那台单色电脑发愣。办公室里既没有黑板,也没有讲台,上司总是到处巡视着,所以只有这一样可以对之发愣的东西:有时,我双手捧着脸对它发愣,头头在室里时,就会来问上一句:喂!怎么了你?我把一只手拿下来,用一个手指到键盘上敲字:屏幕上慢慢悠开始出现一些字。再过一会儿她又来问:你干什么呢?我就把另一只手放下来,用两根手指在键盘上敲字,屏幕上还是在出字,但丝毫也不见快些。假如她再敢来问,我就把两只手全放回下巴底下去,屏幕上还是在出字,好像见了鬼。这台电脑经我改造过。原本它就是老爷货,比我快不了好多,改了以后比我还要慢得多。我住手后五分钟它还要出字,一个接一个地在屏幕上闪现,每个都有核桃大小,显得很多——实际上不多。头头再看到我时,就摇摇头,叹口气,不管我了。所有的字都出完了,屏幕变得乌黑,表面也泛起了白色的反光。它变成了一面镜子,映着我眉毛稀疏,有点虚胖的脸……头头的脸也在这张脸上方出现。她的脸也变得臃肿起来。这个屏幕不是平的,它是一个曲面,像面团里的发酵粉,使人虚胖。她说道:你到底在干些什么……她紧追不舍,终于追进了这个虚胖的世界里。人不该发愣,除非他想招人眼目。但让我不发愣又不可能。 

我的故事另有一种开始。老师说,未来世界是银子的。这位老师的头发编成了高高的发髻,穿着白色的长袍。在她身后没有黑板,是一片粉红色的天幕。虽然时间尚早,但从石柱间吹来的风已经带有乾燥的热意。我盘膝坐在大理石地板上,开始打瞌睡,涂蜡的木板和铁笔从膝上跌落……转瞬之间我又清醒过来,把木板和铁笔抓在手里——但是已经晚了,错过了偷偷打瞌睡又不引起注意的时机。在黑色的眼晕下,老师的眼睛睁大了,雪白的鼻梁周围出现了冷酷的傲慢之色。她打了个榧子,两个高大的黑奴就朝我扑来,把我从教室里拖了出去。如你所知,拖我这么个大个子并不容易,他们尽量把我举高,还是不能使我的肚子离开地面——实际上,我自己缩成了一团,吊在他们的手臂上,像小孩子坐滑梯那样,把腿水平地向前伸去。就是这样,脚还是会落在地下。这时我就缩着腿向前跑动,就如京剧的小丑在表演武大郎——这很有几分滑稽。别的学生看了就笑起来。这些学生像我一样,头顶剃得秃光光,只在后脑上有撮头发和一条小辫子,只有一块遮羞布绕在腰上——他们把我拖到高墙背后,四肢摊开,绑在四个铁环上。此后我就呈X形站着,面对着一片沙漠和几只骆驼。有一片阴影遮着我,随着中午的临近,这块阴影会越来越小,直至不存在,滚烫的阳光会照在我身上。沙漠里的风会把砂粒灌进我的口鼻。我的老师会从这里经过,也许她会带来一瓢水给我解渴,但她多半不会这么仁慈。她会带来一罐蜜糖,刷在我身上。此后蚂蚁会从墙缝里爬出来,云集在我身上——但这都是以后的事了。现在有只骆驼向我走来,把它的嘴伸向我的遮羞布。我想骆驼也缺盐分,它对这条满是汗渍的遮羞布会有兴趣——还有一种可能,就是它是只母骆驼……它把遮羞布吃掉了,继续饶有兴致地盯着我,于是我赤身裸体地面对着一只母骆驼。字典上说,骆驼是论峰的。所以该写:“我赤身裸体地面对着一峰母骆驼”,我压低了嗓子对它说:去,去!找公骆驼玩去……这个故事发生在埃及托勒密王朝时期。我的老师是个希腊裔的贵人——她甚至可以是克利奥佩屈拉本人。如你所知,克利奥佩屈拉红颜薄命,被一条毒蛇咬死了。写这样一个故事,不能说是不尊重老师。 


\section*{三} 

办公室里鸦雀无声,就像在学校里的习题课上。如你所知,学校里有些重大课程设有习题课,把学生圈在教室里做习题——对我来说,这门课叫作“四大力学”,一种不伦不类的大杂烩。老师还没有资格讲这样的重大课程,但她总到习题课上来,坐在门口充当牢头禁子的角色——坐在那里摇头晃脑地打瞌睡。我也来到习题课上,把温热的大手贴在脸上,目不转睛地看着她,发现她摇晃得很有韵律。不时有同学走到她面前交作业,这时她就醒来,微笑着说道:做完了?谢谢你。总得等多数人把习题做完,这节课才能结束。所以她要谢谢每个交作业的人,但我总不在其中。每门课我都不交作业,习题分总是零蛋……老师在习题课上,扮演的正是办公室里头头的角色。 

现在头头不在班上,但我手下的职员还要来找我的麻烦。很不幸的是,现在我自己也当了本室的头头,虽然在公司里我还是别人的手下。据说头头该教手下人如何写作,实际上远不是这样。没人能教别人写作,我也不能教别人写作——但我不能拒绝审阅别人的稿子。他们把稿件送到我办公桌上,然后离去。过上半小时,或者一个小时,我把那篇稿子拿起来,把第一页的第一行看上一遍,再把最后一页最后一行看上一遍,就在阅稿签上签上我的名字。有些人在送稿来时,会带着一定程度的激动,让我特别注意某一页的某一段,这件事我会记住的,虽然他(或者她)说话时,我像一个死人,神情呆滞目光涣散,但我还是在听着。过半小时或一小时之后,我除了看第一行和最后的一行,还会翻到那一页,仔细地看看那一段。看完了以后,有时我把稿子放在桌面上,伸手抓起一支红铅笔,把那一段圈起来,再打上一个大大的红叉——如你所知,我把这段稿子枪毙了。在枪毙稿子时,我看的并不是稿纸,而是盯住了写稿人目不转睛地看着,这个被枪毙的人脸色胀红,眼睛变得水汪汪的,按捺着心中的激动低下头去。假如此人是女的,并且梳着辫子,顺着发缝可以看见头皮上也是通红的——这是枪毙的情形。被毙掉以后,说话的腔调都会改变,还会不停地拉着抽屉。很显然,每个人都渴望被枪毙,但我也不能谁都毙。不枪毙时,我默默地把稿件收拢,用皮筋扎起来,取过阅稿签来签字,从始至终头都不抬。而那个写稿人却恶狠狠地站了起来,把桌椅碰得叮当响,从我身边走过时,假作无心地用高跟鞋的后跟在我脚上狠命地一踩,走了出去。不管怎么狠命,结果都是一样。我不会叫疼的,哪怕整个脚趾甲都被踩掉——有抑郁症的人总是这样的。 

当初我写《师生恋》时,曾兴奋不已——写作的意义就在于此。现在它让我厌烦。我宁愿口干舌燥、满嘴砂粒,从石头墙上被放下来,被人扔到木头水槽里。这可不是个好的洗澡盆:在水槽周围,好多骆驼正要喝水。我落到了它们中间,水花四溅,这使它们暂时后退,然后又拥上来,把头从我头侧、胯下伸下去,为了喝点水。那些在四堵方木垒成的墙中间,积满了混浊、发烫的水。但我别无选择,只能把这种带着羊尿气味的水喝下去——这水池的里侧涂着柏油,这使水的味道更臭。在远处的石阶上,老师扬着脸,雪白的下巴尖削,不动声色地看着我——她的眼睛是紫色的。她把手从袍袖里伸了出来,做了一个坚决的手势,黑奴们又把我拖了出来,带回教室,按在蒲团上,继续那节被瞌睡打断了的热力学课——虽然这样的故事准会被枪毙,但我坚信,克利奥佩屈拉曾给一个东方人讲过热力学,并且一定要他相信,未来的世界是银子做的。 

我坐在办公室的门口,这是头头的位置。如你所知,没人喜欢这个位置……对面的墙是一面窗子,这扇窗通向天顶,把对面的高楼装了进来,还装进来蒙蒙的雾汽。天光从对面楼顶上透了下来,透过楼中间的狭缝,照在雾汽上。有这样的房子:它的房顶分作两半,一半比另一半高,在正中留下了一道天窗。天光从这里透入,照着蒙蒙的雾汽——这是一间浴室。老师没把我拴在外面,而是拴在了浴室里光滑的大理石墙上。我岔开双腿站着——这样站着是很累的。站久了大腿又酸又疼。所以,我时常向前倒去,挂在拴住的双臂上,整个身体像鼓足的风帆,肩头像要脱臼一样疼痛。等到疼得受不了,我再站起来。不管怎么说罢,这总是种变化。老师坐在对面墙下的浴池里,坐在变换不定的光线中。她时常从水里伸出脚来,踢从墙上兽头嘴里注入池中的温水。每当她朝我看来时,我就站直了,把身体紧贴着墙壁,抬头看着天顶,雾气从那里冒了出去,被风吹走。她从水里爬了出来,朝我走来,此时我紧紧闭上眼睛……后来,有只小手捏住我的下巴,来回扳动着说:到底在想什么呢?我也一声不吭。在她看来,我永远是写在墙上的一个符号“X”。X是性的符号。我就是这个符号,在痛苦中拼命地伸展开来……但假如能有一个新故事,哪怕是在其中充当一个符号,我也该满意。 

\section*{四} 

将近中午时,我去见我的头头,呈上那些被我枪毙过的手稿。打印纸上那些红色的笔迹证明我没有辜负公司给我的薪水——这可是个很大的尸堆!那些笔道就如红色的细流在尸堆上流着。我手下的那些男职员们反剪着双手俯卧在地下,扭着脖子,就如宰好的鸡;女职员倒在他们身上。我室最美丽的花朵仰卧在别人身上,小脸上甚是安详——她虽然身轻如燕,但上身的曲线像她的叙事才能一样出色。我一枪正打在她左乳房下面,鲜血从藏青色的上装里流了出来。我室还有另一花朵,身材壮硕,仿佛是在奔逃之中被我放倒了,在尸丛中作奔跑之势,两条健壮的长腿从裙子里伸了出来。她们在我的火力下很性感地倒地,可惜你看不到。我枪毙他们的理由是故事不真实——没有生活依据。上司翻开这些稿子,拣我打了叉子的地方看了起来。我木然地看着窗外射进来的阳光——它照在光滑的地板上,又反射到天花板上,再从天花板上反射下来时,就变成一片弥散的白光——头头合上这些稿子,朝我无声地笑了笑,把它放到案端。然后朝我伸出手来说:你的呢?我呈上几页打印纸。 

在这些新故事里,我是克利奥佩屈拉的男宠或者一条蛇颈龙——后者的长度是五十六公尺,重量是二百吨。假如它爬进了这间办公室,就要把脖子从窗口伸出去,或者盘三到四个圈,用这种曲折委婉的姿式和头头聊天。我期望头头看到这些故事后勃然大怒,拔出把手枪,把我的脑袋轰掉,我的抑郁症就彻底好了。 

我们这里和埃及沙漠不同。我们不仅是写在墙上的符号,还写着各种大逆不道的故事。这些故事送到了头头的案端,等着被红笔叉掉。红笔涂出一个“X”,如你所知,X是性的符号……头头看了我的稿子以后笑了笑,把它们收到抽屉里。这位头头和我年龄相仿,依旧艳丽动人,描着细细的眉毛,嘴唇涂得十分性感。她把手指伸在玻璃板上,手指细长而且惨白,叫人想起了爬在桑叶上的蚕——她长着希腊式的鼻子,绰号就叫克利奥佩屈拉,简称“克”。“克”又一次伸出手来说:还有呢?我再次呈上几页打印纸,这是第十一稿《师生恋》。她草草一看,说道:时间改在秋天啦……就把它放案端那叠稿子的顶端,连一个叉子都没打。虽然看不到自己的脸,但我知道,我的脸变成了灰色。“克”把手放在玻璃板上,脸上容光焕发,说道:你的书市场反应很好,十几年来畅销不衰——用不着费大力气改写。我的脸色肯定已经变成了猪肝色。“克”最懂得怎么羞辱我,就这么草草一翻,就看出这一稿的最大改变:故事的时间改在了秋季。她还说用不着费大力气改写……其实这书稿从我手里交出去以后,还要经过数十道删改,最后出版时,时间又会改回夏季,和第一版一模一样了。这些话严重地伤害了我。她自己也是小说家,所以才会这么坏……我默默地站了起来,要回去工作。“克”也知道这个玩笑开得不好,压低了声音说道:你的稿子我会好好看的。她偷偷脱下高跟鞋,把脚伸了出来,想让我踩一脚。但我没踩她。我从上面跳过去了。 

我在抑郁中回到自己位子上。现在无事可做,只能写我的小说:“老师的脸非常白,眉毛却又宽又黑。但教室里气氛压抑……她把问题又说了一遍,世界是银子的,我很不情愿地应声答道:你说的是热寂之后。这根本不是热力学问题,而是一道谜语:在热寂之后整个宇宙会同此凉热,就如一个银元宝。众所周知,银子是热导最好的物质,在一块银子上,绝不会有一块地方比另一块更热。至于会不会有人因为这么多银子发财,我并不确切知道。这样我就揭开了谜底。 

我又把头转向窗口,那里拦了一道铁栅栏,栅栏上爬了一些常春藤,但有人把藤子截断了,所以常春藤正在枯萎下去。在山坡上,那对松鼠已经不在了。只剩了这面窗子,和上面枯萎的常春藤,这些藤子使我想到了一个暗房,这里横空搭着一些绳子,有些竹夹夹住的胶卷正在上面晾乾。这里光线暗淡,空气潮湿,与一座暗房相仿。 

老师听到了谜底,惊奇的挑起眉毛来。她摇了摇头,回身朝讲台走去。我现在写到的事情,是有生活依据的。”生活“是天籁,必须凝神静听。老师身高大约是一米五五,被紧紧地箍在发皱的绸衫里。她要踮起脚尖才能在黑板上写字。有时头发披散到脸上,她两手都是粉笔沫,就用气去吹头发:两眼朝上看,三面露白,撅起了小嘴,那样子真古怪——但这件事情我已经写了很多遍了。在潮湿的教室里,日光灯一明一灭……”每次我写出这个谜底,都感到沮丧无比。因为不管我乐意不乐意,我都得回到最初的故事,揭开这个谜底:这就像自渎一样,你可以想像出各种千奇百怪的开端,最后总是一种结局:两手粘糊糊……我讨厌这个谜底。我讨厌热寂。既然已经揭穿了谜底,这个故事可以顺利地进行下去。 

现在可以说说在我老师卧室里发生的事情了:“走进那房间的大门,迎着门放了一张软塌塌的床,它把整个房子都占满了,把几个小书架挤到了墙边上。进了门之后,床边紧紧挤着膝盖。到了这里,除了转身坐下之外,仿佛也没什么可做的事情,而且如果我们不转身坐下,就关不上门。等把门关上,我们面对一堵有门的墙,墙皮上有细小的裂纹,凸起的地 方积有细小的灰尘,我们呆在这面高墙的下面。我发现自己在老师沉甸甸手臂的拥抱之中。她抓住我的T恤衫,想把它从我头上拽下来。这件事颇不容易,你可以想象一个小个子女士在角落里搬动电冰箱,这就是当时的情形。后来她说:他妈的!你把皮带解开了呀。皮带束住了短裤,短裤又束住了T恤衫,无怪她拽不掉这件衣服,只能把我拽离地面。此时我像个待绞的死刑犯,那件衣服像个罩子蒙在我头上,什么都看不见,手臂又被袖筒吊到了半空中。我胡乱摸索着解开皮带。老师拽掉了衣服,对我说道:我可得好好看看你——你有点怪。这时我正高举着双手,一副交枪投降的模样。这世界上有不少人曾经交枪投降,但很少会有我这么壮观的投降模样。我的手臂很长,坐在床上还能摸到门框……” 


\section*{五} 

假如你在街上看到我,准会以为我是个打蓝球的,绝不会想到我在写作公司的小说室里上班。我身高两米一十多。但我从来就没上过球场,连想都没敢想过——我太笨了,又容易受伤——这样就白花了很多买衣服和买鞋的钱。我穿的衣服和鞋都是很贵的。每次我上公共厕所,都会有个无聊的小男孩站到我身边,拉开拉锁假装撒尿,其实是想看看我长了一条怎样的货色。我很谦虚地让他先尿,结果他尿不出来。于是,我就抓住他的脖子,把他从厕所里扔出去。 

我的这个东西很少有人看到,和身坯相比,货色很一般。在成熟、甚至是狰狞的外貌之下,我长了一个儿童的身体:很少有体毛,身体的隐秘部位也没有色素沉积——我觉得这是当学生当的,像这样一个身体正逐步地暴露在老师面前,使我羞愧无地——我坐在办公室里写小说,写的就是这些。上大学时我和老师恋爱,这是一个故事。这个故事正逐步暴露在读者面前,使我羞愧无地。看着这些熟悉的字句,我的脸热辣辣的。 

我从旧故事里删掉了这样一些细节:刚一关上卧室的门,老师就用双手勾住我的脖子,努力爬了上来,把小脸贴在了我的额头上,用两只眼睛分别瞪住我的眼睛,厉声喝道:傻呵呵的,想什么呢你!我没想到她会这样问我,简直吓坏了,期期艾艾地说道:没想什么?老师说:混帐!什么叫没想什么?她把我推倒在床垫上,伸手来拽我的衣服……此时我倒不害怕了。我把这些事删掉,原因是:人人都能想到这些。人人都能想到的事就像是编出来的。我总在编故事,但不希望人们看出它是编出来的。 

“在老师的卧室里,我想解开她胸前的扣子,但没有成功。失败的原因是我手指太粗,拿不住细小的东西;还有一个原因是空气太潮,衣料的摩擦系数因此大增。她自己解决了这个问题,从绸衫下面钻了出来,然后把它挂在门背后。门背后有个轻木料做成的架子,是个可以活动的平行四边形,上面有凸起的木钉,她把它作挂衣钩来用,但我认为这东西是一种绘图的仪器。老师留了个娃娃头,她的身材并不像我想象的那么纤细,而是小巧而又结实……”我的故事只有一种开始,每次都是从热力学的教室开始,然后来到了老师的宿舍。然后解老师胸前的扣子,怎么也解不开——这么多年了,我总该有些长进才好。我想让这个故事在别的时间、地点开始,但总是不能成功。 

最近我回学校去过,老师当年住的宿舍楼还在,孤零零地立在一片黄土地上。这片地上满是碎砖乱瓦,还有数不尽的碎玻璃片在闪光。原来这里还有好几座筒子楼,现在都拆了——如果不拆,那些楼就会自己倒掉,因为它们已经太老了。那座楼也变成了一个绿色的立方体:人家把它架在脚手架里,用塑料编织物把它罩住,这样它就变得没门没窗,全无面目,只剩下正面一个小口子,这个口子被木栅栏封住,上面挂了个牌子,上书:电影外景地。听人家说,里面的一切都保留着原状,连走廊里的破柜子都放在原地。什么时候要拍电影,揭开编织袋就能拍,只是原来住在楼里的耗子和蟑螂都没有了——大概都饿死了。要用人工饲养的来充数——电影制片厂有个部门,既养耗子又养蟑螂。假如现在到那里去,电工在铺电线,周围的黄土地上停着发电车、吊车;小工正七手八脚地拆卸脚手架——这说明新版本的师生恋就要开拍了。这座楼的样子就是这样。这个电影据说是根据我的小说改编。我有十几年没见过老师。她现在是什么样子了,我不知道。 

人在公司里只有两件事可做:枪毙别人的稿子或者写出自己的稿子供别人枪毙。别人的稿我已经枪毙完了,现在只能写自己的稿子,在黑色的屏幕上,我垂头丧气地写道:“……她从书架上拿了一盒烟和一个烟灰缸回来。这个烟灰缸上立了一只可以活动的金属仙鹤。等到她取出一支烟时,我就把那只仙鹤扳倒,那下面果然是一只打火机。为老师点烟可以满足我的恋母情结。后来,她把那支烟倒转过来,放到我嘴里。当时我不会吸烟,也吸了起来,很快就把过滤嘴咬了下来,然后那支烟的后半部就在我嘴里解体了,烟丝和烟纸满嘴都是;它的前半截,连同燃烧着的烟头,摊到了我赤裸的胸口上。老师把烟的残骸收拾到烟灰缸里,哈哈地笑起来了,然后她和我并肩躺下。她躺在床上,显得这张床很大;我躺在床上,显得这张床很小;这张床大又不大,小又不小,变成了一样古怪的东西。她钻到我的腋下,拍拍我的胸口说:来,抱一抱。我侧过身来抱住老师——这是此生第一次。在此之前,我谁都没抱过。自己不喜欢,别人也不让我抱。就是不会说话的孩子,见我伸出桅杆似的胳臂去抱他,也会受到惊吓,嚎啕痛哭……后来,我问老师,被我抱住时害不害怕。她看看垂在肩上的胳臂——这样东西像大象的鼻子——摇摇头上的短发,说道:‘不。我不怕你。我怕你干什么?’是啊是啊。我虽然面目可憎,但并不可怕。我不过是个学生罢了。” 



\section*{六} 

今天上午,我室全体同仁——四男二女——都被毙掉了。如今世界上共有三种处决人的方法:电椅、瓦斯、行刑队。我喜欢最后一种方法,最好是用老式的滑膛枪来毙。行刑队穿着英国禁卫军的红色军服,第一排卧倒,第二排跪倒,第三排站立,枪声一响,浓烟弥漫。大粒的平头铅子弹带着火辣辣的疼痛,像飞翔的屎克螂迎面而来,挨着的人纷纷倒地,如果能挨上一下,那该是多么惬意啊——但我没有挨上。我要被钉死在十字架上。我这么大的个子,枪毙太糟蹋了。随着下午来临,天色变得阴暗起来。夜幕就如一层清凉的露水,降临在埃及的沙漠里。此时我被从墙上解了下来,在林立的长矛中,走向沙漠中央的行刑地,走向十字架。克利奥佩屈拉坐在金色的轿子里,端庄而且傲慢。夜幕中的十字架远看时和高大的仙人掌相仿……无数的乌鸦在附近盘旋着。我侧着头看那些乌鸦,担心它们不等我断气就会把我的眼睛啄出来。克利奥佩屈拉把手放在我肩头——那些春蚕似的手指在被晒得红肿的皮肤上带来了一道道的剧痛——柔声说道:你放心。我不让它们吃你。我不相信她的话,抬头看着暮色中那两块交叉着的木头,从牙缝里吸着气说道:没关系,让它们吃罢。对不相信的事情说不在意:这就是我保全体面的方法。到底乌鸦会不会吃我,等被钉上去就知道了。克利奥佩屈拉惊奇地挑起了眉毛,先吸了一口气,然后才说:原来你会说话! 

将近下班时,公司总编室正式通知我说,埃及沙漠里的故事脱离了生活,不准再写了。打电话的人还抱怨我道:瞎写了些什么——你也是个老同志了,怎么一点分寸都不懂呢。居然挨上了总编的枪子儿,我真是喜出望外。总编说话带着囔囔的鼻音,他的话就像一只飞翔的屎克螂。他还说:新版《师生恋》的进度要加快,下个月出集子要收。我没说什么,但我知道我会加快的。至于恐龙的故事,人家没提。看来“克”没把它报上去,但我的要求也不能太高。接到这个电话,我松了一口气——我终于被枪毙了——我决定发一会呆。假如有人来找我的岔子,我就说:我都被枪毙了,还不准发呆吗。提到自己被枪毙,就如人前显贵。请不要以为,我在公司里呆了十几年就没资格挨枪毙了。我一发呆,全室的人都发起呆来,双手捧头面对单色电脑;李清照生前,大概就是这样面对一面镜子。宋代的镜子质量不高,里面的人影面部臃肿,颜色灰暗——人走进这样的镜子,就是为了在里面发愣。今天,我们都是李清照。这种结果可算是皆大欢喜。忽听屋角哗啦一声响,有人拉开椅子朝我走来。原来还有一个人不是李清照……我有一位女同事,不分季节,总穿棕色的长袖套装。她肤色较深,头上梳着一条大辫子,长着有雀斑的圆鼻子和一双大眼睛,像一个卡通里的啮齿动物。现在她朝我走来了。她长得相当好看,但这不是我注意的事。我总是注意到她长得人高马大,体重比一般人为重,又穿着高跟鞋。我从来不枪毙她的稿子,她也从来不踩我——大家相敬如宾。实际上,本室有四男三女,我总把她数漏掉。但她从我身边走过时,我还是要把脚伸出来:踩不踩是她的权利,我总得给她这种机会。怀着这样的心情,我把脚放在可以踩到的地方,但心里忐忑不安。假设有一只猪,出于某种古怪的动机蹲在公路边上,把尾巴伸在路面上让过往的汽车去压,那么听到汽车响时,必然要怀着同样忐忑不安的心情想到自己的尾巴,并且安慰自己说:司机会看到它,他不会压我的……谁知“咯”地一声,我被她踩了一脚,疼痛直接印到了脑子里,与之俱来的,还有失落感——我从旁走过时,“克”都伸出脚来,但我从来不踩;像我这样的身胚踩上一脚,她就要去打石膏啦……这就是说,人家让你踩,你也可以不踩嘛。我禁不住哼了一声。因为这声呻吟,棕色的女同事停了下来,先问踩疼了没有,然后就说:晚上她要和我谈一件事。身为头头,不能拒绝和属下谈话,不管是白天还是晚上。虽然要到晚上谈,但我现在已经开始头疼了。 

“在老师的卧室里,我抱着她,感到一阵冲动,就把她紧紧地搂住,想要侵犯她的身体;这个身体像一片白色的朦胧,朦胧中生机勃发……她狠狠地推了我一把,说道:讨厌!你起开!我放开了她,仰面朝天躺着,把手朝上伸着——一伸就伸到了窗台下的暖气片上。这个暖气片冬天时冷时热,冷的时候温度宜人,热的时候能把馒头烤焦,冬天老师就在上面烤馒头;中午放上,晚上回来时,顶上烤得焦黄,与同合居的烤馒头很相像——同合居是家饭馆,冬天生了一些煤球炉子,上面放着铜制的水壶,还有用筷子穿成串的白面馒头。其实,那家饭店里有暖气,但他们故意要烧煤球炉子——有一回我的手腕被暖气烤出了一串大泡,老师给我涂了些绿药膏,还说了我一顿,但这是冬天的事。夏天发生的事是,我这样躺着,沉入了静默,想着自己很讨厌;而老师爬到我身上来,和我做爱。我伸直了身体,把它伸向老师。但在内心深处还有一点不快——老师说了我。我的记恨心很重。”我知道自己内心不快时是什么样子:那张长长的大脸上满是铅灰色的愁容。如果能避免不快,我尽量避免,所以这段细节我也不想写到。但是今天下午没有这个限制:我已经开始不快了……“她拍拍我的脸说:怎么,生气了?我慢慢地答道:生气干什么?我是太重了,一百一十五公斤。她说:和你太重没有关系——一会儿和你说。但是一会儿以后,她也没和我说什么。 

后来发现,不管做不做爱,她都喜欢跨在我身上,还喜欢拿支圆珠笔在我胸口乱写:写的是繁体字,而且是竖着写,经常把我胸前写得像北京公共汽车的站牌。她还说,我的身体是个躺着很舒服的地方,当然,这是指我的肚子。肚子里盛着些柔软的脏器:大肠、小肠,所以就很柔软,而且冬暖夏凉,像个水床。胸部则不同,它有很多坚硬的肋骨,硌人。里面盛着两片很大的肺,一吸一呼发出噪声。我的胸腔里还有颗很大的心,咚咚地跳着,很吵人。 

这地方爱出汗,也不冬暖夏凉——说实在的,我也不希望老师睡在这个地方。胸口趴上个人,一会儿还不要紧,久了会就透不过气来。如你所知,从小到大,我是公认的天才人物。躺在老师身下时,我觉得自己总能想出办法,让老师不要把我当成一枚鸡蛋来孵着。但我什么办法都没想出来。不但如此,我连动都不能动。只要我稍动一下,她就说:别动……别动。舒服。”我和老师的故事发生了一遍又一遍,每回都是这样的——我只好在她的重压之下睡着了。要是在“棕色的”女同事身下我就睡不着。她太沉了。 

\section*{七} 

随着夜幕降临,下班的时刻来临了——这原本是惊心动魄的时刻。在一片寂静中,“克”一脚踹开了我们的门。她已经化好了妆,换上了夜礼服,把黑色的风衣搭在手臂上,朝我大喝一声道:走,陪我去吃晚饭——看到我愁容满面地趴在办公桌上,她又补了一句:不准说胃疼!似乎我只能跟她到俱乐部里去,坐在餐桌前,手里拿着一把叉子,扎着盘子里的冷芦笋。与此同时,她盘问我,为什么我的稿子里会有克利奥配屈拉——这故事的生活依据是什么。有个打缠头的印度侍者不时的来添上些又冷又酸的葡萄酒,好像嫌我胃壁还没有出血。等到这顿饭吃完,芦笋都变成酱了。我的胃病就是这样落下的。但你不要以为,因为她是头头我就愿意受这种折磨。真正的原因是:她是个有魅力的女人。其实,晚饭我自会安排。我会把我室那朵最美丽的花绑架到小铺里去吃合洛面。就像我怕冷芦笋,她也怕这种面,说这种面条像蛔虫。那家小铺里还卖另一种东西,就是卤煮火烧——但她宁死都不吃肥肉和下水。我吃面时,她侧坐在白木板凳上,抽着绿色的摩尔烟,尽量不往我这边看。但她必须回答我的逼问:在她稿子里那些被我用红笔勾掉的段落中,为什么会有个身高两米一零的男恶棍——这个高度的生活依据何在,是不是全世界的男人都身高两米一零。整个小饭铺弥漫着下水味、泔水味儿,还有民工身上的馊味。她抱怨说,回家马上就要洗头,要不然头发带有抹布味——但你不要以为我是头头她就愿意受这种折磨。真正的原因是:我是个身长两米多的男人。不管身长多少,魅力如何,人的忍耐终归是有限。等到胃疼难忍,摩尔烟抽完,我们已经忍无可忍,挑起眉毛来厉声问道:你到底要干什么?让我陪你上床吗?听到这句问话,我们马上变得容光焕发,说我没这个意思,还温和地劝告说:不要把工作关系庸俗化……其实谁也不想让谁陪着上床,因为谁都不想把工作关系庸俗化——我们不过是寻点乐子罢了。但是,假如没有工作关系,“克”肯定要和我上床,我肯定要和那朵美丽的花上床。工作关系是正常性关系的阻断剂,使它好像是种不正常的性关系。今天晚上我没有跟“克”去吃饭,我只是把头往棕色的女同事那边一扭,说道:我不能去——晚上有事情。“克”看看我,再看看“棕色的”,终于无话可说,把门一摔,就离去了。然后,我继续趴着,把下巴支在桌面上,看着别人从我面前走过。最美丽的花朵最先走过,她穿着黑色的皮衣,大腿上带着坐出的红色压痕,触目惊心——我已经说过我不走,有事情,这就是说,他们可以先走了。这句话就如一道释放令。他们就这样不受惩罚地逃掉了。 

“棕色的”要找我谈话,我猜她不是要谈工资,就是要谈房子。如你所知,我们是作家,是文化工作者,谈这种低俗事情总是有点羞涩,要避开别人。这种事总要等她先开口,她不开口我就只能等着。与此同时,我的同事带着欢声笑语,已经到了停车场上。我觉得自己是个倒霉蛋,但又无可奈何…… 

晚上,公司的停车场的上满是夜雾,伸出手去,好像可以把雾拿到手里——那种粘稠的冷冰冰的雾。这种雾叫人怀念埃及沙漠……天黑以后,埃及沙漠也迅速地冷了下来,从远处的海面上,吹来了带腥味的风。在一片黑暗里,你只能把自己交付给风。有时候,风带来的是海洋的气味,有时带来的是乾燥得令人窒息的烟尘,有时则带来可怕的尸臭。在我们的停车场上,风有时带来浓郁的花香,有时带来垃圾的味道。最可怕的是,总有人在一边烧火煮沥清,用来修理被压坏的车道。沥清熬好之后,他们把火堆熄掉——用的是自己的尿。这股味没法闻。我最讨厌从那边来的风…… 

我读大学时,学校建在一片荒园里。这里的一切亭榭都已倒塌,一切池沼都已乾涸,只余下一片草木茂盛的小山,被道路纵横切割,从天上看来,像个乌龟壳——假如一条太古爬来的蛇颈龙爬到了我们学校,看到的就是这些。它朝着小山俯下头来,想找点吃的东西,发现树叶上满是尘土,吃起来要呛嗓子眼。于是它只好饿着肚子掉头离去。天黑以后,这里亮着疏疏落落的路灯。 

有个男人穿着雨衣,兜里揣着手电筒,在这里无奈地转来转去,吓唬过往的女学生——他是个露阴癖。老师的样子也像个女学生,从这里走过时,也被他吓唬过……看到手电光照着的那个东西,她也愣了一愣,然后抬头看看那张黑影里的脸,说道:真讨厌哪,你!这是冬天发生的事,老师穿着黑色的皮衣,挎着一个蜡染布的包。她总在快速的移动中,一分钟能走一百步——她在我心中的地位无可替代。这也是真实发生的事,但我不能把它写进小说里,因为它脱离了生活——除非这篇小说不叫作《师生恋》,叫作《一个露阴癖的自白》——假如我是那个露阴癖,这就是我的生活。别人也就不能说我脱离生活了。 
\section*{八} 

冬天里,有一次老师来上课,带着她的蜡染布包。包里有样东西直翘翘地露了出来,那是根法国式的棍面包。上课之前她把这根面包从包里拿了出来,放在讲台上。我们的校园很大,是露阴癖出没的场所,老师遇到过,女同学也遇到过。被吓的女同学总是痛哭失声,一副不依不饶的样子。假如那个吓人的家伙被逮住了,那倒好办:她一哭,我们就揍他。把他揍到血肉模糊,她就不忍心再哭了。问题在于谁都没逮住——所以她们总是对着老师不依不饶。老师是我们的班主任,有责任安慰受惊吓的人。在讲课之前,她准备安慰一下那些被惊吓的人,没开口之前先笑弯了腰:原来昨天晚上她又碰上那个露阴癖了。那家伙撩起了雨衣的下摆,用手电照着他的大鸡巴。老师也拿出一个袖珍手电筒,照亮了这根棍面包……结果是那个露阴癖受到了惊吓,惨叫一声逃跑了。讲完了这件事,老师就接着讲她的热力学课。但听课的人却魂不守舍,总在看那根棍面包。那东西有多半截翘在讲台的外面,带着金黄色的光泽。下课后她扬长而去,把面包落在了那里。同学们离开教室时,都小心地绕开它锋端所指。我最后一个离开教室,走以前还端详了它一阵,觉得它的样子很刺激,尤其是那个圆头……然后,这根面包就被遗弃在讲台上,在那里一点点地干掉。我把这件事写进了我的小说,但总是被“克”枪毙掉,并用红笔批道:脱离生活。在红色的叉子底下,她用绿笔在“棍面包”底下画了一道,批道:我知道了。她知道了什么呢?为什么要写到这个露阴癖和这根棍面包,连我自己都不知道。 

晚上,办公室里一片棕色。“棕色的”穿着棕色的套装。头顶米黄色的玻璃灯罩发出暗淡的灯光,溶在潮湿的空气里,周围是黑色的办公家具。墙上是木制的护墙板。现在也不知是几点了。我伸手到抽屉里取出一盒烟来——我有很多年不抽烟了,这盒烟在抽屉里放了很多年,所以它就发了霉,抽起来又苦又涩,但这正是我需要的。办公室里灯光昏暗,像一座热带的水塘——水生植物的茎叶在水里腐烂、溶化,水也因此变得昏暗——化学上把这种水叫作胶体溶液——我现在正泡在胶体溶液里。我正想要打个盹,她忽然开口了。“棕色的”首先提出要看看我的脚丫子,看看它被踩得怎样了。这是从未有过的事:以前他们都是只管踩,不管它怎样的。先是解开重重鞋带,然后这只脚就裸露出来:上面筋络纵横,大脚趾有大号香皂那么大。它穿五十八号鞋,这种鞋必须到鞋厂去定做,每回至少要买两打,否则鞋厂不肯做。总而言之,这只脚还是值得一看的,它和旧时小脚女人的脚恰恰是两个极端。我要是长了一对三寸金莲就走不了路,站在松软的地面上,我还会自己钻到土里去。小脚女人长这双大脚也走不了路,它会左右相绊——但是“棕色的”无心细看,也无心听我解说。她哭起来了。好好的她为什么要哭?就是要长工资,也犯不着哭啊。我觉得自己穿上了一件新衬衣,浆硬的领子磨着脖子,又穿上了挤脚的皮鞋。不要觉得我什么谜都猜得出来。有些谜我猜不出来,还有些谜我根本不想猜。但现在是在公司里。我要回答一切问题,还要猜一切谜。 

穿过夜雾,走上停车场,然后就可以回家了。上了一天班,没人不想回家,虽然在回家的路上可能会遭劫——不久之前,有一回下班以后,我和“棕色的”走在停车场上,拣有路灯地方走着,但还是遇上了一大夥强盗。他们都穿着黑皮衣服,手里拿着锋利的刀子,一下子把我围住。停车场上常有人劫道,但很少见他们成群结队的来。这种劫道的方式颇有古风,但没有经济效益——用不着这么多人。我被劫过多少次,这次最热闹,这使我很兴奋,想凑凑热闹。不等他们开口说话,我就把双手高高举了起来,用雷鸣般的低音说道:请不要伤害我,我投降!脱了衣服才能看见,我的胸部像个木桶,里面盛了强有力的肺。那些小个子劫匪都禁不住要捂耳朵;然后就七嘴八舌地说:吵死了——耳朵里嗡嗡的——大叔,你是唱男低音的吧。原来这是一帮女孩,不知为什么不肯学好,学起打劫来了。其中有个用刀尖指住我的小命根,厉声说道:大叔,脱裤子!我们要你的内裤。周围的香水味呛得我连气都透不过来。真新鲜,还有劫这东西的……这回这个故事非常真实。它根本就是真事。被人拿刀子逼住,这无疑是种生活。我苦笑着环顾四周,说道:小姐们,你们搞错了,我的内裤对你们毫无用处——你们谁也穿不上的。除非两个人穿一条内裤——我看你们也没穷到这个份上。你们应该去劫那位大婶的内裤。结果是刀尖扎了我一下,戳我的女孩说道:少废话,快点脱;迟了让你断子绝孙——好像我很怕断子绝孙似的。别的女孩则七嘴八舌地劝我:我们和别人打了赌,要劫一条男人内裤。劫了小号的裤衩,别人会赖的,你的内裤别人没得说——快脱罢,我们不会伤害你的。这个说法使我很感动:我的内裤别人没得说——我居然还有这种用处。我环顾四周,看到闪亮的皮衣上那些尖尖的小脸,还有细粒的粉刺疙瘩。 

她们都很激动,我也很激动,马上就要说出:姑娘们,转过身去,我马上就脱给你们……我还想知道她们赌了什么。但就在此时,她们认出了我,说道:你就是写《师生恋》那个家伙!书写得越来越臭——你也长得是真寒碜。寒碜就寒碜,还说什么真寒碜。我觉得头面里有点疼了。头疼是动怒的前兆。你可不要提我写的书,除非你想惹我动怒。 

停车场上,所有的路灯从树叶的后面透射出来,混在浓雾里,夜色温柔。不管是在停车场上,还是在沙漠里,都是一天最美好的时光。在停车场上,我被一群坏女孩围住,在沙漠里,我被绑在十字架上,背靠着涂了沥清的方木头,面对着一小撮飘忽不定的篝火。在半干的畜粪堆上,火焰闪动了一阵就熄灭了,剩下一股白烟,还有闪烁不定的炭火。天上看不到一颗星,沙漠里的风变得凛冽起来。那股烟常常飘到我的脸上来,像一把盐一样,让我直流眼泪。因为没有办法把眼泪擦干,就像是在哭。其实我没有哭,我只有一只眼在流泪,因为只熏着了一只。一般人哭起来都是双眼流泪,除非他是个独眼龙。 

此时我扭过头去,看着老师——她就站在我身边,是茫茫黑夜里的一个灰色影子。她把手放在我赤裸的腿上,用尖尖的手指掐我的皮肤,说道:你一定要记住,将来的世界是银子的……这是沙漠里的事。在停车场上,我大腿里侧刺痛难当,刀尖已经深深扎进了肉里——与此同时,我头里有个地方刺疼了起来。这个拿刀子的小丫头真是坏死了。另有一个小丫头比较好,她拿了一支笔塞到我手里,说:老师,等会儿在裤衩上签个字吧。我们是大学中文系的学生,你的小说是我们的范本。我常给一些笨蛋签字,但都是签在扉页上,在裤衩上签字还是头一回。但这件事更让我头疼。我叹了口气说:好吧,这可是你们让我脱的;就把裤子脱了下来。那些女孩低头一看,吓得尖叫一声,掩面返走;原因是我的性器官因为受到惊吓,已经勃起了,在路灯的光下留下长长的黑色影子——样子十分吓人。出了这种事,我禁不住哈哈大笑——假如我不大笑,大概还不会把她们吓跑:那声音好像有一队咆哮的老狗熊迎面扑来。在停车场的路灯下,提着裤子、挺着个大鸡巴,四周是正在逃散的小姐们,是有点不像样子。但非我之罪,谁让她们来劫我呢。 

小姐们逃散之后,一把塑料壳的壁纸刀落在了地上,刀尖朝下,在地下轻轻地弹跳着。我俯身把它拣了起来,摸它的刀片——这东西快得要死,足以使我断子绝孙。我把它收到口袋里,回头去看“棕色的”。这女人站在远处,眯着眼睛朝我这边看着。她像蝙蝠一样瞎,每次下班晚了,都得有人领她走过停车场,否则她就要磕磕碰碰,把脸摔破。上班时别人在她耳畔说笑话,她总是毫无反应。所以她又是个聋子,最起码在办公室里是这样。她大概什么都没看到、没听到。这样最好。我收敛起顽劣的心情,束好裤子,带她走出停车场——一路上什么都没有说。但我注意到,停车场上夜色温柔……当天夜里在睡梦中,我被吊在十字架上,面对着阴燃着的骆驼粪。整个沙漠像一个隐藏在黑夜里的独眼鬼怪。老师在我耳畔低语着,说了些什么我却一句也没记住。她把手伸进我胯下的遮羞布里,那只手就如刀锋,带来了残酷的刺激。就是这种残酷的刺激使我回到了白银时代。 
\section*{九} 

我在办公室里,坐在“棕色的”对面。她还没有开口,但我已经感到很糟糕了。可能她要找我谈的事既不是房子,也不是工资,而是些别的……我既不想和她谈房子,也不想谈工资——我不管房也不管工资,我只管受抱怨。但我更不想谈别的。别的事情对我更坏。 

那天遇劫后,回家洗澡时,我看到胯间有个壁纸刀扎的伤口。它已经结了痂,就像个黑色的线头,对我这样的巨人来说,这样的伤口可以说是微不足道,我还在上面贴了创可贴。但它刺疼不已,好像里面有一根针。我把那把刀找了出来,仔细地看了半天,刀片完好无损,没有理由认为伤口里有什么东西,只好让它疼下去了。也许因为疼痛的刺激,那东西就从头到脚直撅撅的,和在停车场上遇劫时一样。细说起来它还不止是直,从前往后算,大约在三分之一的长度上有点弯曲——往上翘着,像把尼泊尔人用的匕首。用这种刀子捅人,应该往肚子上捅,刀尖自然会往上挑,给人以重伤。总而言之,这种向上弯的样子实在恶毒。假如夜里“棕色的”看见了它,我就会有点麻烦。因为我有责任让她见不到它。这个东西原来又小又老实,还不算太难看,被人用刀子扎了一下,就变又大又不老实,而且丑极了。这就是说,落下后遗症了。 

在我的另一个故事里也有这样一幕:在沙漠里,克利奥佩屈拉把我的缠腰布解开,里面包裹的东西挺立起来,就如沙漠里怒放的仙人掌花。呼啸的风搅动砂砾——在锐利的砂砾中间,它显得十分浑圆,带有模糊不清的光泽,在风里摇摆不定。老师带着笑意对我说:怎么会是这样的?对此我无法解释。我低下头去,看到脚下的麻袋片里包裹的东西:一个铜锤和若干扁头钉子。老师拾起一根钉子,拿到我的面前:钉头像屎克螂一样大,四棱钉体上还带有锻打的痕迹:这就是公元前的工艺水平,比现代的洋钉粗笨,但也有钉得结实的好处。老师就要把我钉死在十字架上,在此之前,她先要亲吻我,左手举着那根钉子,右手把那根直撅撅的东西拨开,踮起脚尖来…… 

我抬起头来,环视四周——灰蒙蒙的沙漠里,立着不少十字架。昨天的同学都被钉在上面。人在十字架上会从白变棕、从棕变黑,最后干缩成一团,变得像一只风干的青蛙、一片烧过的纸片——变成一种熔化后又凝固的坚硬胶状物,再然后在风砂中解体。然后我又去看老师,她已经拿起了铜锤,准备把钉子敲进我的掌心。这是变成风干青蛙的必要步骤。老师安慰我说:并不很疼。我很有幽默感地说道:那你怎么不来试试?她大笑了起来,此时我才发现,老师的声音十分浑厚。 

顺便说一句,我仔细考虑过怎样处死我自己:等到钉穿了双手和双足之后,让老师用一根锋利的木桩洞穿我的心脏。这样她显得比较仁慈——虽然这样的仁慈显得很古怪。在埃及妖后和行将死在十字架上的东方奴隶之间已经说了很多话,这是很罕见的事件……最后,她又一次说道:记住,将来的世界是银子的……此时,我已是鲜血淋漓,在剧痛中颤抖着。只有最残酷的痛苦才能使我离开埃及的沙漠,回到这白银世界里来。 

假如这个故事有寓意的话,它应该是:在剧痛之中死在沙漠里,也比迷失在白银世界里好得多。这个寓意很恶毒。公司领导把它枪毙掉是对的。领导不笨,“克”不笨,我也不笨。我们总是枪毙一切有趣的东西。这是因为越是有趣的东西,就越是包含着恶毒的寓意。 

我们的办公室在一楼,有人说,一楼的房子接地气,接地气的意思是说,这间房子格外潮湿,晚上尤甚。潮气渗透了我的衣服,腐蚀着我的筋骨。潮湿的颜色是棕色的。我的老师也是棕色的,她紧挨着我坐着,把棕色的头发盖在我肩上,告诉我说,未来的世界是银子的。这就是说,这世界早晚要沦为一片冷冰冰的、稀薄的银色混沌,你把一片黄铜含在嘴里,或者把一片锡放在嘴里反复咀嚼,会尝到金属辛辣的味道——这就是混沌的味道。这个前景可不美妙。但是老师的声音毫无悲伧之意——她声调温柔,甚至带有诱惑之意。她把一片棕色的温暖揉进了我的怀里。在这个故事里,老师的身体硕长,嘴唇和乳头都呈紫色。在一阵妙不可言的亢进之中,我进入了一片温暖的潮湿。在这个故事里,我和老师坐在一棵大树的树根上,脚下是热带雨林里四通八达的棕色水系。只有潜入水中,才发现这种棕色透明的水是一片朦胧。有些黄里透绿的大青蛙伸直了腿,一动不动地飘在水里,就像大海里漂着的水母。波光流影在它身上浮动着。你怎么也分不清它是死了,还是活着的。这就是这种动物的谋生之道——无论蛇也好,鳄鱼也罢,都不想吃只死青蛙,会吃坏肚子的……正如在沙漠里有绿洲,埃及也会有热带的雨林和四通八达的水系,老师也会有温柔,温柔就是躺在一片棕色的阴影里,躺在盘根错节的树根上。但是一阵电话铃像针一样扎进了我的脑子。这使我想起有个小子每礼拜三都要在停车场上劫我。我有责任马上出去被他打劫——他等得不耐烦,会拿垒球棒砸我的吉普车。我怀着忐忑的心情等着,不等拿起耳机,我就知道这个电话肯定是场灾祸。我的吉普完蛋了。吉普的零件很难找,因为车子早就停产了。要是去买辆轿车,我又坐不进去。谁让我长这么大个子——我天生是个倒霉蛋……“棕色的”还是光哭不说话。看来这个谜我是必须猜了。我有种种不祥的预感,其中最不祥的一种就是:她要声讨我这根直立的大鸡巴。我没什么可说的,只能代它道歉,因为人家不想看见你,你却被人家看到了。我还要进一步保证说,下次它一定不这样——这样她应该满意了吧。其实下回它会怎样,我也不知道。这女人有怕黑的毛病,下班后得有人陪她走过黑暗的停车场,走到灯火通明的地方。这件事我责无旁贷:一方面,她总是像哑巴一样一声不吭,没人乐意陪她走路;另一方面,我是本室的头头,没人干的事我都要干。 

以后我还要陪她走过停车场,不知什么时候,又会遇上一群坏女孩劫我的内裤——到那时,它又要直立如故,然后“棕色的”又要来声讨我……这就是说,仅仅道歉是不行的。还要让她见到这样东西时,能够不失声痛哭……我准备用老师的话来安慰“棕色的”:“他直他的,我们走我们的路”。这话应该改成我直我的,你走你的路——我怀疑“棕色的”看到了我那个东西,现在正要不依不饶。假如我是露阴癖,此时就该来揍我。但我不是露阴癖。人家用刀子对着我,我才脱裤子的。这一点一定要说清楚。也许我该为那三分之一处弯曲向她道歉,但也要说清楚:人家拿刀子对着它,它才往上弯的…… 
\section*{十} 

公司的保安员用内线电话通知我说:该下班了。他知道有人在等着劫我。所以他是在通知我,赶紧出去给劫匪送钱;不然截匪会砸我的车了。车在学院的停车场上被砸,他有责任,要扣他的工资。我不怕劫匪砸我的车,因为保险公司会赔我。但我怕保安被扣工资——他会记恨我,以后给我离楼最远的车位。车场大得很,从最远的地方走到楼门口有五里路。盛夏时节,走完这段路就快要中暑了。这一系列的事告诉我们的是:文明社会一环扣一环,和谐地运转着,错一环则动全身。现在有一环出了毛病——出在了“棕色的”身上。她突然开口说话了,对我说道:老大哥,我要写小说啊…… 

全公司的人都知道“棕色的”是个缺心眼的人,所以她说出的话不值得重视——下列事件可以证明她的智力水平:本公司有项规定,所有的人每隔两年就要下乡去体验生活——如你所知,生活这个词对写作为生的人来说,有特殊的意义。体验生活,就是在没自来水、没有煤气、没有电的荒僻地方住上半年。根据某种文艺理论,这会对写作大有好处。虽有这项规定,但很少有人真去体验生活——我被轮上了六次,一次也没去。一被轮上我就得病:喘病、糖尿病,最近的一次是皮肤骚痒症。除我之外,别人也不肯去,并且都能及时地生病。只有她,一被轮上就去了。去了才两个星期,就丢盔卸甲地跑了回来。她在乡下走夜路,被四条壮汉按住轮奸了两遍。回来以后,先在医院里住了一星期,然后才来上班。这个女人一贯是沉默寡言的,有一阵子变得喋喋不休,总在说自己被轮奸时的感受:什么第一遍还好受,第二遍有点难忍了云云。后来有关部门给了她一次警告,叫她不要用自己不幸的狭隘经验给大好形势抹黑,她才恢复了常态——又变得一声不吭。才老实了半年,又撒起了癔症。此人是个真正的笨蛋。说起来我也有点惭愧:人家既然笨,我就该更关心她才对嘛。 

透过我的头疼,我看到在一片棕色阴影之中,“棕色的”被关在一个竹笼子里了。这笼子非常小,她在里面蜷成了一团,手脚都被竹篾条拴在笼栅上。菲律宾的某些原始部落搬迁时,就是这样对待他们最宝贵的财产:一只猪。最大快人心的是,人家把她的嘴也拴住了。这样她就不能讲出大逆不道的语言。不管别人怎样看待她,在我眼睛里,她是个女人。她还是我的下属呢。我走向前去,打开竹笼,解开那些竹篾条。“棕色的”透了一口气,马上说道:老大哥,我要写小说!如你所知,我们在写作公司做事,每天都要写小说。她居然还要写小说。这个要求真是太过古怪……但罪不在我。 

我想要劝“棕色的”别动傻念头,但想不出话来。把烟抽完之后,我就开始撕纸。先把一本公用信纸撕碎,又把一扎活页纸毁掉了:一部份变成了雪花状,另一部份做成了纸飞机,飞得办公室里到处都是。顺便说一句,做纸飞机的诀窍在于掌握重心:重心靠前,飞不了多远就会一头扎下来;重心靠后则会朝上仰头,然后屁股朝下的往下掉——用航模的术语来说,它会失速,然后进入螺旋。最后,我终于叠出了最好的纸飞机,重心既不靠前,也不靠后,不差毫厘地就在中央,掷在空中慢慢地滑翔着,一如钉在天上一样,半个钟头都不落地。看到这种绝技,不容“棕色的”不佩服。她擦干了泪水,也要纸来叠飞机。这样我们把办公桌上的全部纸张都变成了这种东西——很不幸的是,这些纸里有一部小说稿子,所以第二天又要满地拣纸飞机,拆开后往一块对,贴贴补补送上去。但这已经是第二天的事了。 

不知不觉地到了午夜,此时我想起了自己是头头,就站起身来,说道:走吧,我送你回家。这是必须的:“棕色的”乘地铁上下班,现在末班车早就开过了。奇怪的是:我的吉普车没被砸坏。门房里的人朝我伸出两个指头,这就是说,他替我垫了二十块钱,送给那个劫道的小玩闹。我朝他点了点头,意思是说,这笔钱我会还他的。保安可不是傻瓜蛋,他不会去逮停车场上的小玩闹——逮倒是能逮到个把,但他们又会抽冷子把车场的车通通砸掉,到那时就不好了。以前发生过这种事:几十辆车的窗玻璃都被砸掉。这就是因为保安打了一个劫匪,这个保安被炒了鱿鱼,然后他就沦为停车场上的劫匪,名声虽不好听,但收入更多。那几十辆车的碎玻璃散在地下,叫我想起了小时的事:那时候人们用暖水瓶打开水。暖水瓶胆用镀银的玻璃制成,碎在地下银光闪闪。来往的人怕玻璃扎脚,用鞋底把它们踩碎。结果是更加银光闪闪。最后有人想到要把碎玻璃扫掉时,已经扫不掉了——银光渗进了地里……在车上“棕色的”又一次开始哭哭啼啼,我感到有点烦躁,想要吼她几句——但我又想到自己是个头头,要对她负责任。所以,我叹了一口气,尽量温存地说道:如果能不写,还是别写罢。听到我这样说,她收了泪,点点头。这就使我存有一丝侥幸之心:也许,“棕色的”不是真想这样,那就太好了。 

送过了“棕色的”我回家。天上下着雨,雨点落在地下,冒着蓝色的火花。有人说,这也是污染所致;上面对此则另有说法。我虽不是化学家,却有鼻子,可以从雨里嗅出一股臭鸡蛋味。但不管怎么说罢,这种雨确实美丽,落在路面上,就如一塘风信子花。我闭灯行驶——开了灯就会糟塌这种好景致。偶而有人从我身边超过,就打开车窗探出头来,对我大吼大叫,可想而知,是在问我是不是活腻了,想早点死。天上在打闪,闪电是紫色的,但听不到雷声。也许我该再编一个老师的故事来解闷,但又编不出来:我脑袋里面有个地方一直在隐隐作痛——这一天从早上八时开始,到凌晨三点才结束,实在是太长了。 
\section*{十一} 

我们生活在白银时代,我在写作公司的小说室里做事。有一位穿棕色衣服的女同事对我说:她要写小说。这就是前因。猜一猜后果是什么?后果是:我失眠了。失眠就是睡不着觉,而且觉得永远也睡不着。身体躺在床上,意识却在黑暗的街道上漫游,在寂静中飞快地掠过一扇扇静止的窗户,就如一只在夜里飞舞的蝙蝠。这好像是在做梦,但睡着以后才能做梦,而且睡过以后就应该不困。醒来之后,我的感觉却是更困了。 

我自己的小说写到了这里:“后来,老师躺在我怀里,把丝一样的短发对着我。这些头发里带着香波的气味。有一段时间,她一声都不吭,我以为她已经睡着了。我探出头去,从背后打量她的身体,从脑后到脚跟一片洁白,腿伸得笔直。她穿着一条浅绿色的棉织内裤。后来,我缩回头来,把鼻子埋在她的头发里。又过了一会儿,她对我说(轻轻地,但用下命令的口吻):晚上陪我吃饭。我在鼻子里哼了一声来答应,她就爬起身来,从上到下地端详我,然后抓住我内裤的两边,把它一把扯了下来,暴露出那个家伙。那东西虽然很激动,但没多大。见了它的模样,老师不胜诧异地说道:怎么会是这样!我感到羞愧无地,但也满足了我的恋母情结。其实,她比我大不了几岁,但老师这个称呼就有这样的魔力。” 

起床以后,我先套上一件弹力护身,再穿上衣服,就迷迷糊糊来上班。路上是否撞死了人,撞死了几个,都一概不知。停车场上雾气稀薄……今天早上不穿护身简直就不敢出门:那东西直翘翘的,像个棍面包。但在我的小说里,我却长了个小鸡鸡。这似乎有点不真实——脱离了生活。但这是十几年前的事——在这十几年里,我会长大。一切都这么合情合理,这该算本真正的小说了罢? 

“我在老师的床上醒来时,房间里只剩了窗口还是灰白色。那窗子上挂了一面竹帘子。我身上盖了一条被单,但这块布遮不住我的脚,它伸到床外,在窗口的光线下陈列着。这间房子里满是女性的气味,和夹竹桃的气味相似。夜晚将临。老师躺在我身后,用柔软的身体摩娑着我”——以前这个情景经常在我梦里出现。它使我感到亲切、安静,但感觉不到性。因为我未曾长大成人。 

现在我长了一脸的粉刺疙瘩,而且长出了腋毛和阴毛,喉结也开始长大。我的声音变得浑厚。更重要的是,那个往上翘的东西总是强项不伏……书上说,这种情况叫青春期。青春期的少年经常失眠。我有点怀疑:三十三岁开始青春期,是不是太晚一点了? 

早上我到了办公室,马上埋头劈里啪啦地打字,偶而抬起头来看看这间屋子,发现所有的人都在劈里啪啦地打字,他们全都满脸倦容,睡眼惺忪,好像一夜没睡——也不知是真没睡还是假没睡。但我知道,我自己一定是这个样子。我是什么样子,他们就是什么样子,所以我不需要带镜子——有的人还在摇头晃脑,好像脑壳有二十斤重。有人用一只手托在下巴上,另一只手用一个指头打字:学我学得还满像呢。只有“棕色的”例外,她什么都不做,只管瞪大了眼睛看着我,眼皮红通通的,大概一夜没睡。此人的特异之处,就是能够对身边的游戏气氛一无所知。我叹了口气,又去写自己的小说了…… 

“晚上,老师叫我陪她去吃饭,坐在空无一人的餐馆里,我又开始心不在焉。记得有那么一秒钟,我对面前的胡桃木餐桌感兴趣,掂了它一把,发现它太重,是种合成材料,所以不是真胡桃木的。还记得在饭快吃完时,我把服务员叫来,让她到隔壁快餐店去买一打汉堡包,我在五分钟内把它们都吃了下去。这没什么稀罕的,像我这样冥思苦想,需要大量的能量。最后付帐时,老师发现没带钱包。我付了帐,第二天她把钱还我,我就收下了。当时觉得很自然,现在觉得有些不妥之处。”假如我知道老师在哪里,就会去找她,请她吃顿饭,或者把那顿饭钱还给她。但我不知道她在哪里。老师早就离开学校了。这就是说,我失去了老师的线索。这实在是桩罪过。 

“我和老师吃完了晚饭,回到学校里去。像往常一样,我跟在她的身后。假如灯光从身后射来,就在地上留下一幅马戏团的剪影:驯兽女郎和她的大狗熊。马路这边的行人抬起头来看我一眼,急匆匆地走过;在马路对面却常有人站下来,死盯盯地看着我——在中国,身高两米一十的人不是经常能见到的。路上老师站住了几次,她一站住,我也就站住。后来我猛然领悟到,她希望我过去和她并肩走,我就走了过去——人情世故可不是我的长项。当时已近午夜,我和老师走在校园里。她一把抓住我肋下的肉,使劲捻着。我继续一声不吭地走着——既然老师要掐我,那就让她掐罢。后来她放开我,哈哈地笑起来了。我问她为什么要笑,她说:手抽筋了。我问她要紧不要紧,她笑得更加厉害,弯下腰去……忽然,她直起身来,朝我大喝一声:你搂着我呀!后来,我就抱着她的肩头,让她抱住我的腰际。感觉还算可以——但未必可以叫作我搂她,就这样走到校园深处,坐在一条长椅上。我把她抱了起来,让她搂着我的脖子。常能看到一些男人在长椅上抱起女伴,但抱着的未必都是他的老师。后来,她叹了一口气,说道:你放手吧。我早就想这样做,因为我感到两臂酸痛。此后,老师就落在了我的腿上。在此之前,我是把她平端着的——我觉得把她举得与肩平高显得尊重,但尊重久了,难免要抽筋。”写完了这一段之后,我把手从键盘上抬了起来,给了自己一个双锋贯耳,险些打聋了——我就这么写着,从来不看过去的旧稿,但新稿和旧稿顶多差个把标点符号。像这么写作真该打两个耳刮子——但我打这一下还不是为了自己因循守旧。我的头疼犯了,打一下里面疼得轻一点…… 
\section*{十二} 

今天早上我醒来之前,又一次闯进了埃及沙漠,被钉在十字架上,就如一只被钉在墙上的蝙蝠。实际上,蝙蝠比我舒服。它经常悬挂在自己的翅膀上,我的胳臂可不是翅膀,而且我习惯于用腿来走路。这样横拉在空中,一时半会儿的还可以,时间长了就受不住。我就如一把倒置的提琴被放置在空中,琴身是肋骨支撑着的胸膛——胸壁被拉得薄到可以透过光来。至于琴颈,就是那个直挺挺的东西。别的部份都不见了。我就这样高悬在离地很远的地方,无法呼吸,就要慢慢地憋死了。此时有人在下面喊我:她是克利奥佩屈拉,裹在白色的长袍里,问我感觉如何。我猛烈地咽口吐沫,润润喉咙,叫她把我放下去,或者爬上来割断我的喉咙。我想这两样事里总会有一样她乐意做的。谁知她断然答道:我不。你经常调戏我。这回我看清楚了:她不是克利奥佩屈拉,而是“克”。我说:我怎么会……你是我的上司,我尊敬还尊敬不过来呢。她说道:不要狡辩了,你经常写些乱七八糟的故事给我看——你什么意思吧。事已至此,辩亦无益。我承认道:好吧,我调戏了你——放我下来。她说:没这么便宜。你不光是调戏,你还不爱我——你还有什么可说的?我无话可说。沉默了一会儿,我忽然咆哮了起来……就这样醒过来了。我失掉了在梦里和“克”辩白清楚的机会:别以为光你在受调戏,我管着七个人,他们天天调戏我……你倒说说看,他们是不是都爱我?!这个情景写在纸上,不像真正的小说。它是一段游戏文章。我整天闷在办公室里,做做游戏,也不算是罪过。这总比很直露地互相倾诉好得多。 

昨天晚上,“棕色的”对我说,她要写真正的小说,这就是说,没有人要她写,是她自己要写的——正如亚里士多德说过的,假话有上千种理由,真话则无缘无故——她还扯上了亚里士多德,好像我听不懂人话似的。我还知道假话比较含蓄,真话比较直露。而这句话则是我听到过的最直露的一句话。如你所知,男女之间有时会讲些很直露的话,那是在卧室里、在床上说的。我实在不知道在什么人之间才会说:“我要写真正的小说”!我的小说就如我在写的这样。虽然它写了很多遍,但我不知道它哪一点够不上“真正的”。 

但“棕色的”所说的那些话就如碘酒倒到我的脑子里,引起了棕色的剧痛。上班以后,我开始一本正经地写着,这肯定有助于小说变成“真正的”。我觉得这一段落肯定是真正的小说:“那天晚上,我一直抱着老师,直到天明,嗅着她身上的女性气味——我觉得她是一种成熟的力量。至于我,我觉得自己是个小孩子。这种想法不能说没有道理,如你所知,现在我刚刚开始青春期,嘴角上正长粉刺疙瘩,当时就是更小的孩子。晚上校园里起了雾,这种白雾带有辛辣的气息。我们这样拥抱着,不知所措……忽然间,老师对我说道:乾脆,你娶了我吧——我听了害起怕来。结婚,这意味着两股成年的力量之间经常举行的交媾,远非我力所能及;但老师让我娶她,我还能不娶吗……但我没法乾脆。好在她马上说道:别怕,我吓你呢。既然是吓我,我就不害怕了。”有关成年力量间的交媾,我是这么想出来的:我现在是室里的头,上面的会也要参加,坐在会场的后排,手里拿着小本本,煞有介事地记着。公司的领导说得兴起时,难免信口雌黄:我们是做文化工作的,要会工作,也要会生活!今天晚上回家,成了家的都要过夫妻生活……活跃一下气氛,对写作也有好处。如你所知,我没成家。回到室里高高兴兴地向下传达。那些成了家的人面露尴尬之色。到了晚上九点半,那些成年的力量洗过了淋浴,脱下睡衣,露出臃肿的身体,开始过夫妻生活。我就在这时打电话过去:老张吗?今天公司交待的事别忘了啊。话筒里传来气急败坏的声音:知道!正做着——我操你妈……说着就挂掉了。我坐在家里,兴高彩烈地在考勤表上打个勾,以便第二天汇报,成年力量的交媾就是这样的。我和老师间的交媾不是成年力量间的那种。它到底该是怎样的,我还没想出来——我太困了。 

我忽然想到:在以前的十稿里,都没有写过老师让我娶她——大概是以前写漏了。现在把它补进去大概是不成的:“克”或者别的上司会把它挑出来,用红笔一圈,批上一句“脱离生活”。什么是生活,什么不是生活,我说了不算:这就是说,我不知道什么叫作生活。我摇摇头,把老师要我娶她那句话抹去了。 

有关夫妻生活,还有些细节需要补充:听到我传达的会议精神,我们室的人忧心忡忡地回家去。在晚上的餐桌上面露暧昧的微笑,鬼鬼祟祟地说:亲爱的,今天公司交待了要过生活……听了这句话,平日最温柔体贴的妻子马上也会变脸,抄起熨斗就往你头上砸。第二天早上,看到血染的绷带,我就知道这种生活已经过完了。当然也有没缠绷带来的,对这种人我就要问一问。比方说,问那朵最美丽的花。她皱着眉头,苦着脸坐在那里,对我的问题(是否过了生活)不理不睬,必须要追问几遍才肯回答:没过!我满脸堆笑地继续:能不能问一句,为什么没过?她恶狠狠地答道:他不行!我兴高彩烈地在考勤表上注明,她没过夫妻生活,原因是丈夫不行。每当上面有这种精神,我都很高兴。 

罗马诗人维吉尔有诗云:下雨天呆在家里,看别人在街上奔走,是很惬意的。所以,老师要我娶了她,我当然不答应。万一学校里布置了要过夫妻生活,我就惬意不起来,而且我也肯定是“不行”。 

我继续写道:“我对老师百依百顺,因为她总能让我称心如意。当然,有时她也要吓吓我。我在长椅上冥思苦想时,她对我耳朵喊道:会想死的,你!我抬头看看她的脸,小声说道:我不会。她说:为什么你不会?我说:因为你不会让我死。她愣了一下,在我腿上直起身来说:臭小子,你说得对。然后,她把绸衫后的乳房放在我脸上,我用鼻子在上面蹭起来。校园里的水银灯颜色惨白,使路上偶而走过的人看起来像些孤魂野鬼,但在绸衫后面,老师的乳房异常温柔——你要知道,在学校里我被视作尼斯湖怪兽,非常孤立。假如没有她肯让我亲近,我可真要死掉了。” 

因为这部小说写了这么多次,这回我想用三言两语说说我和老师的性爱经历:“那时候老师趴在床上,仔细端详我的那个东西。颠过来倒过去看够了以后,她说道:年复一年,咱们怎么一点都不长呢。后来,她又在我身上嗅来嗅去,从胯下嗅到腋下,嗅出这样一个结论:咱们还是没有男人味儿。我一声不吭,但心里恨得要死。看完和嗅完之后,老师跨到我身上来。此时我把头侧过去,看自己的左边的腋窝——这个腋窝大的不得了,到处凹凸不平,而且不长毛,像一个用久了的铝水勺。然后又看右面的腋窝。直到老师来拍我的脸,问我:你怎么了?我才答道:没怎么;然后继续去看腋窝。铝制的东西在水里泡久了,就会变得昏暗,表面还会有些细小的黑斑。我的腋窝也是这样的。躺在这两个腋窝中间,好像太阳穴上扣上了两个铝制水勺——我就这样躺着不动了。” 

“从老师的角度来看我,就会看到一张大脸,高鼻梁、高颧骨,眉棱骨也很高,一天到晚没有任何表情——我知道自己长得什么样子。老师送我到医院去看过病,因为我总是不笑,好像得了面部肌肉麻痹症。经过检查,大夫发现我没有这种毛病,只是说了一句:这孩子可真够丑的。这使老师兴高彩烈,经常冷不防朝我大喝上一声:真够丑的!做爱时我躺着不动,就像从空中看一条泛滥的河流,到处是河水的白光;她的身体就横跨在这条河上。我的那个东西当时虽小,但足够硬梆,而且是直撅撅的;最后还能像成年人一样射精。到了这种时候,她就舔舔舌头,俯下身来告诉我说:热辣辣的。因为我还能热这一下,所以她还是满意的……”这些段落和以前写的完全不同,大概都会被打回来重写,到那时再改回原样吧。我知道怎么写通得过,怎么写通不过。 

但我不大知道什么叫作生活。对于性爱经历,有必要在此补充几句:如你所知,这种事以前是不让写的。假如我写了,上面就要枪毙有关段落,还要批上一句:脱离生活。现在不仅让写,而且每部有关爱情的小说都得有一些,只是不准太过份。这就是说,不过份的性爱描写已经成了生活本身。自从发生了这种变化,我小说里的这些段落就越来越简约。那些成了家的人说:夫妻生活也有变得越来越简约之势。最早他们把这件事叫作静脉注射,后来改为肌肉注射,现在已经改称皮下注射了。这就是说,越扎越浅了。最后肯定连注射都不是,瞎摸两把就算了。我的小说写到最后,肯定连热都不热。 
\section*{十三}

“毕业以后,我还常去看老师。”写到这个地方全书就接近结束了。“我开了一辆黑色的吉普车,天黑以后溜进校园去找她,此时她准在林荫道上游荡,身上穿着我的T恤衫——衫子的下摆长过了她的膝盖,所以她就不用穿别的东西了。但她不肯马上跟我走,让我陪她在校园里溜溜。遇到了熟人,她简单地介绍道:我的学生来接我了。别人抬头看看我,说道:好大的个子!她拍拍我的肚子说:可不是嘛,个子就是大。有些贫嘴的家伙说:学生搞老师,色胆包天嘛!她也拍拍我的肚子说:可不是嘛,胆子就是大……咱们把他扭送校卫队吧。但是她说的不是事实,我胆小如鼠,她一吓我,我就想尿尿。有时她也说句实话:这孩子不爱说话,却是个天才噢。假如有人觉得她穿的衣服古怪,她就解释说:他的T恤衫,穿着很凉快,袖子又可以当蒲扇。有人问,天才床上怎么样(实际情况是,着实不怎么样),她就皱起眉头来,喝道:讨厌!不准问这个问题!然后就拖着我走开,说道:咱们不理他们——老师总是在维护我。”我的稿子总是这么写的,写过很多次了。按说它该是百分之百的真实。其实这事并未发生过。所有我写的事情都未真正发生过。 

也许我该从真正发生过的事情写起——我忽然想到,从老师的角度来看我,是个有趣的想法。老师留着乌黑的短发,长着滑腻的身体。我们学校的公共浴池是用校工厂废弃的车间改建的,原来的窗子用砖砌上了半截,挡住了外来的视线,红砖中间的墙缝里结着灰浆的疙瘩。顺着墙根有一溜排水沟,里面满是湿漉漉的头发。墙边还有一排粗状的水管连接着喷头,但多数喷头已经不见了,只剩下弯曲的水龙头,像旧时铁道上用来给机车上水的水鹤。在没有天花板的屋顶下挂了几个水银灯泡,长明不灭。水管里流着隔壁一家工厂的循环水,也是长流不息。 

这家浴室无人看守,门前的牌子上写着:周一三五女,二四六男,周日检修。这个规定有个漏洞,就是在夜里零点左右会出现男女混杂的情形。一般来说,没有人会在凌晨一点去洗澡,但我就是个例外。我不喜欢让别人看见我的身体,所以专找没人时去洗澡。有一回我站在粗壮的水柱下时,才发现在角落里有个雪白的身体……这件事发生在我上大一时,老师还没教过我们课——从她的角度看来,我罩在一层透明的水膜里,一动不动,表情呆滞,就如被冻在冰柱里一样。她朝我笑了笑,说道:真讨厌哪,你。然后就离去了。这就是一切故事的起因。 

从老师的角度来看我,会看到有一根水柱冻结在我头顶上,我的头发像头盔一样扣在脑袋上。一层水壳结在我的身上,在我身体的凸出部位,则有一些水注分离出来,那是我的耳朵、眉棱骨的外侧、鼻子、下巴。从下巴往下;直到腰际再没有什么凸起的地方了。有一股水柱从小命根上流下来,好像我在尿尿。那东西和一条即将成蛹的蚕有些相似。现在我不怕承认:我虽然人高马大、智力超群,却是个小孩子。直到不久之前,我洗澡和游泳都要避人。虽然我现在能把停车场上的小姐吓跑,但不能抹煞以前的事。老师说过我讨厌之后,就扬长而去,挺着饱满的乳房,迈开坚实的小腿,穿着一条淡绿色的内裤,蹋拉着一双塑料凉鞋。她把绿色绸衫搭在手臂上没穿,大概是觉得在我面前无须遮挡。此时在浴室里,无数的水柱奔流着。我站在水柱里,很不开心。小孩子不会愤怒,只会不开心。这就是这个故事的起因。这件事情是真实的,但我没有写。 

很多年来,我一直在老师的阴影下生活。这位老师的样子如前所述,她曾经拿根棍面包去吓唬露阴癖,还在浴室里碰见过我——但我们之间什么都没发生过。但我一直在写她:这是不是真正的小说,我有点搞不清楚了。也许,我还可以写点别的。比方说,写写我自己。我的故事是这样的:大学毕业以后,他们让我到国家专利局工作:众所周知,爱因斯坦就是在专利局想出了相对论,但我在那儿什么都没想出来。后来他们把我送到了国家实验室、各个研究所,最后让我在大学里教书。所有天才物理学家呆过的地方我都呆过,在哪儿都没想出什么东西来——事实证明,我虽然什么题目都会做,却不是个天才的物理学家;教书我也不行,上了讲台净发愣。最后,他们就不管我了,让我自己去谋生。我干过各种事:在饭店门口拉汽车门,在高级宾馆当侍者——最古怪的工作是在一个叫作丰都城的游乐宫里干的:装成恶鬼去吓唬人。不管干什么,都没有混出自己的房子,要租农民房住,或者住集体宿舍。我睡觉打呼噜,住集体宿舍时,刚一睡着,他们就往我嘴里挤牙膏,虽然夜里两点时刷牙为时尚早。最后我只好到公司来工作。公司一听我在外面到处受人欺负——这是我心地纯洁的标志——马上录取了我。同事都很佩服我的阅历,惊叹道:你居然能在外面找到事情做!但这并不是因为我明白事理,达练人情——我要真有这些本事就不进公司。我能找到这些工作只是因为我个子大罢了。 

当年我在丰都城里掌铡刀,别人把来玩的小姐按到铡刀下,我就一刀铡下去——铡刀片子当然是假的——还不止是假的,它根本就不存在,只是道低能激光。有的小姐就在这时被吓晕过去了,个别的甚至到了需要赶紧更换内裤的程度。另外一些则只是尖叫了一声,爬起来活动一下脖子,伸手到我身上摸一把。我赶紧跳开,说道:别摸——沾一手——全是青灰。不管是被吓晕的还是尖叫的,都很喜欢铡刀这个把戏。到下一个场景,又是我挥舞着钢叉,把她们赶进油锅:那是一锅冒泡的糖浆。看上去吓人,实际只有三十度——泡泡都是空气。这个糖浆浴是很舒服的:我就是这么动员她们往下跳,但没有人听。小姐们此时已经有了经验,不那么害怕,东躲西藏,上蹿下跳,既躲我手上的钢叉,又躲我腰间那根直挺挺的大阴茎。但也有些泼辣的小姐伸手就来拔这个东西,此时我只好跳进油锅去躲避——那是泡沫塑料做的,拔掉了假的,真的就露出来了。既然我跳了油锅,就不再是丰都城里的恶鬼,而是受罪的鬼魂。所以老板要扣我的工资,理由是:我请你,是让你把别人赶下油锅,不是让你下油锅的……作为雇员,我总是尽心尽责,只是时常忘了人家请我来做什么。作为男人,我是个童男子……这就是一切事实。结论是:我自己没什么可写的。 
\section*{十四} 

现在到了交稿的时间,同事们依次走到我面前。我说:放下罢,我马上看。谢谢你。与此同时,我头也不抬,双脚收在椅子下面——我既不肯枪毙他,也不让他踩我的脚。这就是说,我心情很坏。他放下稿子,悄悄地走出门去,就像在死人头前放上鲜花一样。我是这样理解此事:权当我的葬礼提前举行了。最后一个人走到我面前时,我也是如此说。她久久地不肯放下稿子,我也久久地不肯抬头看她。后来,她还是把稿子放下了。但她不肯走出去,和别人一样到屋顶花园去散步,而是走到桌子后面,蹲了下来,双手把我的一只脚搬了出来,放在地面上,然后站起身来,在上面狠命地一踩。这个人就是“棕色的”。我慢慢地抬起头来看着她,发现她的眼睛好像犯了结膜炎一样。我这一夜在失眠,她这一夜在痛哭。虽然她现在正单足立在我的足趾上,但我不觉得脚上比头里更疼——虽然足趾疼使头疼减轻了很多。这种行径和撒娇的坏孩子相仿,但我没有责备她。她见我无动于衷,就俯下身来,对着我的耳朵说:看见你的那东西了——难看死了!她想要羞辱我。但我还是无动于衷,耸了耸肩膀说:难看就难看吧。你别看它不就得了…… 

在我的小说里,我遇到了一个谜语:世界是银子的。我答出了谜底:你说的是热寂之后。现在我又遇到了一个谜语:“棕色的”女同事要写真正的小说。我应该答出谜底:你要写的是……我要是知道谜底就好了。也许你不像我,遇到任何谜语都要知道谜底。但你也不像我,从小就是天才儿童。希腊神话里说,白银时代的人蒙神的恩宠,终生不会衰老,也不会为生计所困。他们没有痛苦,没有忧虑,一直到死,相貌和心境都像儿童。死掉以后,他们的幽灵还会在尘世上游荡。我想他们一定用不着回答这样的问题:什么是真正的小说。如你所知,我一直像个白银时代的人。但自从在停车场上受到了惊吓,我长出一根大鸡巴来了。有了这种丑得要死的东西,我开始不像个白银时代的人了…… 

中午时分,所有的人都到楼顶花园透风去了,“棕色的”没去。抓住这没人的机会,她正好对我“诉求”一番——我不知这个词是什么意思,但我觉得这词很逗。她我面前哀哀地哭着,说道:老大哥,我要写小说啊……大颗大颗的泪珠在她脸上滚着,滚到下巴上,那里就如一颗正在溶化的冰柱,不停地往下滴水。我迷迷糊糊地瞪着她,在身上搜索了一阵,找到了一张纸餐巾(也不知是从哪里抄来的),递给了她。她拿纸在脸上抹着,很快那张纸餐巾就变成了一些碎纸球。穿着长裤在草地上走,裤脚会沾上牛蒡,她的脸就和裤脚相仿。我叹了口气,打开抽屉,取出一条新毛巾来,对她说:不要哭了,就给她擦脸。擦过以后,毛巾上既有眼泪,又有鼻涕,恐怕是不能要了。棕色的不停地打着噎,满脸通红,额头上满是青筋。我略感不快地想到:以后我抽屉里要常备一条新毛巾,这笔开销又不能报销——转而想到:我要对别人负责,就不能这么小气。然后,我对棕色的说:好了,不哭——回去工作吧。她带着哭腔说:老大哥,我做不下去——再扯下去又要哭起来。我赶紧喝住她:做不下事就歇一会儿。她说坐着心烦。我说,心烦的时候,可以打打毛衣,做做习题。她愣了一会说:没有毛衣针。我说:等会儿我给你买——这又是一笔不能报销的开支。我打开写字台边的柜子,从里面拿出一本旧习题集,递给她;叫她千万别在书上写字——这倒不是我小气,这种书现在很难买到了。 

过去,我做习题时,总是肃然端坐,把案端的台灯点亮,把习题书放在桌子的左上方,仔细削一打铅笔,把木屑、铅屑都撮在桌子的右上角,再用橡皮膏缠好每一支笔(不管什么牌子的铅笔,对我来说总是太细),发上一会儿呆,就开始解题了。起初,我写出的字有蚊子大小,后来是蚂蚁大小,然后是跳蚤大小,再以后,我自己都看不到了。所有的问题都沉入了微观世界。我把笔放下,用手支住下巴,沉入冥思苦想之中。“棕色的”情况和我不同,她把身体倚在办公桌上,脖子挺得笔直,眼睛朝下愤怒地斜视着习题纸,三面露白,脸色通红,右手用力按着纸张,左手死命地捏着一支铅笔(她是左撇子),在纸上狠命地戳着——从旁看去,这很像个女凶手在杀人——很快,她就粉碎了一些铅笔,划碎了一些纸张,把办公桌面完全写坏。与此同时,她还大声念着演算的过程,什么阿尔法、贝它,声震屋宇。胆小一点的人根本就不敢在屋里呆着。不管怎么说罢,我把她制住了。现在习题对我不起什么作用,我把这世界所有值得一做的习题都做完了。但我是物理系毕业的,数理底子好。“棕色的”则是学文科的——现有的习题够她做一辈子了。 

大学时期,我在宿舍里,硬把身体挤入桌子和床之间狭窄的空间坐下,面对着一块小小桌面和厚厚的一堆习题集发着呆。我手里拿着一支铅笔,但很少往纸上写,只是把它一节节地捏碎。不知不觉中,老师就会到来。她好像刚从浴室回来,甩着湿淋淋的头发,递给我一张抄着题目的卡片,说道:试试这个——你准不会。我慢慢地把它接过来,但没有看。这世界上没有我不会解的数学题——这是命里注定的事情。还有一件事似乎也是命里注定:我会死于抑郁症。不知不觉之中,老师就爬到了对面的双层床顶上,把双脚垂在我的面前。她用脚尖不停地踢我的额头,催促道:愣什么?快点做题!我终于叹了一口气,把卡片翻了过来,用笔在背面写上答案,然后把它插到老师的趾缝里——她再把卡片拿了起来,研究我写的字,而我却研究起那双脚来:它像婴儿的脚一样朝内翻着。我的嗅觉顺着她两腿中间升了上去,一直升入了皮制的短裙,在那里嗅到了一股夹竹桃的气息。因为这种气味,我拥有了老师洁白娇小的身体,这个身体紧紧地裹在皮革里……她从床上跳了下来,蹲在我的面前,抱住我的脑袋说:傻大个儿,你是个天才——别发愣了!我忽然觉得,我和老师之间什么都发生过——我没有虚构什么。 

我面对着窗子,看到玻璃外面长了几株绿萝。这种植物总是种在花盆里,绕着包棕的柱子生长,我还不知道它可以长在墙角的地下,把藤蔓爬在玻璃上。走近一点看得更清楚:绿萝的蔓条上长有吸盘,就如章鱼的触足一样,这些吸盘吸住玻璃,藤蔓在玻璃上生长,吸盘也像蜗牛一样移动着,留下一道粘液的痕迹,看起来有点恶心。然后它就张开自己的叶子。这些叶子有葵叶大小,又绿又肥,把办公室罩进绿荫里。科学技术在突飞猛进,有人把蜗牛的基因植到绿萝里,造出这种新品种——这不是我这种坐在办公室里臭编的人所能知道的事。 

我知道的是,坐在这些绿萝下,就如坐在藤萝架下。这种藤萝架可以蔓延数千里,人也可以终生走不出藤萝架,这样就会一生都住在一道绿色的走廊里,这未尝不是一种幸福。这不是不能实现的事:只要把人的基因植到蚂蚁里,他(或者她)觉得自己是人,其实只是蚂蚁;此后就可以在一个盆景里得到这种幸福,世界也会因此变得越来越新奇。……我回头看看“棕色的”,在绿荫的遮蔽下,显得更棕了。她吭吭哧哧地和一些三角恒等式纠缠不休。这是初中二年级的功课,她已经有三十五岁了。我不禁哑然失笑:以前我以为自己只有些文学才能,现在才发现,作践起人来,我也是一把好手。我真不知道自己有多聪明——而且我现在还是迷迷糊糊的。我就这么迷迷糊糊地回家去睡觉——再不睡实在也撑不住了。 
\section*{十五} 

天终于晴了。在雾蒙蒙的天气里,我早就忘了晴天是什么样子,现在算是想起来了。晴天就是火辣辣的阳光——现在是下午五点钟,但还像正午一样。我从吉普车里远远地跳出去,小心翼翼地躲开金属车壳,以免被烫着,然后在沾脚的柏油地上走着。远远地闻见一股酒糟味,哪怕是黑更半夜什么都看不见,闻见这股味也知道到家了。这股馊臭的味道居然有提神的功效。闻了它,我又不困了。 

我宿舍的停车场门口支着一顶太阳伞,伞下的躺椅下躺着一个姑娘,戴着墨镜,留着马尾辫,穿着鲜艳的比基尼,把晒黑了的小脚翘在茶几上。我把停车费和无限的羡慕之情递给她,换来了薄薄的一张薄纸片——这是收据,理论上可以到公司去报销。但是报销的手续实在让人厌烦。走过小桥时,下面水面上飘着密密麻麻的薄纸片,我把手上的这一张也扔了下去。这条河里的水是乳白色的,散发着酒糟和淘米水的味道。这股水流经一个造酒厂,或者酱油厂,总之是某个很臭的小工厂;然后穿过黑洞洞的城门洞——我们的宿舍在山上,是座城寨式的仿古院子——门洞里一股刺眼睛的骚味,说明有人在这里尿尿。修这种城门洞就是要让人在里面尿尿。门洞正对着一家韩国烧烤店,在阳光下白得耀眼。在烧烤店的背后,整个山坡上满是山毛榉、槭树,还有小小的水泥房子。所有的树叶都沾满了黑色的粉末,而且是粘糊糊的——叶子上好像有油。山毛榉就是香山的红叶树,但我从没见它红过;到了秋天,这山上一片茄子的颜色。这地方还经常停电。 

为了这一切——这种宿舍、工资,每天要长衣长裤地去上班,到底合算不合算,还是个问题。我现在穿的远不是长衣长裤。刚才在停车场上付费时,我从那姑娘的太阳镜反光里,看清了我自己的模样。我穿着的东西计有:一条一拉得领带,一条很长的针织内裤,里面鼓鼓囊囊的,从内裤两端还露出了宽阔的腹股沟,和黑毵毵的毛——还有一双烤脚的皮鞋,长衣长裤用皮带捆成一捆背在了背上;手里还提着一个塑料冰盒子。那个女人给我收据时,嘴角露出了一丝笑意,可见别人下班时不都是这种穿着。她的嘴角松弛,脖子上的皮也松弛了,不很年轻了。但这不妨碍我对她的羡慕之情。看守停车场和我现在做的事相比,自然是优越无比。 

我的房子在院子的最深处,要走过很长的盘山道才能走到。这是幢水泥平房,从前面走进门厅,就会看到另一座门,通向后院。这两道门一模一样,连门边的窗户也是一模一样。早上起来,我急匆匆地去上班,但时常发现走进了后院。后院里长满了核桃树,核桃年复一年落在地下,青色的果壳裂开,铺在地下,终于把地面染得漆黑。至于核桃坚果,我把它扫到角落里,堆成了一堆。这座院子的后墙镶在山体上,由大块的城砖砌成,这些砖头已经风化了,变成了坚硬的海绵。但若说这堵墙是古代遗留下来的,又不大像。我的结论是:这是一件令人厌恶的假古董——墙上满是黑色的苔藓。在树荫的遮蔽下,我的后院漆黑一团。不管怎么说罢,这总是我自己的家。每当我感到烦闷,想想总算有了自己的家,感觉就会好多了。不知你见没见过看停车场的房子——那种建筑方头方脑,磨砖对缝。有扇窗子对着停车场的入口,窗扇是横拉的,窗下放着一张双屉桌,桌子后面是最好的发愣场所;门窗都涂着棕色的油漆,假如门边不挂牌子,就很容易被误认为收费厕所。这房子孤零零的,和灯塔相似。 

日暮时分,我走到门外,在落日的余晖下伸几个懒腰,把护窗板挂在窗户上,回到屋里来,在黑暗中把门插上,走进里间屋——这间房子却异常明亮。灿烂的阳光透过高处的通气窗,把整个顶棚照亮。如你所知,这屋里有张巨大的床。我的老师穿着短短的皮衣,躺在床上。她的手臂朝上举着,和头部构成一个W形,左手紧握成拳,右手拿着小皮包,脖子上系着一条纱巾——老师面戴微笑。她的双脚穿着靴子,伸到床外。实际上,她是熟睡中的白雪公主。 

我在她身边坐下,床瘪了下去,老师也就朝我倾斜过来。我伸手给她脱去靴子,轻轻地躺了下来,拉过被子把自己盖住,睁大眼睛看着天花板——它正在一点点地暗下去。第二天早上,我又会给老师穿上靴子,到外面上班……老师会沉睡千年,这种过程也要持续千年。我们之间什么都没发生过——虽然那东西一直是直翘翘的。这件事没法写进小说里,因为它脱离了生活。按现在的标准,生活是皮下注射。但这不是真正的生活。什么是真正的生活呢?我又记不得了。这个故事我写了十一遍,我能记住其中的每一句话。但它是真是假,我却记不得了! 

我在家里,脱掉内裤,解开腰上的重重包裹。旧时的小脚女人在密室里,一定也是怀着同样的欣快感,解开自己的裹脚布。那东西获得了解放,弹向空中。我现在有双重麻烦:一是睡不着觉,二是老直着。我还觉得自己在发烧,但到医务室一量体温,总是三十六度五——那东西立在空中,真是丑死了。在学校里,我是天才学生,在公司里我是天才人物。你知道什么是天才的诀窍吗?那就是永远只做一件事。假如要做的事很多,那就排出次序,依次来干。刚才在公司,这个次序是:1、写完我的小说;2、告诉棕色的什么是真正的小说。现在的次序是:1、自渎;2、写完小说;3、告诉棕色的什么是真正的小说。在此之前,我先去找一样东西。这次序又变成了:1、找到那样东西;2、自渎……这样一个男人,赤身裸体,在家里翻箱倒柜,这样子真是古怪透了……但我还是去找了,并把它从床底下拖了出来。把那个破纸箱翻到底,就找到了最初的一稿。打印纸都变成了深黄色,而且是又糟又脆,后来的稿子就不是这样:这说明最早的一稿是木浆纸,后来的则是合成纸。这一稿上还附有鉴定材料:很多专家肯定了它的价值,所以它才能通过。 

现在一个新故事也得经过这样的手续才能出版、搬上银幕——社会对一个故事就是这么慎重。每页打印纸上都有红墨水批的字:属实。以下是签字和年月日。在稿上签字的是我的老师。为了出版这本书,公司把稿子交她审阅,她都批了属实。其实是不属实。不管属实不属实,这些红色的笔迹就让我亢奋。假设小说的女主人公是克利奥佩屈拉,就没人来签字,小说也就出不来。更不好的是:手稿上没有了这些红色笔迹,就不能使我亢奋。 

如你所知,我们所写的一切都必须有“生活”作为依据。我所依据的“生活”就是老师的签字——这些签字使她走进了我的故事。不要以为这是很容易的事:谁愿意被人没滋没味地一遍遍写着呢。老师为我做出了重大的牺牲。后来我到处去找老师,再也找不到——她大概是躲起来了。但是这些签字说明她确实是爱我的——就是这些签字里包含的好意支持着这个故事,使我可以一遍遍地写着,一连写了十一次。 
\section*{十六} 

他们现在说,我这部小说有生活。他们还说,现在缺少写学生生活的小说。我说过,生活这个词有很古怪的用法:在公司内部,我们有组织生活、集体生活。在公司以外,我们有家庭生活、夫妻生活。除此之外,你还可以去体验生活。实际上,生活就是你不乐意它发生但却发生了的事……和真实不真实没有关系。我初写这部小说时,他们总说我的小说没有生活,这不说明别的,只说明当时这篇小说在生活之外,还说明我很想写这篇小说;现在却说有了生活,这不说明别的,只说明它完全纳入了生活的轨道,还说明我现在不想写这篇小说了。 

老师的生活是住在筒子楼里,每天晚上到习题课上打瞌睡,在校园里碰上一个露阴癖;而和一个大个子学生恋爱却不在她的生活之中。她在我的初稿上签字,说我写到的事情都是她的生活,原因恰恰是:我写到的不是她的生活——这件事起初是这样的。结果事情发展下去走了味儿:我一遍遍地写着,她一遍遍地签字,这部小说也变成了她的生活。所以她离开了学校,一走了之。 

早上我去上班之前,要花大量的时间梳妆,把脸刮乾净,在脸上敷上冷霜,描眉画目。这是很必要的,我的脸色白里透青,看上去带点鬼气,眉毛又太稀。然后在腋下喷上香水,来掩饰最近才有的体味。我的形体顾问建议我穿带垫子的内衣,因为我肌肉不够发达。他还建议我用带垫子的护身,但现在用不着了,那东西已经长得很大。然后我出门,在上班的路上还要去趟花店,给棕色的买一束红色的玫瑰花。在花店里,有个穿黑皮短裙的女孩子对我挤眉弄眼,我没理她。后来她又跟我走了一路,一直追到停车场,在我身后说些带挑逗意味的疯话……最后,她终于拦住我的车门,说道:大叔,别假正经了——你到底是不是只鸭?我闷声喝道:滚蛋!把她撵走了。这种女孩子从小就不学好,功课都是零分,中学毕业就开始工作;和我们不是一路人。然后我坐在方向盘后面咳声叹气,想着“棕色的”从来就没有注意过我。要是她肯注意我,和我闲聊几句,起码能省下几道数学题。她解题的速度太快,现有的数学题不够用了。 

有关棕色的女同事要写真正的小说,我现在有如下结论:撇开写得好坏不论,小说无所谓真伪。如你所知,小说里准许虚构,所以没有什么真正的小说。但它可以分成你真正要写的小说和你不想写的小说。还有另外一种区分更有意义:有时候你真正在写小说,但更多的时候你是在过着某种生活。 

这也和做爱相仿:假如一个男人和一个女人双方都想做,那他们就是真正在做爱。假如他们都不想,别人却要求他们做,那就不是做爱,而是在过夫妻生活。我们坐在办公室里,不是在写小说,而是在过写作生活。她在这种生活中过腻了,就出去体验生活——这应该说是个错误。体验到的生活和你在过的生活其实是毫无区别的。我知道,“棕色的”要做的事是:真正地写小说。要做这件事,就必须从所谓的生活里逃开。想要真正地写,就必须到生活之外。但我不敢告诉她这个结论。我胆子很小,不敢犯错误。 

现在“棕色的”每天提前到班上来,坐在办公桌后面,一面打毛衣,一面做习题。她看起来像个狡猾无比的蜘蛛精,一面操作着几十根毛衣针,一面看着习题集——这本习题集拿在一位同事的手里。她嘴里咬着一支牙签,把它咬得粉碎,再吐出来,大喝一声:“翻片儿”!很快就把一本习题集翻完,她才开始口授答案。可怖的是,没有一道做错的。我把同事都动员起来,有的出去找习题,有的给她翻片儿。我到班上以后,把这束玫瑰花献给她,她只闻了一下,就丢进了字纸篓,然后哇哇地叫了起来:老大哥,这些题没有意思!我要写小说!她一小时能做完一本习题集,但想不出真正的小说怎么写,让我告诉她。按理说,我该揍她个嘴巴,但我只叹了一口气,安慰她道:不要急,不要急,我们来想办法;然后坐到自己的位子上了。 在“棕色的”写作生活中,她在写着一个比《师生恋》更无聊的故事。她和我们的不同之处在于,她不会瞎编一些故事来发泄愤怒。因此她就去体验生活,然后被人轮奸了。这说明她很笨,不会生活。既然生活是这样的索然无味,就要有办法把它熬过去。这件事可不那么容易……起码比解习题要难多了。“棕色的”告诉我说:那件事发生以后,她坐在泥地上,忽然就怕得要命。也不知为什么,她想到这些人可能会杀她灭口……她想得很对,强奸妇女是死罪,那些乡下小伙子肯定不想被她指认出来。虽然当时很黑,但她说,看到了那些人在背后打手势。这是件令人诧异的事:我知道,她原来像蝙蝠一样的瞎,在黑地里什么都看不见。但我平时像个太监,被刀尖点着的时候,也变得像一门大炮;所以这件事是可信的。有一个家伙问她:你认不出我们罢?她顺嘴答道:认不出来。你们八个我一个都认不出来。那些人听了以后,马上就走,把她放过去了。这个回答很聪明:明明是四个人,她说是八个。换了我,也想不出这么好的脱身之策。但她因此变得神经兮兮的,让我猜猜她为什么会这么怕死。如你所知,我最擅长猜谜,但这个谜我没猜出来。这谜底是:我这么怕死,说明我是活着的。这真是所罗门式的答案!现在恐怕不能再说她是傻瓜了。实际上,她去体验生活确实是有收获的。首先,她发现了自己不想死,这就是说,她是活着的。既然她是活着的,就有自己的意愿。既然有自己的意愿,就该知道什么是真正在写小说。但她宁愿做个吃掉大量习题的母蝗虫,也不肯往这个方向上想。我也不愿点破这一点:自己在家里闷头就写,不要告诉任何人。这样就是真正在写小说。我不敢犯错误,而且就是犯了错误,也不会让你知道。我注意到“棕色的”总在咬牙签,把齿缝咬得很宽。应该叫她不咬牙签,改吃苹果——照她这个疯狂的样子,一天准能吃掉两麻袋苹果,屙出来的屎全是苹果酱……我现在是在公司里,除了“生活”无事可做。所以,我只能重返大学二年级的热力学教室,打算在那里重新爱上老师。
