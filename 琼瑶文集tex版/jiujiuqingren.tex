\chapter{舅舅情人}

高宗在世的时候,四海清平,正是太平盛世,普天下的货殖流到帝都。长安是当时世界上第一壮丽大城。城里立着皇上的宫城,说不尽的琼楼玉宇,雕梁画栋,无论巴格达的哈里发,还是波斯的皇帝,都没见过这样的宫殿。皇上有世界上最美的后妃,就连宫中的洗衣女,到土耳其的奴隶市场都能卖一斗珍珠的价钱。他还吃着洋人闻所未闻的美味,就连他的御厨泔水桶中的杂物都可以成为欧洲子爵、伯爵,乃至公爵、亲王席上的珍馐。他穿着金线剌绣的软缎,那是全世界的人都没见过的。皇上家里用丝绸做擦桌布,用白玉做磨刀石,用黄金做马桶,用安南的碧玉砌成浴池。他简直什么也不缺,于是他就得了轻微的抑郁症。 

有一天,有一位锡兰的游方僧到长安来。皇帝久仰高僧的大名,请他到宫里宣讲佛法。那和尚在皇帝对面坐下,没有讲佛家的经典,也没有讲佛陀的事迹,只是讲了他一路上的所见所闻。他说月圆的夜晚航行在热带的海面上船尾拖着磷光的航迹。还说在晨光熹微的时候,在船上看到珊礁上的食蟹猴。那些猴子长着狗的脸,在礁盘上伸爪捕鱼。他谈到热带雨林里的食人树。暖水河里比车轮还大的莲花。南方的夜晚,空气里充满了花香,美人鱼浮上水面在月光下展示她的娇躯。皇上富有天下,却没见过这样的景观。他起初想把这胡说八道的和尚斩首,后来又变了主意,放他走了。 

锡兰僧走时,送给皇上一个骨制的手串,上面写满难认的梵文。皇上不认识梵文,他宫里也没有骨制的东西,可是他特别珍视这串珠子。因为把它握在手里里,皇帝就能看见锡兰僧讲到的一切(这当然是心理作用)。他虽然富有,却不能走出皇宫一步。所以他想,做皇帝也未必是一件好事。所有的人都不知道,只有皇帝自己和当过皇帝的人知道,当皇帝会得皇帝病。对花粉过敏,对青草过敏,甚至对新鲜的空气也过敏。如果到宫内最高的云阁上看长安城里的绿荫,下来以后他要鼻塞气重好几天,还要长一身皮疹。除此之外,他还只能吃御厨中精心制作盛在银碗里的食物。如果吃一碗坊间的大锅里熬出盛在粗瓷碗里的羊杂碎,他就会腹泻三天。他也只能和宫内肌肤如雪像花蕊一样娇嫩的女子做爱。如果叫太监从外边弄一个筋粗骨壮的农家女子来,他闻到她身上的汗味就要头晕。听到锡兰僧讲的故事,皇上觉得自己是一个宫禁中的囚徒。于是他再不和后妃嬉戏,再不理朝见的臣子,把自己关在密室中,成天只和那串骨珠亲近。 

皇上在密室的天窗中,看到天上的大雁飞过,看到檐下的铃铛随风摇摆,看到屋脊的阴影在阳光下伸长,消失,又在月光下重现。看到瓦上雪消失,岩松返回青又枯黄。转眼间几度寒暑,他不招后妃侍寝,不问天下大事,只向送饭的太监打听锡兰僧的消息,谁知那和尚一去音讯全无。 

有一天,大食的使节从遥远的西域到来,带来了大食皇帝的国书。皇上虽然心情忧郁,也不能冷落了这使团,因为大食和大唐一样强大。大食的骑兵骑在汗血的天马上,背着弓,口里衔着箭,常常骚扰帝国的边境。大食的皇帝有意修好,正是大唐求之不得的事。皇帝身为人君,有不可推卸的责任去制止边乱。于是,他升殿,带着高贵的微笑去接见使团。他问使节们沿途见到的景色,使节们却听不懂。使节们说话,他也听不懂。皇帝觉得兴味索然,叫宰相陪他们国宴,自己回密室去。他晚上六点钟离开密室,九点钟回去,就在这三个小时中,有人潜入那间屋子,把手串偷走了。皇帝因此而发怒,命令将守在密室门口的宫女和太监严刑拷打,打得他们像猫一样悲鸣。皇帝想把他们都活活打死,后来又改变了主意,把他们交给最仁慈的皇后感化教育,要他们说出是谁偷走了手串。他又召长安里的捕盗高手入宫来现场调查,要他们说出是谁偷走了手串。高手们说不出,皇上大发雷霆,要把他们推出午门斩首。后来又改变主意,赦免他们死刑,只是命令禁卫军把全城捕盗公差的家属全抓到牢里,以免公差们忙于家事不能专心破案。他还命令封闭城门,只留一个门供出入,出城的人都要经过严格的搜查。然后他觉得无聊,就回到密室中去,叫太监们找到手串时通知他一声。 

与此同时,长安城里全体捕盗公差在京兆尹衙门的签事房里集合,讨论案情。时值午夜,人们点起红烛,进宫的几位白胡子和花白胡子的公差痛哭流涕地说到皇恩浩荡,留下他们不值一文的蚁命。当今的圣上仁德光焰无际,草木被恩,连下九流的公差都身受皇恩。如果不能寻回手串,无须皇上动手,他们就要一头碰死。大家听了感动得热泪盈眶,齐声赞美皇帝的恩德,然后静下心来,在灯光下思考皇帝手串的去向,直想到红烛将尽,晨光熹微,谁也想不出一点线索来。 

众所周知,皇城的城墙是磨砖对缝的,高有四丈,墙下日夜站着紫衣禁卫军。长安城里最高明的贼翻越高墙也要借助飞抓绳梯,这种手段在皇城上可无法使用。可是说是皇宫里的人偷走手串呢,那就更不能想像。当今的圣上是百年不遇的仁君,虽升斗小民,也知道敬上,何况是皇城内的人直接身受皇恩?更何况皇帝是世界上一切爱的本源,人人爱皇帝,皇帝爱大家。不管是谁,只要不爱皇帝,就生活在黑暗之中,简直活不过一个小时。在皇城之外,也许还有个把丧心病狂的贼子敢偷圣上的心爱之物,在皇城内这种人绝不可能存在。公差们想到脑门欲碎,一个个倒在长凳上睡着了。 

当五月的热风吹入签事房时,房子里青蝇飞舞。公差们醒来,想到皇上圣心焦虑地等待他们追回手串,就羞愧起来。几位老资格的公差说,大家都到街上去见到形迹可疑之人,就捉回来严加拷问,用这种方法也许能追回圣上的失物。于是大家都到街上去。连勒死贼的公差王安也跟着出去了。 

王安在长安做了十年的公差,从没捉到一个活着的贼。他的身材过于魁梧,按唐尺,身高九尺有余,按现代公制,身高也有两米。膀宽腰细,长髯过腹,浓眉磊眼,声如洪钟。像这样的仪容,根本就不适合当公差。何况他当公差的第一天在街上看到有人行窃,就一链锁住贼的脖子,把他拖到衙门里去。谁知用力过猛,把贼勒死了,从此也就再没捉到过贼。于是全长安的贼无不知王安的大名。他在街头出现,贼就在街尾消失。 

其实像王安这样的人,何必去当公差?他可以当一名紫衣禁卫军。当禁卫军不要武艺,只要身高和胡子,这两样东西王安都具备,他甚至可以到皇城门前城去当执戟郎。唐朝风气与宋明不同,官宦人家的小姐常常出来跑马踏青,她们看到雄壮的执戟郎,就用怀中的果子相赠。郡主、公主也常常飞马出宫入宫,看到仪容出色的武官,就叫他们跟着到她们的密室去,用胡子轻拂自己的娇躯,事后都以价值连城的珠宝做为定情礼物。王安当一名下九流的公差,把他一生的风流艳遇都耽误了。 

王安和公差们一起出来,别人都到通衢大道、热闹的商坊去,谁也不肯和王安结伴而行。他只好和同伴告别,走在坊间的大道上。长安街内一百零八坊,坊坊四里见方,围着三丈高的坊墙,四角的更楼高入云天,坊与坊之间有半里宽的空旷地带,植满了槐树。唐代的长安城多么大呀,大过了罗马,大过了巴比伦,大过巴格达,大过了古往今来的一切城池。王安在坊间的绿荫中走,到处碰不到一个人。 

长安城里多数都是热闹的小城池,可是远离坊门的绿荫地带,却少见人迹,更何况王安朝长安城西北角的鬼方坊走去,那儿更加荒凉。高高的茅草封闭了大路,只剩下羊肠小路。鬼方坊的坊墙,,墙皮斑脱,露出了砌墙的土坯。墙下明渠里流的水像脓一样绿,微风吹过时,树上落下干枯的槐花,好像一阵大雨。 

鬼方坊的更楼呀,全都坍塌啦。四个坊门有三个永久封闭,只剩下一个门供人出入。那榆木的大门都要变成栅栏门啦!正午时分,一只眼的司阍坐在门楼下的阴影中缝衣裳,他就在身上缝衣,好像猴子在捉虱子。走进坊内,只见一片荒凉,到处是断壁残垣,枯树荒草,这个坊已经荒了上百年。 

除了自己和老婆,再加上这位老坊吏,王安再不知道还有谁在这鬼方坊里居住。站在坊门内的空场上,王安极目四望,只看到坊中塌了半截的高塔顶上长满荒草的亭子。土石填满的池塘里长满荆棘,早年的假山挂着几段枯共藤。远处有一道长廊,屋顶塌断了几处,就如巨蟒的骨骼。这荒坊里一片枯黄,见不到几处绿色。 

王安确实知道还有人住在坊中,可是他没见过这个人或者这些人。坊墙的内侧完整,涂满了鸡爪子小人。王安问老司阍这些顽童图画的事,却发现这老头儿又聋又糊涂,口齿不清地说一口最难懂的山西话,完全不能听懂他的意思。王安就沿着坊墙下的小道回家去,沿途研究那些壁画,他觉得这作画技巧很不寻常。 

王安走过一排槐树。说也奇怪,长安城里的槐树不下千万棵,都不长虫子,只有鬼方坊的槐树长槐蚕。才交五月,这一树绿叶已经被虫子吃得精光,只余下一树枯黄的叶脉,就如西域胡人的鬈胡须。有一个穿绿衫的女孩在树下捉槐蚕,她看到王安走来,就站起来叫:“舅舅!舅娘被人捉走了!” 

王安吃了一惊。首先,他不认识这个人。其次,这个女孩真漂亮,披着一头乌油油的黑发,眼睛像泉水一样亮,嘴唇像花儿一样红,两个小小的乳房微微隆起,纤小的手和脚,好像长着鸟的骨骼。最后,她捉了槐蚕就往衣裳里放,她穿一身槐豆染绿的长袍,拦腰束一根丝绳,无数的槐蚕就在腰上的衣内蠕动。王安看了脊背发凉。至于她叫他舅舅,这倒是寻常的事。那时候女孩管成年男子都叫舅舅。 

王安朝她点点头说:“你看到了?是谁来捉她的?” 

“一伙穿紫衣的兵爷,他们叫舅娘跟着走,舅娘不肯,他们就把舅娘捉住,用皮条捆住手脚,放到马背上就走了。临走抽了看门大爷一鞭子,叫他把路修修。这些兵,真横。” 

王安听完这些话,就径直回家去。那个女孩把腰带一松,无数槐蚕落在地上,她把它们用脚踩碎,染了一脚的绿汁,然后就追到王安家里来。 

王安住着一间小小的草房,门扇已被人踢破,家里的家具东倒西歪,好像经过了一场殊死搏斗。王安把家什收拾好,坐在竹床上更衣。脱下旧衣,却没有新衣可换,只好在衣柜里挑一件穿过而不大脏的衣服穿上了。这时他听见有人说:“舅舅的肩真宽,胳膊真粗!”这才发现那个女孩不知什么时候溜了进来,站在阴影中。 

王安说:“甥女儿,你这样不打招呼就进来很不好。” 

女孩说:“舅舅,我的话还没说完呢!舅娘临走时大骂你的祖宗八代,这是怎么回事?” 

“这不干你的事,你刚才在干什么?” 

“捉槐蚕,喂鸡。” 

“那你就再去捉槐蚕吧。” 

女孩想了想说:“舅舅,我不捉槐蚕,鸡也有东西吃。现在我有更重要的事做。舅娘被捉走了,你的衣服没人洗。我给你洗衣服,挣的钱比捉槐蚕一定多。” 

王安确实需要人洗衣服,他就把脏衣服包起来交给她。女孩抱着衣服,闻了闻上面马厩似的气味,却觉得很好闻。她看到王安把头扭过去,好像不爱看这景象,就问: 

“舅舅,舅娘为什么骂你?” 

“皇上丢了东西,要舅舅捉贼,把舅娘捉起来当人质。舅舅破不了案,舅娘就要住哧牢,吃馊饭。所以她骂我。” 

女孩说:“那也不应该,像舅娘这样的女人,嫁了舅舅这样的男人,还不知足吗?别说坐几天牢,丢了命也值!” 

王安又躺到竹床上去,眯起眼睛来想:“她知道我老婆又凶又懒。怎么知道的?” 

王安的老婆很凶悍,十根指头都会抓人。王安知道那些禁卫军来捉她,脸上一定会挂彩,所以她到牢里会比别的女人多吃苦头。因此,必须早点把她救出来。他闭上眼睛,那女孩以为他睡着了,其实王安在回味以前的事。晚上行房之前,他老婆来把玩他的胡须。王安的胡子又软又亮,好像美女的万缕青丝。他老婆把手插到那些胡子之中,白日的凶悍就如被水洗去,只剩下似水柔情。那个女孩看到这些胡子,也想来摸一把,可是他翻了一个身,把胡子压到身下,叫她摸不到,于是她叹一口气,走出门去了。 王安睁开一只眼睛,看那破门里漏进来的阳光,他想起老婆乳头上那七点蜘蛛痣,状如北斗七星。那些痣的颜色,就如名贵的玛瑙上的红绦。那些痣在灯光、月光、星星下都清晰可见,就似王安对她的依恋之情。那女人白天和夜晚是两个人;白天是夜叉,夜里似龙女。白天是胀起脖子的眼镜蛇,晚上是最温顺的波斯猫。她为什么会这样,王安一直弄不明白,越是弄不明白,王安就越爱她。 

第二天,王安一到衙门点卯,发现签事房里一片绝望的气氛。昨天在竹床上打盹时,他的同事在街上捉了上百个贼,搜出几十串骨珠来。经过刑讯,有七八个贼承认骨珠是从宫里偷来。他们把那些骨珠送进宫里,皇上看了大发雷霆,说谁敢送这样的假货来,就把他阉了做太监。 

公差抱怨说,捉到贼搜出骨珠,不经过严刑拷打,没有人知道这珠子是不是从宫里偷的。经过拷打后贼承认是从宫里偷来的,又没有人知道他是不是屈打成招。最后只好请皇帝御览作为最终鉴定,可皇上要把他们阉了做太监。如果被阉了做成太监,就算最终捉到真贼,皇上把老婆发还,她们又没用处了,这种曲折的事情,伟大圣明的天子怎么会体会不到? 

皇上坐在深宫的密室中,眼皮直跳。他知道这是有人在议论他,马上就想到,是那帮黑乌鸦似的公差在嚼舌根子。他在神圣的愤怒之中,想下一道圣旨,把全体公差马上阉掉。可是他马上又变了主意,不发这圣旨了。阉公差,是他有把握能做的事,有把握的事为什么要着急呢? 

皇上平时坐在密室里时,手里总握着那串骨珠。他能够看到热带的雨林,雾气蒸腾的沼泽地,看到暖水河里黑朽的树桩,听到锡兰僧沉重的鼻息。他还能感到锡兰僧在泥水中拔足时沉重的心跳,闻见水沼的气味里合着童身僧侣身上剌鼻的汗酸。直到疲惫之极,他才松开手,让那些灰暗暖润的珠子在指间滑落。现在没有这串珠子,皇上就禁不住焦躁,要走出这间密室,到王座上发号施令,把公差痛责一顿,阉掉京兆尹,把守门的太监和宫女送去杀头。可是他马上改变了主意,决定不出去。这是容易佬的事情,容易佬的事情何必要着急呢? 

就是珠串在手,皇上也有心火上升的时候。那时候他也想走出密室,到皇后身边去。二十七岁的皇后,肌肤像抛光的白玉一样透明。她从出世以来就没吃过饭,全靠喝清汤度日。皇上想闻闻皇后身上的肉香,她身上的奇香与生俱来,有勾魂摄魄的效力,皇上每次闻了以后,都禁不住春情发动。 

行房对娇嫩的皇后来说,无疑是残酷的肉刑。但是皇后从没拒绝过皇帝,也没有过一句怨言。皇帝因此判定,在全世界的人中,只有她真正爱他。所以一想到皇后他总禁不住心花怒放。但是每次这么想过之后,皇帝又改变了主意,到皇后身边去是最容易做的事。容易做的事何必着急呢? 

皇上想追回遗失的手串才是难做的事。可是他又不乐意走出密室。这不是军国大事,不便交给宰相去办,于是他就把追回手串的事,交皇后全权代理。虽然三年不见面,可是他相信,全世界的人只有皇后最明白他的心意。她一定能把手串追回来,他还要人告诉皇后,那虽是一串普通的骨珠,却是锡兰僧长途跋涉时握在右手里的,所以有特殊的意义。 

皇帝说那是一串普通的珠子,可是公差不信,他们认为皇帝身边的东西,一定佛国异宝,起码也是舍利子制成。据说,舍利子那种东西会发出佛光,只有有福气的人和高僧才能看到。所以以后再找到骨珠,应该先送到名山大刹请高僧过目,验明是佛宝之后,再往宫里送。听了这样的议论,王安吐吐舌头,走到签事房外边来。他远眺高耸入云的皇宫,只见飞檐斗拱攒成都市的楼台亭阁,仿佛是空中一片海市蜃楼,这里最矮的阁楼也有十几丈吧? 

如果找到能爬上这样阁楼的人,那么追回手串还有几分希望,试想一个贼有这样的身手,怎么会在大街上被公差捉到?像他的同事那种捉贼的办法,只会把大伙的??和老婆一起送掉。王安想到这些,对同事们的捉贼能力完全丧失了信心,他叹一口气,加家去了。 

王安走回鬼方坊,站在坊墙下看那些壁上的小人,发现他们方头方脑,方口方目。庞大的方身躯下两条麻秆腿,不觉起了同情之心,像这样的人物要是活过来,双腿马上会折断。正在出神,有人在背后叫:“舅舅,你回来啦?” 

王安回过头去,看到那个穿绿衫的女孩站在槐树下,手捧着大沓的衣服。他想:如果这个女孩不捉槐蚕,那倒是蛮可爱的。于是他脸上露出笑意说:“甥女儿,碰上你真凑巧。” 

女孩在阳光下笑起来。“不是凑巧,是我在这儿等你,等了半天啦!” 

王安又板起脸来,他背起手,转身缓缓行去,那女孩在背后跟随。她问:“舅舅,你在看墙上的画,你猜画的是谁?” 

“不知道。” 

“是你呀!” 

王安早知道他可能是那些棺材板似的人物的模特儿,因为那些人的下巴上全长着乱草般的胡子。不过听她这么一说,他还是很气愤。人要长成墙上画的那样,还有什么脸活在人间?他快步走回家去,翻箱倒柜要找一件衣服,把身上这件汗透了的换下来,可是找不到。那女孩说:“舅舅,换我洗的衣服吧!” 

王安在一瞬间想拒绝,可是他改变了主意,脸上又显出笑容,接过衣服来说:“你出去,我换衣服。” 

“舅舅怕什么,我是小孩子。” 

王安不想强迫她出去,就在她面前脱去长衣,裸露出上身。他是毛发很重的人,很以被外人看到自己的胸毛为羞。可是女孩看到王安粗壮的臂膀,宽阔的前胸,觉得心花怒放。她说:“舅舅的胡子真好看。能让我摸一把吧?” 

王安说:“这不行,胡子是男人的威严,怎么随便摸得?” 

“什么威严?舅娘就常摸,我看见的!” 

王安的脸登时红到发紫;她老婆只在行房前抚弄他的胡子。这种事她都看见了,简直是猖狂到了极点。他怒吼一声:“你是怎么看见的?” 

“爬到树上看见的,你怎么瞪眼?我不和你说了!” 

那女孩的脸飞快地涨到通红,瞪圆了眼睛做出一个怒相。她的脾气来得的这么快,倒是王安始料不及的。于是他把自己的怒目金刚相收起来,做出一个笑脸,忽然他闻到一股好闻的青苔味儿,是从衣服里来的,那衣服也很柔软,很干净,于是他和颜悦色地说:“甥女儿,衣服冼得很干净。” 

那女孩气犹未消地说:“是吗?” 

“当然,衣服上还有好闻的青草味。你用草熏过吗?” 

那女孩已经高兴了:“熏什么?我在后边塘里洗的,洗出来就有这股味。” 

王安一听浑身发凉。他知道那水塘,长了一池绿藻,里面全是青蛙和水蛇,塘水和鼻涕一样又浓又绿。早知道她要到那里洗衣服,还不如不叫她洗。但是这种话不便说出口来。于是他到柜里取了铜钱,按一个子儿一件给了洗衣的费用,又加上五文,算做洗得干净的赏钱。然后他叫女孩回家去,他要午睡了。女孩临出门时说: 

“舅舅,我一定要摸摸你的胡子。摸不到不甘心!” 

王安想,这个小鬼头可能是真想这么做的。王安还有话问她,就叫她回来说:“摸摸可以,不准揪。” 

女孩把十指伸开,插到那丝一样的胡须中。她觉得如果一个女人能拥有(当然不是自己长)这么一部胡子时。简直是世界上最大的幸福,就在她沉溺在胡须中时,王安问她: 

“甥女儿,墙上那些小人儿,是谁画的,你知道吗?” 

“是我。” 

王安已经猜到是她,不过他还是佯装不信。女孩说:“这有什么可不信的。我画给舅舅看!” 

她到厨下取了一块木炭,就爬到墙上做作画。她在墙上像壁虎上了纱窗,上下左右移动十分自如。王安想,长安城里那些大盗看到这孩子爬墙的本事,一定会在羞愧中死去。转瞬之间画完一幅画。她从墙上下来,拍拍手上的黑灰说: 

“舅舅,我画得怎么样?” 

王安说;“画得很好。”他点点头,正要走开,忽然看到那女孩对着下沉的夕阳站着,眯缝着眼睛,笑嘻嘻地毫不防备。他便猛然变了主意,像饿虎一样朝她扑去,去势之快捷,连苍鹰捕食都不能与之相比。殊不知那女孩朝地上一扑,比兔子还快地从他胯下爬过,等到王安转过身来,那女孩已经逃到十丈以外,拍着手笑道;“舅舅和我捉迷藏!你捉不到我,明天我再来,今天可要回家了!” 

第二天早上,王安到衙门里去点卯,发现签事房里一片欢腾,那佛手串的案子已经结束。原来圣明仁慈的皇后宣布说是她走进皇上的密室,取去了那串骨珠。公差们兴高采烈地到禁军衙门去接老婆,兵大爷们说,他们未奉旨不便放人。可是,他们也说相信圣旨不时将下,公差们就可以与妻子团聚了。王安对此也深信不疑。他回家里来,洒扫庭院,收拾家具,正忙得不可开交。那个女孩忽然来了,她站在门口,挑起眉毛说: 

“舅舅你在忙什么?难道舅娘要出来了吗?”王安说:“大概是吧。皇后承认是她偷去了珠子,这个案子该结了。” 

女孩说:“我看未必。皇后怎么会偷皇上的珠子?难道她也是贼?” 

王安笑了:“甥女儿,皇后说是她拿了珠子,想来自有她的道理,这种事情我们不便猜测。我想她老人家身为国母,一串骨珠也还担待得下,我对这案子不便关心,倒是你这爬墙的本领叫人佩服,是谁教给你的?” 

“没人教,我天生骨头轻,从小会爬墙。” 

“不管有人教也罢,没人教也罢,反正不是好本领。你把它忘了吧。等你舅娘回来,你和她学学针线。” 

女孩一听立刻火冒八丈,龀牙咧嘴,状如野猫。她恶狠狠地说:“针线我会,不用跟她学。舅舅你不要得意,也许空欢喜一场!” 

王安摇摇头,不再答理她,那女孩说:“舅舅,你还捉不捉我了?” 

王安想起昨天的事,羞得满脸通红。王安到长安之前,在河间府做过九年公差,当时是公差的骄傲,贼子的克星,出手速度之快,足能捉下眼前飞过的小鸟,但是却捉不一以一个小女孩。他摇着头说: 

“甥女儿,你把这事也忘了吧,昨天是我一时糊涂。“ 

“舅舅一点也不糊涂,我就坐在这儿,你再来捉捉看?“ 

王安知道,她就如天上的云,地上的风,谁也捉不到。昨天他被她表面的松懈迷惑,结果大出洋相。今天他不上这个当。他摇摇头说: 

“我何必要捉你?事情已经过去了。” 

那个女孩就走出去。王当躺在竹床上,想到几天之内就可以和老婆相会了。他极力在想像中复原她的倩影,但是这件事很困难。他也为那女孩所惑。当然,不是惑于她的美色。虽然她很美丽,但是尚未长成。王安的妻子在夜里比她要美得多。王当只是沉迷于她的快捷,她玲珑的骨骼,她喜怒无常的性格,这些气质比女色更迷人。 

王安影影绰绰地想起妻子在月夜里坐在竹床上的形象,她高大而丰满,裸露出胸膛,就如一座活玉雕。她在白天的凶暴,似乎全是为了掩饰在夜里的美,这好像是一个梦。可是那女孩在墙上游动的身影就在眼前,她的身子好像没有重量。像这样的人,除非她乐意让你捉住,否则你是无法捉到的。而让她把自己交到别人手里,是一件极费心力的事。谢天谢地,王安不必再为此费心了。就在王安感到轻快的时候,皇上觉得头痛欲裂,周身都是麻烦。皇后说她已经把手串毁了。皇帝只得从密室里走出来,尝试过以前的生活。但是他觉得外面光线晃眼,噪声吵人,山珍海味都不适口,锦墩龙椅都不舒适,宫里的女人浮嚣可憎,因此他又回密室去,召皇后来见面。 

浑身异香的皇后到皇帝面前时,面上浮起了红晕,皇帝觉得她分外光艳照人,所以要说的话也分外难说了出口。他踌躇良久最后痛苦地说:“梓童,朕知道你谏止朕迷恋珠串的苦心,朕也试图照你的意思去办。事实上,朕虽拥有六宫佳丽,除了你之外,却没有一个可以信赖的女人。由于你有天生的异香,由于你对朕的厚爱,朕早已决定终生绝不违拗你的意思。但是这手串实在是朕的生命,朕一定要把它追回。朕的苦恼,希望你能够理解。” 

皇后跪在他面前连称万岁,口称臣妾罪该万死,可是皇帝却出起神来。他看着皇后花一样的面孔,想起自己幼年丧母,从未感到母亲的爱。因此当他爱上皇后之后,就有轻微的犯罪感,每次和皇后做爱时,他感到她肉体的颤栗,就有一种儿奸母的感觉。如果不是因为这个,他绝不会割舍皇后,自己深入密室苦修。于是他苦笑一声,叫皇后平身。又赐她与自己同座。皇帝握着皇后的手说: 

“梓童,朕已有了追回手串的办法,但是却难免要冒犯于你。自从你我结缘以来,你已为我忍受了不少痛苦。为了追回手串,朕又要你为我忍受新的痛苦。因此朕要请求你的原谅。” 

皇后又到皇上面前跪下,口称她能够身为当今国母,全赖皇上的厚恩,她愿为皇上做一切事,惟一不能做的就是追回手串,因为它已经被毁掉了。皇上对这种说法感到厌倦。挥手叫皇后离去。然后在蒲团上静坐了很久,终于下定了决心。他想:皇后已经为他忍受过不少痛苦,再让她忍受点也无妨,这就如顽童烦扰母亲时那种模糊的心境,既然她能受得了生他的痛苦,还有什么受不了的。 

王安再到衙门里去点卯时,发现同事们在签事房里饮酒赌博,到处是放纵松懈的情绪。他还来不及打听出了什么事,就被叫到公堂上去,被按在堂上打了三十大板,做公差的总难免挨打。可是这一回打得非常之轻,那力量连蚊子都拍不死。挨过打之后,王安跪起来,要听听自己挨打的原因。可是官老爷什么也没说,摇头叹气地退堂了,他问打人的公差,今天这三十大板是怎么回事,可是那些人也只顾摇头叹气地离去。于是王安就回签事房去,问出了什么事情。别人说,皇帝早上下了圣旨,要全城的公差继续追查手串的案子,并且是严加追查,一天不破案,全体公差都要挨三十大板。 

公差们说,手串已经被皇后毁去,还要追查,这岂不是向公狗要鹿茸,向母鸡挤奶的事?他们还说,皇上天恩,只赐每天三十大板,就算把大伙全阉割了,把家眷变卖为奴,也是无可奈何的事。王安却没有那么达观,他赶紧回去找那个女孩,找遍了鬼方坊,再也找不到,他就回家来,坐在床上痛悔自己的愚蠢;第一不该冒失地出手抓那孩子,第二不该相信这个案子已经结束,第三不该对那女孩说,要她向老婆学针线。此时她肯定已经远走高飞,他想到自己能够和她住一个坊里,这是何等的侥幸。她又自己找上门来,这是何等的机遇。上天赐给王安这么多机会,他居然让她平安地溜走。简直是活该失去胡子和老婆。 

现在王安只好把希望寄托在皇后身上。他回签事房去,听说皇上已经下旨把她废为庶人。还要京兆衙门把公案和刑具搬进宫去。今天晚上他要亲审废后,要全城的公差都进宫去站堂。王安听到这个消息,吓得面孔铁青,坐在长凳上,好像一段呆木头。 

皇后被贬为庶人之后,就从宫殿里搬进了黑牢。在那儿她被席子上的霉味熏得半死,还被人剥去长袍,除去钗环,换上了罪衣罪裙。这种粗布衣服她从来没穿过,她觉得浑身如虫叮鼠咬。天黑之前,晚霞从窗口映入,照到皇后身上,她觉得周身血迹斑斑,想来到即将到来的羞辱和酷刑,她几次几乎晕死过去。最后有人打开牢门,用锁链锁住她的手足,牵着她去见皇帝。皇后赤足踉跄,走过宫里的石板地,心想:生为绝代佳人,实在是件残酷的事情。 

对于皇后来说,就连更衣这样的小事都是巨大的痛苦。从窗缝里吹进来的风也能使她感到利刃割面的痛苦。出浴时的毛巾不管多么柔软,她都觉得如板锉毛刷。所以活在世上就如忍受一场酷刑。尽管如此,做绝代佳人也比不做好。这就如君王的雨露之恩,来时令人不堪忍受,但是如果不来,更叫人无法活下去。因此皇后决定领受皇帝赐给的刑罚,宁可在刑具下死去,也不改变上谏皇帝的初衷。 

皇后来到皇帝前跪拜时,披散着万缕青丝,脖子上套着铁链。她穿着死囚临刑时穿的褐色衣裙,赤手赤足,用气息奄奄的声音喊道:“犯女XX,愿皇上万岁、万岁、万万岁!”皇上听了有一种奇异的感觉。他叫皇后抬起头来,发现一天不见,皇后已经清简了很多,他以聊天的口吻说: 

“梓童,你披枷戴锁,身着死囚的服装,朕觉得更增妩媚。” 

皇后说,她已经贬为庶人,现在是皇上的阶下囚,请皇上不要以梓童相联系称呼。皇上却说,他觉得阶下囚比皇后更加可爱。皇后就说,只要皇上喜欢,她也乐意做阶下囚。皇帝就挽了她的手到窗口去,让她看庭院中熊熊的烈火,如狼似虎的公差,血迹斑斑的刑具。皇后看了这些东西,只觉得天旋地转,立刻倒在皇上的怀里。 

皇后醒来之后,皇帝对她说:“梓童,现在改变你的决心还不算晚。否则朕只有为就要发生的事情请求你的原谅。” 

皇后明白,无论什么都不可能阻止皇帝追回他的手串,但是她还是说,她的身体归圣上所有,无论置于龙床上还是刑具下,都是正确的用途。 

于是皇帝叫人把她牵出去,几千名公差齐声高叫升堂,几乎把皇后娇嫩的耳膜震破。她被带过公差们站成的人甬道(几乎被男人身上的汗臭熏死),来到公案前跪下,在皇帝面前复述她的供词。皇帝立即命令对废后用刑,拶子刚套上她的十指尚未收紧,皇后的指尖就渗出血来。她像被门夹住尾巴的猫一样惨叫一声,晕死过去。 皇帝命令,用香火把皇后熏醒,再开始刑讯。拶子又收紧了一点儿,皇后在痛苦之中挣扎,却不能晕死过去。她身上的异香随着汗水蒸发,使行刑的公差腿软腰麻。这时皇帝逼问她的供词,皇后仍然不肯更改。皇帝就命令松去拶子,用藤条抽打她的手心,用金针刺入她的足趾。皇后晕厥了几次,终而不肯改口,最后皇帝命令松去皇后的刑具,她立刻瘫软在地昏死过去。 

皇帝命令把皇后送回寝宫,请太医诊治。然后板起脸来,公差扔下手中的水火棍,跪在御前磕头,那情景就如几千人在打夯。皇帝提高嗓子说: 

“朕已知道,你们这些乌鸦,不肯为朕尽心办案,却污蔑说皇后偷走了朕的手串。朕本该把你们全体凌迟处死,奈何还要依仗你们追回失物,只得放你们一条生路。朕这宫中没有石碾石磨,任凭什么人,都不能毁掉手串。而要说那手串为皇后藏匿起来呢,你们的狗头上也长有狗眼,应该看到皇后受刑时的情景。在这种情形之下,她如果能交出手串。绝无不交的可能。故而你们这批狗头,应该死心塌地地到宫外寻找,不要抱有幻想,朕的话你们可明白?” 

公差们抬起头来,齐声应道:“明白!”皇帝脸上露出了笑意说: 

“还有一件事情,朕说与你们知道。朕已下旨到关中各郡招集民间阉猪的好手,七天之内,你们如不能把手串交回御前,朕就要把你们阉掉半边。再过七天还不能破案,就把你们完全阉掉。现在你们马上出去为朕追赶寻失物。滚吧!” 

公差们从宫里出去。顾不上包扎额上的伤口,就到大街上去胡乱捕人。王安不参加捕人的行动。他回去家。出乎他的意料,他家里点着灯,那女孩坐在灯下,见到他进来,她站起来迎接说: 

“舅舅回来了!你的头上怎么破了?” 

听了这句话,王安勃然大怒,这简直是在揭他的短。他尽力装作不动声色,可是还免不了嘴角发抖。那女孩拍手笑道:“舅舅生气了!你来捉住我好了,只要捉住我就可以出尽你的恶习气了!” 

王安更加愤怒,非常想朝她猛扑过去,可是他知道捉不到她,他强笑着到席上去盘腿坐下,要那女孩拿来短几,把灯台放在几上。然后他叫她在对面坐下,和她对坐了许久。 

那女孩的手放在案上,手背和十指瘦骨嶙峋,叫人想起北方冰封悬崖上黑岩石中一缕金子的矿脉。她手肘上洁白的皮肤下暗蓝色的血管,就像雪原上河流,又如初雪后沼泽上众多的小溪。 

王安把双手也放到案上去,把她的双手夹在自己的手中间。 

王安感到她的双手的诱惑,如多年前他老婆的脖子的诱惑一样。王安的老婆在婚前也是个贼,虽无飞檐走壁的奇能。却擅长穿门过户。这原不是王安的案子,可是他为她雪白修长的秀颈所迷惑,一心要把链子套到她的脖子上去。王安这一生绝不贪恋女色,却要为女贼所迷。因此他看到墙上的壁画就会怦然心动,看到女孩在树下捡槐蚕就心悸不安,现地看到灯下案上一双姣好的双腕,手就禁不住轻柔地向上移去。 

十年前,王安看到那修长的脖子,天鹅似的仪容,禁不住起了男人的欲望,因此他就判定这个女人是个贼。看见她从前门走进巨富人家,他就到后门去等。现在他坐在女孩对面,手指轻轻触及她的肌肤,心中的狂荡比十年前有过之而无不及。女孩的腕上传过回夺的悸动,可是她立刻又忍住了,把手腕放在一点点收紧的把握中。王安始终不相信她会被抓住,直到他的手已经握实之后。他猛然用上了十成握力。那女孩“哇”地一声叫出来,猛地挣了起来,却丝毫也挣不动。然后她兴奋地面红耳赤,大叫道:“舅舅,你捉住我了!” 

王安猛想到捉住她也没什么用。他没有一丝证据,不能把她送到衙门里严刑拷打。他觉得受到了她的戏弄,就把手松开了,女孩把手捧到灯下去看,发现腕上印下了深深的青痕,不禁心花怒放,把双腕并着又伸了出来说: 

“舅舅你把木杻(音丑)套在这青痕上,再用链子锁住我的脖子,拉我到衙门去吧!我乐意!” 

王安虽然确信这女孩是贼却不能送她坐牢。他茫然地坐着,一会想说,你把这事忘记忘了吧。一会又想说,你回家去。最后他说: 

“甥女儿,我捉了你又放了,你满意了吧?现在告诉舅舅,皇上的手串你拿了没有?” 

女孩说:“舅舅的话我不大明白,什么满意不满意的,难道你当年也这么捉过舅娘?” 

王安当年站在那家巨富后门的僻巷里,他老婆出来时,他把链子锁在她脖子上。他本该把她拉到衙门去,但是他没有,他把她拖到没有人的地方,动手掏她怀里的赃物,结果看到她乳房上的痣,就再也把持不住,冒犯了她的身体。等到发现她的处女的血染上他的身,王安就不便送她去坐牢,而是娶了她当老婆。如今这女孩问起,他就简略地说过此事,然后说:“甥女,舅舅是怎么一个人,你已经明白了。我现在求你,帮我找回皇上的手串,要不皇上要阉了我们。阉是怎么回事,你知道吗?” 

那女孩面露不悦之色说,她知道什么叫阉,却不懂王安为什么为难。他如果怕阉,可以逃走,至于手串,她可帮不了忙。王安就说: 

“甥女儿,别拿舅舅开心。凭我对你的感觉,你就算不是偷手串的贼,也是大有来历。你一定能帮舅舅寻回手串。至于要我逃脱,是你小孩子不懂事。我怎能扔下舅娘不管?” 

女孩怒起来,跪在席子上说:“舅舅说我是贼为什么不搜我的怀?” 

“那怎么成?搜你舅娘已经很不对了。” 

女孩大发雷霆,尖叫道:“有什么对不对的!既然都是贼,捉住了有的搜,有的不搜,真是岂有此理!”说着她一把把胸襟扯开。王安看到她的胸上也有七点红痣,和他老婆的毫无二致。他因此大吃一惊,两眼发直,然后他才看到她怀里藏了一串珠子。肯定是皇上遗失的,他连忙去抓她的足踝已经迟了,堂屋里就如起了一阵风,女孩一晃就不见了。 

女孩走后,王安想了很久,他忽然彻底揭穿了这个谜。有两点是他以前没有想到的,第一是那女孩和王安的老婆很熟,王安可以想像他老婆在荒坊里很寂寞,如果有一个女孩来做伴她就会把什么都说出来。还有第二点,就是这女孩一直在偷东西。按照规律,地方上出了大案公差领命破案时,总要收家属为质。她想用这种方法把王安的老婆撵走,所以这两年长安城里的大窃案层出不穷。不过王安在衙门里不属于机智干练那一类,所以总也捉不到他老婆头上来。直到她偷到皇帝头上,方才得逞。想明了这两点,王安觉得这案子他已经谙然于胸。他对追回手串又有了信心。他在灯里注入新油,在灯下正襟危坐。他知道那女孩一定会回来的。 

她果然回来了,坐在王安面前吐舌头做鬼脸。王安视若不见,板着脸说: 

“甥女儿,你别挤眉弄眼,这不好看。我问你,你胸上的红点是天生的吗?” 

女孩一听,小脸登时发青。王安又说:“你舅娘对你多好,连奶都给你看,可是你却累得她坐牢,你不觉得可耻吗?“ 

女孩的脸又恢复了原状,她说:“有什么可耻的?我早就想送她进牢房。我听舅娘说,上次舅舅勒死一个贼就在佛前忏悔,发誓道今生再不捉贼,伸左手砍左手,伸右手砍右手。可是你却一连捉了我三次,怎么也不知道羞耻?还不把手砍下来!” 

王安脸红了一下说:“这也没什么可耻的,大人者,言不必信,行不必果,手也不一定要砍。”然后他觉得这样不足以启迪女孩的羞耻心,就说: 

“甥女儿,你胡闹得够了,又偷东西,又点假痣,还把赃物揣在怀里,这全是学你舅娘的旧样。这种小孩子的把戏,你还要耍多久?” 

“舅舅既然说我是小孩子,那我就把这戏耍到底。” 

王安为之语塞。那女孩又说:“其实我并不是小孩子,舅舅伸手捉了我,我就是不折不扣的女贼,你该用对待女贼的态度对我。” 

王安苦笑着说:“舅娘是个苦命人。十年前舅舅无礼强暴了她,到今天她对我还是又抓又咬。这是舅舅的孽债,不知什么时候才能还清。甥女儿,我们不能让舅娘再受苦,否则舅舅的孽债就更深重了!” 

“呸!她算什么苦命人?你这话只好去骗鬼!” 

女孩说,王安的老婆是什么样的人,她比王安还清楚。白天来看时,王安的老婆蓬头垢面音嗓粗哑,显得丑陋不堪。她用男低音说话。说到王安,她说他是一群猪崽子中最下贱的一只。十年前他用铁链子勒着脖子把她强奸了,她说王安的身体毛茸茸的,好像只大猴子。在夜里,因为夫妻的名分和女性的弱点,让他占有了她的肉体。白天想起来,就如喉咙里含了活泥鳅一样恶心,她真恨不得把王安吃掉,以解心头沉郁十年的怒气。然后她给女孩看她指甲上的血迹,说她刚把王安抓得落荒而逃。这时她哈哈大笑,就如坟地上的猫头鹰,她还直言不讳地承认自己是母夜叉,被王安强奸之后,除嫁他别无选择,就如被装进笼子的疯狗,她只有啃铁条消磨时光。 

晚上远看王安的老婆,就发现一切都很不同。她在镜前梳妆着衣,等待王安回来。那时她肩上披着的长发没有一丝散乱,身上穿着锦丝的长袍,用香草熏过,没有一个污点,一个皱褶。她脸上挂着恬静的微笑,用柔和的女中音说话。说王安是公差中的佼佼者,她曾是贼中的佼佼者。最出色的贼一定会爱最出色的公差,就如美丽的死囚会爱英俊的刽子手。那时候她显得又温柔又幸福,又成熟又完美,高大而且丰满。女孩痛恨她佛一样的丰肩,天女一般的宽臀,看到她像大理石雕成的手和修长的双腿,女孩真恨不得死了才好。 

她说到王安对她的冒犯,有和白天很不同的说法,她说当锁链忽然套到她颈上时,在最初的惊慌之后,她又感到一丝甜蜜,这种甜蜜混在铁链的残酷之中。王安锁住她以后,犹豫了很久,这使她想到自己有多么美,然后他牵着她到嫩黄的柳林里去,她隐隐知道要出什么事。那时她跟着铁链走去,脚步蹒跚,有时想喊,可始终没有喊出来。 

强暴来临时,她拼命抗拒过,然后又像水一样顺从。她不记得失去贞操的痛苦,却记得初春上午林梢的迷雾,柳条低垂下来,她的衣服被雪泥弄得一塌糊涂,只好穿上王安的外衣,踏着林荫处半融的残雪回家去,做他的妻子。 

王安的妻子梳妆已毕,敞开胸襟,给女孩看她胸上的痣。她说月夜里,王安把嘴唇深深印在这些痣上。女孩妒火中烧,恨不得把那洁白的乳房和鲜红的痣都用烧红的烙铁毁掉。她束紧腰带,又用布带在臀下系紧,布料下显出她的曲线。她说到王安会用温柔的手把这些结解开,禁不住心花怒放。 

她还说王安的身体,宽阔胸膛,浓重的体毛和铁一样的肌肉,王安就如航行于江海上的航船,有宽阔的船头,厚重的船尾。在两情相悦的时候,她用身体载起这只巨舟,她是水,乳白色的,月光一样的水。所有的女人都是水,但是以前她并不知道。她是独脚贼,没有人告诉她,直到王安这条船升起风帆驶入她的水域。说到这里时,她身上浮起思念丈夫的肉香。女孩闻听这种味儿,恨不得把这娇滴滴、香喷喷的骚娘们儿一刀捅死,以泄心头之恨。 

女孩说,她不相信男女之间只有干那种丑事才能相爱,尤其是像王安这种伟大的男人。试过王安以后,她更加相信,他是被那娘们儿的骚性诱惑了,说完这些话,她就从屋里出去,并没有说怎样她才能把手串交还。 

又过了三天,皇帝对公差寻回手串的能力失去了信心,他下诏说赦免窃珠贼一切罪责。如果贼肯把手串交还,他还要以爵位和国库中的珍宝相赠,他还答应给那人以宫中的美女或禁卫军中的美丈夫。这通诏书一下,长安朝野震动,以为皇帝是疯了。 

只有王安认为皇帝真正圣明。王安相信,任何丢失的东西都可以寻回,捉不到贼,就要用贼想要,或更想要的东西交换。他虽然对这一点深信不疑,可还是想不出怎么才能使那女孩把手串交回来。中午时他坐在家里凝神苦思,下意识地用指头去挖席子,不知不觉把席子抠出一个大洞。 

那时屋外天气很热,阳光把蝉都晒晕了,以致鬼方坊里万籁无声。可是王安屋里是一片凉爽的绿荫,空气里弥漫着夹竹桃的苦味,草叶的芳香,还有干槐花最后的甜香味。他家里摆满了瓶瓶罐罐,里面插着各种各样的绿枝。一旦露出干枯的迹象,女孩就把旧枝条拿出去用新的枝条来代替。现在屋里的树枝、灌木和草叶全是一片新绿。她心满意足,就伏在窗前的席子上睡着了。 

女孩睡着时,没有一丝声息。只有肩头在微微起伏。她睡觉的姿势也很奇特。这说明她所说的并非虚妄。她说她没有家,也不记得有过家。王安没法相信人没有家怎么能长大,但是如果她有过家,就不会以这种姿势睡觉,因为没有人用这种姿势在家里睡觉。 

这女孩搬到王安家里已经两天了。王安以为住在一个屋檐下两天两夜已经足够了解一个女人。可是除了她说过的那些话,王安对她还是一无所知。她对王安说,除了王安的老婆她和谁都不熟识。也许王安的老婆能说出,怎样才能使女孩交去手串。可是她却被关在禁卫军把守的天牢里,不容探视。王安没法向别人打听这女孩的心性,他只好自己来解这个谜。 

他想到昨天晚上,他在她面前更衣,那女孩走过来,用指尖轻轻触及他的肉体。她不像王安老婆那样把手掌和身体附着到他身上。只消看一看,闻一闻,轻轻一触就够了。她在王安面前更衣,毫无扭捏之态,在青色的灯光下王安看到除了两个微微隆起的乳房,她身上再没有什么阻止她跑得快,就如西域进贡给皇帝的猎豹。她骨骼纤细,四肢纤长,好像可以和羚羊赛跑。 

女孩说,她爱王安,如果得不到王安的爱,她一辈子也不会把手串交出来,哪怕王安的老婆死在狱中,哪怕王安因此被处宫刑,也得不到她的同情。王安也准备爱她,可是不知怎么爱才好。如果她再大几岁,或者在市井里住过几年,那么一切都简单了,现在要他去爱简直是岂有此理。 

女孩说,以前她住在终南山中,一年也见不到几个人,在山林里她感到需要爱,才搬到长安城里来。这个哑谜叫王安无从捉摸起,人住在深山无人的地方,也会知道爱吗?她在深山中体会到的爱,也不知有多么怪诞。王安想不出头绪,就把她叫起来问。 

“甥女儿,你在深山里见过飞鸟交尾,或者两条青蛇缠在一起?你听见深秋漫山的金铃子叫,心中可有所感?你也许见过一只雄猫寻母猫的气味而去,或者公山羊们在绝壁上抵角?” 

女孩听了勃然大怒,说:“舅舅,你真讨厌死了,你简直像舅娘一样骚,如果你再这么胡说,我就跑到深山里去,等你被阉了再回来!” 

王安只好让她继续睡觉,他知道她不是个思春昏了头的傻丫头。在胸上点痣,引诱王安去捉,那不过是孩子的恶作剧,她并不喜欢这些。 

王安想来想去,觉得脑筋麻木,他闻到屋里森林般的气味,就动了出去走走的念头。于是他走到坊间的绿荫中去,觉得天气很热。等头顶槐花落尽,真正的酷暑就会来临。 

星星点点的阳光从树叶间漏下来,照在王安身上,光怪陆离,他渐渐忘去心中的烦恼。走进一片浓绿之中,听见极远处一辆牛车在吱吱地响。坊间的道路不只一条,它们弯弯曲曲地在槐林中汇合又分散。王安遇到一只迷路的小蝴蝶,它在荆棘之中奋力扑动翅膀要飞出去。他想到皇帝也是这么奋力地要寻回手串,寻求一条通向月夜下横陈的玉体之路。这些路曲曲弯弯,居然在这里汇合,其中的机缘真不可解。 

王安在心中拿蝴蝶打个赌赛:如果它飞出草丛,那么皇上的手串也能寻回来。所以当蝴蝶的白翅膀在刀丛剑树中挂得粉碎,它那小小的身子和伤残的翅膀一起坠落时,他几乎伤心地叫进来。就在这时那个女孩来到他身边,拉着他的手说: 

“舅舅,出来散步也不叫上我!一起走走吧。” 

王安把蝴蝶的悲哀忘掉,和她一起到更深的绿荫中去。他把她的小手握在手中,感受到一股冷意从手中透入。就想起初见她时,这个女孩在槐树下捡槐蚕的情景。女孩把绿色的活槐蚕揣在怀里,那种冰凉蠕动的感觉是多么奇妙啊!她身上有一种青苔的气味,王安想到女孩在一池绿水中洗衣服,洗出的衣服又柔软又舒适。他们在绿荫中走了很久。王安很放松,很愉快,他感觉她贴体的触觉、嗅觉和遥远的听觉、视觉逐渐分开。她在很近的地方,女孩在很远的地方。当冰凉蠕动的感觉深入内心的时候,王安知道自己在爱了。 

他们回家以后,王安脱去冷湿的衣服。女孩伸出舌尖,尝一尝他胸前的汗味儿。她叫王安是“舅舅情人”。后来这位“舅舅情人”和她在椭圆形的大浴桶里对坐,桶里盛着清凉的水。 

王安看到女孩在一片绿荫之中。他终于伸出一根粗大的手指,按在她胸骨上,不带一点肉欲地说,他爱她,他对她充满了绿色的爱。女孩听见这句话,就从浴桶里跳出去走了,再也没有回来。 

第二天早上,天还没有亮,那串骨珠从密室的天窗飞进来,摔在皇帝的脑袋上。皇帝得回了手串很高兴,就不计较这种交回手串的方式是多么不礼貌。他命令禁卫军把公差的家眷放了,还给每人五两银子压惊钱。王安的老婆回家时天色还没大亮,王安怕她会和他大闹一场,谁知她没有。洗去坐牢时积下的泥垢和汗臭,穿上长裙,她和他做房中的游戏。休息时她说,抓人和撒泼都是坏毛病,她已经决心改了。在黑牢她还看透了一点,就是白天也可以当成黑夜来过。对于她这种达观的态度,王安当然表示欢迎。 

王安的老婆说,她根本不相信能活着回到王安身边,因为她知道这件事是小青(就是那女孩的名字)干的。她知道那女孩会飞檐走壁,偶尔也偷东西。所以当禁卫军把她抓走时,她把王安和小青的祖宗八代都骂遍了。不过骂人不能解决问题。她坐在牢里腐烂潮湿的稻草上,深悔以前没在王安耳边提到她有一位野猫似的小女友,于是她又想通吃醋也是个坏毛病,她也决心要改。 

这些都不足以难倒王安,她深知自己的丈夫是全世界日子机警的公差,尤其是对付女贼时。即便他找不到那女子,她也会自己找上门去。真正困难的是叫她承认自己是贼,而且要她交出赃物。她无法想像王安怎么看透谜底。案发前,有一天傍晚,她和小青在房里聊天时,她说完和自己是水,王安是舟的比喻,就说这是爱的真谛。 

那女孩说,她体会到的爱和她很不同。从前她在终南山下,有一回到山里去,时值仲夏,闷热而无雨,她走到一个山谷里,头上的树叶就如阴天一样严丝合缝,身边是高与人齐的绿草,树干和岩石上长满青苔。在一片绿荫中她走过一个水塘,浅绿色的浮萍遮满了水面,几乎看不到黑色的水面。 

女孩说,山谷里的空气也绝不流动,好像绿色的油,令人窒息,在一片浓绿之中,她看到一点白色,那是一具雪白的骸骨端坐在深草之中。那时她大受震撼,在一片寂静中抚摸自己的肢体,只觉得滑润而冰凉,于是她体会到最纯粹的恐怖,就如王安的老婆被铁链锁住脖子时。然后她又感到爱从恐惧中生化出来,就如绿草中的骸骨一样雪白,像秋后的白桦树干,又滑又凉。 

王安的老婆对她的体会绝不赞同,她在遇到王安之前,脖子上从未挂过锁链,所以当王安锁住她时,她觉得自己已经被占有,那种屈辱与顺从的感觉,怎能用深草中的骸骨比拟,就笑那女孩说:“你去试试,看世上能不能找到一位情郎,给你这种绿色的爱!” 

于是产生了一场口角,那女孩在盛怒中顿足而去。 王安的老婆深知小青一定要王安身上打主意,她却不知她还能把自己搞到牢里去。说完这些话,她就玩王安的胡须,说他是世界上最可爱的大丈夫,连皇帝也不能与之比拟。
