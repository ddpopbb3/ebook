\chapter{有关克隆人}

  "我说过,对克隆人这件事我不准备做任何评论。但我知道人们为这样一种念头苦恼着:假如未来的人是从试管和暖箱里造出来的,没有父母了,他怎么生活呢?假如未来的人们不用自己的身体造孩子了,他们的精神又往何处寄托呢?假如真有这样的事情发生,我可以告慰这些未来的人:孩子们,你们的事业――科学与艺术,也是人造出来的。这就是说,你们虽不是父母的儿女,但还是牛顿、爱因斯坦和莎士比亚的儿女。你们需把人类的事业发扬光大――去做这些事吧。这世界上还有更坏的情况,那就是你的生活与这些前辈的事业全不接轨,去感谢上帝吧,这样的事没有发生在你的身上。" 

节选自《有关克隆人》 1997年第7期《三联生活周刊》 
(待添加...)

\chapter{洋鬼子与辜鸿铭}

我看过一些荒唐的书,因为这些书,我丧失了天真。在英文里,丧失天真(LOSE INNOCENT)兼有变得奸猾的意思,我就是这么一种情形。我的天真丢在了匹兹堡大学的图书馆里。我在那里借了一本书,叫作“一个洋鬼子在中国的快乐经历”,里面写了一个美国人在中国的游历。从表面上看,该洋鬼子是华夏文化的狂热爱好者,清朝末年,他从上海一下船,看了中国人的模样,就喜欢得发狂。别人喜欢我们,这会使我感到高兴,但他却当别论,这家伙是个SADIST,还是个BISEXUAL。用中国话来说,是个双性恋的性虐待狂。被这种人喜欢上是没法高兴的,除非你正好是个受虐狂。 

我和大多数人一样,有着正常的性取向。咱们这些人见到满大街都是漂亮的异性,就会感到振奋。作为一个男人,我很希望到处都是美丽的姑娘,让我一饱眼福-女人的想法就不同,她希望到处都是漂亮小伙子。这些愿望都属正常。古书上说,海上有逐臭之夫。这位逐臭之夫喜欢闻狐臭。他希望每个人都长两个臭腋窝,而且都是熏死狐狸,骚死黄鼠狼的那一种,这种愿望很难叫作正常,除非你以为戴防毒面具是种正常的模样。而那个虐待狂洋鬼子,他的理想是到处都是受虐狂,这种理想肯定不能叫作正常。很不幸的是,在中国他实现了理想。他说他看到的中国男人都是那么唯唯诺诺,头顶剃得半秃不秃,还留了猪尾巴式的小辫子,这真真好看死了。女人则把脚缠得尖尖的,要别人搀着才能走路,走起来那种娇羞无力的苦样,他看了也要发狂…… 

从表面看来,此洋鬼对华夏文化的态度和已故的辜鸿铭老先生的论点很相似-辜老先生既赞成妇女缠足,也赞成男人留辫子。有人说,辜先生是文化怪杰,我同意这个“怪”字,但怪不一定是好意思。以寻常人的角度来看,SADIST就很怪。好在他们并不侵犯别人,只是偷偷寻找性伴侣。有时还真给他们找到了,因为另有一种MASOCHIST(受虐狂),和他们一拍即合。结成了对子,他们就找个僻静地方去玩他们的性游戏,这种地点叫作“密室”-主要是举行一些仪式,享受那种气氛,并不当真动手,这就是西方社会里的S/M故事。但也有些SADIST一时找不着伴儿,我说到的这个就是。他一路找到中国来了。据他说,有些西洋男人在密室里,给自己带上狗戴的项圈,远没有剃个阴阳头,留条猪尾巴好看。他还没见过哪个西洋女人肯于把脚裹成猪蹄子。他最喜欢看这些样子,觉得最为性感-所以他是性变态。至于辜鸿铭先生有什么毛病,我就说不清了。 

那个洋鬼子见到中国人给人磕头,心里兴奋得难以自制:真没法想象有这么性感的姿式——双膝下跪!以头抢地!!口中还说着一些驯服的话语!!!他以为受跪拜者的心里一定欲仙欲死。听说臣子见皇帝要行三磕九叩之礼,他马上做起了皇帝梦:每天作那么快乐的性游戏,死了都值!总而言之,当时中国的政治制度在他看来,都是妙不可言的性游戏和性仪式,只可惜他是个洋鬼子,只能看,不能玩…… 

在那本书里,还特别提到了中国的司法制度。老爷坐在堂上,端然不动,罪人跪在堂下,哀哀地哭述,这情景简直让他神魂漂荡。老爷扔下一根签,就有人把罪人按翻,扒出屁股来,挥板子就打。这个洋鬼子看了几次,感到心痒难熬,简直想扑上去把官老爷挤掉,自己坐那位子上。终于他花了几百两银子,买动了一个小衙门,坐了一回堂,让一个妓女扮作女犯打了一顿,他的变态性欲因此得到了满足,满意而去。在那本书里还有一张照片,是那鬼子扮成官老爷和衙役们的留影。这倒没什么说的,中国古代过堂的方式,确实是一种变态的仪式。不好的是真打屁股,不是假打,并不象他以为的那么好玩。所以,这种变态比S/M还糟。 

我知道有些读者会说,那洋鬼子自己不是个好东西,所以把我们的文化看歪了。这话安慰不了我,因为我已经丧失了天真。坦白地说罢,在洋鬼子的S/M密室里有什么,我们这里就有什么,这种一一对应的关系,恐怕不能说是偶合。在密室里,有些MASOCHIST把自己叫作奴才,把SADIST叫作主人。中国人有把自己叫贱人,奴婢的,有把对方叫老爷的,意思差不多。有些M在密室里说自己是条虫子,称对方是太阳-中国人不说虫子,但有说自己是砖头和螺丝钉的。这似乎说明,我们这里整个是一座密室。光形似说明不了什么,还要神似。辜鸿铭先生说:华夏文化的精神,在于一种良民宗教,在于每个妇人都无私绝对地忠诚其丈夫,忠诚的含义包括帮他纳妾;每个男人都无私地绝对地忠于其君主,国王或皇帝,无私的含义包括奉献出自己的屁股。每个M在密室里大概也是这样忠于自己的S,这是一种无限雌服,无限谄媚的精神。清王朝垮台后,不准纳妾也不准打屁股,但这种精神还在,终于在“文革”里达到了顶峰。在五四时期,辜先生被人叫作老怪物,现在却被捧为学贯中西的文化怪杰,重印他的书。我不知道这是为什么——也许,是为了让虐待狂的洋鬼子再来喜欢我们?

\chapter{救世情结与白日梦}

现在有一种“中华文明将拯救世界”的说法正在一些文化人中悄然兴起,这使我想起了我们年轻时的豪言壮语:我们要解放天下三分之二的受苦人,进而解放全人类。对于多数人来说,不过是说说而已,我倒有过实践这种豪言壮语的机会。七零年,我在云南插队,离边境只有一步之遥,对面就是缅甸,只消步行半天,就可以过去参加缅共游击队。有不少同学已经过去了——我有个同班的女同学就过去了,这对我是个很大的刺激——我也考虑自己要不要过去。过去以后可以解放缅甸的受苦人,然后再去解放三分之二的其他部分;但我又觉得这件事有点不对头。有一夜,我抽了半条春城牌香烟,来考虑要不要过去,最后得出的结论是:不能去。理由是:我不认识这些受苦人,不知道他们在受何种苦,所以就不知道他们是否需要我的解救。尤其重要的是:人家并没有要求我去解放,这样贸然过去,未免自作多情。这样一来,我的理智就战胜了我的感情,没干这件傻事。 

对我年轻时的品行,我的小学老师有句评价:蔫坏。这个坏字我是不承认的,但是“蔫”却是无可否认。我在课堂上从来一言不发,要是提问我,我就翻一阵白眼。像我这样的蔫人都有如此强烈的救世情结,别人就更不必说了。有一些同学到内蒙古去插队,一心要把阶级斗争盖子揭开,解放当地在“内人党”迫害下的人民,搞得老百姓鸡犬不宁。其结果正如我一位同学说的:我们“非常招人恨”。至于到缅甸打仗的女同学,她最不愿提起这件事,一说到缅甸,她就说:不说这个好吗?看来她在缅甸也没解放了谁。看来,不切实际的救世情结对别人毫无益处,但对自己还有点用——有消愁解闷之用。“文化革命”里流传着一首红卫兵诗歌《献给第三次世界大战的勇士》,写两个红卫兵为了解放全世界,打到了美国,“战友”为了掩护“我”,牺牲在“白宫华丽的台阶上”。这当然是瞎浪漫,不能当真:这样随便去攻打人家的总统官邸,势必要遭到美国人民的反对。由此可以得出这样的结论:解放的欲望可以分两种,一种是真解放,比如曼德拉、圣雄甘地、我国的革命先烈,他们是真正为了解放自己的人民而斗争。还有一种假解放,主要是想满足自己的情绪,硬要去解救一些人。这种解放我叫它瞎浪漫。 

对于瞎浪漫,我还能提供一个例子,是我十三岁时的事。当时我堕入了一阵哲学的思辨之中,开始考虑整个宇宙的前途,以及人生的意义,所以就变得本木痴痴;虽然功课还好,但这样子很不讨人喜欢。老师见我这样子,就批评我;见我又不像在听,就掐我几把。这位老师是女的,二十多岁,长得又漂亮,是我单恋的对象,但她又的确掐疼了我。这就使我陷入了爱恨交集之中,于是我就常做种古怪的白日梦,一会儿想象她掉进水里,被我救了出来;一会儿想象她掉到火里,又被我救了出来。我想这梦的前一半说明我恨她,后一半说明我爱她。我想老师还能原谅我的不敬:无论在哪个梦里,她都没被水呛了肺,也没被火烤糊,被我及时地抢救出来了——但我老师本人一定不乐意落入这些危险的境界。为了这种白日梦,我又被她多掐了很多下。我想这是应该的:瞎浪漫的解救,是一种意淫。学生对老师动这种念头,就该掐。针对个人的意淫虽然不雅,但像一回事。针对全世界的意淫,就不知让人说什么好了。 

中国的儒士从来就以解天下于倒悬为己任,也不知是真想解救还是瞎浪漫。五十多年前,梁任公说,整个世界都要靠中国文化的精神去拯救,现在又有人旧话重提。这话和红卫兵的想法其实很相通。只是红卫兵只想动武,所以浪漫起来就冲到白宫门前,读书人有文化,就想到将来全世界变得无序,要靠中华文化来重建全球新秩序。诚然,这世界是有某种可能变得无序——它还有可能被某个小行星撞了呢——然后要靠东方文化来拯救。哪一种可能都是存在的,但是你总想让别人倒霉干啥?无非是要满足你的救世情结嘛。假如天下真的在“倒悬”中,你去解救,是好样的;现在还是正着的,非要在想象中把人家倒挂起来,以便解救之,这就是意淫。我不尊重这种想法。我只尊敬像已故的陈景润前辈那样的人。陈前辈只以解开哥德巴赫猜想为己任,虽然没有最后解决这个问题,但好歹做成了一些事。我自己的理想也就是写些好的小说,这件事我一直在做。李敖先生骂国民党,说他们手淫台湾,意淫大陆,这话我想借用一下,不管这件事我做成做不成,总比终日手淫中华文化,意淫全世界好得多吧。

\chapter{警惕狭隘主义的蛊惑宣传}

罗素曾说,人活在世上,主要是在做两件事:一、改变物体的位置和形状,二、支 使别人这样干。这种概括的魅力在于简单,但未必全面。举例来说,一位象棋国手知道自己 的毕生事业只是改变棋子的位置,肯定会感到忧伤;而知识分子听人说自己干的事不过是用 墨水和油墨来污损纸张,那就不仅是沮丧,他还会对说这话的人表示反感。我靠写作为生, 对这种概括就不大满意:我的文章有人看了喜欢,有人 

看了愤怒,不能说是没有意义的……但话又说回来,喜欢也罢,愤怒也罢,终归是情绪 ,是虚无缥渺的东西。我还可以说,写作的人是文化的缔造者,文化的影响直至千秋万代— —可惜现在我说不出这种影响是怎样的。好在有种东西见效很快,它的力量又没有人敢于怀 疑:知识分子还可以做蛊惑宣传,这可是种厉害东西…… 

在第二次世界大战里,德国人干了很多坏事,弄得他们己都不好意思了。有个德国 将军蒂佩尔斯基这样为自己的民族辩解:德国人民是无罪的,他们受到希特勒、戈培尔之流 蛊惑宣传的左右,自己都不知道自己在干什么。还有人给希特勒所著《我的奋斗》作了一番 统计,发现其中每个字都害死了若干人。德国人在二战中的一切劣迹都要归罪于希特勒在坐 监狱时写的那本破书——我有点怀疑这样说是不是很客观,但我毫不怀疑这种说法里含有一 些合理的成份。总而言之,人做一件事有三种办法,就以希特勒想干的事为例,首先,他可 以自己动手去干,这样他就是个普通的纳粹士兵,为害十分有限;其次,他可以支使别人去 干,这样他只是个纳粹军官;最后,他可以做蛊惑宣传,把德国人弄得疯不疯、傻不傻的, 一齐去干坏事,这样他就是个纳粹思想家了。 

说来也怪,自苏格拉底以降,多少知识分子拿自己的正派学问教人,都没人听,偏 偏纳粹的异端邪说有人信,这真叫邪了门。罗素、波普这样的大学问家对纳粹意识形态的一 些成分发表过意见,精彩归精彩,还是说不清它力量何在。事有凑巧,我是在一种蛊惑宣传 里长大的(我指的是张春 

桥、姚文元的蛊惑宣传),对它有点感性知识,也许我的意见能补大学问家的不足…… 这样的感性知识,读者也是有的。我说得对不对,大家可以评判。 


据我所知,蛊惑宣传不是真话——否则它就不叫作蛊惑——但它也不是蓄意编造的 假话。编出来的东西是很容易识破的。这种宣传本身半疯不傻,作这种宣传的人则是一副借 酒撒疯、假痴不癫的样子。肖斯塔科维奇在回忆录里说,旧俄国有种疯僧,被狂热的信念左 有,信口雌黄,但是人见人怕,他说的话别人也不敢全然不信——就是这种人搞蛊惑宣传能 够成功。半疯不傻的话,只有从借酒撤疯的人嘴里说出来才有人信。假如我说“宁要社会主 义的草不要资本主义的苗”,不仅没人信,老农民还要揍我;非得像江青女士那样,用更年 期高亢的啸叫声说出来,或者像姚文元先生那样,带著怪诞的傻笑说出来,才会有人信。要 搞蛊惑宣传,必须有种什么东西盖著脸(对醉汉来说,这种东西是酒),所以我说这种人是 在借酒撤疯。顺便说一句,这种状态和青年知识分子意气风发的猖狂之态有点分不清楚。虽 然夫子曾曰“不得中行而与之必也狂猖乎”,但我总觉得那种状态不宜提倡。 


其次,蛊惑宣传必定可以给一些人带来快感,纳粹的干年帝国之说,肯定有些德国 人爱听;“文革”里跑步进入共产主义之说,又能迎合一部分急功近利的人。当然,这种快 感肯定是种虚妄的东西,没有任何现实的基础,这道理很简单,要想获得现实的快乐,总要 有物质基础,嘴说是说不出来的:哪怕你想找个干净厕所享受排泄的乐趣,还要付两毛钱呢 ,都找宣传家去要,他肯定拿不出。最简单的作法是煽动一种仇恨,鼓励大家去仇恨一些人 、残害一些人、比如宣扬狭隘的民族情绪,这可以迎合人们野蛮的劣根性。煽动仇恨、杀戮 ,乃至灭绝外民族,都不要花费什么。煽动家们只能用这种方法给大众提供现实的快乐,因 为这是唯一可行的方法——假如有无害的方法,想必他们也会用的。我们应该体谅蛊惑宣传 家,他们也是没办法。 


最后,蛊惑宣传虽是少数狂热分子的事业,但它能够得逞,却是因为正派人士的宽 容。群众被煽动起来之后,有一种惊人的力量。有些还有正常思维能力的人希望这种力量可 以做好事,就宽容它——纳粹在德国初起时,有不少德国人对它是抱有幻想的;但等到这种 非理性的狂潮成了气候,他们后悔也晚了。“文革”初起时,我在学校里,有不少老师还在 积极地帮著发动“文革”哩,等皮带敲到自己脑袋上时,他们连后悔都不敢了。根据我的生 活经验,在中国这个地方,有些人喜欢受益惑宣传时那种快感;有些人则崇拜蛊惑宣传的力 量;虽然吃够了蛊惑宣传的苦头,但对蛊惑宣传不生反感;不唯如此,有些人还像瘾君子盼 毒品一样,渴望著新的蛊惑宣传。目前,有些年轻人的抱负似乎就是要炮制一轮新的蛊惑宣 传——难道大家真的不明白蛊惑宣传是种祸国殃民的东西?在这种情况下,我的抱负只能是 反对蛊祸宣传。我别无选择。

\chapter{京片子与民族自信心}

我生在北京西郊大学区里。长大以后,到美国留学,想要恭维港台来的同学,就说:你国语讲得不坏!他们也很识趣,马上恭维回来:不能和你比呀。北京乃是文化古都,历朝历代人文荟萃,语音也是所有中国话里最高尚的一种,海外华人佩服之至。我曾在美国华文报纸上读到一篇华裔教授的大陆游记,说到他遭服务小姐数落的情形:只听得一串京片子,又急又快,字字清楚,就想起了《老残游记》里大明湖上黑妞说书,不禁目瞪口呆,连人家说什么都没有去想——我们北京人的语音就有如此的魅力。当然,教授愣完了,开始想那些话,就臊得老脸通红。过去,我们北京的某些小姐(尤其是售票员)在粗话的词汇量方面,确实不亚于门头沟的老矿工——这不要紧,语音还是我们高贵。 

但是,这已是昨日黄花。今天你打开收音机或者电视机,就会听到一串“嗯嗯啊啊”的港台腔调。港台人把国语讲成这样也会害臊,大陆的广播员却不知道害臊。有一句鬼话,叫作“那么呢”,那么来那么去,显得很低智,但人人都说。我不知这是从哪儿学来的,但觉得该算到港台的帐上。再发展下去,就要学台湾小朋友,说出“好可爱好高兴噢”这样的鬼话。台湾人造的新词新话,和他们的口音有关。国语口音纯正的人学起来很难听。 

除了广播员,说话港台化最为厉害的,当数一些女歌星。李敖先生骂老K(国民党),说他们“手淫台湾,意淫大陆”,这个比方太过粗俗,但很有表现力。我们的一些时髦小姐糟塌自己的语音,肯定是在意淫港币和新台币——这两个地方除了货币,再没什么格外让人动心的东西。港台人说国语,经常一顿一顿,你知道是为什么吗?他们在想这话汉语该怎么说啊。他们英语讲得太多,常把中国话忘了,所以是可以原谅的。我的亲侄子在美国上小学,回来讲汉语就犯这毛病。犯了我就打他屁股,打一下就好。中国的歌星又不讲英文,再犯这种毛病,显得活像是大头傻子。电台请歌星做节目,播音室里该预备几个乒乓球拍子。乒乒球拍子不管用,就用擀面杖。这样一级一级往上升,我估计用不到狼牙棒,就能把这种病治好。治好了广播员,治好了歌星,就可以治其它小姐的病。如今在饭店里,听见鼻腔里哼出一句港味的“先生”,我就起鸡皮疙瘩。北京的女孩子,干嘛要用鼻腔来说话! 

这篇文章一直在谈语音语调,但语音又不是我真正关心的问题。我关心的是,港台文化正在侵入内地。尤其是那些狗屎不如的电视连续剧,正在电视台上一集集地演着,演得中国人连中国话都说不好了。香港和台湾的确是富裕,但没有文化。咱们这里看上去没啥,但人家还是仰慕的。所谓文化,乃是历朝历代的积累。你把城墙拆了,把四合院扒了,它还在人身上保留着。除了语音,还有别的——就拿笔者来说,不过普普通通一个北方人,稍稍有点急公好义,仗义疏财,有那么一丁点燕赵古风,台湾来的教授见了就说:你们大陆同学,气概了不得…… 

我在海外的报刊上看到这样一则故事:有个前国军上校,和我们打了多年的内战;枪林弹雨都没把他打死。这一方面说明我们的火力还不够厉害,另一方面也说明这个老东西确实有两下子。改革开放之初,他巴巴地从美国跑了回来,在北京的饭店里被小姐骂了一顿,一口气上不来,脑子里崩了血筋,当场毙命。就是这样可怕的故事也挡不住他们回来,他们还觉得被正庄京片子给骂死,也算是死得其所。我认识几位华裔教授,常回大陆,再回到美利坚,说起大陆服务态度之坏,就扼腕叹息道:再也不回去了。隔了半年,又见他打点行装。问起来时,他却说:骂人的京片子也是很好听的呀!他们还说:骂人的小姐虽然粗鲁,人却不坏,既诚实又正直,不会看人下菜碟,专拍有钱人马屁——这倒不是谬奖。八十年代初的北京小姐,就是洛克菲勒冒犯到她,也是照骂不误:“别以为有几个臭钱就能在我这儿起腻,惹急了我他妈的拿大嘴巴子贴你!”断断不会见了港客就骨髓发酥非要嫁他不可——除非是领导上交待了任务,要把他争取过来。粗鲁虽然不好,民族自尊心却是好的,小姐遇上起腻者,用大嘴巴子去“贴”他,也算合理;总比用脸去贴好罢。这些事说起来也有十几年了。如今北京多了很多合资饭店,里面的小姐不骂人,这几位教授却不来了。我估计是听说这里满街的鸟语,觉着回来没意思。他们不来也不要紧,但我们总该留点东西,好让别人仰慕啊。

\chapter{电影·韭菜·旧报纸}

看来,国产电影又要进入一个重视宣传教育的时期。我国电影的从业人员,必须作好艰苦奋斗的思想准备——这是我们的光荣传统。七十年代中期,我在北京的街道工厂当工人,经常看电影,从没花钱买过电影票,都是上面发票。从理论上说、电影票是工会买的,但工会的钱又从哪里来?我们每月只交五分钱的会费。这些钱归根结底是国家出的。严格地说,当时的电影没有票房价值,国家出钱养电影。今后可能也是这样。正如大家常说的,国家也不宽裕,电影工作者不能期望过高。这些都是正经话。 

国家出钱让大家看电影。就是为了宣传和教育。坦白地说,这些电影我没怎么看。七四年、七五年我闲着没事。还去看过几次,到了七七、七八年,我一场电影都没看。那时期我在复习功课考大学,每分钟都很宝贵。除我以外,别的青工也不肯去看。有人要打家具,准备结婚,有人在谈朋友;总之,大家都忙。年轻人都让老师傅去看,但我们厂的师傅女的居多、她们说,电影院里太黑,没法打毛衣——虽然摸着黑也可以打毛衣,但师傅们说:还没学会这种本领。其结果就是,我们厂上午发的电影票,下午都到了字纸篓里。我想说的是,电影要收到宣传教育的结果,必须有人看才成。这可是个严肃的问题。除了编导想办法,别人也要帮着想办法。根据我的切身经历,我有如下建议:假如放映工会包场,电影院里应该有适当的照明,使女工可以一面看电影,一面打毛衣,这样就能把人留在场里。 

当然,电影的宣传教育功能不光体现在城市,还体现在广阔的农村、在这方面我又有切身体验。七十年代初,我在云南插队。在那个地方,电影绝不缺少观众。任何电影都有人看,包括《新闻简报》。但你也不要想到票房收入上去。有观众,没票房,这倒不是因为观众不肯掏钱买票,而是因为他们根本就没有钱。我觉得在农村放电影,更能体现电影的宣传、教育功能。打个比方说,在城市的电影院放电影,因为卖票,就像是职业体育;在农村放电影,就像业余体育。业余体育更符合奥林匹克精神。但是干这种事必需敬业,有献身精神——为此,我提醒电影工作者要艰苦奋斗;放电影的人尤其要有这种精神。我插队时尽和放映员打交道,很了解这件事情。那时候我在队里赶牛车,旱季里,隔上十天半月,总要去接一次放映员,和他们搞得很熟…… 

有一位心宽体胖的师傅分管我们队,他很健谈,可惜我把他的名字忘掉了。我不光接他,还要接他的设备。这些设备里不光有放映机,还有盛在一个铁箱里的汽油发电机。这样他就不用使脚踏机来发电了。赶着牛车往回走时,我对他的工作表示羡慕:想想看,他不用下大田,免了风吹日晒,又有机器可用、省掉了自己的腿,岂不是轻省得很。但是他说,我说得太轻巧,不知道放映员担多大责任。别的不说,片子演到银幕上,万一大头朝下,就能吓出一头冷汗。假如银幕上有伟大领袖在内,就只好当众下跪,左右开弓扇自己的嘴巴,请求全体革命群众的原谅。原谅了还好,要是不原谅,捅了上去,还得住班房——这种事情是有的,而且时常发生。也不知为什么,放映员越怕,就越要出这种事。他说放电影还不如下大田。这是特殊年代里的特殊事件,没有什么普遍意义。但他还说:宣传工作不好干——这就有普遍意义了。就拿放电影来说吧,假如你放商业片,放坏了,是你不敬业;假如这片子有政治意义,放坏了,除了不敬业,还要加一条政治问题。放电影的是这样,拍电影的更是这样。这问题很明白,我就不多说了。 

越不好干的工作,就越是要干,应该有这种精神。我接的这位师傅就是这样。他给我们放电影,既没有报酬,更谈不上红包。我们只管他的饭,就在我们的食堂里吃。这件事说起来很崇高,实际上没这么崇高。我所在的地方是个国营农场,他是农场电影队的,大家同在一个系统,没什么客套。走着走着,他问起我们队的伙食怎样。这可不是瞎问:我们虽是农场,却什么家当都没有,用两只手种地,自己种自己吃,和农民没两样。那时候地种得很坏,我就坦白地说,伙食很槽。种了一些花生,遭了病害,通通死光,已经一年没油吃。他问我有没有菜吃,我说有。他说,这还好。有的队菜地遭了灾,连菜都没有,只能拿豆汤当菜。他已经吃了好几顿豆汤,不想再吃了。我们那里有个很坏的风气,叫作看人下菜碟。首长下来视察就不必说了,就是兽医来阉牛,也会给他煎个荷包蛋。就是放映员来了,什么招待也没有。我也不知是为什么。 我讲这个故事,是想要说明,搞电影工作要艰苦奋斗。没报酬不叫艰苦奋斗,没油吃不叫艰苦奋斗,真正的艰苦马上就要讲到。回到队里,帮他卸下东西,我就去厨房——除了赶牛车,我还要帮厨。那天和往常一样,吃凉拌韭菜。因为没有油,只有这种吃法。我到厨房时,这道菜已经泡制好了,我就给帮着打饭打菜。那位熟悉的放映员来时、我还狠狠地给了他两勺韭菜,让他多吃一些。然后我也收拾家当,准备收摊;就在这时,放映员仁兄从外面猛冲了进来,右手扼住了自己的脖子,舌头还拖出半截,和吊死鬼一般无二。当然,他还有左手:这只于举着饭盆让我看——韭菜里有一块旧报纸。照我看这也没有什么。他问我:韭菜洗了没有,我说洗大概是洗了的,但不能保证洗得仔细。但他又问:你们队的韭菜是不是用大粪来浇?我说:大概也不会用别的东西来浇……然后才想了起来,这大概是队部的旧报纸。旧报纸上只要没有宝像,就有人扯去方便用,报纸就和粪到了一起——这样一想,我也觉得恶心起来,这顿韭菜我也没吃。可钦可佩的是,这位仁兄干呕了一阵,又去放电影了。以后再到了我们队放电影,都是自己带饭,有时来不及带饭,就站在风口处,张大嘴巴说道;我喝点西北风就饱了——他还有点幽默感。需要说明的是,洗韭菜的不是我;假如是我洗的,让我不得好死。这些事是我亲眼所见,放映员同志提心吊胆,在韭菜里吃出纸头,喝着西北风,这就是艰苦奋斗的故事。相比之下,今天的电影院经理。一门心思地只想放商业片,追求经济效益,不把社会效益、宣传工作放在心上,岂不可耻!但活又说回来,光喝西北风如何饱肚,这还需要认真研究。

\chapter{自然景观和人文景观}

我到过欧美的很多城市,美国的城市乏善可陈,欧洲的城市则很耐看。比方说,走到罗马城的街头,古罗马时期的竞技场和中世纪的城堡都在视野之内。这就使你感到置身于几十个世纪的历史之中。走在巴黎的市巾心,周围是漂亮的石头楼房,你可以在铁栅栏上看到几个世纪之前手工打出的精美花饰。英格兰的小城镇保留着过去的古朴风貌,在厚厚的草顶下面,悬挂出木制的啤酒馆招牌。我记忆中最漂亮的城市是德国的海德堡,有一座优美的石桥夹在内卡河上,河对岸的山上是海德堡选帝侯的旧官堡。可以与之相比的有英国的剑桥,大学设在五六百年前的石头楼房里,包围在常春藤的绿荫里——这种校舍不是任何现代建筑可比。比利时的小城市和荷兰的城市,都有无与伦比的优美之处,这种优美之处就是历史。相比之下,美国的城市很是庸俗,塞满了乱糟糟的现代建筑。他们自己都不爱看,到了夏天就跑到欧洲去度假——历史这种东西,可不是想有就能有的呀。 

有位意大利的朋友告诉我说,除了脏点、乱—点,北京城很像一座美国的城市。我想了一下,觉得这是实情——北京城里到处是现代建筑,缺少历史感。在我小的时候就不是这样的,那时的北京的确有点与众不同的风格。举个例子来说,我小时候作在北京的郑工府里,那是一座优美的古典庭院,眼看着它就变得面门全非、塞满了四四方方的楼房,丑得要死。郑王府的遭遇就是整个北京城的缩影。顺便说一句,英国的牛津城里,所有的旧房子,屋主有翻修内部之权,但外观一毫不准动,所以那座城市保持着优美的旧貌。所有的人文景观属于我们只有一次。假如你把它扒掉,再重建起来就不是那么回事了。 

这位意大利朋友还告诉我说,他去过山海关边的老龙头,看到那些新建的灰砖城楼,觉得很难看。我小时候见过北京城的城楼,还在城楼边玩耍过,所以我不得不同意他的意见。真古迹使人留恋之处,在于它历经沧桑直至如今,在它身边生活,你才会觉得历史至今还活着。要是可以随意翻盖,那就会把历史当作可以随意捏造的东西,一个人尽可夫的娼妇;这两种感觉真是大不相同。这位意大利朋友还说,意大利的古迹可以使他感到自己不是属于一代人,而是属于一族人,从亘古到如今。他觉得这样活着比较好,他的这些想法当然是有道理的,不过,现在我们谈这些已经有点晚了。 

谈过了城市和人文景观。也该谈谈乡村和自然景观——谈这些还不晚。房龙曾说,世界上最美丽的乡村就在奥地利的萨尔兹堡附近。那地方我也去过,满山枞木林,农舍就在林中。铺了碎石的小径一尘不染……还有荷兰的牧场,弥漫精心修整的人工美。牧场中央仓放干草的小亭子,油漆得整整齐齐,像是园林工人干的活;因为要把亭子造成那个样于,不但要手艺巧,还要懂什么是好看。让别人看到自己住的地方是—种美丽的自然景观,这也是一种作人的态度。谈论这些域外的风景不是本文土旨,主旨当然还是讨论中国。我前半辈子走南闯北,去过国内不少地方,就我所见,贫困的小山村,只要不是穷到过不下去,多少还有点样。到了靠近城市的地方,人也算有了点钱,才开始难看。家家户户房子宽敞了,院墙也高了,但是样子恶俗,而且门前渐渐和猪窝狗圈相类似。到了城市的近郊,到处是乱倒的垃圾。进到城里以后,街上是干净了, 那是因为有清洁工在扫。只要你往楼道里看一看,阳台上看一眼,就会发现,这里住的人比近郊区的人还要邋遢得多。总的来说,我以为现在到处都是既不珍惜人文景观、也不保护自然景观的邋遢娘们邋遢汉。这种人要吃,要喝,要自己住得舒服,别的一概不管。 

我的这位意大利朋友是个汉学家。他说,中国入只重写成文字的历史,不重保存环境中的历史。这话从—个意大利人嘴里说出来,叫人无法辩驳。人家对待环境的态度比我们强得多。我以为,每个人都有—部分活在自己所在的环境中,这一部分是不会死的,它会保存在那里,让后世的人看到。住海德堡,在剑桥,在萨尔兹堡,你看到的不仅是现世的人,还有他们的先人,因为世世代代的维护,那地方才会像现在这样漂亮。和青年朋友谈这些,大概还有点用。

\chapter{小说的艺术} 

朋友给我寄来一本昆德拉的《被背叛的遗嘱》,这是本谈小说艺术的书。书很长,有些地方我不同意,有些部分我没看懂(这本书里夹杂着五线谱,但我不识谱,家里更没有钢琴);但还是能看懂能同意的地方居多(我对此书有种特别的不满,那就是作者丝毫没有提到现代小说的最高成就:卡尔维诺、尤瑟娜尔、君特·格拉斯、莫迪阿诺,还有一位不常写小说的作者,玛格丽特·杜拉,早在半世纪以前,沃威格就抱怨说,哪怕是大帅的作品,也有纯属冗余的成分。假如他活到了现在,看到现代小说家的作品,这些怨言就没有了。昆德拉不提现代小说的这种成就,是因为同行嫉妒,还是艺术上见解不同。我就不得而知。当然,昆德拉提谁、不提谁,完全是他的自由。但若我来写这本书,一定要把这件事写上。不管怎么说吧,我同意作者的意见,的确存在一种小说的艺术,这种艺术远不是谁都懂得。昆德拉说:不懂开心的人不会懂得任何小说艺术,除了懂得开心,还要懂得更多,才能懂得小说的艺术。但若连开心都不懂,那就只能把小说读糟蹋了。归根结底,昆德拉的话并没有错。 

我自己对读小说有一种真正的爱好,这种爱好不可能由阅读任何其它类型的作品所满足。我自己也写小说,写得好时得到的乐趣,绝非任何其它的快乐可以替代。这就是说,我对小说有种真正的爱好;而这种爱好就是对小说艺术的爱好——在这—点上我可以和昆德拉沟通。我想像一般的读者并非如此,他们只是对文化生活有种泛泛的爱好。现在有种论点,认为当代文学的主要成就是杂文,这或者是事实,但我对此感到悲哀。我自己读杂文,有时还写点杂文。照我看,杂文无非是讲理,你看到理在哪里,径直一讲就可,当然,把道理讲得透彻、讲得漂亮,读起来也有种畅快淋腐的快感。但毕竟和读小说是两道劲儿。写小说则需要深得虚构之美,也需要些无中生有的才能;我更希望能把这件事做好。所以,我虽能把理讲好,但不觉得这是长处,其至觉得这是一种劣根性、需要加以克服。诚然,作为一个人,要负道义的责任,憋不住就得说,这就是我写杂文的动机。所以也只能适当克服,还不能完全克服。 

前不久在报上看到一种论点,说现在杂文取代了小说,负起了社会道义的责任。假如真是如此,那倒是件好事——小说来负道义责任,那就如希腊人所说,鞍子扣到头上来了——但这是仅就文学内部而言。从整个社会而言,道义责任全扣在提笔为文的人身上还是不大对头。从另一方面来看,负道义责任可不是艺术标淮;尤其不是小说艺术的标难。这很重要啊。 

昆德拉的书也主要是说这个问题。写小说的人要让人开心,他要有虚构的才能,并要有施展这种才能的动力——我认为这是主要之点。昆德拉则说,看小说的人要想开心,能够欣赏虚构,并已能宽容虚构的东西——他说这是主要之点。我倒不存这种奢望。小说的艺术首先会形成在小说家的意愿之中,以后会不会遭人背叛,那是以后的事。首先要有这种东西,这才是最主要的。 

昆德拉说,小说传统是欧洲的传统;但若说小说的艺术在中国从未受到重视,那也是不对的。在很多年的,曾有过一个历史的瞬间:年轻的张爱玲初露头角,显示出写小说的才能,傅雷先生发现了这一点,马上写文章说:小说的技巧值得注意。那个时候连张春桥都化名写小说,仅就艺术而言,可算是一团糟。张爱玲确是万绿丛中一点红——但若说有什么遗嘱被背叛了,可不是张爱玲的遗嘱,而是傅雷的遗嘱。天知道张爱玲后来写的那叫什么东西。她把自己的病态当作才能了……人有才能还不叫艺术家,知道珍视自己的才能才叫艺术家呢。笔者行文至此,就欲结束。但对小说的艺术只说了它不是什么,它到底是什么,还一字未提。假如读者想要明白的话,从昆德拉的书里也看不到;应该径直找两本好小说看看。看完了能明白则好,不能明白也就无法可想了,可以去试试别的东西;千万别听任何人讲理,越听越糊涂。任何一门艺术只有从作品里才能看到——套昆德拉的话说,只喜欢看杂文、看评论、看简介的人,是不会懂得任何一种艺术的。

\chapter{旧片重温}

我小时看过的旧片中,有一部对我有持殊意义,是《北国江南》。当时我正上到小学高年级,是学校组织去看的。这是一部农村题材的电影,由秦怡女士主演。我记得她在那部电影里面瞎了眼睛.还记得那部电影惨咧咧的,一点都不好看——当然,这是说电影,不是说秦怡,秦女士一直是很好看的——别的一点都记不得。说实在的,小男孩只爱看打仗的电影,我能在影院里坐到散场,就属难能可贵。这部电影的特殊之处在于:我去看时还没有问题,看过之后就出了问题:阶级斗争问题和路线斗争问题。这种问题我一点都没看出来,说明我的阶级觉悟和路线觉悟都很低。这件事引起了我的警惕,同时也想到,电影不能单单当电影来看,而是要当谜语来猜,谜底就是它问题何在。当然,像这种电影后来还有不少,但这是第一部,所以我牢牢记住了这个片名:《北国江南》。但它实在不对我胃口,所以没有记住内容。 

和我同龄的人会记得,电影开始出问题,是在六十年代中期,准确地说.是六五年以后。在此之前也出过,比方说,电影《武训传》,但那时我太小。六五年我十三岁,在这个年龄发生的事对我们一生都有影响。现在还有人把电影当谜语猜,说每部片子都有种种毛病。我总是看不出来,也可能我这个人比较鲁钝,但是必须承认,六五年六六年那些谜语实在是难猜。举例来说,有—部喜剧片《龙马精神》,说到有一匹瘦马,“脊梁比刀子快,屁股比锥子快,躺下比起来快”。这匹马到了生产队的饲养员大叔手里,就被养得很肥。这部电影的问题是:这匹马起初怎么如此的瘦,这岂不是给集体经济抹黑?这个谜底就大出我的意外。从道理上讲,饲养员大叔把瘦马养肥了,才说明他热爱集体。假如马原来就胖,再把它喂得像一口超级肥猪,走起来就喘,倒不一定是关心集体。但是《龙马精神》还是被枪毙掉了。再比方说,电影《海鹰》,我没看出问题来。但人家还是给它定了罪状。这电影中有个镜头,一位女民兵连长(王晓棠女士饰)登上了丈夫(一位海军军官)开的吉普车,杨尘而去。人家说,这女人不像民兵连长,简直像吉普女郎。所谓吉普女郎,是指解放前和美国兵泡的不正经的女人。说实在的,一般电影观众,除非本人当过吉普女郎,很难看出这种意思来。所以,我没看出这问题,也算是有情可原。几乎所有的电影都被猜出了问题,但没有一条是我能看出来的。最后只剩下了“三战一哈”还能演。三战是《地道战》、《地雷战》、《南征北战》,大多不是文艺片,是军事教育片。这“一哈”是有关一位当时客居我国的亲王的新闻片,这位亲王带着他的夫人,一位风姿绰约的公主,在我国各地游览,片子是彩色的,满好看,上点年纪的读者可能还记得。除此之外,就是《新闻简报》,这是黑白片,内容干篇一律,一点不好看。有一个流行于七十年代的顺口溜,对各国电影做出了概括:朝鲜电影,又哭又笑;日本电影,内部卖票;罗马尼亚电影,莫明其妙;中国电影,《新闻简报》。这个概括是不正确的,起码对我国概括得不正确。当时的中国电影,除《新闻简报》,还剩了点别的。 

这篇文章是从把电影当谜语来猜说起的。在六十年代末七十年代初,大多数的电影都被指出隐含了反动的寓意,枪毙实在是罪有应得。然后开始猜书。书的数量较多,有点猜不过来,但最后大多也有了结论:通通是毒草——红宝书例外。然后就猜人。好好—个人、看来没有毛病,但也被人找出谜底来:不是大叛徒、就是大特务,一个个被关进了牛棚;没被关进去的大都不值得一猜,比方说,我,一个十四五岁的中学生,关我就没啥意思,但我绝不认为自己身上就猜不出什么来。到了这个程度,似乎没有可猜的了吧?但人总能找出事干,这时就猜一切比较复杂的图案。有一种河南出产的香烟“黄金叶”,商标是一张烟叶,叶子上脉络纵横,花里胡哨。红卫兵从这张烟叶上看出有十几条反动标语,还有蒋介石的头像。我找来一张“黄金叶”的烟盒,对着它端详起来,横看看、竖着看,—条也没看出来。不知不觉,大白天的落了枕,疼痛难当,脖子歪了好几个月。好在年龄小,还能正过来。到了这时,我终于得出一个结论:这种胡乱猜疑,实在是扯淡得很。这是个普通猜疑的年代、没都能猜出有来。任何一种东西,只要足够复杂,其个有些难以解释的东西,就被注坏里猜。电影这种产品,信息含量很高,就算是最单纯的电影,所包含的信息也多过“黄金叶”的图案,想要没毛病,根本就不可能。所以,你要是听说某部电影有厂问题,干万不要诧异。我们这代人,在猜疑的年代长大,难免会落下毛病,想从鸡蛋里挑出骨头,这样才显出自己能来,这是很不好的。但你若说,我这篇短文隐含了某些用意,我要承认,你说对了,不是胡乱猜疑。
 
\chapter{我对小说的看法}

我自幼就喜欢读小说,并且一直以为自己可以写小说,直到二十七八岁时,读到了图尼埃尔(Tournier,M.)的一篇小说,才改变了自己的看法。在不知不觉之中,小说已经发生了很大的变化。现代小说和古典小说的区别,就像汽车和马车的区别一样大。现代小说中的精品,再不是可以一目十行往下看的了。为了让读者同意我的意见,让我来举一个例子:杜拉斯(Duras,M.)《情人》的第一句是:“我已经老下。……”无限沧桑尽在其中。如果你仔细读下去,就会发现,每句话的写法大体都是这样的,我对现代小说的看法,就是被《情人》固定下来的。现代小说的名篇总是包含了极多的信息,而且极端精美,让读小说的人狂喜,让打算写小说的人害怕。在经典作家里,只有俄国的契诃夫(Chekhov,A.P.)偶而有几笔写成这样,但远不是通篇都让人敬畏。必须承认,现代小说家曾经使我大受惊吓。我读过的图尼埃尔的那篇小说,叫作《少女与死》.它只是一系列惊吓的开始。 

因为这个发现,我曾经放弃了写小说,有整整十年在干别的事,直到将近四十岁,才回头又来尝试写小说。这时我发现,就是写过一些名篇的现代小说家,平常写的小说也是很一般的。瑞士作家迪伦马特(Durrenmatt,F.)写完了他的书篇《法官和他的刽子手》之后,坦白说,这个长中篇耗去了他好几年的光阴,而且说,今后他不准备再这样写下去了。此后他写很多长篇,虽然都很好看,但不如《法官和他的刽子手》精粹。杜拉斯也说,《情人》经过反复的修改,每—段、每一句都重新安排过。照我看,她的其它小说都不如《情人》好。他们的话让人看了放心,说明现代小说家也不是一群超人,他们有些惊世骇俗的名篇,但是既不多,也不长。虽然如此,找还是认为,现代小说中几个中篇,如《情人》之类,比之经典作家的宏篇巨毫不逊色,爱好古典文学的人也许不会同意我的看法,我也没打算说服他们。但我还是要说,我也爱好过古典文学;而在影视发达的现代,如果没有现代小说,托尔斯泰并不能让我保持阅读的习惯。 

我认为,现代小说的成就建筑在不多几个名篇上,虽然凭这几篇小说很难评上诺贝尔文学奖,但现代小说艺术的顶峰就在其中。我的抱负也是要在一两篇作品里达到这个水平。我也特别喜欢写长中篇(六万字左右),比如我的《未来世界》,就是这么长。《情人》、《法官和他的刽子手》等名篇也是这么长。当然,这样做有东施效颦之嫌,在我写过的小说里,《黄金时代》(《联合报》第十三届中篇小说奖)是最满意的,但是还没达到我希望的水准,所以还要继续努力。
 
\chapter{另一种文化}

我老婆原是学历史的“工农兵大学生”。大学三年级时,有一天.一位村里来的女同学在班上大声说道:我就不知道什么是太监!说完了这话,还作顾盼自雄之状。班上别的同学都跟着说:我也不知道,我也不知道。就我老婆性子直,羞答答地说:啊呀,我可能是知道的,太监就是阉人嘛。人家又说:什么叫作阉人?她就说不出口,闹了个大红脸。当时她是个女孩子,在大庭广众之下承认自己知道什么是太监、阉人,受了很大的刺激;好一阵子灰溜溜的、不敢见人也不敢说话。 

但后来她就走向了反面,不管见到谁,总把这故事讲给别人听,未了还要加上一句恶毒的评论:哼,学历史的大学生不知道什么是太监,书都念到下水里去了!没有客人时,她就把这故事讲给我听。我听了二百来遍,实在听烦了。有一回,禁不住朝她大吼了一声:你就少说几句吧!人家是农村来的,牲口又不穿裤子——没见过阉人,还没见过阉驴吗!这一嗓子又把她吼了个大红脸,这一回可是真的受了刺激,恼羞成怒了,有好几天不和我说话。假如说,这话是说村里来的女同学知道太监是什么,硬说不知道,我自己也觉得过份。假如说,这话是说那位女同学只知道阉驴不知道太监,那我吼叫些什么?所以,我也不知自已是什么意思。不知道自己什么意思,但还是有点意思,这就是种文化呀。 

依我之见,文化有两方面的内容:一种是各种书本知识,这种文化我老婆是有的,所以她知道什么是太监。另一种是各种暖昧的共识,以及各种可意会不可言传的精妙气氛,一切尽在不言中——这种文化她没有,所以,她就不知道要说自己不知道什么是太监。你别看我说得头头是道,在这后一方面我也是个土包子。我倒能管住自己的嘴,但管不住自己的笔。我老婆是乱讲,我是乱写。我们俩都是没文化的野人。 

我老婆读过了博士,现在是社会学家,做过性方面的研究,熟悉这方面的文献——什么homo、S/M,各种乱七八槽,她全知道。这样她就自以为很有学问,所到之处,非要直着脖子嚷嚷不可。有一次去看电影《霸王别姬》,演到关师傅责打徒弟一场,那是全片的重头戏。整个镜头都是男人的臀部,关师傅舞着大刀片(木头的)劈劈啪啪在上面打个不休,被打者还高呼:“打得好!师傅保重!打得好!师傅保重!”相信大家都知道应该看点什么、更知道该怎么看。我看到在场的观众都很感动,有些女孩眼睛都湿润了。这是应该的,有位圈内朋友告诉我,导演拍这一幕时也很激动,重拍了无数次,直到两位演员彻底被打肿。每个观众都很激动,但保持了静默……大家都是有文化的。就是我老婆,假个直肠子驴一样吼了出来:大刀片子不够性感!大刀片子是差了点意思,你就不能将就点吗?这一嗓子把整个电影院的文化气氛扫荡了个干净。所有的人都把异样的目光投向我们,我想找个地缝钻进去,但没有找到。最近,她又闹着要我和她去看《红樱桃》。我就是不去.在家里好好活着,有什么不好,非要到电影院里去找死……这些电影利用了观众的暖昧心理,确实很成功。 

国内的大片还有一部《红粉》。由于《霸王别姬》的前车之鉴,我没和老婆一起看,是自己偷着看的。这回是我瞎操心,这片子没什么能让她吼出来的,倒是使我想打瞌睡。我倒能理解编导的创意:你们年轻人,生在红旗下、长在红旗下,知道什么是妓女吗?好罢,我来讲一个妓女的故事……满心以为我们听到妓女这两个字就会两眼发直;但是这个想法有点过分。在影片里,有位明星刮了头发做尼姑,编导一定以为我们看了大受刺激。这个想法更过分:见了小尼姑就两眼发直,那是阿Q!我们又不是阿Q。有些电影不能使观众感到自己暧昧,而是感到编导暧昧,这就不够成功。 

影视方面的情形就是这样:编导们利用“一切尽在不言”中的文化氛围,确实是大有可为。但我们写稿子的就倒了霉:想要使文字暧昧、可意会不可言传,就只好造些新词、怪词;或者串几句英文。我现在正犯后一种毛病,而且觉得良心平安:英文虽然难懂,但毕竞是种人话,总比编出一种鬼话要强一点罢。 前面所写的homo、S/M,都是英文缩写。虽然难懂,但我照用不误。这主要是因为写出的话不够暧昧,就太过直露,层次也太低。这篇短文写完之后,你再来问我这些缩写是什么意思,我就会说:我也不知道,忘掉了啊。我尤其不认识一个英文单词,叫作Pervert;刚查了字典马上就忘。我劝大家也像我这样。在没忘掉之前,我知道是指一类人,害怕自己的内心世界.所以鬼鬼祟祟的。这些人用中国话来说,就是有点变态。假如有个Pervert站出来说:我就是个Pervert,那他就不是个pervert。当且仅当一个人声称;我就不知道pervert是什么时,他才是个Pervert。假如我说,我们这里有种pervert的气氛,好多人就是Pervert,那我就犯了众怒。假如我说,我们这里没有Pervert的气氛,也没有人是Pervert、那恰恰说明我正是个Pervert。所以,我就什么都不说了。
 
\chapter{中国为什么没有科幻片}

王童叫我回答一个问题:为什么中国没有科幻片。其实,这问题该去问电影导演才对。我认得一两位电影导演,找到一位当面请教时,他就露出一种蒙娜·丽莎的微笑来,笑得我混身起鸡皮疙瘩。笑完了以后他朝我大喝一声:没的还多着的哪!少跟我来这一套……吼得我莫名其妙,不知自己来了哪一套。搞电影的朋友近来脾气都不好,我也不知为什么。 

既然问不出来,我就自己来试着回答这个问题。我在美国时,周末到录相店里租片子,“科幻”一柜里片子相当多,名虽叫作科幻,实际和科学没什么大关系。比方说,《星际大战》,那是一部现代童话片。细心的观众从里面可以看出白雪公主和侠盗罗滨汉等一大批熟悉的身影。再比方说,《侏罗纪公园》。那根本就是部恐怖片。所谓科幻,无非是把时间放在未来的一种题材罢了。当然,要搞这种电影,一些科学知识总是不可少的,因为在人类的各种事业中,有一样总在突飞猛进的发展,那就是科学技术,要是没有科学知识,编出来也不像。 

有部美国片子《苍蝇》,国内有些观众可能也看过,讲一个科学家研究把人通过电缆发送出去。不幸的是,在试着发送自己时,装置里混进了一只苍蝇,送过去以后,他的基因和苍蝇的基因就混了起来,于是自己他就一点点地变成了一只血肉模糊的大苍蝇——这电影看了以后很恶心,因为它得了当年的奥斯卡最佳效果奖。我相信编这个故事的人肯定从维纳先生的这句话里得到了启迪:从理论上说,人可以通过一条电线传输,但是这样做的困难之大,超出了我们的能力。想要得到这种启迪,就得知道维纳是谁:他是控制论的奠基人,得过诺贝尔奖,少年时代是个神童——这样扯起来就没个完了。总而言之,想搞这种电影,编导就不能上电影学院,应该上综合性大学。倒也不必上理科的课,只要和理科的学生同宿舍,听他们扯几句就够用了。据我所知,综合性大学的学生也很希望在校园里看到学电影的同学。尤其是理科的男学生,肯定希望在校园里出现一些表演系的女生……这很有必要。中国的银幕上也出现过科学家的形象,但都很不像样子,这是因为搞电影的没见过科学家。演电影的人总觉得人若得了博士头衔,非疯即傻。实际上远不是这样。我老婆就是个博士。她若像电影上演得那样,我早和她离婚了。 

除了要有点科学知识,搞科幻片还得有点想象力。对于创作人员来说,这可是个硬指标。这类电影把时间放到了未来,脱离了现实的束缚,这就给编导以很大自由发挥的空间——其实是很严重的考验。真到了这片自由的空间里,你又搞不出东西来,恐怕是有点难堪。拍点历史片、民俗片,就算没拍好,也显不出寒碜。缺少科学知识,没有想象力,这都是中国出不了科幻片的原因——还有一个原因,科幻片要搞好,就得搞些大场面,这就需要钱——现在是社会主义初级阶段,没那么多钱。好了,现在我已经有了很完备的答案。但要这么回答王童,我就觉得缺了点什么…… 

我问一位导演朋友中国为什么没有科幻片,人家就火了。现在我设身处地地替他想想:假设我要搞部科幻片,没有科学知识,我可以到大学里听课。没有想象力,我可以喝上二两,然后面壁枯坐。俗话说得好,牛粪落在田里,大太阳晒了三天,也会发酵、冒泡的。我每天喝二两,坐三个小时,年复一年,我就不信什么都想不出来---最好的科幻本子不也是人想出来的吗?搞到后来,我有了很好的本子,又有投资商肯出钱,至于演员嘛,让他们到大学和科研单位里体验生活,也是很容易办到的——搞到这一步,问题就来了:假设我要搞的是《侏罗纪公园》那样的电影。我怎么跟上面说呢?我这部片子,现实意义在哪里?积极意义又在哪里?为什么我要搞这么一部古怪的电影?最主要的问题是:我这部电影是怎样配合当前形势的?这些问题我一个都答不上来,可答不上来又不行。这样一想,结论就出来了:当初我就不该给自己找这份麻烦。
 
\chapter{我的师承——《青铜时代序》} 

 我终于有了勇气来谈谈我在文学上的师承。小时候,有一次我哥哥给我念过查良铮先生译的《青铜骑士》:  

  我爱你,彼得建造的大城  

  我爱你庄严、匀整的面容  

  涅瓦河的流水多么庄严  

  大理石平铺在它的两岸……  

  他还告诉我说,这是雍容华贵的英雄体诗,是最好的文字。相比之下,另一位先生译的《青铜骑士》就不够好: 

  我爱你彼得的营造  

  我爱你庄严的外貌……  

  现在我明白,后一位先生准是东北人,他的译诗带有二人转的调子,和查先生的译诗相比,高下立判。那一年我十五岁,就懂得了什么样的文字才能叫作好。  

  到了将近四十岁时,我读到了王道乾先生译的《情人》,又知道了小说可以达到什么样的文字境界。 道乾先生曾是诗人,后来作了翻译家,文字功夫炉火纯青。他一生坎坷,晚年的译笔沉痛之极。请听听《情人》开头的一段:  

  “我已经老了。有一天,在一处公共场所的大厅里,有一个男人向我走来,他主动介绍自己,他对我说:我认识你,我永远记得你。那时候,你还很年轻,人人都说你很美,现在,我是特为来告诉你,对我来说,我觉得你比年轻时还要美,那时你是年轻女人,与你年轻时相比,我更爱你现在备受摧残的容貌。”  

  这也是王先生一生的写照。杜拉斯的文章好,但王先生译笔也好,无限沧桑尽在其中。查先生和王先生对我的帮助,比中国近代一切著作家对我帮助的总和还要大。现代文学的其它知识,可以很容易地学到。但假如没有像查先生和王先生这样的人,最好的中国文学语言就无处去学。除了这两位先生,别的翻译家也用最好的文学语言写作,比方说,德国诗选里有这样的译诗:  

  朝雾初升,落叶飘零 

  让我们把美酒满斟! 

  带有一种永难忘记的韵律,这就是诗啊。对于这些先生,我何止是尊敬他们——我爱他们。他们对现代汉语的把握和感觉,至今无人可比。一个人能对自己的母语做这样的贡献,也算不虚此生。  

  道乾先生和良铮先生都曾是才华横溢的诗人,后来,因为他们杰出的文学素质和自尊,都不能写作,只能当翻译家。就是这样,他们还是留下了黄钟大吕似的文字。文字是用来读,用来听,不是用来看的——要看不如去看小人书。不懂这一点,就只能写出充满噪声的文字垃圾。思想、语言、文字,是一体的,假如念起来乱糟糟,意思也不会好——这是最简单的真理,但假如没有前辈来告诉我,我怎么会知道啊。有时我也写点不负责任的粗糙文字,以后重读时,惭愧得无地自容,真想自己脱了裤子请道乾先生打我两棍。孟子曾说,无耻之耻,无耻矣。现在我在文学上是个有廉耻的人,都是多亏了这些先生的教诲。对我来说,他们的作品是比鞭子还有力量的鞭策。提醒现在的年轻人,记住他们的名字、读他们译的书,是我的责任。  

  现在的人会说,王先生和查先生都是翻译家。翻译家和著作家在文学史上是不能相提并论的。这话也对,但总要看看写的是什么样的东西。我觉得我们国家的文学次序是彻底颠倒了的:末流的作品有一流的名声,一流的作品却默默无闻。最让人痛心的是,最好的作品并没有写出来。这些作品理应由查良铮先生、王道乾先生在壮年时写出来的,现在成了巴比伦的空中花园了……以他们二位年轻时的抱负,晚年的余晖,在中年时如有现在的环境,写不出好作品是不可能的。可惜良铮先生、道乾先生都不在了……  

  回想我年轻时,偷偷地读到过傅雷、汝龙等先生的散文译笔,这些文字都是好的。但是最好的,还是诗人们的译笔;是他们发现了现代汉语的韵律。没有这种韵律,就不会有文学。最重要的是:在中国,已经有了一种纯正完美的现代文学语言,剩下的事只是学习,这已经是很容易的事了。我们不需要用难听的方言,也不必用艰涩、缺少表现力的文言来写作。作家们为什么现在还爱用劣等的文字来写作,非我所能知道。但若因此忽略前辈翻译家对文学的贡献,又何止是不公道。  

  正如法国新小说的前驱们指出的那样,小说正向诗的方向改变着自己。米兰·昆德拉说,小说应该像音乐。有位意大利朋友告诉我说,卡尔维诺的小说读起来极为悦耳,像一串清脆的珠子洒落于地。我既不懂法文,也不懂意大利文,但我能够听到小说的韵律。这要归功于诗人留下的遗产。  

  我一直想承认我的文学师承是这样一条鲜为人知的线索。这是给我脸上贴金。但就是在道乾先生、良铮先生都已故世之后,我也没有勇气写这样的文章。因为假如自己写得不好,就是给他们脸上抹黑。假如中国现代文学尚有可取之处,它的根源就在那些已故的翻译家身上。我们年轻时都知道,想要读好文字就去要读译著,因为最好的作者在搞翻译。这是我们的不传之秘。随着道乾先生逝世,我已不知哪位在世的作者能写如此好的文字,但是他们的书还在,可以成为学习文学的范本。我最终写出了这些,不是因为我的书已经写得好了,而是因为,不把这个秘密说出来,对现在的年轻人是不公道的。没有人告诉他们这些,只按名声来理解文学,就会不知道什么是坏,什么是好。 

\chapter{明星与癫狂}

笔者在海外留学时,有一次清早起来跑步,见到一些人带着睡袋在街头露宿。经询问,是大影星艾迪·摩菲要到这座城市来巡回演出,影迷在等着买票。摩菲的片子我看过几部,觉得他演得不坏。但花几十块钱买一张票到体育场里看他,我觉得无此必要,所以没有加入购票的行列,而是继续跑步,这样我就在明星崇拜的面前当了一回冷血动物——坦白地说,我一直是这样的冷血动物。顺便说一句,那座城市不大,倒有个很大的体育馆,所以票是富裕的,白天也能买到,根本用不着等一夜。而且那些人根本不是去等买票,而是终夜喝啤酒、放音乐、吵闹不休,最安静的人也在不停地格格傻笑,搞得邻居很有意见。凭良心说,正常人不该是这个样子。至于他们进了体育馆、见到了摩菲之后.闹得就更厉害,险些把体育馆炸掉了。所以我觉得他们排队买票时是在酝酿情绪,以便晚上纵情地闹。此种情况说明.影迷(或称追星族)是有计划、有预谋地把自己置于一场癫狂之中。这种现象并不少见,每有英式足球比赛,或是摇滚歌星的演唱会,就会有人做出这种计划和预谋。当时我很想给埃迪·摩菲写封信、告诉他这些人没见到他时就疯掉了:以免他觉得这么多人都是他弄疯的,受到良心的责备。后来一想,这事他准是知道的,所以就没有写。 

现在我回到国内,翻开报纸的副刊,总能看到有关明星的新闻:谁和谁拍拖,谁和准分手了等等。明星作生意总能挣大钱.写本书也肯定畅销。明星的手稿还没有写出来就可以卖到几百万元,真让笔者羡慕不置。至于那文章,我认为写得真不怎样——不能和我祟拜的作家、也不能和我相比。在电视上可以看到影星唱歌,我觉得唱得实在糟——起码不能和帕瓦洛蒂相比(比我唱的当然要稍好一些,但在歌唱方面,笔者决不是个正面的榜样),但也有人鼓掌。房地产的开发商把昂贵的别墅送给影星,她赏个面子收下了,但绝不去住;开发商还觉得是莫大的荣耀。最古怪的是在万人会场里挤满了人,等某位明星上台去讲几句话,然后就疯狂地鼓掌;这使我想起了文革初的某些场景。我相信,假如有位明星跑到医院去,穿上白大褂,要客串一下外科医生的角色,肯定会有影迷把身体献上任她宰割,而且要求不打麻药;假如跳上民航的客机要求客串机长,飞机上肯定挤满了把生死置之度外的影迷,至于她自己肯不肯拿自己的生命来冒险,则是另一个问题。总而言之,在我们这个社会里,也开始出现了针对明星的癫狂,表面上没有美国闹得厉害,实际上更疯得没底。这种现象使我陷入了沉思之中。 

我认为明星崇拜是一种癫狂症,病根不在明星身上,而是在迫星族的身上。理由很简单:明星不过是—百斤左右的血肉之躯.体内不可能有那么多有害的物质,散发出来时,可以让数万人发狂。所以是追星族自己要癫狂。迫星族为什么要癫狂不是我的题目,因为我不是米歇尔·福柯。但我相信他的说法:正常人和疯子的界线不是那么清楚。笔者四十余岁,年轻时和同龄人一样,发过一种癫狂症,既毁东西又伤人,比追星还要有害。所以,有点癫狂不算有病,这种癫狂没了控制才是有病。总的来说,我不反对这件事,因为人既有这样一股疯劲,把它发泄掉总比郁积着好。在周末花几十元买一张票,把脑子放在家里,到体育场里疯上一阵,回来把脑子装上、再去上班,就如脱掉衣服洗个热水澡,对身心健康有某种好处,也末可知。我既然不反对这种癫狂,也就不会反对这冲癫狂的商业利用(叫作“明星制”吧?)。大众有这种需求、片商或穴头来操办,赚些钱,也算是公道。至于明星本人,在这些癫狂的场合出现,更没有任何可责各的地方。我所反对的,只是对这件事的误解。虽然有这种癫狂,大家并没有疯,这—点很重要。 

如前所述,迫星族常常有计划、有预谋地发一场癫狂;何时何地发作、发多久、发到什么程度、为此花费多少代价,都该由那些人自己来决定。倘若明星觉得自己可以控制这些人的癫狂,肯定是个不合理的想法,因为他把影迷当成了真的疯子。据报载,我国一位女影星晾台,涮了四川上万影迷,这些影迷有点发火了。这位女影星却说,这些影迷不懂什么叫作明星制,还举迈克尔·杰克逊为例。说这位男歌星涮了新加坡大数的歌迷,那些歌迷还觉得满开心云云。我以为女影星的说法是不对的。四川的影迷虽然没有新加坡的歌迷迷得那么凶,但迷到何种程度该由那些人自己来决定。倘若由你决定他们该达到哪个程度,人家就迷到什么程度,有这种想法就不正常。几年前就从报上看到有位男明星开车撞了人,不但不道歉,反要把受害者打一顿。显然,该男明星把受害者看做追星的影迷,觉得他该心甘情愿地挨顿揍,但后者有不同的看法,把他揪到警察那里去了。总而言之,用晾和揍的方法,让大家领略明星制的深奥,恐非正常人所为。最后的结论是:追星族不用我们操心,倒是明星,应该注意心理健康。 

最后再来说点题外之话。国外(尤其是指美国,但不包括港台)对待影星的态度有两重性:既有冷静地欣赏其表演的一面,也有追星起哄的一面。大影星同时也是优秀的演员,演出了一些经典约艺术片。好莱坞的影业也玩闹起哄,但恐怕另有些正经的。他是个有城府的拳师,会耍花拳绣腿;但也另有真相,不让你看到。鉴于这种情形,我怀疑所谓“明星制”,是帝国主义者打过来一颗阴险的糖衣炮弹——当然我也没有任何凭据。只是胡乱猜测——香港的影业已经中弹了。你别看它现在红火,群星灿烂,但早晚要被好莱坞吃掉;不信你就拿两地的片子比比看。至于在大陆,首批中弹的是演员。现在有明星,但没有出色的表演,更没有可以成为经典的艺术片。假如我没理解锗,这些明星还拿玩闹起哄当了真,当真以为自己是些超人。这个游戏玩到此种程度,已经过了,应该回头了。
 
\chapter{在美国左派家作客}

上个礼拜王先生来访问我,问我爱听谁的歌。我实在想不起歌手的名字,就顺口说了个披头士。其实我只是有时用披头士的歌来吵吵耳朵;现在我手上有这四个英国佬的几盒磁盘,LD连一张都没有,像这个样子大概也不算是他们的歌迷。只是一听到这些歌就会想到如烟的往事:好多年以前,我初到美国,深夜里到曼哈顿一位左派家里作客;当时他家里的破录音机上放着披头士的歌。说起来不好意思,我们根本不认识人家,只是朋友的朋友告诉了我们这个地址。夜里一两点钟—头幢了进去,而且一去就是四个人。坦白地说,这根本不是访友,而是要省住旅馆的钱——在纽约住店贵得很。假如不是左派,根本就不会让我们进去,甚至会打电话叫警察来抓我们。但主人见了我们却很高兴,陪我们聊了一夜,聊到了切·格瓦拉,托洛斯基,还有洁然的《金光大道》。这位先生家里有本英文的《金光大道》,中国出版,是朋友的朋友翻译的,我翻了翻,觉得译得并不好。这位朋友谈到了他们沸腾的六七十年代:反战运动、露天集会、大示威、大游行,还讲到从小红书上初次看到“造反有理”时的振奋心情。讲的时候,眼睛卑部冒金光。我们也有些类似的经历,但不大喜欢淡。他老想让我们谈谈中国的红卫兵,我们也不想谈。总的来说,他给我的印象就像某位旧友,当年情同手足,现在却话不投机——我总觉得他的想法有点极左的气味。要是按他的说法,我不必来美国学什么,应该回去接着造反,我不觉得这是个好主意。但不管怎么说,美国的左派人品都非常之好,这一点连右派也不得不承认。 

我记得这位左派朋友留了一头长发。穿着油光水滑的牛仔裤,留了一嘴大胡子,里面有不少白丝。在他那问窄小、肮脏的公寓里,有一位中年妇女,但不是他老婆。还有一个傻呵呵的金发女孩,也不是他的女儿。总的来说,他不像个成功人士。但历史会给他这样的人记上一笔,因为他们曾经挺身而出,反越战,反种族歧视,反对—切不公正。凌晨时分,我们都困了,但他谈意正浓——看来他惯于熬夜。在战斗的六七十年代,他们经常在公园里野营,在火准边上谈着吉它唱上一夜,还抽着大麻烟,这种生活我也有过。只不过不在公园里,是在山坡上。可能是在山边打坝,也可能是上山砍木头,一帮知青在野地里点堆火,噢噢地唱上一夜。至于大麻,我没有抽过。只是有一次烟抽完了,我拿云南出的大叶清茶给自己卷了一支,有鸡腿粗细。拿火柴一点,一团火冒了上来,把我的睫毛燎了个精光。茶叶里没有尼古丁,但有不少咖啡因,我抽了一口,感觉好像太阳穴上挨了两枪,一头栽倒在地。只可惜我们过这样的生活没有什么意义,只是自己受了些罪而已:对此我没什么可抱怨的,只是觉得已经够了,我想要干点别的——这是我和左派朋友最大的不同之处。但不管怎么说,在美国的各种人中,我最喜欢的还是左派。
 
\chapter{为什么老片新拍}

听说最近影视圈里兴起了一阵重拍旧片的浪潮,把一批旧电影重拍成电视连续剧,其中包括《敌后武工队》、《平原枪声》、《铁道游击队》等等。现在《野火春风斗古城》已经拍了出来,正在电视上演着。我看了几眼,虽然不能说全无优点,但也没什么新意。联想到前不久看到一些忠实于原著的历史剧,我怀疑一些电视剧编导正在走一条程式化的老路,正向传统京剧的力向发展。笔者绝不是京剧迷,但认识一位京剧迷。二十年前我当学徒工时,有位老师傅告诉我说,在老北平,他每天晚上都到戏园子坐坐。一出《长坂坡》不知看了多少遍,“谁的赵云”他都看过。对此需要详加解释:过去所有的武生大概都在《长坂坡》里演过赵云;而我师傅则看过一切武生演的赵云。因为还不是所有的男演员都演过杨晓冬、也不是所有的女演员都演过银环,现在我们还不能说淮的杨晓冬、谁的银环都看过;但是事情正朝这个方向发展,因为杨晓冬和银环正在多起来。而且我们也不妨末雨绸缪,把这件事提前说上一说。 

老实说,老片新拍(或者老戏重拍)不是什么新鲜事。我在美国时看过一部《疤脸人》,是大明星艾尔·帕西诺主演的彩色片。片尾忽然冒出一个字幕:以前有过一部电影《疤脸人》,然后就演了旧《疤脸人》人的几个片断。从这几个片断就可以看出,虽然新旧《疤脸入》是同一个故事,但不是同一部电影。我们还知道影片《乱》翻新了莎翁的名剧.至于《战争与和平》,不知被重拍了多少遍。一个导演对老故事有了崭新的体会,就可以重拍;保证观众有一个全新的《疤脸人》或《战争与和平》就是;而且这也是对过去导演的挑战。必须指出,就是这样的老戏重拍,我也不喜欢。但这种老片重拍和我们看到的连续剧还不是一回事。我看到的《野火春风午古城》,不仅忠于小说原著,而且也忠实于老的黑白片;观后感就是让我把早已熟悉的东西过上一遍——就如我师傅每晚在戏园子里把《长板坡》过一遍。前些时候有些历史连续剧,也是把旧小说搬上银幕,也是让大家把旧有的东西过一遍。同是过一遍,现在的连续剧和传统京剧不能比。众所周知,京剧是高度完美的程式化表演。连续剧里程式是有的,完美则说不上。 

我认为,现在中国人里有两种不同的欣赏趣味。一种是旧的,在传统社会相传统戏剧影响下形成的.那就是只喜欢重温旧的东西:另一种是新的,受现代影视影响形成的,只喜欢欣赏新东西。按前一种趣味来看现在的连续剧,大体上还能满意,只是觉得它程式化的程度不够。举例来说,现在连续剧里的银环,相老电影里的银环,长相不一样,表演也不一样,这就使人糊涂。最好勾勾脸,按同一种程式来表演。当然,既已有了程式,编导就是多余的。传统的京剧班子里就没有编导的地位。不过,养几个闲人观众也不反对。若按后一种趣味来看连续剧,就会说:这叫什么?照抄些旧东西,难道编导的艺术工作就是这样的吗?但后一种观众是需要编导的,只是嫌他没把工作做好。总而言之,老戏新拍使编导处于一种两面不讨好的尴尬地位:前一种观众要你的戏,但不要你这个人。后一种观众要你这个人,不要你的戏。换言之,在前一种观众面前,你是尸位素餐地鬼混着。在后一种观众面前,你是不称职或不敬业的编导。照我看来,老戏重拍真是不必要。我有一个作导演的朋友,他告诉我说:你不知道做编导的苦处,好多事都是不得已而为之。他这样一说,我倒是明白了。
 
\chapter{忆苦饭}

我靠写作为生。有人对我说:像你这样写是不行的啊,你没有生活!起初,我以为他想说我是个死人,感到很气愤。忽而想到,“生活”两字还有另一种用法。有些作家常到边远艰苦的地方去住上一段,这种出行被叫作“体验生活”——从字面上看,好像是死人在乍尸,实际上不是的。这是为了对艰苦的生活有点了解,写出更好的作品,这是很好的作法。人家说的生活,是后面一种用法,不是说我要死,想到了这一点,我又回嗔作喜。我虽在贫困地区插过队,但不认为体验得够了。我还差得很远,还需要进一步的体验。但我总觉得,这叫作“体验艰苦生活”比较好。省略了中间两个字,就隐含着这样的意思:生活就是要经常吃点苦头——有专门从负面理解生活的嫌疑。和我同龄的人都有过忆苦思甜的经历:听忆苦报告、吃忆苦饭,等等。这件事和体验生活不是一回事,但意思有点相近。众所周知,旧社会穷人过着牛马不如的生活,吃糠咽菜——菜不是蔬菜,而是野菜——所谓忆苦饭,就是旧社会穷人饭食的模仿品。 

我要说的忆苦饭是在云南插队时吃到的——为了配合某种形势,各队起码要吃一顿忆苦饭,上面就是这样布置的。我当时是个病号,不下大田,在后勤做事,归司务长领导,参加了做这顿饭。当然,我只是下手。真正的大厨是我们的司务长。这位大叔朴直木讷,自从他当司务长,我们队里的伙食就变得糟得很,每顿都吃烂菜叶——因为他说,这些菜太老,不吃就要坏了。菜园子总有点垂垂老矣的菜,吃掉旧的,新的又老了,所以永远也吃不到嫩菜。我以为他泡制忆苦饭肯定很在行,但他还去征求了一下群众意见,问大家在旧社会吃过些啥。有人说,吃过芭蕉树心,有人说,吃过芋头花、南瓜花。总的来说,都不是什么太难吃的东西,尤其是芋头花,那是一种极好的蔬菜,煮了以后香气扑鼻。我想有人可能吃过些更难吃的东西,但不敢告诉他。说实在的,把饭弄好吃的本领他没有,弄难吃的本领却是有的,再教教就更坏了。就说芭蕉树心吧,本该剥出中间白色细细一段,但他叫我砍了一颗笆蕉树来,斩碎了整个煮进了锅里。那锅水马上变得黄里透绿,冒起泡来,像锅肥皂水,散发着令人恶心的苦味…… 

我说过,这顿饭里该有点芋头花。但芋头不大爱开花,所以煮的是芋头杆,而且是刨了芋头剩下的老杆。可能这东西本来就麻,也可能是和芭蕉起了化学反应,总之,这东西下锅后,里面冒出一种很恶劣的麻味。大概你也猜出来了,我们没煮南瓜花,煮的是南瓜藤,这种东西斩碎后是些煮不烂的毛毛虫。最后该搁点糠进去,此时我和司务长起了严重的争执。我认为,稻谷的内膜才叫作糠。这种东西我们有,是喂猪的。至于稻谷的外壳,它不是糠,猪都不吃,只能烧掉。司务长倒不反对我的定义,但他说,反正是忆苦饭,这么讲究干什么,糠还要留着喂猪,所以往锅里倒了一筐碎稻壳。搅匀之后,真不知锅里是什么。做好了这锅东西,司务长高兴地吹起了口哨,但我的心情不大好。说实在的,我这辈子没怕过什么,那回也没有怕,只是心里有点慌。我喂过猪,知道拿这种东西去喂猪,所有的猪都会想要咬死我。猪是这样,人呢? 

后来的事情证明我是瞎操心。晚上吃忆苦饭,指导员带队,先唱“天上布满星”,然后开饭。有了这种气氛,同学们见了饭食没有活撕了我,只是有些愣头青对我怒目而视,时不常吼上一句:“你丫也吃!”结果我就吃了不少。第一口最难,吃上几口后满嘴都是麻的,也说不上有多难吃。只是那些碎稻壳像刀片一样,很难吞咽,吞多了嘴里就出了血。反正我已经抱定了必死的决心,自然没有闯不过去的关口。但别人却在偷偷地干呕。吃完以后,指导员做了总结,看样子他的情况不大好,所以也没多说。然后大家回去睡觉——但是事情当然还没完。大约是夜里十一点,我觉得肠胃搅痛,起床时,发现同屋几个人都在地上摸鞋。摸来摸去,谁也没有摸到,大家一起赤脚跑了出去,奔向厕所,在北回归线下皎洁的月色下,看到厕所门口排起了长队…… 

有件事需要说明,有些不文明的人有放野屎的习惯,我们那里的人却没有。这是因为屎有作肥料的价值,不能随便扔掉。但是那一夜不同,因为厕所里没有空位,大量这种宝贵的资源被抛撒在厕所后的小河边。干完这件不登大雅之事,我们本来该回去睡觉,但是走不了几步又想回来,所以我们索性坐在了小桥上,聊着天,挨着蚊子咬,时不常的到草丛里去一趟,直到肚子完全出清。到了第二天,我们队的人脸色都有点绿,下巴有点尖,走路也有点打晃。像这个样子当然不能下地,只好放一天假。这个故事应该有个寓意,我还没想出来。反正我不觉得这是在受教育,只觉得是折腾人——虽然它也是一种生活。总的来说,人要想受罪,实在很容易,在家里也可以拿头往门框上碰。既然痛苦是这样简便易寻,所以似乎用不着特别去体验。
 
\chapter{祝你平安}

我很少看MTV,但既看电视,总免不了会看见一些。最近看到了孙悦唱的MTV《祝你平安》,心里有些疑惑,想借《演艺圈》的园地,求教于高明。照我看来,那是一首安慰失意情人的歌,对此不当再有其它解释了。 

孙悦唱得很好,歌也很好听。但画面就让人有点看不懂,看懂的地方又让人有气。 

先说我不懂的地方。众所周知,MTV的画面不一定有逻辑,好看就成。但我不懂的是导演的创意。这本是支爱情歌,却串进了女教师和聋哑学生。虽然宏扬主旋律、歌颂人民教师是好的,但却不是这么宏扬法……我老婆没看过这段MTV,却会唱这歌,时不常对我来几句,以示柔情,但我总觉得她在说我是哑巴。这不就是搞串了吗?主旋律是主旋律,男女之情是男女之情,切不可这样胡串。再说,在片首孙悦打扮得像个小蜜,片中才出现了聋哑学生。假如不是我想像力过于丰富,这故事仿佛是说:有一聋哑学校的教师,丢下学生跑到深圳,傍上了大款。回过头来想想被丢下的学生,心中不忍——这歌就不叫《祝你平安》,该叫作《鳄鱼的眼泪》,因为真有良心就该回去教书。就算真有这样的事,也只是极个别的现象,没有普遍意义。 

再说说我自以为看懂了的地方。片子结束时,出现了一个交通警,微笑着做准许车辆通行的手势。照我浅薄的理解,这是对歌曲名称的简单图示:警察同志让你过去,同时一笑,此乃祝你平安之意也——用这种手法来点题。但愿我理解错,因为把别人想得如此低能是有罪的。有个英文歌《Crazy》,请这位导演来拍MTV,就要拍些疯子了,否则没法点题。《我的太阳》可以是这种拍法:请一漂亮女孩搂住陈佩斯。既有“我的”,又有“太阳”,太阳就是陈佩斯的脑袋。你肯定会同意,我的创意虽然直露,尚不如《祝你平安》的结尾那么直。要拍岳飞《满江红》,就得去请食人族,否则不能“饥餐胡虏肉”。按照这种自然主义的逻辑,麦当娜《Like a virgin》该请观众看点什么?难道要请大家当一回大夫,去看那个东西?我们又不是妇科大夫,看了也看不懂……
 
\chapter{我是哪一种女权主义者}

因为太太在作妇女研究.读了—批女权主义的理论书,我们常在—起讨论自己的立场。作为—个知识分子,我们不可避免地会有一种接近某种女权主义的立场。我总觉得,一个人不尊重女权,就不能叫作一个知识分子。但是女权主义的理论门类繁多(我认为这—点并不好),到底是哪一种就很重要了。 

社会主义女权主义者认为,性别之间的不平等是社会制度造成的,要靠社会制度的变革来消除。这种观点在西方带点阶段论的色彩.在中国就不一样了:众所周知,我国现在已是社会主义制度,党主张男女平等,政府重视妇女的社会保障,在这方面成就也不少。但恰恰在这种情况下,我们感到了社会主义女权理论的不足。举个例子来说,现在企业精简职工,很多女职工被迫下岗。假若你要指责企业经理,他就反问道:你何不问问,这些女职工自身的素质如何?像这样的题目报刊上讨论的已经很多了。很明显,一个人的生活不能单纯地依赖社会保障,还要靠自身的努力;而且一个人得到的社会保障越多,目身的努力往往就越少。正如其他女权主义门派指出的那样,社会主义女权主义向社会寻求保障的同时,也就承认了自己是弱者,这是一个不小的失策。在社会主义制度下,得到较多保障的人总是值得羡慕的——我年轻时,大家都羡慕国营企业的工人,因为他们最有保障。但保障和尊严是两回事。 

与此有关的问题是:我们国家的男女是否平等了,在这方面有—点争议。中国人自己以为,在这方面做得已经很不错;但是西方一些观察家不同意。我认为这不是一个问题,而是两个问题,头一个问题是:在我们的社会里,是否把男人和女人同等看待。这个问题有难以评论的性质:众所周知,一有需要,上面就可以规定各级政府里女干部的比例,各级人代会里女代表的比例,我还听说为了配合九五世妇会,出版社正在大出女作家的专辑。因为想把她们如何看待就可以如何看待,这件事就丧失了客观性,而且无法讨论。另一个问题是:在我们国家里,妇女的实际地位如何,她们自身的素质、成就、掌握的决策权,能不能和男性相比。这个问题很严肃,我的意见是:当然不能比。妇女差得很多——也许只有竞技体育例外,但竞技体育不说明什么。我们国家总是从社会主义女权理论的框架出发去关怀女性,分配给她各种东西,包括代表名额。我以为这种关怀是不够的。真正的成就是自己争取来的,而不是分配来的东西。 

西方还有—种激进的女权主义立场,认为女性比男性优越,女人天性热爱和平、关心生态,就是她们优越的证明。据说女人可以有比男人更强烈、持久的性高潮,也是一种优越的证明,我很怀疑这种证明的严肃性。虽然女人热爱自己的性别是值得赞美的,但也不可走火入魔。一个人在坐胎时就有男女之分,我以为这种差异本身是美好的。别人也许不同意,但我以为,见到一种差异,就以为这里有优劣之分,这是一种市侩心理——生为一个女人,好像占了很多便宜。当然,要按这个标准,中国人里市侩更多。他们死乞白赖地想要男孩,并且觉得这样能占到便宜。将来人类很可能只剩下一种性别——男或女。这时候的人知道过去人有性别之分,就会不胜痛惜,并且说:我们的祖先是些市侩。当然,在我们这里,有些女人有激进女权主义者的风貌,中国话叫作“气管炎”。我个人认为,“气管炎”不是中国女性风范的杰出代表。我总是从审美的角度、而不是从势利的角度来看世界,而且觉得自己个是个市侩——当然,这一点还要别人来评判。 

西方女权主义者认为,性之于女权主义理论,正如劳动之于马克思的理论一样重要。这个观点中国人看来很是意外。再过一些年,中国人就会体会到这种说法的含义,现在的潮流正把女人逐渐地往性这个圈子里套。性对于人来说,是很重要的。但是单方面地要求妇女,就很不平等。西方妇女以为自己在这个圈子里丧失了尊严,这是有道理的。但回过头去看看“文化革命”里,中国的妇女和男人除了头发长几寸,就没有了区别,尊严倒是有的,只可惜了无生趣。自由女权主义者认为,男人也该来取悦妇女,这样就恢复了妇女的尊严。假如你不同意这个观点,就要在毫无尊严和了无生趣里选一种了。作为男子,我宁愿自己多打扮,希望这样有助于妇女的尊严,也不愿看到妇女再变成一片蓝蚂蚁,当然,按激进女权的观点,这还远算不上有了弃暗投明的决小。真正有决心应该去作变性手术,起码把自己阉掉。 

我太太现在对后现代女权主义理论着了迷。这种理论总想对性别问题提供一种全新的解读方式。我很同意说,以往的人对性别问题理解得不对——亘古以来,人类在性和性别问题上就没有平常心,开头有点假模假式,后来就有点五迷三道,最后干脆是不三不四,或者是横蛮无理——这些错误主要是男人犯的——这是我对这个问题的看法,但和后现代女权理论没有丝毫的相近之处。那些哲学家、福科的女弟子们,她们对此有着一套远为复杂和深奥的解读方法。我正盼着从中学到一点东西,但还没有学会。 

作为一个男人,我同意自由女权主义,并且觉得这就够了。从这种认同里,我能获得一点平常心,并向其他男人推荐这种想法。我承认男人和女人很不同,但这种差异并不意味着别的:既不意味着某个性别的人比另一种性别的人优越,也不意味着某种性别的人比另一种性别的人高明。一个女孩子来到人世间,应该像男孩一样,有权利寻求她所要的一切。假如她所得到的正是她所要的,那就是最好的——假如我是她的父亲,我也别无所求了。
 
\chapter{承认的勇气}

我很少看电视。有一天偶然打开电视,想看看有没有球赛,谁知里面在演连续剧《年轮》,一对知青正在恋爱——此时想关上也不可能,因为我老婆在旁边,她就喜欢看人恋爱——当时是黑更半夜,一男一女在旷野中,四野无人,只见姑娘忽然惨呼一声,“我是可以教育好的子女”,投入情郎的怀抱。这个场面有点历史的真实性,但我还是觉得,这女孩子讲的话太过古怪了。既然是“子女”,又堪教育,我倒想问问,你今年几岁了。坦白地说,假如我是这位情郎,就要打“吹”的主层。同情归同情,我可不喜欢和糊涂人搞在一起。该剧的作者会为这位当年的姑娘辩护道:什么事情都要放到一定的历史背景下看,当年上面的精神说她是个子女,她就是个子女。这话虽然有道理,但不对我的胃口。我更希望听到这样的解释:这女孩本是个聪明人,只可惜当时正在犯傻;但是这样的解释是很少能听到的。知青文学的作者们总是这样来解释当年的事:这是时代使然,历史使然;好像出了这样的洋相,自己就没有责任了。 

我和同龄人一样,有过各种遭遇。有—阵子,我是黑五类(现在这名字是指黑芝麻、黑米,当时是指人)、后来则被发现需要再教育,就被置于广阔天地之中去滚一身泥巴,炼一颗红心。再后来回到城里,成了工人阶级,本来可以领导一切,但没发现领导了谁。再以后干辛万苦考上了大学,忽而慨然想到:现在总算是个臭老九了——以后的变化还多,就不一一列举。总而言之,人生在世,常常会落到一些“说法”之中。有些说法是不正确的,落到你的头上,你又拿它当了真,时过境迁之后,应该怎样看待自己,就是个严肃的问题。这件事让中国人一说大过复杂(我就是中国人、所以讲得这样复杂),美国人说起来简单:这不就是当了回傻X吗? 

傻X(asshole)这个词,多数美国人是给自己预备的。比方说,感觉自己道入愚弄时,就会说:我觉得自己当了傻X(I feel like an asshole)!心情不好时更会说:我正捉摸我是哪一种傻X。自己遭人愚弄,就坦然承认,那个X说来虽然不雅、但我总觉得这种达观的态度值得学习。相比之下,国人总不肯承认自己傻过,仿佛这样就能使自己显得聪明;除此之外,还要以审美的态度看待自己过去的丑态。像这种傻法,简直连X都不配作了。 

本文的目的是想谈谈我的心路历程。保这样说美国人的好话,有民族虚无主义之嫌,会使该历程的价值大减。其实我想要说的是,承认自己傻过。这是一种美德,而且这种美德并不是洋人教给我的。年轻时我没有这种美德,总觉得自已很聪明,而且永远很聪明,既不会一时糊涂,也不会受愚弄。就算身处逆境,也要高声吟道:天生我才必有用——也不怕风大闪了舌头。忽一日,到工厂里学徒,拜刘二为师,学模具钳工,顺便学会了这种美德。这种美德出于中国哲人的传授,又会使它价值大增。这位哲人长了一双牛一样的眼睛,胡子拉茬,穿着不大干净。我第一次见到他,就听见他在班组里高谈阔论道:我是傻X。对这个论断,刘师傅证明如下:师傅加师母、再加两位世兄,全靠师傅的工资养活,这工资是三十五块五,很不够用,想不出路子搞钱,所以他是傻X。假如你相信是你自己,而不是别人,该为家庭负责.就会相信这个结沦。同理,脑袋扛在肩上,是自己的.也该为它负责,假如自己表现得很傻,就该承认。假如这世上有人愚弄了我,我更是心服口服;既然你能耍了我,那就没什么说的——我是傻X、人生在世有如棋局,输一着就是当了回傻X,懂得这个才叫会下棋。假如我办了什么傻事被你撞见了,你叫我傻X,我是不会介意的。但我不会说别人是傻X,更不会建议别人也说自己是傻X,我知道这是个忌讳。 

我现在有了—种二十岁时没有的智慧。现在我心闲气定地坐在电脑面前写着文章,不会遇到任何人的愚弄,这种状态比年轻时强了很多。当时我被人塞了一脑子的教条,情绪又受到猛烈的煽动,只会干傻事,一件聪明事都办不出来。有了前后两种参照,就能大体上知道什么是对的。这就是我的智慧:有这种智慧也不配叫作智者,顶多叫个成年人。很不幸的是,好多同年人连这种智慧部没有,这就错过了在我们那个年代里能学会的唯一的智慧——知道自己受了愚弄。
 
\chapter{我对国产片的看法}

我很少出去看电影,近来在电影院看过的国产片子,大概只有《红粉》。在《红粉》这部片子里,一个嫖客,两个妓女,生离死别,演出多少悲壮的故事;看了让人起鸡皮疙瘩。由此回想起十多年前看过的一部国产片《庐山恋》,男女主人公在庐山上谈恋爱,狂呼滥喊:“I Iove my motherland...”有如董存瑞炸碉堡。不知别人怎么看,我的感觉是不够妥当。这种不妥当的片子多得不计其数,恕我不一一列举。  

作家纳博科夫曾说,一流的读者不是天生的,他是培养出来的。《庐山恋》还评上了奖,这大概是因为编导对观众的培养之功,但是这样的观众恐怕不能算是一流的。所以我们可以改改纳博科夫的话:三流的影视观众不是天生的。他也是培养出来的,作为欣赏者,我们开头都是二流水平,只有经过了培养,才会特别好或是特别坏。在坏的方面我可以举个例子,最近几年,中央台常演一些历史题材的连续剧,片子一上电视,编导就透过各种媒体说:这部片子的人物、情节。器具。歌舞,我们都是考证过的。我觉得这很没意思。可怪的是,每演这种电视片,报纸上就充满了观众来信,对人物年代做些烦琐考证,我也觉得挺没劲。似乎电视片的编导已经把观众都培养成了考据迷。当然,也有个把漏网之鱼,笔者就是其中之一。但就一般来说,影视的编导就是墨索里尼,总是有理。凭良心说,现在的情况不算坏。文化革命里人们只看八个样板戏,也没人说不好。在那些年月里,也培养了一批只会欣赏样板戏的观众。在现在年月里,也培养了一批只会考证的观众。说到国产片的现状,应该把编导对观众的培养考虑在内。  

作为一条漏网之鱼,我对电影电视有些不同的看法:我想从上面欣赏一些叫作艺术的东西。从这个意义上说,国产片的一些编导犯下了双重罪孽:其一。自己不妥当,其二、把观众也培养得不妥当。不过这种情况已经发生了变化:近年来,中国电影也取得了一些成就,有些片子还在国际上得了奖。我认为这些片子是好的,但也有一点疑问:怎么都这么惨咧咧。苦兮兮的?《霸王别姬》里剁下了一根手指头,《红高梁》里扒下了一张人皮。我们国家最好的导演,对人类的身体都充满了仇恨。单个艺术家有什么风格都可以,但说到群体,就该有另一种标准。打个比方来说,我以为英国文学是好的,自莎士比亚以降,名家辈出;内中有位哈代先生,写出的小说惨绝人寰——但他的小说也是好的。倘若英国作家自莎士比亚以降全是哈代的风格,那就该有另一种评价:英国文学是有毛病的。最近《辛德勒名单》大获成功,我听说有位大导演说:这正是我们的戏路!我们也可以拍这种表现民族苦难的片子。以我之见,按照我们的戏路,这种片子是拍不出来的。除非把活做到银幕之外,请影院工作人员扮成日本兵,手擎染血的假刺刀,随着剧情的进展,来捅我们的肚皮。当然,假如上演这样的片子,剧院外面该挂个牌子:为了下一代,孕妇免进。话虽如此说,我仍然以为张艺谋。陈凯歌不同凡响。不同凡响的证明就是:他们征服了外国的观众,而外国的观众还没有经过中国编导的培养。假如中国故事片真正走向了世界,情况还不知是怎样。  

莫泊桑曾说,提笔为文,就想到了读者。有些读者说:我感动吧…在中国,有些读者会说,请让我们受教育。我举这个例子,当然是想用莫泊桑和读者,来比喻影视编导与观众。敏感的读者肯定能发现其中的可笑之处:作品培养了观众的口味,观众的口味再来影响作者,像这样颠过来。倒过去,肯定是很没劲。特别是,假如编导不妥当,就会使观众不妥当;观众又要求编导不妥当,这样下去大家都越来越不妥当。作为前辈大师,莫泊桑当然知道这是个陷讲,所以他不往里面跳。他说:只有少数出类拔萃的读者才会要求,请凭着你的本心,写出真正好的东西来。他就为这些读者而写。我也想做一个出类拔萃的观众,所以也这样要求:请凭着你的本心去拍片——但是,别再扒人皮了,这样下去有点不妥当。对于已经不妥当的编导,就不知说些什么——也许,该说点题外之语。我在影视圈里也有个把朋友,知道拍片子难:上面要审本子审片,这是一;找钱难,这是二。还有三和四,就没必要一一列举,其中肯定有一条:观众水平低,不过,我不知该怪谁。这只是一时一地的困境,而艺术是永恒的。此时此地,讲这些就如疯话一般。但我偏还觉得自己是一本正经的。
 
\chapter{商业片与艺术片}

去年,好莱坞十部大片在中国上演,引起了一场不大不小的轰动。这类片子我在美国时看了不少,但我远不是个电影迷。初到美国时英文不好,看电影来学习英文——除了在电影院着,还租带子,在有线电视上看,前后看了大约也有上千部。片子看多了,就能分出好坏来;但我是个中国的知识分子,既不买好莱坞电影俗套的帐,也不吃美国文化那一套,评判电影另有一套标准。实际上,世界上所有的文化人评判美国电影、标准都和我差不多。用这个标准来看这十部大片,就是一些不错的商业片,谈不上好。美国电影里有一些真好的艺术片,可不是这个样了。 

作为一个文化入,我认为好莱坞商业片最让人倒胃之处是落俗套。五六十年代的电影来不来的张嘴就唱,抬腿就跳,唱的是没调的歌,跳的是狗撒尿式的踢蹋舞。我在好莱坞电影里看到男女丰人公一张嘴或一抬腿,马上浑身起鸡皮疙瘩,抖作一团;你可能没有同样的反应,那是因为没有我看得多。到了七十年代,西部片大行其道,无非是一个牛仔拔枪就打,全部情节就如我一位美国同学概括的:“Kill everybody”——把所有的人都杀了。等到观众看到牛仔、左轮手枪就讨厌,才换上现在最大的俗套,也就是我们正在看的:炸房子,摔汽车;一直要演到你一看到爆炸就起鸡皮疙瘩,才会换点别的。除了爆炸,还有很多别的俗套。说实在的,我真有点佩服美国片商炮制俗套时那种恬不知耻的劲头。举个例子,有部美国片子《洛基》,起初是部艺术片,讲一个穷移民,生活就如一潭死水——那叙事的风格就像怪腔怪调的布鲁斯,非常的地道。有个拳王挑对手,一下姚到他头上,这是因为他的名字叫“洛基”、在英文的意思里是“经揍”……这电影可能你已经看过了,怪七怪八的,很有点意思。我对它评价不低。假如只拍一集,它会给人留下很好的印象,别人也爱看。无奈有些傻瓜喜欢看电影里揍人的镜头,就有混帐片商把它一集集地拍了下去,除了揍人和挨揍, —点别的都没了。我离开美国时好像已经拍到了《洛基七》或者《洛基八》,弄到了这个地步,就不是电影,根本就是大粪。 

好莱坞商业片看多了,就会联想到《镜花缘》里的直肠国。那里的人消化功能差,一顿饭吃下去,从下面出来,还是一顿饭。为了避免浪费,只好再吃一遍(再次吃下去之前,可能会回回锅,加点香油、味精)。直到三遍五遍,饭不像饭而像粪时,才换上新饭。这个比方多少有点恶心,但我想不到更好的比方了。好莱坞的片商就是直肠国的厨师,美国观众就是直肠国的食客。顺便说一句,国产电影里也有俗套,而且我们早就看腻了……这个话题就到此为止,以免大家恶心。说句公道话,这十部大片有不少长处,特技很出色,演员也演得好,虽然说到头来,也就是些商业俗套,但中国观众才吃第一遍,感觉还很好;总得再看上一些才能觉得味道不对头。 

我说过,美国也有好的艺术片。比方说,沃伦·比提年轻时自己当制片、自己主演的片子就很好。其中有一部《赤色分子》,中国的观众就算没看过,大概也有耳闻。再比方说乌迪·艾伦的影片,从早年的《Banma》(傻瓜),到后来的《汉娜姐妹》,都很好。艺术片和商业片的区别就在于不是俗套。谁能说《末代皇帝》是俗套?谁能说《美国往事》是俗套?美国出产真正的艺术片并不少,只是与大量出产的商业片比,显得少一点而已。然而就是这少量的电影、才是美国电影真正生命之所在。美国搞电影的人自己都说,除了少量艺术精品,好莱坞生产垃圾。制造垃圾的理由是:垃圾能卖钱,精品不卖钱。《美国往事》、《末代皇帝》从筹划到拍成,都是好几年。要总是这样拍电影,片商只好去跳楼…… 

既然艺术片不赚钱,怎么美国人还在拍艺术片?这是最有意思的问题。我以为,没有好的艺术片,就没有好的商业片。好东西翻炒几道才成了俗套,文化垃圾恰恰是精品的碎片。要是投人搞真正的艺术电影,好莱坞现在肯定还在跳狗撒尿的踢蹋舞;让最鲁钝、最没品味的电影观众看了也大发疟疾。无论如何,真正的艺术才是世界上最好的东西。我对去年引进十部大片很赞成,因为前年这这样十部大片都没有。但我觉得自今年起,就该有点艺术片。除此之外,眼睛也别光盯着好莱坞。据我所知,美国一些独立制片人的片子相当好,欧洲的电影就更好。只看好莱坞商业片,是会把人看笨的。
 
\chapter{环境问题}

我生在北京城里。小时候,我爬到院里的高楼顶上——这座楼在西单——四下眺望、经常能看到颐和园的佛香阁。西单离颐和园起码有二十里地。几年前,我住在北大畅春园,离颐和园只行数里之遥,从窗户里看佛香阁,十次倒有八次看不见。北京的空气老是迷迷糊糊的,有点迷眼,又有点呛嗓子,我小时候不是这样。我已经长大了,变成了—条车轴汉了——这是指衬衣领子像车轴而言。在北京城里住,几乎每天都要换衬衣,在国外时,一件衬衣可以穿好几天。世界上有很多以污染闻名的城市:米兰、洛杉矾、伦敦等等,我都去过,只有墨西哥城例外。就我所见。北京城的情况在这些城市里也是坏的。 

但我对北京环境改善充满了信心。这是因为一座现代大都市,有能力很快改善环境,北京是首都,自然会首先改善。不信你到欧美的大城市看看,就会发现有些旧石头居于像瓦窑里面一样黑,而新的石头房子则像雪一样白。找个当地人问问,他们会说:老房子的黑是煤烟熏的,现在没有煤烟,石头墙就不会变黑了,我在美国的匹兹堡留过学,那里是美国的钢铁城市,以污染著称。据当地人说,大约三十年前,当地人出门访友时,要穿一件衬衣,带一件衬衣。身上穿的那件在路上就脏了,到了朋友家里再把带的那件换上。现在的情况是:那里的空气很干净。现代大城市有办法解决环境问题:有财力,也有这种技术。到了非解决不可时,自然就会解决。在这一天到来之前,我们可以戴风镜、带口罩来解决空气不好的问题。 

我现在住的地方在城乡结合部,出门不远,就不归办事处管,而是乡政府的地面。我家楼下是个农贸市场,成天来往着一些砰砰乱响的东西:手扶拖拉机、小四轮、农用汽车等等。这些交通工具有—个共同点:全装着吼声震天、黑烟滚滚的柴油机。因为有这种机器,我认为城市近郊、小城镇等地环境问题更严重。人家总说城市里噪音严重,但你若到郊区的公路边坐上一天,回来大概已经半聋了。县城的城关大多也吵得要命、上那里逛逛、回来时鼻孔里准是黑的。据报道,我国的农用汽车产值超过了正装汽车。叫作农用车,其实它们净往城市和郊区跑。这类地力人烟稠密,和市中心差不个很多。这里的人既有鼻子,又有耳朵,因此造这种车时,工艺也宜考究些,要把环境因素考虑在内才好,否则是用不了几年的。 

在这方面我有—个例子;七四年我在山东烟台一带插队, 见到现在农用车的鼻祖:它是大车改制的,大车已经有两个轮子,在车辕部位装上个转盘,安上抽水磨面的柴油机,下回装上第二个轮子、用三角皮带带动,驾驶员坐在辕上,转弯时推动转盘,连柒油机带底下的轮子一块转。我不知它的正式名称叫什么,只知道它的雅号叫作“宁死不屈”,因为在转急弯时,它会把头一扭、把驾驶员扔下车去,然后就头在后,屁股在前, —路猛冲过去,此时用手枪、冲锋枪去打都不能让它停住,拿火箭筒来打它又来不及,所以叫宁死不屈。当然,最后它多半是冲进路边的店铺,撞在柜台上不动了。但那台肇事的柴油机还在恬不知耻地吼叫着。后来,它被政府部门坚决取缔了。不安全只是原因之一,主要的原因是:它对环境的影响是毁灭性的。那东西吵得厉害,简直是天理难容。跑在烟台二马路上,两边的人都要犯心脏病。发展农用汽车,也要以宁死不屈为鉴。 

说到环境问题,好多人以为这是近代机器文明造成的,其实大谬不然。说到底,环境问题是人的问题。煤烟、柴油机是糟糕,但也是人愿意忍受它。到了下愿忍受时,自然会想出办法来。老北京是座消费城市,虽然没有什么机器,环境也不怎么样:晴天三尺土,雨天一街泥。我从书上看到,旧北京所有的死胡同底部,山墙底下都是尿窝子,过住行人就在那里撒尿。日久天长,山墙另—面就会长出白色的晶体,成分是硝酸铵,经加工可以做鞭炮。有些大妈还用这种东西当盐来炖肉,说用硝来炖肉能炖烂——但这种肉我是不肯吃的。有人说,喝尿可以治百病,但我没有这种嗜好。我宁可得些病。很不幸的是,这些又骚又潮的房子里还要住人,大概不会舒适。天没下雨,听见自己家墙外老是哗哗的,心情也不会好。费孝通先生有篇文章谈“差序格局”,讲到二三十年代江南市镇,满河飘着垃圾,这种环境也个能说是好。我住的地方不远处,有片乱七八槽的小胡同,是外来人口聚集区。有时从那里经过,到处是垃圾。污水到处流,苍蝇到处飞。排水口的筛子上净是粪——根本不成个世界。有一大群人住在一起,只管糟蹋不管收拾,所以就成了这样——此类环境问题源远流长,也没听准说过什么。 

就我所见,一切环境问题都是这么形成的:工业不会造成环境问题,农业也不会造成环境问题,环境问题是人造成的。知识分子悲天悯人的哀号解决不了环境问题,开大会、大游行、全民总动员也解决不了这问题。只要知道一件事就可以解决环境问题:人不能只管糟蹋不管收拾。收拾—下环境就好了,在其中生活也能做个体面人。
 
\chapter{文化的园地}

我在布鲁塞尔等飞机,等“人民快航”。现在的人大概记不得人民快航(People's Express)了,十年前它在美国却是大名鼎鼎,因为它提供最便宜的机票,其国内航班的票比长途车票还要便宜;其国际航班肯定要比搭货船过海便宜——就算你搭得到,在舶上也要吃东西。这笔开销也不小——我乘它到了欧洲,还要乘它回去。很遗憾的是,这家航空公司倒掉了。盛夏时节,欧洲到处是蓝色的人流。大家穿着蓝色的牛仔裤,背着蓝色的帆布包,包上搭着一条小凉席,走到哪儿睡到哪儿,横躺竖卧,弄得候车室、候机厅都像停尸房一样。现在的北京街头也能看到这些人:头发晒得褪了色,脸上晒出了一脸的雀斑,额头晒得红彤彤的,手里拿着旅游地图认着路;只是形不成人流。但我是在这个人流里游遍了欧洲。 

穷人需要便宜的食宿和交通,学生是穷人个最趾高气扬的一种:虽然穷,但前程远大。当时我就是个学生,所以兴高采烈地研究学生旅游书里那些省钱的法子:从纽约市中心前往肯尼迪国际机场,有直通的机场bus,但那本书却建议你乘地铁前往昆士区的北端,再坐昆士区的公共汽车南下。这条路线在地图上像希腊字母欧米伽。那本书这样解释这个欧米伽:要尽量利用城市的公共交通,这种文通工具在世界上任何地方都是便宜的。书上还教你填饱肚子的诀窍:在纽约,可以走进一家中餐馆。要一碗白饭,用桌上的酱油下饭;在巴黎,你可以前往某教堂门口,那里有舍给穷人喝的粥。在布自塞尔,这个诀窍是在下午五点以后前往著名的餐饮城City2,稍微给—点钱,甚于不给钱,就可以把卖剩下来的薯条都包下来。这种薯条又凉、又面,但还可以填饱肚子。这些招儿我没有用过,就是用了也不觉得害臊:我是学生嘛。 

我到布鲁塞尔时,已是初秋。这个季节北欧上空已是一片阴风惨雾,不宜久留,该干啥快去干啥,所以我在机场等飞机。忽然间肠胃轰鸣,那本旅游书上又没有在布鲁塞尔机场找便宜厕所的指导,我只好进了收费厕所。这地方进门要一个美元,合四十比利时法郎,在我印象中,这是全世界的最高价。走进格间,把门一关,门上一则留言深得我心:啊,我的心都碎了……看来是个愤世嫉俗的美国小伙子留在这儿的。他心碎的原因有二:一是被人宰了一刀;二是把自己的问题估计得严重了。至于我,虽然问题是严重的,必须立即解决,不能带上飞机,但也觉得收一美元实在太多。但仔细一看,不禁冷汗直冒:这行字被人批得落花流水——周围密密麻麻用各种字体写着:没水平——没觉悟——层次太低。这行字层次低,却引起了我的共鸣,我的层次也高不了…… 

我在布鲁塞尔等飞机。去了一趟收费厕所,不想走进了一个文化的园地。假如我说,我在那里看到了人文精神的讨论,你肯定不相信。但国外也有高层次的问题:种族问题、环境问题、“让世界充满爱”,还有“I have a dream today”,四壁上写得满满的,这使我冷汗直冒,正襟危坐——坐在马桶上。我相信,有人在这里提到了“终极关怀”。但一定是用德文写的。那地方德文的题字不少,我看不懂。大概还有人提到了后现代,但我也看不懂:那一定是用法文写的,我又不懂法文。那里还有些反着写的问号,不知写些什么。中文却没有,大概是因为该园地收费太贵,同胞们不肯进来——我是个例外。我住了一家学生旅馆,提供免费的早餐:面包片和人造黄油,我把黄油涂得比面包片还要厚,所以跑到这里来了。用英文写出的,大多是些虽很重要,但比较浅薄的问题。比方说,有位先生写道:保护环境。后面就有人批了一句:既然要保护环境,就不要乱写。再以后,又有一句批语:你也在乱写。我很想给他也批上一句:还有你;但又怕别人再来批我。像这样批下去,整个世界都会被字迹批满,所有的环境都要完蛋。还有不少先生提出,要禁止核武器。当时冷战尚未结束、两个核大国在对峙之中。万一哪天走了火,大家都要完蛋。我当然反对这种局面。我只是怀疑坐在马桶上去反对,到底有没有效力。 

布鲁塞尔的那个厕所,又是个世界性的正义论坛。很多留言要求打倒一批独裁者,从原则上说,我都支持。但我不知要打倒些准:要是用中文来写,这些名字可能能认出个把来,英文则一个都不认识。还有些人要求解放一些国家和地区,我都赞成,但我也不知道这些地方在哪里。除此之外,我还不知道我,一个坐在马桶上的人,此时可以为他们做些什么。这些留言都用了祈使句式,主要是促成做一些事的动机——这当然是好的,但这些事到底是什么、怎样来做、由谁来做,通通没有说明。这就如我们的文化园地,总有人在呼吁着。呼吁很重要,但最好说说倒底要干些什么。在那个小隔间里,有句话我最同意,它写在“解放萨尔瓦多”后面:要解放,就回去战斗吧。由此我想到:做成一件事,需要比呼吁更大的勇气和努力。要是你有这些勇气和精力,不妨动手去做。要是没这份勇气和精力,不如闭上嘴,省点唾沫,使厕所的墙壁保待清洁。当然,我还想到了,不管要做什么,都必须首先离开屁股下的马桶圈。这很重要。要是没想到这一点,就会误掉班机。
 
\chapter{论战与道德}

知识分子搞学问,除了闭门造车之外,与人讨论问题也常常是免不了得。在讨论是应该取何种态度,是个蛮有意义的问题。在这方面我有些见闻,虽然还不够广博,但已足够有趣。先父是位逻辑学家,在五十年代曾参加过“逻辑问题大讨论”,所以我虽然对逻辑所知不多,也把当年德论文集找出来细读了一番。对于当年德论争各方谁对谁错,我没有什么意见,但是对论战的态度却很有看法。众所周知,逻辑是一门严谨的科学,只要能争出个对错即可;可实际情况却不是那样,论战的双方都在努力证明对方是“资产阶级”,持有“唯心主义”或“形而上学”得思想方法。相形之下,自己是无产阶级,持有辩证唯物主义的思想方法。在我看来,逻辑问题是对错真伪的问题,扯上这么多,实属多余;而且在五十年代被判定位一名资产阶级分子之后,一个人的生活肯定不是很愉快的,此种论战的方式有恫吓、威胁之意。一般认为,五十年代的逻辑大讨论还算是一次比较平和的讨论,论战各方都没有因为论点前往北大荒;这是必须肯定的。但要说大家表现了多少君子风度,恐怕就说不上了。 

我们这个社会里的论战大多要从平等等讨论转为一方对另一方的批判,这是因讨论的方式决定的;根据我的观察,这些讨论里不是争谁对谁错,而是争谁好谁坏。一旦争出了结果,一方的好人身份既定,另一方是坏蛋就昭然若揭;好人方对坏蛋放当然还有些话要说,不但要批判,还要揭发。根据文献,反右斗争后期,主要是研究右派分子在旧社会的作为,女右派结交男朋友的方式,男右派偷窥女浴室的问题。当然,这个阶段发生的事已经不属于讨论的范畴,但还属论战的延续。再以后就是组织处理等等,更不属于讨论的范围;但是它和讨论有异常显著的因果关系。 

“文化革命”里,我是个小孩子,我住的地方有两派,他们中间的争论不管有没有意义,毕竟是一种争论。我记得有一阵子两派的广播都在朗诵毛主席的光辉著作《将革命进行到底》。倘若你因为双方都在表示自己将革命进行到底的决心,那就错了。大家感兴趣的只是该文中毛主席痛斥反对派是毒蛇的那一段——化成美女的蛇和露出毒牙的蛇,它们虽然已经感到冬天的威胁,但还没有冻僵呢——朗诵这篇文章,当然是希望对方领会到自己是条毒蛇这一事实,并且感到不寒而栗。据我所见,这个希望落空了。后来双方都朗诵另一篇光辉著作《敦促杜聿明等投降书》,这显然是把对方看成了反动派,准备接受他们的投降,但是对方又没有这种自觉性。最后灯结果当然是刀兵相见,打了起来。这以后的事虽然有趣,但已出了本文的范围。 

“文化革命”里的两派之争,有一个阶段,虽不属论战,但也非常有趣,那就是两派都想证明对方成份不纯或者道德败坏;要么发现对方庇护了大叛徒,走资派;要么逮住他们干了有亏德行的事。在后一个方面,只要有某派的一对青年男女呆在一个屋子里,对立面必派出一支精悍队伍埋伏在外面,觉得里面火候差不多了,就踹门进去。我住的地方知识分子成堆,而这些事又都是知识分子所为。从表面上看,双方都是斯文人,其实凶蛮得很。这使我感到,仅用言辞来证明自己比对方道德优越,实在是不容易的事;因此有时侯人们的确很难抑制自己的行动欲望。 

现在,任何一个有理智的人都不会认为,讨论问题的正当方式是把对方说成反动派,毒蛇,并且设法去捉他们的奸;然而,假如是有关谁好谁坏的争论,假如不是因外力而中止,就会得到这种结果。因为你觉得自己是好的,对方式坏的;而对方持有相反的看法,每一句辩驳都会加深恶意。恶意到了一定程度,就会诉诸行动:假设你有权力,就给对方组织处理;有武力,就让对方头破血流;什么都没有的也会恫吓检举。一般来说,真理是越辩越明,但以这种方式争论,总是越辩越不明,而且你在哪个领域争论,哪个领域就遭到损害。而且争论的结果既然是有人好,有人坏;那么好人该有好报,坏人该有坏下场,当然是不言自明。前苏联曾在遗传学方面展开了这种争论,给生物学和生物学家带来了很大的损害。我国在文化领域里有过好多次这种争论,得到了什么结果,也很容易看出来。 

现在我已是个中年人,我们社会里新的轰轰烈烈的文化事件也很少发生了,但我发现人们的论战方式并没有大的改变,还是要争谁好谁坏。很难听的话是不说了,骂人也可以不带脏字。现在最大规模的文化事件就是上演了一部新的电视剧或是电影,到底该为此表示悲哀,还是为之庆幸,我还拿不准;但是围绕着这种文化事件发生的争论之中,还有让人大吃一惊的言论。举例来说,前不久上演了一部电视剧《唐明皇》,有一部分人说不好看,剧组的成员和一部分记者就开了个研讨会,会议纪要登在《中国电视报》上。我记得制片人的发言探讨了反对《唐》剧者的民族精神、国学修为、道德水准诸方面,甚至认为那些朋友的智商都不高;唯一令人庆幸的是,还没有探讨那些朋友的先人祖宗。从此之后,我再不敢去看任何一部国产电视剧,我怕我白发苍苍的老母亲忽然知道自己生了个傻儿子而伤心——因为学习成绩好,我妈一直以为我很聪明。去看电影,尤其是国产电影,也有类似的危险;这种危险表现在两个方面:看了好电影不觉得好,你就不够好;看了坏电影不觉得坏,你就成了坏蛋。有一些电影在国际上得了奖,我看了以后也觉得不坏,但有些评论者说,这些电影简直是在卖国,如此说来,我也有背叛祖国的情绪了——谁感拿自己的人品去冒这个风险? 

我现在既不看国产电影,也不看国产电视剧,而且不看中国当代作家的小说。比方说,贾平凹先生的《废都》,我就坚决不看,生怕看了以后会喜欢----虽然我在性道德上是无懈可击的,但我深知,不是每个人都像我老婆那样了解我。事实上,你只要关心文化领域的事,就可能介入了论战的某一方,自身也不得清白,这种事最好还是避免。假如人人都像我这样,我国的文化事业前景堪虞,不过我也管不了这么多。不管影视也好,文学也罢,倘若属于艺术的范畴,人就可以放心大胆地去欣赏,至不济落个欣赏水平低的评价;一扯到道德问题,就让人裹足不前了。这种怯懦并不是因为我们不重视道德问题,而恰恰似因为我们很重视道德问题。假如我干了不道德的事,我乐于受到指责,并且负起责任;但这种不道德决不能是喜欢或不喜欢某个电影。 

假如我不看电影,不看小说,还可以关心一下正经学问读点理论文章、学术论文。文科的文章往往要说,作者以马列主义为指南,以辩证唯物主义为指导思想,为了什么什么等等。一篇文章我往往只敢看到这里,因为我害怕看完后不能同意作者的观点,就要冒反对马列主义的危险。诚然,我可以努力证明作者口称赞同马列主义,实质上在反对马列,但我又于心不忍,我和任何人都没有这么大的仇恨。 

其实,不光是理论文章,就是电视剧、小说作者也会把自己的动机神圣化;然后把自己的作品神圣化,最后把自己也神圣化;这样一来,他就像天兄下凡时的杨秀清。我对这些人原本有一些敬意,直到去年秋天在北方一小城市里遇到了一批刷猴子的人。他们也用杨秀清的口吻说:为了繁荣社会主义文化,满足大家的精神需求,等等,现在给大家耍场猴戏。我听了以后几乎要气死——猴戏我当然没看。我怕看到猴子翻跟头不喜欢,就背上了反对繁荣社会主义文化的罪名;而且我也希望有人把这些顺嘴就圣化自己的人管一管——电影、电视、小说、理论文章都可以强我喜欢(只要你不强我去看,我可以喜欢),连猴戏也要强我喜欢,实在太过分了——我最讨厌的动物就是猴子,尤其是见不得它做鬼脸。 

现在有很多文人下了海,不再从事文化事业。不管在商界、产业界还是科技界,人们以聪明才智、辛勤劳动来进行竞争。唯独在文化界,赌的是人品、爱国心、羞耻心。照我看来,这有点像赌命,甚至比赌命还严重。这种危险的游戏有何奖品?只是一点小小的文名。所以,你不要怪文人下海。 

假设文化领域里的一切论争都是道德之争、神圣之争,那么争论的结果就该是出人命,重大的论争就该有重大的结果,但这实在令人伤心——一些人不道德、没廉耻,还那么正常地活着,正如孟子所说:无耻无耻,无耻矣!我实在不敢相信,文化界还有这么多二皮脸之人。除了这两种结果,还有第三种结果,那就是大家急赤白脸的争论道德、廉耻,争完了就忘了;这就是说,从起头上就没有把廉耻当廉耻,道德当道德。像这样的道德标准,绝不是像我这样的人能接受的。 

我认为像我这样的人不在少数:我们热爱艺术、热爱科学,认为它们是崇高的事业,但是不希望这些领域里的事同我为人处事的态度、我对别人的责任、我的爱憎感情发生关系,更不愿因此触犯社会的禁忌。这是因为,这两个方面不在一个论域里,而且后一个论域比前者要严重。打个比方,我像本世纪初年的一个爪哇土著人,此种人生来勇敢、不畏惧战争;但是更重视清洁。换言之,生死和清洁两个领域里,他们更看重后者;因为这个原故,他们敢于面对枪林弹雨猛冲,却不敢朝着秽物冲杀。荷兰殖民军和他们作战时,就把屎撅子劈面掷去,使他们望风而逃。当我和别人讨论文化问题时,我以为自己的审美情趣、文化修养在经受挑战,这方面的反对意见就如飞来的子弹,不能使我惧怕;而道德方面的非难就如飞来的粪便那样使我胆寒。我的意思当然不是说现在文化的领域是个屎撅纷飞的场所,臭气熏天——决不是的;我只是说,它还有让我胆寒的气味。所以,假如有人以这种态度论争,我要做得第一件事,就是逃到安全距离之外,然后在好言相劝:算了罢,何必呢?
 
\chapter{皇帝做习题}

明末清初,有批洋人传道士来到中国,后来在朝廷里作了官。其中有人留下了一本日记,后来在中国出版了。里面记载了一些有趣的事,包括他们怎么给中国皇帝讲解欧氏几何学:首先,传教士呈上课本、绘图和测绘的仪器,然后给皇上进讲一些定理,最后还给皇上留了几道习题。等到下一讲,首先讲解上次的习题——《张诚日记》里就是这么记载的,但这些题皇上做了没有,就没有记载。我猜他是做了的:人家给你出了题目,会不会的总要试一试。假如不是皇上不是这样的人,也不会请人来讲几何学。这样一猜之后,我对这位皇上马上就有了亲近之感:他和我有共同的经历,虽然他是个鞑子,又是皇帝,但我还是觉得他比古代汉族的读书人亲近。孔孟程朱就不必说了,康梁也好,张之洞也罢,隔我们都远得很。我们没有死背过《三字经》《四书》,他们没有挖空心思去解过一道几何题。虽然近代中国有些读书人有点新思想,提出新口号曰:“中学为体,西学为用”;但我恐怕什么叫作“西学”,还是鞑子皇帝知道得更多些。 

我相信,读者诸君里有不少解过几何题。解几何题和干别的事不同,要是解对了,自己能够知道,而且会很高兴。要是解得不对,自己也知道没解出来,而且会郁郁寡欢。一个人解对了一道几何题,他的智慧就取得了一点实在的成就,虽然这种成就可能是微不足道的,但对于个人来说,这些成就绝不会是毫无意义。比尔·盖兹可能没解过几何题,他小时候在忙另一件事:鼓捣计算机。《未来之路》里说,他读书的中学里有台小型计算机,但它名不符实,是个像供电用的变压器式的大家伙。有些家长凑钱买下一点机时给孩子们用,所以他有机会接触这台机器,然后就对它着了迷。据他说,计算机有种奇妙之处:你编的程序正确,它绝不会说你错。你编的程序有误,它也绝不会说你对——当然,这台机器必须是好的,要是台坏机器就没有这种好处了。如你所知,给计算机编程和解几何题有共通之处:对了马上能知道对,错了也马上知道错,干干脆脆。你用不着像孟夫子那样,养吾浩然正气,然后觉得自己事事都对。当然,不能说西学都是这样的,但是有些学问的确有这种好处,所以就能成事。成了事就让人羡慕,所以就想以自己为体去用人家——我总觉得这是单相思。学过两天理科的人都知道这不对,但谁都不敢讲。这道理很明白:以其昏昏,使人昭昭,这怎么成呢。 

历史不是我的本行,但它是我胡思乱想的领域——谁都知道近代中国少了一次变法。但我总觉得康梁也好,六君子也罢,倡导变法够分量,真要领导着把法变成,恐怕还是不行的。要建成一个近代国家,有很多技术性的工作要做,迂夫子是做不来的。要是康熙皇帝来领导,希望还大些——当然,这是假设皇上做过习题。
 
\chapter{门前空地}

十年前我在美国,每天早上都要起来跑步,跑过我住的那条街。这条街上满是旧房子,住户一半是学生,另一半是老年人。它的房基高于街道,这就是说,要走上高台阶才到房门口。从房子到人行道,有短短的一道漫坡。这地方只能弄个花坛,不能派别的用场——这就是这条街的有趣之处。这条街上有各民族的住户,比方说,街口住的似是英裔美国人,花坛弄得就很像样子。因为这片空地是漫坡,所以要有护墙,他的护墙是涂了焦油的木材筑成,垒得颇有乡村气氛。花坛里铺了一层木屑,假装是林间空地。中央种了两棵很高的水杉,但也可能是罗汉松——那树的模样介于这两种树之间,我对树木甚是外行,弄不清是什么树。一般来说,美国人喜欢在门前弄片草坪,但是草坪要剪要浇,还挺费事的;种树省心,半年不浇也不会死。 

我们门前也是草坪,但里面寄宿的学生,谁也不去理它,结果长出耐旱的篙子和茅草来,时常长到一人多高。再高时,邻居就打电话来抱怨说这些乱草招蚊子,我们则打电话叫来房东,他用广东话嘟嚷着,骂老美多事,把那些杂草砍倒。久而久之,我们门前就出现个干草垛。然后邻居又抱怨说会失火,然后房东只好来把这些干章运走,上述两栋房子里的人都不想伺候花草,却有这样不同的处理方法。但我们门前比较难看,这是不言而喻的。 

我们左面住了一家意人利人。男主人黝黑黝黑、长了一头银发,遇上我跑步回来,总要拉着我嘀咕一阵,说他要把花坛好好弄弄。照我看,这花坛还不坏。只是砖护墙有些裂缝,里面的土质也不够好,花草都半死不活。这位老先生画了图给我看,那张图画得太过规范,叫我怀疑他是土木工[程师出身。其实他不是,他原来是卖比萨饼的。这件半他筹划来筹划去,迟迟不能开工。 

在街尾处,住了一对中国来的老夫妇,每次我路过,都看到他们在修理花园,有时在砌墙。有时在掘土,使用的工具包括了儿童掘土的玩具铲以及各种报废的厨具,有—回我看到老太大在给老头砌的砖墙匀缝,所用的家什是根筷子。总而言之,他们一直在干活,从来就没停过手。门前的护墙就这么砌了出来,像个弥陀佛,鼓着大肚子。来往行人都躲着走,怕那墙会倒下来.把自己压在下面。他们在花园里摆了几块歪歪扭扭的石头,假装是太湖石。但我很怕这些石头会把老两口绊倒。把他们的门牙磕掉……后来,他们把门廊油得红红绿绿,十分恶俗,还挂上了块破木板钉成的匾,上面写了三个歪歪倒倒的字“蓬莱阁”。我不知蓬莱仙阁是什么样了,所以没有意见,但海上的八仙可能会有不同意见…… 

关于怎样利用门前空地,中国人有各种各样的想法。其中之一是在角落里拦出个茅坑,捞点粪,种菜园子。小时候我住在机关大院的平房里,邻居一位大师傅就是如此行事。他还用废油毡、废铁板在门前造了—间难以言状的古怪房子,用稻草绳子、朽烂的木片等等给自己拦出片领地来,和不计其数的苍蝇快乐地共同生活。据我所见,招来的几乎全是绿荧荧的苍蝇,黑麻蝇很少来,由此可以推断出,同是苍蝇,黑麻蝇比较爱清洁,层次较高;绿豆蝇比较脏、层次也低些。假如这位师傅在美国这样干,有被拉到衔角就地正法的危险。现在我母亲楼下住了另一位帅傅,他在门前堆满了拣来的易拉罐和废纸板,准备去卖钱。他还嫌废纸板不压秤、老在上面浇水,然后那些纸板就发出可怕的味道来,和哈喇的臭咸鱼极为相似。这位老大爷在美国会被关进疯人院——因为他—点都不穷,还要攒这些破烂。每天早上,他去搜索垃圾堆,然后出摊卖早点。我认为,假如你想吃街头的早点,最好先到摊主家里看看……我提起这些事,是想要说明:门前空地虽是你自己的,但在别人的视线之中。你觉得自己是个什么人,就怎么弄好了。 

后来,我的意大利邻居终于规划好了一切,开始造他的花坛。那天早上来了很多黑头发的白种男人,在人行道上大讲意大利语。他们从一辆卡车上卸下一大堆混凝土砌块来,打着嘟噜对行人说”sorry”,因为挡了别人走路。说来你也许不信,他们还带来几样测绘仪器,在那里找水平面呢。总共五米见方的地面,还非弄得横平竖直不可。然后,铺上了袋装腐植土,种了一园子玫瑰花。路过的人总禁不住站下来看,但这是以后的事。花坛刚造好时,是座庄严的四方形建筑。是一本正经建造的,不是胡乱堆的。过往的行人看到,就知道屋主人虽然老了,但也不是苟活在世上。

\chapter{弗洛依德}

我说过,以后写杂文要斯文一些,引经据典。今天要引的经典是弗洛依德。他老人家说过:从某种意义上说,我们每个人都有点歇斯底里——这真是至理名言!所谓歇斯底里,就是按不下心头一股无明火,行为失范。谁都有这种时候,但自打十年前我把弗洛依德全集通读了一遍之后,自觉脾气好多了。古人有首咏雪的打油诗曰:夜来北风寒,老天大吐痰。一轮红日出,便是止痰丸——有些人的痰气简直比雪天的老天爷还大。谁能当这枚止痰丸呢?只有弗洛依德。 

年轻时,我在街道工厂当工人。有位师傅常跑到班长那里去说病了,要请假。班长问他有何症状,他说他看天是蓝色,看地是土色,蹲在厕所里任什么都不想吃。当然,他是在装骚鞑子。看天土色看地蓝色,蹲在臭烘烘茅坑上食欲大开,那才叫作有病——在这些小问题上,很容易取得共识,但大问题就很难说了。举例来说,法国人在马赛曲里唱道:不自由毋宁死;这话有人是不同意的。不信你就找本辜鸿铭的书来看看,里面大谈所谓良民宗教,简直就是在高唱:若自由毋宁死。《独立宣言》里说:我们认为,人人生而平等。这话是讲给英国皇上听的,表明了平民的尊严。这话孟夫子一定反对,他说过:无君无父,是禽兽也——这又简直是宣布说,平民不该有自己的尊严。总而言之,个人的体面与尊严,平等、自由等等概念,中国的传统文化里是没有的,有的全是些相反的东西。我是很爱国的,这体现在:我希望伏尔泰、杰弗逊的文章能归到辜鸿铭的名下;而把辜鸿铭的文章栽给洋鬼子。假如这是事实的话,我会感到幸福得多。 

有时候我想:假如大跃进、文化革命这些事,不是发生在中国,而是发生在外国,该有多好。这些想法很不体面,但还不能说是有痰。有些坏事发生在了中国,我们就说它好,有些鬼话是中国人说的,我们就说它有理,这种作法就叫作有痰气。有些年轻人把这些有痰气的想法写成书,他本人倒不见得是真有痰,不过是哗众取宠罢了。一种普遍存在的事态比这要命得多。举例来说,很多中年人因为文革中上山下乡虚耗了青春,这本是种巨大的痛苦;但他们却觉得很幸福,还说:青春无悔!再比方说,古往今来的中国人总在权势面前屈膝,毁掉了自己的尊严,也毁掉了自己的聪明才智。这本是种痛苦,但又有人说:这很幸福!久而久之,搞到了是非难辨,香臭不知的地步……这就是我们嗓子里噎着的痰。扯完了这些,就可以来谈谈我的典故典故。 

众所周知,有一种人,起码是在表面上,不喜欢快乐,而喜欢痛苦;不喜欢体面和尊严,喜欢奴役与屈辱,这就是受虐狂。弗洛依德对受虐狂的成因有这样一种解释:人若落入一种无法摆脱的痛苦之中,到了难以承受的地步,就会把这种痛苦看作是幸福,用这种方式来寻求解脱——这样一来,他的价值观就被逆转过来了。当然,这种过程因人而异。有些人是不会被逆转的。比方说我吧,在痛苦的重压下,会有些不体面的想法,但还不会被逆转。另有一些人不仅被逆转,而且还有了痰气,一听到别人说自由、体面、尊严等等是好的,马上就怒火万丈——这就有点不对头了,世界上哪有这样气焰万丈的受虐狂?——你就是真有这种毛病,也不要这个样子嘛。

\chapter{有关“伟大一族”}

有位老同学从美国回来探家,我们俩有七八年没见了。他的情况还不错:虽然薪水不很多,但两口子都挣钱,所以还算宽裕。自从美国一别,他的房子买到了第三所,汽车换到了第四辆,至于PC机,只要听说新出来一种更快的,他马上就去买一台,手上过了多少就没了数了。老婆还没有换,也没有这种打算,这正是我喜欢他的地方。虽然没坐过罗尔斯·罗伊斯,没住过棕榈海滩的豪华别墅,手里没有巨额股票,倒有一屁股的饥荒,但就像东北人说的,他起码也“造”了个痛快。我现在房无一间地无一陇,当然只有羡慕的份儿。但我们见面不是光聊这些——这就太过庸俗了。 

我们哥俩都闯荡过四方,种过地,放过牧,当过工人,二十年前在大学里同窗时,心里都曾燃烧起雄心壮志,要开创伟大的事业。所谓伟大的事业,就是要让自己的梦想成真。那时想了些什么,现在我都不好意思说,只好拿别人做例子。比方说微软公司的大老板比尔·盖茨,年轻时想过:要把当时看着不起眼的微处理机做成一种能用的计算机,让人人都能拥有和使用计算机,这样,科学的时代就真正降临人世了——这种梦想的伟大之处就在这里。现在这种梦想在很大程度上变成了真实,他在其中有很大的贡献,这是值得佩服的。至于他在商业上的成功,照我看还不太值得佩服。还有一个例子是马丁·路德·金曾经高呼,“我有一个梦想”,今天在美国的校园里,有时能看到高大英俊的黑人小伙子和白人姑娘拥抱在一起。从这种特别美丽的景象里,可以体会到金博士梦想的伟大。时至今日,我说多了没有意思,脸上也发热。我只能说,像这样的梦想我们也曾有过。 

每个人都有自己的梦想,这些梦想不见得都是伟大事业的起点。鲁迅先生的杂文里提到有这样的人,他梦想的最高境界是在雪天,呕上半口血,由丫环扶着,懒懒地到院子里去看梅花。我看了以后着实生气:人怎么能想这样的事!同时我还想:假如这位先生不那么考究,不要下雪、梅花、丫环搀着等等,光要呕血的话,这件事我倒能帮上忙。那时我是个小伙子,胳臂很有劲儿,拳头也够硬。现在当然不想帮这种忙,过了那个年龄。现在偶尔照照镜子,里面那个人满脸皱纹,我不大认识。走在街上,迎面过来一个庞然大物,仔细从眉眼上辨认,居然是自己当年的梦中情人,于是不免倒吸一口凉气。凉气吸多了就会忘事,所以要赶紧把要说的事说清楚。梦想虽不见得都是伟大事业的起点,但每种伟大的事业必定源于一种梦想——我对这件事很有把握。 

现在的青年里有“追星族”,“上班族”,但想要开创伟大事业的人却没有名目,就叫他们“伟大一族”好了。过去这样的人在校园里(不管是中国校园还是美国校园)是很多的。当盖茨先生穿着一身便装,蓬着一头乱发出现在校园里时,和我们当年一样,属于“伟大一族”。刚回中国时,我带过的那些学生起码有一半属伟大一族,因为他们眼睛里闪烁着梦想的光芒。谁是、谁不是这一族,我一眼就能看出来,但这一族的人数是越来越少了,将来也许会像恐龙一样灭绝掉。我问我哥们儿,现在干嘛呢,他说坐在那里给人家操作软件包,气得我吼了起来:咱们这样的人应该做研究工作——谁给他打软件包?但是他说,人家给钱就得了,管它干什么。我一想也对。谁要是给我一年三四万美元让我“打”软件包,我也给他“打”去了。这说明现在连我也不属伟大一族。但在年轻时,我们有过很宏伟的梦想。伟大一族不是空想家,不是只会从众起哄的狂热分子,更不是连事情还没弄清就热血沸腾的青年。他们相信,任何美好的梦想都有可能成真——换言之,不能成真的梦想本身就是不美好的。假如事情没做成,那是做的不得法;假如做成了,却不美好,倒像是一场恶梦,那是因为从开始就想得不对头。不管结局是怎样,这条路总是存在的——必须准备梦想,准备为梦想工作。这种想法对不对,现在我也没有把握。我有把握的只是:确实有这样的一族。

\chapter{警惕狭隘民族主义的蛊惑宣传}

罗素曾说,人活在世上,主要是在做两件事:一、改变物体的位置和形状,二、支使别人这样干。这种概括的魅力在于简单,但未必全面。举例来说,一位象棋国手知道自己的毕生事业只是改变棋子的位置,肯定会感到忧伤;而知识分子听人说自己干的事不过是用墨水和油墨来污损纸张,那就不仅是沮丧,他还会对说这话的人表示反感。我靠写作为生,对这种概括就不大满意:我的文章有人看了喜欢,有人看了愤怒,不能说是没有意义的……但话又说回来,喜欢也罢,愤怒也罢,终归是情绪,是虚无缥渺的东西。我还可以说,写作的人是文化的缔造者,文化的影响直至千秋万代——可惜现在我说不出这种影响是怎样的。好在有种东西见效很快,它的力量又没有人敢于怀疑:知识分子还可以做蛊惑宣传,这可是种厉害东西…… 

在第二次世界大战里,德国人干了很多坏事,弄得他们己都不好意思了。有个德国将军蒂佩尔斯基这样为自己的民族辩解:德国人民是无罪的,他们受到希特勒、戈培尔之流蛊惑宣传的左右,自己都不知道自己在干什么。还有人给希特勒所著《我的奋斗》作了一番统计,发现其中每个字都害死了若干人。德国人在二战中的一切劣迹都要归罪于希特勒在坐监狱时写的那本破书——我有点怀疑这样说是不是很客观,但我毫不怀疑这种说法里含有一些合理的成份。总而言之,人做一件事有三种办法,就以希特勒想干的事为例,首先,他可以自己动手去干,这样他就是个普通的纳粹士兵,为害十分有限;其次,他可以支使别人去干,这样他只是个纳粹军官;最后,他可以做蛊惑宣传,把德国人弄得疯不疯、傻不傻的,一齐去干坏事,这样他就是个纳粹思想家了。 

说来也怪,自苏格拉底以降,多少知识分子拿自己的正派学问教人,都没人听,偏偏纳粹的异端邪说有人信,这真叫邪了门。罗素、波普这样的大学问家对纳粹意识形态的一些成分发表过意见,精彩归精彩,还是说不清它力量何在。事有凑巧,我是在一种蛊惑宣传里长大的(我指的是张春桥、姚文元的蛊惑宣传),对它有点感性知识,也许我的意见能补大学问家的不足……这样的感性知识,读者也是有的。我说得对不对,大家可以评判。 

据我所知,蛊惑宣传不是真话——否则它就不叫作蛊惑——但它也不是蓄意编造的假话。编出来的东西是很容易识破的。这种宣传本身半疯不傻,作这种宣传的人则是一副借酒撒疯、假痴不癫的样子。肖斯塔科维奇在回忆录里说,旧俄国有种疯僧,被狂热的信念左有,信口雌黄,但是人见人怕,他说的话别人也不敢全然不信——就是这种人搞蛊惑宣传能够成功。半疯不傻的话,只有从借酒撤疯的人嘴里说出来才有人信。假如我说“宁要社会主义的草不要资本主义的苗”,不仅没人信,老农民还要揍我;非得像江青女士那样,用更年期高亢的啸叫声说出来,或者像姚文元先生那样,带著怪诞的傻笑说出来,才会有人信。要搞蛊惑宣传,必须有种什么东西盖著脸(对醉汉来说,这种东西是酒),所以我说这种人是在借酒撤疯。顺便说一句,这种状态和青年知识分子意气风发的猖狂之态有点分不清楚。虽然夫子曾曰“不得中行而与之必也狂猖乎”,但我总觉得那种状态不宜提倡。 

其次,蛊惑宣传必定可以给一些人带来快感,纳粹的干年帝国之说,肯定有些德国人爱听;“文革”里跑步进入共产主义之说,又能迎合一部分急功近利的人。当然,这种快感肯定是种虚妄的东西,没有任何现实的基础,这道理很简单,要想获得现实的快乐,总要有物质基础,嘴说是说不出来的:哪怕你想找个干净厕所享受排泄的乐趣,还要付两毛钱呢,都找宣传家去要,他肯定拿不出。最简单的作法是煽动一种仇恨,鼓励大家去仇恨一些人、残害一些人、比如宣扬狭隘的民族情绪,这可以迎合人们野蛮的劣根性。煽动仇恨、杀戮,乃至灭绝外民族,都不要花费什么。煽动家们只能用这种方法给大众提供现实的快乐,因为这是唯一可行的方法——假如有无害的方法,想必他们也会用的。我们应该体谅蛊惑宣传家,他们也是没办法。 

最后,蛊惑宣传虽是少数狂热分子的事业,但它能够得逞,却是因为正派人士的宽容。群众被煽动起来之后,有一种惊人的力量。有些还有正常思维能力的人希望这种力量可以做好事,就宽容它——纳粹在德国初起时,有不少德国人对它是抱有幻想的;但等到这种非理性的狂潮成了气候,他们后悔也晚了。“文革”初起时,我在学校里,有不少老师还在积极地帮著发动“文革”哩,等皮带敲到自己脑袋上时,他们连后悔都不敢了。根据我的生活经验,在中国这个地方,有些人喜欢受益惑宣传时那种快感;有些人则崇拜蛊惑宣传的力量;虽然吃够了蛊惑宣传的苦头,但对蛊惑宣传不生反感;不唯如此,有些人还像瘾君子盼毒品一样,渴望著新的蛊惑宣传。目前,有些年轻人的抱负似乎就是要炮制一轮新的蛊惑宣传——难道大家真的不明白蛊惑宣传是种祸国殃民的东西?在这种情况下,我的抱负只能是反对蛊祸宣传。我别无选择。

\chapter{诚实与浮嚣}

我念大学本科时,我哥哥在读研究生。我是学理科的,我哥哥是学逻辑学的。有—回我问他:依你之见,在中国人写的科学著作中,哪本最值得一读?他毫不犹豫地答道:费孝通的《江村经济》。现在假如有个年轻人问我这个问题,不管他是学什么的,我的回答还是《江村经济》——但我觉得这本书的名字还是叫作“中国农民的生活”为好。它的长处在于十分诚实地描述了江南农村的生活景象,像这样的诚实在中国人写的书里还未曾有过。同是社会学界的前辈,李景汉先生做过《定县调查》,把一个县的情况搞得清清楚楚。学社会学的人总该读读《定县调查》——但若不学社会学,我觉得可以不读《定县调查》,但不读《江村经济》可不成。中国的读书人有种毛病,总要把某些事实视而不见,这些事实里就包括了中国农民的生活。读书人喜欢做的事情是埋首于故纸堆里,好像故纸之中什么都有了。中国的典籍倒是浩若烟海,但假若没人把事实往纸上写,纸上还是什么都没有。《江村经济》的价值就在于它把事实写到了纸上,在中国这个地方,很少有人做这样的事。马林诺夫斯基给《江村经济》做序,也称赞了费先生的诚实。所以费先生这项研究中的诚实程度,已经达到了国际先进水平。 

这篇文章的主旨不是谈《江村经济》,而是谈诚实。以我之见,诚实就像金子一样,有成色的区别。就以费先生的书为例,在海外发表时,叫作“中国农民的生活”;这是十足赤金式的诚实。在国内发表时叫作《江村经济》,成色就差了一些,虽然它还是诚实的,而且更对中国文人的口味。我们这里有种传统,对十足的诚实甚为不利。有人说,朱熹老夫子做了一世的学问,什么叫作“是”(be),什么叫作“应该是”(should be),从来就没搞清楚过。我们知道,前者是指事实,后者是指意愿,两者是有区别的。人不可能一辈子遇上的都是合心意的事,如果朱夫子总把意愿和事实混为一谈,那他怎么生活呢。所以,当朱夫子开始学术思维时,他把意愿和事实当成了一回事——学术思维确有这样一种特点;不做学问时,意愿和现实又能分开了。不独朱夫子,中国人做学问时部是如此,自打孔子到如今,写文章时都要拿一股劲,讨论国计民生乃至人类的前途这样的大题目,得到一片光明的结论,在这一片光明下,十足的诚实倒显得可羞。在所有重大题目上得出一片光明的结论固然很好,但若不把意愿和事实混为一谈,这却是很难做到的。 

人忠于已知事实叫作诚实;不忠于事实就叫作虚伪。还有些人只忠于经过选择的事实,这既不叫诚实,也不叫虚伪,我把它叫作浮嚣。这是个含蓄的说法,乍看起来不够贴切,实际上还是合乎道理的:人选择事实,总是出于浮嚣的心境。有回,我读一位海外新儒家学者的文集(我对海外的新儒学并无偏见,只是举个例子),作者一会儿引东,一会儿引西,从马克斯·韦伯到现代美国黑人的“寻根文学”引了一个遍,所举例子都不甚贴切,真正该引用的事例他又没有引到。我越看越不懂,就发了狠,非看明白不可。最终看到一篇他在台北的答记者问,把自己所治之学和台湾当局的“文化建设”挂上了钩——看到这里,我算是看明白了。我还知道台湾当局拉拢海外学人是不计工本的,这就是浮嚣的起因——当然,更远的起因还能追溯到科举、八股文,人若把学问当作进身之本来做,心就要往上浮。诚实不是学术界的长处,因为太诚实了,就显得不学术;像费先生在《江衬经济》里表现出的那种诚实,的确是风毛鳞角。有位外国记者问费先生:你觉得中国再过几时才能再出一个费孝通?他答:五十年。这话我真不想信,但恐怕最终还是不得不信。

\chapter{长虫、草帽、细高挑}

近来买了本新出的《哈克贝利·芬历险记》。这本书我小时候很爱看,现在这本是新译的——众所周知,新译的书总是没有老版本好。不过新版本也不是全无长处,篇首多了一篇吐温瞎编的兵工署长通告,而老版本把它删了。通告里说:如有人胆敢在本书里寻找什么结构、道德寓意等等,一律逮捕、流放,乃至枪毙。马克·吐温胆子不小,要是现在国内哪位作家胆敢仿此通告一番:如有人敢在我的书里寻找文化源流或可供解构的东西,一律把他逮捕、流放、枪毙,我看他会第一个被枪毙。现在各种哲学,甚至是文化人类学的观点,都浩浩荡荡杀入了文学的领域。作家都成了文化批评的对象,或者说,成了老太太的尿盆——挨呲儿的货。连他们自己都从哲学或人类学上给自己找写作的依据,看起来着实可怜,这就叫人想起了电影《霸王别姬》里张丰毅演的角色,屁股上挨了板子,还要说:打得好,师傅保重。哲学家说,存在的就是合理的。一种情形既然出现了,就必然有它的原因;再说,批评也是为了作家好。但我现在靠写作为生,见了这种情形,总觉得憋气。 

我家乡有句歇后语:长虫戴草帽,混充细高挑——老家人以为细高挑是种极美丽的身材,连长虫也来冒充。文化批评就是揭去作家头上的草帽,使他们暴露出爬行动物的本色。所谓文学是不存在的,存在的只有文化——这是一种特殊的混沌,大家带着各种丑恶的心态生活在其中。这些心态总要流露出来,这种流露就是写作——假如这种指责是成立的,作家们就一点正经的都没有,是帮混混。我不敢说自己是作家,也不认识几个作家,没理由为作家叫屈。说实在的,按学历我该站在批评的一方,而不是站在受批评的一方。但若说文学事业的根基——写作,是这样一种东西,我还是不能同意。 

过去我是学理科的。按照C·P·格林的观点,正如文学是文学家的文化,科学也是科学家的文化。对科学的文化批评尚未兴起,而且我不认为它有可能兴起。但这不是说没人想要批评科学。人文学者,尤其是哲学家,总想拿数学、物理说事,给它们若干指导。说归说,数学家、物理学家总是不理,说得实在外行时,就拿它当个笑话讲。我当研究生时,有位著名的女人类学家对统计学提出了批评,说没必要搞得这么复杂、高深。很显然,这位女士想要“解构”数学的这一分支。上课之前老师把这批评给大家念了念,师生一起捧腹大笑,其乐也融融——但文学家很少有这种欢笑的机会。数学家笑,是因为假如一个人不演算,也不做公式推导,哪怕你后现代哲学懂得再多,也没有理由对数学说三道四;但这句话文学家就不敢说。同样是文化,怎么会有这种不同的境遇呢?这原因大家恐怕都想到了:文学好像人人都懂,而数学,则远不是人人都懂得。 

罗素先生说得好:人人理应平等,实际上却远不是这样——特别是人与人有知识的差别。这一点在大学里看得最明白:搞科学哲学的教授,尽管名声很大,实际上见了学物理的研究生都要巴结,而物理学家见了数学家,气焰也要减几分,因为就连爱因斯坦都有求职业数学家帮忙的时候。说起一门学问,我会你不会,咱俩就没法平等。看起来,作家们必须从反面理解这种差别:他要巴结的不仅是文艺批评家、文艺理论家,还有哲学家、人类学家、社会学家,甚至要包括每一个文科毕业的学生——只要该学生不是个作家,因为不管谁说出句话来,你听不懂,就只好撅屁股挨打,打你的人火气还特大,我总觉得这事有点不对头。假如挨两下能换来学问,也算挨得值,但就怕碰上蒙事、打几下便宜手的人。我知道一句话,估计除了德宏州的景颇人谁也听不懂:呜!阿靠!卡路来!似乎批评家要想知道意思也得让我打两下,但我没这么坏,不打人也肯把意思说出来:这话是我插队时学来的,意思是:喂,大哥,上哪儿去呀?就凭一句别人听不懂的景颇话打人,我也未免太心黑了一点——那也没有凭几句哲学咒符打人黑。 

文化批评还不全是“呜阿靠卡路来”。它有很大的正面意义,其中最重要的是可以鼓舞作家自爱、自强、自重。一种跨学科的统治一切的欲望,像幽灵一样四处游荡——可怎么偏偏是你遇上了这个鬼?俗话说,老太太买柿子,拣软的捏。但一枚柿子不能怪人家来捏你,要反省自己为什么被捏。对罗素先生的话也可以做适度的推广:人与人不独有知识的差异,还有能力的差异——我的意思是说,写作一道,虽没有很深的学问,也远不是人人都会。作家可以在两个方面表现这种差异:其一是文体,傅雷、汝龙、王道乾,这些优秀翻译家都是文体大师。谁要想解构就去解好了,反正那样的文章你写不出来。其二是想象力,像卡尔维诺《我们的祖先》,尤瑟娜尔的《东方奇观》,里面充满了天外飞龙般的想象力,这可是个硬指标,而且和哲学、人类学、社会学都不搭界。捏不动的硬柿子还有一些,比方说,马克·吐温的幽默。在所有的柿子里,最硬的是莎翁,从文字到故事都无与伦比。当然,搞文化批评的人早就向莎翁开战了,说他的《驯悍记》是男性中心主义的作品。说这个没用,他老人家是人,又没学会喝风屙烟,编几个小剧本到小剧场里搞搞笑,赚几个小钱,这又有什么。再说,人家还有四大悲剧哩--你敢挑四大悲剧的毛病吗?我现在靠写作为生,写上一辈子,总得写出些让别人解构不了的东西。我也不敢期望过高,写到有几分像莎翁就行了。到那时谁想摘我的草帽,就让他摘好了:不摘草帽是个细高挑,摘了还是个细高挑……

\chapter{都市言情剧里的爱情}

看过冯小刚导演的都市言情剧《情殇》,感到这个戏还有些长处。摄影、用光都颇考究,演员的表演也不坏,除主题歌难听,没有太不好的地方。当然,这是把它放在“都市言情剧”这一消闲艺术门类内去看,放到整个艺术的领域里评论,就难免有些苛评——现在我就准备给它点苛评。我觉得自己是文化人,作为此类人士,我已经犯下了两样滔天大罪:第一,我不该看电视剧,这种东西俗得很;第二,我不该给电视剧写评论。看了恶俗的都市言情剧,再写这篇评论文章,我就如毕达哥拉恩学派的弟子,有了吃豆子的恶行,从此要被学院拒之于门外。所幸我还有先例可引:毛姆先生是个正经作家,但他也看侦探小说,而且写过评论侦探小说的文章。毛姆先生使我觉得自己有可能被原谅。当然,是被文化人原谅,不是被言情剧作者原谅——苛刻地评论人家,还想被原谅,显得太虚伪。 

毛姆是这样评论侦探小说的:此类小说自爱伦·坡以来,人才辈出,培养出一大批狡猾的观众,也把自己推入了难堪之境。举例来说, 一旦侦探小说里出现一位和蔼可亲、与世无争的老先生时,狡猾的观众们就马上指出:杀人的凶手就是他,此类情形也发生在我们身边,言情剧的作者也处于难堪的境地。这两年都市言情剧看多了,我们正在变得狡猾:从电视屏幕上看到温柔、漂亮的女主角林幻,我马上就知道她将在这部戏里大受摧残——否则她就不必这样温柔、漂亮了。在言情剧里,一个女人温柔、漂亮,就得倒点霉;假如她长得像我(在现实生活里,女人长得像我是种重大灾难)倒有可能很走运,她还有个变成植物人的丈夫,像根木棍一样睡在病床上,拖着她,使她不便真正移情别恋。从剧情来看,任何一个女人处在女主角的地位,都要移情别恋,因为不管她多么善良、温柔, 总是—个女人,不是一根雌性的木棍,不能永远爱根雄木棍,而且剧里也没把她写成本棍,既然如此,植物人丈夫作用无非是加重对女主角的催残……剧情的发展已经证实了我的预见。 

更狡猾的观众则说,剧作者的用意还不仅如此。请相信,这根木头棍子是颗定时炸弹、一旦林幻真正移情别恋,这根木头棍子就会醒来,这颗定时炸弹就要炸响,使可爱的女主角进—步大受摧残,戏演到现在,加在女土角身上的摧残已经够可怕的了:植物人丈夫一年要二十万医药费,她爱的男人拿不出。有个她不爱的男人倒拿得出,但要地嫁过去才能出这笔钱。对于一个珍视爱情的女人来说,走到了这一步,眼看要被逼成—个感情上的大怪物……我很不希望这种预见被证实,但从剧情的发展来看,又没有别的出路,造出一颗定时炸弹,不让它响,对炸弹也不公平哪。 

毛姆先生曾指出,欣赏通俗作品有种决窍,就是不要把它当真;要把它当作编出来的东西来看,这样就能得到一定的乐趣。常言道:爱与死是永恒的主题,侦探小说的主题是死,言情剧的主题是爱。虽然这两件事是我们生活中的大事,但出现在通俗作品里,就不能当真。此话虽然大有道理,怎奈我不肯照办。 

从长远的观点来看,我们都是要死的。被杀也是一种可能的死因。但任何一个有尊严的人都会拒绝侦探小说里那种死法:把十八英尺长的鳄鱼放到游泳池里,让它咬死你;或者用锐利的冰柱射入你的心脏,最起码要你死于南洋土人使用的毒刺——仿佛这世界上没有刀子也拣不到砖头。其实没有别的理由,只是要你死得怪怪的。这不是死掉,而是把人当猴子耍,凶手对死者太不尊重——我这样认真却是不对的。侦揉小说的作者并没有真的杀过人。所以,在侦探小说里,别的事情都可以当真,唯有死不能当真。 

同理,都市言情剧别的事都可以当真,也只有爱情不能当真。倘若当真,就有很多事无法解释,以《情殇》中的林幻为例,她生为一个女人,长得漂亮也不是她之罪,渴望爱情又有什么不对?但不知为什么,人家给她的却是这样一些男人:第一个只会睡觉,该醒时他不醒,不该醒时他偏醒;就是这么睡,一年却要二十万才够开销——看到睡觉有这么贵,我已经开始失眠;第二个虽然有点像土匪,她也没有挑剔,爱上了,但又没有钱,不能在一起;第三个有钱,可以在一起,她又不爱——看到钱是如此重要,我也想挣点钱,免得害着我老婆;甚至想到去写电视剧——我也不知还有没有第四个和第五个,但我知道假如有,也不会是什么好东西。这世界上不是没有好男人,怎奈人家不给她,拣着坏的给。这个女人就像一头毛驴被驾在车辕上,爱情就像胡萝卜,挂在眼前,不管怎么够,就是吃不着——既然如此,倒不如不要爱情。我想一个有尊严的女人到了这个地步,一定会向上帝抱怨:主啊,我知道你的好意,你把我们分成男人和女人,想让我们生活有点乐趣——可以谈情说爱;但是好心不一定能办好事啊。看我这个样子,你不可怜我吗?倒不如让我没有性别,也省了受这份活罪——我知道有些低等生物蒙你的恩宠,可以无性繁殖。我就像细菌那样分裂繁殖好了。这样晚上睡觉,早上一下变成了两个人,谈恋爱无非是找个伴儿嘛,自己裂成两半儿,不就有伴儿了吗…… 

上帝听了林幻的祷告,也许就安排她下世做个无性繁殖的人,晚上睡觉时是林幻,醒来就变成了林幻一和林幻二,再也不用谈爱情。很不幸的是,这篇祷告词有重大的遗漏,忘记告诉上帝千万不要再把她放进电视剧里。以免剧作者还是可以拿着她分裂的事胡编乱派,让她生不如死。但这已是另一个世界里电视剧作者的题目,非我所能知道。

\chapter{有关“给点气氛”}

我相信,总有些人会渴望有趣的事情,讨厌呆板无趣的生活。假如我有什么特殊之处,那就是:这是我对生活主要的要求。大约十五年前,读过一篇匈牙利小说,题目叫作《会说话的猪》,讲到有一群国营农场的种猪聚在一起发牢骚——这些动物的主要工作是传种。在科技发达的现代,它们总是对着一个被叫作“母猪架子”的人造母猪传种。该架子新的时候大概还有几分像母猪,用了十几年,早就被磨得光秃秃的了——那些种猪天天挺着大肚子往母猪架子上跳,感觉有如一砣冻肉被摔上了案板,难免口出怨言,它们的牢骚是:哪怕在架子背上粘几撮毛,给我们点气氛也好!这故事的结局是相当有教育意义的:那些发牢骚的种猪都被敲掉了。但我总是从反面理解问题:如果连猪都会要求一点气氛,那么对于我来说,一些有趣的事情干脆是必不可少。 

活在某些时代,持有我这种见解会给自己带来麻烦,我就经历过这样的年代——书书没得看,电影电影没得看。整个生活就像个磨得光秃秃的母猪架子,好在我还发现了一件有趣的事情,那就是发牢骚——发牢骚就是架子上残存的一撮毛。大家聚在一起,你一句、我一句,人人妙语连珠,就这样把麻烦惹上身了。好在我还没有被敲掉,只是给自己招来了很多批评帮助。这时候我发现,人和人其实是很隔膜的。有些人喜欢有趣,有些人喜欢无趣,这种区别看来是天生的。作为一个喜欢有趣的人,我当然不会放弃阅读这种获得有趣的机会。结果就发现,作家里有些人拥护有趣,还有些人是反对有趣的。马克·吐温是和我是一头的,或者还有萧伯纳——但我没什么把握。我最有把握的是哲学家罗素先生,他肯定是个赞成有趣的人。摩尔爵士设想了一个乌托邦,企图给人们营造一种最美好的生活方式,为此他对人应该怎样生活作了极详尽的规定,包括新娘新郎该干点什么——看过《乌托邦》的人一定记得,这个规定是:在结婚之前,应该脱光了眼子让对方看一看,以防身上暗藏了什么毛病。这个用意不能说不好,但规定得如此之细就十足让人倒胃,在某些季节里,还可能导致感冒。罗素先生一眼就看出乌托邦是个母猪架子,乍看起来美涣美仑,使上一段,磨得光秃秃,你才会知道它有多糟糕——他没有在任何乌托邦里生活过,就有如此见识,这种先知先觉让人佩服得五体投地——他老人家还说,须知参差多态,乃是幸福的本源。反过来说,呆板无趣就是不幸福——正是这句话使我对他有了把握。一般来说,主张扼杀有趣的人总是这么说的:为了营造至善,我们必须做出这种牺牲,但却忘记了让人们活着得到乐趣,这本身就是善;因为这点小小的疏忽,至善就变成了至恶…… 

这篇文章是从猪要求给点气氛说起的。不同意我看法的人必然会说,人和猪是有区别的。我也认为人猪有别,这体现在人比猪要求得更多,而不是更少。除此之外,喜欢有趣的人不该像那群种猪一样,只会发一通牢骚,然后就被阉掉。这些人应该有些勇气,作一番斗争,来维护自己的爱好。这个道理我直到最近才领悟到。 

我常听人说:这世界上哪有那么多有趣的事情。人对现实世界有这种评价、这种感慨,恐怕不能说是错误的。问题就在于应该做点什么。这句感慨是个四通八达的路口,所有的人都到达过这个地方,然后在此分手。有些人去开创有趣的事业,有些人去开创无趣的事业。前者以为,既然有趣的事不多,我们才要做有趣的事。后者经过这一番感慨,就自以为知道了天命,此后板起脸来对别人进行说教。我以为自己是前一种人,我写作的起因就是:既然这世界上有趣的书是有限的,我何不去试着写几本——至于我写成了还是没写成,这是另一个问题,我很愿意就这后一个问题进行讨论,但很不愿有人就头一个问题来和我商榷。前不久有读者给我打电话,说:你应该写杂文,别写小说了。我很认真的倾听着,他又说:你的小说不够正经——这话我就不爱听了。谁说小说非得是正经的呢。不管怎么说罢,我总把读者当作友人,朋友之间是无话不说的:我必须声明,在我的杂文里也没什么正经。我所说的一切,无非是提醒后到达这个路口的人,那里绝不是只有一条路,而是四通八达的。你可以做出选择。

\chapter{有关黄金时代}

《黄金时代》这本书里,包括了五部中篇小说。其中“黄金时代”一篇,从二十岁时就开始写,到将近四十岁时才完篇,其间很多次的重写。现在重读当年的书稿,几乎每句话都会使我汗颜,只有最后的定稿读起来感觉不同。这篇三万多字的小说里,当然还有不完美的地方,但是我看到了以后,丝毫也沒有改动的冲动。这说明小说有这样一种写法,虽然困难,但还不是不可能。这种写法就叫作追求对作者自己来说的完美。我相信对每个作者来说,完美都是存在的,只是不能经常去追求它。据说迪伦马特写《法官和他的刽子手》,也写了很多年。写完以后说:今后再也不能这样写小说了。这说明他也这样写过。一个人不可能在每篇作品里做到完美,但是完美当然是最好的。有一次,有个女孩子问我怎样写小说,并且说她正有要写小说的念头。我把写“黄金时代”的过程告诉了她。下次再见面,问她的小说写得怎样了,她说,听说小说这么难写,她已经把这个念头放下了。其实在这本书里,大多数章节不是这样呕心沥血地写成的。但我主张,任何写小说的人都不妨试试这种写法。这对自己是有好处的。  

这本书里有很多地方写到性。这种写法不但容易招致非议,本身就有媚俗的嫌疑。我也不知为什么,就这样写了出来。现在回忆起来,这样写既不是为了找些非议,也不是想要媚俗,而是对过去时代的回顾。众所周知,六七十年代,中国处于非性的年代。在非性的年代里,性才会成为生活主题,正如饥饿的年代里吃会成为生活的主题。古人说:食色性也。想爱和想吃都是人性的一部分;如果得不到,就成为人性的障碍。  

然而,在我的小说里,这些障碍本身又不是主题。真正的主题,还是对人的生存状态的反思。其中最主要的一个逻辑是:我们的生活有这么多的障碍,真他妈的有意思。这种逻辑就叫作黑色幽默。我觉得黑色幽默是我的气质,是天生的。我小说里的人也总是在笑,从来就不哭,我以为这样比较有趣。喜欢我小说的人总说,从头笑到尾、觉得很有趣等等。这说明本人的作品有自己的读者群。当然,也有些作者以为哭比较使人感动。他们笔下的人物从来就不笑,总在哭。这也是一种写法。他们也有自己的读者群。有位朋友说,我的小说从来沒让她感动过。她就是个爱哭的人,误读了我的小说,感到很失落。我这样说,是为了让读者不再因为误读我的小说感到失落。   现在严肃小说的读者少了,但读者的水平是大大提高了。在现代社会里,小说的地位和舞台剧一样,正在成为一种高雅艺术。小说会失去一些读者,其中包括想受道德教育的读者,想看政治暗喻的读者,感到性压抑、寻找发泄渠道的读者,无所事事想要消磨时光的读者;剩下一些真正读小说的人。小说也会失去一些作者——有些人会去下海经商,或者搞影視剧本;最后只剩下一些真正写小说的人。我以为这是一件好事。

\chapter{我的精神家园之序言}

  年轻时读萧伯纳的剧本《巴巴拉少校》,有场戏给我留下了深刻的印象:工业巨头安德谢夫老爷子见到了多年不见的儿子斯泰芬,问他对做什么有兴趣。这个年轻人在科学、文艺、法律等一切方面一无所长,但他说自己有一项长处:会明辨是非。老爷子把自己的儿子暴损了一通,说这件事难倒了一切科学家、政治家、哲学家,怎么你什么都不会,就会一个明辨是非?我看到这段文章时只有二十来岁,登时痛下决心,说这辈子我干什么都可以,就是不能做一个一无所能,就能明辨是非的人。因为这个原故,我成了沉默的大多数的一员。我年轻时所见的人,只掌握了一些粗浅(且不说是荒谬)的原则,就以为无所不知,对世界妄加判断,结果整个世界都深受其害。直到我年登不惑,才明白萧翁的见解原有偏颇之处;但这是后话——无论如何,萧翁的这些议论,对那些浅薄之辈、狂妄之辈,总是一种解毒剂。 

  萧翁说明辨是非难,是因为这些是非都在伦理的领域之内。俗话说得好,此人之肉,彼人之毒;一件对此人有利的事,难免会伤害另一个人。真正的君子知道,自己的见解受所处环境左右,未必是公平的;所以他觉得明辨是非是难的。倘若某人以为自己是社会的精英,以为自己的见解一定对,虽然有狂妄之嫌,但他会觉得明辨是非很容易。明了萧翁这重意思以后,我很以做明辨是非的专家为耻——但这已经是二十年前的事了。当时我是年轻人,觉得能洁身自好,不去害别人就可以了。现在我是中年人——一个社会里,中年人要负很重的责任:要对社会负责,要对年轻人负责,不能只顾自己。因为这个原故,我开始写杂文。现在奉献给读者的这本杂文集,篇篇都在明辨是非,而且都在打我自己的嘴。 

  伦理问题虽难,但却不是不能讨论。罗素先生云,真正的伦理原则把人人同等看待。考虑伦理问题时,想替每个人都想一遍是不可能的事,但你可以说,这是我的一得之见,然后说出自己的意见,把是非交付公论。讨论伦理问题时也可以保持良心的清白——这是我最近的体会;但不是我打破沉默的动机。假设有一个领域,谦虚的人、明理的人以为它太困难、太暧昧,不肯说话,那么开口说话的就必然是浅薄之徒、狂妄之辈。这导致一种负筛选:越是傻子越敢叫唤——马上我就要说到,这些傻子也不见得是真的傻,但喊出来的都是傻话。久而久之,对中国人的名声也有很大的损害。前些时见到个外国人,他说:听说你们中国人都在说“不”?这简直是把我们都当傻子看待。我很不客气地答道:物以类聚,人以群分。你认识的中国人都说“不”,但我不认识这样的人。这倒不是唬外国人,我认识很多明理的人,但他们都在沉默中,因为他们都珍视自己的清白。但我以为,伦理问题太过重要,已经不容我顾及自身的清白。 

  伦理(尤其是社会伦理)问题的重要,在于它是大家的事——大家的意思就是包括我在内。我在这个领域里有话要说,首先就是:我要反对愚蠢。一个只会明辨是非的人总是凭胸中的浩然正气做出一个判断,然后加上一句:难道这不是不言而喻的吗?任何受过一点科学训练的人都知道,这世界上简直找不到什么不言而喻的事,所以这就叫做愚蠢。在我们这个国家里,傻有时能成为一种威慑。假如乡下一位农妇养了五个傻儿子,既不会讲理,又不懂王法,就会和人打架,这家人就能得点便宜。聪明人也能看到这种便宜,而且装傻谁不会呢——所以装傻就成为一种风气。我也可以写装傻的文章,不只是可以,我是写过的——“文革”里谁没写过批判稿呢。但装傻是要不得的,装开了头就不好收拾,只好装到底,最后弄假成真。我知道一个例子是这样的:某人“文革”里装傻写批判稿,原本是想搞点小好处,谁知一不小心上了《人民日报》头版头条,成了风云人物。到了这一步,我也不知他是真傻假傻了。再以后就被人整成了“三种人”。到了这个地步,就只好装下去了,真傻犯错误处理还能轻些呀。 

  我反对愚蠢,不是反对天生就笨的人,这种人只是极少数,而且这种人还盼着变聪明。在这个世界上,大多数愚蠢里都含有假装和弄假成真的成分;但这一点并不是我的发现,是萧伯纳告诉我的。在他的《匹克梅梁》里,息金斯教授遇上了一个假痴不癫的杜特立尔先生。息教授问:你是恶棍还是傻瓜?这就是问:你假傻真傻?杜先生答:两样都有点,先生,凡人两样都得有点呀。在我身上,后者的成分多,前者的成分少;而且我讨厌装傻,渴望变聪明。所以我才会写这本书。 

  在社会伦理的领域里我还想反对无趣,也就是说,要反对庄严肃穆的假正经。据我的考察,在一个宽松的社会里,人们可以收获到优雅,收获到精雕细琢的浪漫;在一个呆板的社会里,人们可以收获到幽默——起码是黑色的幽默。就是在我呆的这个社会里,什么都收获不到,这可是件让人吃惊的事情。看过但丁《神曲》的人就会知道,对人来说,刀山剑树火海油锅都不算严酷,最严酷的是寒冰地狱,把人冻在那里一动都不能动。假如一个社会的宗旨就是反对有趣,那它比寒冰地狱又有不如。在这个领域里发议论的人总是在说:这个不宜提倡,那个不宜提倡。仿佛人活着就是为了被提倡。要真是这样,就不如不活。罗素先生说,参差多态乃是幸福的本源——弟兄姐妹们,让我们睁开眼睛往周围看看,所谓的参差多态,它在哪里呢。 

  在萧翁的《巴巴拉少校》中,安德谢夫家族的每一代都要留下一句至理名言。那些话都编得很有意思,其中有一句是:人人有权争胜负,无人有权论是非。这话也很有意思,但它是句玩笑。实际上,人只要争得了论是非的权力,他已经不战而胜了。我对自己的要求很低:我活在世上,无非想要明白些道理,遇见些有趣的事。倘能如我所愿,我的一生就算成功。为此也要去论是非,否则道理不给你明白,有趣的事也不让你遇到。我开始得太晚了,很可能做不成什么,但我总得申明我的态度,所以就有了这本书——为我自己,也代表沉默的大多数。 


                        王小波 

                       1997年3月20日

\chapter{科学与邪道}

从历史书上看到,在三十年代末的德国,很多科学家开始在学校里讲授他们的德国化学、德国数学、德国物理学。有位德国物理学家指出:“有人说科学现在和永远是有国际性的——这是不对的;科学和别的每一项人类创造的东西一样,是有种族性和以血统为条件的。”这话着实有意思。但不知是怎么个种族性法。化学和数学的种族性我没查到,有关物理学的种族性,人家是这么解释的:经典物理是由亚利安人创造的:牛顿、伽利略等等,都是亚利安人,而且大多是北欧血统,所以这门科学是好的;至于现代物理学,都是犹太人搞出来的,所以是邪恶的,必须斩尽杀绝。爱因斯坦是犹太人,他和他的相对论是 “德国物理”的死敌——纳粹物理学家宣称,谁要是称赞相对论,就是喜欢犹太人统治世界,并对“德国人永远沦为无生气的奴隶地位”表示高兴。可想而知,爱因斯坦要是落到德国人手里,肯定没有好。他也知道这一点,所以早早逃到美国去,保住了一条命。德国数学和化学的内容是什么,我不确切知道,但知道它肯定会让纳粹科学家特别开心,让犹太科学家特别不开心——因为一般来说。挨骂总是不开心的事情。 

  过去,在生物学领域里,遗传学曾被认为是资产阶级的邪说,所以就有种无产阶级的生物学——这就是李森科的神圣学派。这种学说我上学时听过一耳朵,好像还有些道理,但不知为什么一定要和遗传学过不去。这股邪风是从前苏联传过来的,老大哥教给我们些好的东西,也教了些邪的歪的。身在那个时代,不会遗传学的人会很高兴,但也有人不高兴,我有位老师,年轻时对现代语言学很有兴趣,常借些新的英文书刊来看。后来有人给他打个招呼说:你这样下去很危险,会滑进资产阶级的泥坑;我们的语言学要以一位前苏联伟人论浯言学问题的小册子为神圣的根基——而你正在背离这个根基。我老师听了很害怕,后来就进了精神病院。他告诉我说,自己是装疯避祸,但我总觉得他是真被吓疯了,因为他讲起这件事来总带着一股胆战心惊的样子。这位老师后来贫困潦倒、提心吊胆,再后来虽然用不着提心吊胆,但大好年华已过。他对这些事当然很不开心。 

  我说的都是过去的事情,现在已经好多了。相对论、遗传学,还有社会学和人类学,都不再是邪恶的学问,我们可以放心地学习了。但有些事情我们还是不明白 ——如果只是外行来摧残科学,我们还可以理解,真正能在科学领域内兴风作浪的,都是懂点科学的人。那些德国和前苏联的学者们,干吗要分裂科学,把它搞褊狭呢?有些史实可以帮助解释这个疑问:从1905年到1931年,有10位德国犹太人,因为在科学上做出贡献得到了诺贝尔奖金,这对某些以纯亚利安血统而自豪的德国科学家来说,未免太多了些。近现代科学取得了很多成就,这些成就大多不是诞生在俄国,难免让俄国科学家气不顺。因此就想把别人的成就贬低,甚至抹煞掉,对自己的成就则夸大,甚至无中生有;以此来证明种族或者这方土地有很大的优越性。中国血统的科学家成就也不少,诺贝尔物理奖、化学奖通通拿到了,虽然他们是美籍,但愿我们能以此为荣。有件事正在使我忧虑:中国人和德国人不同。中国人对证明白己的种族优越从来就不很在意的,他们真正在意的是想要证明自己传统文化的优越性。 

  最近我们听说,从儒家道家、阴阳五行、周易八卦等等之中,即将产生震惊世界的科学成就,前不久我在电视上和一位作家辩论,他告诉我说,有位深谙此道的老者,不用抹胶水,脑门上能贴一叠子钢铺。这件事无论是爱因斯坦还是玻尔都做不到,看来我们的诺贝尔奖又有门了。但我想来想去,怎么也想象不出瑞典科学院的秘书会这样向世界宣布:女士们先生们,这位获奖的科学家能在脑门上贴一大叠钢铺。这是了不起的本领,但诺贝尔奖总不能奖给一个很粘糊的脑门吧。作家这样瞎说还不要紧,科学家也有信这个的。像这样的学问搞了出来,外国人不信怎么办呢?到那时又该说:科学和人类创造的一切东西一样,是以文化和生活方式特异性为基础的。以此为基础,划分出中国的科学,这是好的。还有外国的科学,那是邪恶的,通通都要批倒批臭。中国数学、中国物理和中国化学,都不用特别发明出来,老祖宗都替我们发明好了:中国物理是阴阳,中国化学是五行,中国数学是八卦。到了那时,我们又退回到中世纪去了。

\chapter{与同性恋有关的伦理问题}

1992年,我和李银河合作完成了对中国男同性恋的研究之后,出版了一本专著,写了一些文章。此后,我们仍同研究中结识的朋友保持了一些联系。除此之外,还收到了不少读者的来信。最近几年,虽然没有对这个问题做更深的研究,但始终关注着这一社会问题。 

从1992年到现在,关注同性恋问题的人已经多起来。有不少关于同性恋的研究发表,还有一些人出来做同性恋者的社会工作,我认为这是非常好的事情。当然假如在艾滋病出现之前就能有人来关注同性恋问题,那就更好一些。据我所知,因为艾滋病流行才来关注这个问题,是件很使同性恋者反感的事情。我们的研究是出于社会学方面的兴趣,这种研究角度,调查对象接受起来相对而言比较容易些。 

做科学研究时应该价值中立,但是作为一个一般人,就不能回避价值判断。作为一个研究者,可以回避同性恋道德不道德这类问题,但作为一个一般人就不能回避。应该承认,这个问题曾经使我相当困惑,但是现在我就不再困惑。假定有个人爱一个同性,那个人又爱他;那么此二人之间发生性关系,简直就是不可避免的。不可避免、又不伤害别人的事,谈不上不道德。有些同性恋伴侣也会有很深、很长久的关系。假如他们想要做爱的话,我想不出有什么理由要反对他们。我总觉得长期、固定、有感情的性关系应该得到尊重。这和尊重婚姻是一个道理。 

这几年,我们听到过各种对同性恋的价值判断,有人说:同性恋是一种社会丑恶现象,同性恋不道德,等等。因为我有不少同性恋者朋友,他们都是很好的人,我觉得这种指责是没有道理的,所以这些话曾经使我相当难过。但现在我已经不难过了。这种难过已经变成了一种泛泛的感觉:在我们这里,人对人的态度,有时太过粗暴、太不讲道理。按现代的标准来看,这种态度过于原始──这可能是传统社会的痕迹。假如真是这样,我们或许可以期望将来情况会变得好些。 

我对同性恋者的处境是同情的。尤其是有些朋友有自己的终生恋人,渴望能终生厮守,但现在却是不可能的,这就让人更加同情。不管是同性恋,还是异性恋,对爱情忠贞不渝的人总是让人敬重。但是同性恋圈子里有些事我不喜欢,那就是有些人中间存在的性乱。和不了解的人发生性关系,地点也不考究;不安全、不卫生,又容易冒犯他人。国外有些同性恋者认为,从一而终,是异性恋社会里的陈腐观念,他们就喜欢时常更换性伴。对此我倒无话可说。但一般来说,性乱是社会里的负面现象;是一种既不安定又危险的生活方式。一个有理性的人总能相信,这种生活方式并不可取。 

众所周知,近几年来人们对同性恋现象的关注,是和对艾滋病的关注紧密相连的。但艾滋病和男同性恋的关联,应该说是有很大偶然性的。国外近几年的情况是:艾滋病的主要传播渠道不再是男同性恋,它和其它性传播疾病一样,主要在社会的下层流传。这是因为人们知道了这种病是怎么回事,素质较高的人就改变自己的行为方式来预防它。剩下一些素质不高的人,才会患上这种病。没有钱、没有社会地位、没有文化,人很难掌握自己的命运。我倒以为,假如想要防止艾滋病在中国流行,对于我国的流浪人口,不可掉以轻心。 

艾滋病发现之初,有些人说:这种病是上帝对男同性恋者的惩罚,现在他们该失望了──不少静脉吸毒者也得了艾滋病。我觉得人应该希望有个仁慈的上帝,指望上帝和他们自己一样坏是不对的。我知道有些人生活的乐趣就是发掘别人道德上的“毛病”,然后盼着人家倒霉。谢天谢地,我不是这样的人。 

鉴于本文将在医学杂志上发表,“医者父母心”,一种人文的立场可能会获得更多的共鸣。我个人认为,享受自己的生活对任何人来说都是头等重要的事。性可以带来种种美好的感受,是人生最重要的资源。而同性恋是同性恋者在这方面所有的一切。就我所知,医学没有办法把同性恋者改造成异性恋者──我猜这是因为性倾向和人的整个意识混为一体──所谓矫治,无非是剥夺他的性能力。假如此说属实,矫治就没什么道理。有的人渴慕异性,有些人渴慕同性,但大家对爱情的态度是一样的,歧视和嘲笑是没有道理的。历史上迫害同性恋者最力者,或则不明事理,或则十分偏执──我指的是中世纪的某些天主教士和纳粹分子──中国历史上没有迫害同性恋的例子,这可能说明我们的祖先既明事理,又不十分偏执,这种好传统应该发扬光大。我认为社会应该给同性恋者一种保障,保护他们的正当权益。举例来说,假如有一对同性恋者要结婚,我就看不出有什么不可以。 

至于同性恋者,我希望他们对生活能取一种正面的态度,既能对自己负责,也能对社会负责。我认识的一些同性恋者都有很高的文化素质、很好的工作能力。我总以为,像这样一些朋友,应该能把自己的生活弄得像个样子。我是个异性恋者,我的狭隘经验是:能和自己所爱的女人体面地出去吃饭,在自己家里不受干扰地做爱比较好;至于在街头巷尾勾个性伴,然后在个肮脏的地方瞎弄几下是不好的。当然,现在同性恋者很难得到这样的条件,但这样的生活应该是他们争取的目标。

\chapter{有关“错误的故事”}

  1977年恢复了高考,但我不信大学可以考进去(以前是推荐的),直到看见有人考进去我才信了。然后我就下定决心也要去考,但“文化革命”前我在上初一,此后整整十年没有上学,除了识字,我差不多什么都不会了。离考期只有六个月,根本就来不及把中学的功课补齐。对于这件事,我是这么想的:补习功课无非是为了走进高考的考场,把考题做对。既然如此,我就不必把教科书从头看到尾。干脆,拿起本习题书直接做题就是了。结果是可想而知:几乎每题必错。然后我再对着正确答案去想:我到底忽略了什么?中学的功课对一个成人的智力来说,并不是什么太难猜的东西。就这样连猜带蒙,想出了很多别人没有教过的东西。乱忙了几个月,最后居然也做对不少题。进了考场,我忽然冷汗直冒,心里没底——到底猜得对不对,这回可要见真佛了。 

  现在的年轻人看到此处,必然会猜到:那一年我考上了,要不就不会写这篇文章。他们还会说:又在写你们老三届过五关斩六将的英雄事迹,真是烦死了,我的确是考上了,但并不觉得有何值得夸耀之处。与此相反,我是怀着内心的痛苦在回忆此事。别人在考场上,看到题目都会做,就会高兴;我看到题目都会做,里倒发起虚来。每做出一道题,我心里就要嘀咕一番:这个做法是我猜的,到底对不对呢?所有题都做完,我已经愁肠千结,提前半小时交卷,像丧家犬一样溜出考场。考完之后,别人都在谈论自己能得多少分。我却不敢谈论:得一百分和零分都在我预料之内。虽然成绩不坏,但我还是后怕得很,以后再不敢这样学习。那一年的考生里,像我这样的人还不少,但不是每个人都像我这样怀疑自己。有些考友从考场出来时,心情激动地说:题目都做出来了,这回准是一百分!等发榜一看,几乎是零蛋。这不说明别的,只说明他对考试科目的理解彻底不对。 

  下面一件事是我在海外留学时遇到的。现在的年轻人大可以说,我是在卖弄自己出国留过学。这可不是夸耀,这是又一桩痛苦的经历,虽然发生在别人身上,我却没有丝毫的幸灾乐祸——我上的那所大学的哲学系以科学哲学著称。众所周知,科学哲学以物理为基础,所以哲学系的教授自以为在现代物理方面有很深的修养。忽一日,有位哲学教授自己觉得有了突破性的发现——而且是在理论物理上的发现,高兴之余,发帖子请人去听他的讲座,有关各系的教授和研究生通通都在邀请之列,我也去了,听着倒是蛮振奋的,但又觉得不像是这么回事。听着听着,眼见得听众中有位物理系的教授大模大样,掏出个烟斗抽起烟来。等人家讲完,他把烟斗往凳子腿上一磕,说道:“wrong story !”(错误的故事)就扬长而去。既然谈的是物理,当然以物理教授的意见为准。只见那位哲学教授脸如猪肝色,恨不能一头钻下地去。 

  现在的年轻人又可以说,我在卖弄自己有各种各样的经历。他们爱说什么就说什么好了。我这一生听过各种“wrong story ”,奇怪的是:错得越厉害就越有人信——这都是因为它让人振奋。听得多了,我也算个专家了。有些故事,如“文革”中的种种古怪说法,还可以祸国殃民。我要是编这种故事,也可以发大财,但我就是不编。我只是等故事讲完之后,用烟斗敲敲凳子腿,说一声:这种理解彻底不对。 


(编者:本篇最初发表于1996年11月8 日《南方周末》。)

\chapter{对待知识的态度}

  我年轻时当过知青,当时没有什么知识,就被当作知识分子送到乡下去插队。插队的生活很艰苦,白天要下地干活,天黑以后,插友要玩,打扑克,下象棋。我当然都参加——这些事你不参加,就会被看作怪人。玩到夜里十一二点,别人都累了,睡了,我还不睡,还要看一会儿书,有时还要做几道几何题。假如同屋的人反对我点灯,我就到外面去看书。我插队的地方地处北回归线上,海拔2400米。夜里月亮像个大银盆一样耀眼,在月光下完全可以看书——当然,看久了眼睛有点发花——时隔20多年,当时的情景历历在目。 

  如今,我早已过了不惑之年。旧事重提,不是为了夸耀自己是如何的自幼有志于学。现在的高中生为了考大学,一样也在熬灯头,甚至比我当年熬得还要苦。我举自己作为例子,是为了说明知识本身是多么的诱人。当年文化知识不能成为饭碗,也不能夸耀于人,但有一些青年对它还是有兴趣,这说明学习本身就可成立为一种生活方式。学习文史知识目的在于“温故”,有文史修养的人生活在从过去到现代一个漫长的时间段里。学习科学知识目的在于“知新”,有科学知识的人可以预见将来,他生活在从现在到广阔无垠的未来。假如你什么都不学习,那就只能生活在现时现世的一个小圈子里,狭窄得很。为了说明这一点,让我来举个例子。 

  在欧洲的内卡河畔,有座美丽的城市。在河的一岸是历史悠久的大学城。这座大学的历史,在全世界好像是排第三位——单是这所学校,本身就有无穷无尽的故事。另一岸陡峭的山坡上,矗立着一座城堡的废墟,宫墙上还有炸药炸开的大窟窿。照我这样一说很是没劲,但你若去问一个海德堡人,他就会告诉你,二百年前法国大军来进攻这座宫堡的情景:法军的掷弹兵如何攻下了外层工事,工兵又是怎样开始爆破——在这片山坡上,何处是炮阵地,何处是指挥所,何处储粮,何处屯兵。这个二百年前的古战场依然保持着旧貌,硝烟弥漫——有文化的海德堡人绝不止是活在现代,而是活在几百年的历史里。 

  与此相仿、时候我住在北京的旧城墙下。假如那城墙还在,我就能指着它告诉你:庚子年间,八国联军克天津,破廊坊,直逼北京城下。当时城里朝野陷于权力斗争之中,偌大一个京城竟无人去守,……此时有位名不见经传的营官不等待命令,挺身而出,率健锐营“霆字队”的区区百人,手持新式快枪,登上了左安门一带的城墙,把联军前锋阻于城下,前后有一个多时辰。此人是一个英雄。像这样的英雄,正史上从无记载,我是从野史上看到的。有关北京的城墙,当年到过北京的联军军官写道:这是世界上最伟大的防御工事。它绵延数十里,是一座人造的山脊。对于一个知道历史的中国人来说,他也不会只活在现在。历史,它可不只是尔虞我诈的宫廷斗争,…… 

  作为一个理工科出身的人,其实我更该谈谈科学,说说它如何使我们知道未来。打个比方来说,我上大学时,学了点计算机方面的知识,今天回想起来,都变成了老掉牙的东西。这门科学一日一变,越变越有趣,这种进步真叫人舍不得变老,更舍不得死,……学习科学技术,使人对正在发展的东西有兴趣。但我恐怕说这些太过专业,所以就到此为止。现在的年轻人大概常听人说,人有知识就会变聪明,会活得更好,不受人欺。这话虽不错,但也有偏差。知识另有一种作用,它可以使你生活在过去、未来和现在,使你的生活变得更充实、更有趣。这其中另有一种境界,非无知的人可解。不管有没有直接的好处,都应该学习——持这种态度来求知更可取。大概是因为我曾独自一人度过了求知非法的长夜,所以才有这种想法,……当然,我这些说明也未必能服人。反对我的人会说,就算你说的属实,但我就愿意只生活在现时现世!我就愿意得些能见得到的好处!有用的我学,没用的我不学,你能奈我何?……假如执意这样放纵自己,也就难以说服。罗素曾经说:对于人来说,不加检点的生活,确实不值得一过。他的本意恰恰是劝人不要放弃求知这一善行。抱着封闭的态度来生活,活着真的没什么意思。 

(本篇最初发表于1996年第6 期《辽宁青年》杂志。)

\chapter{智慧与国学}

                 一                    

  我有一位朋友在内蒙古插过队,他告诉我说,草原上绝不能有驴。假如有了的话,所有的马群都要“炸”掉。原因是这样的:那个来自内地的、长耳朵的善良动物来到草原上,看到了马群,以为见到了表亲,快乐地奔了过去;而草原上的马没见过这种东西,以为来了魔鬼,被吓得一哄而散。于是一方急于认表亲,一方急于躲鬼,都要跑到累死了才算。近代以来,确有一头长耳朵怪物,奔过了中国的原野,搅乱了这里的马群,它就是源于西方的智慧。假如这头驴可以撵走,倒也简单。问题在于撵不走。于是就有了种种针对驴的打算:把它杀掉,阉掉,让它和马配骡子,没有一种是成功的。现在我们希望驴和马能和睦相处,这大概也不可能。有驴子的地方,马就养不住。其实在这个问题上?马儿的意见最为正确:对马来说,驴子的确是可怕的怪物。 

  让我们来看看驴子的古怪之处。当年欧几里得讲几何学,有学生发问道,这学问能带来什么好处?欧几里得叫奴隶给他一块钱,还讽刺他道:这位先生要从学问里找好处啊!又过了很多年,法拉第发现了电磁感应,演示给别人看,有位贵妇人说:这有什么用?法拉第反问道:刚生出来的小孩子有什么用?按中国人的标准,这个学生和贵妇有理,欧几里得和法拉第没有理:学以致用嘛,没有用处的学问哪能叫做学问。西方的智者却站在老师一边,赞美欧几里得和法拉第,鄙薄学生和贵妇。时至今日,我们已经看出,很直露地寻求好处,恐怕不是上策。这样既不能发现欧氏几何,也不能发现电磁感应,最后还要吃很大的亏。怎样在科学面前掩饰我们要好处的暖昧心情,成了一个难题。 

  有学者指出,中国传统的思维方式有重实用的倾向。他们还以为,这一点并不坏。抱着这种态度,我们很能欣赏一台电动机。这东西有“器物之用”,它对我们的生活有些贡献。我们还可以像个迂夫子那样细列出它有“抽水之用”、“通风之用”,等等。如何得到“之用”,还是个问题,于是我们就想到了发明电动机的那个人——他叫做西门子或者爱迪生。他的工作对我们可以使用电机有所贡献,换言之,他的工作对器物之用又有点用,可以叫做“器物之用之用”。像这样林林总总,可以揪出一大群:法拉第、麦克斯韦,等等,分别具有“之用之用之用”或更多的之用。像我这样的驴子之友看来,这样来想问题,岂止是有点笨,简直是脑子里有块榆木疙瘩,嗓子里有一口痰。我认为在器物的背后是人的方法与技能,在方法与技能的背后是人对自然的了解,在人对自然了解的背后,是人类了解现在、过去与未来的万丈雄心。按老派人士的说法,它该叫做“之用之用之用之用”,是末节的末节。一个人假如这样看待人类最高尚的品行,何止是可耻,简直是可杀。而区区的物品,却可以叫“之用”,和人亲近了很多。总而言之,以自己为中心,只要好处;由此产生的狼心狗肺的说法,肯定可以把法拉第、爱迪生等人气得在坟墓里打滚。 

  在西方的智慧里,怎样发明电动机,是个已经解决了的问题,所以才会有电动机。罗素先生就说,他赞成不计成败利钝地追求客观真理。这话还是有点绕。我觉得西方的智者有一股不管三七二十一,总要把自己往聪明里弄的劲头儿,为了变得聪明,就需要种种知识。不管电磁感应有没有用,我们先知道了再说。换言之,追求智慧与利益无干?这是一种兴趣。现代文明的特快列车竟发轫于一种兴趣,说来叫人不能相信,但恐怕真是这样。 

  中国人还认为,求学是痛苦的,学海无涯苦做舟。学童不仅要背四书五经,还要挨戒尺板子,仅仅是因为考虑到他们的承受力,才没有动用老虎凳。学习本身很痛苦,必须以更大的痛苦为推动力,和调教牲口没有本质的区别。当然,夫子曾说,学而时习之,不亦说乎?但他老人家是圣人,和我们不一样。再说,也没人敢打他的板子。从书上看,孟子曾从思辨中得到一些快乐。但春秋以后到近代,再没有中国人敢说学习是快乐的了。一切智力的活动都是如此,谁要说动脑子有乐趣,最轻的罪名也是不严肃——顺便说一句,我认为最严肃的东西是老虎凳,对坐在上面的人来说,更是如此。据我所知,有些外国人不是这样看问题。维特根斯坦在临终时,回顾自己一生的智力活动时说:告诉他们,我度过了美好的一生。还有一个物理学家说:我就要死了,带上两道难题去问上帝。在天堂里享受永生的快乐他还嫌不够,还要在那里讨论物理!总的来说,学习一事,在人家看来快乐无比,而在我们眼中则毫无乐趣,如同一个太监面对后宫佳丽。如此看来,东西方两种智慧的区别,不仅是驴和马的区别,而且是叫驴和骟马的区别。那东西怎么就没了,真是个大问题! 

  作为驴子之友,我对爱马的人也有一种敬意。通过刻苦的修炼来完善自己,成为一个敬祖宗畏鬼神、俯仰皆能无愧的好人,这种打算当然是好的。惟一使人不满意的是,这个好人很可能是个笨蛋。直愣愣地想什么东西有什么用处,这是任何猿猴都有的想法。只有一种特殊的裸猿(也就是人类),才会时时想到“我可能还不够聪明”!所以,我不满意爱马的人对这个问题的解答。也许在这个问题上可以提出一个骡子式的折衷方案:你只有变得更聪明,才能看到人间的至善。但我不喜欢这样的答案。我更喜欢驴子的想法:智慧本身就是好的。有一天我们都会死去,追求智慧的道路还会有人在走着。死掉以后的事我看不到。但在我活着的时候,想到这件事,心里就很高兴。 


                 二 

  物理学家海森堡给上帝带去的那两道难题是相对论和湍流。他还以为后一道题太难,连上帝都不会。我也有一个问题,但我不想向上帝提出,那就是什么是智慧。假如这个问题有答案,也必定在我的理解范围之外。当然,不是上帝的人对此倒有些答案,但我总是不信。相比之下我倒更相信苏格拉底的话:我只知道自己一无所知。罗素先生说,虽然有科学上的种种成就,但我们所知甚少,尤其是面对无限广阔的未知,简直可以说是无知的。与罗素的注释相比,我更喜欢苏格拉底的那句原话,这句话说得更加彻底。他还有些妙论我更加喜欢:只有那些知道自己智慧一文不值的人,才是最有智慧的人。这对某种偏向是种解毒剂。 

  如果说我们都一无所知,中国的读书人对此肯定持激烈的反对态度:孔夫子说自己知天命而且不逾矩,很显然,他不再需要知道什么了。后世的人则以为:天已经生了仲尼,万古不长如夜了。再后来的人则以为,精神原子弹已经炸过,世界上早没有了未解决的问题。总的来说,中国人总要以为自己有了一种超级的知识,博学得够够的、聪明得够够的,甚至巴不得要傻一些。直到现在,还有一些人以为,因为我们拥有世界上最博大精深的文化遗产,可以坐待世界上一切寻求智慧者的皈依——换言之,我们不仅足够聪明,还可以担任联合国救济署的角色,把聪明分绐别人一些。我当然不会反对这样说:我们中国人是全世界、也是全宇宙最聪明的人。一种如此聪明的人,除了教育别人,简直就无事可干。 

  马克·吐温在世时,有一次遇到了一个人,自称能让每个死人的灵魂附上自己的体。他决定通过这个人来问候一下死了的表兄,就问道:你在哪里?死表哥通过活着的人答道:我在天堂里。当然,马克·吐温很为表哥高兴,但问下去就不高兴了——你现在喝什么酒?灵魂答道:在天堂里不喝酒。又问抽什么烟?回答是不抽烟。再问干什么?答案是什么都不干,只是谈论我们在人间的朋友,希望他们到这里和我们相会。这个处境和我们有点相像,我们这些人现在就无事可干,只能静待外国物质文明破产,来投靠我们的东方智慧。这话梁任公1920年就说过,现在还有人说。洋鬼子在物质堆里受苦,我们享受天人合一的大快乐,正如在天堂里的人闲着没事拿人间的朋友磕磕牙,我们也有了机会表示自己的善良了。说实在的,等人来这点事还是洋鬼子给我们找的。要不是达·伽马找到好望角绕了过来,我们还真闲着没事于。从汉代到近代,全中国那么多聪明人,可不都在闲着:人文学科弄完了,自然科学没的弄。马克·吐温的下一个问题,我国的一些人文学者就不一定爱听了:等你在人间的朋友们都死掉,来到了你那里,再谈点什么?是啊是啊,全世界的人都背弃了物质文明,投奔了我们,此后再干点什么?难道重操旧业,去弄八股文?除此之外,再搞点考据、训诂什么的。过去的读书人有这些就够了,而现在的年轻人未必受得了。把拥有这种超级智慧比作上天堂,马克·吐温的最后一个问题深得我心:你是知道我的生活方式的,有什么方法能使我不上天堂而下地狱,我倒很想知道!言下之意是:忍受地狱毒火的煎熬,也比闲了没事要好。是啊是啊!我宁可做个苏格拉底那样的人,自以为一无所知,体会寻求知识的快乐,也不肯做个“智慧满盈”的儒士,忍受这种无所事事的煎熬! 


                 三 

  我有位阿姨,生了个傻女儿,比我大几岁,不知从几岁开始学会了缝扣子。她大概还学过些别的,但没有学会。总而言之,这是她惟一的技能。我到她家去坐时,每隔三到五分钟,这傻丫头都要对我狂嚎一声:“我会缝扣子!”我知道她的意思:她想让我向她学缝扣子。但我就是不肯,理由有二:其一,我自己会缝扣子;其二,我怕她扎着我。她这样爱我,让人感动。但她身上的味也很难闻。 

  我在美国留学时,认得一位青年,叫做戴维。我看他人还不错,就给他讲解中华文化的真谛,什么忠孝、仁义之类。他听了居然不感动,还说:“我们也爱国。我们也尊敬老年人。这有什么?我们都知道!”我听了不由得动了邪火,真想扑上去咬他。之所以没有咬,是因为想起了傻大姐,自觉得该和她有点区别,所以悻悻然地走开,心里想道:妈的!你知道这些,还不是从我们这里知道的。礼义廉耻,洋人所知没有我们精深,但也没有儿奸母、子食父、满地拉屎。东方文化里所有的一切,那边都有,之所以没有投入全身心来讲究,主要是因为人家还有些别的事情。 

  假如我那位傻大姐学会了一点西洋学术,比方说,几何学,一定会跳起来大叫道:人所以异于禽兽者,几稀!这东西就是几何学!这话不是没有道理,的确没有哪种禽兽会几何学。那时她肯定要逼我跟她学几何,如果我不肯跟她学,她定要说我是禽兽之类,并且责之以大义。至于我是不是已经会了一些,她就不管了。我的意思当然不是说她能学会这东西,而是说她只要会了任伺一点东西,都会当作超级智慧,相比之下那东西是什么倒无所谓。由这件事我想到超级知识的本质。这种东西罗素和苏格拉底都学不会,我学起来也难。任何知识本身,即便烦难,也可以学会。难就难在让它变成超级,从中得到大欢喜、大欢乐,无限的自满、自足、手之舞之足之蹈之的那种品行。这种品行我的那位傻大姐身上最多,我身上较少。至于罗素、苏格拉底两位先生,他们身上一点都没有。 

  傻大姐是个知识的放大器,学点东西极苦,学成以后极乐。某些国人对待国学的态度与傻大姐相近。说实在的,他们把它放得够大了。拉封丹寓言里,有一则《大山临盆》,内容如下:大山临盆,天为之崩,地为之裂,日月星辰,为之无光。房倒屋坍,烟尘滚滚,天下生灵,死伤无数……最后生下了一只耗子。中国的人文学者弄点学问,就如大山临盆一样壮烈。当然,我说的不止现在,而且有过去,还有未来。 

  正如迂夫子不懂西方的智慧,也能对它品头论足一样,罗素没有手舞足蹈的品行,但也能品出其中的味道——大概把对自己所治之学的狂热感情视作学问本身乃是一种常见的毛病,不独中国人犯,外国人也要犯。他说:人可能认为自己有无穷的财源,而且这种想法可以让他得到一些(何止是一些!罗素真是不懂——王注)满足。有人确实有这种想法,但银行经理和法院一般不会同意他们。银行里有账目,想骗也骗不成;至于在法院里,我认为最好别吹牛,搞不好要进去的。远离这两个危险的场所,躲在人文学科的领域之内,享受自满自足的大快乐,在目前还是可以的;不过要有人养。在自然科学里就不行:这世界上每年都有人发明永动机,但谁也不能因此发财。顺便说一句,我那位傻大姐,现在已经五十岁了,还靠我那位不幸的阿姨养活着。 


(本篇最初发表于1995年第11期《读书》杂志。)

\chapter{思维的乐趣}

                 一 

  二十五年前,我到农村去插队时,带了几本书,其中一本是奥维德的《变形记》,我们队里的人把它翻了又翻,看了又看,以致它像一卷海带的样子。后来别队的人把它借走了,以后我又在几个不同的地方见到了它,它的样子越来越糟。我相信这本书最后是被人看没了的。现在我还忘不了那本书的惨状。插队的生活是艰苦的,吃不饱,水土不服,很多人得了病,但是最大的痛苦是没有书看,倘若可看的书很多的话,《变形记》也不会这样悲惨地消失了。除此之外,还得不到思想的乐趣。我相信这不是我一个人的经历:傍晚时分,你坐在屋檐下,看着天慢慢地黑下去,心里寂寞而凄凉,感到自己的生命被剥夺了。当时我是个年轻人,但我害怕这样生活下去,衰老下去。在我看来,这是比死亡更可怕的事。 

  我插队的地方有军代表管着我们,现在我认为,他们是一批单纯的好人,但我还认为,在我这一生里,再没有谁比他们使我更加痛苦过了。他们认为,所谓思想的乐趣,就是一天二十四小时都用毛泽东思想来占领,早上早请示,晚上晚汇报,假如有闲暇,就去看看说他们自己“亚古都”的歌舞。我对那些歌舞本身并无意见,但是看过二十遍以后就厌倦了。假如我们看书被他们看到了,就是一场灾难,甚至“著迅鲁”的书也不成——小红书当然例外。顺便说一句,还真有人因为带了旧版的鲁迅著作给自己带来了麻烦。有一个知识可能将来还有用处,就是把有趣的书换上无趣的皮。我不认为自己能够在一些宗教仪式中得到思想的乐趣,所以一直郁郁寡欢。像这样的故事有些作者也写到过,比方说,茨威格写过一部以此为题材的小说《象棋》,可称是现代经典,但我不认为他把这种痛苦描写得十全十美了。这种痛苦的顶点不是被拘押在旅馆里没有书看、没有合格的谈话伙伴,而是被放在外面,感到天地之间同样寂寞,面对和你一样痛苦的同伴。在我们之前,生活过无数的大智者,比方说,罗素、牛顿、莎士比亚,他们的思想和著述可以使我们免于这种痛苦,但我们和他们的思想、著述,已经被隔绝了。一个人倘若需要从思想中得到快乐,那么他的第一个欲望就是学习。我承认,我在抵御这种痛苦方面的确是不够坚强,但我绝不是最差的一个。举例言之,罗素先生在五岁时,感到寂寞而凄凉,就想道:假如我能活到七十岁,那么我这不幸的一生才度过了十四分之一!但是等他稍大一点,接触到智者的思想的火花,就改变了想法。假设他被派去插队,很可能就要自杀了。 

  谈到思想的乐趣,我就想到了我父亲的遭遇。我父亲是一位哲学教授,在五六十年代从事思维史的研究。在老年时,他告诉我自己一生的学术经历,就如一部恐怖电影。每当他企图立论时,总要在大一统的官方思想体系里找自己的位置,就如一只老母鸡要在一个大搬家的宅院里找地方孵蛋一样。结果他虽然热爱科学而且很努力,在一生中却没有得到思维的乐趣,只收获了无数的恐慌。他一生的探索,只剩下了一些断壁残垣,收到一本名为《逻辑探索》的书里,在他身后出版。众所周知,他那一辈的学人,一辈子能留下一本书就不错。这正是因为在那些年代,有人想把中国人的思想搞得彻底无味。我们这个国家里,只有很少的人觉得思想会有乐趣,却有很多的人感受过思想带来的恐慌,所以现在还有很多人以为,思想的味道就该是这样的。 


                 二 

  “文化革命”之后,我读到了徐迟先生写哥德巴赫猜想的报告文学,那篇文章写得很浪漫。一个人写自己不懂得的事就容易这样浪漫。我个人认为,对于一个学者来说,能够和同行交流,是一种起码的乐趣。陈景润先生一个人在小房子里证数学题时,很需要有些国外的数学期刊可看,还需要有机会和数学界的同仁谈谈。但他没有,所以他未必是幸福的,当然他比没定理可证的人要快活。把一个定理证了十几年,就算证出时有绝大的乐趣,也不能平衡。但是在寂寞里枯坐就更加难熬。假如插队时,我懂得数论,必然会有陈先生的举动,而且就是最后什么都证不出也不后悔;但那个故事肯定比徐先生作品里描写的悲惨。然而,某个人被剥夺了学习、交流、建树这三种快乐,仍然不能得到我最大的同情。这种同情我为那些被剥夺了“有趣”的人保留着。 

  “文化革命”以后,我还读到了阿城先生写知青下棋的小说,这篇小说写得也很浪漫。我这辈子下过的棋有五分之四是在插队时下的,同时我也从一个相当不错的棋手变成了一个无可救药的庸手。现在把下棋和插队两个词拉到一起,就能引起我生理上的反感。因为没事干而下棋,性质和手淫差不太多。我决不肯把这样无聊的事写讲小说里。 

  假如一个人每天吃一样的饭,干一样的活,再加上把八个样板戏翻过来倒过去地看,看到听了上句知道下句的程度,就值得我最大的同情。我最赞成罗素先生的一句话:“须知参差多态,乃是幸福的本源。”大多数的参差多态都是敏于思索的人创造出来的。当然,我知道有些人不赞成我们的意见。他们必然认为,单一机械,乃是幸福的本源。老子说,要让大家“虚其心而实其腹”,我听了就不是很喜欢;汉儒废黜百家,独尊儒术,在我看来是个很卑鄙的行为。摩尔爵士设想了一个细节完备的乌托邦,但我像罗素先生一样,决不肯到其中去生活。在这个名单的末尾是一些善良的军代表,他们想把一切从我头脑中驱除出去,只剩一本270 页的小红书。在生活的其他方面,某种程度的单调、机械是必须忍受的,但是思想决不能包括在内。胡思乱想并不有趣,有趣是有道理而且新奇。在我们生活的这个世界上,最大的不幸就是有些人完全拒绝新奇。 

  我认为自己体验到最大快乐的时期是初进大学时,因为科学对我来说是新奇的,而且它总是逻辑完备,无懈可击,这是这个平凡的尘世上罕见的东西。与此同时,也得以了解先辈科学家的杰出智力。这就如和一位高明的棋手下棋,虽然自己总被击败,但也有机会领略妙招。在我的同学里,凡和我同等年龄、有同等经历的人,也和我有同样的体验。某些单调机械的行为,比如吃、排泄、性交,也能带来快感,但因为过于简单,不能和这样的快乐相比。艺术也能带来这样的快乐,但是必须产生于真正的大师,像牛顿、莱布尼兹、爱因斯坦那样级别的人物,时下中国的艺术家,尚没有一位达到这样的级别。恕我直言,能够带来思想快乐的东西,只能是人类智慧至高的产物。比这再低一档的东西,只会给人带来痛苦;而这种低档货,就是出于功利的种种想法。 


                 三                  

  有必要对人类思维的器官(头脑)进行“灌输”的想法,时下正方兴未艾。我认为脑子是感知至高幸福的器官,把功利的想法施加在它上面,是可疑之举。有一些人说它是进行竞争的工具,所以人就该在出世之前学会说话,在三岁之前背诵唐诗。假如这样来使用它,那么它还能获得什么幸福,实在堪虞。知识虽然可以带来幸福,但假如把它压缩成药丸子灌下去,就丧失了乐趣。当然,如果有人乐意这样来对待自己的孩子,那不是我能管的事,我只是对孩子表示同情而已。还有人认为,头脑是表示自己是个好人的工具,为此必须学会背诵一批格言、教条——事实上,这是希望使自己看上去比实际上要好,十足虚伪。这使我感到了某种程度的痛苦,但还不是不能忍受的。最大的痛苦莫过于总有人想要用种种理由消灭幸福所需要的参差多态。这些人想要这样做,最重要的理由是道德;说得更确切些,是出于功利方面的考虑。因此他们就把思想分门别类,分出好的和坏的,但所用的标准很是可疑。他们认为,假如人们脑子里灌满了好的东西,天下就会太平。因此他们准备用当年军代表对待我们的态度,来对待年轻人。假如说,思想是人类生活的主要方面,那么,出于功利的动机去改变人的思想,正如为了某个人的幸福把他杀掉一样,言之不能成理。 

  有些人认为,人应该充满境界高尚的思想,去掉格调低下的思想。这种说法听上去美妙,却使我感到莫大的恐慌。因为高尚的思想和低下的思想的总和就是我自己;倘若去掉一部分,我是谁就成了问题。假设有某君思想高尚,我是十分敬佩的;可是如果你因此想把我的脑子挖出来扔掉,换上他的,我绝不肯,除非你能够证明我罪大恶极,死有余辜。人既然活着,就有权保证他思想的连续性,到死方休。更何况那些高尚和低下完全是以他们自己的立场来度量的,假如我全盘接受,无异于请那些善良的思想母鸡到我脑子里下蛋,而我总不肯相信,自己的脖子上方,原来是长了一座鸡窝。想当年,我在军代表眼里,也是很低下的人,他们要把自己的思想方法、生活方式强加给我,也是一种脑移植。菲尔丁曾说,既善良又伟大的人很少,甚至是绝无仅有的,所以这种脑移植带给我的不光是善良,还有愚蠢。在此我要很不情愿地用一句功利的说法:在现实世界上,蠢人办不成什么事情。我自己当然希望变得更善良,但这种善良应该是我变得更聪明造成的,而不是相反。更合况赫拉克利特早就说过,善与恶为一,正如上坡和下坡是同一条路。不知道何为恶,焉知伺为善?所以他们要求的,不过是人云亦云罢了。 

  假设我相信上帝(其实我是不信的),并且正在为善恶不分而苦恼,我就会请求上帝让我聪明到足以明辨是非的程度,而绝不会请他让我愚蠢到让人家给我灌输善恶标准的程度。假若上帝要我负起灌输的任务,我就要请求他让我在此项任务和下地狱中做—选择,并且我坚定不移的决心是:选择后者。 


                 四                  

  假如要我举出一生最善良的时刻,那我就要举出刚当知青时,当时我一心想要解放全人类,丝毫也没有想到自己。同时我也要承认,当时我愚蠢得很,所以不仅没干成什么事情,反而染上了一身病,丢盔卸甲地逃回城里。现在我认为,愚蠢是一种极大的痛苦;降低人类的智能,乃是一种最大的罪孽。所以,以愚蠢教人,那是善良的人所能犯下的最严重的罪孽。从这个意义上说,我们决不可对善人放松警惕。假设我被大奸大恶之徒所骗,心理还能平衡;而被善良的低智人所骗,我就不能原谅自己。 

  假如让我举出自己最不善良的时刻,那就是现在了。可能是因为受了一些教育,也可能是因为已经成年,反正你要让我去解放什么人的话,我肯定要先问问,这些人是谁,为什么需要帮助;其次要问问,帮助他们是不是我能力所及;最后我还要想想,自己直奔云南去挖坑,是否于事有补。这样想来想去,我肯定不愿去插队。领导上硬要我去,我还得去,但是这以后挖坏了青山、造成了水土流失等等,就罪不在我。一般人认为,善良而低智的人是无辜的。假如这种低智是先天造成的,我同意。但是人可以发展自己的智力,所以后天的低智算不了无辜——再说,没有比装傻更便当的了。当然,这结论绝不是说当年那些军代表是些装傻的奸邪之辈——我至今相信他们是好人。我的结论是:假设善恶是可以判断的,那么明辨是非的前提就是发展智力,增广知识。然而,你劝一位自以为已经明辨是非的人发展智力,增广见识,他总会觉得你让他舍近求远,不仅不肯,还会心生怨恨。我不愿为这样的小事去得罪人。 

  我现在当然有自己的善恶标准,而且我现在并不比别人表现得坏。我认为低智、偏执、思想贫乏是最大的邪恶。按这个标准,别人说我最善良,就是我最邪恶时;别人说我最邪恶,就是我最善良时。当然我不想把这个标准推荐给别人,但我认为,聪明、达观、多知的人,比之别样的人更堪信任。基于这种信念,我认为我们国家在“废黜百家,独尊儒术”之后,就丧失了很多机会。 

  我们这个民族总是有很多的理由封锁知识、钳制思想、灌输善良,因此有很多才智之士在其一生中丧失了学习、交流、建树的机会,没有得到思想的乐趣就死掉了。想到我父亲就是其中的一个,我就心中黯然;想到此类人士的总和有恒河沙数之多,我就趋向于悲观。此种悲剧的起因,当然是现实世界里存在的种种问题。伟大的人物总认为,假设这世界上所有的人都像他期望的那样善良——更确切地说,都像他期望的那样思想,“思无邪”,或者“狠斗私字一闪念”,世界就可以得救。提出这些说法的人本身就是无邪或者无私的,他们当然不知邪和私是什么,故此这些要求就是:我没有的东西,你也不要有。无数人的才智就此被扼杀了。考虑到那恒河沙数才智之土的总和是一种难以想象的庞大资源,这种想法就是打算把整个大海装入一个瓶子之中。我所看到的事实是,这种想法一直在实行中,也就是说,对于现实世界的问题,从愚蠢的方面找办法。据此我认为,我们国家自汉代以后,一直在进行思想上的大屠杀;而我能够这样想,只说明我是幸存者之一。除了对此表示悲伤之外,我想不到别的了。 


                 五 

  我虽然已活到了不惑之年,但还常常为一件事感到疑惑:为什么有很多人总是这样的仇恨新奇,仇恨有趣。古人曾说:天不生仲尼,万古长如夜。但我有相反的想法。假设历史上曾有一位大智者,一下发现了一切新奇、一切有趣,发现了终极真理,根绝子一切发现的可能性,我就情愿到该智者以前的年代去生活,这是因为,假如这种终极真理已经被发现,人类所能做的事就只剩下了依据这种真理来做价值判断。从汉代以后到近代,中国人就是这么生活的。我对这样的生活一点都不喜欢。 

  我认为,在人类的一切智能活动里,没有比做价值判断更简单的事了。假如你是只公兔子,就有做出价值判断的能力——大灰狼坏,母兔子好;然而兔子就不知道九九表。此种事实说明,一些缺乏其他能力的人,为什么特别热爱价值的领域。倘若对自己做价值判断,还要付出一些代价;对别人做价值判断,那就太简单、太舒服了。讲出这样粗暴的话来,我的确感到羞愧,但我并不感到抱歉。因为这种人士带给我们的痛苦实在太多了。 

  在一切价值判断之中,最坏的一种是:想得太多、太深奥、超过了某些人的理解程度是—种罪恶。我们在体验思想的快乐时,并没有伤害到任何人;不幸的是,总有人觉得自己受了伤害。诚然,这种快乐不是每一个人都能体验到的,但我们不该对此负责任。我看不出有什么理由要取消这种快乐,除非把卑鄙的嫉妒计算在内 ——这世界上有人喜欢丰富,有人喜欢单纯;我未见过喜欢丰富的人妒恨、伤害喜欢单纯的人,我见到的情形总是相反。假如我对科学和艺术稍有所知的话,它们是源于思想乐趣的浩浩江河,虽然惠及一切人,但这江河决不是如某些人所想象的那样,为他们而流,正如以思想为乐趣的人不是为他们而生一样。 

  对于一位知识分子来说,成为思维的精英,比成为道德精英更为重要。人当然有不思索、把自己变得愚笨的自由;对于这一点,我是一点意见都没有的。问题在于思索和把自己变聪明的自由到底该不该有。喜欢前一种自由的人认为,过于复杂的思想会使人头脑昏乱,这听上去似乎有些道理。假如你把深山里一位质朴的农民请到城市的化工厂里,他也会因复杂的管道感到头晕,然而这不能成为取消化学工业的理由。所以,质朴的人们假如能把自己理解不了的事情看作是与己无关的事,那就好了。 

  假如现在我周围的世界又充满了“文革”时的军代表和道德教师,只能使我惊,不能使我惧。因为我已经活到了四十二岁。我在大学里遇到了把知识当作幸福来传播的数学教师,他使学习数学变成了一种乐趣。我遇到了启迪我智慧的人。我有幸读到了我想看的书——这个书单很是庞杂,从罗素的《西方哲学史》,一直到英国维多利亚时期的地下小说。这最后一批书实在是很不堪的,但我总算是把不堪的东西也看到了。当然,我最感谢的是那些写了好书的人,比方说,萧伯纳、马克。吐温、卡尔维诺、杜拉斯等等,但对那些写了坏书的人也不怨恨。我自己也写了几本书,虽然还没来得及与大陆读者见面,但总算获得了一点创作的快乐。这些微不足道的幸福就能使我感到在一生中稍有所得,比我父亲幸福,比那些将在思想真空里煎熬一世的年轻人幸福。作为一个有过幸福和痛苦两种经历的人,我期望下一代人能在思想方面有些空间来感到幸福,而且这种空间比给我的大得多。而这些呼吁当然是对那些立志要当军代表和道德教师的人而发的。

\footnote{本篇最初发表于1994年第9 期《读书》杂志。
《思维的乐趣》是李银河编选的一本王小波的杂文集,也是他生前出版的最后一本书。其中包括《沉默的大多数》、《理想国与哲人王》、《一只特立独行的猪》、《我的精神家园》等名篇。书中作者以特有的幽默的口吻,独到地阐释了自己对当代中国思想领域种种的反思。 

《思维的乐趣》,北岳文艺出版社,太原,1996,ISBN 7-5378-1650-6 
《思维的乐趣》,中国人民大学出版社,北京,2005,ISBN 7-300-06539-2
}

\chapter{花刺子模信使问题}

  据野史记载,中亚古国花刺子模有一古怪的风俗,凡是给君王带来好消息的信使,就会得到提升,给君王带来坏消息的人则会被送去喂老虎。于是将帅出征在外,凡麾下将士有功,就派他们给君王送好消息,以使他们得到提升;有罪,则派去送坏消息,顺便给国王的老虎送去食物。花剌子模是否真有这种风俗并不重要,重要的是这个故事所具有的说明意义,对它可以举一反三。敏锐的读者马上就能发现,花刺子模的君王有一种近似天真的品性,以为奖励带来好消息的人,就能鼓励好消息的到来,处死带来坏消息的人,就能根绝坏消息。另外,假设我们生活在花刺子模,是一名敬业的信使,倘若有一天到了老虎笼子里,就可以反省到自己的不幸是因为传输了坏消息。最后,你会想到,我讲出这样一个古怪故事,必定别有用心。对于这最后一点,必须首先承认。 

  从某种意义上说,学者的形象和花刺子模信使有相像处,但这不是说他有被吃掉的危险。首先,他针对研究对象,得出有关的结论,这时还不像信使;然后,把所得的结论报告给公众,包括当权者,这时他就像个信使;最后,他从别人的反应中体会到自己的结论是否受欢迎,这时候他就像个花刺子模的信使。中国的近现代学者里,做“好消息信使”的人很多,尤其是人文学者。比方说,现在大家发现了中华文化是最好的文化,世界的前途倚赖东方文明。不过也有“坏消息信使”,此人叫做马寅初。五十年代初,马寅初提出了新人口论。当时以为,只要把马老臭批一顿,就可以根绝中国的人口问题,后来才发现,问题不是这么简单。 

  假如学者能知道自己报告的是好消息还是坏消息,这问题也就简单了。这方面有一个例子是我亲身所历。我和李银河从 1989 年开始一项社会学研究,首次发现了中国存在着广泛的同性恋人群,并且有同性恋文化。当时以为这个发现很有意义,就把它报道出来,结果不但自己倒了霉,还带累得一家社会学专业刊物受到本市有关部门的警告。这还不算,还惊动了该刊一位顾问(八十多岁的老先生),连夜表示要不当顾问。此时我们才体会到这个发现是不受欢迎的,读者可以体会到我们此时是多么的惭愧和内疚。假设禁止我们出书,封闭有关社会学杂志,就可以使中国不再出现同性恋问题,这些措施就有道理。但同性恋倾向是遗传的,封刊物解决不了问题,所以这些措施一点道理都没有。值得庆幸的是,北京动物园的老虎当时不缺肉吃。由此得出花刺子模信使问题第一个结论是:对于学者来说,研究的结论会不会累及自身,是个带有根本性的问题。这主要取决于在学者周围有没有花刺子模君王类的人。 

  假设可以对花刺子模君王讲道理,就可以说,首先有了不幸的事实,然后才有不幸的信息,信使是信息的中介,尤其的无辜。假如要反对不幸,应该直接反对不幸的事实,此后才能减少不幸的信息。但是这个道理有一定的复杂性,不是君王所能理解。再说,假如能和他讲理,他就不是君王。君王总是对的,臣民总是不对。君王的品性不可更改,臣民就得适应这种现实。假如花剌子模的信使里有些狡猾之徒,递送坏消息时就会隐瞒不报,甚至滥加篡改。鲁迅先生有篇杂文,谈到聪明人和傻子的不同遭遇,讨论的就是此类现象。据我所知,学者没有狡猾到这种程度,他们只是仔细提防着自己,不要得出不受欢迎的结论来。由于日夜提防,就进入了一种迷迷糊糊的心态,乃是深度压抑所致。与此同时,人人都渴望得到受欢迎的结论,因此连做人都不够自然。现在人们所说的人文科学的危机,我以为主要起因于此。还有一个原因在经济方面——挣钱太少。假定可以痛快淋漓地做学问,再挣很多的钱,那就什么危机都没有了。 

  我个人认为,获得受欢迎的信息有三种方法:其一,从真实中索取、筛选;其二,对现有的信息加以改造;其三,凭空捏造。第一种最困难。第三种最为便利,在这方面,学者有巨大的不利之处,那就是凭空捏造不如奸佞之徒。假定有君王专心要听好消息,与其养学者,不如养一帮无耻小人。在中国历史上,儒士的死敌就是宦官。假如学者下海去改造、捏造信息,对于学术来说,是一种自杀之道。因此学者往往在求真实和受欢迎之中,苦苦求索一条两全之路,文史学者尤其如此。我上大学时,老师教诲我们说,搞现代史要牢记两个原则,一是治史的原则,二是党性的原则。这就是说,让历史事实按党性的原则来发生。凭良心说,这节课我没听懂。在文史方面,我搞不清的东西还很多。不过我也能体会到学者的苦心。 

  在中国历史上,每一位学者都力求证明自己的学说有巨大经济效益、社会效益。孟子当年鼓吹自己的学说,提出了“仁者无敌”之说,有了军事效益,和林彪的 “精神原子弹”之说有异曲同工之妙。学术必须有效益,这就构成了另一种花刺子模。学术可以有实在的效益,不过来得极慢,起码没有嘴头上编出来的效益快;何况对于君主来说,“效益”就是一些消息而已。最好的效益就是马上能听见的好消息。因为这个原因,学者们承受着一种压力,要和骗子竞赛语惊四座,看着别人的脸色做学问,你要什么我做什么。必须说明的是,学者并没有完全变狡猾,这一点我还有把握。 

  假如把世界上所有的学者对本学科用途的说明做一比较,就可发现大致可以分为两种,一种说:科学可以解决问题,但就如中药铺里的药材可以给人治病一样,首先要知识完备,然后才能按方抓药,治人的病。照这种观点,我们现在所治之学,只是完备药店的药材,对它能治什么病不做保证。另一种说道,本人所治之学对于现在人类所遇到的问题马上就有答案,这就如卖大力丸的,这种丸药百病通治,吃下去有病治病,无病强身。中国的学者素来有卖大力丸的传统,喜欢做妙语以动天听。这就造成了一种气氛,除了大力丸式的学问,旁的都不是学问。在这种压力之下,我们有时也想做几句惊人之语,但又痛感缺少想象力。 

  我记得冯友兰先生曾提出要修改自己的《中国哲学史》,以便迎合时尚和领袖,这是变狡猾的例子——罗素先生曾写了一本《西方哲学史》,从未提出为别人做修改,所以冯先生比罗素狡猾——但是再滑也滑不过佞人。从学问的角度来看,冯先生已做了最大的牺牲,但上面也没看在眼里。佞人不做学问,你要什么我编什么,比之学人利索了很多——不说是天壤之别,起码也有五十步与百步之分。二三十年前,一场红海洋把文史哲经通通淹没。要和林彪比滑头,大伙都比不过,人文学科的危机实质上在那时就已发生了。 

  罗素先生修西方哲学史,指出很多伟大的学者都有狡猾的一面(比如说,莱布尼兹),我仔细回味了一下,也发现了一些事例,比如牛顿提出了三大定理之后,为什么要说上帝是万物运动的第一推动力?显然也是朝上帝买个好。万一他真的存在,死后见了面也好说话。按这种标准我国的圣贤滑头的事例更多,处处在拍君王的马屁,仔细搜集可写本《中国狡猾史》。中国古代的统治者都带点花刺子模君王气质,我国的文化传统里有“文死谏”之说,这就是说,中国常常就是花刺子模,这种传统就是号召大家做敬业的信使,拿着屁股和脑壳往君王的刀于板子上撞。很显然,只要不是悲观厌世,谁也不喜欢牺牲自己的脑袋和屁股。所以这种号召也是出于滑头分子之口,变着法说君王有理,这样号召只会起反作用。对于我国的传统文化、现代文化,只从诚实的一面理解是不够的,还要从狡猾的一面来理解。扯到这里,就该得出第二个结论:花剌子模的信使早晚要变得滑头起来,这是因为人对自己的处境有适应能力。以我和李银河为例,现在就再不研究同性恋问题了。 

  实际上,不但是学者,所有的文化人都是信使,因为他们产出信息,而且都不承认这些信息是自己随口编造的,以此和佞人有所区别。大家都说这些信息另有所本,有人说是学术,有人说是艺术,还有人说自己传播的是新闻。总之,面对公众和领导时,大家都是信使,而且都要耍点滑头:拣好听的说或许不至于,起码都在提防着自己不要讲出难听的来——假如混得不好,就该检讨一下自己的嘴是不是不够甜。有关信使,我们就讲这么多。至于君主,我以为可以分为两种,一种是粗暴型的君主,听到不顺耳的消息就拿信使喂老虎;另一种是温柔型,到处做信使们的思想工作,使之自觉自愿地只报来受欢迎的消息。这样他所管理的文化园地里,就全是使人喜闻乐见的东西了。这后一种君主至今是我们怀念的对象。凭良心说,我觉得这种怀念有点肉麻,不过我也承认,忍受思想工作,即便是耐心细致的思想工作,也比喂老虎好过得多。 

  在得出第三个结论之前,还有一点要补充的——有句老话叫做“久居鲍鱼之肆不闻其臭”,这就是说,人不知自己是不是身在花刺子模,因此搞不清自己是不是有点滑头,更搞不清自己以为是学术、艺术的那些东西到底是真是假。不过,我知道假如一个人发现自己进了老虎笼子,那么就可以断言,他是个真正的信使。这就是第三个结论。余生也晚,赶不上用这句话去安慰马寅初先生,也赶不上去安慰火刑架上的布鲁诺,不过这话留着总有它的用处。 

  现在我要得出最后一个结论,那就是说,假设有真的学术和艺术存在的话,在人变得滑头时它会离人世远去,等到过了那一阵子,人们又可以把它召唤回来—— 此种事件叫做“文艺复兴”。我们现在就有召唤的冲动,但我很想打听一下召唤什么。如果是召唤古希腊,我就赞成,如果是召唤花刺子模,我就反对。我相信马寅初这样的人喜欢古希腊,假如他是个希腊公民,就会在城邦里走动,到处告诉大家:现在人口太多,希望朋友们节制一下,要是滑头分子,就喜欢花刺子模,在那里他营造出了好消息,更容易找到买主。恕我说得难听,现在的人文知识分子在诚恳方面没几个能和马老相比。所以他们召唤的东西是什么,我连打听都不敢打听。

\footnote{本篇最初发表于1995年第3 期《读书》杂志。}

\chapter{知识分子的不幸}

  乔叟《坎特伯雷故事集》里,有这样一个故事,有位武士犯了重罪,国王把他交给王后处置。王后命他回答一个问题:什么是女人最大的心愿?这位武士当场答不上来,王后给了他一个期限,到期再答不上来,就砍他的脑袋。于是,这位武士走遍天涯去寻求答案。最后终于找到了,保住了自己的头;假如找不到,也就不成其为故事。据说这个答案经全体贵妇讨论,一致认为正确,就是:“女人最大的心愿就是有人爱她。”要是在今天,女权主义者可能会有不同看法,但在中世纪,这答案就可以得满分啦。 

  我也有一个问题,是这样的:什么是知识分子最害怕的事?而且我也有答案,自以为经得起全球知识分子的质疑,那就是:“知识分子最怕活在不理智的年代。”所谓不理智的年代,就是伽利略低头认罪,承认地球不转的年代,也是拉瓦锡上断头台的年代;是茨威格服毒自杀的年代,也是老舍跳进太平湖的年代。我认为,知识分子的长处只是会以理服人,假如不讲理,他就没有长处,只有短处,活着没意思,不如死掉。丹麦王子哈姆雷特说:活着呢,还是死去,这是问题。但知识分子赶上这么个年代,死活不是问题。最大的问题是:这个倒霉的年头儿何时过去。假如能赶上这年头过去,就活着;赶不上了就犯不着再拖下去。老舍先生自杀的年代,我已经懂事了,认识不少知识分子。虽然我当时是个孩子,但嘴很严,所以也是他们谈话的对象。就我所知,他们最关心的正是赶得上赶不上的问题。在那年头死掉的知识分子,只要不是被杀,准是觉得赶不上好年头了。而活下来的准觉得自己还能赶上——当然,被改造好了、不再是知识分子的人不在此列。因此我对自己的答案颇有信心,敢拿这事和天下人打赌,知识分子最大的不幸,就是这种不理智。 

  下一个问题是:我们所说的不理智,到底是因何而起?对此我有个答案,但不愿为此打睹,主要是怕对方输了赖账:此种不理智,总是起源于价值观或信仰的领域。不很久以前,有位外国小说家还因作品冒犯了某种信仰,被下了决杀令,只好隐姓埋名躲起来。不管此种宗教的信仰者怎么看,我总以为,因为某人写小说就杀了他是不理智的。所幸这道命令已被取消,这位小说家又可以出来角逐布克奖了。对于这世界上的各种信仰,我并无偏见,对有坚定信仰的人我还很佩服,但我不得不指出,狂信会导致偏执和不理智。有一篇歌词,很有点说明意义: 

    跨过大海,尸浮海面,     跨过高山,尸横遍野,     为天皇捐躯,     视死如归。 

  这是一首日本军歌的歌词,从中不难看出,对天皇的狂信导致了最不理智的死亡欲望。一位知识分子对歌中唱到的风景,除了痛心疾首,不应再有其他评价。还有一支出于狂信的歌曲,歌词如下: 

    无产阶级文化大革命,就是好!     就是好来     就是好啊,就是好!…… 

  这四个“就是好”,无疑根绝了讲任何道理的可能性。因为狂信,人就不想讲理。我个人以为,无理可讲比尸横遍野更糟;而且,只要到了无理可讲的地步,肯定也要尸横遍野,“文化革命”里就死人不少,还造成了全民知识水平的大倒退。 

  当然,信仰并不是总要导致狂信,它也不总是导致不理智。全无信仰的人往往不堪信任,在我们现在的社会里,无信仰无价值的人正给社会制造麻烦,谁也不能视而不见。十年前,我在美国,和我的老师讨论这个问题,他说:对一般人来说,有信仰比无信仰要好。起初我不赞成,后来还是被他说服了。 

  十年前我在美国,适逢里根政府要通过一个法案,要求所有的中小学在课间安排一段时间,让所有的孩子在教师的带领下一起祷告。因为想起了“文化革命”里的早请示,我听了就摇头,险些把脑袋摇了下来。我老师说:这件事你可以不同意,但不要这样嗤之以鼻——没你想的那么糟。政府没有强求大家祈祷新教的上帝。佛教孩子可以念阿弥陀佛,伊斯兰教的孩子可以祷告真主,中国孩子也可以想想天地祖宗——各自向自己的神祈祷,这没什么不好。但我还是要摇头。我老师又说:不要光想你自己!十几岁的孩子总不会是知识分子吧。就算他是无神论者,也可以在祷告时间反省一下自己的所作所为。这种道理说服了我,止住了我的摇头疯:不管是信神,还是自珍自重,人活在世界上总得有点信念才成。就我个人而言,虽是无神论者,对于无限广阔的未知世界,多少还有点猜测;我也有个人的操守,从不逾矩,其依据也不是人人都能接受的,所以也是一种信念。从这个意义上说,我理应不反对别人信神、信祖宗,或者信天命——只要信得不过分。在学校里安排段祈祷的时间,让小孩子保持虔诚的心境,这的确不是坏主意——当时我是这样想,现在我又改主意了。 

  时隔十年,再来考虑信仰问题,我忽然发现,任何一种信仰,包括我的信仰在内,如果被滥用,都可以成为打人的棍子、迫害别人的工具。渎神是罪名,反民族反传统、目无祖宗都是罪名。只要你能举出一种可以狂信而无丧失理智危险的信仰,无须再说它有其他的好处,我马上就皈依它——这种好处比其他所有好处加起来,都要大得多啊。 

  现在,有这样一种信仰摆在了我们面前。请相信,对于它的全部说明,我都考虑过了。它有很多好处:它是民族的、传统的、中庸的、自然的、先进的、惟一可行的;论说都很充分。但我不以为它可以保证自己不是打人的棍子,理由很简单,它本身就包括了很多大帽子,其分量足以使人颈骨折断:反民族、反传统、反中庸、反自然……尤其是头两顶帽子,分量简直是一目了然的。就连当初提倡它的余英时先生,看到我们这里附和者日众,也犯起嘀咕来了。最近他在《二十一世纪》杂志上著文,提出了反对煽动民族狂热的问题。在我看来,就是因为看到了第一顶帽子的分量。金庸先生小说里曾言:“武林至尊,宝刀屠龙;号令天下,莫敢不从!”民族狂热就是把屠龙刀啊。余先生不肯铸出宝刀,再倒持太阿,以柄授人——这证明了我对海外华人学者一贯的看法:人家不但学术上有长处,对于切身利害也很惊警,借用打麻将的术语,叫做“门儿清”! 

  至于国内的学者,门儿清就不是他们的长处。有学者说,我们搞的是学术研究,不是搞意识形态——嘿,这由得了你吗?有朝一日它成了意识形态,你的话就是罪状:胆敢把我们民族伟大的精神遗产扣押在书斋里,不让它和广大群众见面!我敢打赌,甚至敢赌十块钱:到了这有朝一日,整他准比整我还厉害。 

  说到信仰,我和我老师有种本质的不同。他老人家是基督徒,又对儒学击节赞赏;他告诉我说,只要身体条件许可,他每年都要去趟以色列——他对犹太教也有兴趣;至于割没割包皮,因为没有和他老人家同浴的机会,我不知道。但我知道,他是一个信仰的爱好者。我相信他对我的看法是:可恨的无神论者,马基雅弗利分子。我并不以此为耻。说到马基雅弗利,一般人都急于和他划清界线,因为他胆敢把道义、信仰全抛开,赤裸裸地谈到利害;但是真正的知识分子对他的评价不低,赤裸裸地谈利害,就接近于理智。但我还是不当马基雅弗利分子——我是墨子的门徒,这样把自己划在本民族的圈子里面,主要是想防个万一。顺便说一句,我老师学问很大,但很天真;我学问很小,但老奸巨猾。对于这一点,他也佩服。用他的原话来说,是这样的:你们大陆来的同学,经历这一条,别人没法比啊。 

  我对墨子的崇拜有两大原因:其一,他思路缜密,有人说他发现了小孔成像——假如是真的,那就是发现了光的直线传播,比朱子只知阴阳二气强了一百多倍 ——只可惜没有完备的实验记录来证明。另外,他用微积分里较老的一种方法来论证无穷(实际是论兼爱是可能的。这种方法叫德尔塔- 伏赛语言),高明无比;在这方面,把孔孟程朱捆在一起都不是他的个儿。其二,他敢赤裸裸地谈利害。我最佩服他这后一点。但我不崇拜他兼爱无等差的思想,以为有滥情之嫌。不管怎么说,墨子很能壮我的胆。有了他,我也敢说自己是中华民族的赤诚分子,不怕国学家说我是全盘西化了。 

  作为墨子门徒,我认为理智是伦理的第一准则,理由是:它是一切知识分子的生命线。出于利害,它只能放到第一。当然,我对理智的定义是:它是对知识分子有益,而绝不是有害的性质。——当然还可以有别的定义,但那些定义里一定要把我的定义包括在内。在古希腊,人最大的罪恶是在战争中砍倒橄榄树。在现代,知识分子最大的罪恶是建造关押自己的思想监狱。砍倒橄榄树是灭绝大地的丰饶,营造意识形态则是灭绝思想的丰饶;我觉得后一种罪过更大——没了橄榄油,顶多不吃色拉;没有思想人就要死了。信仰是重要的,但要从属于理性——如果这是不许可的,起码也该是鼎立之势。要是再不许可,还可以退而求其次——你搞你的意识形态,我不说话总是可以的吧。最糟的是某种偏激之见主宰了理性,聪明人想法子自己来害自己。我们所说的不幸,就从这里开始了。 

  中国的人文知识分子,有种以天下为己任的使命感,总觉得自己该搞出些给老百姓当信仰的东西。这种想法的古怪之处在于,他们不仅是想当牧师、想当神学家,还想当上帝(中国话不叫上帝,叫“圣人”),可惜的是,老百姓该信什么,信到哪种程度,你说了并不算哪,这是令人遗憾的。还有一条不令人遗憾,但却要命:你自己也是老百姓;所以弄得不好,就会自己屙屎自己吃。中国的知识分子在这一节上从来就不明白,所以常常会害到自己。在这方面我有个例子,只是想形象说明一下什么叫自己屙屎自己吃,没有其他寓意:我有位世伯,“文革”前是工读学校的校长,总拿二十四孝为教本,教学生说,百善孝为先,从老莱娱亲、郭解埋儿,一路讲到卧冰求鱼。学生听得毛骨悚然,他还自以为得计。忽一日,来了“文化革命”,学生把他驱到冰上,说道:我们打听清楚了,你爸今儿病了,要吃鱼 ——脱了衣服,趴下吧,给我们表演一下卧冰求鱼——我世伯就此落下病根,健康全毁了。当然,学生都是混蛋,但我世伯也懊悔当初讲得太肉麻。假如不讲那些肉麻故事,挨揍也是免不了,但学生怎么也想不出这么绝的方法来作践他。他倒愿意在头上挨皮带,但岂可得乎……我总是说笑话来安慰他:你没给他们讲“割股疗亲”,就该说是不幸之中的大幸,要不然,学生片了你,岂不更坏?但他听了不觉得可笑。时至今日,一听到二十四孝,他就浑身起鸡皮疙瘩。 

  我对国学的看法是:这种东西实在厉害。最可怕之处就在那个“国”字。顶着这个字,谁还敢有不同意见?这种套子套上脖子,想把它再扯下来是枉然的;否则也不至于套了好几千年。它的诱人之处也在这个“国”字,抢到这个制高点,就可以压制一切不同意见;所以它对一切想在思想领域里巧取豪夺的不良分子都有莫大的诱惑力。你说它是史学也好,哲学也罢,我都不反对——倘若此文对正经史学家哲学家有了得罪之处,我深表歉意——但你不该否认它有成为棍子的潜力。想当年,像姚文元之类的思想流氓拿阶级斗争当棍子,打死打伤了无数人,现在有人又在造一根漂亮棍子。它实在太漂亮了,简直是完美无缺。我怀疑除了落进思想流氓手中变成一种凶器之外,它还能有什么用场。鉴于有这种危险,我建议大家都不要做上帝梦,也别做圣人梦,以免头上鲜血淋漓。 

  对于什么叫美好道德、什么叫善良,我有个最本分的考虑:认真地思索,真诚地明辨是非,有这种态度,大概就可算是善良吧。说具体些,如罗素所说,不计成败利钝地追求客观真理,这该是种美德吧?知识本身该算一种善吧?科学知识分子说这就够了,人文知识分子却来扳杠。他们说,这种朴素的善恶观,造成了多少罪孽!现代的科技文明使人类迷失了方向,科学又造出了毁灭世界的武器。好吧,这些说法也对,可是翻过来看看, 

  人文知识分子又给思想流氓们造了多少凶器、多少混淆是非的烟雾弹!翻过来倒过去,没有一种知识分子是清白无辜的。所以我建议把看不清楚的事撇开,就从知识分子本身的利害来考虑问题——从这种利害出发,考虑我们该有何种道德、何种信念。至于该给老百姓(包括我们自己在内)灌输些什么,最好让领导上去考虑。我觉得领导上办这些事能行,用不着别人帮忙。 

  作为一个知识分子,我对信念的看法是:人活在世上,自会形成信念。对我本人来说,学习自然科学、阅读文学作品、看人文科学的书籍,乃至旅行、恋爱,无不有助于形成我的信念,构造我的价值观。一种学问、一本书,假如不对我的价值观发生作用(姑不论其大小,我要求它是有作用的),就不值得一学,不值得一看。有一个公开的秘密就是:任何一个知识分子,只要他有了成就,就会形成自己的哲学、自己的信念。托尔斯泰是这样,维纳也是这样。到目前为止,我还看不出自己有要死的迹象,所以不想最终皈依什么——这块地方我给自己留着,它将是我一生事业的终结之处,我的精神墓地。不断地学习和追求,这可是人生在世最有趣的事啊,要把这件趣事从生活中去掉,倒不如把我给阉了……你有种美好的信念,我很尊重,但要硬塞给我,我就不那么乐意:打个粗俗的比方,你的把把不能代替我的把把,更不能代替天下人的把把啊。这种看法会遭到反对,你会说:有些人就是笨,老也形不成信念,也管不了自己,就这么浑浑噩噩地活着,简直是种灾难!所以,必须有种普遍适用的信念,我们给它加点压力,灌到他们脑子里!你倒说说看,这再不叫意识形态,什么叫意识形态?假如你像我老师那么门儿清,我也不至于把脑袋摇掉,但还是要说:不是所有的人都那么笨,总要留点余地呀。再说,到底要灌谁?用多大压力?只灌别人,还是连你在内?灌来灌去,可别都灌傻了呀。在科技发达的二十一世纪,你给咱们闹出一窝十几亿傻人,怎么个过法嘛…… 

\footnote{本篇最初发表于1996年第2 期《东方》杂志。}

\chapter{迷信与邪门书}

  我家里有各种各样的书,有工具书、科学书和文学书,还有戴尼提、气功师一类的书,这些书里所含的信息各有来源。我不愿指出书名,但恕我直言,有一类书纯属垃圾。这种书里写着种种古怪异常的事情,作者还一口咬定都是真的,据说这叫人体特异功能。 

  人脑子里有各种各样的东西,有可靠的知识,有不可靠的猜测,还有些东西纯属想入非非。这些东西各有各的用处,我相信这些用处是这样的:一个明理的人,总是把可靠的知识作为根本;也时常想想那些猜测,假如猜测可以验证,就扩大了知识的领域;最后,偶尔他也准许自己想入非非,从荒诞的想象之中,人也能得到一些启迪。当然,人有能力把可信和不可信的东西分开,不会把怪诞的想像当真——但也有例外。 

  当年我在农村插队,见到村里有位妇女撒癔症,自称狐仙附了体,就是这种例外。时至今日,我也不能证明狐仙鬼怪不存在,我只知道他们不大可能存在,所以狐仙附体不能认定是假,只能说是很不可信。假设我信有狐仙附了我的体,那我是信了一件不可信的事,所以叫撒了癔症。当然,还有别的解释,说那位妇女身上有了“超自然的人体现象”,或者是有了特异功能(自从狐仙附体,那位大嫂着实有异于常人,主要表现在她敢于信口雌黄),自己不会解释,归到了狐仙身上;但我觉得此说不对。在学大寨的年代里,农村的生活既艰苦,又乏味;妇女的生活比男人还要艰苦。假如认定自己不是个女人,而是只狐狸,也许会愉快一些。我对撒癔症的妇女很同情,但不意味着自己也想要当狐狸。因为不管怎么说,这是一种病态。 

  我还知道这样一个例子,我的一位同学的父亲得了癌症,已经到了晚期,食水俱不能下,静脉都已扎硬。就在弥留之际,忽然这位老伯指着顶棚说,那里有张祖传的秘方,可以治他的病。假如找到了那张方子,治好了他的病,自然可以说,临终的痛苦激发了老人家的特异功能,使他透过顶棚纸,看到了那张家传秘方。不幸的是,把顶棚拆了下来也没找到。后来老人家终于在痛苦中死去。同学给我讲这件事,我含泪给他解释道:伯父在临终的痛苦之中,开始想入非非,并且信以为真了。 

  我以为,一个人在胸中抹煞可信和不可信的界限,多是因为生活中巨大的压力。走投无路的人就容易迷信,而且是什么都信(马林诺夫斯基也是这样来解释巫术的)。虽然原因让人同情,但放弃理性总是软弱的行径。我还以为,人体特异功能是件不可信的事,要让我信它,还得给我点压力,别叫我“站着说话不腰疼”。比方说,让我得上癌症,这时有人说,他发点外气就能救我,我就会信;再比方说,让我是个犹太人,被关在奥斯维辛,此时有人说,他可以用意念叫希特勒改变主意,放了我们大家,那我不仅会信,而且会把全部财物(假如我有的话)都给他,求他意念一动。我现在正在壮年,处境尚佳,自然想循科学和艺术的正途,努力地思索和工作,以求成就;换一种情况就会有变化。在老年、病痛或贫困之中,我也可能相信世界上还有些奇妙的法门,可以呼风唤雨、起死回生。所以我对事出有因的迷信总抱着宽容的态度。只可惜有种情况叫人无法宽容。 

  在农村还可以看到另一种狐仙附体的人,那就是巫婆神汉。我以为他们不是发癔症,而是装神弄鬼,诈人钱物。如前所述,人在遇到不幸时才迷信,所以他们又是些趁火打劫的恶棍。总的来说,我只知道一个词,可以指称这种人,那就是“人渣”。各种邪门书的作者应该比人渣好些,但凭良心说,我真不知好在哪里。 

  我以为,知识分子的道德准则应以诚信为根本。假如知识分子也骗人,让大家去信谁?但知识分子里也有人信邪门歪道的东西,这就叫人大惑不解。理科的知识分子绝不敢在自己的领域里胡来,所以在诚信方面记录很好。就是文史学者也不敢编造史料,假造文献。但是有科学的技能,未必有科学素质;有科学的素质,未必有科学的品格。科学家也会五迷三道。当然,我相信他们是被人骗了。老年、疾病和贫困也会困扰科学家,除此之外,科学家只知道什么是真,不知道什么是假。更不谙弄虚作假之道,所以容易被人骗。 

  小说家是个很特别的例子,他以编故事为主业;既知道何谓真,更知道何谓假。我自己就是小说家,你让我发誓说写出的都是真事,我绝不敢,但我不以为自己可以信口雌黄到处骗人。我编的故事,读者也知道是编的。我总以为写小说是种事业、是种体面的劳动;有别于行骗。你若说利用他人的弱点进行欺诈,干尽人所不齿的行径,可只因为是个小说家,他就是个好人了,我抵死也不信。这是因为虚构文学一道,从荷马到如今,有很好的名声。 

  我还以为,知识分子应该自尊、敬业。我们是一些堂堂君子,从事着高尚的事业;所有的知识分子都是这样看自己和自己的事业,小说家也不该例外。现在市面上有些书,使我怀疑某人是这么想的:我就是个卑鄙小人,从事着龌龊的事业。假如真有这等事,我只能说:这样想是不好的。 

  最近,有一批自然科学家签名,要求警惕种种伪科学,此举来得非常及时。《老残游记》上说,中国有“北拳南革”两大祸患。当然,“南革”的说法是对革命者的污蔑,但“北拳”的确是中国的一大隐患。中国人——尤其是社会的下层——有迷信的传统,在社会动荡、生活有压力时,简直就是渴望迷信。此时有人来装神弄鬼,就会一哄而起,造成大的灾难。这种流行性的迷信之所以可怕,在于它会使群众变得不可理喻。这是中国文化传统里最深的隐患;宣扬种种不可信的东西,是触发这种隐患。作家应该有社会责任感,不可为一点稿酬,就来为祸人间。

\chapter{生命科学与骗术}

  我的前半生和科学有缘,有时学习科学,有时做科学工作,但从未想到有一天自己会充当科学的辩护士,在各种江湖骗子面前维护它的名声——这使我感到莫大的荣幸。身为一个中国人,由于有独特的历史背景,很难理解科学是什么。 

  我在匹兹堡大学的老师许倬云教授曾说,中国人先把科学当作洪水猛兽,后把它当作呼风唤雨的巫术,直到现在,多数学习科学的人还把它看成宗教来顶礼膜拜;而他自己终于体会到,科学是个不断学习的过程。但是,这种体会过于深奥,对大多数中国人不适用。在大多数中国人看来,科学有移山倒海的威力,是某种叫作“科学家”的人发明出的、我们所不懂的古怪门道。基于这种理解,中国人很容易相信一切古怪门道都是科学;其中就包括了可以呼风唤雨的气功和让药片穿过塑料瓶的特异功能。我当然要说,这些都不是科学。要把这些说明白并不容易——对不懂科学的人说明什么是科学,就像要对三岁孩子说明什么是性一样,难于启齿。 

  物理学家维纳曾说,在理论上人可以通过一根电线来传输;既然如此,你怎么能肯定地说药片不可能穿过药瓶?爱因斯坦说,假如一个车厢以极高的速度运动,其中的时间就会变慢;既然如此,三国时的徐庶为什么就不能还在人间?答案是:维纳、爱因斯坦说话,不该让外行人听见。我还听说有位山里人进城,看见城里的电灯,就买个灯泡回家,把它用皮绳吊起来,然后指着它破口大骂:“妈的,你为什么不亮!”很显然,城里人点电灯,也不该让山里人看到。现在的情况是:人家听也听到了,看也看到了;我们负有解释之责。我的解释是这样的:科学对于公众来说,确实犯下了过于深奥的罪孽。虽然如此,科学仍然是理性的产物。它是世界上最老实、最本分的东西,而气功呼风唤雨、药片穿瓶子,就不那么老实。 

  大贤罗素曾说,近代以来,科学建立了权威。这种权威和以往一切权威都不同,它是一种理性的权威,或者说,它不是一种真正的权威。科学所说的一切,你都不必问它是从谁嘴里说出来的、那人可不可信,因为你可以用纸笔或者试验来验证。虽然不是每个人都有验证数学定理的修养,更不见得拥有实验室,但也不出大格 ——数学修养可以学出来,试验设备也可以置办。数学家证明了什么,总要把自己的证明写给人看;物理学家做出了什么,也要写出实验条件和过程。总而言之,科学家声称自己发明、发现了什么,都要主动接受别人的审查。 

  我们知道,司法上有无罪推定一说,要认定一个人有罪,先假设他是无罪的,用证据来否定这个假设。科学上认定一个人的发现,也是从他没发现开始,用证据来说明他确实发现了。敏感的读者会发现,对于个人来说,这后一种认定,是个有罪推定。举例来说,我王某人在此声明自己最终证明了哥德巴赫猜想(我当然不是认真说的!),就等于把自己置于骗子的地位。直到我拿出了证明,才能脱罪。鉴于此事的严重性,我劝读者不要轻易尝试。 

  假如特异功能如某些作家所言,是什么生命科学大发现的话,在特异功能者拿出足以脱罪的证明之前,把他们称为骗子,显然不是冒犯,因为科学的严肃性就在于此。现在有几位先生努力去证明特异功能有鬼,当然有功于世道,但把游戏玩颠倒了——按照前述科学的规则,我们必须首先推定:特异功能本身就是鬼,那些人就是骗子;直到他们有相反的证据。如果有什么要证明的,也该让他们来证明。 

  现在来说说科学的证明是什么。它是如此的清楚、明白、可信,绝不以权威压人,也绝不装神弄鬼。按罗素的说法,这种证明会使读者感到,假如我不信他所说的就未免太笨。按维纳所说的条件(他说的条件现在做不到),假如我不相信人可以通过电线传输,那我未免太笨;按爱因斯坦所说的条件(他说的条件现在也做不到),假如我不相信时间会变慢,也未免太笨。这些条件太过深奥,远不是特异功能的术者可以理解的。虽然那些人可能看过些科普读物,但连科普都没看懂。在大家都能理解的条件之下,不但药片不能穿过塑料瓶,而且任何刚性的物体都不可能穿过比自身小的洞而且毫发无损,术者说药片穿过了分子间的缝隙,显然是不要脸了。那些术者的证明,假如有谁想要接受,就未免太笨。如果有人持相反的看法,必然和“骗”字有关;或行骗、或受骗。假如我没有勇气讲这些话,也就不配作科学的弟子。因为我们已经被逼到了这个地步,假如不把这个“骗”字说出来,就只好当笨蛋了。 

  关心“特异功能”或是“生命科学”的人都知道,像药片穿瓶子、耳朵识字这类的事,有时灵,有时不灵。假如你认真去看,肯定碰上他不灵,而且也说不出什么时候会灵。假如你责怪他们:为什么不把特异功能搞好些再出来表演,就拿他们太当真了。仿此我编个笑话,讲给真正的科学家听:有一位物理学家致电瑞典科学院说:本人发现了简便易行的方法,可以实现受控核聚变,但现在把方法忘掉了。我保证把方法想起来,但什么时候想起来不能保证。在此之前请把诺贝尔物理奖发给我。当然,真正的物理学家不会发这种电报,就算真的出了忘掉方法的事,也只好吃哑巴亏。我们国家的江湖骗子也没发这种电报,是因为他们的层次太低。他们根本想不到骗诺贝尔奖,只能想到混吃混喝,或者写几本五迷三道的书,骗点稿费。 

  按照许倬云教授的意见,中国人在科学面前,很容易失去平常心。科学本身太过深奥,这是原因之一。民族主义是另一个原因。假设特异功能或是生命科学是外国人发明的,到中国来表演,相信此时它已深深淹没在唾液和粘痰的海洋里。众所周知,现代科学发祥于外国,中国人搞科学,是按洋人发明的规则去比赛规定动作。很多人急于发明新东西,为民族争光。在急迫的心情下,就大胆创新,打破常规,创造奇迹。举例来说,五八年大跃进时就发明了很多东西。其中有一样,上点岁数的都记得:一根铁管,一头拍扁后,做成单簧管的样子,用一片刀片做簧片。他们说,冷水从中通过,就可以变成热水,彻底打破热力学第二定律。这种东西叫作“超声波”,被大量制造,下在澡堂的池子里。据我所见,它除了割破洗澡者的屁股,别无功能;我还见到一个人的脚筋被割断,不知他现在怎样了。“特异功能”、“生命科学”就是九十年代的“超声波”。 

  “超声波”的发明者是谁,现在已经不可考;但我建议大家记下现在这些名字,同时也建议一切人:为了让自己的儿女有脸作人,尽量不要当骗子。很显然,这种发明创造,丝毫也不能为民族争光,只是给大家丢丑,所以让那些假发明的责任者溜掉有点不公道。我还建议大家时时想到:整个人类是一个物种,科学是全人类的事业,它的成就不能为民族所专有,所以它是全人类的光荣;这样就能有一些平常心。有了平常心,也就不容易被人骗。 

  我的老师曾说,科学是个不断学习的过程。学习科学,尤其要有平常心。如罗素所言,科学在“不计利害地追求客观真理”。请扪心自问,你所称的科学,是否如此淳朴和善良。尤瑟纳尔女士说:“当我计算或写作时,就超越了性别,甚至超越了人类。”请扪心自问,你所称的科学,是否是如此崇高的事业。我用大师们的金玉良言劝某些成年人学好。不用别人说,我也觉得此事有点可笑。 

  现在到了结束本文的时候,可以谈谈我对所谓“生命科学”的看法了。照我看,这里包含了一些误会。从表面上科学只认理不认人,仿佛它是个开放的领域,谁都能来弄一把;但在实际上,它又是最困难的事业,不是谁都能懂,所以它又最为封闭。从表面上看,科学不断创造奇迹,好像很是神奇,但在实际上,它绝无分毫的神奇之处——如马林诺夫斯基所言,科学是对真正事实的实事求是——它创造的一切,都是本分得来的;其中包含的血汗、眼泪和艰辛,恐非外人所能知道。但这不是说,你只要说有神奇的事存在,就会冒犯到我。我还有些朋友相信基督死了又活过来,这比药片穿瓶更神奇!这是信仰,理当得到尊重。科学没有理由去侵犯合理的宗教信仰。但我们现在见到的是一种远说不上合理的信仰在公然强奸科学——一个弱智、邪恶、半人半兽的家伙,想要奸污智慧女神,它还流着口水、吐着粘液、口齿不清地说道:“我配得上她!她和我一样的笨!”——我想说的是:你搞错了。换个名字,到别处去试试吧。

\chapter{电脑特技与异化}

  《侏罗纪公园》、《玩具总动员》获得成功以后,电影中的电脑特技就成了个热门话题。咱们这里也有人炒这个题目,写出了大块文章,说电脑特技必然导致电影人的异化云云。我对这问题也有兴趣,但不是对炒有兴趣,而是对特技有兴趣,电脑做出的效果虽然不错,但还不能让我满意。听说做特技要用工作站,这种机器不是我能买得起的,软件也难伺候,总得有一帮专家聚在一起,黑天白日地干,做出的东西才能看。有朝一日技术进步了,用一台PC机就能做电影,软件一个人也能伺候过来,那才好呢。到了那时,我就不写小说,写点有声有色的东西。说句实在话,老写这方块字,我早就写烦了。有关文章的作者一定会惊呼道:连小说的作者(即我)也被异化了。但这种观点不值一驳。你说电脑特技是异化,比之搭台子演戏,电影本身才是异化呢。演戏还要化妆,还不如灰头土脸往台上一站。当然上台也是异化,不如不上台。整个表演艺术都没有,这不是更贴近生活吗。说来说去,人应该弃绝一切科学、技术和艺术的进步,而且应该长一脸毛,拖条尾巴,见了人龇出大牙噢噢地叫唤——你当然知道它是谁,它是狒狒。比之人类,它很少受到异化,所以更像我们的共同祖先——猴子。当然,狒狒在低等猴类面前也该感到惭愧,因为它也被异化了。这样说来说去,所有的动物都该感到惭愧,只有最原始的三叶虫和有关批判文章的作者例外。 

  像这样理解异化的概念,可能有点歪批,但也没有把电脑科技叫做异化更歪。 

  除了异化之外,还有个概念叫做同化。在生物学上指生物从外界取得养分,构造自己的机体。作为艺术家,我认为一切技术手段都是我们同化的目标。假如中国的电影人连电脑特技这样的手段都同化不了,干脆散伙算了。我希望艺术家都长着一颗奔腾的心,锐意进取,你当然也可以说,这姓王的被异化得太厉害,把心脏都成了电脑的CPU。说句老实话吧,我不相信有关文章的作者真的这么仇恨电脑。 

  所有的东西都涨价,就是电脑在降价,它有什么可恨的呢。他们这样说,主要是因为电脑特技是外国人先搞出来,并且先用在电影上的。假如这种技术是中国人的发明,并且在我国的重点影片上首先采用,我就不相信谁还会写这种文章——资本主义国家弄出了新玩意,先弄它一下。不管有理没理,态度起码是好的。有朝一日,上面有了某种精神,咱们的文章早就写了,受表扬不说,还赚了个先知先见之明。像这种事情以前也有过,但不是发生在中国,而是发生在早年的苏联,也不是发生在电影界,而是发生在物理学界。 

  当时爱因斯坦的相对论刚刚问世,有几位聪明人盘算了一下,觉得该弄它一下,就写几篇文章批判了一番。爱因斯坦看了觉得好笑,写了首打油诗作为回敬—— 批判文章我没看到,爱老师的打油诗是读过的,当然,等我读到打油诗时,爱老师和写文章的老师都死掉了。对于后者来说,未尝不是好事,要不别人见到时说他一句:批判相对论,你还是物理学家呢你。难免也会臊死。 

  我总觉得,未来的电影离不了电脑特技,正如今日的物理学离不了相对论;所以上面也不会有某种精神。当然,我也不希望有关作者被臊死。这件事没弄对,但总会有弄对的时候。 


\footnote{本篇最初发表于1997年4 月3 日《戏剧电影报》。}

\chapter{沉默的大多数}

君特·格拉斯在《铁皮鼓》里,写了一个不肯长大的人。小奥斯卡发现周围的世界太过荒诞,就暗下决心要永远做小孩子。在冥冥之中,有一种力量成全了他的决心,所以他就成了个侏儒。这个故事太过神奇,但很有意思。人要永远做小孩子虽办不到,但想要保持沉默是能办到的。在我周围,像我这种性格的人特多──在公众场合什么都不说,到了私下里则妙语连珠,换言之,对信得过的人什么都说,对信不过的人什么都不说。起初我以为这是因为经历了严酷的时期(文革),后来才发现,这是中国人的通病。龙应台女士就大发感慨,问中国人为什么不说话。她在国外住了很多年,几乎变成了个心直口快的外国人。她把保持沉默看做怯懦,但这是不对的。沉默是一种人类学意义上的文化,一种生活方式。它的价值观很简单:开口是银,沉默是金。一种文化之内,往往有一种交流信息的独特方式,甚至是特有的语言,有一些独有的信息,文化可以传播,等等。这才能叫作文化。 

沉默有自己的语言。举个住楼的人都知道的例子:假设有人常把一辆自行车放在你门口的楼道上,挡了你的路,你可以开口去说:打电话给居委会;或者直接找到车主,说道:同志,五讲四美,请你注意。此后他会用什么样的语言来回答你,我就不敢保证。我估计他最起码要说你“事儿”,假如你是女的,他还会说你“事儿妈”,不管你有多大岁数,够不够做他妈。当然,你也可以选择沉默的方式来表达自己对这种行为的厌恶之情:把他车胎里的气放掉。干这件事时,当然要注意别被车主看见。 


还有一种更损的方式,不值得推荐,那就是在车胎上按上个图钉。有人按了图钉再拔下来,这样车主找不到窟窿在哪儿,补带时更困难。假如车子可以搬动,把它挪到难找的地方去,让车主找不着它,也是一种选择。这方面就说这么多,因为我不想编沉默的辞典。 

一种文化必有一些独有的信息,沉默也是有的。戈尔巴乔夫说过这样的话:有一件事是公开的秘密,假如你想给自己盖个小房子,就得给主管官员些贿赂,再到国家的工地上偷点建筑材料。这样的事干得说不得,属于沉默;再加上讲这些话时,戈氏是苏共总书记,所以当然语惊四座。还有一点要补充的,那就是:属于沉默的事用话讲了出来,总是这么怪怪的。 

沉默也可以传播。在某些年代里,所有的人都不说话了,沉默就像野火一样四下漫延着。把这叫作传播,多少有点过甚其辞,但也不离大谱。在沉默的年代里,人们也在传播小道消息,这件事破坏了沉默的完整性。好在这种话语我们只在一些特定的场合说,比方说,公共厕所。最起码在追查谣言时,我们是这样交待的:这话我是在厕所里听说的!这样小道消息就成了包含着排便艰巨的呓语,不值得认真对待。另外,公厕虽然也是公共场合,但我有种强烈的欲望,要把它排除在外,因为它太脏了。 

我属于沉默的大多数。从我懂事的年龄,就常听人们说:我们这一代,生于一个神圣的时代,多么幸福;而且肩负着解放天下三分之二受苦人的神圣使命,等等;在甜蜜之余也有一点怀疑:这么多美事怎么都叫我赶上了。再说,含蓄是我们的家教。 

在三年困难时期,有一天开饭时,每人碗里有一小片腊肉。我弟弟见了以后,按捺不住心中的狂喜,冲上阳台,朝全世界放声高呼:我们家吃大鱼大肉了!结果是被我爸爸拖回来臭揍了一顿。经过这样的教育,我一直比较深沉。所以听到别人说:我们多么幸福、多么神圣时,别人在受苦,我们没有受等等,心里老在想着:假如我们真遇上了这么多美事,不把它说出来会不会更好。当然,这不是说,我不想履行自己的神圣职责。对于天下三分之二的受苦人,我是这么想的:与其大呼小叫说要去解放他们、让人家苦等,倒不如一声不吭。忽然有一天把他们解放,给他们一个意外惊喜。总而言之,我总是从实际的方面去考虑,而且考虑得很周到。智者千虑尚且难免一失,何况当年我只是个小孩子。我就没想到这些奇妙的话语只是说给自己听的,而且不准备当真去解放谁。总而言之,家教和天性谨慎,是我变得沉默的起因。 

与沉默的大多数相反,任何年代都有人在公共场合喋喋不休。我觉得他们是少数人,可能有人会不同意。如福科先生所言,话语即权力。当我的同龄人开始说话时,给我一种极恶劣的印象。有位朋友写了一本书,写的是自己在文革中的遭遇,书名为《血统》。可以想见,她出身不好。她要我给她的书写个序。这件事使我想起来自己在那些年的所见所闻。文革开始时,我十四岁,正上初中一年级。有一天,忽然发生了惊人的变化,班上的一部份同学忽然变成了红五类,另一部份则成了黑五类。我自己的情况特殊,还说不清是哪一类。当然,这红和黑的说法并不是我们发明出来,这个变化也不是由我们发起的。照我看来,红的同学忽然得到了很大的好处,这是值得祝贺的。黑的同学忽然遇上了很大的不幸,也值得同情。我不等对他们一一表示祝贺和同情,一些红的同学就把脑袋刮光,束上了大皮带,站在校门口,问每一个想进来的人:你什么出身?他们对同班同学问得格外仔细,一听到他们报出不好的出身,就从牙缝里迸出三个字:“狗崽子!”当然,我能理解他们突然变成了红五类的狂喜,但为此非要使自己的同学在大庭广众下变成狗崽子,未免也太过份。这使我以为,使用话语权是人前显贵,而且总都是为了好的目的。现在看来,我当年以为的未必对,但也未必全错。 

话语有一个神圣的使命,就是想要证明说话者本身与众不同,是芸芸众生中的娇娇者。现在常听说的一种说法是:中国人拥有世界上最杰出的文化,在全世界一切人中最聪明。对此我不想唱任何一种反调,我也不想当人民公敌。我还持十几岁时的态度:假设这些都是实情,我们不妨把这些保藏在内心处不说,“闷兹蜜”。这些话讲出来是不好的,正如在文革时,你可以因自己是红五类而沾沾自喜,但不要到人前去显贵,更不要说别人是狗崽子。根除了此类话语,我们这里的话就会少很多,但也未尝不是好事。 

现在我要说的是另一个题目:我上小学六年级时,暑期布置的读书作业是《南方来信》。那是一本记述越南人民抗美救国斗争的读物,其中充满了处决、拷打和虐杀。 

看完以后,心里充满了怪怪的想法。那时正在青春期的前沿,差一点要变成个性变态了。总而言之,假如对我的那种教育完全成功,换言之,假如那些园丁、人类灵魂的工程师对我的期望得以实现,我就想像不出现在我怎能不嗜杀成性、怎能不残忍,或者说,在我身上,怎么还会保留了一些人性。好在人不光是在书本上学习,还会在沉默中学习。这是我人性尚存的主因。 

现在我就在发掘沉默,但不是作为一个社会科学工作者来发掘。这篇东西大体属于文学的范畴,所谓文学就是:先把文章写到好看,别的就管他妈的。现在我来说明自己为什么人性尚存。文化革命刚开始时,我住在一所大学里。有一天,我从校外回来,遇上一大夥人,正在向校门口行进。走在前面的是一夥大学生,彼此争论不休,而且嗓门很大;当然是在用时髦话语争吵,除了毛主席的教导,还经常提到“十六条”。所谓十六条,是中央颁布的展开文化革命的十六条规定,其中有一条叫作“要文斗、不要武斗”,制定出来就是供大家违反之用。在那些争论的人之中,有一个人居于中心地位。但他双唇紧闭,一声不吭,唇边似有血迹。在场的大学生有一半在追问他,要他开口说话,另一半则在维护他,不让他说话。文化革命里到处都有两派之争,这是个具体的例子。至于队伍的后半部分,是一帮像我这么大的男孩子,一个个也是双唇紧闭,一声不吭,但唇边没有血迹,阴魂不散地跟在后面。有几个大学生想把他们拦住,但是不成功,你把正面拦住,他们就从侧面绕过去,但保持着一声不吭的态度。这件事相当古怪,因为我们院里的孩子相当的厉害,不但敢吵敢骂,而且动起手来,大学生还未必是个儿,那天真是令人意外的老实。我立刻投身其中,问他们出了什么事,怪的是这些孩子都不理我,继续双唇紧闭,两眼发直,显出一种坚忍的态度,继续向前行进──这情形好像他们发了一种集体性的癔症。 

有关癔症,我们知道,有一种一声不吭,只顾扬尘舞蹈;另一种喋喋不休,就不大扬尘舞蹈。不管哪一种,心里想的和表现出来的完全不是一回事。我在北方插队时,村里有几个妇女有癔症,其中有一位,假如你信她的说法,她其实是个死去多年的狐狸,成天和丈夫(假定此说成立,这位丈夫就是个兽奸犯)吵吵闹闹,以狐狸的名义要求吃肉。但肉割来以后,她要求把肉煮熟,并以大蒜佐餐。很显然,这不合乎狐狸的饮食习惯。所以,实际上是她,而不是它要吃肉。至于文化革命,有几分像场集体性的癔症,大家闹的和心里想的也不是一回事。但是我说的那些大学里的男孩子其实没有犯癔症。后来,我揪住了一个和我很熟的孩子,问出了这件事的始末:原来,在大学生宿舍的盥洗室里,有两个学生在洗脸时相遇,为各自不同的观点争辩起来。争着争着,就打了起来。其中一位受了伤,已被送到医院。另一位没受伤,理所当然地成了打人凶手,就是走在队伍前列的那一位。这一大夥人在理论上是前往某个机构(叫作校革委还是筹委会,我已经不记得了)讲理,实际上是在校园里做无目标的布朗运动。 

这个故事还有另一个线索:被打伤的学生血肉模糊,有一只耳朵(是左耳还是右耳已经记不得,但我肯定是两者之一)的一部份不见了,在现场也没有找到。根据一种安加莎·克里斯蒂式的推理,这块耳朵不会在别的地方,只能在打人的学生嘴里,假如他还没把它吃下去的话;因为此君不但脾气暴燥,急了的时候还会咬人,而且咬了不止一次了。我急于交待这件事的要点,忽略了一些细节,比方说,受伤的学生曾经惨叫了一声,别人就闻声而来,使打人者没有机会把耳朵吐出来藏起来,等等。总之,此君现在只有两个选择,或是在大庭广众之中把耳朵吐出来,证明自己的品行恶劣,或者把它吞下去。我听到这些话,马上就加入了尾随的行列,双唇紧闭,牙关紧咬,并且感觉到自己嘴里仿佛含了一块咸咸的东西。 

现在我必须承认,我没有看到那件事的结局;因为天晚了,回家太晚会有麻烦但我的确关心着这件事的进展,几乎失眠。这件事的结局是别人告诉我的:最后,那个咬人的学生把耳朵吐了出来,并且被人逮住了。不知你会怎么看,反正当时我觉得如释重负:不管怎么说,人性尚且存在。同类不会相食,也不会把别人的一部份吞下去。当然,这件事可能会说明一些别的东西:比方说,咬掉的耳朵块太大,咬人的学生嗓子眼太细,但这些可能性我都不愿意考虑。我说到这件事,是想说明我自己曾在沉默中学到了一点东西,而这些东西是好的。这是我选择沉默的主要原因之一:从话语中,你很少能学到人性,从沉默中却能。假如还想学得更多,那就要继续一声不吭。 

有一件事大多数人都知道:我们可以在沉默和话语两种文化中选择。我个人经历过很多选择的机会,比方说,插队的时候,有些插友就选择了说点什么,到“积代会”上去“讲用”,然后就会有些好处。有些话年轻的朋友不熟悉,我只能简单地解释道:积代会是“活学活用毛主席著作积极分子代表大会”,讲用是指讲自己活学活用毛主席著作的心得体会。参加了积代会,就是积极分子。而积极分子是个好意思。 

另一种机会是当学生时,假如在会上积极发言,再积极参加社会活动,就可能当学生干部,学生干部又是个好意思。这些机会我都自愿地放弃了。选择了说话的朋友可能不相信我是自愿放弃的,他们会认为,我不会说话或者不够档次,不配说话。因为话语即权力,权力又是个好意思,所以的确有不少人挖空心思要打进话语的圈子,甚至在争夺“话语权”。我说我是自愿放弃的,有人会不信──好在还有不少人会相信。 

主要的原因是进了那个圈子就要说那种话,甚至要以那种话来思索,我觉得不够有意思。据我所知,那个圈子里常常犯着贫乏症。 

二十多年前,我在云南当知青。除了穿着比较乾净、皮肤比较白晰之外,当地人怎么看待我们,是个很费猜的问题。我觉得,他们以为我们都是台面上的人,必须用台面上的语言和我们交谈──最起码在我们刚去时,他们是这样想的。这当然是一个误会,但并不讨厌。还有个讨厌的误会是:他们以为我们很有钱,在集市上死命地朝我们要高价,以致我们买点东西,总要比当地人多花一两倍的钱。后来我们就用一种独特的方法买东西:不还价,甩下一叠毛票让你慢慢数,同时把货物抱走。等你数清了毛票,连人带货都找不到了。起初我们给的是公道价,后来有人就越给越少,甚至在毛票里杂有些分票。假如我说自己洁身自好,没干过这种事,你一定不相信;所以我决定不争辩。终于有一天,有个学生在这样买东西时被老乡扯住了;但这个人决不是我。那位老乡决定要说该同学一顿,期期艾艾地憋了好半天,才说出:哇!不行啦!思想啦!斗私批修啦!后来我们回家去,为该老乡的话语笑得打滚。可想而知,在今天,那老乡就会说:哇!不行啦!五讲啦!四美啦!三热爱啦!同样也会使我们笑得要死。从当时的情形和该老乡的情绪来看,他想说的只是一句很简单的话,那一句话的头一个字发音和洗澡的澡有些相似。我举这个例子,绝不是讨了便宜又要卖乖,只是想说明一下话语的贫乏。用它来说话都相当困难,更不要说用它来思想了。话语圈子里的朋友会说,我举了一个很恶劣的例子──我记住这种事,只是为了丑化生活;但我自己觉得不是的。还有一些人会说,我们这些熟练掌握了话语的人在嘲笑贫下中农,这是个卑劣的行为。说实在的,那些话我虽耳熟,但让我把它当众讲出口来,那情形不见得比该老乡好很多。我希望自己朴实无华,说起话来,不要这样绕嘴,这样古怪,这样让人害怕。这也是我保持沉默的原因之一。 

中国人有句古话:敬惜字纸。这话有古今两种通俗变体:古代人们说,用印了字的纸擦屁股要瞎眼睛;现代有种近似科学的说法:用有油墨的纸擦屁股会生痔疮。其实,真正要敬惜的根本就不是纸,而是字。文字神圣。我没听到外国有类似的说法,他们那里神圣的东西都与上帝有关。人间的事物要想神圣,必须经过上帝或者上帝在人间代理机构的认可。听说,天主教的主教就需要教皇来祝圣。相比之下,中国人就不需要这个手续。只要读点书,识点字,就可以写文章。写来写去,自祝自圣。这件事有好处,也有不好处。好处是达到神圣的手续甚为简便,坏处是写什么都要带点“圣”气,就丧失了平常心。我现在在写字,写什么才能不亵渎我神圣的笔,真是个艰巨的问题。古代和近代有两种方法可以壮我的胆。古代的方法是,文章要从夫子曰开始。近代的方法是从“毛主席教导我们说”开始。这两种方法我都不拟采用。其结果必然是:这篇文字和我以往任何一篇文字一样,没有丝毫的神圣性。 

我们所知道、并且可以交流的信息有三级:一种心知肚明,但既不可说也不可写,另一种可说不可写,我写小说,有时就写出些汉语拼音来。最后一种是可以写出来的。 

当然,说得出的必做得出,写得出的既做得出也说得出;此理甚明。人们对最后这类信息交流方式抱有崇敬之情。在这方面我有一个例子:我在云南插队时,有一阵是记工员。队里的人感觉不舒服不想上工,就给我写张假条。有一天,队里有个小伙子感觉屁股疼,不想上工。他可以用第一种方式通知我,到我屋里来,指指屁股,再苦苦脸,我就会明白。用第二种方法也甚简便。不幸他用了第三种方式。我收到那张条子,看到上面写着“龟头疼”,就照记下来。后来这件事就传扬开来,队里的人还说,他得了杨梅大疮,否则不会疼在那个部位上。因此他找到我,还威胁说要杀掉我。经过核实原始凭据,发现他想按书面语言,写成臀部疼,不幸写成了“电布疼”,除此之外,还写得十分歪歪斜斜。以致我除了认做龟头疼,别无他法。其实呢,假如他写屁股疼,我想他是能写出的;此人既不是龟头疼,也不是屁股疼,而是得了痔疮;不过这一点已经无关紧要了。要紧的是人们对于书面话语的崇敬之情。假如这种话语不仅是写了出来,而且还印了出来,那它简直就是神圣的了。但不管怎么说罢,我希望人们在说话和写文章时,要有点平常心。屁股疼就说屁股疼,不要写电布疼。至于我自己,丝毫也不相信有任何一种话语是神圣的。缺少了这种虔诚,也就不配来说话。 

我所说的一切全都过去了。似乎没有必要保持沉默了。如前所述,我曾经是个沉默的人,这就是说,我不喜欢在各种会议上发言,也不喜欢写稿子。这一点最近已经发生了改变,参加会议时也会发言,有时也写点稿。对这种改变我有种强烈的感受,有如丧失了童贞。这就意味着我违背了多年以来的积习,不再属于沉默的大多数了。 

我还不致为此感到痛苦,但也有一点轻微的失落感,我们的话语圈从五十年代起,就没说过正常的话:既鼓吹过亩产三十万吨钢,也炸过精神原子弹。说得不好听,它是座声名狼籍的疯人院。如今我投身其中,只能有两种可能:一是它正常了,二是我疯掉了,两者必居其一。我当然想要弄个明白,但我无法验证自己疯没疯。在这方面有个例子:当年里根先生以七十以上的高龄竞选总统,有人问他:假如你当总统以后老糊涂了怎么办?里根先生答道:没有问题。假如我老糊涂了,一定交权给副总统。然后人家又问:你老糊涂了以后,怎能知道自己老糊涂了?他就无言以对。这个例子对我也适用:假如我疯掉了,一定以为自己没有疯。我觉得话语圈子比我容易验证一些。 

假如你相信我的说法,沉默的大多数比较谦虚、比较朴直、不那么假正经,而且有较健全的人性。如果反过来,说那少数说话的人有很多毛病,那也是不对的。不过他们的确有缺少平常心的毛病。 

几年前,我参加了一些社会学研究,因此接触了一些“弱势群体”,其中最特别的就是同性恋者。做过了这些研究之后,我忽然猛省到:所谓弱势群体,就是有些话没有说出来的人。就是因为这些话没有说出来,所以很多人以为他们不存在或者很遥远。在中国,人们以为同性恋者不存在。在外国,人们知道同性恋者存在,但不知他们是谁。有两位人类学家给同性恋者写了一本书,题目就叫做。然后我又猛省到自己也属于古往今来最大的一个弱势群体,就是沉默的大多数。这些人保持沉默的原因多种多样,有些人没能力、或者没有机会说话;还有人有些隐情不便说话;还有一些人,因为种种原因,对于话语的世界有某种厌恶之情。我就属于这最后一种。 

对我来说,这是青少年时代养成的习惯,是一种难改的积习。小时候我贫嘴聊舌,到了一定的岁数之后就开始沉默寡言。当然,这不意味着我不会说话──在私下里我说的话比任何人都不少──这只意味着我放弃了权力。不说话的人不仅没有权力,而且会被人看做不存在,因为人们不会知道你。 

我曾经是个沉默的人,这就是说,我不喜欢在各种会议上发言,也不喜欢写稿子。 

这一点最近已经发生了改变,参加会议时也会发言,有时也写点稿。对这种改变我有种强烈的感受,有如丧失了童贞。这就意味着我违背了多年以来的积习,不再属于沉默的大多数了。我还不至为此感到痛苦,但也有一点轻微的失落感。现在我负有双重任务,要向保持沉默的人说明,现在我为什么要进入话语的圈子;又要向在话语圈子里的人说明,我当初为什么要保持沉默,而且很可能在两面都不落好。照我看来,头一个问题比较容易回答。我发现在沉默的人中间,有些话永远说不出来。照我看,这件事是很不对的。因此我就很想要说些话。当然,话语的圈子里自然有它的逻辑,和我这种逻辑有些距离。虽然大家心知肚明,但我还要说一句,话语圈子里的人有作家、社会科学工作者,还有些别的人。出于对最后一些人的尊重,就不说他们是谁了──其实他们是这个圈子的主宰。我曾经是个社会科学工作者,那时我想,社会科学的任务之一,就是发掘沉默。就我所知,持我这种立场的人不会有好下场。不过,我还是想做这件事。 

第二个问题是:我当初为什么要保持沉默。这个问题难回答,是因为它涉及到一系列复杂的感觉。一个人决定了不说话,他的理由在话语圈子里就是说不清的。但是,我当初面对的话语圈和现在的话语圈已经不是一个了──虽然它们有一脉相承之处。 

在今天的话语圈里,也许我能说明当初保持沉默的理由。而在今后的话语圈里,人们又能说明今天保持沉默的理由。沉默的说明总是要滞后于沉默。倘若你问,我是不是依然部份地保持了沉默,就是明知故问──不管怎么说,我还是决定了要说说昨天的事。但是要慢慢地说。 

七八年前,我在海外留学,遇上一位老一辈的华人教授。聊天的时候他问:你们把太太叫作“爱人”──那么,把lover叫做什么?我呆了一下说道:叫作“第三者”罢。他朝我哈哈大笑了一阵,使我感觉受到了暗算,很不是滋味。回去狠狠想了一下,想出了一大堆:情人、傍肩儿、拉边套的、乱搞男女关系的家伙、破鞋或者野汉子,越想越歪。人家问的是我们所爱的人应该称作什么,我竟答不上来。倘若说大陆上全体中国人就只爱老婆或老公,别人一概不爱,那又透着虚伪。最后我只能承认:这个称呼在话语里是没有的,我们只是心知肚明,除了老婆和老公,我们还爱过别人。以我自己为例,我老婆还没有和我结婚时,我就开始爱她。此时她只是我的女朋友。根据话语的逻辑,我该从领到了结婚证那一刻开始爱她,既不能迟,也不能早。不过我很怀疑谁控制自己感情的能力有这么老到。由此可以得到两个推论:其一,完全按照话语的逻辑来生存,实在是困难得很。其二:创造话语的人是一批假正经。沿着第一个推理前进,会遇上一堆老话。越是困难,越是要上;存天理灭人欲嘛──那些陈糠烂谷子太多了,不提也罢。让我们沿着第二条道路前进:“爱人”这个字眼让我们想到什么?做爱。这是个外来语,从makelove硬译而来。本土的词儿最常用有两个,一个太粗,根本不能写。另外一个叫作“敦伦”。这个词儿实在有意思。假如有人说,他总是以敦厚人伦的虔敬心情来干这件事,我倒想要认识他,因为他将是我所认识的最不要脸的假正经。为了捍卫这种神圣性,做爱才被叫作“敦伦”。 

现在可以说说我当初保持沉默的原因。时至今日,哪怕你借我个胆子,我也不敢说自己厌恶神圣。我只敢说我厌恶自己说自己神圣,而且这也是实情。 

在一个科幻故事里,有个科学家造了一个机器人,各方面都和人一样,甚至和人一样的聪明,但还不像人。因为缺少自豪感,或者说是缺少自命不凡的天性。这位科学家就给该机器人装上了一条男根。我很怀疑科学家的想法是正确的。照我看来,他只消给机器人装上一个程序,让他到处去对别人说:我们机器人是世界上最优越的物种,就和人是一样的了。 

但是要把这种经历作为教学方法来推广是不合适的。特别是不能用咬耳朵的方法来教给大家人性的道理,因为要是咬人耳的话,被咬的人很疼,咬猪耳的话,效果又太差。所以,需要有文学和社会科学。我也要挤入那个话语圈,虽然这个时而激昂、时而消沉,时而狂吠不止、时而一声不吭的圈子,在过去几十年里从来就没教给人一点好的东西,但我还要挤进去。

\chapter{不新的《万历十五年》}

黄仁宇先生的《万历十五年》很早就在中国出版了,因为选了家好的出版社(三联),所以能够不断重印。我手里这一本是95年底第4次印刷的,以后还有可能再印。这是本老书,但以新书的面目面市。这两年市面上好书不多,还出了些“说不”的破烂。相比之下我宁愿说说不新的《万历十五年》:旧的好书总比新的烂书好。 

黄先生以明朝的万历十五年为横断面,剖开了中国的传统社会:这个社会虽然表面上尊卑有序,实际上是乱糟糟的。书里有这么个例子:有一天北京城里哄传说皇上要午朝了,所有的官员(这可是一大群人)赶紧都赶到城市的中心,挤在一起像个骡马大集,把皇宫的正门堵了个严严实实,但这件事皇上自己都不知道,把他气得要撒癔症。假如哪天早上你推门出去,看到外面楼道上挤满了人,都说是你找来的,但你自己不知道有这么回事,你也要冒火,何况是皇上。他老人家一怒之下罚了大家的俸银──这也没有什么,反正大家都有外快。再比方说,中国当时军队很多,机构重叠,当官的很威武,当兵的也不少,手里也都有家伙,但都是些废物。极少数的倭寇登了陆,就能席卷半个中国。黄先生从政治、经济、军事、文化各个方面来考察,到处都是乱糟糟;偏偏明朝理学盛行,很会摆排场,高调也唱得很好。用儒学的标准来看,万历年间不能说是初级阶段,得说是高级阶段,但国家的事办得却是最不好,要不然也不会被区区几个八旗兵亡掉。由此得出一个结论说,仅靠儒家的思想管理一个国家是不够的,还得有点别的;中国必须从一个靠尊卑有序来管理的国家,过渡到靠数目字来管理的国家。 

我不是要和黄先生扳杠,若说中国用数字来管理就会有前途,这个想法未免太过天真──数数谁不会呢。大跃进时亩产三十万斤粮,这不是数目字吗?用这种数字来管理,比没有数字更糟,这是因为数字可以是假的,尤其是阿拉伯数字,在后面添起0来太方便,让人看了打怵。万历年间的人不识数吗?既知用原则去管理社会不行,为什么不用数字来管? 

黄先生又说,中国儒家的原则本意是善良的,很可以作道德的根基,但在治理国家时,宗旨的善良不能弥补制度的粗疏。这话我相信后半句,不信前半句。我有个例子可以证明它行不通。这例子的主要人物是我的岳母,一个极慈爱的老太太。次要人物是我:我是我丈母娘的女婿,用老话来说,我是她老人家的“半子”──当然不是下围棋时说的半个子,是指半个儿子──她对我有权威,我对她有感情,这是不言而喻的。我家的卫生间没有挂镜子,因为是水泥墙,钉不进钉子。有一天老太太到我们家来,拿来了一面镜子和一根钉子,说道:拿锤子来,你把钉子钉进墙里,把镜子挂上。我一看这钉子,又粗又钝。除非用射钉枪来发射,决钉不进墙里──实际上这就是这钉子的正确用途。细心考虑了一下,我对岳母解释道:妈,你看这水泥,又硬又脆,差不多和玻璃一样。我呢,您是知道的,不是一支射钉枪,肯定不能把它一下打进墙里,要打很多下,水泥还能不碎吗?结果肯定是把墙凿个坑,钉子也钉不上──我说得够清楚的了吧?老太太听了瞪我一眼道:我给你买了钉子,又这么大老远给你送来,你连试都不试?我当然无话可说。过了一会儿,地上落满了水泥碎块,墙上出现了很多浅坑。老太太满意了,说道:不钉了,去吃饭。结果是我家浴室的墙就此变了麻子,成了感情和权威的牺牲品。过些时候,遇到我的大舅子,才知道他家卫生间也是水泥墙,上面也有很多坑,也是用钝钉子钉出来的;他不愿毁坏自己的墙,但更不愿伤害老太太的感情。按儒家的标准,我岳母对待我们符合仁的要求,我们对待我岳母也符合仁的标准,结果在墙上打了些窟窿。假设她连我的PC机也管起来,这东西肯定是在破烂市上也卖不出去,我连吃饭的家伙都没有了。善良要建立在真实的基础上,所以让我去选择道德的根基,我愿选实事求是。 

我说《万历十五年》是本好书,但又这样鸡蛋里挑骨头式的找它的毛病。这是因为此书不会因我的歪批而贬值,它的好处是显而易见的。它是一面镜子,照见了我们的前辈──古时候的读书人,或者叫作儒生们──是怎样作人做事的。古往今来的读书人,从经典里学到了一些粗浅的原则,觉得自己懂了春秋大义,站出来管理国家,妄断天下的是非屈直,结果把一切都管得一团糟。大明帝国是他们交的学费,大清帝国又是他们交的学费。老百姓说:罐子里养王八,养也养不大。儒学的罐子里长不出现代国家来。万历十五年是今日之鉴,尤其是人文知识分子之鉴,我希望他们读过此书之后,收拾起胸中的狂妄之气,在书斋里发现粗浅原则的热情会有所降低,把这些原则套在国家头上的热情也会降低。少了一些造罐子的,大家的日子就会好过了。

\chapter{个人尊严}

在国外时看到,人们对时事做出价值评判时,总是从两个独立的方面来进行:一个方面是国家或者社会的尊严,这像是时事的经线;另一个方面是个人的尊严,这像是时事的纬线。回到国内,一条纬线就像是没有,连尊严这个字眼也感到陌生了。 

提到尊严这个概念,我首先想到的英文词“dignity”,然后才想到相应的中文词。在英文中,这个词不仅有尊严之义,还有体面、身份的意思。尊严不但指人受到尊重,它还是人价值之所在。从上古到现代,数以亿万计的中国人里,没有几个人有过属于个人的尊严。举个大点的例子,中国历史上有过皇上对大臣施廷杖的事,无论是多大的官,一言不和,就可能受到如此当众羞辱,高官尚且如此,遑论百姓。除了皇上一人,没有一个人能有尊严。有一件最怪的事是,按照传统道德,挨皇帝的板子倒是一种光荣,文死谏嘛。说白了就是:无尊严就是有尊严。此话如有任何古怪之处,罪不在我。到了现代以后,人与人的关系、个人与集体的关系,仍有这种遗风──我们就不必细说文革中、文革前都发生过什么样的事情。到了现在,已经不用见官下跪,也不会在屁股上挨板子,但还是缺少个人的尊严。环境就是这样,公共场所的秩序就是这样,人对人的态度就是这样,不容你有任何自尊。 

举个小点的例子,每到春运高潮,大家就会在传媒上看到一辆硬座车厢里挤了三四百人,厕所里也挤了十几人。谈到这件事,大家会说国家的铁路需要建设,说到铁路工人的工作难做,提到安全问题,提到所有的方面,就是不提这些民工这样挤在一起,好像一个团,完全没有了个人的尊严──仿佛这件事很不重要似的。当然,只要民工都在过年时回家,火车总是要挤的;谁也想不出好办法。但个人的尊严毕竟大受损害;这件事总该有人提一提才对。另一件事现在已是老生常谈,人走在街上感到内急,就不得不上公共厕所。一进去就觉得自己的尊严一点都没了。现在北京的公厕正在改观,这是因为外国人到了中国也会内急,所以北京的公厕已经臭名远扬。假如外国人不来,厕所就要臭下去;而且大街上改了,小胡同里还没有改。我认识的一位美国留学生说,有一次他在小胡同里内急,走进公厕撒了一泡尿,出来以后,猛然想到自己刚才满眼都对黄白之物,居然能站住了不倒,觉得自己很了不起,就急忙来告诉我。北京的某些街道很脏很乱,总要到某个国际会议时才能改观,这叫借某某会的东风。不光老百姓这样讲,领导上也这样讲。这话听起来很有点不对味。不雅的景象外人看了丢脸,没有外人时,自己住在里面也不体面──这后一点总是被人忘掉。 

作为一个知识分子,我发现自己曾有一种特别的虚伪之处,虽然一句话说不清,但可以举些例子来说明。假如我看到火车上特别挤,就感慨一声道:这种事居然可以发生在中华人民共和国的土地上!假如我看到厕所特脏,又长叹一声:唉!北京市这是怎么搞的嘛!这其中有点幽默的成份,也有点当真。我的确觉得国家和政府的尊严受到了损失,并为此焦虑着。当然,我自己也想要点个人尊严,但以个人名义提出就过于直露,不够体面──言必称天下,不以个人面目出现,是知识分子的尊严所在。 

当然,现在我把这做为虚伪提出,已经自外于知识分子。但也有种好处,我找到了自己的个人面目。有关尊严问题,不必引经据典,我个人就是这么看。但中国忽视个人尊严,却不是我的新发现。从大智者到通俗作家,有不少人注意到一个有中国特色的现象:罗素说,中国文化里只重家族内的私德,不重社会的公德公益,这一点造成了很要命的景象;费孝通说,中国社会里有所谓“差序格局”,与己关系近的就关心,关系远的就不关心或少关心;结果有些事从来就没人关心。龙应台为这类事而愤怒过,三毛也大发过一通感慨。读者可能注意到了,所有指出这个现象的人,或则是外国人,或则曾在国外生活过,又回到了国内。没有这层关系的中国人,对此浑然不觉。笔者自己曾在外国居住四年,假如没有这种经历,恐怕也发不出这种议论──但这一点并不让我感到开心。环境脏乱的问题,火车拥挤的问题,社会秩序的问题,人们倒是看到了。但总从总体方面提出问题,讲国家的尊严、民族的尊严。其实这些事就发生在我们身边,削我们每个人的面子──对此能够浑然无觉,倒是咄咄怪事。 

人有无尊严,有一个简单的判据,是看他被当作一个人还是一个东西来对待。这件事有点两重性,其一是别人把你当做人还是东西,是你尊严之所在。其二是你把自己看成人还是东西,也是你的尊严所在。挤火车和上公共厕所时,人只被当身体来看待。这里既有其一的成份,也有其二的成份;而且归根结蒂,和我们的文化传统有关。 说来也奇怪,中华礼仪之邦,一切尊严,都从整体和人与人的关系上定义,就是没有个人的位置。一个人不在单位里、不在家里,不代表国家、民族,单独存在时,居然不算一个人,就算是一块肉。这种算法当然是有问题。我的算法是:一个人独处荒岛而且谁也不代表,就像鲁滨孙那样,也有尊严,可以很好的活着。这就是说,个人是尊严的基本单位。知道了这一点,火车上太挤了之后,我就不会再挤进去而且浑然无觉。

\chapter{艺术与关怀弱势群体}

前不久在《中华读书报》上看到一篇文章,作者在北大听戴锦华教授的课,听到戴教授盛赞林白的《一个人的战争》,就发问道:假如你有女儿,想不想让她看这本书?戴教授答曰:否。 

于是作者以为自己抓到了理,得意洋洋地写了那篇文章。 

读那篇文章时,我就觉得这是一片歪理,因为同样的话也可以去问谢晋导演。谢导的儿子是低智人,笔者的意思不是对谢导不敬,而是说:假如谢导持有上述文章作者的想法,拍电影总以儿子能看为准,中国的电影观众就要吃点苦头。大江健三郎也有个低智儿子,若他写文章以自己的儿子能看为准绳,那就是对读者的不敬。 

但我当时没有作文反驳,因为有点吃不准,不知戴教授有多大。倘若她是七十岁的老人,儿女就当是我的年龄,有一本书我都不宜看,那恐怕没有什么人宜看。 

昨天在一酒会上见到戴教授,发现她和我岁数相仿,有儿女也是小孩子,所以我对自己更有把握了。因为该文作者的文艺观乃是以小孩子为准绳,可以反驳他(或者她)的谬见。很不幸的是,我把原文作者的名字忘了,在此申明,不是记得有意不提。 

任何社会里都有弱势群体,比方说,小孩子、低智人──顺便说一句,孩子本非弱势,但在父母心中就弱势得很。以笔者为例,是一绝顶聪明的雄壮大汉,我妈称呼我时却总要冠个傻字──社会对弱势人群当有同情之心。 

文明国家各种福利事业,都是为此而设。但我总觉得,科学、艺术不属福利事业,不应以关怀弱势群体为主旨。这样关怀下去没个底。就以弱智人为例,我小时候邻居有位弱智人,喜欢以屎在墙上涂抹,然后津津有味地欣 

赏这些图案。如果艺术的主旨是关怀弱势群体,恐怕大家都得去看屎画的图案。倘若科学的主旨是关怀弱势群体,恐怕大家都得变成蜣螂一类──我对这种前景深为忧虑。 

最近应朋友之邀,作起了影视评论,看了一些国产影视剧,发现这种前景就在眼前,再看到上述文章,就更感忧虑。以不才之愚见,我国的文学工作者过于关怀弱势群体,与此同时,自己正在变成一个奇特的弱势群体──起码是比观众、读者为弱。戴锦华教授很例外地不在其中,难怪有人看她不顺眼。 

笔者在北大教过书,知道该校有个传统:教室的门是敞开的,谁都可以听。这是最美好的传统,体现了对弱势群体的关怀。但不该是谁都可以提问。罗素先生曾言,人人理应平等,但实际上做不到,其中最特殊的就是知识的领域……要在北大提问,修养总该大体上能过得去才好。 

说完了忧虑,可以转入正题。我以为科学和艺术的正途不仅不是去关怀弱势群体,而且应当去冒犯强势群体。使最强的人都感到受了冒犯,那才叫作成就。以爱因斯坦为例,发表相对论就是冒犯所有在世的物理学家;他做得很对。艺术家也当如此,我们才有望看到好文章。以笔者为例,杜拉斯的《情人》、卡尔维诺的《我们的祖先》,还有许多书都使我深感被冒犯,总觉得这样的好东西该是我写出来的才对。 我一直憋着用同样的冒犯去回敬这些人──只可惜卡尔维诺死了。如你所见,笔者犯着眼高手低的毛病。不过我也有点好处:起码我能容下林白的《一个人的战争》。

\chapter{王朔的作品}

与王朔有关的影视作品我看了一些,有的喜欢,有的不喜欢。有些作品里带点乌迪·艾伦的风格,这是我喜欢的。有些作品里也冒出些套话,这就没法喜欢。总的来说,他是有艺术成就的,而且还不小;当然,和乌迪·艾伦的成就相比,还有不小的距离。现在他受到一些压力,说他的作品没有表达真善美,不够崇高等等。对此我倒有点看法。有件事大家可能都知道:艺术的标准在世界上各个地方是不同的。以美国的标准为例,到了欧洲就会被视为浅薄。我知道美国有部格调高尚的片子,说上帝本人来到了美国,变成了一个和蔼可亲的美国老人,到处去助人为乐;听见别人顺嘴溜出一句:感谢上帝……就接上一句:不客气!相信这个故事能使读者联想到一些国产片。这种片子叫欧洲人,尤其是法国人看了,一定会觉得浅薄。法国人对美国电影的看法是:除了乌迪·艾伦的电影,其它通通是狗屎一堆。 

相反,一些优秀的欧洲电影,美国人却没有看过。比方说,我小时看过一些极出色的意大利电影,如《罗马十一时》之类,美国人连听都没听说过。为此我请教过意大利人,他们皱着鼻子说道:美国人看我们的电影?他们看不懂!把知识分子扣除在外,仅就一般老百姓而论,欧洲人和美国人在文化上有些差异:欧洲、尤其是南欧的老百姓喜欢深刻的东西,美国人喜欢浅薄的东西;这一点连后者自己也是承认的。这种区别是因为欧洲有历史,美国没有历史所致。 

因为有这种区别,所以对艺术的认识也有深浅的不同。假定你有深刻的认识,对浅薄的艺术就会视为庸俗──这正是欧洲人对美国电影的看法。现在来谈谈我们中国人民是哪一种人。我毫不怀疑,因为有五千年的文明史,我们是全世界最深刻的人民。这一点连自以为深奥的欧洲人也是承认和佩服的。我在国外时,从电视上看到这样一件事:美籍华人建筑师贝先生主持了卢浮宫改造工程;法国人不服,有人说:美国人有什么文化?凭什么来动我们的卢浮宫?对此,贝先生从容答道:我有文化,我是中国人哪;对方也就哑口无言了。顺便说说,乌迪·艾伦的电影,充满了机智、反讽,在美国电影里是绝无仅有的。这也难怪,他虽是美国籍,却是犹太人,犹太文化当然不能小看。他的电影,能搞到手的我都看过,我觉得他不坏;但对我来说,还略嫌浅薄。略嫌浅薄的原因除中华文化比犹太文化历史悠久之外,还有别的。这也难怪,在美国的中国人当时不过百万,作为观众为数太少;他也只能迁就一下一般浅薄的美国观众。正因为中国的老百姓有历史、有文化、很深刻,想在中国搞出正面讴歌的作品可不容易啊;无论是美国导演还是欧洲导演,哪怕是犹太导演,对我们来说,都太浅薄。我认为,真善美是一种老旧的艺术标准;新的艺术标准是:搞出漂亮的、有技巧的、有能力的东西。批判现实主义是艺术的一支,它就不是什么真善美。王朔的东西在我看来基本属于批判现实主义,乌迪·艾伦也属这一类。这一类的艺术只有成熟和深刻的观众才能欣赏。 

在我看来,所谓真善美就是一种甜腻腻的正面描写,在一个成熟的现代国度里,一流的艺术作品没有不包括一点批判成份的。因此,从批判转入正面歌颂往往意味着变得浅薄。王朔和他的创作集体在影视圈、乃至文化圈里都是少数派。对于上述圈子里的多数派,我有这样一种意见:现在中青年文化人之大多数,对文化的一般见识,比之先辈老先生们,不唯没有提高,反而大幅度下降。为了防止激起众怒,我要声明:我自己尤其远不如老先生们。五六十年代的意大利的优秀电影一出现,老先生们就知道是好东西,给予“批判现实主义杰作”的美誉。现在的文化人不要说这种见识,连这样的名词都不知道,只会把“崇高”之类的名词径直讲出口来,也不怕直露。当然,大家不乏讴歌主旋律的决心,但能力,或者干脆说是才能,始终是个主要问题。多数的影视作品善良的创作动机是不容怀疑的,但都不好看。 

在此情况下,应该想到自己的艺术标准浅于大众;和有五千年文明史的中国人民之一般水平不符,宜往深处开掘──不要看不起小市民,也不要看不起芸芸众生。毛主席曾言:高贵者最愚蠢,卑贱者最聪明。你搞出的影视作品让人家看了身上爆起三层鸡皮疙瘩,谁聪明谁笨,也就不言自明。搞影视的人常抱怨老百姓口味太刁;这意思无非是说老百姓太聪明,自己太笨。我倒觉得不该这样子不打自招,这就显得更笨了。我觉得王朔过去的反嘲、反讽风格,使我们能见到深一层的东西。最近听说他要改变风格,向主流靠拢,倒使我感到忧虑。王朔是个聪明人。根据我的人生经验,假如没有遇上车祸,聪明人很不容易变笨。可能他想要耍点小聪明,给自己的作品披上一层主旋律的外衣,故作崇高之状。但是,中国人都太聪明,耍小聪明骗不了谁,只能骗骗自己。就拿他最近的的《红樱桃》来说,虽然披了一层主旋律的外衣,其核心内容和美国电影《九周半》还是一类。把这些不是一类的东西嫁接在一起,看上去真是不伦不类。照这个样子搞下去,广电部也未必会给他什么奖励,还要丢了观众。两样都没得到,那才叫倒霉。

\chapter{我怎样做青年的思想工作}

我有个外甥,天资聪明,虽然不甚用功,也考进了清华大学——对这件事,我是从他母系的血缘上来解释的,作为他的舅舅之一,我就极聪明。这孩子爱好摇滚音乐,白天上课,晚上弹吉它唱歌,还聚了几个同好,自称是在“排演”,但使邻居感到悲愤;这主要是因为他的吉它上有一种名为噪声发生器的设备,可以弹出砸碎铁锅的声音。要说清华的功课,可不是闹着玩的,每逢考期临近,他就要熬夜突击准备功课;这样一来就找不着时间睡觉。几个学期下来,眼见得尖嘴猴腮,两眼乌青,瘦得可以飘起来。他还想毕业后以摇滚音乐为生。不要说他父母觉得灾祸临门,连我都觉得玩摇滚很难成立为一种可行的生活方式——除非他学会喝风屙烟的本领。 

作为摇滚青年,我外甥也许能找到个在酒吧里周末弹唱的机会,但也挣不着什么钱;假如吵着了酒吧的邻居,或者遇到了要“整顿”什么,还有可能被请去蹲派出所——这种事我听说过。此类青年常在派出所的墙根下蹲成一排,状如在公厕里,和警察同志做轻松之调侃。当然,最后还要家长把他们领出来。这孩子的父母,也就是我的姐姐、姐夫,对这种前景深感忧虑,他们是体面人,丢不起这个脸。所以长辈们常要说他几句,但他不肯听。最不幸的是,我竟是他的楷模之一。我可没蹲过派出所,只不过是个自由撰稿人,但不知为什么,他觉得我的职业和摇滚青年有近似之处,口口声声竟说:舅舅可以理解我! 

因为这个缘故,不管我愿意不愿意,我都要负起责任,劝我外甥别做摇滚乐手,按他所学的专业去做电气工程师。虽然在家族之内,这事也属思想工作之类。按说该从理想、道德谈起,但因为在甥舅之间,就可以免掉,径直进入主题:“小子,你爸你妈养你不容易。好好把书念完,找个正经工作罢,别让他们操心啦。”回答当然是:他想这样做,但办不到。他热爱自己的音乐。我说:有爱好,这很好。你先挣些钱来把自己养住,再去爱好不迟。摇滚音乐我也不懂,就听过一个“一无所有”。歌是满好听的,但就这题目而论,好像不是一种快乐的生活。我外甥马上接上来道:舅舅,何必要快乐呢?痛苦是灵感的源泉哪。前人不是说:没有痛苦,叫什么诗人?——我记得这是莱蒙托夫的诗句。连这话他都知道,事情看来很有点不妙了…… 

痛苦是艺术的源泉,这似乎无法辩驳:在舞台上,人们唱的是“黄土高坡”、“一无所有”,在银幕上,看到的是《老井》、《菊豆》、《秋菊打官司》。不但中国,外国也是如此,就说音乐罢,柴科夫斯基“如歌的行板”是千古绝唱,据说素材是俄罗斯民歌“小伊万”,那也是人民痛苦的心声。美国女歌星玛瑞·凯瑞,以黑人灵歌的风格演唱,这可是当年黑奴们唱的歌……照此看来,我外甥决心选择一种痛苦的生活方式,以此净化灵魂,达到艺术的高峰,该是正确的了。但我偏说他不正确,因为他是我外甥,我对我姐姐总要有个交待。因此我说:不错,痛苦是艺术的源泉;但也不必是你的痛苦……柴科夫斯基自己可不是小伊万;玛瑞·凯瑞也没在南方的种植园里收过棉花;唱黄土高坡的都打扮得珠光宝气;演秋菊的卸了妆一点都不悲惨,她有的是钱……听说她还想嫁个大款。这种种事实说明了一个真理:别人的痛苦才是你艺术的源泉;而你去受苦,只会成为别人的艺术源泉。因为我外甥是个聪明孩子,他马上就想到了,虽然开掘出艺术的源泉,却不是自己的,这不合算——虽然我自己并不真这么想,但我把外甥说服了。他同意好好念书,毕业以后不搞摇滚,进公司去挣大钱。 

取得了这个成功之后,这几天我正在飘飘然,觉得有了一技之长。谁家有不听话的孩子都可以交给我说服,我也准备收点费,除写作之外,开辟个第二职业——职业思想工作者。但本文的目的却不是吹嘘我有这种本领,给自己作广告。而是要说明,思想工作有各种各样的作法。本文所示就是其中的一种:把正面说服和黑色幽默结合起来,马上就开辟了一片新天地……

\chapter{文明与反讽}

据说在基督教早期,有位传教士(死后被封为圣徒)被一帮野蛮的异教徒逮住,穿在烤架上用文火烤着,准备拿他做一道菜。该圣徒看到自己身体的下半截被烤得滋滋冒泡,上半截还纹丝未动,就说:喂!下面已经烤好了,该翻翻个了。 

烤肉比厨师还关心烹调过程,听上去很有点讽刺的味道。那些野蛮人也没办他的大不敬罪──这倒不是因为他们宽容。人都在烤着了,还能拿他怎么办。如果用棍子去打、拿鞭子去抽,都是和自己的午餐过不去。烤肉还没断气,一棍子打下去,将来吃起来就是一块淤血疙瘩,很不好吃。 

这个例子说明的是:只要你不怕做烤肉,就没有什么阻止你说俏皮话。但那些野蛮人听了多半是不笑的:总得有一定程度的文明,才能理解这种幽默──所以,幽默的圣徒就这样被没滋没味的人吃掉了。 

本文的主旨不是拿人做烤肉,而是想谈谈反讽──照我看,任何一个文明都该容许反讽的存在,这是一种解毒剂,可以防止人把事情干到没滋没味的程度。谁知动笔一写,竟写出件烧烤活人的事,我也不知道是为什么。 

让我们进入正题,且说维多利亚女王时期,英国的风气极是假正经。上等人说话都不提到腰以下的部位,连裤子这个字眼都不说,更不要说屁股和大腿。为了免得引起不良的联想,连钢琴腿都用布遮了起来。 

还有桩怪事,在餐桌上,鸡胸脯不叫鸡胸脯,叫作白肉。鸡大腿不叫鸡大腿,叫作黑肉──不分公鸡母鸡都是这么叫。这么称呼鸡肉,简直是脑子有点毛病。照我看,人若是连鸡的胸脯、大腿都不敢面对,就该去吃块砖头。问题不在于该不该禁欲,而在于这么搞实在是没劲透了。 

英国人就这么没滋没味的活着,结果是出了件怪事情:就在维多利亚时期,英国出现了一大批匿名出版的地下小说,通通是匪夷所思的色情读物。直到今天,你在美国逛书店,假如看到书架上钉块牌子,上书“维多利亚时期”,架子上放的准不是假正经,而是真色情……坦白地说,维多利亚时期的地下小说我读了不少──你爱说我什么就说什么好了。 

我不爱看色情书,但喜欢这种逆潮流而动的事──看了一些就开始觉得没劲。这些小说和时下书摊上署名“黑松林”的下流小册子还是有区别的,可以看出作者都是有文化的人。其中有一些书,还能称得上是种文学现象。有一本还有剑桥文学教授作的序,要是没有品,教授也不会给它写序。 

我觉得一部份作者是律师或者商人,还有几位是贵族。这是从内容推测出来的。至于书里写到的事,当然是不敢恭维。看来起初的一些作者还怀有反讽的动机,一面捧腹大笑,一面胡写乱写;搞到后来就开始变得没滋没味,把性都写到了荒诞不经的程度。所以,问题还不在于该不该写性,而在于不该写得没劲。 

过了一个世纪,英国的风气又是一变。无论是机场还是车站,附近都有个书店,布置得怪模怪样,霓虹灯乱闪,写着小孩不准入内,有的进门还要收点钱。就这么一惊一乍的,里面有点啥?还是维多利亚时期的小说以及它们的现代翻写本,这回简直是在犯贫。 

终于,福尔斯先生朝这种现象开了火。这位大文豪的作品中国人并不陌生,《法国中尉的女人》、《石屋藏娇》,国内都有译本。特别是后一本书,假如你读过维多利亚时期的原本,才能觉出逗来。有本维多利亚时期的地下小说,写一个光棍汉绑架了一个小姑娘,经过一段时间,那女孩爱上他了──这个故事被些无聊的家伙翻写来翻写去,翻到彻底没了劲。福尔斯先生的小说也写了这么个故事,只是那姑娘被关在地下室里,先是感冒了,后来得了肺炎,然后就死掉了。 

当然,福尔斯对女孩没有恶意,他只是在反对犯贫。总而言之,当一种现象(不管是社会现象还是文学现象)开始贫了的时候,就该兜头给它一瓢凉水。要不然它还会贫下去,就如美国人说的,散发出屁眼气味──我是福尔斯先生热烈的拥护者。 

我总觉的文学的使命就是制止整个社会变得无趣……当然,你要说福尔斯是反色情的义士,我也没什么可说的。你有权利把任何有趣的事往无趣处理解。但我总觉得福尔斯要是生活在维多利亚时期,恐怕也不会满足于把鸡腿叫作黑肉。他总要闹点事,写地下小说或者还不至于,但可能像王尔德一样,给自己招惹些麻烦。我觉得福尔斯是个反无趣的义士。 

假如我是福尔斯那样的人,现在该写点啥?我总禁不住想向《红楼梦》开火。其实我还有更大的题目,但又不想作死──早几年兴文化衫,有人在胸口印了几个字:活着没劲,觉得自己有了点幽默感,但所有写应景文章的人都要和这个人玩命,说他颓废──反讽别的就算了罢,这回只谈文学。 

曹雪芹本人不贫,但写各种“后梦”的人可是真够贫的。然后又闹了小一个世纪的红学。我觉得全中国无聊的男人都以为自己是贾宝玉,以为自己不是贾宝玉的,还算不上是个无聊的男人。看来我得把《红楼梦》反着写一下──当然,这本书不会印出来的:刚到主编的手里,他就要把我烤了。罪名是现成的:亵渎文化遗产,民族虚无主义。那位圣徒被烤的故事在我们这里,也不能那样讲,只能改作:该圣徒在烤架上不断高呼“我主基督万岁”,“圣母马利亚万岁”,“打倒异教徒”,直至完全烤熟。连这个故事也变得很没劲了。

\chapter{摆脱童稚状态}

李银河所译约翰盖格农《性社会学》第十七章“性环境”,集中叙述了美国对含有性内容的作品审查制度的变迁,因而成为全书最有神彩的一章。美国在两次大战前对“色情作品”的审查是最严的,受到打击的决不止是真正的色情作品。就以作家为例,不但海明威、雷马克有作品被禁,连最为“道学”的列夫托尔斯泰也上了禁书榜。在本世纪二十年代,美国的禁书榜上不但包括了乔依斯的《尤利西斯》,劳伦斯的《恋爱中的女人》等等,拉伯莱斯的《阿拉伯之夜》和雷马克的《西线无战事》也只能出节本。事有凑巧,我手上正好有一本国内出版的《西线无战事》,也是节本,而且节得上气不接下气。这种相似之处,我相信不仅仅是有趣而已。以前我们谈到国内对书刊、影视某些内容过于敏感时,总是将原因归纳为国情不同,社会制度不同,假如拿美国的三十年代和现在中国做个对比,就很容易发现新的线索。 

自一次大战后,美国对色情作品的检查呈稳步L升之势。一方面对性作品拼命压制,一方面严肃文学中性主题不断涌现,结果是从联邦到州、市政府开出了长得吓人的禁书书单。遭难的不只是上述作家,连圣经和莎翁的戏剧也只能通过节本和青少年见面。圣经抽掉了《雅歌》,莎翁抽掉了所谓猥亵的内容,结果是孩子们简直就看不明白。当然,受到限制的不仅是书刊,电影也没有逃出审查之网。在电影里禁止表现娼妓,长时间的作爱,禁止出现裸体、毒品、混血儿(!!)、性病、生育和嘲笑神职人员的镜头。 

当时严格的检查制度有其理论,这种理论认为一切对性的公开正面(非谴责性)的讨论都会导致性活动的泛滥,因为性知识是性行为的前兆。这就是说,性冲动是强大的,一受刺激就会自动表达出来。与此相辅相成的是另一个理论:性是危险的,人是薄弱的,必须控制性来保护人。这种观点和时下主张对文学作品严加控制的观点甚是相似。在我们国家里,现在正有人认为青少年的性犯罪和书籍、录相带有关系;还有一些家长反映孩子看了与性有关的书刊,影响了学习。因此主张对有性内容的书刊、录相严加限制。 

但是在我看来,像这样的观点因为是缺少科学训练的人提出的,多少总有点混乱不清的地方。比方说二十年代美国这种理论。在科学上我们只能承认它是一种假设,必须经过验证才能成立;而且它又是一种最糟不过的假设,定义不清,以致无法设计一种检验方法。我在报刊上看到一些统计数字,指出有多少性犯罪的青少年看过“不良”书刊或者黄色录相带,但是这样立论是错误的。 

实际上有效的立论应是指出有多少看过“不良”书刊的青少年犯了罪。在概率论上这是两个不同的反验概率,没有确定的关系,也不能够互相替代。至于家长说孩子看了与性有关的书刊,影响了学习,实际上是提出了一个因果模型??看某些书刊--影响学习。 

有经验的社会学家都会同意,建立一个可靠的因果模型是非常困难的。就以前述家长的抱怨为例,首先你要证明,你的孩子是先看了某些书刊,而后学习成绩才下降的:其次你要证明没有一个因素既影响到孩子看某种书,也影响到孩子的学习,我知道有一个因素要影响到这两件事,就是孩子的性成熟。故而上述家长的抱怨不能成立。现在的孩子营养好,性成熟早,对性知识的需求比他们的父母要早。据我所知,这是造成普遍忧虑的一个原因。假如家长只给他们馒头和咸菜吃,倒可以解决问题(使其性成熟期晚些到来)。以上论述要说明的是,关于色情作品对青少年的腐蚀作用,公众从常识的观点得出的结论和专家能做出的结论是不一样的,倘非如此,专家就不成其为专家。 

当然,人们给所谓色情作品定下的罪名不仅是腐蚀青少年,而且是腐蚀社会。在这方面书中有一个例子,就是六十年代的丹麦试验,1967年,丹麦开放了色情文学(真正的色情文学)作品,1969年开放了色情照片,规定色情作品可以生产,并出售给十六岁以上的公民。这项试验有了两项重要结果:其一是,丹麦人只是在初开禁时买了一些色情品,后来就不买或是很少买,以致在开禁几年后,所有的色情商店从哥本哈根居民区绝迹,目前只在两个小小的地区还在营业,而且只靠旅游者生存。本书作者对此的结论是:“人有多种兴趣,性只是其中的一种,色情品又只是其中一个小小的侧面。几乎没有人会把性当做自己的主要生活兴趣,把色情品当作自己的主要生活兴趣的人就更少见丹麦试验的第二个重大发现是色情业的开放对某些类型的犯罪有重大影响。猥亵儿童发案率下降了百分之八十,露阴癖也有大幅度下降。暴力污辱罪(强奸,狠亵)也减少了。其它犯罪数没有改变。这个例子说明色情作品的开放会减少而不是增加性犯罪,笔者引述这个例子,并不是主张什么,只是说明有此一事实而已。 

美国对色情作品的审查浪潮在二次大战后忽然退潮了,本书作者的观点是:这和美国从一个保守的、乡村为主的、单一清教国家,转变成了多元的国家有关。前者是反移民、反黑人、反共、排外的,社会掌握在道德警察千里;后来变成了一个都市化、工业化的社会,那种严格检查的背景就不存在了。这种说明对我们甚有意义,我们国家也是一个以乡村为主的国家。至于清教传统,我们没有过。清教徒认为人本性是恶的,必须加以限制。我们国家传统哲学认为人性本善,但是一到了“慕少艾”的年龄,他就不再是好东西了。所以对于青春期以后的人,两边的看法是完全一样的。本书作者给出了一个美国色憎开放程度的时间表,在此列出,以备参考:早于四十年代:任何女性的裸体或能引起这类联想的东西,包括掀起的衣裙、乳头的暗示,都属禁止;’四十年代:色情杂志上出现裸女背影;五十年代:乳房的侧影;,六十年代:出现乳头;《花花公子》杂志上出现女性阴部;七十年代:男性主殖器出现在《维瓦》和《花花女郎》杂志上,女性的阴唇出现在《阁楼》和《花花公子》杂志上,每当杂志走得更远时,审查员就大声疾呼,灾难就要降临;但是后来也没闹什么灾。所以这些人就落人了喊“狼来了”那个孩子的窘境。 

《性社会学》这本书里把对影视出版的审查,看作一种性环境。这种审查的主要目标是色情作品,所以含有性内容的严肃作品在这里只是被“捎带“的,所谓严肃作品,在我看来应该是虽然写到了性,但不以写性为目的的作品。这其中包括了以艺术上完美为目标的文学、影视作品,社会学、人类学的专业书,医学心理学的一部分书。据我所知,这类作品有时会遇到些麻烦。从某种意义上讲,严肃的作家、影视从业人员也可以算做专家,从专家的角度来看审查制度,应该得到什么样的结论呢? 

改革开放之初,聂华苓、安格尔夫妇到中国来,访问了我国一批老一代作家。安格尔在会见时间:你们中国的作品里,怎么没有写性呢?性是生活中很重要的事呀。我国一位年长的作家答道:我们中国人对此不感兴趣!这当然是骗洋鬼子的话,实际情况远非如此,但是洋鬼子不吃骗,又问道:你们中国有好多小孩子,这是怎么一回事?这句话的潜台词就是这些孩子不是你捏着鼻子,忍着恶心造出来的罢。当然,我们可以回答:我们就是像吃苦药那样做这件事!但是这样说话就等于承认我们都是伪君子。 

事实上性在中国人生活里也是很重要的事,我们享受性生活的态度和外国入没有什么不同。在这个方面没必要装神弄鬼。既然它重要,自然就要讨论。严肃的文学不能回避它,社会学和人类学要研究它,艺术电影要表现它;这是为了科学和艺术的缘故,然而社会要在这方面限制它,于是,问题就不再是性环境,而是知识环境的问题了。 

《性社会学》这本书描述了二十年代美国是怎样判决淫秽书的:起诉人从大部头书里摘出一段来,念给陪审员听,然后对他们说:难道你希望你们的孩子读这样的书吗?结果海明威。劳伦斯、乔伊斯就这样被禁掉了。我不知道我们国家里现在有没有像海明威那样伟大的作家,但我知道假如有的话,他一定为难以发表作品而苦恼。海明威能写出让起诉人满意的书吗?不能。 

我本人就是个作者。任何作者的书出版以后,会卖给谁他是不能够控制的。假如一位严肃作家写了性,尽管其本心不是煽情、媚俗,而是追求表达生活的真谛,也不能防止这书到了某个男孩子手里,起到手淫前性唤起的作用。故此社会对作家的判决是:因为有这样的男孩子存在,所以你的书不能出。这不是太冤了吗?但我以为这样的事还不算冤,社会学家和心理学家比他还要冤。事实上社会要求每个严肃作者、专业作者把自己的读者想象成十六岁的男孩子,而且这些男孩似乎还是不求上进、随时要学坏的那一种。 

我本人又是个读者,年登不惑,需要看专业书,并且喜欢看严肃的文学书,但是市面上只有六十二个故事的《十日谈》,节本《金瓶梅》,和被宰得七零八落的雷马克;还有一些性心理学性社会学的书,不容气他说,出得完全是乌七八糟。前些日子买了一本福科的《性史》,根本看不懂,现在正想办法找英文本来看。这种情形对我是一种极大的损害,在此我毫不谦虚他说,我是个高层次的读者,可是书刊检查却拿我当十六岁的孩子看待。 

这种事情背后隐含着一个逻辑,就是我们国家的出版事业必须就低不就高,一本书能不能出,并不取决于它将有众多的有艺术鉴赏力或者有专业知识的读者,这本书应该对他们有益;而是取决于社会上存在着一些没有鉴赏力或没有专业知识的读者,这本书不能对他们有害。对我来说,书刊审查不是个性环境,而是个知识环境,对其他知识分子也是这样的,这一点是《性社会学》上没有提到的,二三十年代,有头脑的美国人,如海明威等,全在欧洲呆着,后来希特勒把知识分子又都撵回到美国去,所以美国才有了科学发达、人文汇萃的时代。假如希特勒不在欧洲烧书、杀犹大人,我敢说现在美国和欧洲相比,依然是个土得掉渣的国家,我不敢说国内人材凋零是书刊检查之故,但是美国如果现在出了希特勒,我们国内的人材一定会多起来。 

假如说市场上有我需要的书,可能会不利于某些顽劣少年的成长的话,有利于少年成长的书也不适合于我们。

\chapter{从INTERNET说起}

我的电脑还没连网,也想过要和Internet连上。据说,网上黄毒泛滥,还有些反动的东西在传播,这些说法把我吓住了。前些时候有人建议对网络加以限制,我很赞成。说实在的,哪能容许信息自由的传播。但假如我对这件事还有点了解,我要说:除了一剪子剪掉,没有什么限制的方法。那东西太快,太邪门了。现代社会信息爆炸,想要审查太困难,不如禁止方便。假如我作生意,或者搞科技,没有网络会有些困难。但我何必为商人、工程师们操心?在信息高速网上,海量的信息在流动。但是我,一个爬格子的,不知道它们也能行。所以,把Internet剪掉罢,省得我听了心烦。 

Internet是传输信息的工具。还有处理信息的工具,就是各种个人电脑。你想想看,没有电脑,有网也接不上。再说,磁盘、光盘也足以贩黄。必须禁掉电脑,这才是治本。这回我可有点舍不得——大约十年前,我就买了一台个人电脑。到现在换到了第五台。花钱不说,还下了很多工夫,现在用的软件都是我自己写的。我用它写文章,做科学工作:算题,做统计——顺便说一句,用电脑来作统计是种幸福,没有电脑,统计工作是种巨大的痛苦。 

但是它不学好,贩起黄毒来了,这可是它自己作死,别人救不了它。看在十年老交情上,我为它说几句好话:早期的电脑是无害的。那种空调机似的庞然大物算起题来嘎嘎做响,没有能力演示黄毒。后来的486、586才是有罪的:这些机器硬件能力突飞猛进,既能干好事,也能干坏事,把它禁了吧……但现在要买过时的电脑,不一定能买到。为此,可以要求IBM给我们重开生产线,制造早期的PC机。洋鬼子听了瞪眼,说:你们是不是有毛病?回答应该是:我们没毛病,你才有毛病——但要防止他把我们的商务代表送进疯人院。当然,如果决定了禁掉一切电脑,我也能对付。我可以用纸笔写作,要算统计时就打算盘。不会打算盘的可以拣冰棍棍儿计数——满地拣棍儿是有点难看,但是——谢天谢地,我现在很少作统计了。 

除了电脑,电影电视也在散布不良信息。在这方面,我的态度是坚定的:我赞成严加管理。首先,外国的影视作品与国情不符,应该通通禁掉。其次,国内的影视从业人员良莠不齐,做出的作品也多有不好的……我是写小说的,与影视无缘,只不过是挣点小钱。王朔、冯小刚,还有大批的影星们,学历都不如我,搞出的东西我也看不入眼。但他们可都发大财了。应该严格审查——话又说回来,把Internet上的通讯逐贞看过才放行,这是办不到的;一百二十集的连续剧从头看到尾也不大容易。倒不如通通禁掉算了。 

文化大革命十年,只看八个样板戏不也活过来了嘛。我可不像年轻人,声、光、电、影一样都少不了。我有本书看看就行了。说来说去,我把流行音乐漏掉了。这种乌七八糟的东西,应该首先禁掉。年轻人没有事,可以多搞些体育锻炼,既陶冶了性情,又锻炼了身体……这样禁来禁去,总有一天禁到我身上。我的小说内容健康,但让我逐行说明每一句都是良好的信息,我也做不到。再说,到那时我已经吓傻了,哪有精神给自己辨护。电影电视都能禁,为什么不能禁小说?我们爱读书,还有不识字的人呢,他们准赞成禁书。好吧,我不写作了,到车站上去扛大包。我的身体很好,能当搬运工。别的作家未必扛得动大包……我赞成对生活空间加以压缩,只要压不到我;但压来压去,结果却出乎我的想像。 

海明威在《钟为谁鸣》说过这个意思:所有的人是一个整体,别人的不幸就是你的不幸。所以,不要问丧钟是为谁而鸣——它就是为你而鸣。但这个想法我觉得陌生,我就盼着别人倒霉。五十多年前,有个德国的新教牧师说:起初,他们抓共产党员,我不说话,因为我不是工会会员;后来,他们抓犹太人,我不说话,因为我是亚利安人。后来他们抓天主教徒,我不说话,因为我是新教徒……最后他们来抓我,已经没人能为我说话了。众所周知,这里不是纳粹德国,我也不是新教牧师。所以,这些话我也不想记住。

\chapter{《代价论》、乌托邦与圣贤}

郑也夫先生的《代价论》在哈佛燕京丛书里出版了,书在手边放了很长时间都没顾上看——我以为如果没有精力就读一本书,那是对作者的不敬。最近细看了一下,觉得也夫先生文笔流畅,书也读得很多,文献准备得比较充分。就书论书,应该说是本很好的书;但就书中包含的思想而论,又觉得颇为抵触。 

说来也怪,我太太是社会学家,我本人也做过社会科学的研究工作,但我对一些社会科学家的思想越来越觉得隔膜。这本书的主旨,主要是中庸思想的推广,还提出一个哲理:任何一种社会伦理都必须付出代价,做什么事都要把代价考虑在内等等。这些想法是不错的,但我总觉有些问题当作技术问题看比当原则问题更恰当些。当你追求一种有利效果时,有若干不利的影响随之产生,这在工程上最常见不过,有很多描述和解决这种问题的数学工具——换言之,如果一心一意地要背弃近代科学的分析方法,自然可以提出很多的原则,但这些原则有多大用处就很难说了。 

中庸的思想放在一个只凭感觉做事的古代人脑子里会有用——比方说他要蒸馒头,记住中庸二字,就不会使馒头发酸或者碱大。但近代的化工技师就不需要记住中庸的原则,他要做的是测一下Ph值,再用天平去称量苏打的份量。总而言之,我不以为中庸的思想有任何高明之处,当然这也可能是迷信分析分析方法造成的一种偏见。我听到社会学家说过,西方人发明的分析方法已经过时,今后我们要用中国人发明的整合方法作研究;又听到女权主义者说,男人发明的理性的方法过时了,我们要用感性的方法作研究。但我总以为,作研究才是最主要的。 

《代价论》分专章讨论很多社会学专题,有些问题带有专门性我不便评论。但有一章论及乌托邦的,我对这个问题特别有兴趣。“乌托邦”这个名字来自摩尔的同名小说,作为一种文学题材,它有独特的生命力。除了有正面乌托邦,还有反面乌托邦。这后一种题材生命力尤旺。作为一种制度,它确有极不妥之处。首先,它总是一种极端国家主义的制度,压制个人;其次,它僵化没有生命力。最后,并非最不重要,它规定了一种呆板的生活方式,在其中生活一定乏味得要死。近代思想家对它多有批判,郑先生也引用了。但他又说,乌托邦可以激励人们向上,使大家保持蓬勃的朝气,这就是我所不能同意的了。 

乌托邦是前人犯下的一个错误。不管哪种乌托邦,总是从一个人的头脑里想像出来的一个人类社会,包括一个虚拟的政治制度、意识形态、生活方式,而非自然形成的人类社会。假如它是本小说,那倒没什么说的。要让后世的人都到其中去生活,就是一种极其猖狂的狂妄。现世独裁者的狂妄无非是自己一颗头脑代天下苍生思想,而乌托邦的缔造者的是用自己一次的思想,带替千秋万代后世人的思想,假如不把后世人变得愚蠢,这就无论如何也不可能成功。现代社会的实践证明,不要说至善至美的社会,就是个稍微过得去的社会,也少不了亿万人智力的推动。无论构思乌托邦,还是实现乌托邦,都是一种错误,所以我就不明白它怎能激励人们向上。我们曾经经历过乌托邦鼓舞出的蓬勃朝气,只可惜那是一种特殊的愚蠢而已。 

从郑也夫的《代价论》扯到乌托邦,已经扯得够远的了。下一步我又要扯到圣贤身上去,这题目和郑先生的书没有一丝一毫的关系。讨厌乌托邦的人上溯它的源头,一直寻到柏拉图和他的《理想国》,然后朝他猛烈开火攻击。中国的自由派则另有攻击对象,说种种不自由的始作俑者。此时此地我也不敢说自己是个自由派,但我觉得这种攻击有些道理。罗素先生攻击柏拉图是始作俑者,给他这样一个罪名:一代又一代的青年读了理想国,胸中燃烧起万丈雄心,想当莱库格斯或一个哲人王,只可惜对权势的爱好总是使他们误入歧途。这话我想了又想,终于想到:说理想国的爱好者们爱好权势,恐怕是不当的指责。莱库格斯就不说了,哲人王是什么?就是圣贤啊。

\chapter{道德堕落与知识分子}

道德堕落与知识分子看到《东方》杂志一期上王力雄先生的大作《渴望堕落》,觉得很有趣。我同意王先生的一些论点,但是在本质上,我站在王先生的对立面上,持反对王先生的态度。我喜欢王先生直言不讳的文风,只可惜那种严肃的笔调是我学不来的。 

一、知识分子的罪名之一:亵读神圣 

如王先生所言,现在一些知识分子放弃了道德职守,摆脱了传统价值观念的束缚,正在“痞”下去,具体的表现是言语粗俗,放弃理想,厚颜无耻,亵读神圣。我认为,知识分子的语言的确应当斯文些,关心的事情也该和大众有些区别。不过这些事对于知识分子只是未节,他真正的职责在于对科学和文化有所贡献;而这种贡献不是仅从道德上可以评判的,甚至可以说,它和道德根本就不搭界。 

举例来说,达尔文先生在基督教社会里提出了进化论,所以有好多人说他不道德。我们作为旁观者,当然可以说:一个科学理论,你只能说它对不对,不能拿道德来评说。但假若你是个教士,必然要说达尔文亵读神圣。鉴于这个情况,我认为满脑子神圣教条的人只宜作教士,不适于作知识分子,最起码不适于当一流的知识分子。倘若有人说,对于科学家来说,科学就是神圣的;我也不同意。我的一位老师说过,中国人对于科学的认识,经历过若干个阶段。 

首先,视科学如洪水猛兽,故而砍电杆,毁铁路(义和团的作为);继而视科学如巫术,以为学会几个法门,就可以船坚炮利;后来就视科学力神圣的宗教,拜倒在它面前。他老人家成为一位有成就的历史学家后,才体会到科学是个不断学习的过程。我认为他最后的体会是对的,对于每个知识分子而言,他毕生从事的事业,只能是个不断学习的过程;而不是顶礼膜拜。爱因斯但身为物理学家,却不认为牛顿力学神圣,所以才有了相对论。这个例子说明,对于知识分子来说,知识不神圣??我们用的字眼是:真实、可信、完美;到此为止。而不是知识的东西更不神圣。所以,对一位知识分子的工作而言,亵读神圣本身不是罪名,要看他有没有理由这样做。 

二、知识分子罪名之二:厚颜无耻 

另一个问题是知识分子应不应该比别人更知耻。过去在西方社会里,身为一个同性恋者是很可耻的,计算机科学的奠基人图林先生就是个同性恋者,败露后自杀了,死时正在有作为的年龄。据说柴科夫斯基也是这样死的。按王先生的标准,这该算知耻近勇罢。但我要是生于这两位先生的年代,并且认识他们,就会劝他们“无耻”地活下去。我这样做,是出于对科学和音乐的热爱。在一个社会里,大众所信奉的价值观,是不是该成为知识分子的金科玉律呢?我认为这是可以存疑的。当年罗素先生在纽约教书,有学生问他对同性恋有何看法。他用他那颗伟大学者的头脑考虑后,回答了。这回答流传了出去,招来一个没甚文化的老太太告了他一状,说他诲盗诲淫,害得他老人家失了教席,灰头土脸地回英格兰去。这个故事说明的是:不能强求知识分子与一般人在价值观方面一致,这是向下拉齐。除了价值观的基本方面,知识分子的价值体系应该有点独特的地方,举例来说,画家画裸体模特,和小流氓爬女浴室窗户不可以等量齐观,虽然在表面上这两种行为有点像。 

三、知识分子的其它罪名 

王先生所举知识分子的罪名,多是从价值观或者道德方面来说的。我党得多少带点宋明理学或者宗教的气味。至于说知识分子言语粗俗,举的例子是电视片中的人物,或者电影明星。我以为这些人物不典型,是不是知识分子都有疑问。假如有老外问我,中国哪些人学识渊博,有独立见解,我说出影星、歌星的名字来,那我喝的肯定是不止二两啦。现在有些知识分子下了海,引起了王先生很大的忧虑。其实下了海就不是知识分子了,还说人家干什么。我觉得知识分子就该是喜欢弄点学问的人,为此不得不受点穷;而非特意的喜欢熬穷。假如说安于清贫、安于住筒子楼、安于营养不良是好品格,恐怕是有点变态。所谓身体发肤,受之父母,和自己过不去,就是和爹娘过不去。再说,咱们还有妻子儿女。 

王先生文章里提到的人物主要是作家,我举这些例子净是科学家,或许显得有点文不对题。作家也是知识分子,但是他们的事业透明度更大:字人人识,话人人懂(虽然意思未必懂),所以格外倒霉。我认为,在知识分子大家庭里,他们最值得同情,也最需要大家帮助。我听说有位老先生对贾平凹先生的《废都)有如下评价:“国家将亡,必有妖孽”。不管贾先生这本书如何,老先生言重了。真正的妖孽是康生、姚文元之辈,只不过他们猖狂时来头甚大,谁也惹不起。将来咱们国家再出妖孽(我希望不要再出了),大概还是那种人物。像这样的话我们该攒着,见到那种人再说。科学家维纳认为,人在做两种不同性质的事,一类如棋手,成败由他的最坏状态决定,也就是说,一局里只要犯了错误就全完了。还有一类如发明家,只要有一天状态好,做成了发明,就成功了,在此之前犯多少次糊涂都可以。贾先生从事的是后一类工作,就算《废都》没写好,将来还可以写出好书。这样看问题,才是知识分子对待知识分子的态度。玉先生说,知识分子会腐化社会,我认为是对的,姚文元也算个知识分子,却喜欢咬别的知识分子,带动了大家互相咬,弄得大家都像野狗。他就是这样腐化了社会。 

四、知识分子的真实罪孽 

如果让我来说中国知识分子的罪状,我也能举出一堆:同类相残(文人相轻),内心压抑,口是心非……不过这样说话是不对的。首先,不该对别人滥做价值判断。其次,说话要有凭据。所以,我不能说这样的话。我认为中国的知识分子只在一个方面有欠缺:他们的工作缺少成绩,尤其是缺少一流的成果。以人口比例来算,现代一切科学文化的成果,就该有四分之一出在中国。实际上远达不到这个比例。学术界就是这样的局面,所以我们劝年轻人从事学术时总要说:要耐得住寂寞!好像劝寡妇守空房一样。除了家徒四壁,还有头脑里空空如也,这让人怎么个熬法嘛。在文学方面,我同意王先生所说的,中国作家已经痞掉了;从语言到思想,不比大众高明。但说大家的人品有问题,我认为是不对的。没有杜拉斯,没有昆德拉,只有王朔的调侃小说。顺便说一句,我认为王朔的小说挺好看,但要说那就是“modernclassic”,则是我万难接受、万难领会的。痞是不好的,但其根源不在道德上。真正的原因是贫乏。没有感性的天才,就不会有杜拉斯《情人》那样的杰作;没有犀利的解析,也就没有昆德拉。作家想要写出不同流俗之作,自己的头脑就要在感性和理性两方面再丰富些,而不是故作清高就能解决问题的。我国的作家朋友只要提高文学修养,还大有机会。就算遇到了挫折,还可以从头开始嘛。 

五、知识分子该干什么? 

王先生的文章里,我最不能同意的就是结尾的一段。他说,中国社会的精神结构已经千疮百孔,知识分子应司重建之责。这个结构是指道德体系吧。我还真没看见疮在哪里、孔在哪里。有些知识分子下了海,不过是挣几个小钱而已,还没创建“王安”、“苹果”那样的大公司呢,王先生就说我们“投机逐利”。文章没怎么写,就“厚颜无耻”。还有丧失人格、渴望堕落、出卖原则、亵读神圣(这句话最怪,不知王先生信什么教)、藐视理想。倘若这些罪名一齐成立,也别等红卫兵、褐衫队来动手,大伙就一齐吊死了罢,别活着现眼。但是我相信,王先生只是顺嘴说说,并没把咱们看得那么坏。 

最后说说知识分子该干什么。在我看来,知识分子可以干两件事:其一,创造精神财富;其二,不让别人创造精神财富。中国的知识分子后一样向来比较出色,我倒希望大伙在前一样上也较出色。“重建精神结构”是好事,可别建出个大笼子把大家关进去;再造出些大棍子,把大家揍一顿。我们这个国家最敬重读书人,可是读书人总是不见太平。大家可以静下心来想想原因。

\chapter{东西方快乐观区别之我见}

东西方精神的最大区别在于西方人沉迷于物欲,而东方人精于人与人的关系;前者从征服中得到满足,后者从人与人的相亲相爱中汲取幸福。 

一次大战刚结束时,梁任公旅欧归来,就看到前一种精神的不足;那个时候列强竞相掠夺世界,以致打了起来,生灵涂炭——任公觉得东方人有资格给他们上一课;而当时罗素先生接触了东方文明以后,也觉得颇有教益。 

现在时间到了世纪末,不少东方人还觉得有资格给西方人上一课。这倒不是因为又打了大仗,而是西方人的物欲毫无止境,搞得能源、生态一齐闹了危机;而人际关系又是那么冷酷无情。但是这一课没有听众,急得咱们自己都抓耳挠腮。 

这种物欲横流的西方病,我们的老祖宗早就诊断过。当年孟子见梁惠王,梁惠王问利,孟子就说,上下交征利而国危矣。所谓利,就是能满足物质欲望的东西。在古代,生产力有限,想要利,就得从别人那里夺,争的凶了就要打破头。现代科技发达,可以从开发自然里得到利益,搞得过了头,又要造成生态危机。孟子提出一种东西作为“利”的替代物,这个暂且不提。我们来讨论一下西方病的根源。 

笔者既学过文,又学过理,两边都是糊里糊涂,且有好做不伦不类的类比之恶习。不管怎样,大家可以听听这种类比可有道理。人可以从环境中得到满足,这种满足又成为他行动的动力。比方说,冷天烧了暖气觉得舒服,热天放了冷气又觉得舒服,结果他就要把房间恒到华氏70度,购买空调机,耗费无数电力;骑车比走路舒服,坐车又比骑车舒服,结果是人人买汽车,消耗无数汽油。由此看来,舒服了还要更舒服,正是西方人掠夺自然的动力。 

这在控制论上叫作正反馈,社会就相当于一个放大器,人首先有某种待满足的物欲,在欲望推动下采取的行动使欲望满足,得到了乐趣,这都是正常的。乐趣又产生欲望,又反馈回去成了再做这行动的动力,于是越来越凶,成了一种毛病。 

玩过无线电的人都知道,有时候正反馈讨厌得很,状似抽疯:假如话筒和喇叭串了,就会闹出这种毛病,喇叭里的声音又进了话筒,放大数百倍出来再串回去,结果就是要吵死人——行话叫作“自激”。 

在我们这里看来,西方社会正在自激,舒服了还要更舒服,搅到最后,连什么是舒服都不清不楚,早晚把自己烧掉了完事。这种弊病的根源在于它是个欲望的放大器——它在满足物欲方面能做得很成功,当然也有现代技术在做它的后盾。孟老夫子当年就提出要制止这种自激,提出个好东西,叫作“仁义”,仁者,亲亲也,义者,敬长也,亲亲敬长很快乐,又不毁坏什么,这不是挺好的吗(见《孟子》)。 

有关自激像抽疯,还可以举出一个例子。凡高级动物脑子里都有快乐中枢,对那地方施以刺激,你就乐不可支。据说吸毒会成瘾,就是因为毒品直接往那里作用。有段科普文章里说到有几个缺德科学家在海豚脑子里装了刺激快乐中枢的电极,又给海豚一个电键,让它可以自己刺激自己。结果它就抽了疯,废寝忘食地狂敲不止。我当然不希望他们是在寻海豚的开心,而希望他们是在做重要的试验。不管怎么说吧,上下交征利,是抽这种疯,无止境地开发自然,也是抽这种疯。 

我们可以教给西方人的就是:咱们可以从人与人的关系里得到乐趣。当然,这种乐趣里最直接的就是性爱,但是孟子毫不犹豫地把它挖了出去,虽然讲出的道理很是牵强——说“慕少艾”不是先天的“良知良能”,是后天学坏了,现代人当然要得出相反的结论。实际原因也很简单,它可能导致自激。 

孟子说,乐之实,乃是父子之情,手足之情(顺便说说,有注者说这个“乐”是音乐之“乐”,我不大信)。再辅之以礼,就可以解决一切社会问题。这是孟子的说法,但我不大信服;他所说的那种快乐也可以自激,就如孟子自己说的:“乐则生矣,生则恶可已也,恶可已,则不知足之蹈之手之舞之”,谁要说这不叫抽疯,那我倒想知道一下什么是抽疯。而且我认为,假如没有一大帮人站在一边拍巴掌,谁也抽不到这种程度——孟夫子本人当然例外。 

中国人在人际关系里找到了乐趣,我们认为这是自己的一大优点。因为有此优点,我们既不冷漠,又不自私,而且人与自然的关系和谐。中国社会四平八稳,不容易出毛病。这些都是我们的优点,我也不敢妄自菲薄。但是基督曾说,不要只看到别人眼里有木刺,没准儿自己眼里还有大梁呢。中国的传统道德,讲究得过了头,一样会导致抽疯式的举动。这是因为中国的传统社会在这方面也是个放大器。人行忠孝节义,就能得忠臣孝子节妇义士的美名,这种美名刺激你更去行忠孝节义,循环往复,最后你连自己在干什么都搞不清。 

举例言之,我们讲究孝道,人人都说孝子好。孝子一吃香,然后也能导致正反馈,从而走火入魔:什么郭解埋儿啦,卧冰求鱼啦,谁能说这不是自激现象?再举一例,中国传统道德里要求妇女守身如玉,从一而终,这可是个好道德罢?于是人人盛赞节烈妇女。翻开历史一看,女人为了节烈,割鼻子拉耳朵的都有。鼻子耳朵不比头发指甲,割了长不出来,而且人身上有此零件,必有用处;拿掉了肯定有不便处。若是为“节烈”之名而自杀,肯定是更加不妥的了。此类行为,就像那条抽疯的海豚。 

文化革命中大跳忠字舞时,也是抽的这种疯;你越是五迷三道,晕头胀脑,大家就越说你好,所以当时九亿人民都像发了四十度的高烧。不用我说,你就能发现,这正是孟子说的那种手舞足蹈的现象。经历了文化革命的中国人,用不着我来提醒,就知道它是有很大害处的。“忠”可算是有东方特色的,而且可以说它是孝的一种变体,所以东方精神发扬到了极致,和西方精神一样的不合理,没准还会更坏。我们这里不追求物欲的极大满足,物质照样不够用。 

正如新儒家学者所说,我们的文化重人,所以人多了一定好,假如是自己的种,那就更好:作父母的断断不肯因为穷、养不起就不生,生得多了,人际关系才能极大丰富,对不对?于是你有一大帮儿子就有人羡慕。结果中国有十二亿人,虽然都没有要求开私家车,用空调机,能源也是不够用。只要一日三餐的柴禾,就能把山林砍光,只要有口饭吃,地就不够种。偶而出门一看,到处是人山人海,我就觉得咱们这里自激得很厉害。 

虽然就个体而言没有什么过分的物欲,就总体来看还是很过分,中国人一年烧掉十亿吨煤,造出无数垃圾,同样也超过地球的承受力。现在社会虽然平稳,拿着这么多的人口也是头疼。故而要计划生育,这就使人伦的基础大受损害。倘若这种东方特色不能改变,那就只能把大家变到身高三寸,那么所有的中国人又可以快乐的生活,并且享受优越的人际关系。可以预言,过个三五百年,三寸又嫌太高。就这么缩下去,一直缩到风能吹走,看来也不是好办法。 

本文的主旨,在于比较东西方不同的快乐观。罗素在讨论伦理问题时曾经指出,人人都希求幸福,假如说,人得到自己希求的东西就是幸福,那就言之成理。倘若说因为某件事是幸福的,所以我们就希求它,那就是错误的。谁也不是因为吃是幸福的才饿的呀。幸福的来源,就是不计苦乐、不计利弊、自然存在的需要,这种需要的种类、分量,都不是可以任意指定的。当然,这是人在正常时的情形,被人哄到五迷三道,晕头转向的人不在此列。 

马尔库塞说西方社会有病,是说它把物质消费本身当成了需要,消费不是满足需求,而是满足起哄。我能够理解这种毛病是什么,但是缺少亲身体验。假如把人际关系和谐本身也当成需要,像孟子说的那样:行孝本身是快乐的,所以去行孝,当然就更是有病,而且这种毛病我亲身体验过了(在文化革命里人人表忠心的时候)。 

人满足物质欲望的结果是消费,人际关系的和谐也是人避免孤独这一需要的结果。一种需要本身是不会过分的,只有人硬要去夸大它,导致了自激时才会过分。饿了,找个干净饭馆吃个饭,有什么过分?想要在吃饭时显示你有钱才过分。你有个爸爸,你很爱他,要对他好,有什么过分?非要在这件事上显示你是个大孝子,让别人来称赞才过分。需要本身只有一分,你非把它弄到十分,这原因大家心里明白,社会对个人不是只起好作用,它还是个起哄的场所,干什么事都要别人说好,赢得一些彩声,正是这件事在导致自激。东方社会有东方的起哄法,西方有西方的起哄法。而且两边比较起来,还是东方社会里的人更爱起哄。假如此说是正确的,那么真正的幸福就是让人在社会的法理、公德约束下,自觉自愿的去生活;需要什么,就去争取什么;需要满足之后,就让大家都得会儿消停。这当然需要所有的人都有点文化修养,有点独立思考的能力,并且对自己的生活负起责任来,同时对别人的事少起点哄。 

这当然不容易,但这是唯一的希望。看到人们在为物质自激,就放出人际关系的自激去干扰;看到人在人际关系里自激,就放出物质方面的自激去干扰;这样激来扰去,听上去就不是个道理。搞得不好,还能把两种毛病一齐染上:出了门,穷极奢欲,非奔驰车不坐,非毒蛇王八不吃,甚至还要吃金箔、屙金屎;回了家,又满嘴仁义道德,整个一个封建家长,指挥上演种种草菅人命的丑剧(就像大邱庄发生过的那样);要不就走向另一极端,对物质和人际关系都没了兴趣,了无生趣——假如我还不算太孤陋寡闻,这两样的人物我们在当代中国已经看到了。

\chapter{工作与人生}

我现在已经活到了人生的中途,拿一日来比喻人的一生,现在正是中午。人在童年时从朦胧中醒来,需要一些时间来克服清晨的软弱,然后就要投入工作;在正午时分,他的精力最为充沛,但已隐隐感到疲惫;到了黄昏时节,就要总结一日的工作,准备沉入永恒的休息。按我这种说法,工作是人一生的主题。这个想法不是人人都能同意的。我知道在中国,农村的人把生儿育女看作是一生的主题。把儿女养大,自己就死掉,给他们空出地方来——这是很流行的想法。在城市里则另有一种想法,但不知是不是很流行:它把取得社会地位看作一生的主题。站在北京八宝山的骨灰墙前,可以体会到这种想法。我在那里看到一位已故的大叔墓上写着:副系主任、支部副书记、副教授、某某教研室副主任,等等。假如能把这些“副”字去掉个把,对这位大叔当然更好一些,但这些“副”字最能证明有这样一种想法。顺便说一句,我到美国的公墓里看过,发现他们的墓碑上只写两件事:一是生卒年月,二是某年至某年服兵役;这就是说,他们以为人的一生只有这两件事值得记述:这位上帝的子民曾经来到尘世,以及这位公民曾去为国尽忠,写别的都是多余的,我觉得这种想法比较质朴……恐怕在一份青年刊物上写这些墓前的景物是太过伤感,还是及早回到正题上来罢。 

我想要把自己对人生的看法推荐给青年朋友们:人从工作中可以得到乐趣,这是一种巨大的好处。相比之下,从金钱、权力、生育子女方面可以得到的快乐,总要受到制约。举例来说,现在把生育作为生活的主题,首先是不合时宜;其次,人在生育力方面比兔子大为不如,更不要说和黄花鱼相比较;在这方面很难取得无穷无尽的成就。我对权力没有兴趣,对钱有一些兴趣,但也不愿为它去受罪——做我想做的事(这件事对我来说,就是写小说),并且把它做好,这就是我的目标。我想,和我志趣相投的人总不会是一个都没有。 

根据我的经验,人在年轻时,最头疼的一件事就是决定自己这一生要做什么。在这方面,我倒没有什么具体的建议:干什么都可以,但最好不要写小说,这是和我抢饭碗。当然,假如你执意要写,我也没理由反对。总而言之,干什么都是好的;但要干出个样子来,这才是人的价值和尊严所在。人在工作时,不单要用到手、腿和腰,还要用脑子和自己的心胸。我总觉得国人对这后一方面不够重视,这样就会把工作看成是受罪。失掉了快乐最主要的源泉,对生活的态度也会因之变得灰暗…… 

人活在世上,不但有身体,还有头脑和心胸——对此请勿从解剖学上理解。人脑是怎样的一种东西,科学还不能说清楚。心胸是怎么回事就更难说清。对我自己来说,心胸是我在生活中想要达到的最低目标。某件事有悖于我的心胸,我就认为它不值得一做;某个人有悖于我的心胸,我就觉得他不值得一交;某种生活有悖于我的心胸,我就会以为它不值得一过。罗素先生曾言,对人来说,不加检点的生活,确实不值得一过。我同意他的意见:不加检点的生活,属于不能接受的生活之一种。人必须过他可以接受的生活,这恰恰是他改变一切的动力。人有了心胸,就可以用它来改变自己的生活。 

中国人喜欢接受这样的想法:只要能活着就是好的,活成什么样子无所谓。从一些电影的名字就可以看出来:《活着》、《找乐》……我对这种想法是断然地不赞成,因为抱有这种想法的人就可能活成任何一种糟糕的样子,从而使生活本身失去意义。高尚、清洁、充满乐趣的生活是好的,人们很容易得到共识。卑下、肮脏、贫乏的生活是不好的,这也能得到共识。但只有这两条远远不够。我以写作为生,我知道某种文章好,也知道某种文章坏。仅知道这两条尚不足以开始写作。还有更加重要的一条,那就是:某种样子的文章对我来说不可取,绝不能让它从我笔下写出来,冠以我的名字登在报刊上。以小喻大,这也是我对生活的态度。

\chapter{积极的结论}

我小的时候,有一段很特别的时期。有一天,我父亲对我姥姥说,一亩地里能打三十万斤粮食,而我的外祖母一位农村来的老太大,跳着小脚叫了起来:“杀了俺俺也不信。”她还算了一本细帐,说一亩地上堆三十万斤粮,大概平地有两尺厚的一层。当时我们家里的人都攻击我姥姥觉悟太低,不明事理。我当时只有六岁,但也得出了自己的结论:我姥姥是错误的。事隔三十年,回头一想,发现我姥姥还是明白事理的。亩产三十万斤粮食会造成特殊的困难,那么多的粮食谁也吃不了,只好堆在那里,以致地面以每十年七至八米的速度上升,这样的速度在地理上实在是骇人听闻;几十年后,平地上就会出现一些山峦,这样水田就会变成旱田,旱田则会变成坡地,更不要说长此以往,华北平原要变成喜玛拉雅山了。 

我十几岁时又有过一段很特别的时期。我住的地方(我家在一所大学里)有些大学生为了要保卫党中央、捍卫毛主席而奋起,先是互相挥舞拳头,后用长矛交战,然后就越来越厉害。我对此事的看法不一定是正确的,但我认为,北京城原来是个很安全的地方,经这些学生的努力之后,在它的西北郊出现了一大片枪炮轰鸣的交战地带,北京地区变得带有危险性,故而这种作法能不能叫作保卫,实在值得怀疑。有一件事我始终想知道:身为二十世纪后半期的人,身披销甲上阵与人交战,白刀子进红刀子出,自我感觉如何?当然,我不认为在这辈子里还能有机会轮到我来亲身体验了。但是这些事总在我心中徘徊不去。等到我长大成人,到海外留学,还给外国同学讲起过这些事,他们或则直楞楞地看着我,或则用目光寻找台历--我知道,他们想看看那一天是不是愚人节。当然,见到这种反应,我没兴趣给他们讲这些事了。 

说到愚人节,使我想起报纸上登过的一条新闻:国外科学家用牛的基因和西红柿做了一个杂种,该杂种并不到处跑着吞吃马粪和腐植质,而是老老实实长在地上,结出硕大的果实。用这种牛西红柿做的番茄酱带有牛奶的味道,果皮还可以做鞋子。这当然是从国外刊物的愚人节专号上摘译的。像这样离奇的故事我也知道不少,比方说,用某种超声波哨子可以使冷水变热,用砖头砌的炉灶填上煤末子就可以炼出钢铁;但是这些故事不是愚人节的狂想,而是我亲眼所见。有一些时期,每一逃诩是愚人节。我在这样的气氛里长大,有一天,上级号召大家去插队、到广阔天地里,“滚一身泥巴,炼一颗红心”,我就去了,直到现在也没有认真考较一下,自己的心脏是否因此更红了一些。这当然也是个很特别的时期。消极地回顾自己的经历是不对的悲观、颓废、怀疑都是不对的。但我做的事不是这样,我正在从这些事件中寻找积极的结论,这就完全不一样了。 

在我插队不久就遇到了这样一件事,有一天,军代表把我们召集起来,声色惧厉地喝斥道:“你们这些人,口口声声要保卫毛主席,现在却是毛主席保卫了你们,还保卫了红色江山,等等。然后就向我们传达说,出了林彪事件,要我们注意盘查行人(我们在边境上)。散了会后,我有好一段时间心中不快--像每个同龄人一样,誓死保卫毛主席的口号我是喊过的。当然,军代表比我们年长,又是军人,理当在这件事上有更多的责任,这是问题的-个方面;另一方面,知青娃子实在难管,出了事先要昨唬我们一顿,这也是军代表政治经验老到之处。但是这些事已经不能安慰我了,因为我一向以为自已是个老实人,原来是这样的不堪信任--我是一个说了不算的反复小人!说了要保卫毛主席,结果却没有保卫。我对自己要求很严,起码在年轻时是这样的。经过痛苦的反思,我认为自己在这件事上是无能为力的,假如不是当初说了不负责任的话现在就可以说是清白无辜了。我说过自己正在寻找积极的结论现在就找到了一个。假设我们说话要守信义,办事情要有始有终,健全的理性实在是必不可少”。 

有关理性,哲学家有很多讨论,但根据我的切身体会,它的关键是:凡不可信的东西就不信,像我姥姥当年对待亩产三十万斤粮的态度,就叫做有理性。但这一点有时候不容易做到,因为会导致悲观和消极,从理性和乐观两样东西里选择理性颇不容易。理性就像贞操,失去了就不会再有;只要碰上了开心的事,乐观还会回来的。不过这一点很少有人注意到。从逻辑上说,从一个错误的前题什么都能推出来;从实际上看,一个扯谎的人什么都能编出来。所以假如你失去了理性,就会遇到大量令人诧异的新鲜事物,从此迷失在万花筒里,直到碰上了钉子。假如不是遇到了林彪事件,我至今还以为自己真能保卫毛主席哩。 

我保持着乐观、积极的态度,起码在插队时是这样的。直到有一天患上了重病,加上食不裹腹,病得要死。因此我就向领导要求回城养病。领导上不批准,还说我的情绪有问题。这使我猛省到,当时的情绪很是悲伤。不过我以为人生了病就该这样。旧版《水浒传》上,李逵从梁山上下去接母亲,路遇不测,老母被老虎吃了。他回到山寨,对宋江讲述了这个悲惨的故事之后,书上写着“宋江大笑。”你可以认为宋江保持了积极和乐观的态度,不过金圣叹有不同的意见,他把那句改成了“李逵大哭”。我同意金圣叹的意见,因为人遇到了不幸的事件就应该悲伤,哪有一天到晚呵呵傻笑的。当时的情形是这样的:虽然形势一片大好(这一点现在颇有疑问),但我病得要死,所以我觉得自己有理由悲伤。这个故事这样讲,显得有点突兀,应当补充些缘由:伴随着悲伤的情绪,我提出要回城去养病;领导上不批准,还让我高兴一点,“多想想大好形势”。现在想起来情况是这样:四人帮例行逆施,国民经济行将崩溃,我个人又病到奄奄一息,简直该悲伤死才好。不过我认为,当年那种程度的悲伤就够了。 

我认为,一个人快乐或悲伤,只要不是装出来的,就必有其道理。你可以去分享他的快乐,同情他的悲伤,却不可以命令他怎样怎样,因为这是违背人类的天性的。众所周知,人可以令驴和马交配,这是违背这两种动物的天性的,结果生出骡子来,但骡子没有生殖力,这说明违背天性的事不能长久。我个人的一个秘密是在需要极大快乐和悲伤的公众场合却达不到这种快乐和悲伤应有的水平,因而内心惊恐万状,汗下如雨。一九六八年国庆时,我和一批同学拥到了金水桥畔,别人欢呼雀跃,流下了幸福的眼泪,我却恨不能找个地缝钻下去。还有一点需要补充的,那就是作为一个男性,我很不容易晕厥,这更加重了我的不幸。我不知道这些话有没有积极意义,但我知道,按当年的标准,我在内心里也是好的、积极向上的,或者说,是“忠”的,否则也不会有勇气把这些事坦白出来。我至今坚信,毛主席他老人家知道了我,一个十七岁的中学生的种种心事,必定会拍拍我的脑袋说:好啦,你能做到什么样就做到什么样罢,不要勉强了。但是这样的事没有发生(恐怕主要的原因是我怕别人知道这些卑鄙的心事,把它们隐藏得很深,故而没人知道),所以我一直活得很紧张。西洋人说,人人衣柜里有一具骷髅,我的骷髅就是我自己;我从不敢想像自己当了演员,走上舞台,除非在做恶梦时。这当然不是影射什么,我只是在说自己。 

有关感情问题,我的结论如下,在这方面我们有一点适应能力。但是不可夸大这种能力,自以为想笑就能笑、想哭就能哭。假如你扣我些工资,我可以不抱怨;无缘无故打我个右派,我肯定要怀恨在心。别人在这方面比我强,我很佩服,但我不能自吹说达到了他的程度。我们不能欺骗上级,误导他们。这是老百姓应尽的义务。 

麦克阿瑟将军写过一篇祈祷文,代他的儿子向上帝讨一些品行。各种品行要了一个遍,又要求给他儿子以幽默感。假设别的东西不能保持人的乐观情绪,幽默感总能。据我所见,我们这里年轻人没有幽默感,中老年人倒有。在各种讨论会上,时常有些头顶秃光光的人,面露蒙娜丽莎式的微笑,轻飘飘地抛出几句,让大家忍俊不禁。假如我理解正确的话,这种幽默感是老奸巨猾的一种,本身带有消极的成分。不要问我这些人是谁,我不是告密者;反正不是我,我头顶不秃。我现在年登不惑,总算有了近于正常的理性;因为无病无灾,又有了幽默感,所以遇到了可信和不可信的事,都能应付自如。不过,在我年轻的时候,既没有健全的理性,又没有幽默感,那是怎么混过来的,实在是个大疑问。和同龄人交流,他们说,自己或则从众,或则听凭朴素的感情的驱动。这种状态,或者可以叫作虔诚。 

但是这样理解也有疑问。我见到过不少虔信宗教的人,人家也不干荒唐事。最主要的是:信教的人并不缺少理性,有好多大科学家都信教,而且坚信自己的灵魂能得救;人家的虔诚在理性的轨道之内,我们的虔诚则带有不少黑色幽默的成分,如此看来,问题不在于虔诚。必须指出的是,宗教是在近代才开始合理的,过去也干过烧女巫、迫害矣谒等勾当。我们知道,当年教会把布鲁诺烧死了。就算我虔信宗教,也不会同意这种行为--我本善良,我对这一点极有把握,所以肯定会去劝那些烧人的人:诸位,人家只不过是主张曰心说,烧死他太过分了。别人听了这样的话,必定要拉我同烧,这样我马上会改变劝说的方向,把它对准布鲁诺:得了吧,哥们儿,你这是何苦?去服个软儿吧。这就是我年轻时作人的态度,这当然算不上理性健全,只能叫作头脑糊涂;用这样的头脑永远也搞不清楚曰心说对不对。如果我说中国人里大多数都像我,这肯定不是个有积极意义的结论。我只是说我自己,好像很富柔韧性。因为我是柔顺的,所以领导上觉得让我怎样都成,甚至在病得要死时也能乐呵呵。这是我的错误。其实我没那么柔顺。 

我的积极结论是这样的:真理直率无比,坚硬无比,但凡有一点柔顺,也算不了真理。安徒生有一篇童话(光荣的荆棘路),就是献给这些直率、坚硬的人,不过他提到的全是外国人。作为中国的知识分子,理应有自己的榜样。此刻我脑子里浮现出一系列名字:陈寅格教授,冯友兰教授,等等。说到陈教授,我们知道,他穷毕生精力,考据了一篇很不重要的话本,《再生缘》。想到这件事,我并不感到有多振奋,只是有点伤感。

\chapter{极端体验}

唐朝有位秀才先生,才高八斗,学富五车,因慕李太白为人,自起名为李赤——我虽没见过他,但能想像出他的样子:一位翩翩佳公子。有一天,春日融融,李赤先生和几个朋友出城郊游。走到一处野外的饭馆,朋友们决定在此吃午饭。大家入席以后,李赤起身去方便。去了就不回来,大家也没理会。忽听外面一声暴喊,大家循声赶去,找到了厕所里。只见李赤先生头在下,脚在上,倒插在粪桶里。这景象够吓人的。幸亏有位上厕所的先生撞见了,惊叫了一声,迟了不堪设想……大伙赶紧把他拔出来,打来清水猛冲了几桶。还好,李赤先生还有气,冷水一激又缓了过来。别人觉得有个恶棍躲在厕所里搞鬼,把李赤拦腰抱起,栽进了粪桶里,急着要把他逮住。但李赤先生说,是自己掉进去的。于是众人大笑,说李先生太不小心了,让他更衣重新入席——但却忽略了一件事:李先生不是跳水队员,向前跳水的动作也不是非常熟练,怎么能一失足就倒插在粪桶里。所以,他是自己跳下去的。李赤的故事古书里提到了多次,《唐文粹》里有柳宗元的《李赤传》,《酉阳杂》里好像也有,都提到了李先生跳粪桶或跳茅坑,但都无法解释他为什么要跳。我忽然发现,这件事我能解释: 

有些人秉性特殊,寻常生活不能让他们满足。他们需要某种极端体验:喜欢被人捆绑起来,加以羞辱和拷打——人各有所好,这不碍我们的事。其中还有些人想要golden shower,也就是把屎尿往头上浇。这才是真正惊世骇俗的嗜好。据说在纽约和加州某些俱乐部里,有人在口袋里放块黄手绢,露出半截来,就表明自己有这种嗜好。我觉得李赤先生就有这种嗜好,只是他不是让别人往头上浇,而是自己要往里跳。这种事解释得太详细了难免恶心,我们只要明白极端体验是个什么意思就够了。 

现在是太平年月,大约在三十年前吧,整个中国乱哄哄的,有些人生活在极端体验里。这些人里有几位我认识,有些是学校里的老师,还有一些是大院里的叔叔、阿姨。他们都不喜欢这种横加在头上的极端体验,就自杀了:跳楼的跳楼,上吊的上吊,用这种方法来解脱苦难。也许有些当年闹事的人觉得这些事还满有意思的,但我劝他们替死者家属想想。死者已矣,留给亲友的却是无边的黑夜…… 

然后我就去插队,走南闯北,这种事情见得很多。比方说,在村里开会,支书总要吆喝“地富到前排”,讲几句话,就叫他们起来“撅”着。那些地富有不少比我岁数还小。原来农村的规矩是地富的子女还叫地富,就那么小一个村子,大家抬头不见低头见,撅在大伙面前,头在下腚在上,把脸都丢光,这也是种极端体验罢。当然,现在不叫地富,大家都是社员了。做出这项决定的人虽已不在人世了,但大家都会怀念他的——总而言之,那是一个极端体验的年代;虽然很惊险、很刺激,但我一点都不喜欢。不喜欢自己体验,也不喜欢看到别人体验。现在有些青年学人,人已经到了海外,拿到了博士学位和绿卡,又提起那个年代的种种好处来,借某个村庄的经验说事儿,老调重弹:想要大家再去早请示、晚汇报、学老三篇,还煞有介事地总结了毛泽东思想育新人的经验。听了这些话,我满脊梁乱起鸡皮疙瘩。 

我有些庸人的想法:吃饱了比饿着好,健康比有病好,站在粪桶外比跳进去好。但有人不同意这种想法,比方说,李赤先生。大家宴饮已毕,回城里去,走到半路,发现他不见了。赶紧回去找,发现他又倒栽进了粪桶里。这回和上回不同,拖出来一看,他已经没气了。李赤先生的极端体验就到此结束——一玩就把自己玩死,这可是太极端了,没什么普遍意义。我觉得人不该淹死在屎里,但如你所知,这是庸人之见,和李赤先生的见解不同——李赤先生死后面带幸福的微笑,只是身上臭哄哄的。 

我这个庸人又有种见解:太平年月比乱世要好。这两种时代的区别,比新鲜空气和臭屎的区别还要大。近二十年来,我们过着太平日子,好比呼吸到了一点新鲜空气,没理由再把我们栽进臭屎里。我是中国的国民,我对这个国家的希望就是:希望这里永远是太平年月。不管海外的学人怎么说我们庸俗,丧失了左派的锐气,我这个见解终不肯改。现在能太太平平,看几本书,写点小文章,我就很满意了。我可不想早请示、晚汇报,像文化革命里那样穷折腾。至于海外那几位学人,我猜他们也不是真喜欢文化革命——他们喜欢的只是那时极端体验的气氛。他们可不想在美国弄出这种气氛,那边是他们的安身立命之所。他们只想把中国搞得七颠八倒,以便放暑假时可以过来体验一番,然后再回美国去,教美国书、挣美国钱。这主意不坏,但我们不答应:我们没有极端体验的瘾,别来折腾我们。真正有这种瘾的人,何妨像李赤先生那样,自己一头扎向屎坑。

\chapter{奸近杀}

《廊桥遗梦》上演之前,有几位编辑朋友要我去看,看完给他们写点小文章。现在电影都演过去了,我还没去看。这倒不是故作清高,主要是因为围绕着《廊桥遗梦》有种争论,使我觉得很烦,结果连片子都懒得看了。有些人说,这部小说在宣扬婚外恋,应该批判。还有人说,这部小说恰恰是否定婚外恋的,所以不该批判。于是,《廊桥遗梦》就和“婚外恋”焊在一起了。我要是看了这部电影,也要对婚外恋作一评判,这是我所讨厌的事情。对于《廊桥遗梦》,我有如下基本判断:第一,这是编出来的故事,不是真的。第二,就算是真的,也是美国人的事,和我们没有关系。有些同志会说,不管和我们有没有关系,反正这电影我们看了,就要有个道德评判。这就叫我想起了近二十年前的事:当时巴黎歌剧院来北京演《茶花女》,有些观众说:这个茶花女是个妓女啊!男主角也不是什么好东西,玛格丽特和阿芒,两个凑起来,正好是一对卖淫嫖娼人员!要是小仲马在世,听了这种评价,一定要气疯。法国的歌唱家知道了这种评论,也会说:我们到这里演出,真是干了件傻事。演一场歌剧是很累的,唱来唱去,底下看见了什么?卖淫嫖娼人员!从那时到现在,已经过了十几年。我总觉得中国的观众应该有点长进——谁知还是没有长进。 

小时候,我有一位小伙伴,见了大公鸡踩蛋,就拣起石头狂追不已,我问他干什么,他说要制止鸡耍流氓。当然,鸡不结婚,搞的全是婚外恋,而且在光天化日之下做事,有伤风化;但鸡毕竟是鸡,它们的行为不足以损害我们——我就是这样劝我的小伙伴。他有另一套说法:虽然它们是鸡,但毕竟是在耍流氓。这位朋友长着鸟形的脸,鼻涕经常流过河,有点缺心眼——当然,不能因为人家缺心眼,就说他讲的话一定不对。不知为什么,傻人道德上的敏感度总是很高,也许这纯属巧合。我们要讨论的问题是:在聪明人的范围之内,道德上的敏感度是高些好,还是低些好。 

在道德方面,全然没有灵敏度肯定是不行的,这我也承认。但高到我这位朋友的程度也不行:这会闹到鸡犬不宁。他看到男女接吻就要扔石头,而且扔不准,不知道会打到谁,因此在电影院里成为一种公害。他把石头往银幕上扔,对看电影的人很有点威胁。人家知道他有这种毛病,放电影时不让他进;但是石头还会从墙外飞来。你冲出去抓住他,他就发出一阵傻笑。这个例子说明,太古板的人没法欣赏文艺作品,他能干的事只是扰乱别人…… 

我既不赞成婚外恋,也不赞成卖淫嫖娼,但对这种事情的关切程度总该有个限度,不要闹得和七十年代初抓阶级斗争那样的疯狂。我们国家五千年的文明史,有一条主线,那就是反婚外恋、反通奸,还反对一切男女关系,不管它正当不正当。这是很好的文化传统,但有时也搞得过于疯狂,宋明理学就是例子。理学盛行时,科学不研究、艺术不发展,一门心思都在端正男女关系上,肯定没什么好结果。中国传统的士人,除了有点文化之外,品行和偏僻小山村里二十岁守寡的尖刻老太婆也差不多。我从清朝笔记小说中看到一则纪事,比《廊桥遗梦》短,但也颇有意思。这故事是说,有一位才子,在自己的后花园里散步,走到篱笆边,看到一对蚂蚱在交尾。要是我碰上这种事,连看都不看,因为我小时候见得太多了。但才子很少走出书房,就停下来饶有兴致地观看。忽然从草丛里跳出一个花里胡哨的癞蛤*?,一口把两个蚂蚱都吃了,才子大惊失色,如梦方醒……这故事到这里就完了。有意思的是作者就此事发了一通感慨,大家可以猜猜他感慨了些什么…… 

坦白地说,我看书看到这里,掩卷沉思,想要猜出作者要感慨些啥。我在这方面比较鲁钝,什么都没猜出来。但是从《廊桥遗梦》里看到了婚外恋的同志、觉得它应该批判的同志比我要能,多半会猜到:蚂蚱在搞婚外恋,死了活该。这就和谜底相当接近了。作者的感慨是:“奸近杀”啊。由此可以重新解释这个故事:这两只蚂蚱在篱笆底下偷情,是两个堕落分子。而那只黄里透绿,肥硕无比的癞蛤*?,却是个道德上的义士,看到这桩奸情,就跳过来给他们一点惩诫——把他们吃了。寓意是好的,但有点太过离奇:癞蛤蟆吃蚂蚱,都扯到男女关系上去,未免有点牵强。我总怀疑那只蛤*?真有这么高尚。它顶多会想:今天真得蜜,一嘴就吃到了两个蚂蚱!至于看到人家交尾,就义愤填膺,扑过去给以惩诫——它不会这么没气量。这是因为,蚂蚱不交尾,就没有小蚂蚱;没有小蚂蚱,癞蛤蟆就会饿死。

\chapter{京片子与民族自尊心}

我生在北京西郊大学区里。长大以后,到美国留学,想要恭维港台来的同学,就说:你国语讲得不坏!他们也很识趣,马上恭维回来:不能和你比呀。北京乃是文化古都,历朝历代人文荟萃,语音也是所有中国话里最高尚的一种,海外华人佩服之至。我曾在美国华文报纸上读到一篇华裔教授的大陆游记,说到他遭服务小姐数落的情形:只听得一串京片子,又急又快,字字清楚,就想起了《老残游记》里大明湖上黑妞说书,不禁目瞪口呆,连人家说什么都没有去想——我们北京人的语音就有如此的魅力。当然,教授愣完了,开始想那些话,就臊得老脸通红。过去,我们北京的某些小姐(尤其是售票员)在粗话的词汇量方面,确实不亚于门头沟的老矿工——这不要紧,语音还是我们高贵。 

但是,这已是昨日黄花。今天你打开收音机或者电视机,就会听到一串“嗯嗯啊啊”的港台腔调。港台人把国语讲成这样也会害臊,大陆的广播员却不知道害臊。有一句鬼话,叫作“那么呢”,那么来那么去,显得很低智,但人人都说。我不知这是从哪儿学来的,但觉得该算到港台的帐上。再发展下去,就要学台湾小朋友,说出“好可爱好高兴噢”这样的鬼话。台湾人造的新词新话,和他们的口音有关。国语口音纯正的人学起来很难听。 

除了广播员,说话港台化最为厉害的,当数一些女歌星。李敖先生骂老K(国民党),说他们“手淫台湾,意淫大陆”,这个比方太过粗俗,但很有表现力。我们的一些时髦小姐糟塌自己的语音,肯定是在意淫港币和新台币——这两个地方除了货币,再没什么格外让人动心的东西。港台人说国语,经常一顿一顿,你知道是为什么吗?他们在想这话汉语该怎么说啊。他们英语讲得太多,常把中国话忘了,所以是可以原谅的。我的亲侄子在美国上小学,回来讲汉语就犯这毛病。犯了我就打他屁股,打一下就好。中国的歌星又不讲英文,再犯这种毛病,显得活像是大头傻子。电台请歌星做节目,播音室里该预备几个乒乓球拍子。乒乒球拍子不管用,就用擀面杖。这样一级一级往上升,我估计用不到狼牙棒,就能把这种病治好。治好了广播员,治好了歌星,就可以治其它小姐的病。如今在饭店里,听见鼻腔里哼出一句港味的“先生”,我就起鸡皮疙瘩。北京的女孩子,干嘛要用鼻腔来说话! 

这篇文章一直在谈语音语调,但语音又不是我真正关心的问题。我关心的是,港台文化正在侵入内地。尤其是那些狗屎不如的电视连续剧,正在电视台上一集集地演着,演得中国人连中国话都说不好了。香港和台湾的确是富裕,但没有文化。咱们这里看上去没啥,但人家还是仰慕的。所谓文化,乃是历朝历代的积累。你把城墙拆了,把四合院扒了,它还在人身上保留着。除了语音,还有别的——就拿笔者来说,不过普普通通一个北方人,稍稍有点急公好义,仗义疏财,有那么一丁点燕赵古风,台湾来的教授见了就说:你们大陆同学,气概了不得…… 

我在海外的报刊上看到这样一则故事:有个前国军上校,和我们打了多年的内战;枪林弹雨都没把他打死。这一方面说明我们的火力还不够厉害,另一方面也说明这个老东西确实有两下子。改革开放之初,他巴巴地从美国跑了回来,在北京的饭店里被小姐骂了一顿,一口气上不来,脑子里崩了血筋,当场毙命。就是这样可怕的故事也挡不住他们回来,他们还觉得被正庄京片子给骂死,也算是死得其所。我认识几位华裔教授,常回大陆,再回到美利坚,说起大陆服务态度之坏,就扼腕叹息道:再也不回去了。隔了半年,又见他打点行装。问起来时,他却说:骂人的京片子也是很好听的呀!他们还说:骂人的小姐虽然粗鲁,人却不坏,既诚实又正直,不会看人下菜碟,专拍有钱人马屁——这倒不是谬奖。八十年代初的北京小姐,就是洛克菲勒冒犯到她,也是照骂不误:“别以为有几个臭钱就能在我这儿起腻,惹急了我他妈的拿大嘴巴子贴你!”断断不会见了港客就骨髓发酥非要嫁他不可——除非是领导上交待了任务,要把他争取过来。粗鲁虽然不好,民族自尊心却是好的,小姐遇上起腻者,用大嘴巴子去“贴”他,也算合理;总比用脸去贴好罢。这些事说起来也有十几年了。如今北京多了很多合资饭店,里面的小姐不骂人,这几位教授却不来了。我估计是听说这里满街的鸟语,觉着回来没意思。他们不来也不要紧,但我们总该留点东西,好让别人仰慕啊。

\chapter{打工经历}

在美留学时,我打过各种零工。其中有一回,我和上海来的老曹去给家中国餐馆装修房子。这家餐馆的老板是个上海人,尖嘴猴腮,吝啬得不得了;给人家当了半辈子的大厨,攒了点钱,自己要开店,又有点烧得慌——这副嘴脸实在是难看,用老曹的话来说,是一副赤佬像。上工第一天,他就对我们说:我请你们俩,就是要省钱,否则不如请老美。这工程要按我的意思来干。要用什么工具、材料,向我提出来,我去买。别想揩我的油…… 

以前,我知道美国的科技发达、商业也发达,但我还不知道,美国还是各种手艺人的国家。我们打工的那条街上就有一大窝,什么电工、管子工、木工等等,还有包揽装修工程的小包工头儿;一听见我们开了工,就都跑来看。先看我们抡大锤、打钎子,面露微笑,然后就跑到后面去找老板,说:你请的这两个宝贝要是在本世纪内能把这餐馆装修完,我输你一百块钱。我脸上着实挂不住,真想扔了钎子不干。但老曹从牙缝里啐口吐沫说:不理他!这个世纪干不完,还有下个世纪,反正赤佬要给我们工钱…… 

俗话说,没有金刚钻,别揽磁器活。要是不懂怎么装修房子就去揽这个活,那是我们的错。我虽是不懂,但有一把力气,干个小工还是够格的。人家老曹原是沪东船厂的,是从铜作工提拔起来的工程师,专门装修船舱的,装修个餐馆还不知道怎么干吗……他总说,现在的当务之急是买工具、租工具,但那赤佬老板总说,别想揩油。与其被人疑为贪小便宜,还不如闷头干活,赚点工钱算了。 

等把地面打掉以后,我们在这条街上赢得了一定程度的尊敬。顺便说一句,打下来的水泥块是我一块块抱出去,扔到垃圾箱里,老板连个手推车都舍不得租。他觉得已经出了人工钱,再租工具就是吃了亏。那些美国的工匠路过时,总来聊聊天,对我们的苦干精神深表钦佩。但是他们说,活可不是你们俩这种干法。说实在的,他们都想揽这个装修工程,只是价钱谈不拢。下一步是把旧有的隔断墙拆了。我觉得这很简单,挥起大锤就砸——才砸了一下,就被老板喝止。他说这会把墙里的木料砸坏。隔断墙里能有什么木料,不过是些零零碎碎的破烂木头。但老板说,要用它来造地板。于是,我们就一根根把这些烂木头上的钉子起出来。美国人见了问我们在干什么,我如实一说,对方捂住肚子往地下一蹲,笑得就地打起滚来。这回连老曹脸上都挂不住了,直怪我太多嘴…… 

起完了钉子,又买了几块新木料,老板要试试我们的木匠手艺,让我们先造个门。老曹就用锯子下起料来:我怎么看,怎么觉得这锯子不像那么回事儿,锯起木头来直拐弯儿。它和我以前见过的锯子怎么就那么不一样呢。正在干活,来了一个美国木匠。他笑着问我们原来是干啥的。我出国前是个大学教师,但这不能说,不能丢学校的脸。老曹的来路更不能说,说了是给沪东船厂丢脸。我说:我们是艺术家。这话不全是扯谎。我出国前就发表过小说,至于老曹,颇擅丹青,作品还参加过上海工人画展……那老美说:我早就知道你们是艺术家!我暗自得意:我们身上的艺术气质是如此浓郁,人家一眼就看出来了。谁知他又补充了一句,工人没有像你们这么干活的!等这老美一走,老曹就扔下了锯子,破口大骂起来。原来这锯子的正确用途,是在花园里锯锯树杈…… 

我们给赤佬老板干了一个多月,也赚了他几百块钱的工钱,那个餐馆还是不像餐馆,也不像是冷库,而是像个破烂摊。转眼间夏去秋来,我们也该回去上学了。那老板的脸色越来越难看,天天催我们加班。催也没有用,手里拿着手锤铁棍,拼了命也是干不出活来的。那条街上的美国工匠也嗅出味来了,全聚在我们门前,一面看我们俩出洋像,一面等赤佬老板把工程交给他们。在这种情况下,连老曹也绷不住,终于和我一起辞活不干了。于是,这工程就像熟透的桃子一样,掉进了美国师傅的怀里。本来,辞了活以后就该走掉。但老曹还要看看美国人是怎么干活的。他说,这个工程干得窝囊,但不是他的过错,全怪那赤佬满肚子馊主意。要是由着他的意思来干,就能让洋鬼子看看中国人是怎么干活的…… 

美国包工头接下了这个工程,马上把它分了出去,分给电工、木工、管子工,今天上午是你的,下午是他的,后天是我的,等等。几个电话打出去,就有人来送工具,满满当当一卡车。这些工具不要说我,连老曹都没见过。除了电锯电刨,居然还有用电瓶的铲车,可以在室内开动,三下五除二,就把我们留下的破烂从室内推了出去。电工上了电动升降台,在天花板上下电线,底下木工就在装配地板,手法纯熟之极。虽然是用现成的构件,也得承认人家干活真是太快了。装好以后电刨子一跑,贼亮;干完了马上走人,运走机械,新的工人和机械马上开进来……转眼之间,饭馆就有个样儿了……我和老曹看了一会儿,就灰溜溜地走开了。这是因为我们都当过工人,知道怎么工作才有尊严。

\chapter{我的精神家园}

  我十三岁时,常到我爸爸的书柜里偷书看。那时候政治气氛紧张,他把所有不宜摆在外面的书都锁了起来,在那个柜子里,有奥维德的变形记,朱生豪译的莎翁戏剧,甚至还有十日谈。柜子是锁着的,但我哥哥有捅开它的方法。他还有说服我去火中取栗的办法:你小,身体也单薄,我看爸爸不好意思揍你。但实际上,在揍我这个问题上,我爸爸显得不够绅士派,我的手脚也不太灵活,总给他这种机会。总而言之,偷出书来两人看,挨揍则是我一人挨,就这样看了一些书。虽然很吃亏,但我也不后悔。 

  看过了变形记,我对古希腊着了迷。我哥哥还告诉我说:古希腊有一种哲人,穿着宽松的袍子走来走去。有一天,有一位哲人去看朋友,见他不在,就要过一块涂蜡的木板,在上面随意挥洒,画了一条曲线,交给朋友的家人,自己回家去了。 

  那位朋友回家,看到那块木板,为曲线的优美所折服;连忙埋伏在哲人家左近,待他出门时闯进去,要过一块木板,精心画上一条曲线……当然,这故事下余的部分就很容易猜了:哲人回了家,看到朋友留下的木板,又取一块蜡板,把自己的全部心胸画在一条曲线里,送给朋友去看,使他真正折服。现在我想,这个故事是我哥哥编的。但当时我还认真地想了一阵,终于傻呵呵地说道:这多好啊。时隔三十年回想起来,我并不羞愧。井底之蛙也拥有一片天空,十三岁的孩子也可以有一片精神家园。此外,人有兄长是好的。虽然我对国家的计划生育政策也无异议。 

  长大以后,我才知道科学和艺术是怎样的事业。我哥哥后来是已故逻辑大师沈有鼎先生的弟子,我则学了理科;还在一起讲过真伪之分的心得、对热力学的体会;但这已是我二十多岁时的事。再大一些,我到国外去旅行,在剑桥看到过使牛顿体会到万有引力的苹果树,拜伦拐着腿跳下去游水的“拜伦塘”,但我总在回想幼时遥望人类智慧星空时的情景。千万丈的大厦总要有片奠基石,最初的爱好无可替代。所有的智者、诗人,也许都体验过儿童对着星光感悟的一瞬。我总觉得,这种爱好对一个人来说,就如性爱一样,是不可少的。 

  我时常回到童年,用一片童心来思考问题,很多烦难的问题就变得易解。人活着当然要做一番事业,而且是人文的事业;就如有一条路要走。假如是有位老学究式的人物,手执教鞭戒尺打着你走,那就不是走一条路,而是背一本宗谱。我听说前苏联就是这么教小孩子的:要背全本的普希金、半本莱蒙托夫,还要记住俄罗斯是大象的故乡(萧斯塔科维奇在回忆录里说了很多)。我们这里是怎样教孩子的,我就不说了,以免得罪师长。我很怀疑会背宗谱就算有了精神家园,但我也不想说服谁。安徒生写过光荣的荆棘路,他说人文的事业就是一片着火的荆棘,智者仁人就在火里走着。当然,他是把尘世的嚣嚣都考虑在内了,我觉得用不着想那么多。用宁静的童心来看,这条路是这样的:它在两条竹篱笆之中。篱笆上开满了紫色的牵牛花,在每个花蕊上,都落了一只蓝蜻蜓。这样说固然有煽情之嫌,但想要说服安徒生,就要用这样的语言。维特根斯坦临终时说:告诉他们,我度过了美好的一生。这句话给人的感觉就是:他从牵牛花丛中走过来了。虽然我对他的事业一窍不通,但我觉得他和我是一头儿的。 

  我不大能领会下列说法的深奥之处:要重建精神家园、恢复人文精神,就要灭掉一切俗人——其中首先要灭的,就是风头正健的俗人。假如说,读者兜里的钱是有数的,买了别人的书,就没钱来买我的书,所以要灭掉别人,这个我倒能理解,但上述说法不见得有如此之深奥。假如真有这么深奥,我也不赞成——我们应该像商人一样,严守诚实原则,反对不正当的竞争。让我的想法和作品成为嚣嚣尘世上的正宗,这个念头我没有,也不敢有。既然如此,就必须解释我写文章(包括这篇文章)的动机。坦白地说,我也解释不大清楚,只能说:假如我今天死掉,恐怕就不能像维特根斯坦一样说道:我度过了美好的一生;也不能像斯汤达一样说:活过,爱过,写过。我很怕落到什么都说不出的结果,所以正在努力工作。

\chapter{卖唱的人们}

有一次,我在早上八点半钟走过北京的西单北大街,这个时间商店都没有开门,所以人行道上空空荡荡,只有满街飞扬的冰棍纸和卖唱的盲人。他们用半导体录音机伴奏,唱着民歌。我到过欧美很多地方,常见到各种残疾人乞讨或卖唱,都不觉得难过,就是看不得盲人卖唱。这是因为盲人是最值得同情的残疾人,让他们乞讨是社会的耻辱。再说,我在北京见到的这些盲人身上都很脏,歌唱得也过于悲惨;凡事他们唱过得歌我再也不想听到。当时满街都是这样的盲人,就我一个明眼人,我觉得这种景象有点过分。我见过各种各样的卖唱者,就属那天早上看到的最让人伤心。我想,最好有个盲人之家,把他们照顾起来,经常洗洗澡,换换衣服,再有辆面包车接送他们各处卖唱,免得都挤在西单北大街--但是最好别卖唱。很多盲人有音乐天赋,可以好好学一学,做职业艺术家。美国就有不少盲人音乐家,其中有几个还很有名。 

本文的宗旨不是谈如何关怀盲人,而是谈论卖唱--当然,这里说的卖唱是广义的,演奏乐器也在内。我见过各种卖唱者,其中最怪异的一个是在伦敦塔边上看到的。这家伙有五十岁左右,体壮如牛,头戴一顶猎帽,上面插了五彩的鸵鸟毛,这样他的头就有点像儿童玩的羽毛球;身上穿了一件麂皮茄克,满是污渍,但比西单的那些盲人干净--那些人身上没有污渍,整个人油亮油亮的--手里弹着电吉他,嘴上用铁架子支了一只口琴,脚踩着一面踏板鼓,膝盖栓有两面钹,靴子跟上、两肘栓满了铃,其他地方可能也藏有一些零碎,因为从声音听来,不止我说的这些。他在演奏时,往好听里说,是整整一支军乐队,往难听里说,是一个修理黑白铁的工场。演奏着一些俗不可耐的曲子。初看时不讨厌,看过一分钟,就得丢下点零钱溜走,否则就会头晕,因为他太吵人。我不喜欢他,因为他是个哗众取宠得家伙。他的演奏没有艺术,就是要钱。 

据我所见,卖唱不一定非把身上弄得很脏,也不一定要要哗众取宠。比方说,有一次我在洛杉矶乘地铁,从车站出来,走过一个很大的过厅。这里环境很优雅,铺着红地毯,厅中央放了一架钢琴。有一个穿黑色燕尾服的青年坐在钢琴后面,琴上放了一杯冰水。有人走过时,他并不多看你,只弹奏一曲,就如向你表示好意。假如你想回报他的好意,那是你的事。无心回报时,就带着这好意走开。我记得我走过时,他弹奏的是“八音盒舞曲”,异常悠扬。时隔十年,我还记得那乐曲,和他的样子,他非常年轻。人在年轻时,可能要做些服务性的工作,糊口或攒学费,等待进取的时机,在公共场所演奏也是一种。这不要紧只要无损于尊严就可。我相信,这个青年一定会有很好的前途。 

下面我要谈的是我所见过的最动人的街头演奏,这个例子说明在街头和公共场所演奏,不一定会有损个人尊严,也不一定会使艺术蒙羞--只可惜这几个演奏者不是真为钱而演奏。一个夏末的星期天,我在维也纳,阳光灿烂,城里空空荡荡,正好欣赏这座伟大的城市。维也纳是奥匈帝国的首都,帝国已不复存在,但首都还是首都。到过那座城市的人会同意,“伟大”二字决非过誉。在那个与莫扎特等伟大名字联系在一起的歌剧院附近,我遇上三个人在街头演奏。不管谁在这里演奏,都显得有点不知寒碜。只有这三个人例外。拉小提琴的是个金发小伙子,穿件毛衣、一条宽松的裤子,简朴但异常整洁。他似是这三个人的头头,虽然专注于演奏,但也常看看同伴,给他们无声的鼓励。有一位金发姑娘在吹奏长笛,她穿一套花呢套裙,眼睛里有点笑意。还有一个东亚女孩坐着拉大提琴,乌黑的齐耳短发下一张白净的娃娃脸,穿着短短的裙子,白袜子和学生穿的黑皮鞋;她有点慌张,不敢看人,只敢看乐谱。三个人都不到二十岁,全都漂亮之极。至于他们的音乐,就如童声一样,是一种天籁。这世界上没有哪个音乐家会说他们演奏得不好。我猜这个故事会是这样的:他们三个是音乐学院的同学,头一天晚上,男孩说:敢不敢到歌剧院门前去演奏?金发女孩说:敢!有什么不敢的!至于那东亚女孩,我觉得她是我们的同胞。她有点害羞,答应了又反悔,反悔了又答应,最后终于被他们拉来了。除了我们之外,还有十几个人在听,但都远远地站着,恐怕会打扰他们。有时会有个老太太走近去放下一些钱,但他们看都不看,沉浸在音乐里。我坚信,这一幕是当日维也纳最美丽的风景。我看了以后有点嫉妒,因为他们太年轻了。青年的动人之处,就在于勇气,和他们的远大前程。

\chapter{我为什么要写作}

有人问一位登山家为什么要去登山——谁都知道登山这件事既危险,又没什么实际的好 处,他回答道:“因为那座山峰在那里。”我喜欢这个答案,因为里面包含着幽默感——明明是自己想要登山,偏说是山在那里使他心里痒痒。除此之外,我还喜欢这位登山家干的事,没来由地往悬崖上爬。它会导致肌肉疼痛,还要冒摔出脑子的危险,所以一般人尽量避免爬山。用热力学的角度来看,这是个反熵的现象,所发趋害避利肯定反熵。 

现在把登山和写作相提并论,势必要招致反对。这是因为最近十年来中国有过小说热、 诗歌热、文化热,无论哪一种热都会导致大量的人投身写作,别人常把我看成此类人士中的一个,并且告诫我说,现在都是什么年月了,你还写小说(言下之意是眼下是经商热,我该下海去经商了)?但是我的情形不一样。前三种热发生时,我正在美国念书,丝毫没有受到感染。我们家的家训是不准孩子学文科,一律去学理工。因为这些缘故,立志写作在我身上是个不折不扣的反熵过程。我到现在也弄不明白自己为什么要干这件事,除了它是个反熵过程这一点。 

有关我立志写作是个反熵过程,还有进一步解释的必要。写作是个笼统的字眼,还要看写什么东西。写畅销小说、爱情小诗等等热门东西,应该列入熵增过程之列。我写的东西一点不热门,不但挣不了钱,有时还要倒贴一些。严肃作家的“严肃”二字,就该做如此理解。据我所知,这世界上有名的严肃作家,大多是凑合也算不上。这样说明了以后,大家都能明白我确实在一个反熵过程中。 

我父亲不让我们学文科,理由显而易见。在我们成长的时代里,老舍跳了太平湖,胡风关了监狱,王实味被枪毙了。以前还有金圣叹砍脑壳等等实例。当然,他老人家也是屋内饮酒,门外劝水的人,自己也是个文科的教授,但是他坦白地承认自己择术不正,不足为训。我们兄弟姐妹五个就范此全学了理工科,只我哥哥例外。考虑到我父母脾气暴躁、吼声如雷,你得说这种选择是个熵增过程。而我哥哥那个例外是这么发生的:七八年考大学时,我哥哥是北京木城漳煤矿最强壮的青年矿工,吼起来比我爸爸音量还要大。无论是动手揍他,还是朝他吼叫,我爸爸自己都挺不好意思,所以就任凭他去学了哲学:在罗辑学界的泰斗沈有鼎先生的门下当了研究生。考虑到符号逻辑是个极专门的学科(这是从外行人看不懂得逻 辑文章来说),它和理工科差不太多的。从以上的叙述,你可以弄明白我父亲的意思。他希望我们每个人都学一种外行人弄不懂而又是有功世道的专业,平平安安地度过一生。我父亲一生坎坷,他又最爱我们,这样的安排在他看来最自然不过。 

我自己的情形是这样的:从小到大,身体不算强壮,吼起来音量也不够大,所以一直本分为人。尽管如此,我身上总有一股要写小说的危险情绪。插队的时候,我遇上一个很坏的家伙(他还是我们的领导,属于在我国这个社会里少数坏干部之列),我就编了一个故事,描写他从尾骨开始一寸寸变成了一条驴,并且把它写出来,以泄心头之愤。后来读了一些书,我发现卡夫卡也写了个类似的故事,搞得我很不好意思。还有一个故事,女主人公长了蝙蝠的翅膀,并且头发是绿色的,生活在水下。这些二十岁前的作品我都烧掉了。在此一提是要说明这种危险倾向的由来。后来我一直抑制着这种倾向,念完了本科,到美国去留学。我哥哥也念完了硕士,也到美国去留学。我在那边又开始写小说,这种危险的倾向再也不能抑制了。 

在美国时,我父亲去世了。回想他让我们读理科的事,觉得和美国发生的事不是一个逻辑。这让我想起了前苏联元帅图哈切夫斯基对大音乐家萧斯塔科奇说的话来:“我小的时候,很有音乐天才。只可惜我父亲没钱给我买把小提琴!假如有了那把小提琴,我现在就坐在你的乐池里。”这段话乍看不明其意,需要我提示一句:这次对话发生在苏联的三十年代, 说完了没多久,图元帅就一命呜呼。那年头专毙元帅将军,不大毙小提琴手。文化革命里跳楼上吊的却是文人居多。我父亲在世时,一心一意地要给我们每人都弄把小提琴。这把小提琴就是理工农医任一门,只有文科不在其内,这和美国发生的事不一样,但是结论还是同一个——我该去干点别的,不该写小说。 

有关美国的一切,可以用一句话来描述:“Americans business is business”,这句话 的意思就是说,那个国家永远是在经商热中,而且永远是一千度的白热。所以你要是看了前文之后以为那里有某种气氛会有助于人立志写作就错了。连我哥哥到了那里都后悔了,觉得不该学逻辑,应当学商科或者计算机。虽然他依旧未证出的逻辑定理,但是看到有钱人豪华的住房,也免不了唠叨几句他对妻儿的责任。 

在美国有很强大的力是促使人去挣钱,比方说洋房,有些只有一片小草坪,有的有几百亩草坪,有的有几千亩草坪,所以仅就住房一项,就能产生无穷无尽的挣钱的动力。再比方说汽车,有无穷的档次和价格。你要是真有钱,可以考虑把肯尼迪遇刺时坐的汽车买来坐。 还有人买下了前苏联的战斗机,驾着飞上天。在那个社会里,没有人受得了自己的孩子对同 伴说:我爸爸穷。我要是有孩子,现在也准在那里挣钱。而写书在那里也不是个挣钱的行当,不信你到美国书店里看看,各种各样的书涨了架子,和超级市场里陈列的卫生纸一样多 ——假如有人出售苦心积虑一页页写出的卫生纸,肯定不是好行当。除此之外,还有好多人的书没有上架,窝在他自己的家里。我没有孩子,也不准备要。作为中国人,我是个极少见的现象。但是人有一张脸,树有一张皮,别人都有钱挣,自己却在干可疑的勾当,脸面上也过不去。 

在美国时,有一次和一位华人教授聊天,他说他女儿很有出息,放着哈佛大学人类学系奖学金不要,自费去念一般的大学的lawschool如此反潮流,真不愧是书香门第。其实这是舍小利而趋大利,受小害而避大害。不信你去问问律师挣多少钱,人类学家又挣多少钱。和我聊天的这位教授是个大学问家,特立独行之辈。一谈到了儿女,好像也不大特立独行了。 

说完了美国、苏联,就该谈谈自己。到现在为止,我写了八年小说,也出了几本书,但是大家没怎么看到。除此之外,我还常收到谩骂性的退稿信,这时我总善意地想:写信的人 准是领导那里挨了骂,找我撒气。提起王小波,大家准会想到宋朝的四川拉杆子的那一位, 想不起我身上。我还在反熵过程中。顺便说一句,人类的存在,文明的发展就是个反熵过 程,但是这是说人类。具体说到自己,我的行为依旧无法解释。再顺便说一句,处于反熵过程中,绝不只是我一个人。在美国,我遇上过支起摊来卖托洛斯基、格瓦拉、毛主席等人的书的家伙,我要和他说话,他先问我怕不怕联帮调查局——别的例子还很多。在这些人身上,你就看不到水往低处流、苹果掉下地,狼把兔子吃掉的宏大的过程,看到的现象,相当 于水往山上流,苹果飞上天,兔子吃掉狼。我还可以说,光有熵增现象不成。举例言之,大家都顺着一个自然的方向往下溜,最后准会在个低洼的地方汇齐,挤在一起像粪缸里的蛆。 但是这也不能解释我的行为。我的行为是不能解释的,假如你把熵增现象看成金科玉律的话。 

当然,如果硬要我用一句话直截了当地回答这个问题,那就是:我相信我自己有文学才能,我应该做这件事。但是这句话正如一个嫌疑犯说自己没杀人一样不可信。所以信不信由你罢。

%% Last updated: 2008|05|17  00:53:16 by Van Tae

A person asks a mountain climber why he climbs mountains. Everyone knows mountain climbing is dangerous. There's no practical benefit. The mountain climber answers, " Because the mountain is there." I like this answer because it contains a kind of humor: he clearly wants to climb mountains, but he says that the mountain is there to make the other guy's heart itch. Beside this, I also like what the mountain climber does. He climbs dangerous cliffs for no apparent reason. It makes for muscle aches, even has the risk of losing one's head. Therefore ordinary people do all they can to avoid climbing mountains. From the perspective of thermodynamics, it's a phenomenon that goes against entropy. Whatever seeks out harm and avoids rewards must act against entropy. 

Now let us compare mountain climbing to writing, which is bound to invite objections from some. This is because in the recent decade, China had a novel boom, a poetry boom, and a culture boom. No matter which boom, each has led many people to throw themselves into writing. Others often see me as one of these people and warn me, "What era are you living in? Why are you writing novels (implying there's a business boom going on. I should get down and go into business)? " But my situation is not like this. When those three booms occured, I was studying in the United States and wasn't at all influenced. The instruction in our house was not to allow the kids to study the humanities, but to encourage all to study engineering and the sciences. Because of this, determining to write has been an anti-entropic process for me. Even now I don't understand why I do it, except for the fact that it is against entropy. 

I need to explain further how my resolution write is against entropy. Writing is a general word. We must also look at what one is writing. Writing best-sellers, romance novels, and other popular items should be considered as following entropy [and quite natural]. My writings are not at all popular. Not only do they not make me any money, sometimes they even cost me money. This is what the "serious" of "serious writer" means. From what I know, most serious writers in the world can't make ends meet even when they improvise. When explained like this, everyone realizes that I am really in a process that is against entropy. 

My dad didn't allow us to study humanities in school. The reason was obvious. When we were growing up, Laose (the famous novelist) jumped in Lake Taiping, Hu Feng (the literary critic) was imprisoned, and Wang Shiwei (the writer of political tracts and translator of Marxist writings) was sentenced to death and shot. Before them, there was Jin Shengtan (the famous Qing dynasty scholar and critic) whose head was chopped off and other examples. Of course, my old man was also the kind to urge liquor in private while telling you publicly to drink water. He himself was a professor in the humanities, but he honestly acknowledged that his chosen profession was not right and shouldn't be taken as an example. Thus, among us children, five gave in and studied engineering and science, my older brother being the only exception. Considering my parents' quick temper and loud roars, you can say that such a choice was a pro-entropic process. But my older brother's exception happened like this: when taking the university entrance exam in 1978, he was the strongest young miner at Beijing's Mu Cheng Zhang coal mine. His roar was even louder than my dad's. My dad felt rather embarrassed hitting him or shouting at him, and so he allowed him to study philosophy: to become a graduate student studying with Shen Youding, an authority in the field of logic. Considering that symbolic logic is an extremely specialized field (a layman would not understand an article on logic), it was not that different from engineering or science. From what I have said, you can understand my dad's desire. He wished that each of us can study a specialty that laymen couldn't understand but which could still make a contribution to the wolrd, so that we can each live our lives in peace. My dad had a rough life and loved us above all things. He saw this kind of arrangement as the most natural. 

My own situation was this. From childhood to adulthood, I've never been that physically strong. I couldn't shout loud enough. Thus I was always content with my lot. Despite this, I always felt the dangerous wish to write novels. When I was sent to work in the coutryside [during the Cultural Revolution], I ran across an atrocious guy (he was even our boss, one of the few bad leaders in our country). So I made up a story, describing how he was transformed into a donkey little by little, starting with his tailbone, and I wrote it down to vent some anger. Later I read some books and found that Kafka also wrote a similar story, which made me a little embarrassed. There was another story in which the female protagonist had the wings of a bat, had green hair, and lived under water. I burned all these works from twenty years ago. I mention them here to explain the source of my dangerous inclination. Later I always suppressed this inclination, finishing my undergraduate studies and going to the U.S. for graduate work. My older brother also obtained his Master's degree and went to the U.S. for graduate studies. I started writing novels again there. This dangerous inclination could no longer be suppressed. 

My dad passed away while I was in the U.S. Reflecting on his request for us to study engineering and science, I felt it didn't fit with the logic of my experiences in the U.S. It reminded me of what former Marshal Tukhackevsky of the Soviet Union said to the great composer Shostakovich: "When I was little, I had a lot of musical talent. Unfortunately, my dad didn't have the money to buy me a violin! If I had that violin, I could be sitting in your orchestra now." This statement doesn't make sense at first and needs some explanation. It was made in the Soviet Union of the 1930s. Not long after making the statement, Marshal Tukhachevsky was executed. In those years, one specialized in executing marshals and general, very rarely executing violinists. But during the Cultural Revolution, there more writers and artists who committed suicide by jumping from buildings or hanging themselves. When my dad was still alive, he wholeheartedly wanted to give each of us a violin. This violin was choosing science, engineering, agriculture, or medicine. It didn't matter which one, only not the humanities. This wasn't the same as my experiences the U.S., but the conclusion was the same -- I should do something else and not write novels. 

One sentence can describe everything in America: "American's business is business." What this statement means is that business will always be hot in this country. Moreover, it will always be a one-thousand-degree white heat. Thus, you would be mistaken if you read the foregoing and thought that the atmosphere there would help one resolve to write. Even my older brother had regrets after arriving there and felt that he shouldn't have studied logic but instead should have studied business or computers. Even though he still wasn't able to prove it by logical theorems, when he saw the luxurious houses of the rich, he couldn't help but say a few words about his responsibility towards his wife and kids. 

There is a strong force in America compelling people to make money. Let's say a house: some people have several hundred acres of lawn; some have several thousand. Therefore if one only had a place to live, one could develop the endless urge to make money. Or let's say a car: there are endless categories and prices. If you really had money, you could consider purchasing the car Kennedy was riding when he was assassinated to ride around. There are also people buying former Soviet fighter planes, piloting them around in the air. In that society, those who have no money have to suffer their kids telling their friends: "My dad is poor." If I had children, I would definitely still be there, making money. But writing is not a money-making profession there, either. If you don't believe me, you can go see for yourself in American bookstores. All kinds of books overflow the shelves, as many as the rows of toilet paper stocked at supermarkets. If someone is selling toilet paper, painstakingly written page by page, it can't be a good profession. Beside this, there remain many people whose books have yet to appear on the shelves, still nesting back at home. I don't have children and am not prepared to have any. As a Chinese person, I am a rare phenomenon. However, a person has a face, a tree has a bark. Others are all making money, but I am involved in this dubious business. I really wouldn't have a face to show my kids. 

When I was in America, I was chatting with a Chinese professor once. He said his daughter was very successful. But she gave up a scholarship to study in a humanities department at Harvard to pay her own way to study at an average law school. He said going against the tide like this was certainly not worthy of a scholar family. In reality, this was giving up a small reward to seek a big one, suffering a small harm to avoid a big one. If you don't believe me, go ask how much money lawyers make and how much humanities scholars make. This professor I chatted with was very learned, very independent-minded. But when talking about his children, he seemed not very independent. 

Having talked about America and the Soviet Union, I should say something about myself. At this point, I have written novels for eight years and have published a few books, but not too many have read them. Beside this, I also often receive jeering rejection letters for my manuscripts. When this happens, I always think benevolently: the person writing the letter must have gotten chewed out by his boss and using me to vent. If you ask people about Wang Xiaobo, I'm sure they will think of the guy who used to pull poles in Sichuan province during the Song dynasty. They won't think of me. I am still in the process of moving against entropy. I'll say this: The existence of humanity, the development of civilization is a anti-entropic process. But that's humanity. When talking about myself specifically, I still can't explain my actions. I'll add another point: I am definitely not the only person in the process of acting against entropy. In America, I met a guy who set up a table on the sidewalk to sell books by Trotsky, Guevara, Chairman Mao, and others. When I spoke to him, he first asked me if I'm afraid of the FBI. There are many other examples. In these people, you will not see the magnificent process of water flowing from high to low, apples falling downwards, wolves eating rabbits. The phenomenon you see is akin to water flowing uphill, apples flying into the sky, rabbits eating wolves. I can also say, following entropy is not enough. For instance, if everyone followed the natural course of sliding downwards, in the end, we will all certainly gather together in a place down below, crowded together like maggots in a septic tank. But this also cannot explain my actions. My actions cannot be explained, if you take the phenomenon of increasing entropy as an immutable principle. 

Of course, if I am forced to use one straightforward sentence to answer this question, then it would be: I believe I have literary talent; I should do this. But this statement is just as unbelieveable as that made by a suspect who says that he didn't kill anyone. So it's up to you to believe it or not.

\chapter{跳出手掌心}

近来读了C.P.斯诺的《两种文化》。这本书里谈到的事倒是不新鲜,比方说,斯诺先生把知识分子分成了科学知识分子和文学(人文)知识分子两类,而且说,有两种文化,一种是科学文化,一种是文学(人文)文化。现在的每个知识分子,他的事业必定在其中一种之中。 

我要谈到的事,其实与斯诺先生的书只有一点关系,那就是,我以为,把两种文化合在一起,就是人类前途所系。这么说还不大准确,实际上,是创造了这两种文化的活动——人类的思索,才真正是人类前途之所系。尤瑟纳尔女士借阿德里安之口云,当一个人写作或计算时,就超越了性别,甚至超越了人类——当你写作和计算时,就是在思索。思索是人类的前途所系,故此,思索的人,超越了现世的人类。这句话讲得是非常之好的,只是讲得过于简单。实际上,并不是每一种写作或计算都可以超越人类。这种情况并不多见,但是非常的重要。 

现在我又想起了另一件事,乍看上去离题甚远:八十年代,美国通过了一个计划,拨出几百亿美元的资金,要在最短时间之内攻克癌症。结果却不令人满意,有些人甚至说该计划贻人笑柄,因为花了那么多钱,也没找出一种特效疗法。这件事说明,有了使不尽的钱,也不见得能做出突破性的发现。实际上,人类历史上任何一种天才的发现都不是金钱直接作用的结果。金钱、权力,这在现世上是最重要的东西,是人类生活的一面,但还有另一面。说到天才的发现,我们就要谈到天才、灵感、福至心灵、灵机一动等等,决不会说它们是某些人有了钱、升了官,一高兴想出来的。我要说的就是:沉默地思索,是人类生活的另外一面。就以攻克癌症为例,科学家默默地想科学、做科学,不定哪一天就做出一个发现,彻底解决了这个问题。但是,如果要约定一个期限,则不管你给多少钱也未必能成功。对于现代科技来说,资金设备等等固然重要,但天才的思想依然是最主要的动力。一种发现或发明可以赚到很多钱,但有了钱也未必能造出所要的发明。思索是一道大门,通向现世上没有的东西,通到现在人类想不到的地方。以科学为例,这个道理就是明明白白的。 

倘若我说,科学知识分子比人文知识分子人品高尚,肯定是不对的。科学知识分子里也有卑鄙之徒,比方说,前苏联的李森科。但我未听到谁对他的学说说过什么太难听的话,更没有听到谁做过这样细致的分析:李森科学说中某个谬误,和他的卑鄙内心的某一块是紧密相连的。倘若李森科不值得尊敬,李森科所从事的事业——生物学——依旧值得尊重。在科学上,有错误的学说,没有卑鄙的学说;就是李森科这样卑鄙的人为生物学所做的工作也不能说是卑鄙的行径。这样的道德标准显然不能适用于现在中国的艺术论坛,不信你就看看别人是怎样评论贾平凹先生的《废都》的。很显然,现在在中国,文学不是一种超越现世、超越人类的事业。我们评论它的标准,和三姑六婆评价身边发生的琐事的标准,没有什么不同。贾先生写了一部《废都》,就如某位大嫂穿了旗袍出门,我们不但要说衣服不好看,还要想想她的动机是什么,是不是想要勾引谁。另外哪位先生或女士写了什么好书,称赞他的话必是功在世道人心,就如称赞哪位女士相夫教子、孝敬公婆是一样的。当然,假如我说现在中国对文艺只有这样一种标准,那就是杜拉斯的《情人》问世不久,一下就出了四种译本(包括台湾的译本),电影《辛德勒的名单》国内尚未见到,好评就不绝于耳。我们说,这些将是传世之作,那就不是用现世的标准、道德的标准来评判的。这种标准从来不用之于中国人。由此得到一个结论,那就是在文学艺术的领域,外国人可以做超越人类的事业,中国人却不能。 

在文学艺术及其他人文的领域之内,国人的确是在使用一种双重标准,那就是对外国人的作品,用艺术或科学的标准来审评;而对中国人的作品,则用道德的标准来审评。这种想法的背后,是把外国人当成另外一个物种,这样对他们的成就就能客观地评价;对本国人则当作同种,只有主观的评价,因此我们的文化事业最主要的内容不是它的成就,而是它的界限;此种界限为大家所认同,谁敢越界就要被群起而攻之。当年孟子如此来评价杨朱和墨子:“无君无父,是禽兽也。”现在我们则如此地评价《废都》和一些在国外获奖的电影。这些作品好不好可以另论,总不能说人家的工作是“禽兽行”,或者是“崇洋媚外”。身为一个中国人,最大的痛苦是忍受别人“推己及人”的次数,比世界上任何地方的人都要多。我要说的不是自己不喜欢做中国人(这是我最喜欢的事),我要说的是,这对文化事业的发展很是不利。 

我认为,当我们认真地评价艺术时,所用的标准和科学上的标准有共通之处,那就是不依据现世的利害得失,只论其对不对(科学)、美不美(艺术)。此种标准我称为智慧的标准。假设有一种人类之外的智能生物,我们当然期望它们除了理解人类在科学上的成就之外,还能理解人类在艺术上的成就,故此,智慧就超越了人类。有些人会以为人类之外的东西能欣赏人类的艺术是不可能的,那么我敢和你打赌,此种生物在读到尤瑟纳尔女士的书时,读到某一句必会击节赞赏,对人类拥有的胸襟给予肯定;至于它能不能欣赏《红楼梦》,我倒不敢赌。但我敢断言,这种标准是存在的。从这种标准来看,人类侥幸拥有了智慧,就该善用它,成就种种事业,其中就包括了文学艺术在内。用这样的标准来度量,小说家力图写出一本前所未有的书,正如科学家力图做出发现,是值得赞美的事。当然,还有别的标准,那就是念念不忘自己是个人,家住某某胡同某某号,周围有三姑六婆,应该循规蹈矩地过一生,倘有余力,就该发大财,当大官,让别人说你好。这后一种标准是个人幸福之所系,自然不可忘记,但作为一个现代知识分子,前一种标准也该记住一些。 

一个知识分子在面对文化遗产时,必定会觉得它浩浩洋洋,仰之弥高。这些东西是数千年来人类智慧的积累,当然是值得尊重的。不过,我以为它的来源更值得尊重,那就是活着的人们所拥有的智慧。这种东西就如一汪活水,所有的文化遗产都是它的沉积物。这些活水之中的一小份可以存在于你我的脑子里,照我看来,这是世界上最美好的事情。保存在文化遗产里的智慧让人尊敬,而活人头脑里的智慧更让人抱有无限的期望。我喜欢看到人们取得各种成就,尤其是喜欢看到现在的中国人取得任何一种成就。智慧永远指向虚无之境,从虚无中生出知识和美;而不是死死盯住现时、现事和现在的人。我认为,把智慧的范围限定在某个小圈子里,换言之,限定在一时、一地、一些人、一种文化传统这样一种界限之内是不对的;因为假如智慧是为了产生、生产或发现现在没有的东西,那么前述的界限就不应当存在。不幸的是,中国最重大的文化遗产,正是这样一种界限,就像如来佛的手掌一样,谁也跳不出来;而现代的主流文化却诞生在西方。 

在中国做知识分子,有一种传统的模式,可能是孔孟,也可能是程朱传下来的,那就是自己先去做个循规蹈矩的人,做出了模样,做出了乐趣,再去管别人。我小的时候,从小学到中学,班上都有这样的好同学,背着手听讲,当上了小班长,再去管别人。现在也是这样,先是好好地求学,当了知名理论家、批评家,再去匡正世道人心。当然,这是做人的诀窍。做个知识分子,似乎稍嫌不够;除了把世道和人心匡得正正的,还该干点别的。由这样的模式,自然会产生一种学堂式的气氛,先是求学,受教,攒到了一定程度,就来教别人,管别人。如此一种学堂开办数千年来,总是同一些知识在其中循环,并未产生一种面向未来、超越人类的文化——谁要骂我是民族虚无主义,就骂好了,反正我从小就不是好同学——只产生了一个极沉重的传统,无数的聪明才智被白白消磨掉。倘若说到世道人心,我承认没有比中国文化更好的传统——所以我们这里就永远只有世道人心,有不了别的。 总之,说到知识分子的职责,我认为还有一种传统可循:那就是面向未来,取得成就。古往今来的一切大智者无不是这样做的。这两种知识分子的形象可以这样分界,前一种一世的修为,是要做个如来佛,让别人永世跳不出他的手掌心;后一种是想在一生一世之中,只要能跳出别人的手掌心就满意了。我想说的就是,希望大家都做后一种知识分子,因为不管是谁的手掌心,都太小了。

\chapter{道德保守主义及其他}

为《东方》的社会伦理漫谈专栏写文章时,我怀有一种特殊的责任感,期待自己的工作能为提高社会的道德水平做出一点贡献。然而作为中国的知识分子,随时保持内省的状态是我们的传统,不能丢掉。 

我记得在我之前写这个专栏的何怀宏先生,写过一篇讨论全社会的道德水平能否随经济发展提高的文章,得出了“可以存疑”的结论。对于某些人来说,何先生的结论不能令人满意。结论似乎应当是可以提高而且必须提高。如果是这样,那篇文章就和大多数文章一样,得到一种号召积极行动的结论。 

号召积极行动的结论虽好,但不一定合理。再说,一篇文章还没有读,结论就已知道,也不大有趣。我认为,目前文化界存在着一种“道德保守主义”,其表现之一就是多数文章都会得到这种结论。 

在道德这个论域,假如不持保守的立场,就不会一味地鼓吹提高全社会的道德水平。举例言之,假如你持宋儒的观点,就会认为,全社会没有了再醮的寡妇,所有的女孩子都躲在家里等待“父母之命、媒妁之言”,道德水平就是很高的,应该马上朝这个方向努力;而假设你是“五四”之后的文化人,就会认为这种做法道德水平有多高是有问题的,也就不急于朝那个方面努力。这个例子想要说明的是,当你急于提高全社会道德水平时,也许已经忽略了社会伦理方面发生的变革;而且这种变革往往受到了别的因素的影响,实际上是不可避免的。事实上,因为我们国家很大一部分人的生活方式正在改变,这种变革也正在发生,所以如何去提高道德水平是个最复杂的问题;而当我们这样提出问题时,也就丧失了提高道德水平的急迫感。 

前年夏天,我到外地开一个会——在此声明,我很少去开会,这个会议的伙食标准也不高——看到一位男会友穿了一件文化衫,上面用龙飞凤舞的笔迹写着一串英文:OK,Let’s pee!总的来说,这个口号让人振奋,因为它带有积极、振奋的语调,这正是我们都想听到的。但是这个pee是什么意思不大明白,我觉得这个字念起来不大对头。回来一查,果不出我所料,是尿尿的意思。搞明白了全句的意思,我就觉得这话不那么激动人心了。众所周知,我们已过了要人催尿的年龄,在小便这件事上无须别人的鼓励。 

我提到这件事,不是要讨论如何小便的问题,而是想指出,在做一件事之前,首先要弄明白是在干什么,然后再决定是不是需要积极和振奋。 

这只是我个人的意见,当然,有些人在这类事情上一向以为,无论干的是什么,积极和振奋总是好的。假如倒回几年,到了“文化革命”里,连我也是这样的人。当年我坚信,一切方向问题都已解决,只剩下一件事,“毛主席挥手我前进”,所以在回忆年轻时代的所作所为之时,唯一可以感到自豪的事就是:那段时间我一直积极而振奋,其他的事都只能令我伤心。 

我个人认为,一个社会的道德水准取决于两个方面,一是价值取向,二是在这些取向上取得的成就;很显然,第一个方面是根本。倘若取向都变了,成就也就说不上,而且还会适得其反。因此,要提高社会的道德水准就要解决两方面的问题。一、弄清哪一种价值取向比较可取;二、以积极进取的态度来推进它。坦白地说,我只关心第一个问题。换言之,我最关心pee是要干什么,在搞明白它是什么意思之前,对OK,Let’s中包含的强烈语气无动于衷。我知道自己是个挺极端的例子;另一种极端的例子是对干什么毫不关心,只关心积极进取,狂热推动。我觉得自己所处的这个极端比较符合知识分子的身份,并为处于另一极端的朋友捏一把冷汗。假如他们凑巧持一种有益无害的价值取向,行为就会很好;假如不那么凑巧,就要成为一种很大的祸害。因为这个原故,他们的一生是否能于社会有益、于人类有益,就不再取决于自己,而是取决于机遇。正因为有这样的人存在,思考何种社会伦理可取的人的责任就更重大了。 

我本人关心社会伦理问题,是从研究同性恋始。我做社会学研究,但是这样一个研究题目当然和社会伦理问题有关系。现在有人说,同性恋是一种社会丑恶现象,我反对这种说法,但不想在此详加讨论——我的看法是,同性恋是指一些人和他们的生活,说人家是种社会现象很不郑重。我要是说女人是种社会现象,大家以为如何?——我只想转述一位万事通先生在澡堂里对这个问题发表的宏论,他说:“同性恋那是外国的高级玩艺儿,我们这里有些人就会赶时髦……这艾滋病也不是谁想得就配得的!”在他说这些话时,我的一位调查对象就在一边坐着。后者告诉我说,他的同性恋倾向是与生俱来的。他既不是想赶时髦,也不是想得艾滋病。他还认为,生为一个同性恋者,是世间最沉重的事。我想,假如这位万事通先生知道这一切,也不会对同性恋做出轻浮、赶时髦这样的价值评判,除非他对自己说出的话是对是错也不关心。我举这个例子是想说明:伦理道德的论域也和其他论域一样,你也需要先明白有关事实才能下结论,而并非像某些人想象的那样,只要你是个好人,或者说,站对了立场,一切都可以不言自明。不管你学物理也好,学数学也罢,都得想破了脑袋,才能得到一点成绩;假设有一个领域,你在其中想都不用想就能得到大批的成绩,那倒是很开心的事。不过,假如我有了这样的感觉,一定要先去看看心理医生。 

在本文开始的时候,提出了“道德保守主义”这样一种说法。我以为“道德保守主义”和不问价值取向是否合理、只求积极进取的倾向,在现象上是一回事,虽然它们在逻辑上没有什么联系。这主要是因为假如你不考虑价值取向这样一个主要问题(换言之,你以为旧有的价值取向都是对的,无须为之动脑子),就会节省大量的精力,干起呼吁、提倡这类事情时,当然精力充沛,无人能比。 

举例来说,有关传统道德里让寡妇守节,我们知道,有人说过饿死事小,失节事大;又有人说过饿死事极小,失节事极大。这些先生没有仔细考虑过让寡妇守节是否合理,此种伦理是否有必要变革,所以才能如此轻松地得出要丧偶女士饿死这样一个可怕的结论。 

喜欢萧伯纳的朋友一定记得,在《巴巴拉少校》一剧里,安德谢夫先生见到了平时很少见到的儿子斯泰芬。老先生要考较一下儿子,就问他能干点什么。他答道:干什么都不行,我的特长在于明辨是非。假如我理解得对,斯泰芬先生是说他在伦理道德方面有与生俱来的能力。安德谢夫把斯泰芬狠狠损了一顿,说道:你说的那件事,其实是世界上最难的事。 

当然,这位老爷子不是在玩深沉,他的意思是说,你要明辨是非,就要把与此有关的一切事都搞清。这是最高的智慧,绝不是最低的一种。这件事绝不轻松,是与非并不是不言自明的。 

在伦理道德的论域里,有两种不同的态度:一种认为,只有详细地考虑有关证据,经过痛苦的思索过程,才能搞清什么是对,什么是错——我就是这样考虑伦理问题的;另一种认为,什么是对什么是错根本无须考虑,只剩下了如何行动的问题——我嫉妒这种立论的方式,这实在太省心。假设有位女子风华绝代,那么她可以认为,每个男人都会爱上她,而且这么想是有理由的。但我很难想象,什么样的人才有资格相信自己一拍脑袋想出来的东西就是对的;现在能想出的唯一例子就是圣灵充满的耶稣基督。我这辈子也不会自大到这种程度。还有一种东西可以拯救我们,那就是相信有一种东西绝对是对的,比如一个传统,一本小红书,你和它融为一体时,也就达到了圣灵充满的境界。 

在这种状态下,你会感到一切价值取向上的是与非都一目了然,你会看到那些没有被“充满”的人都是那么堕落,因而充满了道德上的紧迫感。也许有一天,我会向这种诱惑屈服,但现在还不肯。

\chapter{我看文化热}

我们已经有了好几次文化热:第一次好像是在八五年,我正在海外留学,有朋友告诉我说,国内正在热着。到八八年我回国时,又赶上了第二次热。这两年又来了一次文化批评热,又名“人文精神的讨论”。看来文化热这种现象,和流行性感冒有某种近似之处。前两次热还有点正经,起码介绍了些国外社会科学的成果,最近这次很不行,主要是在发些牢骚:说社会对人文知识分子的态度不端正,知识分子自己也不端正;夫子曰,君子喻于义,小人喻于利,我们要向君子看齐——可能还说了些别的。但我以为,以上所述,就是文化批评热中多数议论的要点。在文化批评热里王朔被人臭骂,正如《水浒传》里郓城县都头插翅虎雷横在勾栏里遭人奚落:你这厮若识得子弟门庭时,狗头上生角!文化就是这种子弟门庭,决不容痞子插足。如此看来,文化是一种以自我为中心的价值观,还有点党同伐异的意思;但我不愿把别人想得太坏,所以就说,这次热的文化,乃是一种操守,要求大家洁身自好,不要受物欲的玷污。我们文化人就如唐僧,俗世的物欲就如一个母蝎子精,我们可不要受她的勾引,和那个妖女睡觉,丧了元阳,走了真精,此后不再是童男子,不配前往西天礼佛——这样胡扯下去,别人就会不承认我是文化人,取消我讨论文化问题的权利。我想要说的是,像这样热下去,我就要不知道文化是什么了。 

我知道一种文化的定义是这样的:文化是一个社会里精神财富的积累,通过物质媒介(书籍、艺术品等等)传诸后世或向周围传播。根据这种观点,文化是创造性劳动的成果。现在正热着的观点却说,文化是种操守,是端正的态度,属伦理学范畴。我也不便说哪种观点更对。但就现在人们呼吁的“人文精神的回归”,我倒知道一个例子:文艺复兴。这虽是个历史时期,但现在还看得见、摸得着。为此我们可以前往佛罗伦萨,那里满街都是文艺复兴时期的建筑,这种建筑是种人文的成果。佛罗伦萨还有无数的画廊、博物馆,走进去就可以看见当时的作品——精妙绝伦,前无古人。由于这些人文的成果,才可以说有人文精神。倘若没有这些成果,佛罗伦萨的人空口说白话道:“我们这里有过一种人文精神”,别人不但不信,还要说他们是骗子。总而言之,所谓人文精神,应当是对某个时期全部人文成果的概括。 

现在可以回过头去看看,为什么在中国,一说到文化,人们就往伦理道德方面去理解。我以为这是种历史的误会。众所周知,中国文化的最大成就,乃是孔孟开创的伦理学、道德哲学。这当然是种了不得的大成果,如其不然,别人也不会承认有我们这种文化。很不幸的是,这又造成了一种误会,以为文化即伦理道德,根本就忘了文化应该是多方面的成果——这是个很大的错误。不管怎么说,只有这么一种成果,文化显得单薄乏味。打个比方来说,文化好比是蔬菜,伦理道德是胡萝卜。说胡萝卜是蔬菜没错,说蔬菜是胡萝卜就有点不对头——这次文化热正说到这个地步,下一次就要说蔬菜是胡萝卜缨子,让我们彻底没菜吃。所以,我希望别再热了。

\chapter{文化之争}

罗素先生在《权力论》一书里,提到有一种僧侣的权力,过去掌握在教士们手里。他还说,在西方,知识分子是教士的后裔。另外,罗素又说,中国的儒学也拥有僧侣的权力。这就使人想到,中国知识分子是儒士的后裔。教士和儒士拥有的知识来自一些圣书,《圣经》或者《论语》之类。而近代知识分子,即便不是全部,起码也是一部分人,手里并没有圣书。他们令人信服,全凭知识;这种知识本身就可以取信于人。奇怪的是,这后一种知识并不能带来权力。 

把儒学和宗教并列,肯定会招来一些反对。儒学没有凭借神的名义,更没有用天堂和地狱来吓唬人。但它也编造了一个神话,就是假如你把它排除在外,任何人都无法统治,天下就会乱作一团,什么秩序、伦理、道德都不会有。这个神话唬住了一代又一代的中国人,直到现在还有人相信。罗素说,对学者的尊敬从来就不是出于真知,而是因为想象中他具有的魔力。我认为,儒学的魔力就是统治神话的魔力。当然,就所论及的内容来说,儒学是一种哲学,但是圣人说的那些话都是些断语,既没有什么证据,也没有什么逻辑。假如不把统治的魔力估计在内,很难相信大家会坚信不移。 

罗素所说的“真知”是指科学。这种知识,一个心智正常的人,只要肯花工夫,就能学会。众所周知,科学不能解决一切问题,特别是在价值的领域。因此有人说它浅薄。不过,假如你真的花了些时间去学,就会发现,它和儒学有很大的不同。 

我们知道,儒士的基本功是要背书,把圣人说过的每一句话都牢牢地记住。我相信,假如孔子或者孟子死而复生,看到后世的儒生总在重复他们说过的只言片语,一定会感到诧异。当然,也不能说这些儒生只是些留声机。因为他们在圣人之言前面都加上了前缀“夫子曰”。此种怪诞的情形提示了儒学的精神:让儒士成为圣人的精神复制品。按我的理解,这种复制是通过背诵来完成的。从另一个方面来说,背诵对儒士也是有利可图的。我们知道,有些人用背诵《韦氏大字典》的方式来学习英文。与过去背圣人书可以得到的利益相比,学会英文的利益实在太小。假设你真的成为圣人的精神复制品,就掌握了统治的魔力,可以学而优则仕,当个官老爷;而会背诵字典的人只能去当翻译,拿千字20元的稿酬。这两种背诵真不可同日而语。 

现在我们来看看科学。如果不提它的复杂性,它是一些你知道了就会同意的东西。它和“君君、臣臣、父父、子子”不同,和“天人合一”也不同。这后两句话我知道了很多年,至今还没有同意。更重要的是,科学并不提倡学者成为某种精神的复制品,也不自称有某种魔力。因为西方知识分子搞出了这种东西,所以不再受人尊重。假如我们相信罗素先生的说法,西方知识分子就是这样拆了自己的台。可恨的是,他们不但拆了自己的台,还要来拆中国知识分子的台。更可恨的是,有些中国知识分子也要来拆自己的台——晚生正是其中的一人。 

自从近代以来,就有一种关于传统文化的争论。我们知道,文化是人类的生活方式,它有很多方面。而此种争论总是集中在如何对待传统哲学之上,所以叫做“文化之争”多少有点名不副实。在争论之中,总要提到中外有别,中国有独特的国情。照我看,争论中有一方总在暗示着传统学术统治的魔力,并且说,在中国这个地方,离开了这种魔力是不行的。假如我理解得不错,说中国离开了传统学术独特的魔力就不行,不是一个问题,而是两个问题。其一是说,作为儒学传统嫡系子孙的那些人离开了这种魔力就不成。其二是说,整个中国的芸芸众生离开了这种魔力就不行。把这两件事伙在一起来说,显然是很不恰当。如果分开来说,第一个问题就很是明白。儒学的嫡系子孙们丧失了统治的魔力之后,就沦为雇员,就算当了教授、研究员,地位也不可与祖先相比。对于这种状况,罗素先生有个说明:“知识分子发现他们的威信因自己的活动而丧失,就对当代世界感到不满。”他说的是西方的情形。在中国,这句话应该改为:某些中国知识分子发现自己的权威因为西方知识分子的活动而丧失,所以仇恨西洋学术和外国人。至于第二个问题,却是越说越暧昧难明。我总是在怀疑,有些人心里想着第一个问题,嘴上说着第二个问题。凭良心说,我很希望自己怀疑错。 

我们知道,优秀的统帅总是选择于己有利的战场来决战。军事家有谋略是件好事,学者有谋略好不好就值得怀疑。赞成传统文化的人现在有一种说法,以为任何民族都要尊重自己的文化传统,否则就没有前途。晚生以为,这种说法有选择战场的嫌疑。在传统这个战场上,儒士比别人有利。不是儒士的人有理由拒绝这种挑战。前不久晚生参与了一种论战,在论战中,有些男士以为现在应当回到传统,让男主外女主内;有些女士则表示反对。很显然,在传统这个战场上,男人比女人有利。我虽是男人,却站到了女人一方;因为我讨厌这种阴谋诡计。 

现在让我们回到正题。罗素先生曾说,他赞成人人平等。但很遗憾的是,事实远不是这样。人和人是不平等的,其中最重要的,是人与人有知识的差异。这就提示说,由知识的差异可以产生权力。让我们假设世界上的人都很无知,唯有某个人全知全能,那么此人就可能掌握权力。中国古代的圣贤和现代的科学家相比,寻求知识的热情有过之而无不及。在圣贤中,特别要提出朱熹,就我所知,他的求知热情是古往今来的第一人。科学家和圣贤的区别在于,前者不但寻求知识,还寻求知识的证明。不幸的是,证明使知识人人可懂,他们就因此丧失了权力。相比之下,圣贤就要高明很多。因此,他们很快就达到了全知全觉的水平,换言之,达到了“内圣”的境界;只是这些知和觉可靠不可靠却大成问题。我们知道,内圣和外王总是联系在一起的。假如我们说,圣贤急于内圣,是为了外王,就犯了无凭据地猜度别人内心世界的错误。好在还有朱熹的话来作为佐证:他也承认,自己格物致知,是为了齐家治国平天下。 

现在,假如我说儒家的道德哲学和伦理学是全然错误的,也没有凭据。我甚至不能说这些东西是令人羞愧的知识。不过,这些知识里的确有令人羞愧的成分,因为这种知识的追随者,的确用它攫取了僧侣的权力。至于这种知识的发明人,我是指孔子、孟子,不包括朱熹,他们是无辜的。因为他们没有想获得、更没有享受到这种权力。倘若今日仍有人试图通过复兴这种知识来获得这种权力,就可以用孟子的话来说他们:“无耻之耻,无耻矣。”当然,有人会说,我要复兴国学,只是为了救民于水火,振兴民族的自尊心。这就等于说,他在道德上高人一等,并且以天下为己任。我只能说,这样赤裸裸地宣扬自己过于直露,不是我的风格;同时感到,僧侣的权力又在叩门。僧侣的权力比赤裸裸的暴虐要好得多,这我是承认的。虚伪从来就比暴力好得多。但我又想,生活在二十世纪末,我们有理由盼望好一点的东西。当然,对我这种盼望,又可以反驳说,身为一个中国人,你也配!——此后我除了向隅而泣,就想不到别的了。

\chapter{“行货感”与文化相对主义}

《水浒传》上写到,宋江犯了法,被刺配江州,归戴宗管。按理他该给戴宗些好处,但他就是不给。于是,戴宗就来要。宋江还是不给他,还问他:我有什么短处在你手里,你凭什么要我的好处?戴宗大怒道:还敢问我凭什么?你犯在我的手里,轻咳嗽都是罪名!你这厮,只是俺手里的一个行货!行货是劣等货物,戴宗说,宋江是一件降价处理品,而他自己则以货主自居。我看到这则故事时,只有十二岁,从此就有了一种根深蒂固的行货感,这是一种很悲惨的感觉。在我所处的这个东方社会里,没有什么能冲淡我的这种感觉——这种感觉中最悲惨的,并不是自己被降价处理,而是成为货物这一不幸的事实。最能说明你是一件货物的事就是:人家拿你干了什么或对你有任何一种评价,都无须向你解释或征得你的同意。我个人有过这种经历:在我十七岁时,忽然就被装上了火车,经长途运输运往云南,身上别了一个标签:屯垦戍边。对此我没有什么怨言,只有一股油然而生的行货感。对于这件事,在中国的文化传统里早有解释:普天之下,莫非王土;率土之滨,莫非王臣……是啊,普天之下,莫非王士,我不是王;率上之滨,莫非王臣,我又不是王。我总觉得这种解释还不如说我是个行货更直接些。 

古埃及的人以为,地球是圆的——如你所知,这是事实;古希腊的人却以为,地是一块平板,放在了大鲸鱼的背上,鲸鱼漂在海里,鲸鱼背上一痒,就要乱蹭,然后就闹地震——这就不是事实。罗素先生说,不能因此认为埃及人聪明,希腊人笨。埃及人住在空旷的地方,往四周一看,圆圆一圈地平线,得出正确的结论不难。希腊人住在多山、多地震的滨海地区,难怪要想到大海、鲸鱼。同样是人,生在旷野和生在山区,就有不同的见识。假若有人生为行货,见识一定和生为货主大有不同。后一方面的例子有美国《独立宣言》,这是两百年前一批北美的种植园主起草的文件,照我们这里的标准,通篇都是大逆不道的语言。至于前一方面的例子,中国的典籍里多的是,从孔孟以降,讲的全是行货言论,尤其是和《独立宣言》对照着读,更是这样。我对这种言论很不满,打算加以批判。但要有个立脚点:我必须证明自己不是行货——身为货物,批判货主是不对的。 

这些年来,文化热常盛不衰,西方的学术思潮一波波涌进了中国。有一些源于西方的学术思想正是我的噩梦——这些学术思想里包括文化相对主义、功能学派,等等。说什么文化是生活的工具(马林诺夫斯基的功能论),没有一种文化是低等的(文化相对主义),这些思想就是我的噩梦。从道理上讲,这些观点是对的,但要看怎么个用法;遇上歪缠的人,什么好观点都要完蛋。举例来说,江州大牢里的宋江,他生活在一种独特的文化之中(我们可以叫它宋朝的牢狱文化),按照这种文化的定义,他是戴宗手里的行货,他应该给戴宗送好处。他若对戴宗说,人人生而平等,我也是一个人,凭什么说我是宗货物?咱们这种文化是有毛病的。戴宗就可以说:宋公明,根据文化相对主义的原理,没有一种文化有毛病,咱们这种文化很好,你还是安心当我的行货吧。宋江若说:虽然这种文化很好,但你向我要好处是敲诈我,我不能给。戴宗又可以说:文化是生活的工具,既然在我们的文化里你得给我好处,这件事自有它的功能,你还是给了吧。如果不给,我就要按咱这种文化的惯例,用棍子来打你了——你先不要不满意,打你也有打你的功能。这个例子可以说明文化人类学的观点经不住戴宗的歪曲、滥用。实际上,没有一种科学能经得起歪曲、滥用。但有一些学者学习西方的科学,就是为了用东方的传统观念来歪曲的。从文化相对主义,就能歪曲出一种我们都是行货的道理来。 

我们知道,非洲有些地方有对女孩行割礼的习惯,这是对妇女身心的极大摧残。一些非洲妇女已经起而斗争,反对这种陋习。假如非洲有些食洋不化的人说:这是我们的文化,万万动不得,甚至搬出文化相对主义来,他肯定是在胡扯。文化相对主义是人类学家对待外文化的态度,可不是让宋公明当行货,也不是让非洲的女孩子任人宰割。人生活在一种文化的影响之中,他就有批判这种文化的权利。我对自己所在的文化有所批评,这是因为我生活在此地,我在这种文化的影响之下,所以有批判它的权利。假设我拿了绿卡,住在外国,你说我没有这种权利,我倒无话可说。这是因为,人该是自己生活的主宰,不是别人手里的行货。假如连这一点都不懂,他就是行尸走肉,而行尸走向是不配谈论科学的。

\chapter{我看国学}

我现在四十多岁了,师长还健在,所以依然是晚生。当年读研究生时,老师对我说,你国学底子不行,我就发了一回愤,从《四书》到二程、朱子乱看了一通。我读书是从小说读起,然后读四书;做人是从知青做起,然后做学生。这样的次序想来是有问题。虽然如此,看古书时还是有一些古怪的感慨,值得敝帚自珍。读完了《论语》闭目细思,觉得孔子经常一本正经地说些大实话,是个挺可爱的老天真。自己那几个学生老挂在嘴上,说这个能干啥,那个能干啥,像老太太数落孙子一样,很亲切。老先生有时候也鬼头鬼脑,那就是“子见南子”那一回。出来以后就大呼小叫,一口咬定自己没“犯色”。总的来说,我喜欢他,要是生在春秋,一定上他那里念书,因为那儿有一种“匹克威克俱乐部”的气氛。至于他的见解,也就一般,没有什么特别让人佩服的地方。至于他特别强调的礼,我以为和“文化革命”里搞的那些仪式差不多,什么早请示晚汇报,我都经历过,没什么大意思。对于幼稚的人也许必不可少,但对有文化的成年人就是一种负担。不过,我上孔老夫子的学,就是奔那种气氛而去,不想在那里长什么学问。 

《孟子》我也看过了,觉得孟子甚偏执,表面上体面,其实心底有股邪火。比方说,他提到墨子、杨朱,“无君无父,是禽兽也”,如此立论,已然不是一个绅士的作为。至于他的思想,我一点都不赞成。有论家说他思维缜密,我的看法恰恰相反。他基本的方法是推己及人,有时候及不了人,就说人家是禽兽、小人;这股凶巴巴恶狠狠的劲头实在不讨人喜欢。至于说到修辞,我承认他是一把好手,别的方面就没什么。我一点都不喜欢他,如果生在春秋,见了面也不和他握手。我就这么读过了孔、孟,用我老师的话来说,就如“春风过驴耳”。我的这些感慨也只是招得老师生气,所以我是晚生。 

假如有人说,我如此立论,是崇洋媚外,缺少民族感情,这是我不能承认的。但我承认自己很佩服法拉第,因为给我两个线圈一根铁棍子,让我去发现电磁感应,我是发现不出来的。牛顿、莱布尼兹,特别是爱因斯坦,你都不能不佩服,因为人家想出的东西完全在你的能力之外。这些人有一种惊世骇俗的思索能力,为孔孟所无。按照现代的标准,孔孟所言的“仁义”啦,“中庸”啦,虽然是些好话,但似乎都用不着特殊的思维能力就能想出来,琢磨得过了分,还有点肉麻。这方面有一个例子:记不清二程里哪一程,有一次盯着刚出壳的鸭雏使劲看。别人问他看什么,他说,看到毛茸茸的鸭雏,才体会到圣人所说“仁”的真意。这个想法里有让人感动的地方,不过仔细一体会,也没什么了不起的东西在内。毛茸茸的鸭子虽然好看,但再怎么看也是只鸭子。再说,圣人提出了“仁”,还得让后人看鸭子才能明白,起码是辞不达意。我虽然这样想,但不缺少民族感情。因为我虽然不佩服孔孟,但佩服古代中国的劳动人民。劳动人民发明了做豆腐,这是我想象不出来的。 

我还看过朱熹的书,因为本科是学理工的,对他“格物”的论述看得特别的仔细。朱子用阴阳五行就可以格尽天下万物,虽然阴阳五行包罗万象,是民族的宝贵遗产,我还是以为多少有点失之于简单。举例来说,朱子说,往井底下一看,就能看到一团森森的白气。他老人家解释适,阴中有阳,阳中有阴(此乃太极图之象),井底至阴之地,有一团阳气,也属正常。我相信,你往井里一看,不光能看到一团白气,还能看到一个人头,那就是你本人(我对这一点很有把握,认为不必做实验了)。不知为什么,这一点他没有提到。可能观察得不仔细,也可能是视而不见,对学者来说,这是不可原谅的。还有可能是井太深,但我不相信宋朝就没有浅一点的井。用阴阳学说来解释这个现象不大可能,也许一定要用到几何光学。虽然要求朱子一下推出整个光学体系是不应该的,那东西太过复杂,往那个方向跨一步也好。但他根本就不肯跨。假如说,朱子是哲学家、伦理学家,不能用自然科学家的标准来要求,我倒是同意的。可怪的是,咱们国家几千年的文明史,就是出不了自然科学家。 

现在可以说,孔孟程朱我都读过了。虽然没有很钻进去,但我也怕钻进去就爬不出来。如果说,这就是中华文化遗产的主要部分,那我就要说,这点东西太少了,拢共就是人际关系里那么一点事,再加上后来的阴阳五行。这么多读书人研究了两千年,实在太过分。我们知道,旧时的读书人都能把四书五经背得烂熟,随便点出两个字就能知道它在书中什么地方。这种钻研精神虽然可佩,这种做法却十足是神经病。显然,会背诵爱因斯坦原著,成不了物理学家;因为真正的学问不在字句上,而在于思想。就算文科有点特殊性,需要背诵,也到不了这个程度。因为“文革”里我也背过毛主席语录,所以以为,这个调调我也懂——说是诵经念咒,并不过分。 

二战期间,有一位美国将军深入敌后,不幸被敌人堵在了地窖里,敌人在头上翻箱倒柜,他的一位随行人员却咳嗽起来。将军给了随从一块口香糖让他嚼,以此来压制咳嗽。但是该随从嚼了一会儿,又伸手来要,理由是:这一块太没味道。将军说:没味道不奇怪,我给你之前已经嚼了两个钟头了!我举这个例子是要说明,四书五经再好,也不能几千年地念;正如口香糖再好吃,也不能换着人地嚼。当然,我没有这样地念过四书,不知道其中的好处。有人说,现代的科学、文化,林林总总,尽在儒家的典籍之中,只要你认真钻研。这我倒是相信的,我还相信那块口香糖再嚼下去,还能嚼出牛肉干的味道,只要你不断地嚼。我个人认为,我们民族最重大的文化传统,不是孔孟程朱,而是这种钻研精神。过去钻研四书五经,现在钻研《红楼梦》。我承认,我们晚生一辈在这方面差得很远,但也未尝不是一件好事。四书也好,《红楼梦》也罢,本来只是几本书,却硬要把整个大千世界都塞在其中。我相信世界不会因此得益,而是因此受害。 

任何一门学问,即便内容有限而且已经不值得钻研,但你把它钻得极深极透,就可以挟之以自重,换言之,让大家都佩服你;此后假如再有一人想挟这门学问以自重,就必须钻得更深更透。此种学问被无数的人这样钻过,会成个什么样子,实在难以想象。那些钻进去的人会成个什么样子,更是难以想象。古宅闹鬼,树老成精,一门学问最后可能变成一种妖怪。就说国学吧,有人说它无所不包,到今天还能拯救世界,虽然我很乐意相信,但还是将信将疑。

\chapter{中国知识分子与中古遗风}

一、什么是知识分子? 

我到现在还不确切知道什么人算是知识分子,什么人不算。 

插队的时候,军代表就说过我是“小资产阶级知识分子”。那一年我只有十七岁,上过六年小学,粗识些文字,所以觉得“知识分子”这四个字受之有愧。顺便说一句,“小资产”这三个字也受之有愧,我们家里吃的是公家饭,连家具都是公家的,又没有在家门口摆摊卖香烟,何来“小资产”?至于说到我作为一个人,理应属于某个阶级,我倒是不致反对,但到现在我也不知道“知识青年”算什么阶级。假如硬要比靠,我以为应当算是流氓无产者之类。这些已经扯得太远了。我们国家总以受过某种程度的教育尺度来界定知识分子,外国人却不是这样想的。 

我在美国留学时,和老美交流过,他们认为工程师,牙医之类的人,只能算是专业人员,不算知识分子,知识分子应该是在大学或者研究部门供职,不坐班也不挣大钱的那些人。照这个标准,中国还算有些知识分子。《纽约时报》有一次对知识分子下了个定义,我不敢引述,因为那个标准说到了要“批判社会”,照此中国就没有或几乎没有知识分子。还有一个定义是在消闲刊物上看来的,我也不大敢信。照那个标准,知识分子全都住在纽约的格林威治村,愤世疾俗,行为古怪,并且每个人都以为自己是世界上最后一个知识分子。所以我们还是该以有一份闲差或教职为尺度来界定现在的知识分子,以便比较。 

如果到历史上去找知识分子,先秦诸子和古希腊的哲学家当然是知识分子,但是距离太遥远。到了中古,我们找到的知识分子的对应物就该是这样的:在中国,是一些进了县学或者州学的读书人,在等着参加科举考试的时候,能领到些米或者柴火;学官不时来考较一下,实在不通的要打一顿;等到中了科举当了官,恐怕就不能算是知识分子;所研究的学问,属于伦理学或者道德哲学之类。而在欧洲,是些教士或修道士,通晓拉丁文,打一辈子光棍,万一打熬不住,搞了同性恋,要被火烧死,研究的学问是神学,一个针尖上能立几个天使之类。虽然生活清苦,两边的知识分子都有远大的理想。这边以天下为己任,不亦重乎?那边立志献身于上帝,不亦高尚乎?当然,两边都出了些好人物。咱们有关汉卿,曹雪芹,人家有哥白尼,布鲁诺,不说是平分秋色,起码是各有千秋。所以在中古时中外知识分子很是相像。到了近代就不像了。 

二、中国的知识分子的中古遗风 

现代中国的知识分子,相比之下中古的遗风多些,首先表现在受约束上。试举一例,有一位柯老说过,知识分子两大特点,一是懒,二是贱……三天不打,尾巴就翘到天上去了。他老人家显出了学官的嘴脸。前几天我在电视剧《针眼儿胡同》里听见一位派出所所长也说了类似的话,此后我一直等待正式道歉,还没等到。顺便说说,当年军代表硬要拿我算个知识分子,也是要收拾我。此种事实说明,中国知识分子的屁股离学官的板子还不太远。而外国的例子是有一位赫赫有名的福柯,颇有古希腊的遗风,是公开的同性恋者,未听说法国人要拿他点天灯。 

不管怎么说,中外知识分子还是做着一样的事,只是做法不同-否则也不能都被叫做知识分子-这就是做自己的学问和关注社会。做学问的方面,大家心里有数,我就不加评论了。至于关注社会,简直是一目了然-关心的方式大不相同。中国知识分子关注社会的伦理道德,经常赤膊上阵,论说是非;而外国的知识分子则是以科学为基点,关注人类的未来;就是讨论道德问题,也是以理性为基础来讨论。弗罗姆,马尔库塞的书,国内都有译本,大家看看就明白了。人家那里热衷于伦理道德的,主要是些教士,还有一些是家庭妇女(我听说美国一些抵制色情协会都是家庭妇女在牵头-可能有以偏盖全之处)。我敢说大学教授站在讲坛上,断断不会这样说:你们这些罪人,快忏悔吧,……这与身份不符。因为口沫飞溅,对别人大做价值评判,层次很低。教皇本人都不这样,我在电视上看到过他,笑眯眯的,说话很和气,遇到难以教化的人,就说:我为你祷告,求上帝启示于你-比之我国某位作家动不动就“警告XXX”,真有天壤之别。据我所知,教皇博学多识,我真想把他也算个知识分子,就怕他不乐意当。 

我国知识分子在讨论社会问题时,常说的一件事就是别人太无知。举例言之,我在海外求学时,在《人民日报》(海外版)上看到了一篇文章,就说现在大学生水平太低,连“郭鲁茅巴”都不知道,我登时就如吃了一闷棍。我想这是个蒙古人,不知为什么我该知道他。想到了半夜才想出来,原来他是郭沫若,鲁迅,茅盾,巴金四位先生。一般来说,知识的多寡是个客观的标准,但把自编的黑话也列入知识的范畴,就难说有多客观了。现在中学生不知道李远哲也是个罪名-据我所知,学化学的研究生也未必能学到李先生的理论;他们还有个罪名是“追星族”,鬼迷心窍,连杨振宁,李政道,李四光是谁都不知道。据我所知,这三位先生的学问实在高深,中学生根本不该懂,不知道学问,死记些名字,有何必要?更何况记下这些名字之后屈指一算,多一半都入了美国籍,这是给孩子灌输些什么?还有一个爱说的话题就是别人“格调低下”,我以为这句话的意思是说:“兄弟我格调甚高,不是俗人!”我在一篇匈牙利小说里看到过这种腔调,小说的题目叫《会说话的猪》。总的来说,这类文章的要点是说别人都不够好,最后呼吁要大大提高全社会的道德水平,否则就要国将不国。这种挑别人毛病的文章,国外的报刊上也有。只是挑出来的毛病比较靠谱,而且没有借着贬别人来抬自己。如果把道德的伦理功能概括为批判和建设两个方面,以上所说的属于批判方面。我不认为这是批判社会-这是批判人。知识分子的批判火力对两类人最为猛烈:一类是中学生;另一类是踩着地雷断了腿的同类。这道理很明白-别人咱也惹不起。 

现在该说说建设的方面了。这些年来,大家蜂拥而上赞美过的正面形象,也就是《渴望》里面的一位妇女。该妇女除了长得漂亮之外,还像是封建时期一个完美的小媳妇。当然,大伙是从后一个方面,而不是前一个方面来赞美她;这也是中古的遗风。不过,要旌表一个戏中人,这可太古怪了。我们知识分子的正面形象则是:谢绝了国外的高薪聘请,回国服务。想要崇高,首先要搞到一份高薪聘请,以便拒绝掉,这也太难为人了;在知识分子里也没有普遍意义。所以,除了树立形象,还该树立个森严的道德体系,把大家都纳入体系。从道德上说事,就人人都被说着了。 

所谓道德体系,是价值观念里跟人有关的部分。有人说他森严点好,有人说他松散点好,我都没有意见。主要的问题是,价值观念不是某个人能造出来的(人类学上有些说法,难以一一引述),道德体系也不是说立哪个就能立起哪个。就说儒家的道德体系吧,虽然是孔孟把他造了出来,要不是大一统的中央帝国拿它有用,恐怕早被人忘掉了。现在的知识分子想造道德体系,关上门就可以造。造出来人家用不用,那就是另一回事了,我们当然可以潜心于伦理学,道德哲学,营造一批道德体系,供社会挑选,或是向社会推荐-但是这件事也没见有人干。当年冯定老先生就栽在这上面,所以现在的知识分子都学乖了,只管呼吁不管干,并且善用一种无主句:“要如何如何”。此种句式来源于《圣经。创世记》:“上帝说,要有光,于是有了光”,真是气魄宏伟。上帝的句式,首长用用还差不多。咱们用也就是跟着起哄罢了。 

现在可以说说中国现代知识分子的中古遗风是什么了。他既不像远古的中国知识分子(如孔孟,杨朱,墨子)那样建立道德体系,也不像现代欧美知识分子跨价值观的立论(价值中立)。最爱干的事是拿着已有的道德体系说别人,如前所述,这正是中古的遗风。倒霉的是,在社会的转型时期,已有的道德体系不完备,自己都说不清;于是就哀叹:人心不古,世道浇漓,道德武器船不坚,炮不利,造新船新炮又不敢。其实可以把开船打炮的事交给别人干-但咱们又怕失业。当然,知识分子也是社会的一分子,也该有公民热情,针砭时弊也是知识分子该干的事;不过出于公民热情去做事时,是以公民的身份,而非知识分子的身份,和大家完全平等。这个地位咱们又接受不了,非要有点知识分子特色不可。照我看这个特色就是中古特色。 

三、中国知识分子该不该放弃中古遗风? 

现在中国知识分子在关注社会时,批判找不着目标,颂扬也找不着目标,只一件事找得着目标:呼吁速将大任降给我们,这大任乃是我们维护价值体系的责任,没有它我们就丧失了存在的意义。要论价值体系的形成,从自然地理到生活方式都有一份作用,其功能也是关系到每一个人,维护也好,变革也罢,总不能光知识分子说了算哪。要社会把这份责任全交给你,得有个理由。总不能说我除了这件事之外旁的干不来吧?凭我妙笔生花,词儿多?那就是把别人当傻子了。凭我是个好人?这话人人会说,故而不能认真对待。我知道有人很想说,历史上就是我们负这责任。这不是个道理,历史上男子可以三妻四妾,妇女还裹脚哪,咱们可别讲出这种糊涂油蒙了心的话来找挨骂。再说,拉着历史的车轮逆转,咱们这些人是拉不动的。说来说去,只能说凭我清楚明白。那么我只能凭思维能力来负这份责任,说那些说得清的事;把那些说不清的事,交付公论。现代的欧美知识分子就是这么讨论社会问题:从人类的立场,从科学的立场,从理性的立场,把价值的立场剩给别人。咱们能不能学会? 

最后说说中国知识分子的传统。当然,他有“士”的传统。有人说,他先天下之忧而忧,后天下之乐而乐(悲观主义者?)。有人说,他以天下为己任(国际主义者?),我看都不典型。最典型的是他自以为道德清高(士有百行),地位崇高(四民之首),有资格教训别人(教化于民)。这就是说,我们是这样看自己的。问题是别人怎样看我们。我所见到的事,实属可怜,“脱裤子割尾巴”地混了这么多年,才混到工人阶级队伍里,可谓“心比天高,命比纸薄”!在这种情况下,我建议咱们把“士”的传统忘掉为好,因为不肯忘就是做白日梦了。如果我们讨论社会问题,就讲硬道理:有什么事,我知道,别人还不知道;或者有什么复杂的问题,我想通了,别人想不通;也就是说,按现代的标准来表现知识分子的能力。这样虽然缺少了中国特色,但也未见得不好。

\chapter{理想国与哲人王}

罗素先生评价柏拉图的《理想国》时说,这篇作品有一个蓝本,是斯巴达和它的立法者莱库格斯。我以为,对于柏拉图来说,这是一道绝命杀手。假如《理想国》没有蓝本,起码柏拉图的想象力值得佩服。现在我们只好去佩服莱库格斯,但他是个传说人物,真有假有尚存疑问。由此所得的结论是:《理想国》和它的作者都不值得佩服。当然,到底罗素先生有没有这样阴毒,还可以存疑。罗素又说,无数青年读了这类著作,燃烧起雄心,要做一个莱库格斯或者哲人王。只可惜,对权势的爱好,使人一再误入歧途。顺便说一句,在理想国里,是由哲学家来治国的。倘若是巫师来治国,那些青年就要想做巫师王了。我很喜欢这个论点。我哥哥有一位同学,他在“文化革命”里读了几本哲学书,就穿上了一件蓝布大褂,手里掂着红蓝铅笔,在屋里踱来踱去,看着墙上一幅世界地图,考虑起世界革命的战略问题了。这位兄长大概是想要做世界的哲人王,很显然,他是误入歧途了,因为没听说有哪个中国人做了全世界的哲人王。 

自柏拉图以降,即便不提哲人王,起码也有不少西方知识分子想当莱库格斯。这就是说,想要设计一整套制度、价值观、生活方式,让大家在其中幸福地生活;其中最有名的设计,大概要算摩尔爵士的《乌托邦》。罗素先生对《乌托邦》的评价也很低,主要是讨厌那些繁琐的规定。罗素以为参差多态是幸福的本源,把什么都规定了就无幸福可言。作为经历了某种“乌托邦”的人,我认为这个罪状太过轻微。因为在乌托邦内,对什么是幸福都有规定,比如:“以苦为乐,以苦为荣”,“宁要社会主义的草,不要资本主义的苗”之类。在乌托邦里,很难找到感觉自己不幸福的人,大伙只是傻愣愣的,感觉不大自在。以我个人为例,假如在七十年代,我能说出罗素先生那样充满了智慧的话语,那我对自己的智力状况就很满意,不再抱怨什么。实际上,我除了活着怪没劲之外,什么都说不出来。 

本文的主旨不是劝人不要做莱库格斯或哲人王。照我看,这是个兴趣问题,劝也是没有用的。有些人喜欢这种角色,比如说,我哥哥的那位同学;有人不喜欢这种角色,比如说,我。这是两种不同的人。这两类人凑在一起时,就会起一种很特别的分歧。据说,人脖子上有一道纹路,旧时刽子手砍人,就从这里下刀,可以干净利索地切下脑袋。出于职业习惯,刽子手遇到不认识的人,就要打量他脖子上的纹,想象这个活怎么来做;而被打量的人总是觉得不舒服。我认为,对于敬业的刽子手,提倡出门时戴个墨镜是恰当的,但这已是题外之语。想象几个刽子手在一起互相打量,虽然是很有趣的图景,但不大可能发生,因为谢天谢地,干这行的人绝不会有这么多。我想用刽子手比喻喜欢、并且想当哲人王的人,用被打量的人比喻不喜欢而且反对哲人王的人。这个例子虽然有点不合适,但我也想不到更好的例子。另外,我是写小说的,我的风格是黑色幽默,所以我不觉得举这个例子很不恰当。举这个例子不是想表示我对哲人王深恶痛绝,而是想说明一下“被打量着”是一种什么样的感觉。 

众所周知,哲人王降临人世,是要带来一套新的价值观、伦理准则和生活方式。假如他来了的话,我就没有理由想象自己可以置身于事外。这就意味着我要发生一种脱胎换骨的变化,而要变成个什么,自己却一无所知。如果说还有比死更可怕的事,恐怕就是这个。因为这个原故,知道有人想当哲人王,我就觉得自己被打量着。 

我知道,这哲人王也不是谁想当就能当,他必须是品格高洁之士,而且才高八斗,学富五车。在此我举中国古代的哲人王为例——这只是为了举例方便,毫无影射之意——孔子是圣人,也很有学问。夏礼、周礼他老人家都能言之。但假如他来打量我,我就要抱怨说:甭管您会什么礼,千万别来打量我。再举孟子为例,他老人家善养浩然之气,显然是品行高洁,但我也要抱怨道:您养正气是您的事,打量我干什么?这两位老人家的学养再好,总不能构成侵犯我的理由。特别是,假如学养的目的是要打量人的话,我对这种学养的性质是很有看法的。比方说,朱熹老夫子格物、致知,最后是为了齐家、治国、平天下。因为本人不姓朱,还可以免于被齐,被治和被平总是免不了的。假如这个逻辑可以成立,生活就是很不安全的。很可能在我不知道的地方,有一位我全然不认识的先生在努力地格、致,只要他功夫到家,不管我乐意不乐意,也不管他打算怎样下手,我都要被治和平,而且根本不知自己会被修理成什么模样。 

就我所知,哲人王对人类的打算都在伦理道德方面。倘若他能在物质生活方面替我们打算周到,我倒会更喜欢他。假如能做到,他也不会被称为哲人王,而会被称为科学狂人。实际上,自从有了真正的科学,科学家表现得非常本分。这主要是因为科学就是教人本分的学问,所以根本就没出过这种狂人。至于中国的传统学术,我就不敢这么说。起码我听到过一种说法,叫做“学而优则仕”,当然,若说学了它就会打量人,可能有点过分;但一听说它又出现了新的变种,我就有点紧张。国学主张学以致用,用在谁身上,可以不问自明——当然,这又是题外之语。 

至于题内之语,还是我们为什么要怕哲人王的打量。照我看来,此君的可怕之处首先在于他的宏伟志向:人家考虑的问题是人类的未来,而我们只是人类的几十亿分之一,几乎可以说是不存在。《水浒传》的牢头禁子常对管下人犯说:你这厮只是俺手上的一个行货……一想到哲人王,我心中难免有种行货感。顺便说一句,有些话只有哲人才能说得出来,比如尼采说:到女人那里去不要忘了带上鞭子。我要替女人说上一句:我们招谁惹谁了。至于这类疯话气派很大,我倒是承认的。总的来说,哲人王藐视人类,比牢头禁子有过之无不及。主张信任哲人王的人会说:只有藐视人类的人才能给人类带来更大利益。我又要说:只有这种人才能给人类带来最大的祸害。从常理来说,倘若有人把你当做了nothing,你又怎能信任他们? 

哲人王的又一可怕之处,在于他的学问。在现代社会里,人人都有不懂的学问,科学上的结论不足以使人恐惧,因为这种结论是有证据和推导过程的,对于有理性的人,这些说法是你迟早会同意的那一种。而哲学上的结论就大不相同,有的结论你抵死也不会同意,因为既没有证据也没有推导,哲人王本人就是证明,而结论本身又往往非常的严重。举例来说,尼采先生的结论对一切非受虐狂的女性就很严重;就这句话而论,我倒希望他能活过来,说一句“我是开个玩笑”,然后再死掉。当然,我也盼着中国古代的圣人活过来,把存天理灭人欲、饿死事小失节事大之类的话收回一些。 

我说哲人王的学问可怕,丝毫也不意味着对哲学的不敬。哲学不独有趣,还足以启迪智慧,“文化革命”里工农兵学哲学时说:哲学就是聪明学,我以为并不过分。若以为哲学里种种结论可以搬到生活里使用,恐怕就不尽然。下乡时常听老乡抱怨说:学了聪明学反而更笨,连地都不会种了。至于可以使人成王的哲学,我认为它可以使王者更聪明,老百姓更笨。罗素是个哲学家,他说:真正的伦理准则把人人同等看待。很显然,他的哲学不能使人成王。孔子说:民可使由之,不可使知之。像这样的哲学就能使人(首先是自己)成王。孔丘先生被封为大成至圣先师,子子孙孙都是衍圣公,他老人家果然成了个哲人王。 

时值今日,还有人盼着出个哲人王,给他设计一种理想的生活方式,好到其中去生活;因此就有人乐于做哲人王,只可惜这些现代的哲人王多半不是什么好东西,人民圣殿教的故事就是一例。不但对权势的爱好可以使人误入歧途,服从权势的欲望也可以使人误入歧途。至于我自己,总觉得生活的准则。伦理的基础,都该是些可以自明的东西。假如有未明之处,我也盼望学者贤明的意见,只是这些学者应该像科学上的前辈那样以理服人,或者像苏格拉底那样,和我们进行平等的对话。假如像某些哲人那样讲出些晦涩、偏执的怪理,或者指天划地、口沫飞溅地做出若干武断的规定,那还不如让我自己多想想的好。不管怎么说,我不想把自己的未来交给任何人,尤其是哲人王。

\chapter{救世情结与白日梦}

现在有一种“中华文明将拯救世界”的说法正在一些文化人中悄然兴起,这使我想起了我们年轻时的豪言壮语:我们要解放天下三分之二的受苦人,进而解放全人类。对于多数人来说,不过是说说而已,我倒有过实践这种豪言壮语的机会。七零年,我在云南插队,离边境只有一步之遥,对面就是缅甸,只消步行半天,就可以过去参加缅共游击队。有不少同学已经过去了——我有个同班的女同学就过去了,这对我是个很大的刺激——我也考虑自己要不要过去。过去以后可以解放缅甸的受苦人,然后再去解放三分之二的其他部分;但我又觉得这件事有点不对头。有一夜,我抽了半条春城牌香烟,来考虑要不要过去,最后得出的结论是:不能去。理由是:我不认识这些受苦人,不知道他们在受何种苦,所以就不知道他们是否需要我的解救。尤其重要的是:人家并没有要求我去解放,这样贸然过去,未免自作多情。这样一来,我的理智就战胜了我的感情,没干这件傻事。 

对我年轻时的品行,我的小学老师有句评价:蔫坏。这个坏字我是不承认的,但是“蔫”却是无可否认。我在课堂上从来一言不发,要是提问我,我就翻一阵白眼。像我这样的蔫人都有如此强烈的救世情结,别人就更不必说了。有一些同学到内蒙古去插队,一心要把阶级斗争盖子揭开,解放当地在“内人党”迫害下的人民,搞得老百姓鸡犬不宁。其结果正如我一位同学说的:我们“非常招人恨”。至于到缅甸打仗的女同学,她最不愿提起这件事,一说到缅甸,她就说:不说这个好吗?看来她在缅甸也没解放了谁。看来,不切实际的救世情结对别人毫无益处,但对自己还有点用——有消愁解闷之用。“文化革命”里流传着一首红卫兵诗歌《献给第三次世界大战的勇士》,写两个红卫兵为了解放全世界,打到了美国,“战友”为了掩护“我”,牺牲在“白宫华丽的台阶上”。这当然是瞎浪漫,不能当真:这样随便去攻打人家的总统官邸,势必要遭到美国人民的反对。由此可以得出这样的结论:解放的欲望可以分两种,一种是真解放,比如曼德拉、圣雄甘地、我国的革命先烈,他们是真正为了解放自己的人民而斗争。还有一种假解放,主要是想满足自己的情绪,硬要去解救一些人。这种解放我叫它瞎浪漫。 

对于瞎浪漫,我还能提供一个例子,是我十三岁时的事。当时我堕入了一阵哲学的思辨之中,开始考虑整个宇宙的前途,以及人生的意义,所以就变得本木痴痴;虽然功课还好,但这样子很不讨人喜欢。老师见我这样子,就批评我;见我又不像在听,就掐我几把。这位老师是女的,二十多岁,长得又漂亮,是我单恋的对象,但她又的确掐疼了我。这就使我陷入了爱恨交集之中,于是我就常做种古怪的白日梦,一会儿想象她掉进水里,被我救了出来;一会儿想象她掉到火里,又被我救了出来。我想这梦的前一半说明我恨她,后一半说明我爱她。我想老师还能原谅我的不敬:无论在哪个梦里,她都没被水呛了肺,也没被火烤糊,被我及时地抢救出来了——但我老师本人一定不乐意落入这些危险的境界。为了这种白日梦,我又被她多掐了很多下。我想这是应该的:瞎浪漫的解救,是一种意淫。学生对老师动这种念头,就该掐。针对个人的意淫虽然不雅,但像一回事。针对全世界的意淫,就不知让人说什么好了。 

中国的儒士从来就以解天下于倒悬为己任,也不知是真想解救还是瞎浪漫。五十多年前,梁任公说,整个世界都要靠中国文化的精神去拯救,现在又有人旧话重提。这话和红卫兵的想法其实很相通。只是红卫兵只想动武,所以浪漫起来就冲到白宫门前,读书人有文化,就想到将来全世界变得无序,要靠中华文化来重建全球新秩序。诚然,这世界是有某种可能变得无序——它还有可能被某个小行星撞了呢——然后要靠东方文化来拯救。哪一种可能都是存在的,但是你总想让别人倒霉干啥?无非是要满足你的救世情结嘛。假如天下真的在“倒悬”中,你去解救,是好样的;现在还是正着的,非要在想象中把人家倒挂起来,以便解救之,这就是意淫。我不尊重这种想法。我只尊敬像已故的陈景润前辈那样的人。陈前辈只以解开哥德巴赫猜想为己任,虽然没有最后解决这个问题,但好歹做成了一些事。我自己的理想也就是写些好的小说,这件事我一直在做。李敖先生骂国民党,说他们手淫台湾,意淫大陆,这话我想借用一下,不管这件事我做成做不成,总比终日手淫中华文化,意淫全世界好得多吧。

\chapter{百姓·洋人·官}

小时候,每当得到了一样只能由一人享受的好东西而我们是两个人时,就要做个小游戏来决定谁是幸运者。如你所知,这种把戏叫作“石头、剪子、布”,这三种东西循环相克,你出其中某一样,正好被别人克住,就失败了。这种游戏有个古老的名称,叫作“百姓、洋人、官”,我相信这名称是清末民初流传下来的,当时洋人怕中国的老百姓,中国的官又怕洋人。《官场现形记》写到了不少实例:中国的老百姓人多,和洋人起了争执,就蜂拥而上,先把他臭揍一顿——洋人怕老百姓,是怕吃眼前亏。洋人到了衙门里,开口闭口就是要请本国大使和你们皇上说话,中国的官怕得要死——不但怕洋人,连与洋人有来往的中国人都怕,这种中国人多数是信教的,你到了衙门里,只要说一句“小的是在教的”,官老爷就不敢把你当中国百姓看待,而是要当洋人来巴结。书里有个故事,说一位官老爷听说某人“在教”,就去巴结,拿了猪头三牲到人家的庙里上供,结果被打得稀烂撵了出来——原来是搞错了,人家在的不是洋人的天主教,而是清真古教。 

小说难免有些夸张,但当时有这种现象,倒是无可怀疑。现在完全不同了。洋人在中国,只要不做坏事,就不用怕老百姓。我住的小区里立有一块牌子,写有文明公约,其中有一条,提醒我见了外国人,要“不卑不亢,以礼相待”,人家没有理由怕我。至于我国政府,根本就不怕洋人。在对外交涉中,就是做了些让步,也是合乎道理的。就说保护知识产权罢,盗版软件、盗版VCD,那是偷人家外国的东西;再说市场准入罢,人家外国的市场准你入,你的市场不准人家入,这生意是没法做的。如果说打击国内的盗版商、开放市场就是怕了洋人,肯定是恶意的中伤。还有中国政府在国际事务中的“不出头”政策,这也合乎道理,要出头就要把大票的银子白白交给别人去花,我们舍不得,跟怕洋人没有关系。在这个方面,我完全赞成政府,尤其这最后一条。 

既然情况发生了变化,我再说这些似乎是无的放矢——但我的故事还没讲完呢。无论石头、剪子、布,还是百姓、洋人、官,都是循环相克的游戏。这种古老的游戏还有一个环节是老百姓怕官。这种情况现在应该没有了——现在不是封建社会了,老百姓不该怕官。政府机关也要讲道理、依法办事,你对政府部门有什么意见,既可以反映上去,又可以到检察机关去告——理论上是这样的。但中国是个官本位国家,老百姓见了官,腿肚子就会筛起糠来,底气不足,有民主权利,也不敢享受。对于绝大多数平头百姓来说,情况还是这样。 

最近有本畅销书《中国可以说不》,对我国的对外关系发了些议论。我草草翻了一下,没怎么看进去。现在对这本书有些评论,大多认为书的内容有些偏激。还有人肯定这本书,说是它的意义在于老百姓终于可以说外国人,地位因此提高了。

\chapter{对中国文化的布罗代尔式考证}

萧伯纳是个爱尔兰人,有一次,人家约他写个剧本来弘扬爱尔兰民族精神,他写了《英国佬的另一个岛》,有个剧中人对爱尔兰人的生活态度做了如下描述:“一辈子都在弄他的那片土,那只猪,结果自己也变成了一块土,一只猪……”不知为什么,我看了这段话,脸上也有点热辣辣。这方面我也有些话要说,萧伯纳的态度很能壮我的胆。 

1973年,我到山东老家去插队。有关这个小山村,从小我姥姥已经给我讲过很多,她说这是一个四十多户人家的小山村,全村有一百多条驴。我姥姥还说,驴在当地很有用,因为那里地势崎岖不平,耕地多在山上,所以假如要往地里送点什么,或者从地里收获点什么,驴子都是最重要的帮手。但是我到村里时,发现情况有很大的变化,村里不是四十户人,而是一百多户人,驴子一条都不见了。村里人告诉我说,我姥姥讲的是二十年前的老皇历。这么多年以来,人一直在不停地生出来,至于驴子,在学大寨之前还有几条,后来就没有了。没有驴子以后,人就担负起往地里运输的任务,当然不是用背来驮,而是用小车来推。当地那种独轮车载重比小毛驴驮得还要多些,这样人就比驴有了优越性。在所有的任务里,最繁重的是要往地里送粪”——其实那种粪里土的成分很大——一车粪大概有三百斤到四百斤的样子,而地往往在比村子高出二三百米的地方。这就是说,要把二百公斤左右的东西送到80层楼上,而且早上天刚亮到吃早饭之前要往返十趟。说实在话,我对这任务的艰巨性估计不足。我以为自己长得人高马大,在此之前又插过三年队,别人能干的事,我也该能干,结果才推了几趟,我就满嘴是胆汁的味道。推了两天,我从城里带来的两双布鞋的后跟都被豁开了,而且小腿上的肌肉总在一刻不停的震颤之中。后来我只好很丢脸地接受了一点照顾,和一些身体不好的人一道在平地上干活。好在当地人没有因此看不起我,他们还说,像我这样初来乍到的人,能把这种工作坚持到三天之上,实在是不容易。就连他们这些干惯了的人都觉得这种工作太过辛苦,能够歇上一两天,都觉得是莫大的幸福。 

时隔二十年,我把这件事仔细考虑了一遍,得到的一个结论是这样的:用人来取代驴子往地里送粪,其实很不上算。因为不管人也好,驴也罢,送粪所做的功都是一样多,我们(人和驴)都需要能量补充,人必须要吃粮食,而驴子可以吃草;草和粮食的价值大不相同。事实上,一个人在干推粪这种活和干别的活时相比,食量将有一个很可观的增长,这就导致了粮食不够吃,所以不得不吃下一大批白薯干。白薯干比之正经粮食便宜了很多,但在集市上也要卖到两毛钱一斤;而在集市上,最好的草(可以苫房顶)是三分钱一斤,一般做饲料的草顶多值两分钱。我不认为自己在吃下一斤白薯干之后,可以和吃了十斤干草的驴比赛负重,而且白薯干还异常难吃,噎人,难消化,容易导致胃溃疡;而驴在吃草时肯定不会遇到同样的困难。在此必须强调指出,此种白薯干是生着切片晾的,假设是煮熟了晾出的那种甜甜的东西,就绝不止两毛钱一斤。有关白薯干的情况,还可以补充几句,它一进到了食道里就会往上蹦,不管你把它做成发糕还是面条,只要不用大量的粮食来冲淡,都有同样的效果。因此我曾设想改进一下进食的方式,拿着大顶来吃饭,这样它往上一蹦就正好进到胃里,省得我痛苦地向下咽,但是我没有试验过,我怕被别人看到后难以解释。白薯干原来是猪的口粮,这种可怜的动物后来就改吃人屙的屎。据我在厕所兼猪圈里的观察。它们一遇到吃薯干屙出的屎,就表现出愤怒之状,这曾使我在出恭时良心大感痛苦——这个话题就说到这里为止。由此可见,我姥姥在村里时,四十户人家、一百多条驴是符合经济规律的。当然,我在村里时,一百多户人家没有驴,也符合经济规律。前者符合省钱的规律,后者符合就业的规律。只有“一百户人家加一百条驴”不符合经济规律,因为没有那么多的事可做。于是,驴子就消失了。有关这件事,可以举出一件恰当的反例:在英国产业革命前夕,有过一次圈地运动,英国农民认为这是“羊吃人”;而在我的老家则是人吃驴,而且是货真价实的吃。村里人说,有一阵子老是吃驴肉,但我去晚了没赶上,只赶上了吃白薯干。当然,在这场人和驴的生存竞争中,我当时坚定地站在人这一方,认为人有吃掉驴子的权利。 

最近我读到布罗代尔先生的《十五至十八世纪的物质文明、经济和资本主义》,才发现这种生存竞争不光是在我老家存在,也不限于在人和驴之间,更不限于本世纪七十年代,它是一种广泛存在的历史事实。十六世纪到中国来的传教士就发现,与西欧相比,中国的役畜非常少,对水力和风力的利用也不充分。这就是说,此种生存竞争不光在人畜之间存在,还存在于人与浩浩荡荡的自然力之间。这次我就不能再站在人的立场上反对水和风了,因为这种对手过于低级,胜之不武。而且我以为,中国的文化传统里,大概是有点问题。众所周知,我们国家的传统文化是一种人本的文化,但是它和西方近代的人本主义完全不同。在我们的文化里,只认为生命是好的,却没把快乐啦、幸福啦、生存状态之类的事定义在内;故而就认为,只要大家都能活着就好,不管他们活得多么糟糕。由此导致了一种古怪的生存竞争,和风力,水力比赛推动磨盘,和牲口比赛运输——而且是比赛一种负面的能力,比赛谁更不知劳苦,更不贪图安逸! 

中国史学界没有个年鉴学派,没有人考证一下历史上的物质生活,这实在是一种遗憾——布罗代尔对中国物质生活的描述还是不够详尽——这件事其实很有研究的必要。在中国人口稠密的地带,根本就见不到风车、水车,这种东西只在边远地方有。我们村里有盘碾子,原来是用驴子拉的,驴没了以后改用人来推。驴拉碾时需要把眼蒙住,以防它头晕。人推时不蒙眼,因为大家觉得这像一头驴,不好意思。其实人也会晕。我的切身体会是:人只有两条腿,因为这种令人遗憾的事实,所以晕起来站都站不住。我还听到一个真实的故事,陈永贵大叔在大寨曾和一头驴子比赛负重,驴子摔倒,永贵大叔赢了。我认为,那头驴多半是个小毛驴,而非关中大叫驴。后一种驴子体态壮硕,恐非人类所能匹敌——不管是哪一种驴,这都是一个伟大的胜利,证明了就是不借手推车,人也比驴强。我认识的一位中学老师曾经用客观的态度给学生讲过这个故事(未加褒贬),结果在“文化革命”里被斗得要死。这最后一件事多少暗示出中国为什么没有年鉴学派。假如布罗代尔是中国人,写了一本有关中国农村物质生活的书,人和驴比赛负重的故事他是一定要引用的,白纸黑字写了出来,“文化革命”这一关他绝过不去。虽然没有年鉴学派那样缜密的考证,但我也得出了结论:在现代物质文明的影响到来之前,在物质生活方面有这么一种倾向,不是人来驾驭自然力、兽力;这就要求人能够吃苦耐劳、本分。当然,这种要求和传统文化对人的教诲甚是合拍,不过孰因孰果很难说明白。我认为自己在插队时遭遇的一切,是传统社会物质文明发展规律走到极端所致。 

在人与兽、人与自然力的竞争中,人这一方的先天条件并不好。如前所述,我们不像驴子那样有四条腿、可以吃草,也不像风和水那样浑然无觉,不知疲倦。好在人还有一种强大的武器,那就是他的智能、他的思索能力。假如把它对准自然界,也许人就能过得好一点。但是我们把枪口对准了自己,发明了种种消极的伦理道德,其中就包括了吃大苦、耐大劳,“存天理、灭人欲”;而苦和累这两种东西,正如莎翁笔下的爱情,你吃下的越多,它就越有,“所以两者都是无穷无尽的了!”(引自《罗米欧与朱丽叶》) 

这篇文章写到了这里,到了得出结论的时候了。我认为中国文化对于物质生活的困苦,提倡一种消极忍耐的态度,不提倡用脑子想,提倡用肩膀扛;结果不但是人,连驴和猪都深受其害。假设一切现实生活中的不满意、不方便,都能成为严重的问题,使大家十分关注,恐怕也不至于搞成这个样子,因为我们毕竟是些聪明人。虽然中国人是如此的聪明,但是布罗代尔对十七世纪中国的物质生活(包括北京城里有多少人靠拣破烂为生)做了一番描述之后下结论道:在这一切的背后,“潜在的贫困无处不在”。我们的祖先怎么感觉不出来?我的结论是:大概是觉得那么活着就不坏吧。

\chapter{人性的逆转}

有位西方的发展学者说:贫穷是一种生活方式。言下之意是说,有些人受穷,是因为他不想富裕。这句话是作为一种惊世骇俗的观点提出的,但我狭隘的人生经历却证明此话大有道理。对于这句话还可以充分地推广:贫困是一种生活方式,富裕是另一种生活方式;追求聪明是一种人生的态度,追求愚蠢则是另一种生活态度。在这个世界上,有一些人在追求快乐,另一些人在追求痛苦;有些人在追求聪明,另一些人在追求愚蠢。这种情形常常能把人彻底搞糊涂。 

洛克先生以为,人人都追求快乐,这是不言自明的。以此为基础,他建立了自己的哲学大厦。斯宾诺莎也说,人类行为的原动力是自我保存。作为一个非专业的读者,我认为这是同一类的东西,认为人趋利而避害,趋乐而避苦,这是伦理学的根基。以此为基础,一切都很明白。相比之下,我们民族的文化传统大不相同,认为礼高于利,义又高于生,这样就创造了一种比较复杂的伦理学。由此产生了一个矛盾,到底该从利害的角度来定义崇高,还是另有一种先验的东西,叫做崇高——举例来说,孟子认为,人皆有恻隐之心,这是人先天的良知良能,这就是崇高的根基。我也不怕人说我是民族虚无主义,反正我以为前一种想法更对。从前一种想法里产生富裕,从后一种想法里产生贫困;从前一种想法里产生的总是快乐,从后一种想法里产生的总是痛苦。我坚定不移地认为,前一种想法就叫做聪明,后一种想法就叫做愚蠢。笔者在大学里学的是理科,凭这样的学问底子,自然难以和专业哲学家理论,但我还是以为,这些话不能不说。 

对于人人都追求快乐这个不言自明的道理罗素却以为不尽然,他举受虐狂作为反例。当然,受虐狂在人口中只占极少数。但是受虐却不是罕见的品行。七十年代,笔者在农村插队,在学大寨的口号鞭策下,劳动的强度早已超过了人力所能忍受的极限,但那些工作却是一点价值也没有的。对于这些活计,老乡们概括得最对:没别的,就是要给人找些罪来受。但队干部和积极分子们却乐此不疲,干得起码是不比别人少。学大寨的结果是使大家变得更加贫穷。道理很简单:人干了艰苦的工作之后,就变得很能吃,而地里又没有多长出任何可吃的东西。这个例子说明,人人都有所追求,这个道理是不错的,但追求的却可以是任何东西:你总不好说任何东西都是快乐吧。 

人应该追求智慧,这对西方人来说是很容易接受的道理;苏格拉底甚至把求知和行善画上了等号。但是中国人却说“难得糊涂”,仿佛是希望自己变得笨一点。在我身上,追求智慧的冲动比追求快乐的冲动还要强烈,因为这个原故,在我年轻时,总是个问题青年、思想改造的重点对象。我是这么理解这件事的:别人希望我变得笨一些。谢天谢地,他们没有成功。人应该改变自己,变成某种样子,这大概是没有疑问的。有疑问的只是应该变聪明还是变笨。像这样的问题还能举出一大堆,比方说,人(尤其是女人)应该更漂亮、更性感一些,还是更难看、让人倒胃一些;对别人应该更粗暴、更野蛮一些,还是更有礼貌一些;等等。假如你经历过中国的七十年代,就会明白,在生活的每一个方面,都有不同的答案。你也许会说,每个国家都有自己的国情,每个时代都有自己的风尚,但我对这种话从来就不信。我更相信乔治·奥威尔的话:一切的关键就在于必须承认一加一等于二;弄明白了这一点,其他一切全会迎刃而解。 

我相信洛克的理论。人活在世上,趋利趋乐暂且不说,首先是应该避苦避害。这种信念来自我的人生经验:我年轻时在插队,南方北方都插过。谁要是有同样的经历就会同意,为了谋生,人所面临的最大任务是必须搬动大量沉重的物质:这些物质有时是水,有时是粪土,有时是建筑材料,等等。到七十年代中期为止,在中国南方,解决前述问题的基本答案是:一根扁担。在中国的北方则是一辆小车。我本人以为,这两个方案都愚不可及。在前一个方案之下,自肩膀至脚跟,你的每一寸肌肉、每一寸骨骼都在百十公斤重物的压迫之下,会给你带来腰疼病、腿疼病。后一种方案比前种方案强点不多,虽然车轮承担了重负,但车上的重物也因此更多。假如是往山上推的话,比挑着还要命。西方早就有人在解决这类问题,先有阿基米德,后有牛顿。卡特,所以在一二百年前就把这问题解决了。而在我们中国,到现在也没解决。你或者会以为,西方文明有这么一点小长处,善于解决这种问题,但我以为这是不对的。主要的因素是感情问题。、西方人以为,人的主要情感源于自身,所以就重视解决肉体的痛苦。中国人以为,人的主要情感是亲亲敬长,就不重视这种问题。这两种想法哪种更对?当然是前者。现在还有人说,西方人纲常败坏,过着痛苦的生活——这种说法是昧良心的。西方生活我见过,东方的生活我也见过。西方人儿女可能会吸毒,婚姻可能会破裂,总不会早上吃两片白薯干,中午吃两片白薯干,晚上再吃两片白薯干,就去挑一天担子,推一天的重车!从孔孟到如今,中国的哲学家从来不挑担、不推车。所以他们的智慧从不考虑降低肉体的痛苦,专门营造站着说话不腰疼的理论。 

在西方人看来,人所受的苦和累可以减少,这是一切的基础。假设某人做出一份牺牲,可以给自己或他人带来很多幸福,这就是崇高——洛克就是这么说的。孟子不是这么说,他的崇高另有根基,远不像洛克的理论那么能服人。据我所知,孟子远不是个笨蛋。除了良知良能,他还另有说法。他说反对他意见的人(杨朱、墨子)都是禽兽。由此得出了崇高的定义:有种东西,我们说它是崇高,是因为反对它的人都不崇高。这个定义一直沿用到了如今。细想起来,我觉得这是一种模糊不清的混蛋逻辑,还不如直说凡不同意我意见者都是王八蛋为好。总而言之,这种古怪的论证方式时常可以碰到。 

在七十年代,发生了这样一回事:河里发大水,冲走了一根国家的电线杆。有位知青下水去追,电杆没捞上来,人也淹死了。这位知青受到表彰,成了革命烈士。这件事引起了一点小小的困惑:我们知青的一条命,到底抵不抵得上一根木头?结果是困惑的人惨遭批判,结论是:国家的一根稻草落下水也要去追。至于说知青的命比不上一根稻草,人家也没这么说。他们只说,算计自己的命值点什么,这种想法本身就不崇高。坦白地说,我就是困惑者之一。现在有种说法,以为民族的和传统的就是崇高的。我知道它的论据:因为反民族和反传统的人很不崇高。但这种论点吓不倒我。 

过去欧洲有个小岛,岛上是苦役犯服刑之处。犯人每天的工作是从岛东面挑起满满的一挑水,走过崎岖的山道,到岛西面倒掉。这岛的东面是地中海,水从地中海里汲来。西面也是地中海,这担水还要倒回地中海去。既然都是地中海,所以是通着的。我想,倒在西面的水最终还要流回东面去。无价值的吃苦和无代价的牺牲大体就是这样的事。有人会说,这种劳动并非毫无意义,可以陶冶犯人的情操、提升犯人的灵魂;而有些人会立刻表示赞成,这些人就是那些岛上的犯人——我听说这岛上的看守手里拿着鞭子,很会打人。根据我对人性的理解,就是离开了那座岛屿,也有人会保持这种观点。假如不是这样,劳动改造就没有收到效果。在这种情况下,人性就被逆转了。 

从这个例子来看,要逆转人性,必须有两个因素:无价值的劳动和暴力的威胁,两个因素缺一不可。人性被逆转之后,他也就糊涂了。费这么大劲把人搞糊涂有什么好处,我就不知道,但想必是有的,否则不会有这么个岛。细想起来,我们民族的传统文化里就包含了这种东西。举个例子来说,朝廷的礼节。见皇上要三磕九叩、扬尘舞蹈,这套把戏耍起来很吃力,而且不会带来任何收益,显然是种无代价的劳动。但皇上可以廷杖臣子,不老实的马上拉下去打板子。有了这两个因素,这套把戏就可以耍下去,把封建士大夫的脑子搞得很糊涂。回想七十年代,当时学大寨和抓阶级斗争总是一块搞的,这样两个因素就凑齐了。我下乡时,和父老乡亲们在一起。我很爱他们,但也不能不说:他们早就被逆转了。我经历了这一切,脑子还是不糊涂,还知道一加一等于二,这只说明一件事:要逆转人性,还要有第三个因素,那就是人性的脆弱。 

我认为七十年代是我们宝贵的精神财富,这个看法和一些同龄人是一样的。七十年代的青年和现在的青年很不一样,更热情、更单纯、更守纪律、对生活的要求更低,而且更加倒霉。成为这些人中的一员,是一种极难得的际遇,这些感受和别人是一样的。有些人认为这种经历是一种崇高的感受,我就断然反对,而且认为这种想法是病态的。让我们像奥威尔一样,想想什么是一加一等于二,七十年代对于大多数中国人来说,是个极痛苦的年代。很多年轻人做出了巨大的自我牺牲,而且这种牺牲毫无价值。想清楚了这些事,我们再来谈谈崇高的问题。就七十年代这个例子来说,我认为崇高有两种:一种是当时的崇高,领导上号召我们到农村去吃苦,说这是一种光荣。还有一种崇高是现在的崇高,忍受了这些痛苦、做出了自我牺牲之后,我们自己觉得这是崇高的。我觉得这后一种崇高比较容易讲清楚。弗洛伊德对受虐狂有如下的解释:假如人生活在一种无力改变的痛苦之中,就会转而爱上这种痛苦,把它视为一种快乐,以便使自己好过一些。对这个道理稍加推广,就会想到:人是一种会自己骗自己的动物。我们吃了很多无益的苦,虚掷了不少年华,所以有人就想说,这种经历是崇高的。这种想法可以使他自己好过一些,所以它有些好作用。很不幸的是它还有些坏作用:有些人就据此认为,人必须吃一些无益的苦、虚掷一些年华,用这种方法来达到崇高。这种想法不仅有害,而且是有病。 

说到吃苦、牺牲,我认为它是负面的事件。吃苦必须有收益,牺牲必须有代价,这些都属一加一等于二的范畴。我个人认为,我在七十年代吃的苦、做出的牺牲是无价值的,所以这种经历谈不上崇高;这不是为了贬低自己,而是为了对现在和未来发生的事件有个清醒的评价。逻辑学家指出,从正确的前提能够推导出正确的结论,但从一个错误的前提就什么都能够推导出来。把无价值的牺牲看作崇高,也就是接受了一个错误的前提。此后你就会什么鬼话都能说出口来,什么不可信的事都肯信——这种状态正确的称呼叫做“糊涂”。人的本性是不喜欢犯错误的,所以想把他搞糊涂,就必须让他吃很多的苦——所以糊涂也很难得呀。因为人性不总是那么脆弱,所以糊涂才难得。经过了七十年代,有些人对人世间的把戏看得更清楚,他就是变得更聪明。有些人对人世间的把戏更看不懂了,他就是变得更糊涂。不管发生了哪种情况,七十年代都是我们的宝贵财富。 

我要说出我的结论,中国人一直生活在一种有害哲学的影响之下,孔孟程朱编出了这套东西,完全是因为他们在社会的上层生活。假如从整个人类来考虑问题,早就会发现,趋利避害,直截了当地解决实际问题最重要——说实话,中国人在这方面已经很不像样了——这不是什么哲学的思辨,而是我的生活经验。我们的社会里,必须有改变物质生活的原动力,这样才能把未来的命脉握在自己的手里。

\chapter{肚子里的战争}

我年轻时,有一回得了病,住进了医院。当时医院里没有大夫,都是工农兵出身的卫生员——真正的大夫全都下到各队去接受贫下中农再教育去了。话虽如此说,穿着白大褂的,不叫他大夫又能叫什么呢。我入院第一天,大夫来查房,看过我的化验单,又拿听诊器把我上下听了一遍,最后还是开口来问:你得了什么病。原来那张化验单他没看懂。其实不用化验单也能看出我的病来:我浑身上下像隔夜的茶水一样的颜色,正在闹黄疸。我告诉他,据我自己的估计,大概是得了肝炎。这事发生在二十多年前,当时还没听说有乙肝,更没有听说丙肝丁肝和戊肝,只有一种传染性肝炎。据说这一种肝炎中国原来也没有,还是三年困难时吃伊拉克蜜枣吃出来的——叫做蜜枣,其实是椰枣。我虽没吃椰枣,也得了这种病。大夫问我该怎么办,我说你给我点维生素吧——我的病就是这么治的。说句实在话,住院对我的病情毫无帮助。但我自己觉得还是住在医院里好些,住在队里会传染别人。 

在医院里没有别的消遣,只有看大夫们给人开刀。这一刀总是开向阑尾——应该说他们心里还有点数,知道别的手术做不了。我说看开刀可不是瞎说的,当地经常没有电,有电时电压也极不稳,手术室是四面全是玻璃窗的房子,下午两点钟阳光最好,就是那时动手术——全院的病人都在外面看着,互相打赌说几个小时找到阑尾。后来我和学医的朋友说起此事,他们都不信,说阑尾手术还能动几个钟头?不管你信也好,不信也罢,我看到的几个手术没有一次在一小时之内找着阑尾的。做手术的都说,人的盲肠太难找——他们中间有好几位是部队骡马卫生员出身,参加过给军马的手术,马的盲肠就很大,骡子的盲肠也不小,哪个的盲肠都比人的大,就是把人个子小考虑在内之后,他的盲肠还是太小。闲着没事聊天时,我对他们说:你们对人的下水不熟悉,就别给人开刀了。你猜他们怎么说?“越是不熟就越是要动——在战争中学习战争!”现在的年轻人可能不知道,这后半句是毛主席语录。人的肠子和战争不是一码事,但这话就没人说了。我觉得有件事情最可恶:每次手术他们都让个生手来做,以便大家都有机会学习战争,所以阑尾总是找不着。刀口开在什么部位,开多大也完全凭个人的兴趣。但我必须说他们一句好话:虽然有些刀口偏左,有些刀口偏右,还有一些开在中央,但所有的刀口都开在了肚子上,这实属难能可贵。 

我在医院里遇上一个哥们,他犯了阑尾炎,大夫动员他开刀。我劝他千万别开刀——万一非开不可,就要求让我给他开。虽然我也没学过医,但修好过一个闹钟,还修好了队里一台手摇电话机。就凭这两样,怎么也比医院里这些大夫强。但他还是让别人给开了,主要是因为别人要在战争里学习战争,怎么能不答应。也是他倒霉,打开肚子以后,找了三个小时也没找到阑尾,急得主刀大夫把他的肠子都拿了出来,上下一通紧倒。小时候我家附近有家小饭铺,卖炒肝、烩肠,清晨时分厨师在门外洗猪大肠,就是这么一种景象。眼看天色越来越暗,别人也动手来找,就有点七手八脚。我的哥们被人找得不耐烦,撩开了中间的白布帘子,也去帮着找。最后终于在太阳下山以前找到,把它割下来,天也就黑了,要是再迟一步,天黑了看不见,就得开着膛晾一宿。原来我最爱吃猪大肠;自从看过这个手术,再也不想吃了。 

时隔近三十年,忽然间我想起了住院看别人手术的事,主要是有感于当时的人浑浑噩噩,简直是在发疯。谁知道呢,也许再过三十年,再看今天的人和事,也会发现有些人也是在发疯。如此看来,我们的理性每隔三十年就有一次质的飞跃——但我怀疑这么理解是不对的。理性可以这样飞越,等于说当初的人根本没有理性。就说三十年前的事吧,那位主刀的大叔用漆黑的大手捏着活人的肠子上下倒腾时,虽然他说自己在学习战争,但我就不信他不知道自己是在胡闹。由此就得到一个结论:一切人间的荒唐事,整个社会的环境虽是一个原因,但不主要。主要的是:那个闹事的人是在借酒撒疯。这就是说,他明知道自己在胡闹,但还要闹下去,主要是因为胡闹很开心。 

我们还可以得到进一步的推论:不管社会怎样,个人要为自己的行为负责——但作为杂文的作者,把推论都写了出来,未免有直露之嫌,所以到此打住。住医院的事我还没写完呢:我在医院里住着,肝炎一点都不见好,脸色越来越黄;我的哥们动了手术,刀口也总是长不上,人也越来越瘦。后来我们就结伴回北京来看病。我一回来病就好了,我的哥们却进了医院,又开了一次刀。北京的大夫说,上一次虽把阑尾割掉了,但肠子没有缝住,粘到刀口上成了一个瘘,肠子里的东西顺着刀口往外冒,所以刀口老不好。大夫还说,冒到外面还是万分幸运,冒到肚子里面,人就完蛋了。我哥们倒不觉得有什么幸运,他只是说:妈的,怪不得总吃不饱,原来都漏掉了。这位兄弟是个很豪迈的人,如果不是这样,也不会拿自己的内脏给别人学习战争。

\chapter{椰子树与平等} 

二十多年前,我在云南插队。当地气候炎热,出产各种热带水果,就是没有椰子。整个云南都不长椰子,根据野史记载,这其中有个缘故。据说,在三国以前,云南到处都是椰子,树下住着幸福的少数民族。众所周知,椰子有很多用处,椰茸可以当饭吃,椰子油也可食用。椰子树叶里的纤维可以织粗糙的衣裙,椰子树干是木材。这种树木可以满足人的大部分需要,当地人也就不事农耕,过着悠闲的生活。忽一日,诸葛亮南征来到此地,他要教化当地人,让他们遵从我们的生活方式:干我们的活,穿我们的衣服,服从我们的制度。这件事起初不大成功,当地人没看出我们的生活方式有什么优越之处。首先,秋收春种,活得很累,起码比摘椰子要累;其次,汉族人的衣着在当地也不适用。就以诸葛先生为例,那身道袍料子虽好,穿在身上除了捂汗和捂痱子,捂不出别的来;至于那顶道冠,既不遮阳,也不挡雨,只能招马蜂进去做窝。当地天热,摘两片椰树叶把羞处遮遮就可以了。至于汉朝的政治制度,对当地的少数民族来说,未免太过烦琐。诸葛先生磨破了嘴皮子,言必称孔孟,但也没人听。他不觉得自己的道理不对,却把帐算在了椰子树身上:下了一道命令,一夜之间就把云南的椰树砍了个精光;免得这些蛮夷之人听不进圣贤的道理。没了这些树,他说话就有人听了——对此,我的解释是,诸葛亮他老人家南征,可不是一个人去的,还带了好多的兵,砍树用的刀斧也可以用来砍人,砍树这件事说明他手下的人手够用,刀斧也够用。当地人明白了这个意思,就怕了诸葛先生。我这种看法你尽可以不同意——我知道你会说,诸葛亮乃古之贤人,不会这样赤裸裸地用武力威胁别人;所以,我也不想坚持这种观点。 

对于此事,野史上是这么解释的:蛮夷之人,有些稀奇之物,就此轻狂,胆敢渺视天朝大邦;没了这些珍稀之物,他们就老实了。这就是说,云南人当时犯有轻狂的毛病,这是一种道德缺陷。诸葛先生砍树,是为了纠正这种毛病,是为他们好。我总觉得这种说法有点太过惊世骇俗。人家有几样好东西,活得好一点,心情也好一点,这就是轻狂;非得把这些好东西毁了,让人家心情沉痛,这就是不轻狂——我以为这是野史作者的意见,诸葛先生不是这样的人。 

野史是不能当真的,但云南现在确实没有椰子,而过去是有的。所以这些椰树可能是诸葛亮砍的。假如这不是耍野蛮,就该有种道义上的解释。我觉得诸葛亮砍椰子时,可能是这么想的:人人理应生来平等,但现在不平等了:四川不长椰树,那里的人要靠农耕为生;云南长满了椰树,这里的人就活得很舒服。让四川也长满椰树,这是一种达到公平的方法,但是限于自然条件,很难做到。所以,必需把云南的椰树砍掉,这样才公平。假如有不平等,有两种方式可以拉平:一种是向上拉平,这是最好的,但实行起来有困难;比如,有些人生来四肢健全,有些人则生有残疾,一种平等之道是把所有的残疾人都治成正常人,这可不容易做到。另一种是向下拉平,要把所有的正常人都变成残疾人就很容易:只消用铁棍一敲,一声惨叫,这就变过来了。诸葛先生采取的是向下拉平之道,结果就害得我吃不上椰子。在云南时,我觉得嘴淡时就啃几个木瓜。木瓜淡而无味,假如没熟透,啃后满嘴都是麻的。但我没有抱怨木瓜树,这种树内地也是不长的。假如它的果子太好吃,诸葛先生也会把它砍光啦。 

我这篇文章题目在说椰子,实质在谈平等问题,挂羊头卖狗肉,正是我的用意。人人理应生来平等,这一点人人都同意。但实际上是不平等的,而且最大的不平等不是有人有椰子树,有人没有椰子树。如罗素先生所说,最大的不平等是知识的差异——有人聪明有人笨,这就是问题之所在。这里所说的知识、聪明是广义的,不单包括科学知识,还包括文化素质,艺术的品味,等等。这种椰子树长在人脑里,不光能给人带来物质福利,还有精神上的幸福;这后一方面的差异我把它称为幸福能力的差异。有些作品,有些人能欣赏,有些人就看不懂,这就是说,有些人的幸福能力较为优越。这种优越最招人嫉妒。消除这种优越的方法之一就是给聪明人头上一闷棍,把他打笨些。但打轻了不管用,打重了会把脑子打出来,这又不是我们的本意。另一种方法则是:一旦聪明人和傻人起了争执,我们总说傻人有理。久而久之,聪明人也会变傻。这种法子现在正用着呢。

\chapter{体验生活} 

我靠写作为生。有人对我说:像你这样写是不行的啊,你没有生活!起初,我以为他想说我是个死人,感到很气愤。忽而想到,“生活”两字还有另一种用法。有些作家常到边远艰苦的地方去住上一段,这种出行被叫做“体验生活”——从字面上看,好像是死人在诈尸,实际上不是的。这是为了对艰苦的生活有点了解,写出更好的作品,这是很好的做法。人家说的生活,是后面一种用法,不是说我要死,想到了这一点,我又回嗔作喜。我虽在贫困地区插过队,但不认为体验得够了。我还差得很远,还需要进一步的体验。但我总觉得,这叫做“体验艰苦生活”比较好。省略了中间两个字,就隐含着这样的意思:生活就是要经常吃点苦头——有专门从负面理解生活的嫌疑。和我同龄的人都有过忆苦思甜的经历:听忆苦报告、吃忆苦饭,等等。这件事和体验生活不是一回事,但意思有点相近。众所周知,旧社会穷人过着牛马不如的生活,吃糠咽菜——菜不是蔬菜,而是野菜。所谓忆苦饭,就是旧社会穷人饭食的模仿品。  

我要说的忆苦饭是在云南插队时吃到的——为了配合某种形势,各队起码要吃一顿忆苦饭,上面就是这样布置的。我当时是个病号,不下大田,在后勤做事,归司务长领导,参加了做这顿饭。当然,我只是下手。真正的大厨是我们的司务长。这位大叔朴实木讷,自从他当司务长,我们队里的伙食就变得糟得很,每顿都吃烂菜叶——因为他说,这些菜太老,不吃就要坏了。菜园子总有点垂垂老矣的菜,吃掉旧的,新的又老了,所以永远也吃不到嫩菜。我以为他炮制忆苦饭肯定很在行,但他还去征求了一下群众意见,问大家在旧社会吃过些啥。有人说,吃过芭蕉树心,有人说,吃过芋头花、南瓜花。总的来说,都不是什么太难吃的东西,尤其是芋头花,那是一种极好的蔬菜,煮了以后香气扑鼻。我想有人可能吃过些更难吃的东西,但不敢告诉他。说实在的,把饭弄好吃的本领他没有,弄难吃的本领却是有的。再教教就更坏了。就说芭蕉树心吧,本该剥出中间白色细细一段,但他叫我砍了一棵芭蕉树来,斩碎了整个煮进了锅里。那锅水马上变得黄里透绿,冒起泡来,像锅肥皂水,散发着令人恶心的苦味……  

我说过,这顿饭里该有点芋头花。但芋头不大爱开花,所以煮的是芋头秆,而且是刨了芋头剩下的老秆。可能这东西本来就麻,也可能是和芭蕉起了化学反应,总之,这东西下锅后,里面冒出一种很恶劣的麻味。大概你也猜出来了,我们没煮南瓜花,煮的是南瓜藤,这种东西斩碎后是些煮不烂的毛毛虫。最后该搁点糠进去,此时我和司务长起了严重的争执。我认为,稻谷的内膜才叫做糠。这种东西我们有,是喂猪的。至于稻谷的外壳,它不是糠,猪都不吃,只能烧掉。司务长倒不反对我的定义,但他说,反正是忆苦饭,这么讲究干什么,糠还要留着喂猪,所以往锅里倒了一筐碎稻壳。搅匀之后,真不知锅里是什么。做好了这锅东西,司务长高兴地吹起了口哨,但我的心情不大好。说实在的,我这辈子没怕过什么,那回也没有怕,只是心里有点慌。我喂过猪,知道拿这种东西去喂猪,所有的猪都会想要咬死我。猪是这样,人呢?  

后来的事情证明我是瞎操心。晚上吃忆苦饭,指导员带队,先唱“天上布满星”,然后开饭。有了这种气氛,同学们见了饭食没有活撕了我,只是有些愣头青对我怒目而视,时不常吼上一句:“你丫也吃!”结果我就吃了不少。第一口最难,吃上几口后满嘴都是麻的,也说不上有多难吃。只是那些碎稻壳像刀片一样,很难吞咽,吞多了嘴里就出了血。反正我已经抱定了必死的决心,自然没有闯不过去的关口。但别人却在偷偷地干呕。吃完以后,指导员做了总结,看样子他的情况不大好,所以也没多说。然后大家回去睡觉——但是事情当然还没完。大约是夜里十一点,我觉得肠胃搅痛,起床时,发现同屋几个人都在地上摸鞋。摸来摸去,谁也没有摸到,大家一起赤脚跑了出去,奔向厕所,在北回归线那皎洁的月色下,看到厕所门口排起了长队……  

有件事需要说明,有些不文明的人有放野尿的习惯,我们那里的人却没有。这是因为屎有做肥料的价值,不能随便扔掉。但是那一夜不同,因为厕所里没有空位,大量这种宝贵的资源被抛撒在厕所后的小河边。干完这件不登大雅之事,我们本来该回去睡觉,但是走不了几步又想回来,所以我们索性坐在了小桥上,聊着天,挨着蚊子咬,时不常地到草丛里去一趟。直到肚子完全出清。到了第二天,我们队的人脸色都有点绿,下巴有点尖,走路也有点打晃。像这个样子当然不能下地,只好放一天假。这个故事应该有个寓意,我还没想出来。反正我不觉得这是在受教育,只觉得是折腾人——虽然它也是一种生活。总的来说,人要想受罪,实在很容易,在家里也可以拿头往门框上碰。既然痛苦是这样简便易寻,所以似乎用不着特别去体验。 

\chapter{关于崇高}

七十年代发生了这样一回事:河里发大水,冲走了一根国家的电线杆。有位知青下水去追,电线杆没捞上来,人却淹死了。这位知青受到表彰,成了革命烈士。这件事在知青中间引起了一点小小的困惑:我们的一条命,到底抵不抵得上一根木头?结果是困惑的人惨遭批判,不瞒你说,我本人就是困惑者之一,所以对这件事记忆犹新。照我看来,我们吃了很多年的饭才长到这么大,价值肯定比一根木头高;拿我们去换木头是不值的。但人家告诉我说:国家财产是大义之所在,见到它被水冲走,连想都不要想,就要下水去捞。不要说是木头,就是根稻草,也得跳下水。他们还说,我这种值不值的论调是种落后言论——幸好还没有说我反动。  

实际上,我在年轻时是个标准的愣头青,水性也好。见到大水冲走了木头,第一个跳下水的准是我,假如水势太大,我也可能被淹死,成为烈士,因为我毕竟还不是鸭子。这就是说,我并不缺少崇高的气质,我只是不会唱那些高调。时隔二十多年,我也读了一些书,从书本知识和亲身经历之中,我得到了这样一种结论:自打孔孟到如今,我们这个社会里只有两种人。一种编写生活的脚本,另一种去演出这些脚本。前一种人是古代的圣贤,七十年代的政工干部;后一种包括古代的老百姓和近代的知青。所谓上智下愚、劳心者治人劳力者治于人,就是这个意思吧。从气质来说,我只适合当演员,不适合当编剧,但是看到脚本编得太坏时,总禁不住要多上几句嘴,就被当落后分子来看待。这么多年了,我也习惯了。  

在一个文明社会里,个人总要做出一些牺牲——牺牲“自我”,成就“超我”——这些牺牲就是崇高的行为。我从不拒绝演出这样的戏,但总希望剧情合理一些——我觉得这样的要求并不过分。举例来说,洪水冲走国家财产,我们年轻人有抢救之责,这是没有疑问的,但总要问问捞些什么。捞木头尚称合理,捞稻草就太过分。这种言论是对崇高唱了反调。现在的人会同意,这罪不在我:剧本编得实在差劲。由此就可以推导出:崇高并不总是对的,低下的一方有时也会有些道理。实际上,就是唱高调的人见了一根稻草被冲走,也不会跳下水,但不妨碍他继续这么说下去。事实上,有些崇高是人所共知的虚伪,这种东西比堕落还要坏。  

人有权拒绝一种虚伪的崇高,正如他有权拒绝下水去捞一根稻草。假如这是对的,就对营造或提倡社会伦理的人提出了更高的要求:不能只顾浪漫煽情,要留有余地;换言之,不能够只讲崇高,不讲道理。举例来说,孟子发明了一种伦理学,说亲亲敬长是人的良知良能,孝敬父母、忠君爱国是人间的大义。所以,臣民向君父奉献一切,就是崇高之所在。孟子的文章写得很煽情,让我自愧不如,他老人家要是肯去做诗,就是中国的拜伦;只可惜不讲道理。臣民奉献了一切之后,靠什么活着?再比方说,在七十年代,人们说,大公无私就是崇高之所在。为公前进一步死,强过了为私后退半步生。这是不讲道理的:我们都死了,谁来干活呢?在煽情的伦理流行之时,人所共知的虚伪无所不在;因为照那些高调去生活,不是累死就是饿死——高调加虚伪才能构成一种可行的生活方式。从历史上我们知道,宋明理学是一种高调。理学越兴盛,人也越虚伪。从亲身经历中我们知道,七十年代的调门最高。知青为了上大学、回城,什么事都干出来了。有种虚伪是不该受谴责的,因为这是为了能活着。现在又有人在提倡追逐崇高,我不知道是在提倡理性,还是一味煽情。假如是后者,那就是犯了老毛病。  

与此相反,在英国倒是出现了一种一点都不煽情的伦理学。让我们先把这相反的事情说上一说——罗素先生这样评价功利主义的伦理学家:这些人的理论虽然显得卑下,但却关心同胞们的福利,所以他们本人的品格是无可挑剔的。然后再让我们反过来说——我们这里的伦理学家既然提倡相反的伦理,评价也该是相反的。他们的理论虽然崇高,但却无视多数人的利益;这种偏执还得到官方的奖励,在七十年代,高调唱得好,就能升官——他们本人的品行如何,也就不好说了。我总觉得有煽情气质的人唱高调是浪费自己的才能:应该试试去写诗——照我看,七十年代的政工干部都有诗人的气质——把营造社会伦理的工作让给那些善讲道理的人,于公于私,这都不是坏事。 

\chapter{谦卑学习班}

朋友们知道我在海外留学多年,总要羡慕地说,你可算是把该看的书都看过了。众所周知,我们这里可以引进好莱坞的文化垃圾,却不肯给文人方便,设家卖国外新书的文化书店。如果看翻译的书,能把你看得连中国话都忘了。要是到北京图书馆去借,你就是老死在里面也借不到几本书。总而言之,大家都有想看而看不到的书。说来也惭愧,我在国外时,根本没读几本正经书,专拣不正经的书看。当时我想,正经书回来也能看到,我先把回来看不到的看了吧。我可没想到回来以后什么都看不到——要是知道,就在图书馆里多泡几年再回来。根据我的经验,人从不正经的书里也能得到教益。  

我就从一本不正经的书里得到了一些教益。这本书的题目叫做《我是<花花公子>的编辑》,里面尽是荒唐的故事,但有一则我以为相当正经。这本书标明是纪实类的书,但我对它的真实性有一点怀疑。这故事是这么开始的:有一天,洛杉矾一家大报登出一则学习班的广告:教授谦卑。学费两千元。住宿在内,膳食自理。本书的作者接到主编的指示:去看看出了什么怪事。他就驱车出发,一路上还在想着:我也太狂傲了,这回报社给报销学费,让我也学点谦卑。等到到了学习班的报名处,看到了一大批过了气的名人:有文体明星、政治家、文化名人、道德讲演家,甚至还有个把在电视上讲道的牧师。美国这地方有点古怪:既捧人,也毁人。以电影明星为例,先把你捧到不知东西南北,口出狂言道:我是有史以来最伟大的男(女)演员。然后就开始毁。先是老百姓看他(她)的狂相不顺眼,纷纷写信或打电话到报社、电视台贬他,然后,那些捧人的传媒也跟着转向,把他骂个一文不值——这道理很简单:报纸需要订户,电视台也需要收视率,美国老百姓可是些得罪不起的人哪。在我们这里就不是这样,所以也没有这样的学习班——这样一来,一个名人就被毁掉了。作者在这个学习班上见到的全是大名人,这些家伙都因为太狂,碰了钉子,所以想要学点谦卑。此时,他想到:和他们相比,我得算个老实人——狂傲这两个字用在我身上是不恰当的。当然,他还没见到我们中国的明星,要是见到了,一定会以为自己就是道德上的完人了。  

且说这个学习班,设在一个山中废弃的中学里,要门没门,要窗没窗,只有满地的鹿粪和狐狸屎。破教室的地上放了一些床垫子,从破烂和肮脏程度来看,肯定是大街上拣来的垃圾。那些狂傲的名人好不容易才弄清是要他们睡在这些垫子上,知道以后,就纷纷向工作人员嚷道:两千块钱的住宿就是这样的吗?人家只回答一句话:别忘了你是来学什么的!有些人就说:说得对,我是来学谦卑的,住得差点,有助于纠正我道德上的缺陷;有些人还是不理解,还是吵吵闹闹。但吵归吵,人家只是不理。等到中午吃饭时,那破学校的食堂里供应汉堡包,十块钱一份,面包倒是很大,生菜叶子也不少——毛驴会喜欢的——就是没有肉。有些狂傲的名人就吼了起来:十块钱一个的汉堡包就该是这样的吗?牛肉在哪儿?(顺便说一句,“Where is the beef!”是句成语,意思是“别蒙事呀!”)得到的回答是:别忘了你是来学什么的!就这样,吃着净素,睡着破床垫,每天早上在全校唯一能流出冷水的破管子前面排着长队盥洗。此书的作者是个老油子,看了这个破烂的地点和这些不三不四的工作人员,心里早就像明镜似的,但他也不来说破。除了吃不好睡不好,这个学习班还实行着封闭式管理,不到结业谁也不准回家——当然,除非你不想结业,也不要求退还学费,就可以回家。这些盛气凌人的家伙被圈在里面,很快就变得与一伙叫化子相仿。除了这种种不便,这个班还总不上课,让学员在这破烂中学里溜达,美其名曰反省自己。学习班的办公室里总是挤满了抱怨的人,大家都找负责人吵架,但这位负责人也有一手,总是笑容可掬地说道:要是我是你,就不这样气急败坏——要知道,在上帝面前,我们可都是罪人哪。至于课,我们会上的。听了以后保证你们会满意。长话短说,这个鬼学习班把大家耗了两个礼拜,这帮名人居然都坚持了下来,只是天天闹着要听课。  

最后,上课的时刻终于来到了。校方宣布,主讲者是个伟大的人,很不容易请到。所以这课只讲一堂,讲完了就结业。于是,全体学员都来到了破礼堂里,见到了这位演讲人。原书花了整整三页来形容他,但我没有篇幅,只能长话短说:此人有点像歌星,有点像影星,有点像信口雌黄的政治家,又有几分像在讲台上满嘴撒村的野狐禅牧师——为了使中国读者理解,还要加上一句,他又像个有特异功能的大气功师。总而言之,他就是那个我们花钱买票听他嚷嚷的人。这么个家伙往台上一站,大家都倍感亲切,因而鸦雀无声。此人说道:我的课只讲一句话,讲完了整个学习班就结束……虽然只是一句话,大家记住了,就会终生受用不尽,以后永不会狂傲——听好了:You are an asshole!同时,他还把这话写在了黑板上,然后一摔粉笔,扬长而去。这话只能用北京俗话来翻译:你是个傻×!  

礼堂里先是鸦雀无声,然后就是卷堂大乱。有人感到大受启发,说道:有道理,有道理!原来我是个傻×呀。还有人愤愤不平,说道:就算我真是个傻×,也犯不着花两千块钱请人来告诉我!至于该书作者,没有介人争论,径直开车下山去找东西吃——连吃两个礼拜的净素可不是闹着玩的。如前所述,我对这故事的真实性有点怀疑,但我以为,真不真的不要紧,要紧的是要有教育意义——中国常有人不惜代价,冒了被踩死的危险。挤进体育馆一类的地方,去见见大名人,在里面涕泪直流,出来后又觉得上当。这道理是这样的:用不着花很多钱,受很多罪,跑好远的路,洗耳恭听别人说你是傻×。自己知道就够了。

\chapter{荷兰牧场与父老乡亲}

我到荷兰去旅游,看到运河边上有个风车,风车下面有一片牧场,就站下来看,然后被震惊了。这片牧场在一片低洼地里,远低于运河的水面,茵茵的绿草上有些奶牛在吃草。乍看起来不过是一片乡村景象,细看起来就会发现些别的:那些草地的中央隆起,四周环以浅沟;整个地面像瓦愣铁一样略有起伏;下凹的地方和沟渠相接;浅沟通向深沟,深沟又通向渠道。所有的渠道都通到风车那里。这样一来,哪怕天降大雨,牧场上也不会有积水。水都流到沟渠里,等着风车把它抽到运河里去。如果没有这样精巧的排水系统,这地方就不会有牧场,只会有沼泽地。站在运河边上,极目所见,到处是这样井然有序的牧场,这些地当然不是天生这样,它是人悉心营造的结果。假如这种田园出于现代工程技术人员之手,那倒也罢了。实际上,这些运河、风车、牧场,都是十七世纪时荷兰人的作品。我从十七岁就下乡插队,南方北方都插过,从来没见过这样的土地。 

我在山东老家插过两年队,什么活都干过。七四年的春夏之交,天还没有亮,我就被一阵哇哇乱叫的有线广播声吵起来了。这种哇哇的声音提醒我们,现在已经是电子时代。然后我紧紧裤腰带,推起独轮车,给地里送粪。独轮车很不容易叫我想起现在是电子时代。俗话说得好,种地不上粪,等于瞎胡混;我们老家的人就认这个理。独轮车的好处在于它可以在各种糟糕的路上走,绕过各种坑和石头;坏处在于它极难操纵,很容易连人带车一起翻掉。我们老家的人在提高推车技巧方面不遗余力,达到了杂技的水平。举例来说,有人可以把车推过门槛,有人可以把它推上台阶。但不管技巧有多高,还是免不了栽跟头,而且总造成鼻青脸肿的后果。现在我想,与其在车技上下苦功,还不如把路修修——我在欧洲游玩时,发现那边的乡间道路极为美好——但这件事就是没人干。不要说田间的路,就是村里的路也很糟;说不清是路还是坑。我们老家那些地都在山上。下乡时我带了几双布鞋,全是送粪时穿坏了。整双鞋像新的一样,只是后跟豁开了。我的脚脖子经常抽筋,现在做梦梦到推粪上山,还是要抽筋。而且那些粪也不过是美其名为粪,实则是些垫猪圈的土,学大寨时要凑上报数字,常常刚垫上就挖出来,猪还来不及在上面排泄呢……我去起圈时,猪老诧异地看着我。假如它会说话,肯定要问问我:抽什么疯呢?有时我也觉得不好意思,就揍它。被猪看成笨蛋,这是不能忍受的。 

坦白地说,我自己绝不可能把一车粪推上山——坡道太陡,空手走都有点喘。实际上山边上有人在接应:小车推到坡道上,就有人用绳子套住,在前面拉,和两人之力,才能把车弄上山去。这省了我的劲儿,但从另一个角度来说就更笨了。这道理是这样的:这一车粪有一百公斤,我和小车加起来,也快有一百公斤了,为了送一百公斤的粪,饶上我这一百公斤已经很笨,现在又来了一个人,这就不止是一百公斤。刨去做无效功不算,有效功不过是送上去一些土,其中肥料的成份本属虚无飘渺……好在这些蠢事猪是看不到的;假如看到的话,不知它会怎么想:土里只要含有微量它老人家的粪尿,人就要不惜劳力送上高山——它会因此变成自大狂,甚至提出应该谁吃谁的问题。 

从任何意义上说,送粪这种工作决不比从低洼地里提水更有价值。这种活计本该交给风能去干,犯不着动用宝贵的人体生物能。我总以为,假如我老家住了些十七世纪的荷兰人,肯定遍山都是缆车、索道——他们就是那样的人:工程师、经济学家、能工巧匠。至于我老家的乡亲,全是些勤劳朴实,缺少心计的人。前一种人的生活比较舒服,这是不容争辩的。 

现在可以说说我是种什么人。在老家时,我和乡亲们相比,显得更加勤劳朴实、更加少心计。当年我想的是:我得装出很能吃苦的样子,让村里的贫下中农觉得我是个好人,推荐我去上大学,跳出这个火坑……顺便说一句,我虽有这种卑鄙的想法,但没有得逞。大学还是我自己考上的。既然他们没有推荐我,我就可以说几句坦白的话,不算占了便宜又卖乖:村里的那些活,弄得人一会儿腰疼,一会儿腿疼。尤其是拔麦子,拔得手疼不已,简直和上刑没什么两样——十指连心嘛,干嘛要用它们干这种受罪的事呢。当年我假装很受用,说什么身体在受罪,思想却变好了,全是昧心话。说良心话就是:身体在受罪,思想也更坏了,变得更阴险,更奸诈……当年我在老家插队时,共有两种选择:一种朴实的想法是在村里苦挨下去,将来成为一位可敬的父老乡亲;一种狡猾的想法就是从村里混出去,自已不当父老乡亲,反过来歌颂父老乡亲。这种歌颂虽然动听,但多少有点虚伪……站在荷兰牧场面前,我发现还有第三种选择。对于个人来说,这种选择不存在,但对于一个民族来说,它不仅存在,而且还是正途。

\chapter{我看老三届}

我也是“老三届”,本来该念书的年龄,我却到云南挖坑去了。这件事对我有害,尚在其次,还惹得父母为此而忧虑。有人说,知青的父母都要因儿女而减寿,我家的情况就是如此。做父母的总想庇护未成年的儿女,在特殊年代里,无力庇护,就代之以忧虑。身为人子,我为此感到内疚,尤其是先父去世后更是如此。当然,细想起来,罪不在我,但是感情总不能自已。  

在上山下乡运动中,两千万知青境遇不同;有人感觉好些,有人感觉坏些。讨论整个老三届现象,就该把个人感情撤除在外,有颗平常心。老三届的人对此会缺少平常心,这是可以理解的。从历史的角度来看,这件事极不寻常。怎么就落在我们身上,这真叫活见鬼了。人生在什么国度,赶上什么样的年月,都不由自己来决定。所以这件事说到底,还是造化弄人。  

上山下乡是件大坏事,对我们全体老三届来说,它还是一场飞来的横祸。当然,有个别人可能会从横祸中得益,举例来说,这种特殊的经历可能会有益于写作,但整个事件的性质却不可因此混淆。我们知道,有些盲人眼睛并没有坏,是脑子里的病,假如脑袋受到重击就可能复明。假设有这样一位盲人扶杖爬上楼梯,有个不良少年为了满足自己无聊的幽默感,把他一脚踢了下去,这位盲人因此复了明,但盲人滚下楼梯依然是件惨痛的事。尤其是踢盲人下楼者当然是个下流胚子,决不能因为该盲人复明就被看成是好人。这是一种简单的逻辑,大意是说,坏事就是坏事,好事就是好事,让我们先言尽于此。至于坏事可不可以变成好事,已经是另一个问题了。  

我有一位老师,有先天的残疾,生下来时手心朝下,脚心朝上,不管自己怎么努力,都不能改变手脚的姿态。后来他到美国,在手术台上被人大卸八块又装了起来,勉强可以行走,但又多了些后遗症。他向我坦白说,对自己的这个残疾,他一直没有平常心:我在娘胎里没做过坏事,怎么就这样被生了下来?后来大夫告诉他说,这种病有六百万分之一的发生几率,换言之,他中了个一比六百万的大彩。我老师就此恢复了平常心。他说:所谓造化弄人,不过如此而已。这个彩我认了。他老人家在学术上有极大的成就,客观地说,和残疾是有一点关系的:因为别人玩时他总在用功。但我没听他说过:谢天谢地,我得了这种病!总而言之,在这件事上他是真正地有了平常心。顺便说一句,他从没有坐着轮椅上台“讲用”。我觉得这样较好。对残疾人的最大尊重,就是不把他当残疾人。  

坦白地说,身为老三届,我也有没有平常心的时候,那就是在云南挖坑时。当时我心里想:妈的!比我们大的可以上大学,我们就该修理地球?真是不公平!这是一类想法。这个想法后来演变成:比我们小的也直接上大学,就我们非得先挖坑后上学,真他妈的不公平。另一类想法是:我将来要当作家,吃些苦可能是大好事。陀思妥耶夫斯基还上过绞首台哪。这个想法后来演变成:现在的年轻人没吃苦,也当不了作家。这两种想法搅在一起,会使人彻底糊涂。现在我出了几本书,但我却以为,后一种想法是没有道理的。假定此说是有理的,想当作家的人就该时常把自己吊起来,想当历史学家的人就该学太史公去掉自己的男根,想当音乐家的人就该买个风镐来家把自己震聋,以便像贝多芬,想当画家的人就该割去自己的耳朵,混充凡·高。什么都想当的人就得把什么都去掉,像个梆子,听起来就不是个道理。总的来说,任何老三届优越的理论都没有平常心。当然,我也反对任何老三届恶劣的说法。老三届正在壮年,耳朵和男根齐备,为什么就不如人。在身为老三届这件事上,我也有了平常心:不就是荒废了十年学业吗?这个彩老子也认了。现在不过四十来岁,还可以努力嘛。  

现在来谈谈那种坏事可以变好事,好事也可以变坏事的说法。它来源于伟人,在伟大的头脑里是好的,但到了寻常人的头脑里就不起好作用,有时弄得人好赖不知,香臭不知。对我来说,好就是好,坏就是坏,这个逻辑很够用。人生在世,会遇到一些好事,还会遇上些坏事。好事我承受得起,坏事也承受得住。就这样坦荡荡做个寻常人也不坏。  

本文是对《中国青年研究》第四期上彭泗清先生文章的回应。坦白地说,我对彭先生的文章不满,起先是因为他说了老三届的坏话。在我看来,老三届现象、老三届情结,是我们这茬人没有平常心造成的。人既然不是机器,偶尔失去平衡,应该是可以原谅的。但是仔细想来,“文革”过了快二十年了,人也不能总是没有平常心哪,老三届文人的一些自我吹嘘的言论,连我看着都肉麻。让我们先言尽于此:对于彭先生所举老三届心态的种种肉麻之处,我是同意的。  

然后再说说我对彭先生的不满之处。彭先生对老三届的看法是否定的,对此我倒不想争辩,想争的是他讲出的那一番道理。他说老三届有种种特殊遭遇,所以他们是些特殊的人;这种特殊的人不怎么高明——这是一种特别糟糕的论调。翻过来,说这种特殊的人特别好,也同样的糟。这个论域貌似属于科学,其实属于伦理;它还是一切法西斯和偏执狂的策源地。我老师生出来时脚心朝上,但假如说的不是身体而是心智,就不能说他特殊。老三届的遭遇是特别,但我看他们也是些寻常人。对黑人、少数民族、女人,都该做如是观。罗素先生曾说,真正的伦理原则把人人同等看待。我以为这个原则是说,当语及他人时,首先该把他当个寻常人,然后再论他的善恶是非。这不是尊重他,而是尊重“那人”,从最深的意义上说,更是尊重自己——所有的人毕竟属同一物种。人的成就、过失、美德和陋习,都不该用他的特殊来解释。You are special,这句话只适于对爱人讲。假如不是这么用,也很肉麻。

\chapter{盖茨的紧身衣}

比尔·盖茨在《未来之路》一书里写道:随着现代信息技术的发展,工程师已有能力营造真实的感觉。他们可以给人戴上显示彩色图像的眼镜,再给你戴上立体声耳机,你的所见所闻都由计算机来控制。只要软硬件都过硬,人分不出电子音像和真声真像的区别。可能现在的软硬件还称不上过硬,尚做不到这一点,但过去二十年里,技术的进步是惊人的,所以对这一天的到来,一定要有心理准备。  

光看到和听到还不算身历其境,还要模拟身体的感觉。盖茨先生想出一种东西,叫做VR紧身衣,这是一种机电设备,像一件衣服,内表面上有很多伸缩的触头,用电脑来控制,这样就可以模仿人的触觉。照他的说法,只要有二十五到三十万个触点,就可以完全模拟人全身的触感——从电脑技术的角度来说,控制这些触头简直是小儿科。有了这身衣服,一切都大不一样。比方说,电脑向你输出一阵风,你不但可以看到风吹杨柳,听到风过树梢,还可以感到风从脸上流过——假如电脑输出的是美人,那就不仅是她的音容笑貌,还有她的发丝从你面颊上滑过——这是友好的美人,假如不友好,来的就是大耳刮子——VR紧身衣的概念就是如此。作为学食品科技的人,我觉得还该有个面罩连着一些香水瓶,由电脑控制的阀门决定你该闻到什么气味,但假若你患有鼻炎,就会觉得面罩没有必要。总而言之,VR紧身衣的概念就是如此。估计要不了二十年,科学就能把它造出来,而且让它很便宜,像今天的电子游戏机一样,在街上出售;穿上它就能前往另一个世界,假如软件丰富,想上哪儿就能上哪儿,想遇上谁就能遇上谁,想干啥就能干啥,而且不花什么代价——顶多出点软件钱。到了那一天,不知人们还有没有心思阅读文本,甚至识不识字都不一定。我靠写作为生,现在该作出何种决定呢?  

大概是在六七十年代吧,法国有些小说家就这样提出问题:在电影时代,小说应该怎么写?该看到的电影都演出来了,该听到的广播也播出来了。托尔斯泰在《战争与和平》里花几十页写出的东西,用宽银幕电影几个镜头就能解决。还照经典作家的写法,没有人爱看,顶多给电影提供脚本——如我们所知,这叫生产初级产品,在现代社会里地位很低。在那时,电影电视就像比尔·盖茨的紧身衣,对艺术家来说,是天大的灾难。有人提出,小说应该向诗歌的方向发展。还有人说,小说该着重去写人内心的感受。这样就有了法国的新小说。还有人除了写小说,还去搞搞电影,比如已故的玛格丽特·杜拉斯。我对这些作品很感兴趣,但凭良心说,除杜拉斯的《情人》之外,近十几年来没读到过什么令人满意的小说。有人也许会提出最近风靡一时的《廊桥遗梦》,但我以为,那不过是一部文字化的电影。假如把它编成软件,钻到比尔·盖茨的紧身衣里去享受,会更过瘾一些。相比之下,我宁愿要一本五迷三道的法国新小说,也不要一部《廊桥遗梦》,这是因为,从小说自身的前途来看,写出这种东西解决不了问题。  

真正的小说家不会喜欢把小说写得像电影。我记得米兰·昆德拉说过,小说和音乐是同质的东西。我讨厌这个说法,因为好像这世界上没有了音乐,就说不出小说该像什么了;但也不能不承认,这种说法有些道理。小说该写人内在的感觉,这是没有疑问的。但仅此还不够,还要使这些感觉组成韵律。音乐有种连贯的、使人神往的东西,小说也该有。既然难以言状,就叫它韵律好了。  

本文的目的是要纪念已故的杜拉斯,谈谈她的小说《情人》,谁知扯得这样远——现在可以进入主题。我喜欢过不少小说,比方说,乔治·奥威尔的《1984》,还有些别的书。但这些小说对我的意义都不能和《情人》相比。《1984》这样的书对我有帮助,是帮我解决人生中的一些疑惑,而《情人》解决的是有关小说自身的疑惑。这本书的绝顶美好之处在于,它写出一种人生的韵律。书中的性爱和生活中别的事件,都按一种韵律来组织,使我完全满意了。就如达·芬奇画出了他的杰作,别人不肯看,那是别人的错,不是达·芬奇的错;米开朗琪罗雕出了他的杰作,别人不肯看,那是别人的错,不是米开朗琪罗的错。现代小说有这样的杰作,人若不肯看小说,那是人的错,不是小说的错。杜拉斯写过《华北情人》后说,我最终还原成小说家了。这就是说,只有书写文本能使她获得叙事艺术的精髓。这个结论使我满意,既不羡慕电影的镜头,也不羡慕比尔·盖茨的紧身衣。 

\chapter{关于格调}

最近我出版了一本小说《黄金时代》,有人说它格调不高,引起了我对格调问题的兴趣。各种作品、各种人,尤其是各种事件,既然有高有低,就有了尺度问题。众所周知,一般人都希望自己格调高,但总免不了要干些格调低的事。这就使得格调问题带有了一定的复杂性。  

当年有人问孟子,既然男女授受不亲,嫂子掉到水里,要不要伸手去拉。这涉及了一个带根本性的问题,假如“礼”是那么重要,人命就不要了吗?孟子的回答是:用手去拉嫂子是非礼,不去救嫂子则“是豺狼也”,所以只好从权,宁愿非礼而不做豺狼。必须指出,在非礼和豺狼之中做一选择是痛苦的,但这要怪嫂子干吗要掉进水里。这个答案有不能令人满意的地方,但不是最坏,因为他没有说戴上了手套再去拉嫂子,或者拉过了以后再把手臂剁下来。他也没有回答假如落水的不是嫂子而是别的女人,是不是该去救。但是你不能对孟子说,在生活里,人命是最重要的,犯不着为了些虚礼牺牲它——说了孟夫子准要和你翻脸。另一个例子是舜曾经不通知父亲就结了婚。孟子认为,他们父子关系很坏,假如请示的话,可能一辈子结不了婚;他还扯上了一些不孝有三无后为大的话,结论是舜只好从权了。这个结论同样不能令人满意,因为假如舜的父亲稍稍宽容,许可舜和一个极为恶毒的女人结婚,不知孟子的答案是怎样的。假如让舜这样一位圣贤娶上一个恶毒的妇人,从此在痛苦中生活,我以为不够恰当。倘若你说,在由活里,幸福是最重要的,孟老夫子也肯定要和你翻脸。但不管怎么说,一个理论里只要有了“从权”这种说法,总是有点欠严谨。好在孟子又有些补充说明,听上去更有道理。  

有关礼与色孰重的问题,孟子说,礼比色重,正如金比草重。虽然一车草能比一小块金重,但是按我的估计,金子和草的比重大致是一百比一——搞精确是不可能的,因为草和草还不一样。这样我们就有了一个换算关系,可以作为生活的指南,虽然怎么使用还是个问题。不管怎么说,孟子的意思是明白的,生活里有些东西重,有些东西轻。正如我们现在说,有些事格调高,有些事格调低。假如我们重视格调高的东西,轻视格调低的东西,自己的格调就能提升。  

作为一个前理科学生,我有些混账想法,可能会让真正的人文知识分子看了身上长鸡皮疙瘩。对于“礼”和“色”,大致可以有三到四种不同的说法。其一,它们是不同质的东西,没有可比性;其二,礼重色轻,但是它们没有共同的度量;最后是有这种度量,礼比色重若干,或者一单位的礼相当于若干单位的色;以上的分类恰恰就是科学上说的定类(nominal)、定序(ordinal)、定距(interval)和定比(ratio)这四种尺度(定距和定比的区别不太重要)。这四种尺度越靠后的越精密、格调既然有高低之分,显然属于定序以后的尺度。然而,说格调仅仅是定序的尺度还不能令人满意——按定序的尺度,礼比色重,顺序既定,不可更改,舜就该打一辈子光棍。如果再想引入事急从权的说法,那就只能把格调定为更加精密的尺度,以便回答什么时候从权,什么时候不可从权的问题——如果没个尺度,想从权就从权,礼重色轻就成了一句空话。于是,孟子的格调之说应视为定比的尺度,以格调来度量,一份礼大致等于一百份色。假如有一份礼,九十九份色,我们不可从权;遇到了一百零一份色就该从权了。前一种情形是在一百和九十九中选了一百,后者是从一百和一百零一中选了一百零一。在生活中,作出正确的选择,就能使自己的总格调得以提高。  

对于作品来说,提升格调也是要紧的事。改革开放之初有部电影,还得过奖的,是个爱情故事。男女主角在热恋之中,不说“我爱你”,而是大喊“I love my motherland!”场景是在庐山上,喊起来地动山摇,格调就很高雅,但是离题太远。国外的电影拍到这类情节,必然是男女主角拥抱热吻一番,这样格调虽低,但比较切题。就爱情电影而言,显然有两种表达方式,一种格调高雅,但是晦涩难解。另一种较为直接,但是格调低下。按照前一种方式,逻辑是这样的:当男主角立于庐山之上对着女主角时,心中有各种感情:爱祖国、爱人民、爱领袖、爱父母,等等。最后,并非完全不重要,他也爱女主角。而这最后一点,他正急于使女主角知道。但是经过权衡,前面那些爱变得很重,必须首先表达之,爱她这件事就很难提到。而女主角的格调也很高雅,她知道提到爱祖国、爱人民等等,正是说到爱她的前奏,所以她耐心地等待着。我记得电影里没有演到说出“I love you”,按照这种节奏,拍上十几个钟头就可以演到。改革开放之初没有几十集的连续剧,所以真正的爱情场面很难看到。外国人在这方面缺少训练,所以对这部影片的评价是:虽然女主角很迷人,但不知拍了些啥。  

按照后一种方式,男主角在女主角面前时,心里也爱祖国、爱上帝,等等。但是此时此地,他觉得爱女主角最为急迫,于是说,我爱你,并且开始带有性爱意味的身体接触。不言而喻,这种格调甚为低下。这两种方式的区别只在于有无经过格调方面的加权运算,这种运算本身就极复杂,导致的行为就更加复杂。后一种方式没有这个步骤,显得特别简捷,用现时流行的一个名词,就是较为“直露”。这两种方式的区别在于前者以爱对方为契机,把祖国人民等等—一爱到,得到了最高的总格调。而后者径直去爱对方,故而损失很大,只得到了最低的总格调。  

说到了作品,大家都知道,提升格调要受到某种制约。“文革”里有一类作品只顾提升格调,结果产生了高大全的人物和高大全的故事,使人望之生厌。因为这个原故,领导上也说,要做到政治性与艺术性的统一——作品里假如只有格调,就不成个东西。这就是说,格调不是评价作品唯一的尺度。由此就产生了一个问题,另外那种东西和格调是个什么关系?这个问题孟子肯定会这么回答:艺术与格调,犹色与礼也。作品里的艺术性,或则按事急从权的原则,最低限度地出现;或则按得到最高格调的原则,合理地搭配。比如说,径直去写男女之爱,得分为一,搭配成革命的爱情故事,就可以得到一百零一分。不管怎么说,最后总要得到高大全。  

我反对把一切统一到格调上,这是因为它会把整个生活变成一种得分游戏。一个得分游戏不管多么引人入胜,总不能包容全部生活,包容艺术,何况它根本就没什么意思。假如我要写什么。我就根本不管它格调不格调;正如谈恋爱时我决不从爱祖国谈起。  

现在可以谈谈为什么别人说我的作品格调低——这是因为其中写到了性。因为书中人物不是按顺序干完了格调高的事才来干这件格调低的事,所以它得分就不高。好在评论界没有按礼与色一百比一的比例来算它的格调,所以在真正的文学圈子里对它的评价不低,在海外还得过奖。假如说,这些人数学不好,不会算格调,我是不能承认的。不说别人,我自己的数学相当好,任何一种格调公式我都能掌握、我写这些作品是有所追求的,但这些追求在格调之外。除此之外,我还怀疑,人得到太多的格调分,除了使别人诧异之外,没有实际的用处。  

坦白地说,我对色情文学的历史有一点了解。任何年代都有些不争气的家伙写些丫丫乌的黄色东西,但是真正有分量的色情文学都是出在“格调最高”的时代。这是因为食色性也,只要还没把小命根一刀割掉,格调不可能完全高。比方说,英国维多利亚时期出了一大批色情小说,作者可以说有相当的文学素质;再比方说,“文化革命”里流传的手抄小说,作者的素质在当时也算不错。要使一个社会中一流的作者去写色情文学,必须有极严酷的社会环境和最不正常的性心理。在这种情况下,色情文学是对假正经的反击。我认为目前自己尚写不出真正的色情文学,也许是因为对环境感觉鲁钝。前些时候我国的一位知名作者写了《废都》,我还没有看。有人说它是色情文学,但愿它不是的,否则就有说明意义了。  

维多利亚时期的英国人和“文革”时的中国人一样,性心理都不正常。正常的性心理是把性当作生活中一件重要的事,但不是全部。不正常则要么不承认有这么回事,要么除此什么都不想。假如一个社会的性心理不正常,那就会两样全占。这是因为这个社会里有这样一种格调,使一部分人不肯提到此事,另一部分人则事急从权,总而言之,没有一个人有平常心。作为作者,我知道怎么把作品写得格调极高,但是不肯写。对于一件愚蠢的事,你只能唱唱反调。

\chapter{关于幽闭型小说}

张爱玲的小说有种不同凡响之处,在于她对女人的生活理解得很深刻。中国有种老女人,面对着年轻的女人,只要后者不是她自己生的,就要想方设法给她罪受:让她干这干那,一刻也不能得闲,干完了又说她干得不好;从早唠叨到晚,说些尖酸刻薄的话——捕风捉影,指桑骂槐。现在的年轻人去过这种生活,一天也熬不下来。但是传统社会里的女人都得这么熬。直到多年的媳妇熬成了婆,这女人也变得和过去的婆婆一样刁。张爱玲对这种生活了解得很透,小说写得很地道。但说句良心话,我不喜欢。我总觉得小说可以写痛苦,写绝望,不能写让人心烦的事,理由很简单:看了以后不烦也要烦,烦了更要烦,而心烦这件事,正是多数中国人最大的苦难。也有些人烦到一定程度就不烦了——他也“熬成婆”了。  

像这种人给人罪受的事,不光女人中有,男人中也有,不光中国有,外国也有。我在一些描写航海生活的故事里看到过这类事,这个折磨人的家伙不是婆婆,而是水手长。有个故事好像是马克·吐温写的:有这么个千刁万恶的水手长,整天督着手下的水手洗甲板,擦玻璃,洗桅杆。讲卫生虽是好事,但甲板一天洗二十遍也未免过分。有一天,水手们报告说,一切都洗干净了。他老人家爬到甲板上看看,发现所有的地方都一尘不染,挑不出毛病,就说:好吧,让他们把船锚洗洗吧。整天这样洗东西,水手们有多心烦,也就不必再说了,但也无法可想:四周是汪洋大海,就算想辞活不干,也得等到船靠码头。实际上,中国的旧式家庭,对女人来说也是一条海船,而且永远也靠不了码头。你要是烦得不行,就只有跳海一途。这倒不是乱讲的,旧式女人对自杀这件事,似乎比较熟练。由此可以得到这样的结论:这种故事发生的场景,总是一个封闭的地方,人们在那里浪费着生命;这种故事也就带点幽囚恐怖症的意味。  

本文的主旨,不是谈张爱玲,也不是谈航海小说,而是在谈小说里幽闭、压抑的情调。家庭也好,海船也罢,对个人来说,是太小的囚笼,对人类来说,是太小的噩梦。更大的噩梦是社会,更准确地说,是人文生存环境。假如一个社会长时间不进步,生活不发展,也没有什么新思想出现,对知识分子来说,就是一种噩梦。这种噩梦会在文学上表现出来。这正是中国文学的一个传统。这是因为,中国人相信天不变道亦不变,在生活中感到烦躁时,就带有最深刻的虚无感。这方面最好的例子,是明清的笔记小说,张爱玲的小说也带有这种味道:有忧伤,无愤怒;有绝望,无仇恨;看上去像个临死的人写的。我初次读张爱玲,是在美国,觉得她怪怪的。回到中国看当代中青年作家的作品,都是这么股味。这时才想到:也许不是别人怪,是我怪。  

所谓幽闭类型的小说,有这么个特征:那就是把囚笼和噩梦当作一切来写。或者当媳妇,被人烦;或者当婆婆,去烦人;或者自怨自艾;或者顾影自怜;总之,是在不幸之中品来品去。这种想法我很难同意。我原是学理科的,学理科的不承认有牢不可破的囚笼,更不信有摆不脱的噩梦;人生唯一的不幸就是自己的无能。举例来说,对数学家来说,只要他能证明费尔马定理,就可以获得全球数学家的崇敬,自己也可以得到极大的快感,问题在于你证不出来。物理学家发明了常温核聚变的方法,也可马上体验幸福的感觉,但你也发明不出来。由此就得出这样的结论,要努力去做事,拼命地想问题,这才是自己的救星。  

怀着这样的信念,我投身于文学事业。我总觉得一门心思写单位里那些烂事,或者写些不愉快的人际冲突,不是唯一可做的事情。举例来说,可以写《爱丽丝漫游奇境记》这样的作品,或者,像卡尔维诺《我们的祖先》那样的小说。文学事业可以像科学事业那样,成为无边界的领域,人在其中可以投入澎湃的想象力。当然,这很可能是个馊主意。我自己就写了这样一批小说,其中既没有海船,也没有囚笼,只有在它们之外的一些事情。遗憾的是,这些小说现在还在主编手里压着出不来,他还用一种本体论的口吻说道:他从哪里来?他是谁?他到底写了些什么? 

\chapter{关于“媚雅”}

前不久在报纸上看到一篇文章,谈到有关“媚俗”与“媚雅”的问题。作者认为,米兰·昆德拉用出来一个词儿,叫做“媚俗”,是指艺术家为了取悦大众,放弃了艺术的格调。他还说,我们国内有些小玩闹造出个新词“媚雅”,简直不知是什么意思。这个词的意思我倒知道,是指大众受到某些人的蛊惑或者误导,一味追求艺术的格调,也不问问自己是不是消受得了。在这方面我有些经验,都与欣赏音乐有关。高雅音乐格调很高,大概没有疑问。我自己在音乐方面品味很低,乡村音乐还能听得住,再高就受不了。  

大约十年前,我在美国,有一次到波士顿去看个朋友。当时正是盛夏,为了躲塞车,我天不亮就驾车出发,天傍黑时到,找到了朋友,此时他正要出门。他说,离他家不远有个教堂,每晚里面都有免费的高雅音乐会,让我陪他去听。说实在的,我不想去,就推托道:听高雅音乐要西装革履、正襟危坐。我开了一天的车,疲惫不堪,就算了吧。但是他说,这个音乐会比较随便,属大学音乐系师生排演的性质。你进去以后只要不打瞌睡、不中途退场就可。我就去了、到了门口才知道是演奏布鲁克纳的两首交响曲。我的朋友还拉我在第一排正中就座,听这两首曲子——在这里坐着,连打呵欠的机会都没有了。我觉得这两首曲子没咸没谈、没油没盐,演奏员在胡吹、胡拉,指挥先生在胡比画,整个感觉和晕船相仿。天可怜见,我开了十几个小时的车,坐在又热又问的教堂里,只要头沾着点东西,马上就能睡着;但还强撑着,把眼睛瞪得滚圆,从七点撑到了九点半!中间有一段我真恨不能一头碰死算了……布鲁克纳那厮这两首鸟曲,真是没劲透了!  

如前所述,我在古典音乐方面没有修养,所以没有发言权。可能人家布鲁克纳音乐的春风是好的,不入我这俗人的驴耳。但我总觉得,就算是高雅的艺术,也有功力、水平之分,不可以一概而论。总不能一入了高雅的门槛就是无条件的好——如此立论,就是媚雅了。人可以抱定了媚雅的态度,但你的感官马上就有不同意见,给你些罪受……  下一个例子我比较有把握——不是我俗,而是表演高雅音乐的人水平低所致。这回是听巴赫的合唱曲,对曲子我没有意见,这可不是崇拜巴赫的大名,是我自己听出来的。这回我对合唱队有点意见。此事的起因是我老婆教了个中文班,班上有个学生是匹兹堡市业余乐团的圆号手,邀我们去听彩排,我们就去了。虽不是正式演出,作为观众却不能马虎,因为根本就没有几个观众。所以我认真打扮起来——穿上三件套的西服。那件衣服的马甲有点瘦,但我老婆说,瘦衣服穿起来精神;所以我把吃牛肉吃胀的肚腩强箍了下去,导致自己的横膈膜上升了一寸,有点透不过气来。就这样来到音乐学院的小礼堂,在前排正中入座。等到幕启,见到合唱队,我就觉得出了误会:合唱队正中站了一位极熟的老太太。我在好几个课里和她同学——此人没有八十,也有七十五——我记得她是受了美国政府一项“老年人重返课堂”项目的资助,书念得不好,但教授总让她及格,我对此倒也没有什么意见。看来她又在音乐系混了一门课,和同学一起来演唱。很不幸的是,人老了,念书的器官会退化,歌唱的器官更会退化,这歌大概也唱不好。但既然来了,就冲这位熟识的老人,也得把这个音乐会听好——我们是有这种媚雅的决心的。说句良心话,业余乐团的水平是可以的,起码没走调;合唱队里领唱的先生水平也很高。及至轮到女声部开唱,那位熟识的老太太按西洋唱法的要求把嘴张圆,放声高歌“亚美路亚”,才半声,眼见得她的假牙就从口中飞了出来,在空中一张一合,做要咬人状,飞过了乐池,飞过我们头顶,落向脑后第三排;耳听得“亚美路亚”变成了一声“噗”!在此庄重的场合,唱着颂圣的歌曲,虽然没假牙口不关风,老太太也不便立即退场,瘪着嘴假作歌唱,其状十分古怪……请相信,我坐在那里很严肃地把这一幕听完了,才微笑着鼓掌。所有狂野粗俗的笑都被我咽到肚子里,结果把内脏都震成了碎片。此后三个月,经常咳出一片肺或是一片肝。但因为当时年轻,身体好,居然也没死。笔者行文至此,就拟结束。我的结论是:媚雅这件事是有的,而且对俗人来说,有更大的害处。

\chapter{卡拉OK和驴鸣镇}

有一次,愁容骑土堂·吉诃德和他忠实的侍从桑乔·潘萨走在路上,遇到一伙手持刀杖去打冤家的乡下人。这位高尚的骑士问乡下人为什么要厮杀,听到了这么一个故事:在一个镇子上,住了两个朋友。有一天,其中一位走失了一条驴子,就找朋友帮忙。他们进山去找——那位帮忙的朋友说:山这么大,怎么找呢。我有一样不登大雅的雕虫小技,假如你也会一点,事情就好办了。失驴的朋友说:这是怎样的技巧呢?那位帮忙者说:他会学驴叫。假如失驴者也会,大家就可以分头学着驴叫在山上巡游,那迷途的驴子听到同类的呼唤,肯定会走出来和他们会合。那失驴者答道:好计策!至于学驴叫,我岂止是会一点,简直是很精通啊!让我们依计而行吧。于是,两位朋友分头走进了山间小道,整个荒山上响起了阵阵驴鸣……  

我住的这座楼隔音很坏,往户中有不少人买了卡拉OK机器,从早唱到晚。黑更半夜,我躺在床上听到OK之声,一面把脑袋往被窝里扎,一面就想起了这个故事——且听我把故事讲完:这两位朋友分头去寻驴,在林子深处相会了。失驴的朋友说:怎么,竟是你吗!我是不轻易恭维人的,但我要说,仅从声音上判断,你和一头驴子是没有任何区别的……那帮忙的朋友答道:朋友,同样的话我正要用来说你!你的声音很宏亮,音度很坚强,节奏很准确。在我的长项上,我从不佩服任何人的,但我对你要五体投地,俯首称臣了!——这也正是笔者的感触。你可以去查七八级人民大学新生的体检记录,我的肺活量在两千人里排第一,可以长嚎一分钟不换气,引得全校的人都想掐死我;但总想在半夜敲邻居的门,告诉她,在嚎叫方面我对他已是五体投地——现在言归正传,那失驴者听到赞誉之后说:以前,我以为自己是个一无所长的人。现在听了你的赞誉,再不敢妄自菲薄,我也是有一技之长的人了……后来,这两位朋友又去寻驴,每次都把对方当成驴,聚在了一起。最后,总算是找到了,这可怜的畜生被狼吃得只剩些残余。那帮忙的朋友说:我说它怎么不答应!就算它死了,只要是完整的,听了你的召唤,也一定会起来回答。而那失驴的朋友却说:虽然失了驴,但也发现了自己的才能,我很开心!于是,这两个朋友下山去,把这故事告诉路人:不想给本镇招来了“驴鸣镇”的恶名——隐含的意思就是镇上全是驴。故事开始时见到的那伙人,就是因为被人称为驴鸣镇人,而去拼命。如前所述,我觉得自己住在驴鸣楼里,但不想为此和人拼命。 

我总想提醒大家一句,人在歌唱时听不到自己的声音。在卡拉OK时,面对五彩画面觉得挺美时,也许发出的全不类人声。茶余酒后,想过把歌星瘾时,也可以唱唱。但干这种勾当,最好在歌厅酒楼等吵不着人的地方;就是嗓子好,也请把嗓门放低些,留点余地——别给餐厅留下“驴鸣餐厅”的恶名。

\chapter{关于文体}

自从我开始写作,就想找人谈谈文体的问题,但总是找不到。和不写作的人谈,对方觉得这个题目索然无味;和写作的人谈,又有点谈不开。既然写作,必有文体,不能光说别人不说自己。文体之于作者,就如性之于寻常人一样敏感。  

把时尚排除在外,在文学以内讨论问题,我认为最好的文体都是翻译家创造出来的。傅雷先生的文体很好,汝龙先生的文体更好。查良铮先生的译诗、王道乾先生翻译的小说——这两种文体是我终生学习的榜样。必须承认,我对文体有特殊的爱好,别人未必和我一样。但我相信爱好文学的人会同意我这句话:优秀文体的动人之处,在于它对韵律和节奏的控制。阅读优美的文字会给我带来极大的快感。好多年以前,我在云南插队,当地的傣族少女身材极好。看到她们穿着合身的筒裙婀娜多姿地走路,我不知不觉就想跟上去。阅读带来的快感可以和这种感觉相比。我开始写作,是因为受了好文章的诱惑——我自己写得怎样,当然要另说。  

前辈作家中,有一部分用方言来写作,或者在行文中带出方言的影响来,我叫它方言体。其中以河北和山西两地的方言最为常见。河北人说话较慢,河北方言体难免拖沓。至于山西方言体,我认为它有难懂的毛病——最起码“圪蛋”(据说山西某些地区管大干部叫大“圪蛋”)这个词对山西以外的读者来说,就不够通俗。“文化革命”中出版的文艺作品方言体的很多,当时的作者以为这样写更乡土些,更乡土就更贴近工农兵,更贴近工农兵也就更革命——所以说,方言体也就是革命体。当然,不是每种方言都能让人联想到革命。必须是老根据地所在省份的方言才有革命的气味。用苏白写篇小说,就没有什么革命的气味。  

自方言体之后,影响最大的文体应该是苏晓康写报告文学的文体,或称晓康体。这种文体浮嚣而华丽,到现在还有人模仿。念起来时最好拖着长腔,韵味才足,并且好用三个字的词组,比如“共和国”、“启示录”之类。在晓康体里,前者是指政府,后者是指启示,都属误用。晓康体写多了,人会退化成文盲的。  

现在似乎出现了一种新的文体。我们常看到马晓晴和葛优在电视屏幕上说一种话,什么“特”这个,“特”那个,其实是包含了特多的傻气,这种文体与之相似。所以我们就叫它撒娇打痴体好了。其实用撒娇打痴体的作者不一定写特字,但是肯定觉得做个聪明人特累。时下一些女散文作家(尤其是漂亮的)开始用撒娇打痴体写作。这种文体不用写多了,只消写上一句,作者就像个大头傻子。我也觉得自己活得特累,但不敢学她的样子。我全凭自己的聪明混饭吃。这种傻话本该是看不进去的,但把书往前一翻,看到了作者像:她蛮漂亮的,就感觉她是在搔首弄姿,而且是朝我来的。虽然相片漂亮,真人未必漂亮;就算满脸大麻子,拍照前还不会用腻子腻住?但不管怎么说吧,那本书我还真看下去了——当然,读完就后悔了。赶紧努力把这些傻话都忘掉,以免受到影响。作者怕读坏文章,就是怕受坏影响。  

以上三种文体的流行,都受到了时尚的左右。方言体流行时,大家都羡慕老革命;晓康体流行时,大家都在虚声恫吓;而撒娇打痴体之流行,使我感觉到一些年轻的女性正努力使自己可爱一些。一个漂亮女孩冒点傻气,显得比较可爱——马晓晴就是这么表演的。我们还知道西施有心绞痛并因此更加可爱,心绞痛也该可以形成一种文体。以此类推,更可爱的文体应该是:“拿硝酸甘油来!”但这种可爱我们消受不了。我们已经有了一些医学知识,知道心绞痛随时有可能变成心肌梗塞,塞住了未必还能活着。大美人随时可能死得直翘翘,也就不可爱了。   如前所说,文体对于作者,就如性对寻常人一样重要。我应该举个例子说明我对恶劣文体的感受。大约是在七○年,盛夏时节,我路过淮河边上一座城市,当时它是一大片低矮的平房。白天热,晚上更热。在旅馆里睡不着,我出来走走,发现所有的人都在树下乘凉。有件事很怪:当地的男人还有些穿上衣的,中老年妇女几乎一律赤膊。于是,水银灯下呈现出一片恐怖的场面。当时我想:假如我是个天阉,感觉可能会更好一点。恶劣的文字给我的感受与此类似:假如我不识字,感觉可能会更好。 

\chapter{欣赏经典}


有个美国外交官,二三十年代在莫斯科呆了十年。他在回忆录里写道:他看过三百遍《天鹅湖》。即使在芭蕾舞剧中《天鹅湖》是无可争辩的经典之作,看三百遍也太多了,但身为外交官,有些应酬是推不掉的,所以这个戏他只能一遍又一遍地看,看到后来很有点吃不消。我猜想,头几十次去看《天鹅湖》,这个美国人听到的是柴科夫斯基优美的音乐,看到的是前苏联艺术家优美的表演,此人认真地欣赏着,不时热烈地鼓掌。看到一百遍之后,观感就会有所不同,此时他只能听到一些乐器在响着,看到一些人在舞台上跑动,自己也变成木木痴痴的了。看到二百遍之后,观感又会有所不同。音乐一响,大幕拉开,他眼前是一片白色的虚空——他被这个戏魇住了。此时他两眼发直,脸上挂着呆滞的傻笑,像一条冬眠的鳄鱼——松弛的肌肉支持不住下巴,就像冲上沙滩的登陆艇那样,他的嘴打开了,大滴大滴的哈喇子从嘴角滚落,掉在膝头。就这样如痴如醉,直到全剧演完,演员谢幕已毕,有人把舞台的电闸拉掉,他才觉得眼前一黑。这时他赶紧一个大嘴巴把自己打醒,回家去了。后来他拿到调令离开苏联时,如释重负地说道:这回可好了,可以不看《天鹅湖》了。  

如你所知,该外交官看《天鹅湖》的情形都是我的猜测——说实在的,他流了哈喇子也不会写进回忆录里——但我以为,对一部作品不停地欣赏下去,就会遇到这三个阶段。在第一个阶段,你听到的是音乐,看到的是舞蹈——简言之,你是在欣赏艺术。在第二个阶段,你听到一些声音,看到一些物体在移动,觉察到了一个熟悉的物理过程。在第三个阶段,你已经上升到了哲学的高度,最终体会到芭蕾舞和世间一切事物一样,不过是物质存在的形式而已。从艺术到科学再到哲学,这是个返璞归真的过程。一般人的欣赏总停留在第一阶段,但有些人的欣赏能达到第二阶段。比方说,在电影《霸王别姬》里,葛优扮演的戏霸就是这样责备一位演员:“别人的”霸王出台都走六步,你怎么走了四步?在实验室里,一位物理学家也会这样大惑不解地问一个物体:别的东西在真空里下落,加速度都是一个G,你怎么会是两个G?在实验室里,物理过程要有再现性,否则就不成其为科学,所以不能有以两个G下落的物体。艺术上的经典作品也应有再现性,比方说《天鹅湖》,这个舞剧的内容是不能改变的。这是为了让后人欣赏到前人创造的最好的东西。它只能照老样子一遍遍地演。  

经典作品是好的,但看的次数不可太多。看的次数多了不能欣赏到艺术——就如《红楼梦》说饮茶:一杯为品,二杯是解渴的蠢物,三杯就是饮驴了。当然,不管是品还是饮驴,都不过是物质存在的方式而已,在这个方面,没有高低之分……  

“文化革命”里,我们只能看到八个样板戏。打开收音机是这些东西,看个电影也是这些东西。插队时,只要听到广播里音乐一响,不管轮到了沙奶奶还是李铁梅,我们张嘴就唱;不管是轮到了吴琼花还是洪常青,我们抬腿就跳。路边地头的水牛看到我们有此举动,怀疑对它有所不利,连忙扬起尾巴就逃。假如有人说我唱的跳的不够好,在感情上我还难以接受:这就是我的生活——换言之,是我存在的方式,我不过是嚷了一声,跳了一个高,有什么好不好的?打个比方来说,犁田的水牛在拔足狂奔时,总要把尾巴像面小旗子一样扬起来,从人的角度来看有点不雅,但它只会这种跑法。我在地头要活动一下筋骨,就是一个倒踢紫金冠——我就会这一种踢法,别的踢法我还不会哪。连这都要说不好,岂不是说,我该死掉?根据这种情形,我认为自己对八个样板戏的欣赏早已到了第三个阶段,我们是从哲学的高度来欣赏的,但这些戏的艺术成就如何,我确实是不知道。莫斯科歌舞剧院演出的《天鹅湖》的艺术水平如何,那位美国外交官也不会知道。你要是问他这个问题,他只会傻呵呵地笑着,你说好,他也说好,你说不好,他也说不好……  

在一生的黄金时代里,我们没有欣赏到别的东西,只看了八个戏。现在有人说,这些戏都是伟大的作品,应该列入经典作品之列,以便流传到千秋万代。这对我倒是种安慰——如前所述,这些戏到底有多好我也不知道,你怎么说我就怎么信,但我也有点怀疑,怎么我碰到的全是经典?就说《红色娘子军》吧,作曲的杜鸣心先生显然是位优秀的作曲家,但他毕竟不是柴可夫斯基……芭蕾和京剧我不懂,但概率论我是懂的。这辈子碰上了八个戏,其中有两个是芭蕾舞剧,居然个个是经典,这种运气好得让人起疑。根据我的人生经验,假如你遇到一种可疑的说法,这种说法对自己又过于有利,这种说法准不对,因为它是编出来自己骗自己的。当然,你要说它们都是经典,我也无法反对,因为对这些戏我早就失去了评判能力。

\chapter{思想与害臊}

我年轻时在云南插队,仅仅几十年前,那里还是化外蛮邦。因为这个缘故,除了山青水秀之外,还有民风淳朴的好处。我去的时候,那里的父老乡亲除了种地,还在干着一件吃力的事情:表示自己是些有思想的人。在那个年月里,在会上发言时,先说一句时髦的话语,就是有思想的表示。这件事我们干起来十分轻松,可是老乡们干起来就难了。比方说,我们的班长想对大田里的工作发表意见——这对他来说本没有什么困难,他是个老庄稼人嘛——他的发言要从一句时髦话语开始,这句话可把他难死了。从他蠕动的嘴唇看来,似要说句“斗私批修”这样的短语,不怎么难说嘛——但这是对我而言,对他可不是这样。只见他老脸胀得通红,不住地期期艾艾,豆大的汗珠滚滚而下,但最后还是没把这句话憋出来,说出来的是:鸡巴哩,地可不是这么一种种法嘛!听了这样的妙语,我们赶紧站起来,给他热烈鼓掌。我喜欢朴实的人,觉得他这样说话就可以。但他对自己有更高的要求,总要使自己说话有思想。 

据说,旧时波兰的农妇在大路上相遇,第一句话总说:圣母玛丽亚是可赞美的:外乡人听了摸不着头脑,就说:是呀,她是可以赞美,你就赞美吧。这就没有理解对方的意思。对方不是想要赞美圣母,而是要表示自己有思想。我们那时说话前先来一句“最高指示”,也是这个意思。在《红楼梦》里,林黛玉和史湘云在花园里联句,忽然冒出些颂圣的诗句。作者大概以为,林史虽是闺阁中人,说话也总要有思想才对。至于我们的班长,也是这样想的,只是没有林妹妹那样伶牙俐齿。也不知为什么,时髦话语使他异常害臊,拼了命也讲不出口;讲出的总是些带X的话。这就使全体男知青爱上了他。每次他在大会上发言之前,我们都屏息静等,等到他一讲出话来就鼓掌欢呼,这使他的毛病越来越重了。 

有一次,我们认和别的队赛篮球,我们的球队由他带领——说来你可能不信,我们班长会打篮球。球艺虽然不高,但常使对方带伤,有时是胸腔积血,有时是睾丸血肿;他可是个了不起的中锋,我们队就指着他的勇悍赢球——两支队伍立在篮球场上。对方的队长念了—段毛主席语录。轮到他时,他居然顺顺当当讲出话来,也不带x,这使我们这些想鼓掌的人很是失望。谁知他被当裁判的指导员恶狠狠地吹了一哨,还训斥他道:最高指示是最高指示,革命口号是革命口号,不可以乱讲!然后就他就被换下场来,脸色铁青坐在边上。原来他说了一句:最高指示,毛主席万岁!指导员觉得他讲得不对。最高指示是毛主席的话,他老人家没有说过自己万岁。所以这话是不对。但我总觉得不该和质朴的人叫真,有思想就行了嘛。自从被吹了一哨,我们班长就不敢说话了,带X不带X的话都不敢说,几乎成了哑巴…… 

当年那些时髦话语都表达了—个意思,那就是对权力的忠顺态度——这算不上什么秘密,那个年月提倡的就是忠字当头。但是同样的话,有人讲起来觉得害臊,有人讲起来却不觉得害臊,这就有点深奥。害臊的入不见得不忠、不顺,就以我们班长而论,他其实是个最忠最顺的人,但这种忠顺是他内心深处的感情,实际上是一种阴性的态度,不光是忠顺,还有爱,所以不乐意很直露地不惧肉麻地当众披露。我们班长的忠顺表现在他乐意干活,把地种好;但让他在大庭广众中说这些话,就是强人所难。用爱情来打比方,有些男性喜欢用行动来表示爱情,不喜欢把“我爱你”挂在嘴上。我们班长就是这么一种情况。另外有些人没有这种感觉,讲起这些话来不觉得肉麻;但是他们内心的忠顺程度倒不见得更大——正如有些花花公子满嘴都是“我爱你”,真爱假爱却很难说。 

如前所述,我插队的地方民风淳朴,当地人觉得当众表示自己的雌伏很不好意思;所以“有思想”这种状态,又成了“害臊”的同义语。不光是我们班长这么想.多数人都这么想。这件事有我的亲身经历为证:有一次我在集上买东西,买的是一位傣族老大娘的菠萝蜜。需要说明的是,当地人以为知青都很有钱.同样—件东西,卖给我们要贵三倍,所以我们的买法是趁卖主不注意,扔下合理的价钱,把想买的东西抱走。有人把这种买法叫作偷,但我不这么想——当然,我现在也不这么买东西了。那一天我身上带的钱少了,搁下的钱不怎么够。那位傣族老太大——用当地话来说,叫作蔑巴——就大呼小叫地追了过来,朝我大喝一声:不行啦!思想啦!斗私批修啦……然后趁我腰一软,腿一颤,把该菠萝蜜——又叫作牛肚子果——抢了回去。如你所知,这位蔑巴说这些有思想的话,意思是:你不害臊吗!这些话收到了效果,我到现在想起了这件事,还觉得羞答答的:为吃口牛肚子果,被人说到了思想上去,真是臊死了。

\chapter{苏东坡与东坡肉}

我父亲是教逻辑的教授,我哥哥是修逻辑的Ph·D。我自己对逻辑学也有兴趣,这种兴趣是从对逻辑学家的兴趣发展来的:本世纪初年,罗素发现了以自己名字命名的悖论,连忙写信告诉弗雷泽,顺便通知弗雷泽,他经营了半生的体系、因为这个悖论的发现有了重大的漏洞。弗雷泽考虑了一番,回信说:我要是知道什么是正确的结论就好了,……我觉得这个弗雷泽简直逗死了,他要是有女儿,我一定要娶了做老婆,让他做我的老岳丈,话又说回来,就算弗雷泽有女儿。做我的姥姥一定比做老婆合适得多。这样弗雷泽就不是我的老岳丈,而是我的曾外公啦。我在美国上学时还遇见过一件类似的事:有一回在课堂上,有个胖乎乎的女同学在打瞌睡,忽然被老师叫起来提问。可怜她根本没听,怎么能答得上来。在美国,不但老师可以问学生,学生也可以问老师。万一老师被问住,就说一句:问得好!不回答问题,接着讲课。这位女同学迷迷糊糊,拖着长声说道:This is a good question、(问得好)……差点把大家的肚皮笑破。下课后,我打量了她好半天,发现她太胖,又有狐臭。这才打消了不轨之心——弗雷泽就有这么逗。让我们书归正传,另一个有趣的逻辑学家是维特根斯坦,罗素请他来英国,研究一下出书的问题。 

维特根斯坦没有路费,又不肯朝罗素借。最后罗素买下了维特根斯坦留在剑桥的一些旧家具——我觉得他们俩都很逗。受这种浅薄的幽默感驱使,我学过数理逻辑,开头还有兴趣,后来学到了烦难的东西,就学不进去了。 

我对数学也有过兴趣,这种兴趣是从对方程的兴趣发展来的。人们老早就知道二次方程有公式解,但二次以上的方程呢?在十九世纪以前,人们是不知道的。在十七世纪,有个意大利数学家,又是一位教授,他对三次方程的解法有点心得。有天下午,外面下着雨,在教室里,他准备对学生讲讲这些心得。忽听“喀嚓”一声巨响,天上打下来个落地雷,擦着教室落在花园里——青色的电光从狭窄的石窟照进来,映得石墙上一片惨白。教授手捂着心口,对学生们转过身来,说道:先生们,我们触及了上帝的秘密,......一我读到这个故事时,差点把肠子笑断了。三次方程算个啥,还值得打雷——教授把上帝看成个小心眼了。数学我也学了不少,学来学去没了兴趣,也搁下了。类似的学科还有物理学、化学,初学时兴趣都很大,后来就没兴趣了,现在未必记得多少。 

总而言之,我对研究学问这件事和研究学问的人有兴趣,对这门学问本身没什么兴趣。所有的功课我都是这么学的,但我的成绩竟都是五分。只有一门功课例外,那就是计算机编程。我学的时候还要穿纸带,没意思透了。这一门学科里没有名人轶事,除了这门科学的奠基人图林先生是同性恋,败露后自杀了。我既不是同性恋,也不想自杀,所以我对计算机没兴趣,得的全是三分。但我现在时常用得着它,所以还要买书看看,关心一下最新的进展,以免用时抓瞎。这是因为我写文章的软件是自己编的,别人编的软件我既使不惯,也信不过,就这么点原因。但就因为这点小原因,我在编程序这件事上,还真正有点修为。由此可见,对研究某种学问这件事感兴趣和对这门学问本身感兴趣可以完全是两回事。 

这篇小文章想写我的心路历程,但有一件别人的事情越过了这个历程,我决定也把它写上。“文革”中期,我哥哥去看一位多年不见的高中同学。走进那间房子,我哥哥被惊呆了:这间房子有整整的一面被巨幅的世界地图占满了。这位同学身着蓝布大褂,足蹬布底的黑布鞋,手掂红蓝铅笔,正在屋里踱步,而且对家兄的出现视而不见。据家兄说,这位先生当时梳了个中分头,假如不拿红蓝铅笔,而是挟着把雨伞,就和 那张伟大领袖去安源的画一模一样了。我哥哥耐心地等待了一会儿,才小声问道:能不能请教一下……你这是在干吗呢?他老人家不理我哥哥,又转了两圈,才把手指放到嘴上,说道:嘘,我在考虑世界革命的战略问题。然后我哥哥就回家来,脸皮乌紫地告诉我此事。然后我们哥俩就捧腹大笑,几乎笑断了肠子…… 

罗素、弗雷泽研究逻辑,是对逻辑本身感兴趣,要解决逻辑领域的问题,正如毛主席投身革命事业,也是对革命本身感兴趣,要解决中国社会的问题。在解决问题的过程中,这些先辈自然会有些事迹,让人很感兴趣。如果把对问题本身的兴趣抹去,只追求这些事迹,就显得多少有点不对头。所以,真正有出息的人是对名人感兴趣的东西感兴趣,并且在那上面做出成就,而不是仅仅对名人感兴趣。 

古时候有位书生,自称是苏东坡的崇拜者。有人问他:你是喜欢苏东坡的诗词呢,还是喜欢他的书法?书生答道:都不是的。我喜欢吃东坡肉……东坡肉炖得很烂,肥而不腻,的确很好吃。但只为东坡肉来崇拜苏东坡,这实在是个太小的理由。

\chapter{一只特立独行的猪}

  插队的时候,我喂过猪、也放过牛。假如没有人来管,这两种动物也完全知道该怎样生活。它们会自由自在地闲逛,饥则食渴则饮,春天来临时还要谈谈爱情;这样一来,它们的生活层次很低,完全乏善可陈。人来了以后,给它们的生活做出了安排:每一头牛和每一口猪的生活都有了主题。就它们中的大多数而言,这种生活主题是很悲惨的:前者的主题是干活,后者的主题是长肉。我不认为这有什么可抱怨的,因为我当时的生活也不见得丰富了多少,除了八个样板戏,也没有什么消遣。有极少数的猪和牛,它们的生活另有安排。以猪为例,种猪和母猪除了吃,还有别的事可干。就我所见,它们对这些安排也不大喜欢。种猪的任务是交配,换言之,我们的政策准许它当个花花公子。但是疲惫的种猪往往摆出一种肉猪(肉猪是阉过的)才有的正人君子架势,死活不肯跳到母猪背上去。母猪的任务是生崽儿,但有些母猪却要把猪崽儿吃掉。总的来说,人的安排使猪痛苦不堪。但它们还是接受了:猪总是猪啊。

  对生活做种种设置是人特有的品性。不光是设置动物,也设置自己。我们知道,在古希腊有个斯巴达,那里的生活被设置得了无生趣,其目的就是要使男人成为亡命战士,使女人成为生育机器,前者像些斗鸡,后者像些母猪。这两类动物是很特别的,但我以为,它们肯定不喜欢自己的生活。但不喜欢又能怎么样?人也好,动物也罢,都很难改变自己的命运。

  以下谈到的一只猪有些与众不同。我喂猪时,它已经有四五岁了,从名分上说,它是肉猪,但长得又黑又瘦,两眼炯炯有光。这家伙像山羊一样敏捷,一米高的猪栏一跳就过;它还能跳上猪圈的房顶,这一点又像是猫——所以它总是到处游逛,根本就不在圈里呆着。所有喂过猪的知青都把它当宠儿来对待,它也是我的宠儿——因为它只对知青好,容许他们走到三米之内,要是别的人,它早就跑了。它是公的,原本该劁掉。不过你去试试看,哪怕你把劁猪刀藏在身后,它也能嗅出来,朝你瞪大眼睛,噢噢地吼起来。我总是用细米糠熬的粥喂它,等它吃够了以后,才把糠对到野草里喂别的猪。其他猪看了嫉妒,一起嚷起来。这时候整个猪场一片鬼哭狼嚎,但我和它都不在乎。吃饱了以后,它就跳上房顶去晒太阳,或者模仿各种声音。它会学汽车响、拖拉机响,学得都很像;有时整天不见踪影,我估计它到附近的村寨里找母猪去了。我们这里也有母猪,都关在圈里,被过度的生育搞得走了形,又脏又臭,它对它们不感兴趣;村寨里的母猪好看一些。它有很多精彩的事迹,但我喂猪的时间短,知道得有限,索性就不写了。总而言之,所有喂过猪的知青都喜欢它,喜欢它特立独行的派头儿,还说它活得潇洒。但老乡们就不这么浪漫,他们说,这猪不正经。领导则痛恨它,这一点以后还要谈到。我对它则不止是喜欢——我尊敬它,常常不顾自己虚长十几岁这一现实,把它叫做“猪兄”。如前所述,这位猪兄会模仿各种声音。我想它也学过人说话,但没有学会——假如学会了,我们就可以做倾心之谈。但这不能怪它。人和猪的音色差得太远了。

  后来,猪兄学会了汽笛叫,这个本领给它招来了麻烦。我们那里有座糖厂,中午要鸣一次汽笛,让工人换班。我们队下地干活时,听见这次汽笛响就收工回来。我的猪兄每天上午十点钟总要跳到房上学汽笛,地里的人听见它叫就回来——这可比糖厂鸣笛早了一个半小时。坦白地说,这不能全怪猪兄,它毕竟不是锅炉,叫起来和汽笛还有些区别,但老乡们却硬说听不出来。领导上因此开了一个会,把它定成了破坏春耕的坏分子,要对它采取专政手段——会议的精神我已经知道了,但我不为它担忧——因为假如专政是指绳索和杀猪刀的话,那是一点门都没有的。以前的领导也不是没试过,一百人也治不住它。狗也没用:猪兄跑起来像颗鱼雷,能把狗撞出一丈开外。谁知这回是动了真格的,指导员带了二十几个人,手拿五四式手枪;副指导员带了十几人,手持看青的火枪,分两路在猪场外的空地上兜捕它。这就使我陷入了内心的矛盾:按我和它的交情,我该舞起两把杀猪刀冲出去,和它并肩战斗,但我又觉得这样做太过惊世骇俗——它毕竟是只猪啊;还有一个理由,我不敢对抗领导,我怀疑这才是问题之所在。总之,我在一边看着。猪兄的镇定使我佩服之极:它很冷静地躲在手枪和火枪的连线之内,任凭人喊狗咬,不离那条线。这样,拿手枪的人开火就会把拿火枪的打死,反之亦然;两头同时开火,两头都会被打死。至于它,因为目标小,多半没事。就这样连兜了几个圈子,它找到了一个空子,一头撞出去了;跑得潇洒之极。以后我在甘蔗地里还见过它一次,它长出了獠牙,还认识我,但已不容我走近了。这种冷淡使我痛心,但我也赞成它对心怀叵测的人保持距离。

  我已经四十岁了,除了这只猪,还没见过谁敢于如此无视对生活的设置。相反,我倒见过很多想要设置别人生活的人,还有对被设置的生活安之若素的人。因为这个原故,我一直怀念这只特立独行的猪。

\chapter{我在荒岛上迎接黎明}

我在荒岛上迎接黎明.太阳初升时,忽然有十万支金喇叭齐鸣.阳光穿过透明的空气,在暗蓝色的天空飞过. 

在黑暗尚未褪去的海面上燃烧着十万支蜡烛.我听见天地之间钟声响了,然后十万支金喇叭又一次齐鸣. 我忽然泪如雨下,但是我心底在欢歌.有一柄有弹性的长剑从我胸中穿过,带来了剧痛似的巨大快感. 这是我一生最美好的时刻,我站在那一个门坎上,从此我将和永恒连结在一起.因为确确实实地知道我已经胜利,所以那些燃烧的字句就在我眼前出现,在我耳中轰鸣.这是一首胜利之歌,音韵铿锵,有如一支乐曲。 我摸着水湿过的衣袋,找到了人家送我划玻璃的那片硬质合金.于是我用有力的笔迹把我的诗刻在石壁上,这是我的胜利纪念碑.在这孤零零的石岛上到处是风化石,只有这一片坚硬而光滑的石壁.我用我的诗把它刻满,又把字迹加深,为了使它在这人迹罕至的地方永久存在. 

在我小的时候,常有一种冰凉的恐怖使我从睡梦中惊醒,我久久地凝视着黑夜。我不明白我为什么会死. 

到我死时,一切感觉都会停止,我会消失在一片混沌之中.我害怕毫无感觉,宁愿有一种感觉会永久存在.哪怕它是疼. 

长大了一点的时候,我开始苦苦思索。我知道宇宙和永恒是无限的,而我自己和一切人一样都是有限的. 

我非常非常不喜欢这个对比,老想把它否定掉.于是我开始去思考是否有一种比人和人类都更伟大的意义. 

想明白了从人的角度看来这种意义是不存在的以后,我面前就出现了一片寂寞的大海.人们所做的一切都不过是些死前的游戏. 在冥想之中长大了以后,我开始喜欢诗.我读过很多诗,其中有一些是真正的好诗.好诗描述过的事情各不相同,韵律也变化无常,但是都有一点相同的东西.它有一种水晶般的光辉,好象是来自星星,真希望能永远读下去,打破这个寂寞的大海.我希望自己能写这样的诗.我希望自己也是一颗星星.如果我会发光,就不必害怕 

黑暗.如果我自己是那么美好,那么一切恐惧就可以烟消云散.于是我开始存下了一点希望---如果我能做到. 

那么我就战胜了寂寞的命运.

\chapter{拒绝恭维}

  在美国时,常看“笑星”考斯比的节目。有一次他讲了这么一个笑话:小时候,他以为自己就是耶稣基督。这是因为每次他一人在家时,都要像一切小鬼一样,把屋里闹得一团糟。他妈回家时,站在门口,看到家里像发过一场大水,难免要目瞪口呆,从嘴角滚出一句来:啊呀,我的耶稣基督……他以为是说他呢。这种事情经常发生,他的这种

想法也越来越牢
固,以至于后来到了教堂里,听到大家热情地赞美基督,他总以为是在
夸他,心里难免麻酥酥的,摇头晃脑暗自臭美一番。人家高叫“赞美耶稣我们的救主”,他就禁不住要答应出来。再以后,他爹他妈发现这个小鬼头不正常,除了给他两个大耳光,还带他去看心理医生;最后他终于不胜痛苦地了解到,原来他不是耶稣,也不是救世主——当然,这个故事讲到这个地步,就一点都不逗了。这后半截是我加上的。

  我小的时候,常到邻居家里去玩。那边有个孩子,比我小好几岁,经常独自在家。他不乱折腾,总是安安静静跪在一个方凳上听五斗橱上一个匣子——那东西后来我们拆开过,发现里面有四个灯,一个声音粗哑的舌簧喇叭,总而言之,是个破烂货——里面说着些费解的话,但他屏息听着。终于等到一篇文章念完,广播员端正声音,一本正经地说道:革命的同志们,无产阶级革命派的战友们……这孩子马上很清脆地答应了两声,跳到地上扬尘舞蹈一番。其实匣子里叫的不是他。刚把屁股帘摘掉没几天,他还远够不上是同志和战友,但你也挡不住他高兴。因为他觉得自己除了名字张三李四考斯比之外,终于有了个冠冕堂皇的字号,至于这名号是同志、战友还是救世主,那还在其次。我现在说到的,是当人误以为自己拥有一个名号时的张狂之态。对于我想要说到的事,这只是个开场白。

  当你真正拥有一个冠冕堂皇的字号时,真正臭美的时候就到了。有一个时期,匣子里总在称赞革命小将,说他们最敢闯,最有造反精神。所有岁数不大,当得起那个“小”字的人,在臭美之余,还想做点什么,就拥到学校里去打老师。在我们学校里,小将们不光打了老师,把老师的爹妈都打了。这对老夫妇不胜羞辱,就上吊自杀了。打老师的事与我无关,但我以为这是极可耻的事。干过这些事的同学后来也同意我的看法,但就是搞不明白,自己当时为什么像吃了蜜蜂屎一样,一味地轻狂。国外的文献上对这些事有种解释,说当时的青春期少男少女穿身旧军装,到大街上挥舞皮带,是性的象征。但我觉得这种解释是不对的。我的同龄人还不至于从性这方面来考虑问题。

  小将的时期很快就结束了,随后是“工人阶级领导一切”的时期。学校里有了工人师傅,这些师傅和过去见到的工人师傅不大一样,多少都有点晕晕乎乎、五迷三道,虽然不像革命小将那么疯狂,但也远不能说是正常的。然后就是“三支两军时期”,到处都有军代表。当时的军代表里肯定也有头脑清楚、办事稳重的人,但我没有见到过。最后年轻人都被派往农村,接受贫下中农的再教育,学习后者的优秀品质。下乡之前,我们先到京郊农村去劳动,作为一次预演。那村里的人在我们面前也有点不够正常——寻常人走路不应该把两腿叉得那么宽,让一辆小车都能从中推过去,也不该是一颠一颠的模样,只有一条板凳学会了走路才会是这般模样。在萧瑟的秋风中,我们蹲在地头,看贫下中农晚汇报,汇报词如下:“最最敬爱的伟大领袖毛主席——我们(读做‘母恩’)今天下午的活茬是:领着小学生们敛芝麻。报告完毕。”我一面不胜悲愤地想到自己长了这么大的个子,居然还是小学生,被人领着敛芝麻;一面也注意到汇报人兴奋的样子,有些人连冻出的清水鼻涕都顾不上擦,在鼻孔上吹出泡泡来啦。现在我提起这些事情,绝不是想说这些朴实的人们有什么不对,而是试图说明,人经不起恭维。越是天真、朴实的人,听到一种于己有利的说法,证明自己身上有种种优越的素质,是人类中最优越的部分,就越会不知东西南北,撒起癔症来。我猜越是生活了无趣味,又看不到希望的人,就越会竖起耳朵来听这种于己有利的说法。这大概是因为撒癔症比过正常的生活还快乐一些吧——说到了这一点,这篇文章也临近终结。

  八十年代之初,我是人民大学的学生。有一回被拘到礼堂里听报告,报告人是一位青年道德教育家——我说是被拘去的,是因为我并不想听这个报告,但缺席要记旷课,旷课的次数多了就毕不了业。这位先生的报告总是从恭维听众开始。在清华大学时,他说:这里是清华大学,是全国最高学府呀;在北大则说:这里是有五四传统的呀;在人大则说:这是有革命传统的学校呀。总之,最后总要说,在这里做报告他不胜惶恐。我听到他说不胜惶恐时,禁不住舌头一转,鼻子底下滚出一句顶级的粗话来。顺便说一句,不管到了什么地方,我首先要把当地的骂人话全学会。这是为了防一手,免得别人骂我还不知道,虽然我自己从来不骂人,但对于粗话几乎是个专家。为了那位先生的报告我破例骂了一回,这是因为我不想受他恭维。平心而论,恭维人所在的学校是种礼貌。从人们所在的民族、文化、社会阶层,乃至性别上编造种种不切实际的说法,那才叫做险恶的煽动。因为他的用意是煽动一种癔症的大流行,以便从中渔利。人家恭维我一句,我就骂起来,这是因为,从内心深处我知道,我也是经不起恭维的。

\footnote{本篇最初发表于1996年第8期《三联生活周刊》杂志。}

\chapter{盛装舞步}

  初入大学的门槛,我发现有个同学和我很相像:我们俩都长得人高马大,都是一副睡不醒的样子,而且都能言善辩。后来发现,他不仅和我同班,而且同宿舍,于是感情就很好。每天吃完了晚饭,我要在校园里散步,他必在路边等我,伸出手臂说:年兄请——这家伙把我叫做年兄,好像我们是同科的进士或者举人。我也说:请。于是就手臂挽着手臂(有点像一对情人),在校园里遛起弯来,一路走,一路高谈阔论。像这个样子在美国是有危险的,有些心胸狭隘的家伙会拿枪来打我们。现在走在上海街头恐怕也

不行,但是七十年代末八十
年代初,在北京的一所校园的角落里遛遛,还没什么大问
题。当然,有时也有些人跟在我们身后,主要是因为这位年兄博古通今,满肚子都是典故;而我呢,如你所知,能胡编是我吃饭的本事,我们俩聊,听起来蛮有意思的。有些同班同学跟着我们,听我们胡扯——从纪晓岚一路扯到爱因斯坦,这些前辈在天之灵听到我们的谈话内容可能会不高兴。到了期中期末,功课繁忙,大家都去准备考试,没人来听我们胡扯,散步的就剩下我们两个人。

  我们俩除了散步,有时还跳跳踢踏舞。严格地说,还不是踢踏舞。此事的起因是:这位年兄曾在内蒙插队,对马儿极有感情,一看到电视上演到马术比赛,尤其是盛装舞步,他马上就如痴如狂。我曾给他出过这样的主意:等放了暑假,你回插队的地方,弄匹马来练练好了。他却说:我们那里只有小个子蒙古马,骑上去它就差不多了,怎忍心让它来跳舞——再说,贫下中牧也不会答应,他们常说:糟蹋马匹的人不得好死。然后,他忽然有了一个重要的发现:啊呀年兄,咱们俩合起来是四条腿,和马的腿一样多嘛!……他建议我们来练习盛装舞步,我也没有不同意见——反正吃饱了要消消食。两条大汉扣着膀子乱跳,是有点古怪,但我们又不是在大街上跳,而是在偏僻小路上跳,所以没有妨碍谁。再说,我们俩都是出了名的特立独行之士,无论是老师,还是学生干部,全都懒得来管我们。后来有一天,有个男同学经过我们练习舞步的地方——记得他是上海人,戴副小眼镜——他看了我们一阵,然后冲到我们面前来说:像你们俩这样可不行——不像话。说完就走了。

  这位同学走了以后,我们停了一会儿。年兄问道:刚才那个人说了什么?我说:不知道。这个人好像有毛病——咱们怎么办?年兄说:不理他,接着跳!直到操练完毕,我们才回宿舍拿书,去阅览室晚自习。第二天傍晚,还在老地方,那位小眼镜又来了。他皱着眉头看了我们半天,忽然冲过来说:那件事还没公开化呢!说完就又走了。这回我们连停都懒得停,继续我们的把戏。但不要以为我们是傻子,我知道人家说的那件事是同性恋。很不巧的是,我们俩都是坚定的异性恋者,我的情况尚属一般,年兄不仅是坚定的异性恋,而且还有点骚——见了漂亮女生就两眼放光,口若悬河。当然,同样的话,年兄也可以用来说我。所以实际情况是:说我们俩是同性恋,不仅不正确,而且很离谱。那天晚上那位眼镜看到的,不是同性恋者快乐的舞蹈,而是一匹性情温良的骏马在表演左跨步……文化人类学指出,不同文化、不同价值观的人之间,会发生误解,明明你在做这样一件事,他偏觉得你在做另外的事,这就是件误解的例子。你若说,我们不该引起别人的误会,这也是对的。但我们躲到哪儿,他就追到哪儿,老在一边乱嘀咕。

  我和年兄在校园里操练舞步,有人看了觉得很可耻,但我们不理睬他。我猜这个人会记恨我们,甚至在心里用孟夫子的话骂我们:“无耻之耻,无耻矣!”我们不理他,是因为他把我们想错了。顺便说一句,孟老夫子的基本方法是推己及人,这个方法是错误的。推己往往及不了人,不管从谁那儿推出我们是同性恋都不对,因为我们不是的。但这不是说,我们拒绝批评。批评只要稍微有点靠谱,我们就听。有一天,我们正在操练舞步,有个女同学从那儿经过,笑了笑说:狗撒尿。然后飘然而去。我们的步法和狗撒尿不完全一样,说实在的,要表演真正的狗撒尿步法,非职业舞蹈家不可,远非我二人的胯骨力所能及;但我们忽然认为,盛装舞步还是用马匹来表演为好。

  我早就从大学毕业了,靠写点小文章过活,不幸的是,还是有人要误解我。比方说,我说人若追求智慧,就能从中得到快乐;就有人来说我是民族虚无主义者——他一点都不懂我在说什么。他还说理性已经崩溃了,一个伟大的、非理性的时代就要降临。如此看来,将来一定满世界都是疯子、傻子。我真是不明白,满世界都是疯子和傻子,这就是民族实在主义吗?既然谁都不明白谁在说些什么,就应该互不答理才对。我在这方面做得不错,我从来不看有痰气的思辨文章(除非点了我的名),以免误解。至于我写的这种幽默文章,也不希望它被有痰气的思辨学者看到。

\chapter{科学的美好}

  我原是学理科的,最早学化学。我学得不坏,老师讲的东西我都懂。化学光懂了不成,还要做实验,做实验我就不行了。用移液管移液体,别人都用橡皮球吸液体,我老用嘴去吸——我知道移液管不能用嘴吸,只是橡皮球经常找不着——吸别的还好,有一

回我竟去吸浓
氨水,好像吸到了陈年的老尿罐里,此后有半个月嗓子哑掉了。做毕业论
文时,我做个萃取实验,烧瓶里盛了一大瓶子氯仿,滚滚沸腾着,按说不该往外跑,但我的装置漏气,一会儿就漏个精光。漏掉了我就去领新的,新的一会儿又漏光。一个星期我漏掉了五大瓶氯仿,漏掉的起码有一小半被我吸了进去。这种东西是种麻醉药,我吸进去的氯仿足以醉死十条大蟒。说也奇怪,我居然站着不倒,只是有点迷糊。在这种情况下,我还把实验做了出来,证明我的化学课学得蛮好。但是老师和同学一致认为我不适合干化学。尤其是和我在一个实验室里做实验的同学更是这样认为,他们也吸进了一些氯仿,远没我吸得多,却都抱怨说头晕。他们还称我为实验室里的人民公敌。我自己也是这样想的:继续干化学,毒死我自己还不要紧,毒死同事就不好了。我对这门科学一直恋恋不舍:学化学的女孩很多,有不少长得很漂亮。

  后来我去学数学,在这方面我很有天分。无论是数字运算,还是公式推导,我都像闪电一样快,只是结果不一定全对。人家都说,我做起数学题来像小日本一样疯狂:我们这一代人在银幕上见到的日本人很多,这些人总是头戴战斗帽,挺着刺刀不知死活地冲锋,别人说我做数学题时就是这么个模样。学数学的女孩少,长得也一般。但学这门科学我害不到别人,所以我也很喜欢。有一回考试,我看看试题,觉得很容易,就像刮风一样做完了走人。等分数出来,居然考了全班的最低分。找到老师一问,原来那天的试题分为两部分,一半在试题纸的正面,我看到了,也做了。还有一半在反面,我根本就没看见。我赶紧看看这些没做的题,然后说:这些题目我都会做。老师说,知道你会,但是没做也不能给分。他还说什么“就是要整整你这屁股眼大掉了心的人”。这就是胡说八道了。谁也不能大到了这个地步。一门课学到了要挨整的程度,就不如不学。

  我现在既不是化学家,也不是数学家,更不是物理学家。我靠写文章为生,与科技绝缘——只是有时弄弄计算机。这个行当我会得不少,从最低等的汇编语言到最新潮的C++全会写,硬件知识也有一些。但从我自己的利益来看,我还不如一点都不会,省得整夜不睡,鼓捣我的电脑,删东加西,最后把整个系统弄垮,手头又没有软件备份。于是,在凌晨五点钟,我在朋友家门前踱来踱去,抽着烟;早起的清洁工都以为我失恋了,这门里住着我失去的恋人,我在表演失魂落魄给她看。其实不是的,电脑死掉了,我什么都干不了,更睡不着觉。好容易等到天大亮了,我就冲进去,向他借软件来恢复系统——瞎扯了这么多,现在言归正传。我要说的是:我和科学没有缘分,但是我爱科学,甚至比真正的科学家还要爱得多些。

  正如罗素先生所说,近代以来,科学建立了一种理性的权威——这种权威和以往任何一种权威不同。科学的道理不同于“夫子曰”,也不同于红头文件。科学家发表的结果,不需要凭借自己的身份来要人相信。你可以拿一枝笔,一张纸,或者备几件简单的实验器材,马上就可以验证别人的结论。当然,这是一百年前的事。验证最新的科学成果要麻烦得多,但是这种原则一点都没有改变。科学和人类其他事业完全不同,它是一种平等的事业。真正的科学没有在中国诞生,这是有原因的。这是因为中国的文化传统里没有平等:从打孔孟到如今,讲的全是尊卑有序。上面说了,拿煤球炉子可以炼钢,你敢说要做实验验证吗?你不敢。炼出牛屎一样的东西,也得闭着眼说是好钢。在这种框架之下,根本就不可能有科学。

  科学的美好,还在于它是种自由的事业。它有点像它的一个产物互联网(Internet)——谁都没有想建造这样一个全球性的电脑网络,大家只是把各自的网络连通,不知不觉就把它造成了。科学也是这样的,世界上各地的人把自己的发明贡献给了科学,它就诞生了。这就是科学的实质。还有一样东西也是这么诞生的,那就是市场经济。做生意的方法,你发明一些,我发明一些,慢慢地形成了现在这个东西,你看它不怎么样,但它还无可替代。一种自由发展而成的事业,总是比个人能想出来的强大得多。参与自由的事业,像做自由的人一样,令人神往。当然,扯到这里就离了题。现在总听到有人说,要有个某某学,或者说,我们要创建有民族风格的某某学,仿佛经他这么一规划、一呼吁,在他画出的框子里就会冒出一种真正的科学。老母鸡“格格”地叫一阵,挣红了脸,就能生一个蛋,但科学不会这样产生。人会情绪激动,又会爱慕虚荣。科学没有这些老病,对人的这些毛病,它也不予回应。最重要的是:科学就是它自己,不在任何人的管辖之内。

  对于科学的好处,我已经费尽心机阐述了一番,当然不可能说得全面。其实我最想说的是:科学是人创造的事业,但它比人类本身更为美好。我的老师说过,科学对中国人来说,是种外来的东西,所以我们对它的理解,有过种种偏差:始则惊为洪水猛兽,继而当巫术去理解,再后来把它看做一种宗教,拜倒在它的面前。他说这些理解都是不对的,科学是个不断学习的过程。我老师说得很对。我能补充的只是:除了学习科学已有的内容,还要学习它所有、我们所无的素质。我现在不学科学了,但我始终在学习这些素质。这就是说,人要爱平等、爱自由,人类开创的一切事业中,科学最有成就,就是因为有这两样做根基。对个人而言,没有这两样东西,不仅谈不上成就,而且会活得像一只猪。比这还重要的只有一样,就是要爱智慧。无论是个人,还是民族,做聪明人才有前途,当笨蛋肯定是要倒霉。大概是在一年多以前吧,我写了篇小文章讨论这个问题,论证人爱智慧比当笨蛋好些。结果冒出一位先生把我臭骂一顿,还说我不爱国——真是好没来由!我只是论证一番,又没强逼着你当聪明人。你爱当笨蛋就去当吧,你有这个权利。

\footnote{科学的美好本篇最初发表于1997年第1期《金秋科苑》杂志。发表时题目为“向科学学习什么”。}

\chapter{优越感种种}

  我在美国留学时,认识不少犹太人——教授里有犹太人,同学里也有犹太人。我和他们处得不坏,但在他们面前总有点不自在。这是因为犹太教说,犹太人是上帝的选民;换言之,只有他们可以上天堂,或者是有进天堂的优先权,别人则大抵都是要下地狱的。我和一位犹太同学看起来都是一样的人,可以平等相交,但也只是今生今世的 事。死了以后就会完全 两样:他因为是上帝的选民,必然直升天堂;而我则未被选中, 所以是地狱的后备力量。地狱这个地方我虽没去过,但从书上看到了一些,其中有些地方就和全聚德烤鸭店的厨房相仿。我到了那里,十之八九会像鸭子一样,被人吊起来烤——我并不确切知道,只是这样猜测。本来可以问问犹太同学,但我又不肯问,怕他以为我是求他利用自己选民的身份,替我在上帝面前美言几句,给我找个在地狱里烧锅炉的事干,自己不挨烤,点起火来烤别人——这虽是较好的安排,但我当时年轻气盛,傲得很,不肯走这种后门。我对犹太同学和老师抱有最赤诚的好感,认为他们既聪明,又勤奋;就是他们节俭的品行也对我的胃口:我本人就是个省俭的人。但一想到他们是选民,我不是选民,心里总有点不对劲。

  我们民族的文化里也有这一类的东西:以天朝大国自居,把外国人叫做“洋鬼子”。这虽是些没了味的老话,但它的影响还在。我有几位外国朋友,他们有时用自嘲的口气说:我是个洋鬼子。这就相当于我对犹太同学说:选民先生,我是只地狱里的烤鸭。讽刺意味甚浓。我很不喜欢听到这样的话——既不愿听到人说别人是鬼子,也不愿听人说自己是洋鬼子。相比之下,尤其不喜欢听人说别人是洋鬼子。这世界上各个民族都有自己的文化,这些文化都有特异性,就如每个人都与别人有些差异。人活在世上,看到了这些差异,就想要从中得出于己有利的结果。这虽是难以避免的偏执,但不大体面。我总觉得,这种想法不管披着多么深奥的学术外衣,终归是种浅薄的东西。

  对于现世的人来说,与别人相较,大家都有些先天的特异性,有体质上的,也有文化上的。有件事情大家都知道:日耳曼人生来和别的人有些不同:黄头发、蓝眼睛、大高个儿,等等。这种体质人类学上的差异被极个别的混账日耳曼人抓住,就成了他们民族优越的证据,结果他们就做了很多伤天害理的事。犹太民族则是个相反的例子:他们相信自己是上帝的选民,但在尘世上一点坏事都不做。我喜欢犹太人,但我总觉得,倘他们不把选民这件事挂在心上,是不是会好些?假如三四十年代的欧洲犹太人忘了这件事,对自己在尘世上的遭遇可能会更关心些,对纳粹分子的欺凌可能会做出更有力的反抗:你也是人,我也是人,我凭什么伸着脖子让你来杀?我觉得有些被屠杀的犹太人可能对上帝指望得太多了一点——当然,我也希望这些被屠杀的人现在都在天堂里,因为有那么多犹太人被纳粹杀掉,我倒真心希望他们真是上帝的选民;即使此事一真,我这非选民就要当地狱里的烤鸭,我也愿做这种牺牲——这种指望恐怕没起好作用。这两个例子都与特异性有关。当然,假如有人笃信自己的特异性一定是好的,是优越、正义的象征,举一千个例子也说服不了他。我也不想说服谁,只是想要问问,成天说这个,有什么用?

  还有些人对特异性做负面的理解。我知道这么个例子,是从人类学的教科书上看来的:在美国,有些黑人孩子对自己的种族有自卑感,觉得白孩子又聪明又好看,自己又笨又难看。中国人里也有崇洋媚外的,觉得自己的人种不行,文化也不行。这些想法是不对的。有人以为,说自己的特异性无比优越是惟一的出路,这又使我不懂了。人为什么一定用一件错事来反对另一件错事呢?除非人真是这么笨,只能懂得错的,不能懂得对的,但这又不是事实。某个民族的学者对本民族的人民做这种判断,无异是说本族人民是些傻瓜,只能明白次等的道理,不能懂得真正的道理,这才是民族虚无主义的想法。说来也怪,这种学者现在甚多,做出来的学问一半像科学,一半像宣传;整个儿像戈培尔。戈培尔就是这样的:他一面说日耳曼人优越,一面又把日耳曼人当傻瓜来愚弄。我认识一个德国人,一提起这段历史,他就觉得灰溜溜的见不得人。灰溜溜的原因不是怀疑本民族的善良,而是怀疑本民族的智慧:“怎么会被纳粹疯子引入歧途了呢?那些人层次很低嘛。”这也是我们要引以为戒的啊。

\footnote{优越感种种:本篇最初发表于1996年8月2日《南方周末》。}

\chapter{有关天圆地方}

  现在我经常写点小文章,属杂文或是随笔一类。有人告诉我说,没你这么写杂文的!杂文里应该有点典故,有点考证,有点文化气味。典故我知道一些,考证也会,但就是不肯这么写。年轻时读过莎翁的剧本《捕风捉影》,有一场戏是一个使女和就要出嫁的小姐耍贫嘴,贫到后来有点荤。其中有一句是这么说的:“小姐死后进天堂,一定是脸朝上!”古往今来的莎学家们引经据典,考了又考,注了又注,文化气氛越来越浓烈,但越注越让人看不懂。只有一家注得简明,说:这是个与性有关的、粗俗不堪的比喻。这就没什么文化味,但照我看来,也就是这家注得对。要是文化氛围和明辨是非不可兼得的话,我宁愿明辨是非,不要文化氛围。但这回我想改改作风,不再耍贫嘴,我也引经据典地说点事情,这样不会得罪人。

  罗素先生说,在古代的西方,大概就数古希腊人最为文明,比其他人等聪明得多。但要论对世界的看法,他们的想法就不大对头——他们以为整个世界是个大沙盘,搁在一条大鲸鱼的背上。鲸鱼又漂在一望无际的海上。成年扛着这么个东西,鲸鱼背上难受,偶尔蹭个痒痒,这时就闹地震。古埃及的人看法比他们正确,他们认为大地是个球形,浮在虚空之中。埃及人还算过地球的直径,居然算得十分之准。这种见识上的差异源于他们住的地方不同:埃及人住在空旷的地方,举目四望,周围是一圈地平线,和蚂蚁爬上篮球时的感觉一模一样,所以说地是个球。希腊人住在多山的群岛上,往四周一看,支离破碎,这边山那边海。他们那里还老闹地震,所以就想出了沙盘鲸鱼之说。罗素举这个例子是要说,人们的见识总要受处境的限制,这种限制既不知不觉,又牢不可破——这是一个极好的说明。

  中国古人对世界的看法是:天圆地方,人在中间,堂堂正正,这是天经地义。谁要对此有怀疑,必是妖孽之类。这是因为地上全是四四方方的耕地,天上则是圆圆的穹隆盖,睁开眼一看,正是天圆地方。其实这说法有漏洞,随便哪个木匠都能指出来:一个圆,一个方,斗在一起不合榫。要么都圆,要么都方才合理,但我不记得哪个木匠敢跳出来反对天经地义。其实哪有什么天经地义,只有些四四方方的地界,方块好画呀。人自己把它画出来,又把自己陷在里面了。顺便说一句,中国文人老说:三光日月星,还自以为概括得全面。但随便哪个北方的爱斯基摩人听了都不认为这是什么学问。天上何止有三光?还有一光——北极光!要是倒回几百年去,你和一个少年气盛的文人讲这些道理,他不仅听不进,还要到衙门里去揭发你,说你是个乱党——其实,想要明白些道理,不能觉得什么顺眼就信什么,还要听得进别人说。当然,这道理只对那些想要知道真理的人适用。

\footnote{有关天圆地方:本篇最初发表于1996年12月20日《南方周末》。}

\chapter{高考经历}

 1978年我去考大学。在此之前,我只上过一年中学,还是十二年前上的,中学的功课或者没有学,或者全忘光。家里人劝我说:你毫无基础,最好还是考文科,免得考不上。但我就是不听,去考了理科,结果考上了。家里人还说,你记忆力好,考文科比较有把握。我的记忆力是不错,一本很厚的书看过以后,里面每个细节都能记得,但是书里的人名地名年代等等,差不多全都记不得。

  我对事情实际的一面比较感兴趣:如果你说的是种状态,我马上就能明白是怎样一种情形;如果你说的是种过程,我也马上能理解照你说的,前因如何,后果则会如何。不但能理解,而且能记住。因此,数理化对我来说,还是相对好懂的。最要命的是这类问题:一件事,它有什么样的名分,应该怎样把它纳入名义的体系——或者说,对它该用什么样的提法。众所周知,提法总是要背的。我怕的就是这个。文科的鼻祖孔老夫子说,必也正名乎。我也知道正名重要。但我老觉得把一件事搞懂更重要——我就怕名也正了,言也顺了,事也成了,最后成的是什么事情倒不大明白。我层次很低,也就配去学学理科。

  当然,理科也要考一门需要背的课程,这门课几乎要了我的命。我记得当年准备了一道题,叫做十次路线斗争,它完全是我的噩梦。每次斗争都有正确的一方和错误的一方,正确的一方不难回答,错误的一方的代表人物是谁就需要记了。你去问一个基督徒:谁是你的救主?他马上就能答上来:他是我主耶稣啊!我的情况也是这样,这说明我是个好人。若问:请答出著名的十大魔鬼是谁?基督徒未必都能答上来——好人记魔鬼的名字干什么。我也记不住错误路线代表人物的名字,这是因为我不想犯路线错误。但我既然想上大学,就得把这些名字记住。“十次路线斗争”比这里解释的还要难些,因为每次斗争都分别是反左或反右,需要—一记清,弄得我头大如斗。坦白说,临考前一天,我整天举着双手,对着十个手指一一默诵着,总算是记住了所有的左和右。但我光顾了记题上的左右,把真正的左右都忘了,以后总也想不起来。后来在美国开车,我老婆在旁边说往右拐,或者往左拐我马上就想到了陈独秀或者王明,弯却拐不过来,把车开到了马路牙子上,把保险杠撞坏。后来改为揪耳朵,情况才有好转,保险杠也不坏了——可恨的是,这道题还没考。一门课就把我考成了这样,假如门门都是这样,肯定能把我考得连自己是谁都忘掉。现在回想起来,幸亏我没去考文科——幸亏我还有这么点自知之明。如果考了的话,要么考不上,要么被考傻掉。

  我当年的“考友”里,有志文科的背功都相当了得。有位仁兄准备功课时是这样的:十冬腊月,他穿着件小棉袄,笼着手在外面溜达,弓着个腰,嘴里念念叨叨,看上去像个跳大神的老太婆。你从旁边经过时,叫住他说:来,考你一考。他才把手从袖子里掏出来,袖子里还有高考复习材料,他把这东西递给你。不管你问哪道题,他先告诉你答案在第几页,第几自然段,然后就像炒豆一样背起来,在句尾断下来,告诉你这里是逗号还是句号。当然,他背的一个字都不错,连标点都不会错。这位仁兄最后以优异的成绩考进了一所著名的文科大学——对这种背功,我是真心羡慕的。至于我自己,一背东西就困,那种感觉和煤气中毒以后差不太多。跑到外面去挨冻倒是不困,清水鼻涕却要像开闸一样往下流,看起来甚不雅。我觉得去啃几道数学题倒会好过些。

  说到数学,这可是我最没把握的一门课,因为没有学过。其实哪门功课我都没学过,全靠自己瞎琢磨。物理化学还好琢磨,数学可是不能乱猜的。我觉得自己的数学肯定要砸,谁知最后居然还及了格。听说那一年发生了一件怪事:京郊某中学毕业班的学生,数学有人教的,可考试成绩通通是零蛋,连个得零点五分的都没有。把卷子调出来一看,都答得满满的,不是白卷。学生说,这门课听不大懂,老师让他们死记硬背来的。不管怎么说吧,也不该都是零分。后来发现,他们的数学老师也在考大学,数学得分也是零。别人知道了这件事都说:这班学生的背功真是了得。不是吹牛,要是我在那个班里,数学肯定得不了零分——老师让我背的东西,我肯定记不住。既然记不住,一分两分总能得到。

\footnote{本篇最初发表于1997年第11期《三联生活周刊》杂志。}
