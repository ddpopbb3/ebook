\chapter{立新街甲一号与昆仑奴}

我住在立新街甲一号的破楼里。庚子年间,有一帮洋主子在此据守,招来了成千上万的义和团大叔,把它围了个水泄不通。他们搬来红衣炮、黑衣炮、大将军、过江龙、三眼铳、榆木喷、大抬杆儿、满天星、一声雷、一窝蜂、麻雷子、二踢脚、老头冒花一百星,铁炮铜炮烟花炮,鸟枪土枪滋水枪,装上烟花药、炮仗药、开山药、鸟枪药、耗子药、狗皮膏药,填以榴弹、霰弹、燃烧弹、葡萄弹、臭鸡蛋、犁头砂、铅子儿砂,对准它排头燃放,打了它一身窟窿,可它还是挺着不倒。直到八十多年后,它还摇摇晃晃地站着,我还得住在里面。 

这房子公道讲,破归破,倒也宽敞。我一个人住一个大阁楼,除了冬天太冷,夏天太热,也说不出有什么不妥当。但是我对它深恶痛绝,因为十几年前我住在这里时,死了爹又死了妈,从此成了孤儿。住在这里我每夜都做噩梦,因此我下定决心,不搬出去就不恋爱,不结婚。古代一位将军出门打仗,下令“灭此朝食”,不把对面那帮狗娘养的杀个净光净,绝不开饭!他的兵都有一条皮带,把肚子束紧,所以一个个那么苗条可爱。我的决心也这么坚定。隆冬的傍晚,我和小胡在炉边对坐,我说在这小屋里结婚是对我的侮辱。古人形容男女弄玉吹萧时有诗云:小楼吹彻玉笙寒。在这个破楼前吹玉笙,不相宜,只能吹洋铁皮喇叭,不像谈恋爱,倒像收破烂。古人云,要做东床快婿、这个阁楼里就这么一张床,如何去做?古人形容夫妻相敬,有言道,举案齐眉。准在我这屋里个案,小心憧了脑袋。古人形容夫妻相戏,有词云:嚼烂红绒,笑向檀郎唾。要是一位女士误嫁人我这狗窝,恐怕唾过来的不是红绒,是一口粘痰。 

小胡说,她也有同感。她要嫁出去,不住这个破房子。俗话称出嫁为出阁,那就是要搬出这个破楼阁。古诗云:雕栏玉砌应犹在,只是朱颜改。试问此楼,雕栏何在?玉砌何在?古词云:佳人难得,倾国。别人连国都倾了,她却倾不了一个破楼,真她娘没道理!所以她就等着那一天,要“仰天长笑出门去”!出门者,嫁人也。长笑一声出了这狗窝,未婚夫乘大号奔驰车来接。阿房宫,八百里,未央宫,深如水。自古华厦住佳人,不成咱是个蓬头鬼? 

听了她这个长歌行,我心里真有点不高兴。当时我们俩正在煤球炉上涮羊肉,炉台上放着韭花酱、卤虾油一类的东西。我偷眼看看她,只见此人高大粗壮,毛衣里凸出两个大乳房,就如提篮里露出两棵大号洋白菜,粗胳膊粗腿。吃得发热时满脸通红,脑袋上还梳一条大辫子,越发显得大得不得了。她骑在我的椅子上,那椅子那么单薄,我和椅子都提心吊胆,等着那咔嚓一声。咔嚓之前是椅子,咔嚓之后是劈柴。看来她还没本钱,勾上一位高于子弟搬出去,让这破楼里只剩我一个人和耗子做伴儿。她这么吹嘘,纯是出于一股自恋倾向。 

吃完了羊肉她告退,回自己房里做画去了。此女风雅如是,是何家闺秀耶?她是电影院画广告牌儿的。和我一样,是无亲无故的一条光杆儿。本小生志向不凡,官居何职抑袭何爵耶?我是豆制品厂磨豆浆的。我比她还不如,她还上了几年美专,鄙人只是个熟练工,除了开闸放水泡豆子,合电门开钢磨磨豆浆,大约并无什么可吹嘘的。那一天她走以后,我站在窗前,只见窗外银花飞舞,天地同色,就想到一千多年前,王二在雪地里卖狗肉汤时,也是如此的寂寞而凄凉。那时候正是唐初盛世,长安城里有四方人物。王二在小巷里别人房檐下支起几片草排,在炭火池中安一个瓦罐,罐里就是他要卖掉的狗肉汤。那时候天色向晚,外面飞旋的雪幕后已经显出淡淡的灰色。王二坐在条凳上,毡鞋被雪水湿透了,说不出的寒冷。他把脚放到炭火中去烤。可炭火将熄,也没有什么暖意。没有人来买他的狗肉汤,一个也没有。 

地上的雪越来越厚,天快黑了。有一个黑人从对面人家的后门里出来。天寒地冻,他却只围一块腰布;肌肤黑如墨亮如漆,在雪中倒算是相映生趣。黑人身上的肌肉才叫肌肉,块块隆起又不粗笨。他头上一层短短的卷发,圆鼻子圆脸,一双圆眼睛,看上去很好玩。那黑人说:“王老板,你卖完了没有?如果卖完了还有汤剩下,请给我一碗。我冷得受不了,你的汤真是御寒的妙品!” 

这位黑哥们儿常来要汤喝,平常王二也就给他了。可是今天他心情坏,不想给他这碗汤,就说: 

“昆仑奴,你老来喝汤,却不给钱。这碗汤是白来的吗?煮这碗汤要用伢狗肉。你来想一想:这伢狗出了娘胎,好不容易长到这么大,人却不容它与小母狗亲热,就把它打死煮进了汤锅!你再看我这煨汤的瓦罐,它是清明前河底的寒泥烧成,所以才经火不炸。挖泥时河水好不寒冷,只有童子之身才能抵挡得住。所以年老的瓦工一辈子都不敢亲近女人。你再看这汤里的胡椒桂叶,全是南国生成,飘洋过海到泉州,走万里水旱路到黄河边。黄河的航船过三门,要从激流中上行到关中。千人挽,万人撑。一个不小心落下水,那就尸骨无存。一碗汤不足惜,可是中间有多少血和泪!你闲着没事儿一碗一碗地喝,这可不大对劲!” 

昆仑奴说:“王老板,我知道这汤来得不容易,可是我身上冷,需要这碗汤来御寒。我生在东非草原上,哪见过雪,哪见过冰?这都是因为酋长卖我做奴隶。我在地中海上摇船,背上挨了鞭子,又浇上海水!人家把我在拜占庭卖掉,我又渡过水色如墨的黑海,赤足走过火热的沙漠,爬过冰川雪山,涉过陷人的流沙河。如今在伟大的长安城里,天上下着大雪,我却没有御寒的衣服。猫和狗都有充足的食物,可是我在挨饿!真主啊,请你为我的苦难做证!难道人身为奴隶,就不配在隆冬喝一碗御寒的狗肉汤?你让我向谁去求得怜悯?主人吗?富人的心是皮革做的。王老板,一碗汤对你算得了什么?你不会因此变穷的!” 

有好多雪片飞到昆仑奴身上,在那儿融化,变成雪水流下去。王二把他拉到草棚里来,让他在身边坐下,接过他的大碗,舀一碗热汤给他。他拍拍黑人的脊梁说:“昆仑奴,喝吧!” 

昆仑奴喝汤时,王二看着乱纷纷的雪幕背后楼台的轮廓,心里有说不出的感慨,这种远眺华厦的感觉,古今并无不同。我站在窗前,看到脚下是一片平阔的雪地,雪地那边是新楼。那楼不算好看,不过它叫我想起很多地名,楼上有广西柳州的水泥,如果那边也在下雪,雪花会在竹林间飞舞,南来避寒的候鸟会不知所措地瞅瞅。秦皇岛的玻璃———一想到秦皇岛,就想起在冬季灰色的海面上行进的大轮船。钢制的门窗与石景山紫色的烟雾有关。送暖的暖气片产在河北南皮县。南皮我没去过,不过这个地名有历史感——曹操和袁绍在那儿打过仗。袁绍的兵穿鱼鳞铁甲,曹操的兵的皮甲上镶着铜星。可是在我的屋顶上满是窟窿,叫人想起渔光曲——爹爹留下这张网,靠它还要过一冬。铁斗里的煤球叫人想起煤炭铺里穿长衫的胖掌柜,还有恶霸地主牟二黑子。王二站在这破屋檐下,身穿工作服,瘦长脸上面色阴沉,而一位穿红毛衣的少女在新楼里倚着雪白的窗纱远眺雪景。这种感觉,古今无不同。雪景也是古今无不同。昆仑奴喝下一碗热汤,黑檀似的身躯上有了光泽。王二看了很高兴,就说: 

“昆仑奴,到我家去吧,我要招待你。” 

昆仑奴也很高兴,收起木碗,随王二走过铺满了白雪的小巷。那时候他就如白玉的棋盘上一枚黑色的棋子。走到王二那用木片搭起的小屋门前,他惊叹一声: 

“原来中国也有穷人呀!” 

王二生起炭火,用狗油炒狗肝,把狗肉干在火上烤软。他烫热了酒,把菜和肉放在短几上,端到席上去。昆仑奴坐在他对面,披着狗皮。他们开始吃喝、谈笑,度过这漫漫长夜。当户外梨花飞舞,雪光如昼时,人不想沉沉睡去。这种感觉,古今无不同。 

小胡睡不着觉,爬上来聊天。聊天可以,你该问问我困不困。可是她根本不想办这个手续。她坐在我对面,谈到和男朋友吹了的事。这话题使我感到屈辱,因为我没有任何女朋友。然后她又说我个儿矮。混账,你说我个矮,我就说你腿粗。她说腿粗跑步可以治,个矮只有压面机能治。这真是岂有此理,她盼我跳压面机自杀,好得我的遗产。我这个人有好古癖,收藏颇丰、除了破椅子破床板,我还有一箱子线装书。当然,珍本善本是没有的。那些书用纪念章、邮票和豆腐干换不来。我有这么一批书:《三字经》、《千家诗》、《罗通扫北》、《小五义》、《南唐二主词》、《太平广记》、《朱子语类》、《牛马经》、《麻衣神相》、《南华经》、《净土经》,还有光绪十年的皇历。为这些破书,逼我惨死,可谓狠毒矣。地下室还有一批破烂,那一年游承德捡的普陀宗胜之庙房上的铜瓦;游东陵拣回的一个琉璃兽头;长城上的砖头;黄陵边的瓦片。北京修地铁,挖出的各种破烂,其中有一奇形木片,经我考证那是元代穷人买不起手纸用的刮具。此物大英博物馆都没有收藏,可谓无价之宝。小胡逼我死掉,大概志在得此奇珍异宝。 

小胡说,那件宝贝她不想要。她不惟不希望我早死,还盼我能活得长久。所以她要帮我解决困难,为我介绍女朋友。现在的男子身高不足一米八十者,都被列入二级残废。我之身高尚不足一米七,属于微生物一级,女孩子根本看不见。她要起到显微镜的作用,让她们通过她看到我。说完这些伤天害理的话,她打了个呵欠下楼睡觉去了。 

她走以后,我心里很不安定。我有三种感觉:第一是屈辱感,这不必解释,是因为我个儿矮。第二是施恩图报的感觉。本人系有大恩于小胡者。十几年前,在同一天,因为同一个事故,我们俩都成了孤儿。当时我们是中学生,在同一个中学读书,同住在这座破楼里,因为这些共同点,我对她是有求必应。半夜她要上厕所,总把我从阁楼上叫下来,在门前站岗。每隔五秒钟她叫我名字,有一次不应她马上嚎出来。她可是一面出清直肠一面叫我的,这种一心二用的方式是不是挺可恶?要没有我,她早被屎憋死啦!如今她在我面前,居然不避圣讳说出一个矮字来,良心何在!第三,我对她还有一种嫉妒之心。此人五体不全之阴人耳,居然上了美专。而我是如此地热爱艺术,也画一手好素描,就进不了美专的门。这只是因为我有点色弱,红的绿的分不大清楚。其次,她长得比我还高。当然,她极为粗笨。不过嫉妒心一上来,我又觉得她高大健美,和观音菩萨差不多。这桩事儿不能想,一想奇妒难熬。 

这三种感觉,即屈辱感、图报感、嫉妒感,正是古今一般同。那天晚上昆仑奴在王二家问:“王老板,你家里怎么没有女人服侍?”王二心里的屈辱感就油然而生。在唐朝的长安城里,一个又贫又贱的小贩,就如现时之一位一米六八的二级工,根本搞不到对象。此时王二家里灯光如豆,雪光映壁,火盆里炭火熊熊,昆仑奴头上起了油汗。王二双手把一盆烩狗筋捧到昆仑奴面前,昆仑奴接下来,放在案上。王二又取一把铜勺,在衣襟上一拭,再次双手捧到昆仑奴面前,昆仑奴接下来,放在羹盆边。这都是对待贵客的礼节,王二做得一丝不苟。因此他想:昆仑奴,你是一个奴隶。我把你请到家里来,待以上宾之礼,希望你也自觉一点,别问人家难堪的问题。 

谁知那黑人又问:“王老板,难道你也像我们奴隶一样,没女人服伺吃饭吗?”王二一听,更加不悦。他想:你要不识趣,别怪我也问出不好听的来。于是他说: 

“昆仑奴,听说你们是树上结的果子,是真的吗?” 

昆仑奴一听,把眼珠子都瞪圆了,说:“谁说的?人还有树上结的吗?你们唐朝人都是树上结的?” 

“我们当然是母亲生的啦!但是你们就不同了。听说非洲有一种大树,名为黑檀,高有百丈,粗有十人不能合抱者,锯之则流血。树叶大如蒲团,树枝上脐带挂着一树的小黑孩。自挂果至成熟,历时十个月,熟则坠地,能言语能行走。波斯商人在树下等着,捡起来贩为奴隶。因为是树生的果实,所以男身者,有男之形无男之实,不能御女成胎;女形者有女之态无女之实,亦不能怀孕生子。我们大唐只有皇帝才得用阉人为太监,所以王侯之家不惜以重金购进黑奴,在内宅中服务。也许你不是树上结的,不过别的黑人却可能是树上结的?” 

昆仑奴说这是谣言,非洲绝没有能结出人的树。黑人也如其他人一样,是母亲腹中所生。在非洲时,每逢旱季,他也常和肤色黝黑的女子到草原上去,在空旷无人的所在性交,到下一个雨季,小娃娃就出生了。那些娃娃的皮肤也如黑玉一般,闪着光泽,叫人想起蓝天下那些快乐时光。那时草原上吹着白色的热风,羚羊、斑马、大象、猎豹,都在干同样的事。他知道这谣言的来源,因为黑奴很值钱,所以主人很希望他们能够增殖。他们往往把男女黑奴关在一个笼子里,但是结果总让他们失望。笼子不是草原,笼子里没有草原上的风。笼里的女人也是奴隶,谁乐意传下奴隶的孽种!啊,黑非洲,黑非洲!说到非洲,昆仑奴哭起来。 

王二又问,公侯内宅里的姑娘,难道不漂亮吗?她们对昆仑奴不好吗?昆仑奴对那些女孩,难道就没有感情?昆仑奴说,那些姑娘都像月亮一样的漂亮,心地也很善良。她们对他也很好。如果他挨了鞭子,她们就会伸出嫩葱般的手指来抚摸他的黑脊梁,洒下同情的眼泪。昆仑奴挨饿的时候,她们还省下点心给他吃。昆仑奴也爱她们,不过那只是一种兄妹之情。于是王二想,他是多么地身在福中不知福啊! 

昆仑奴说,在王二家里做客,又温暖又快活。下次他要带个姑娘来,让她也享受这种乐趣。三更时他起身告退,回主人家去,给王二留下嫉妒和期望。王二羡慕那黑人,有与美丽女郎朝夕相处的幸福,这种感觉,古今无不同。 

转眼间冬去春来,暖和的风从破楼一百多个窟窿里吹进来。从窗口往外看,北京城里一片嫩黄烟柳世界。在屋里也能感到懒洋洋的春意,这种感觉古今无不同。我想得到唐代的王二是怎么感觉春意的:当阳光照到桑皮纸糊的木格门上时,他把洗净的瓦罐放到格于下层。把辣椒、桂叶用纸包好,放到架子上层。如果它们经过雨季不发霉,下个冬天就不必再买。他取出铜锅,用柴灰擦去铜绿,准备去卖阳春面。心里在盘算煮汤的牛骨是什么价钱,青葱、嫩韭是什么价钱,面汤里放几滴麻油才合适。春意熏熏时,他做这种事感到兴奋,也许卖阳春面能多赚一点钱,胜过了狗肉汤。 

我也想为春天做点事:到长城边远足,到玉渊潭游泳,到西郊去看古墓,可是哪一样都做不成。西郊的古墓全没啦,上面盖了楼房。长城现在是马蜂窝,爬满了人。我也不像十几岁时了,要从历史中寻求安慰。二十岁以前,我和小胡在初春去游泳,从冷水里爬出来,小风一吹浑身通红。现在可不行,我见了冷水浑身发紫,嘴唇乌青,像老太太踩了电门一样狂抖。这都是因为抽了十几年烟,内脏受了损害。因此我只能一个人呆在家里。 

傍晚时分小胡回家来,站在楼梯口叫我。她可真是臭美得紧啦!头戴太阳帽,身穿鹅黄色的毛衣,细条绒的裤子,猪皮冒充的鹿皮鞋,背上背着大画夹,叫我下去看她的画。我马上想到本人夭折了的美术生涯,托故不去。过了一会儿,她又爬上来,身上换了一套天蓝色的运动装。这套衣服也是对我的伤害,因为它是我买来给自己穿的。穿了一天之后,发现别人看我的眼色不对劲儿。原来它是淡紫色的,这种颜色正是青春靓女们的流行色。演出了这场性倒错的丑剧之后,我只好把这套衣服送给她,让她穿上来刺激我。第一,我是半色盲,买衣服时必须由她来指导,如果自行出动,结果正合她意。第二,我个矮,我的衣服她也能穿。我正伤心得要流鼻血,她却说要报告我一个好消息。原来她给我介绍的对象就要到来,要我马上吃饭,吃饱后盛装以待。我就依计而行。饭后穿得体体面面地坐在椅子上出神儿,心里想这事不大对劲儿。我也应该给这位身高腿粗的伙计介绍个对象。我们车间的技术员圆头圆脑,火气旺盛,老穿一件海魂衫,像疯了一样奔来跑去,推荐给她正合适。正在想这个事,她在楼下喊我,我就下去,如待宰之绵羊走进她的房间。你猜我看见了什么?我看见一个娘们坐在床上,身上穿着葱绿的丝绵小夹袄,腿上穿一件猩红的呢子西装裤,足蹬千层底圆口布鞋。我这眼睛不大管事,所以没法确定她身上的颜色。该女人白净面皮,鼻子周围有几粒浅麻子,梳一个大巴巴头,看起来就如西太后从东陵里跑了出来。凭良心说,长得也还秀气,不过对我非常无礼。下面是现场记录,从我进了门开始: 

该女人举手指着我的鼻子,唉声嗲气地说:“就是他呀!” 

小胡坐到她身边去,说:“没错儿!” 

这就验明正身,可以枪毙了。该女人眯起眼睛来看我,这不是因为我和基督变容一样,光焰照人,而是这娘们要露一手职业习惯给我瞧瞧,她老人家是一位自封的画家。然后—— 

该女人又说:“行哦,挺有特点。鹰钩鼻子卷毛头,脸色有点黑,像拉丁人。” 

小胡浪笑几声说:“他在学校里外号就叫拉丁人!” 

该女人间:“脾气怎么样?”就如一位兽医问病时说:“吃草怎么样?” 

小胡说:“凶!在学校里和人打架,一拳把三合板墙打了个窟窿!他发了脾气,连我都敢打!不过一般来说,还算遵纪守法。” 

然后两个女人就咬起耳朵来,叽叽喳喳。我在一边抽烟,什么话也不说。过了一会儿,她送那娘们出去,又在过道里咬了半天耳朵。然后她回来间: 

“怎么样,你有什么看法?” 

我先问那女人走远了没有,得到肯定的答复后才说:“这算啥玩艺?一个老娘们嘛!而且还小看人!” 

她听了就皱起眉头来说;“你不觉得她很有性格,很有特点?” 

我说这人好像有精神病。她很不高兴,说这是她的好朋友,要我把嘴放干净点儿。后来她又说,对方还说可以谈呢,我这么坚决拒绝,真是岂有此理。我跟她说:你少跟我说这些,免得招我生气!说完我就回楼上去了。在那儿我想:我也不必给她介绍对象。不知为什么,这种事有点伤感情。 

过了半个钟头,小胡忽然很冲动地跑到楼上,脸色通红地宣布说,她发现自己干了件很糟糕的事,希望我不要介意。后来就没了下文。她好像在等我说下文,我又好像在等她的下文,于是就都发起呆来。这种窘境,也是古今一般同。春天的午夜,昆仑奴到王二家做第二次访问。他没和佳人携手而来,却背来了一个沉重的大包袱。王二担心这是赃物,他是本分买卖人,不愿当窝赃的窝主。他想叫昆仑奴把东西送回去,但是不好意思开口。他对昆仑奴还有所期待。 

我也不知自己在期待什么,只觉得嘴唇沉重,舌头沉重,什么也说不出。我就如唐之王二,默默地等待昆仑奴打开包袱。包袱里坐着一个绝代尤物。那是一位金发碧眼的女郎,穿着轻罗的衣服,皮肤像雪一样白,像银子一样闪亮。嘴唇像花一样红,像蜜糖一样湿润。她跳起来,在屋里走动,操着希腊口音说:“这就是自由人的住处吗?我闻到的就是自由的气味吗?” 

王二家里充满了烟味、生皮子味、霉味和臭味,可是她以为这就是自由的气息,大口地呼吸。她对什么都有兴趣,要王二把壁架上的纸包打开,告诉她什么是辣椒什么是桂叶,把梁上的葫芦里的种子倒出来,告诉她什么是葱籽,什么是菜籽。她还以为墙上挂的饼铛是一种乐器,男用的瓦夜壶是酒器。她就如一个记者一样问东问西,这也不足为奇。原来那些内院的姑娘都想出来看看,而她是第一个中选者。她有详尽报告的义务。后来她穿上王二的破衣服,用布包了头面,到外面走了一小圈,看过了外面的千家灯火,就回来吃自由的阳春面。她宣布自由的面好得很,但又不敢多吃。饭后他们三人同桌饮酒,女孩起身跳了一段胡旋艳舞。原来她正是跳胡旋舞的舞姬。 

胡旋舞在唐朝十分有名。一听胡旋两个字,光棍就口角流涎。女孩起舞时,把轻罗的衣服脱下来,浑身只穿了一条金锻子的三角裤,她的裸体美极了。王二把眼睛眯起来,尽量不看她那粉樱桃似的乳头,轮廓完美的胸膛,修长的玉腿,丝一般的美发。他的心脏感到重压,呼吸困难。就如久日饥渴的人见不得丰盛的酒筵。王二看到这位金发妖姬,也有点头晕。 

五更时,昆仑奴要回去,他把那位舞姬又打到包袱里。女孩儿说:“大哥,你让我露出头来看看外面好不好?”可是昆仑奴说不行。爬墙时树枝剐破了你的小脸儿主人间起来怎么说?咱们都要完蛋。他们就这样走了。不知为什么,王二微微感到有点失望。这个女人美则美矣,却像个幻影不可捉摸。他又寄希望于下一个来观光的女人,这种感觉,真是古今一般同。 

小胡在我对面坐了很久,我们什么都没有说。后来她微感失望地叹了一口气,这股窘意就过去了。她开始谈房子的事,听到这种话题,我也微感失望,但是我们还是就这个问题谈了很久。 

话头从甲一号的破楼扯起,它在庚子年间被打了一身窟窿,应该拆了,可是教皇不答应。他说当拳民攻击破楼时,上帝保佑了此楼,所以要让它永远不倒,以扬耶和华之威。他还说了些上帝不老,此楼不倒之类的疯活,然后请一位主教来修理此此楼。如果当时把这楼好好修修,它不至于这么破。可惜该主教把它用青灰抹了抹就卖给了一个商人。商人付款后,墙上的青灰落下来,他一看此楼是一副蜂窝煤的嘴脸,就对自己抠响了驳壳枪,最后血糊淋拉地跳进北海。然后这座破楼里住满了想自杀又没胆量的人们,自然是越来越破的没溜啦。 

这些解放前的事儿是我考证出来的。解放后,为置甲一号这破楼于死地,头儿们制定了上百个计划。计有大跃进建房计划、抓革命促生产扒旧楼建新楼计划、批林批孔建新楼计划、批臭宋江再建梁山计划、批倒“四人帮”盖新楼计划、房产复兴百年大规划、排干扰建房计划、拔钉子建房计划等等。但是这破楼老拆不倒,新房也建不起来。经事后分析,这房子有大批的反动派做后盾,计有(国外不计)右倾机会主义分子、走资派、林秃子、孔老二、“四人帮”、宋江、卢俊义、司马光、董仲舒、孟柯、颜回等等从中作祟。现在的反动派是小胡和我,我们俩赖着不搬,是钉子户。现在报纸上批钉子户,不弱于当年批宋江的火力。我实在为自己和宋江并列感到羞辱——他算什么玩艺儿?在水游传里没干一件露脸的事儿,最不要脸的是一刀桶死了如花少女阎婆惜。我确实想搬走,可是没地方可去。头儿们说,我在破楼里是寄居的性质,不能列入新楼计划。可是厂里有豆腐干往的地方,没我住的地方呀! 

小胡说,她也想搬出去,可是一到公司里要房,领导就勃然大怒说:“你也来闹事,在甲一号楼不是住得挺好的吗?”电影公司一到分房时,全体更年期妇女的脸就如猴屁股一样红起来,毛发也根根百立。老头子们就染头发,生怕分房前被列入退休名册。在这种情况之下,她只好把希望寄托在男朋友身上。如果嫁到有房的人家,剩下我一个就好办啦。甲一号还能不给我一套新房?春天到来,她穿上春装在街上一走,路边的男子回头率颇高。凭她这等身材相貌,嫁出去不成什么问题。所以我只有坐在家里,净等她的胜利消息! 

小胡的一切都是跟我学的,而且每一项都是青出于蓝。首先是我画两笔画,她也学着画,结果学出点名堂。现在光业余时间画小人书就有不少收入。我好古成癖,她也跟着学,结果画法有汉砖、敦煌画之风,在画坛上也小有名气。我会胡说八道,她也跟着学,从一个腼腆的小女孩,学到大嘴啦啦。我一长青春痘,就喊出要找对象的口号,不过一个也没找着。可是她谈过无数男朋友,常常搂着一个在楼道里“叭叽”,好像在向我示威。只有一样本事她没有学会,就是站着撒尿。 

夏天到了。豆腐厂改为一律早班,这样造出的豆腐,中午和下午上市,不用过夜,就不会酸。一到夏天我就困得死去活来,因为凌晨两点凉爽的时候,别人正睡得安稳,我却出门去蘑豆浆。到中午我回来时,阳光已经把薄铁皮的屋顶晒得火热。我在下面躺着,似睡非睡,似醒非醒,纯粹是发晕。到口干得不能忍受时,就喝脸盆里的清水。每天都能喝掉一盆。就这么熬到太阳偏西,阁楼才刚刚有点凉风,可以睡一会儿了,小胡又爬上来。这时我真盼她早点找到主儿嫁出去,哪怕嫁给宋江也罢! 

小胡上来时穿着短衫短裤,右手端着一个大碗。碗里是热气腾腾的馄钝汤。这么大热的大,她请我吃这种东西,简直就像潘金莲对付武大郎。左子提着的东西更可恶,那是一个水桶。她要借我的房子洗澡,把我轰到她房里去。她的房问朝西,现在就加点着了的探照灯。她来了我只好坐起来,看见她那对大奶于东摇西晃,我就如见了拳王阿里的拳头,太阳穴一阵阵发炸。顺手拿过镜子来一照,眼珠子通红。我说:“小胡,你不能这么干。我也是个人,他妈的,你怎么不给我人权?”这种话对她不起作用。她说:“呀!上来看看你不好吗?一天没见了,你不想我?”我什么都教给她了,就是没教她要脸,因为我自己也不要脸。后来她说,她上来不单是和我闲扯谈,还有要紧的事情。但是她说起这件要紧的事儿,又没有要紧的样子,倒像要给我上一大课。第一,这房子实在住不得了。夏天是这样热,以致她的头发不用去理发馆,自己就打起卷来。冬天呢,能把人冻死。春秋天刮大风,满屋都是沙土,可以练习跳远儿。除此之外,它还随时有可能塌倒。因此就有第二,有必要从这里搬出去。豆腐厂和电影公司不能解决这个问题。男朋友也爱莫能助。最后只剩下甲一号。她已经和头儿们谈了很多次,以我们两人的名义和他们谈条件。然后她就解释为什么自己去和人家谈判。她说这里绝无看不起我的意思,只是因为她是二十三级干部,而我是二级工。干部比较受人尊重,这是一个有利条件。而且她姓胡,胡这个姓比较少,所以容易引起重视。姓王的太多了,多到不成体统。所以姓王的去谈事情就没人答理。她就这么有一搭没一搭地胡扯,渐渐扯到没影的地方去。我知道她心里有鬼,就说:“你要说房子问题,就直说吧!” 

她的脸当时就红了,结巴着说:经过反复交涉,头儿们答应给一套房子,交换条件是两个人都搬出去。这有什么可脸红的?给一套你就先搬进去,我到头儿们问口搭小棚住。古人云,先有太极,后有两仪,两仪生四象,四象生八卦,八八六十四,循环无穷,乃孔明八阵图也。故而世上事,有一就有二,只怕他不松口。小胡说,你不要臭美,甲一号准不知咱俩是没溜儿的人?人家会轻易上当吗?这一套房子不是这么来的,她对人家说,我们是一对情人,不久就要结婚,当然这是骗他们的。说到这儿她愉眼看看我,我当然有点儿晕乎,不过没什么外在的表示。她就继续说下去:她告诉他们,在破楼里,我们俩天天演戏。半夜三更她会站在门口长叹一声: 

“啊,王二,王二,为什么你是王二?” 

我就说:“听了你的话,我从此不叫王二。”混充罗密欧与朱丽叶,在阳台说情话哩。或者是唱山歌“胡家溜溜的大姐,人材溜溜的好,王家溜溜的大哥,看上溜溜的她。”还唱越剧:“小别重逢胡XX!” 

这些鬼话我听了起了一身鸡皮疙瘩。就凭她那男性化的公鸭嗓和我这驴鸣似的歌喉,真要唱有可能把西山上的狼招来。头儿们听了将信将疑。要说信,我们俩在一个楼里住了多年,真要搞上了也算不上什么新闻。要说不信,谁不知这两个家伙大嘴啦啦,什么都敢说?头儿们就组织专案组去调查。首先查到十几年前给我们发抚恤金的会计,她说有一次我们没去领钱,她就给送来,发现我们两个小孩在楼道里十分亲呢地斗殴,敲到双方都是满头大包犹不肯住手,打完了架又在一个锅里吃饭。居委会的大娘们揭发了当年我带小胡爬树摘桑葚的事,以及某一天我出门时她从楼上探身出来大叫:“给我带包妇女卫中纸来,不带花了你!”最后的事例有小胡前天在小卖部给我买了一条男用针织裤权。专案组根据这些材料,下结论道:胡王恋爱一案,可以基本肯定。因此头儿们代表组织上宣布,什么时候交来结婚证和永不翻案(即离婚)的保证书,什么时候姓胡的和姓王的就能领到一套两居室的住房证和钥匙。她说为了这套房子我们可以假结婚,结了再离,房产科又不是法院,无法制止。 

虽然说是假结婚,她说起来还是有点结巴,我也有点儿喘。等到说完了这一节,她又辩才自如,立论说,由于假结婚,她将受到重大损失,将来再找对象时,人家总要怀疑她有个孩子养在乡下姥姥家。但是为了我们的共同福利她已不惜火中取栗。不知为什么我对她的胡扯失去了兴趣,就干脆说:“不必废话了,明天就去登记。” 

决定了这件事以后,小胡要洗澡,我按惯例该到她房里烤着去。可是今天本人别出心裁,从窗口爬上了房顶。一出来我就后悔了,因为太阳虽已西斜,屋顶的铁板还挺烙脚,坐下又觉得烙屁股。此时阁楼里已响起了溅水声,我欲旧无路,只好在房上吃完了馄饨,就坐下发傻。这时我看到一位少女从对面新楼里走出来,身穿洁白的连衣裙,真是秀色可餐。我以前没见过她,也不知道她叫什么名字,因此就爱心大炽。这种心境,正是古今一般同。 

话说王二和昆仑奴拉上了关系,就常在家里接待王侯家里的姑娘。他真是大开眼界,见过了跳肚皮舞的阿拉伯女郎,跳草裙舞的南洋少女,跳土风舞的黑人姑娘。这种女孩个个美得很,人也十分热情,不过他对她们只存欣赏之心,绝没动过爱欲。有一天昆仑奴说,他要带一位特殊的姑娘来,要王二早做准备。当然,特殊的姑娘也是奴隶,但是这一位身价不同。原来王侯家里的女奴分为三等,最下者为丫环仆妇。针线娘子洗衣妇,大抵是长安城里穷人家养不起卖给大户人家者,身价不过三两五两七两八两。门卫不禁止他们随意出门,所以也不必带她们出来。更高级的是歌姬舞娘,都是从四方贩来之绝色绝艺者,身价几十两、几百两不等,不能出门宅一步,王二看过的都是这种人。最高的身价在千两至万两之间,在内宅里养着,也不唱歌,也不跳舞,也不操家务,也不大吃,也不大喝,也不大走路,也不大说话,只管坐着充当摆设。如今有这么一位听说王二家好玩得要命,也要来看看。昆仑奴不好厚此薄彼,只得答应,他特地来关照王二,要他把家里好好收拾一下。于是王二把房子彻底清扫,换上一张新草席,借了上等茶具,就在家里静等。 

是夜昆仑奴来时,背了个极大的包,好像里面是大肚子弥勒佛。开包后先是三重棉絮,六层绸缎,八层轻纱,然后才是这位佳人。这是位中国少女,在席上坐得笔直,从始至终,眼帘低垂。她穿着白软缎的衣裙,脸色苍白有如贫血,面目极其娟秀,嘴极其小,鼻极其直,眉极其细,身材也极其苗条,肩极其削,腰极其细,手指极其细长,脚极其小。坐了许久,才发出如蚊鸣的细声,请求一口茶。王二急取黄泥炉,紫砂壶,燃神川之炭,烹玉泉之水,彻清明前之雀舌茶,又把细磁茶具洗涮二十通后,浅斟奉上。少女润唇之后,把茶杯放下,又坐半个更次,乃出细声曰: 

“多谢款待。盛情今生难报,留待来世。”然后就离去了。 

王二见过这位女郎,顿时失魂落魄,爱了个发昏章第十一。虽然她在他对面坐过,他却如在十里地之外见过她似的,回想起来只有一点模糊的轮廓。他想,这才是女人!极其高贵极其纯洁,想到她就有天上人间之感。这种感觉,正是古今一般同。 

第二天,我要和小胡登记结婚,这件事想起来就忐忑不安。等到阁楼没了声息,我从窗子里爬回去,只见桌子上留一张条子,上书: 

1、今晚不聊天了。 

2、明天下午三点钟办事处门口见,请着白色西服。 

3、明晚上我请客。 

屋子里到处是水渍,还有一种淡淡的石灰水气味。闻见这种味儿,就想起小胡来,觉得她很不错。古人云,环肥燕瘦各有态。她是属于环肥那一种。无论怎么说,我不能拒绝这种结论,即小胡是漂亮女孩。只要不是神经病似的非绝代佳人不娶,大概也可以满意了。 

当然,我对身轻如燕,举止端庄,沉默寡言者更为倾心。这种感觉,正是古今一般同。当年王二在家里见过这样一位佳人,就爱心大炽,一再托昆仑奴传后请她再来。她拒绝了好几次,最后终于来了,坐在王二对面,还是低垂着眼帘,什么都不说。王二一再劝诱她稍进饮食,她终于从盘里取一粒樱桃吃下去,流泪说道:“情孽。”然后又什么也不说了。到天明前,她和昆仑奴一起离去,王二想问她什么时候再来,但恐怕太唐突,就没有问。 

我一直睡不着。到半夜时分,小胡轻轻地爬上楼来,坐柱对面的椅子上,沉默了好久以后,忽然问我睡着了没有。她显然是明知故问。我翻身坐起来,看着窗前的月光。是夜有薄云,故而月光也如一抹石灰水,就如她身上白色的内衣一样淡薄。我想到如下事实: 

以前我们都有凌云壮志,非绝代佳人不娶,非白马王子不嫁。所谓绝代佳人者,自然是身轻如燕,沉默寡言者,而非高大健美,大嘴啦啦者。至于白马王子,身高一米九十以上,面白无须。因此我们结成同仇敌忾的统一战线,立志开拓我们的世界,看今夜的形势,只怕要壮志成灰。 

小胡忽然哭起来,提到如下事实: 

小时候她被人揪小辫子(其实是她先招惹了别人),要我给她撑腰,而我跑去以后,只要叉着腰在一边站着,喝道:“你揍他!我不信你揍不过!”她得了我的教唆,就扑过去又抓又咬。 

半夜里我叫她参加我的午夜行动,从窗户里爬出去骑在屋脊上。屋脊非常光削,她感觉它要把她从下到上一切两半,就像猪崽子一样嚎叫,却被我厉声喝止。下来以后我还打了她两拳,打在腰眼上。 

小胡说,这种行为很野蛮,我这么对待她不公道,她要求立即改变,因此我过去和她拥抱接吻。这种身体接触是平生第一次,我非常的兴奋。但是想起我的绝代佳人计划,又有点害羞。于是我放开她,回到板床上坐下,又觉得心有未曾。幸好她跟过来,两个人楼在一起,觉得很不错。我的手放肆起来,此时有如下想法: 

小胡和我这么搂着,实在是很自然的事。 

假结婚是扯谈。 

于是我说,现在我们这样,虽然非常之好,可是我的绝代佳人和她的白马王子计划岂不是完全失败?但是小胡说,现在很快活,这显然是伟大胜利,怎么能说是失败? 

那位绝代佳人第三次到王二家去,带了一个小丫头和很多东西。昆仑奴几乎背不动,当她和王二对坐无言时,小丫头就勤快地动起手来。先挂起罗销帐,又陈放好博山炉,在炉里点上檀香。她在草席上铺上猩猩毡,又在毡上铺上象牙细席,放上一对鸳鸯枕,就和昆仑奴到门外去嗑瓜子儿。王二和她静坐多时,终于拉着手到帐里去。在那儿他怀着虔诚的心情为她宽衣解带,扶她在席上躺下。然后定睛一看,席上是一个女人的裸体,并非什么不可思议的怪物,只不过腿非常细长,脐窝非常小而浅,腰非常细,乳房小而圆,非常精致,肋骨非常细,如同猫肋一样。王二就胆壮起来,先正襟危坐,如抚琴一般轻抚她身体三匝,又俯身在她的樱唇上一吻,然后就宽衣拉下帐子完成夫妇大礼的其他部分。 

我也和小胡行了夫妇之大礼,不过弄得不依古格,乱七八糟,就连我这嗜古成厮的人都不能克己复礼,可见人心不古,世道浇漓。但是礼毕时,我们俩都很满意。这种感觉,大概古今无不同。 根据史籍记载,王二和那位美女行过礼之后就逃到外乡去做豆腐为生,和我的职业一模一样。昆仑奴回主人家去。不久此事败露了,那位主人派了三十个兵去捉他,可是没想到这位黑先生在非洲以爬树捉猴子、跑步追羚羊为生。他见势不好,把木碗别在腰里拔腿就跑,大兵根本追不上,终于跑得无影无踪,音信全无,一直跑回非洲去了。
