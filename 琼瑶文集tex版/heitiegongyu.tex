\chapter{黑铁公寓}

一 

我很小时就离开了学校,做过各种各样的事情,现在我在学校里当电工。人家看到我时说:嘿,这小电工。他们说我怎么看都不像十八岁,想当电工就不能低于十八岁 ——这又有什么呢,岁数的问题我们来想办法。一年前我在开大货车,那时候我二十岁,警察看我不像,就塞点钱好了。两年前我在街上摆烟摊,人家问我多大了,我说二十五岁。今年我十八岁,真是越活越年轻了。你想要我几负,我就可以几岁,你要要什么证明文件我都能找来,要不然我还能在外面混吗?总而言之,我现在梳着油亮的分头,穿着贼亮的皮鞋,跷着二郎腿坐在传达室里,很像一位电工大爷,这可比驾车跑长途好多了。甭管驾驶证上几岁,我知道自己很爱打瞌睡,常把车开进沟里,开货车我是太小了点。摆烟摊受人欺负,又挣不来钱。而跟货车到新疆哈密瓜呢,我又吃不了这种苦。在机关学校里混事是最舒服的了。 

学校的入口立着两根粗大的门柱,门柱之间是紧闭着的黑漆铁栅栏大门。学生从旁门出入。经过传达室窗外时,他们盯着我看。我坐在看门教养的木板床上,看着自己的脚尖,偶尔把脚尖移开,朝痰盂吐口痰。我知道他们在看什么:这小子年纪轻轻,怎么不去上中学,跑到这里来坐着。这可叫没办法的事——俗话说得好,各人有各人的造化。我的造化还是小的,我有个表哥,比我大不了多少,已经做了多年的生意,挣了不少钱。现在他要百心竿头更进一步:他要开公寓了。 



所有上过小学的人都要上中学,所有上过中学的人都要上大学。所有上过大学的人,都必须住在有营业执照的公寓里。据说公寓里特别好,别人想住都住不进去。假如你生在我们的时代,对这些想必已经耳熟能详,但你也可能生在后世,所以我要说给你知道——假如有样东西人人都说好,那它必定不好,这是一定之理。 

所以假如你在上学的年龄,一定要从学校里逃掉,这是当务之急——逃掉以后怎么谋生就成了问题。我一直在给人打工,我表哥在做生意。做别的倒也罢了,他居然做起公寓来了。这行当不但对品行、阅历有种种要求,还要年满三十五岁周岁。要是我记的不错,我表哥顶多比我大一岁——也就是说,不满十八岁。但你到了他的面前一定会打消这一个想法:我表哥头顶光秃秃,两腮和月球的表面相仿。额头上有三道抬头纹,配上又黑又粗的眉毛和一脸奸笑,就像一根四十五岁的老油条,这都是吃药吃的。在眼前这个社会里,人只有过了求学的年纪才能有前途。在这方面,撒谎只能解决一部分问题。这家伙拿着类固醇、睾丸酮一类的药物当家常便饭来吃,还劝我也吃,但我可不想拿自己的身体来开玩笑。顺便说一句,这家伙不但手背、脚背、胸口、小腹上满是黑毛,连背上都长着。至于他那杆大枪,让人看了都替他害臊——说实话,我今年只有十六出头,我可不想长这种东西。 

我表哥先骗下了公寓管理员的证书,又骗下了公寓的营业执照,然后租下了学校对面的旧仓库,在里面装修房子。他说,我还是离你近点好,有事找你商量时近便些。他说自己最近经常一阵一阵地犯糊涂,脑子不管用了,照我看是吃药吃的。最近一段他住在我这里,每天早上,他拿几十片药,放在捣臼里捣碎,加把麦片用牛奶一冲,就那么吃下去,日久天长哪有不犯糊涂的。牛奶和麦片都是我买的,他从来就不买。连方便面他都不买,但却忘不了吃。他抽我的烟,喝我的茶,牙刷用他自己的,但使用我的牙膏。惟一肯往我这里拿的就是药,而我又不吃药。我看药他也没花钱买,准是找拣破烂的要的。拣破烂的什么药都能拣到,要知道有公费医疗。我表哥是个铁公鸡——一毛不拔。他还以此为荣,说道:要不然,我就攒出开公寓的钱了? 

有关我表哥,还可以说得更多一些:我们经常搭伙干事,他嫌我懒,我嫌他抠,所以总是弄不长。现在我们处于拆伙的状态:我当我的电工,他跑他的买卖。但不管他干什么,我还得去搭把手,理由很简单:总共就这一门亲戚。要是回家亲戚会多些,但我不敢回家——一进家门居委会就会找来,抓我去上工读学校,工读学校也是学校噢。 



我表哥的房子装修好了,他搬了过来,带着他的家具、杂物,还有六个房客。家具装在大卡车上,由搬家公司的人搬上楼去,房客装在一辆黑玻璃的面包车上,一直没有露面。那辆面包车窗子像黑铁公寓的窗子一样,装着铁栅栏,有个武装警卫坐在车里,还有几个站在了周围。等到一切都安顿好了,才把面包车的门打开,请房客们下车。原来这些房客都是女的。有两位有四十来岁,看上去像学校里的教授。有三位有三十来岁,看上去像学校里的讲师。还有一位只有二十多岁,像一个研究生,或者是高年级同学。大家都拖着沉重的脚镣,手里提着一个转塑料垃圾袋,里面盛着换洗衣服,只有那个女孩没提塑料袋。她们从车上下来,顺着墙根站成了一排,等着我表哥清点人数。 

我表哥搬家那天,北京城里刮着大风,天空被尘暴弄得灰蒙蒙的,照在地面上的阳光也变得惨白。有两位房客戴着花头巾,有三位房客戴着墨镜,其他人没有戴。我表哥说:老师们,搬家是好事情,大家高兴一点——这回的房子真不赖。但她们听了无动于衷,谁也不肯高兴。我想这是很自然的,披枷戴锁站在过往行人面前,谁也高兴不起来。我听说监狱里的犯人犯了错误时,就给他们戴上脚镣作为惩罚——这还是因为他们已经在监狱里,没别的地方可送了。给犯人戴的脚镣是生铁铸的,房客们戴的脚镣是不锈钢做的,样子小巧别致。但它仍然是脚镣,不是别的东西。我表哥干笑着说:脚镣是租来的,这不是搬家吗,万一跑丢一个就不好了——咱们平时不戴这种东西。我表哥像别的老北京一样,喜欢说“咱们”来套近乎,但我觉得他这个“咱们”十足虚伪,因为他没戴这种东西。这些房客里有五个戴着手铐或者拇指铐——这后一种东西也非常的小巧,像两个连在一起的顶针,把两手的大拇指铐在了一起。不过这也不是什么好东西,因为假如没有钥匙,不把大拇指砍掉是取不下来的,而把拇指砍掉了就会立刻成为残废。她们双手并在前面提着袋子,像动物园里的狗熊在作揖。我表哥又说,手铐出门时才戴,不是总戴着的。那个年轻的女孩倒是没戴手铐,双手被一条皮绳子反绑在了身后。她挺起胸膛,好像就要从容就义的样子。我表哥解释说:咱们讨厌手铐,所以用根绳子。我听说癌病房里的病人总拿死和别人开玩笑,已婚的女人和未婚的女人间总拿性来开玩笑,这些笑话也是“咱们、咱们”地说着吧。但我觉得我表哥的笑话十足虚伪,因为他自己并没有用根绳子吗。所有要住进公寓的人肘弯都扣着一根铁环,被一根铁链串在一起,只有我表哥例外。 

我表哥告诉我说,这六个房客是从劳动局领来的,都还不错,为此没少给主办人好处。他说他一早起来,租车、租铁链子、租脚镣,忙了个要死,刚才还满地爬着往别人脚上拴链子。他还抱怨我没去帮他的忙。这话没道理,我在学校里做事。人家找电工马上就得到,如果不到会炒了我的。虽然腰里挂着BP机,我也不敢走远了。他让我今天下午别走了——他进了六个大活人。他的意思是让我留下给他出出主意。我表哥被药物催的秃头秃脑,别人原看不出他几岁,但一张嘴就漏馅儿,别人别致了这些话,要是再猜不出我们是谁就是傻子了。我一直在偷眼看那皮绳反绑的女孩,只见她对身边一个房客说:欧阳,两个小流氓。小流氓想必是指我们了。我听了也不生气:我们俩岁数不大,而且的确不是好人。那位欧阳还不错,答道:小流氓就小流氓吧,总比老流氓强。——也不知强在哪里。我表哥耳朵聋没听见,要是听见了准要动手打人。对他这个人,我还是有一点了解的…… 

房客们都穿着郑重的秋季服装——呢子的上衣和裙子,这些衣服都是很贵的;脸上涂了很重的粉,嘴唇涂得鲜艳欲滴。只有一个人例外,那个年轻的女孩没有化妆。她穿着花格衬衫,袖子挽到肘上,那个扣住手臂的钢环被掩在袖子里。下襟束在腰带里,那条小牛皮的腰带好像是名牌。腿上穿着褪色的牛仔裤,脚下穿一双雪白的运动鞋。那条不锈钢的脚镣亮晶晶的,镣环扣在套着白袜子的脚腕上。背着手,姿势挺拔,四下张望着——她排在队尾。我一直盯住了她看,她的领口敞开着,露出了锁骨和一部分胸口,随着呼吸平缓地起伏着。后来她转过身去背对着我——她的小臂修长,手腕被黑色的皮箱纠缠着。有时候她握紧拳头,把双手往上举着,这样双臂就构成个愤怒的W形;有时又把手放下来,平静地搭在对岸的手臂上,这样就构成了一个平静的一字形。与此同时,别的房客低着头,一动都不动。直到一切都安顿好了,我表哥才说:好,进去吧。房客们从黑铁分寓的前门鱼贯而入,像一伙被逮住的女贼。那个女孩走在最后,她在我脚上踩了一脚,说:小坏蛋,看什么你?我翻翻白眼儿说:又看不坏,看看怎么了? 



二 

黑铁公寓是一座四四方方的混凝土城堡,从外面看起来是浅灰色的,但它名副其实,因为它里面非常的黑。在高高的天花板上,亮着一盏遥远的水银灯,照着这间宽大的房子,好像一座篮球馆内部的样子,但是这里没有篮球架子。从底层的中央乘长降机到达四楼,你会发现自己在十字交叉的通道的中心。每条通道通向一个窗子,窗子的大小刚够区别白天和黑夜。在通道两边,雕花的黑漆铁栏杆后面,就是黑铁公寓的房间——房间里的一切都一览无余——你怎么也不肯同意,像这样的小房间可以要那么多的房钱。但是人家也璋,他们径直把你推进其中的一间,然后你就得为这间房子付钱了。隆冬时节黑铁公寓里面流动着透明的暖风,从铺在地面上的橡胶地毯上方流过,黑铁公寓里面一尘不染,多亏了有效的中央空调系统。这里有第一流的房间服务——一日三餐都有人从铁门上的送饭口送进来。从这个口子送进来的还有内衣和卫生纸、袋装茶和袋装咖啡——在动物园里,人们也是这样给笼养的猛兽送东西,只是不送袋装咖啡——住在这个笼子里,你大概也用不着别的东西。这个地方过去是座旧仓库,现在是黑铁公寓。打听了这所公寓的房钱之后,你会得出这样一个结论:这黑铁公寓可真是够黑的。 

那个穿花格衬衫的女孩住在门口,她说我们是两个小流氓,如果说是指我们不肯上学流窜在外,那就说得完全对。但流氓还有一层意思,指在两性关系上行为不端的人。在这方面她只说对了一半。说对了一半——对的那半是我表哥。他和所有搞得到的女孩之间全都不干不净,满脑子都是下流主意,称为小流氓不为过。至于我呢,虽然从初二就离开了学校到社会上混事,但始终洁身自好,和一切女孩之间都是清白的。我喜欢知识,找了一大堆书在看,但我表哥呢,除了药典什么都不看……他身上的味也难闻,好像一个马厩。就这么个家伙,在房客面前还有点腼腆,和我小声嘀咕道:怎么办呢,这可都是些有学问的人哪。我说,还有什么怎么办的,先把那根穿羊肉串的签子拔了吧。我表哥看了我一眼,然后才领悟到这是指把房客们连在一起的铁链子。这些房客都站在公寓的走廊里,哪间房都进不去。他从口袋里掏出一大把小钥匙来给我,我就去开那些锁在手臂上的锁——这种小锁是人家锁信箱的,一块五一把。虽然也挣不开,但我表哥也够会省钱的了。每打开一个,那人就径直走开,走进自己房间里:谁住哪间房早就交待过了。开到队尾时,碰上了那个女孩。她瞪我一眼说:你才是羊肉串!我和表哥说话声音很轻,但她还是听见了。后来知道,她是个音乐家。音乐家耳朵不灵怎么成呢。 

在公寓装修好之前,表哥住在我宿舍里,睡在我双层床的上铺上。他在那时放响屁,声如裂帛。只要响上几次,屋里的气味就和山羊圈相仿。他还拿我的脸盆洗脸,洗过以后水都不倒——那水就如一锅隔宿的羊肉汤。那所公寓是我设计、我监工,预算也是我造的——平日好学不倦就有这种好处。遗憾的是用的全是他的钱,我表哥付清了给我的劳务费,所以公寓是他的。我表哥满肚子都是糠,但也有两点让人不能不佩服:一是能省钱,二是能吃苦。省钱的情形我说过了一些,但还没说到主要的:我们出去吃饭,他要把盘底的菜汤全舔光。不但舔自己桌上的,还舔邻桌上的。舔盘子不值得佩服,干着这种丑事,面不改色,坦坦荡荡,这就让人佩服了。至于吃苦,那真是没说的。大冬天到新疆去贩瓜,押闷罐车回来,车厢又不能喝酒——瓜见了酒味马上被催熟烂掉——跑上一趟回来,两个耳朵全生了冻疮,像贴了两摊干鸡屎。在澡堂子里泡两个小时,出门买张硬座票,又上路去新疆——这样做事你行么?当然,你要是贩过瓜,就知道主要的难处在于车过河南时,黑更半夜,当地那些苦哈哈撬开车门就抢瓜,此时你要抄起根棍子兜头就打,把头顶着的麻袋片、棉帽打飞,把脑子打出来。干这事我也行,要论心毒手狠,我们表兄弟俩差不太多。我就是吃不了苦,而我表哥就是上不了台面。房客都进了自己的房间,他还拿眼睛瞅我,问我该怎么办。我伸手挥动按钮开关,只听轰的一声响,所有的铁门一齐关住,把房客关了起来。表哥从口袋里拿出一块抹布(他管这叫手绢)擦擦脑门说:真该死!还忘了有这么个开关。表弟,你该一样一样再对我说说。我看他是慌的。现在走廊上空空荡荡,每个房客都坐在自己房间里的床上一声不吭。整个公寓在屋顶的水银灯光下鸦雀无声,看起来满像样的。表哥很高兴,说道:多么好啊。表弟,咱们拿出来捋一管吧——庆祝庆祝。他就喜欢做这种惊世骇俗的建议,以此显示自己是特立独行之士,倒不一定真要这么做。我说,这是你的公寓,要庆祝你庆祝,要捋你捋。房客在自己的笼子里听到了这样的鬼话,全都无动于衷,只有那个穿花格衬衫的女孩皱了一下眉头。 



把房客锁上以后,我们俩到办公室里喝咖啡。这间房子和房客的大屋不同,有一个很大的窗户。满屋黑色的家具,散发着一股醋酸味。假如我记得不错,冰醋酸是种粘合剂。这里的一切都是新的,brand new——我正在学英文,不知不觉就要来上一句。我舀了一些咖啡豆,放进磨里磨着。表哥躺进了黑皮沙发,马上又跳了起来,看着那些咖啡豆说:小二(这是我的小名),咱们是不是太过牛逼了?在我表哥的词典里,牛逼指奢华,还有很多辞意,在此不能一一开列。我告诉他说:不牛逼。我们喝掉咖啡,留着发票,就可以上账。这笔钱叫做管理旨,按国家的财务制度,最后算在房客头上。他听了满脸通红,说道:财务制度真,我算种上了铁杆庄稼了——当然,此间的牛逼,又是英文wonderful之意。他还让我帮他算算自己有多牛逼——此处之牛逼又是每月收入之意。我说你且慢牛逼,管不住房客有你的好看。上面吊销你的执照,叫你血本无归。他说:能管住的。今天这不是第一次慌了吗?然后他又说起第一次来,刚动手摸摸,自己就先流了——这是个下流比喻。我能听懂,但不接茬。后来我要回学校,表哥送我出来。走在走廊上,看到每个房客还规规矩矩坐在自己的床上,叉开双腿,眼睛看着我们——这好有一比,在幼儿园小班里,大家排除去屙屎,屙完不敢站起来,都在看阿姨的眼色。看来大家都懂规矩,这就省我表哥的事了。 

我和表哥走过走廊时,迎着每个客户的目光,心里微微有陶醉之意——尤其是当房客比较年轻,比较漂亮时,更是这样。走过403室门口时,迎上了那位欧阳的目光。这位房客肤色黝黑,身材颀长。除了穿花格衬衫的姑娘,这公寓里就属她漂亮。她朝我们一举铐住的双手说:就这么一直铐住我们吗?语调里颇有责怪之意。我们俩确实是忘了房客身上的镣铐应该早点打开,这是我们的不妥之处。照我看来,应该把别人的镣铐都打开,留着欧阳的,因为谁都不开口,显得她太牛逼。但我表哥不是这么理解问题,他一拍脑袋道:说得是!脚镣是租的,按小时算钱,得早点还哪。说着他就拿钥匙,打开每间房门,卸掉脚镣,把它们束成了一捆扛在肩上说:我去还脚镣,手铐你开吧。——说完就跑了。此后公寓里就剩了我一个人。在这座公寓里,有八座紧闭的笼门,里面有六个被束缚着的女孩。我手上有五把手铐的钥匙。 



三 

我逐一打开笼门,去给房客开手铐。如你所知,我没上过大学,连初中都没读完,但我绝非浅薄之士。我知道威严来自礼貌。每开一副手铐之前,我都微微躬躬身子说道:对不起了,阿姨。等手铐开了以后,她们都揉揉手说:谢谢。人家住公寓也不是一天两天了,油头粉面的小流氓也见过一些,想必知道嘴越甜心越毒这个道理,所以都是乖乖的。就是403室的欧阳,一开了铐就把我推开,一头闯进了卫生间。过了好半天才随着水箱的轰鸣声回来,嘴和手都是湿的。我瞪着她说:怎么也不说个谢谢?她把双手都伸了过来道:好了,反正尿也撒完了。你不妨再把我铐上。我马上答道:何必这样呢,阿姨?就住在附近,以后常见面。她愣了一下,假笑着说:是呀。谢谢你了。小表弟。妈的,谁是你表弟?你是我的表嫂吗?我一点都不喜欢她。 



有关我自己,还要做些自我介绍。我脸色惨白,个子倒是蛮高的,但软绵绵的没有劲儿。穿什么上衣都显大,穿什么裤子都嫌肥。眼睛像患了甲亢一样凸出,脸上有很多鲜红的小斑点。不知什么地方没长(月真)道,叫人一眼就能看出小来。但你也不要小看我,知道我的人都说:这孙子手特黑。这当然是个比方,实际上我的手一点都不黑而是雪白雪白,四季温凉。看相的说,男生女手,大富大贵,但这一点到现在我还没看出来——我走进401室,对坐在床上的女孩说:阿姨,你转过身去,我给你解绳子。她马上站了起来,转过身去。那双交叉在一起的洁白手臂又呈现在我面前了。 

有件事你可能早就看出来了:现在你很少能看到青年,也很少看到中年人,能见到的中青年里还有不少像我表哥那样是假的。这是因为你看到的人都没有文化,老年人常常错过了受教育的机会,小孩子还没有受教育。而中青年已经受过了教育,后悔也来不及了。所以当眼前这位女孩说“两个小流氓”时,欧阳答道:总比老流氓好吧。——不是流氓的人一定要落到流氓手里,而流氓非老即小,你别无选择了。我拖过一把椅子来,想要解开捆在手臂上的皮条:这不是一根皮条,是一束细皮条,系了很多扣。我一个一个解着,但注意力都在手臂上。在屋顶那盏水银灯照耀下,手臂上反射着暗淡的光。我禁不住在上面吻了一下。她冷冷地问道:怎么回事?我答道:阿姨,我喜欢你。她听了一哆嗦,大概是气的。 

我表哥在房客面前张慌失措,是因为他没有文化,搞不来太复杂的事,所以发慌。我有一些文化,虽然还不够多,但已能壮我的胆子。我一面给401室的女孩解绳扣,一面把脸贴在她手臂上。她的臀位很高,腿很长。裹在粗布底下的臀部也让我神魂颠倒。我还毅然告诉她说:阿姨,你的腰很细,腿也很直。她听了一发抖个不止。等到绳子解开了,她转过身,扬起手来,看样子想要抽我个嘴巴。我坐着不动,决定让她抽一下,但她没有抽下来——大概是想清楚了吧——把手往外一指说:你出去,我要换衣服。我站了起来,把椅子拖开,眼睛直视着她,郑重说道:我爱你,这是真的。然后退出了房间,把门锁上了。 



以上的叙述会给你一个印象,好像我表哥脸皮很薄,我脸皮很厚——起码在两性关系上是这样。实际上远不是这样。公寓装修好之前,我回自己宿舍里去,十次里有九次遇上表哥搂着个女孩坐在我铺位上。如前所述,他的铺位是上铺,如果坐上去,也许整个床都要塌掉,所以我也不好抱怨什么。他们经常把我的床搞得很乱,而我是很讲整洁的。次数多了,表哥也觉得不好意思,就对女孩说:既然碰上了,你和我表弟也玩玩——表哥的厚颜无耻就到了如此程度。那女孩不是“鸡”(打鸡我表哥还舍不得钱哩),把小嘴一噘说:我不。遇上这种场面,我总是不动声色地朝他们走去,说声“对不起”,从床底下掏出几本书来,包在报纸里,拿着走了。出了门还听到女孩说:你表弟怎么这样怪?表哥说:他就这样。看着吧,早晚坏在这上……他说早晚要坏,是指我喜欢读书。在这种情况下,我就拿着书到地下室去读。现在我表哥搬走了,我可以在自己的房间里读书了。 

晚上我可以回自己宿舍去读书。现在有各种各样的书,有纸质的书,这种书可以拿在手里读,听见有人敲门就把它塞到床底下;有光盘书,这种书要用有光驱的PC机来读。我的抽屉里锁了一台笔记本电脑,可以读光盘书。别人看到了,我就说自己在打游戏。还有网络版的书,看那种书要有Net PC。我在地下室里装了一台,谁了看不见,但那地方太冷,太潮,呆不久。相比之下,我还是爱看纸做的书,尤其是小开本的,这种书藏起来方便。书太多了,该不完,而且我读书是要避人的,因为我住在黑铁公寓之外。相比之下,住在公寓里的人就没有这个问题。 

在公寓里,我把大家都放开,退到走廊上。所有的房客都动了起来,收拾自己的东西,把衣物放进床头柜,把几本随身携带的书放在桌面上,打开案头灯调整角度、试试亮度,更有人把桌上的Net PC也打开了,阴暗的公寓里又多了一种monitor的光亮。我在走廊上慢慢走过时,里面的人都警觉地抬起头来,举着手里的书,或者把屁股从椅子上挪开一半指着眼前的键盘问道:可以吗?真实我想耸耸肩膀说:随你们的便。后来又觉得不妥。这些人在公寓里住久了,听到走廊上有人走过就问可以不可以,所以我说:当然可以。她们也就安心去做事。又过了一会儿,整个公寓又恢复了平静,大家都在看书或者看荧屏。多也常做这些事,但没有人看到。自己在看书时,有人在背后看着,这种感觉我没有体验过。说老实话,我有点羡慕。后来我表哥回来了,悄悄地走了进来,站在我身后——此人走路像只猫,很难听到,我是从他身上带的冷气感觉到的。他站着看了好半天,才开口说道:很牛逼,不是吗?这个牛逼我就不知是什么意思,所以也不接茬。过一会他又说:你知道她们干什么呢?我说不知道。他说:她们给我挣钱呢。我表哥就知道钱,但他说得也对。她们在寻求知识,但也在给我表哥挣着钱。这后一点让人想起来不那么太愉快。 

现在我在自己屋里看书,既不必闻我表哥的屁味,也不为他翻身的声音所骚扰,但我还是静不下心来。这间房子里空无一人,没有人从我面前走过,我也不必举起这本书来对他请示道:可以吗?因此这里缺少读书的气氛。 



四 

我住的宿舍离学校的南墙很近,学校的南墙又和我表哥开的公寓很近,有一段南墙是砌锅炉的耐火砖砌的,黄碜碜的,看起来很古怪。墙下有窄窄的一条草坪,出了南墙就能看见,总没人浇水,但草还活着。草坪里种了一丛丛的月季,夏天草坪上满是西瓜皮。草坪前面是马路,过了马路就到了公寓门前。那儿原是个很大的工厂,有很多几层的厂房,有铁道贯穿其中,铁路边上有货栈。总而言之,那地方空房子多得很,以前没发现它有什么用处,现在发现了——我表哥搬来后,又搬来好几家,南墙外面那条马路很快就变成了公寓一条街。这对我有些好处:我是电工,我表哥的房子又是我设计的。有很多人找我做活,下电线,设计房子。这段时间外快挣得很多。 



下雪那天下午,黑铁公寓的管理员在办公室里喝茶,看到401的红灯亮了起来。红灯连闪了两下才熄灭了,这表示住户想要出去散步。此时办公室里只有他一个人。他把脚从桌子上拿下来,穿上大头靴子,套上他的黑皮茄克,从办公室里出去,走到401门前,看到里面的女孩已经准备停当:她把头发束成了马尾辫,脸上化了淡妆,穿着白色的衬衣,黑色的紧身裤,脚上穿着长统皮靴——看来她已经知道外面在下雪。她手里拿了一个白信封。这位管理员是个秃顶的彪形大汉,他从皮带上提起钥匙串,把铁门打开。此时那个女孩把信封塞到他上衣口袋里——信封里是小费。管理员说:用不着这样——然后又改口道:用不着现在给。但是钱已经给了。管理员看了这间房子:这时的每一样家具都是黑色的,黑色的矮床,床上罩着黑色的床罩,黑色的钢管椅子,黑色的终端台上,放着黑色的PC机——机器是关着的。一切都收拾得井井有条,用不着人尽督促、管理之责。正如他平时常说的,401的房客最让人省心。桌面上还有一个黑色的磁杯子,里面盛着冒气的热咖啡。管理员建议道:先把咖啡喝了吧。那个女孩没有回答,只是面露不耐烦之色——这位房客虽让人省心,但是很高傲。于是他走向那张几乎看不见的黑皮沙发,叉开双腿坐了下来。那个女孩走到他面前,站到他两腿之间,然后转过身去,跪在地板上,把双手背到身后。管理员在牙缝里出了一口气,俯下身去,用手按住她的后脑,让她把头低得更低,直至面颊贴到冷冰冰的地板,然后从袖筒里掏出一根麂皮绳索,很熟练地把她的双手反绑在身后。我说的这件事发生在黑铁时代,黑铁时代的人有很多怪癖。这位管理员像一位熟练的理发师在给女顾客洗头,一面缠绕着绳子,一面说:紧了说话啊。但那个女孩没有说话——看来松紧适中。等到捆绑完毕,他把她扶了起来,转过她的身子,左右端详了一番,看到脸上没有沾到土,头发也没有散乱,就从衣架上拿起黑色的斗篷,给她围在身上,系好了带子。随后他又看到墙上还挂有一顶黑色的女帽,就把它拿到手里,想要戴到她的头上。但那女孩摇了摇头,于是他又把帽子挂在墙上,然后打开了铁门,让她走在前面,两个人一起到漫天的大雪里去散步。 


(编者注:窝子录入完成于2004年9月21日。未校对稿。)
