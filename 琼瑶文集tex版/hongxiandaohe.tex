\chapter{红线盗盒}

肃宗时薛嵩在湖南做沅西节度使,加兵部尚书、户部左侍郎、平南大将军衔,是文从一品、武一品的大员。妻常氏,封安国夫人。子薛湃,封龙骑尉。沅西镇领龙陵、凤凰两军,治慈利等七州八县,镇所在凤凰寨,显赫一时。 

有一天早上,薛嵩早起到后院去。此时晨光熹微,池水不兴波,枝头鸟未啼,风不起雾未聚,节度大人在后园,见芭蕉未黄,木瓜未熟,菠萝只长到拳头大小。这一园瓜果都不堪食。节度大人看了,有点嘴酸。正在没奈何时,忽然竹林里刷啦啦响,好似猪崽子抢食一样,钻出一个刺客来,此人浑身涂着黑泥,只露眼白和白牙;全身赤裸,只束条丁字带儿,胸前一条皮带,上挂七八把小平斧,手握一口明晃晃的刀,径奔薛节度而来,意欲行刺。薛节度手无寸铁,无法和刺客理论,只得落荒而逃。那刺客不仅是追,还飞了薛嵩一斧,从额角擦过。薛嵩直奔到檐下,抢一条苦竹枪在手(此物是一条青竹制成,两端削尖,常用来担柴担草,俗称尖担是也),转身要料理这名刺客。那刺客见薛节度有枪在手,就不敢来见高低,转身就跑。薛嵩奋起神威,大吼一声,目眺尽裂,把手中枪掷出去,正中那刺客后心,把他扎了个透心凉。办完了这桩事儿,他觉得脸上麻麻痒痒,好像有蚂蚁在爬,伸手一摸,沾了一手血。原来那一斧子并不是白白从额面擦过去的,它带走了核桃大小一块皮肉。他赶紧跑回屋去。这间屋子可不是什么青堂瓦舍,而是一问摇摇晃晃的竹楼。竹板地板木板墙。房里也没有绸缎的帷幕,光秃秃的到处一览无遗。他叫侍妾红线给他包扎伤口。这位侍妾也非细眉细目粉雕也似的美人——头上梳风头髻,插紫金钗,穿丝纱衣袍,临镜梳妆者。此女披散着一头乌发,在板铺上睡着未起,一看薛嵩像血葫芦一样跑了进来,不惟不大叫一声晕厥过去,反而大叫一声迎将过来。她身上不着一丝,肤色如古铜且发亮,长臂长腿,皮肉紧绷绷,矫捷如猿猱,不折不扣是个小蛮婆。 

如前所述,薛嵩早起所赏之园,以及他府第和侍妾的状况,根本不像大唐一位节度使,倒像本地一位酋长。不过这只是表面现象,事实上他毕竟是天朝大邦的官员,有很高的文明水平。红线为他包扎伤口,被他当胸一掌推出三尺。节度大人说: 

“你真是没道理!我是主,你是奴;我是男,你是女;我是天,你是地;如今我坐在地上,你站着给我裹伤,倒似我给你行礼一般!” 

红线只好跪下给他裹伤,嘴里说,她不过是看他中原人长得好看,就跑来跟了他,谁知他有这么多讲究,又是跪又是拜,花样翻新。闲话少说,裹好伤以后,薛嵩穿上贴衣的细甲,提一条短抢,红线拿上藤牌短刀,到园子里看那个死刺客。红线略一打量,就说: 

“这不是山里人,而是山下湖边的汉人。” 

薛嵩说:“放屁,你看这家伙光着身子抹一身黑泥,不是山里的蛮子是什么?你说他不是山里人,无非是为你的蛮族同胞开脱。”红线说:“他的确不是山里人。首先,他用手斧行刺。山里的部落有善用吹筒的,有善用标枪的,但绝无用飞斧的。第二,他的牙齿洁白,从来没嚼过槟榔。所以他是山下的汉人,往身上抹一身泥巴,混充是蛮人。”薛嵩说:“混账!放屁!岂有此理!”红线只好跪下来说:“奴婢知错了,奴婢罪该万死。”薛嵩对她在教化方面的进步表示满意,就说:“姑念尔是初犯,本老爷免于责罚,快给我上山去把马套下来。”他伸出一只手,把红线拽起来,叫她快点跑。 

等红线把马拉来时,薛嵩已经着装完毕:身上穿二指厚海兽皮镶铁的重铠,头戴一顶熟铜大盔,背插银装锏,腰悬漆裹铁胎大弓和一壶狼牙箭,手提七十斤重的浑铁大枪,骑在枣骝嘶风马上,威风凛凛,仪表堂堂。不过这种武装在此地极不适宜,因为此地山高林密,到处是沟谷池塘,万一马惊了把他甩在塘里,会水也要淹死。依红线的意见,他不如骑一条大牯牛出去,不必穿甲,拿个大藤牌护身;枪锏都不必带,带一把长刀就够用。当然这些话是蛮婆的蠢主意,薛嵩完全听不进,他打马出去,立在当街,喝令他的兵集合——那些兵部躺在各处竹楼檐下的绳床上,嚼槟榔的,看斗鸡的,干什么的都有。薛嵩吆喝一早晨,才点起二百名亲兵。他命令打一通鼓,拉开寨门,就浩浩荡荡出发,刺客的尸首就驮在队尾的牲口上。他要到这九洞十八山的瑶山苗寨问一问,是谁派刺客来刺他。 

薛嵩上山去找酋长们问罪,去时披坚执锐,好不威风,回来时横担在马背上脸色排红,人事不知。他手下的兵轮流扛着那条大抢,也累得气喘吁吁。这倒不是吃了败仗。薛嵩这一条枪虽不及开国名将罗士信、秦叔宝那两条枪有名,可在正德年间,使枪的名家就数着他啦,岂能在这种地方栽跟头?实际上他上山以后并没和人开仗,就从马上栽了下来。回到寨里.红线一看薛嵩的症候,就叫亲兵卸去他的盔甲,把他放在竹床上。此时节度大人胸前胁下,无数鲜红的小颗粒清晰可见。红线叫大兵提来井水,一桶一桶往他身上浇,泼到第七桶,节度大人悠悠醒转。原来山上虽然凉快,可毕竟是六月酷热的大气,穿海兽皮的厚甲不甚相宜。节度大人披甲出门,不单捂了一身痱子,而且中了暑。 

节度大人醒来时,只见自己像刚出世一样精赤条条,面前站满了手下的兵,这可不得了!他这个身体,虽不比皇上的御体,但是身为文武双一品的朝廷大员,起码可以称为贵体,岂能容闲杂人等随便来看?更何况他身上长满了扉子。薛嵩是堂堂的一条好汉,而痱子是小孩子长的东西,所以既然长了痱子就应该善加掩饰,怎么能拿来展览?薛嵩把手下人都轰出去,关起门来要就这个过失对红线实施家法,也就是说,用竹板打她的手心。可是那个小蛮婆发了性子,吼声如雷,说老娘好意救你,倒落下好多不是,这他妈的就叫文明啦!她还把孔圣人、孟圣人,以及大唐朝的列祖列宗一齐拿来咒骂。薛嵩见她不服教化,也只好罢休。他叫她拿饭来吃,今宵早点睡,明天起绝早再上山去找酋长们问罪。 

红线把节度大人的晚膳拿来——诸位,这可不是羊炙鱼脍之类的大唐名菜,盛在细磁盘白玉碗里;而是生胸鱼、牛肉干耙、酸菜臭笋之流,盛在竹筒木碗之中。红线给薛嵩上菜根本谈不上举案齐眉,只是横七竖八端上桌来。这女人好像有点得意忘形,端上菜以后就粗声粗气地说: 

“吃吧!” 

把薛嵩气得要发疯。如果她是薛嵩的正妻,薛嵩就要按七出之条出了她。如果她是长安家里的侍妾,薛嵩就要把她臭揍一顿,卖给人贩子。可是此地是荒山野岭,使不得这一套。他只好忍气吞声地吃饭。吃到一半,他忽然想到这蛮女今天这么趾高气扬,想必做下了什么露脸的事情,不妨问上一问。这一间就问出来,早上薛嵩出去以后,又有两位身上涂黑泥的大爷到家里来找他,被红线使铁叉叉翻,吊在后园的竹林里。薛嵩一听大喜,跑到后园一看,那儿果然吊着两个人。这一下薛嵩连饭也顾不上吃,连忙跑到家里,开箱子取出一品大员的大红袍穿上,戴上乌纱帽,束上碧玉带,一边穿衣一边告诉红线法律方面的事,按大唐的制度,节度使不问刑名,案子应该交地方官审理。不过这个案子是行刺本节度,所以可以接军法审理。说完这些活,他就兴冲冲出门去,叫军政司升帐审那两个刺客。 

这个案子倒不难审。两个刺客一到堂上不等用刑就招了供。薛嵩问明情由,给那两位立下罪名,一是偷越关津,擅入沅西镇地面;二是身怀利器,擅入节度府第,行刺朝廷方面大员,按军法推出辕门斩首。等到把这两人斩了,薛节度回家去,坐在铺上生闷气。再看那红线,在一边又开腿坐着,丢砂包捉羊拐,玩得十分开心,气得他拍席喝道:“小贼婆,高兴什么?” 

红线闻声十分踊跃地奔过来,跪在薛高面前,气壮如牛地吼道:“奴婢知错了!奴婢罪该万死!!” 

薛嵩被她搅得没了脾气,只好把她拉起来说:“得啦,起来说话,我现在倒运得很,遇上一件糟心事,只好和你商量。” 

“启禀家主爷,奴婢罪该万死得很啦,我不知道你说的是哪一出。” 

“还能是哪一出?就是早上那两个刺客的事。” 

“嗅!那两个刺客!你问出来了吧,他们是苗人还是瑶人?” 

一说起那两个刺客的种族,薛嵩脸色有点阴沉。红线说:“是不是又要给你跪下来?”薛嵩说:“这倒不必,那些人果然如你所说,全是汉人,他们是两湖节度使田承嗣帐下的外宅男,奉差来取薛某的首级。”红线说:她十分知罪,首先,她为三阴弱质,头发长见识短;其次,她乃蛮夷之人,不遵王化,因此她这个小奴家就不知什么叫外宅男,以及他们为什么要取薛嵩的首级。薛嵩说,这件事十分荒唐,这位两湖节度使田承嗣,管着洞庭周围数十州县,所治部是鱼米之乡,物产丰饶,不知起了什么痰气,还要来抢薛嵩的地盘儿。田老头自称有哮喘病,热天难过,要薛嵩借一片山给他避暑。怎奈薛嵩名义上领有两军七州八县,实际上能支配的也就是这凤凰寨周围的弹丸之地,没地方可借。田承嗣索地未遂,就坏了良心,派他的外宅男来行刺。所谓外宅男者,二等于儿子是也。像这类的干儿子田老头有三千余人,都是两湖一带的勇士,受日老头豢养,愿为其效死力者。这种坏东西今后还要大批到来,杀不胜杀,防不胜防,真不知该怎么对付。红线说,这都怪节度相公当初没听她的话。要按她的意见,当初建寨时,只消种上一圈儿剑麻或是霸王鞭,此时,早长到密密层层,猪崽子也挤不进,刺客要不是长虫,根本爬不进来。现在立了一圈寨栅,窟窿比墙还大,什么都挡不住。薛嵩说,这种话毫无意思,现在去种剑麻也晚了。红线说,家主老爷自称是文一品,武一品,又是大唐的勋戚,在皇上面前很有面子的。只消写一纸奏章,送到长安去,皇上就会治田承嗣的罪——最低限度也要打几十下手心。薛嵩愁眉苦脸地说,这种事皇上多半是不管。那年头群藩割据,潼关以东朝廷号令不行,想管也管不了。于是红线说,她还有个主意,就是他们上山去投靠他的“爹地”。她的“爹地”是个大酋长,管十几座寨子,住在他那儿,薛嵩的安全一定没问题。薛嵩说,这可不成。他是朝廷命宫,天朝的大员,岂能托庇于蛮酋之下?夫子曰,三军可以夺帅,匹夫不可以夺志。万不可如此行。红线就说,她没有其他的主意了,除非他回长安去。回长安也不坏,她想跟着去见见那个花花世界。不过薛嵩家里还有妻室,又有公公婆婆大姑子小姑子等等,数以百计。现在侍候薛嵩一个老爷,又要跪又要拜,当耍子也还可以,再加上老太爷老太太大奶奶二奶奶等等,那就肯定不好玩。 

听了红线的活,薛嵩长叹一声。他不能回长安去,不过这话不能讲给红线听。她虽是贴身侍妾,但是非我族类,不可以托以腹心。他想,我到湘西,原是图做二军七州八县的节度使,为朝廷建功立业,得一个青史扬名,教后世的人也喝一声彩。好一个薛嵩,不愧是薛仁贵之孙,薛平贵之子!谁知遇上这么一种哭笑不得的局面,眼下又冒出了田承嗣,也来凑这份热闹,真他妈的操蛋得很。然后他想:二军七州八县没弄着,只弄上一个小蛮婆。这娘们不待父母之命媒的之言就跑了来,可算是淫奔不才之流;我和她揽到一块,有损名声。最后他又想:这蛮婆也不坏,头发很黑,眼睛很大,腿很长,身腰很好;天真烂漫,说什么信什么。套一句文来说就是:蛮婆可教也。眼下再不把她好好利用一下,就更亏了,他把这意思一说,红线十分踊跃:“是!领相公钧旨!“就躺下来,既没有罗纳帐,又没有白玉枕。薛大人抱着她就地一滚。这项工作刚开始,只听后门嘎嘎一响,薛嵩撇下红线就去抓枪。可是红线比他还快,顺手抓一方磨石就掷出去,只听“哇”的一声,正打在一个人面门上,那人提一口刀,正从门外抢进来。薛嵩十分恼火:行刺拣这个时候来,真该天诛地灭,千刀万剐。于是他挥起大枪杀出去,一到后院,就有七八个人跳出来和他交手。这帮人手段高强,更兼勇悍绝伦,薛嵩打翻了两个,余者犹猛扑不止。要不是红线舞牌挥刀来助,这场争斗不知会有什么结果。那伙人见薛、红二人勇猛,唿哨一声退去,把伤员都救走,足见训练有素。后面是一片竹林,薛嵩腿上也挂了一点伤,所以他无心去追。回到屋里,红线拾起刺客丢下的刀一看,禁不住惊呼一声: 

“哇!这刀可以剃头嘛?” 

薛嵩一看,认得是巴东的杀牛刀,屠干牛而刃不卷,颇值些钱的。刺客先生用这种刀,大概不是无名之辈,他觉得今晚上事态严重,十之八九要栽。首先,他这凤凰寨里只有几十个人,其余的兵散居于寨外的林里,各拣近溪傍塘之处开一片园子,搭一幢竹楼居住;其次,住在寨圈里这几十个人,也是这么七零八落。原来他的兵也和他一样,都搞上了蛮婆。蛮婆就喜欢这种住法,他们说这样又干净又清静。现在他要集合队伍,最远的兵住在十里之外,这么黑灯瞎火怎么叫得齐?薛嵩正在着急,红线说: 

“启禀老爷,奴婢有个计较。” 

“少胡扯!不是讲礼法的时候!有什么主意快说!” 

“禀老爷,这帮家伙在后园里不走,想必是等他们的伙计来帮忙。我们赶紧爬出去,找个秃山头守住。今晚月亮好,老爷的弓又强,在空旷地方,半里地内准一露头你就把他射死,不强似守在这儿等死。” 

这真是好主意。两人掀开一片地板,红线拿着弓箭,嘴里衔一口短刀。薛嵩拿了弓箭,背了官印,钻下去顺着水沟爬到林子里。这儿黑得出手不辨五指,只听见刺客吹竹哨联络,此起彼落,不知有多少人到来。薛嵩也不顾朝廷大员的体面,跟在红线背后像狗一样爬。爬出寨栅,才站起来跑,又跑了好一阵,才出了林子上了山头。是夜月明如昼,站在山头上看四下的草坡,一览无遗。薛嵩把弓上了弦,摇摇那壶箭,沉甸甸有五六十支,他觉得安全有了保障,长叹一声说: 

“红线,你的主意不坏!这一日大难不死都是你的功劳!” 

正说之间,山下寨子里轰一声火起,烧的正是薛节度的府第,火头蹿起来,高出林梢三丈有余。寨里有人乱敲梆子,高声呐喊,却不见有人去救火,那火光照得四下通红。薛嵩这才发现自己浑身上下不着一丝,尚不及红线在脖子上系一条红领带。薛嵩一看这情景,就撅起嘴皱起眉,大有愁肠千结的意思。红线不识趣,伸手来扳他的肩。 

薛嵩一把把她推开,说:“滚蛋!我烦得要死!” 

“呀!有什么可烦的,奴婢罪该万死,还不成吗?” 

薛嵩说,这回不干她的事,山下一把火,烧去了祖传的甲枪还是小事,还把他的袍服全烧光。他是朝廷的一品大员,总不能披着芭蕉叶去见人。在这种荒僻地方,再置一套袍服谈何容易。不过这种愁可以留着明天发。这两位就在山头上背抵背坐下,各守一方。红线毕竟是个孩子,闹了半夜就困了,直耷拉头,薛嵩用肘捅她一下说: 

“贱婢,这是什么所在,汝尚敢瞌睡乎?我辈的性命只在顷刻!” 

红线大着舌头说:“小贱人困得当不得,你老人家只得担待罢!” 

说完她一头睡倒,再也叫不醒。她一睡着,薛嵩的困劲也上来了,他白天中过暑,又挂了两处彩,只觉得晕晕沉沉,眼皮下坠,于是他把红线摇起来,说: 

“红线,我也很困!你得起来陪我,不然两人一齐睡过去,恐怕就都醒不过来了!” 

红线发着懒说:“启禀大人,奴婢真地困得很啦。你叫我起来干什么?天亮了吗?” 

她坐在那儿两眼发直,说的全是梦话,转眼之间又睡熟了。薛嵩用脚踢了她腰眼一下,这下不仅醒过来,而且火了。 

“混账!我刚睡着!你他娘的又是大人,又是老爷,把便宜都占全,值一会夜就不成吗?老娘又跪你,又拜你,又喊你老爷,又挨你打,连觉也不能睡?我偏要睡!”说完她又睡倒了。 

薛嵩一个人坐在山头上四下眺望,忽然一阵悲从中来,他禁不住长吁短叹,“唉!流年不利,闹得我有家难回!”这股伤。心劲儿上来,禁不住流了几滴英雄泪。红线在睡梦中听见,就爬起来,怯生生来拉薛嵩的手。 

“老爷,你怎么了你?你老人家这个脸子真难看。好啦,奴婢知罪啦,你来动家法罢!” 

薛嵩说:“你回去睡吧。老爷我的精神劲儿上来,守到天明不成问题。”红线说,听见老爷叹气,就像烙铁烙心一样难受,她也睡不着。用文词儿来说,生死有命,富贵在天,叹之何为。薛嵩曰:事关薛氏百年声威,非汝能知者。红线说,但讲何妨。某虽贱品,亦有能解主忧者。这一番对答名垂千古。唐才子袁郊采其事入《甘泽谣》,历代附庸者如过江之鲫,清代才子乐钓赞曰:“田家外宅男,薛家内记室;铁甲三千人,哪敌一青衣。金合书生年,床头子夜失。强邻魂胆消,首领向公乞。功成辞罗绔,夺气殉无匹。洛妃去不远,千古怀烟质!” 

洛妃当是湘妃之误。近蒙薛姓友人赠予秘本《薛氏宗谱》一卷,内载薛姓祖上事机洋,多系前人未记者。余乃本此秘籍成此记事,以正视听。该书年久,纸页尽紫,真唐代手本也!然余妻小胡以其为紫菜,扯碎入汤做馄饨矣。唐代纸墨,啖之亦甘美。闲话少说,单说那晚薛嵩坐在山头上,对红线自述优怀。据《甘泽谣》所载:“嵩乃具告其事,曰:我承祖上遗业,受国家重恩,一旦失其疆土,即数百年勋业尽矣。”语颇简约,且多遗漏,今从薛氏秘本补齐如下: 

红线:照奴婢看,打冤家输到光屁股逃上山,也不是什么太悲惨的事儿。过两天再杀回去就是啦。老爷何必忧虑至此。 

薛嵩:这事和你讲不明白。我要是光棍贫儿,市井无赖出身,混到这步田地,也就算啦。奈何本人是名门之后,搞成眼下这个样子,就叫有辱先人。我的曾祖,也就是你的太上老爷,名讳叫做薛十四,是唐军中一个伙夫,身高不及六尺,驼背鸡胸,手无缚鸡之力,一生碌碌无为。我的祖父,也就是你的太老爷,名讳叫做薛仁贵,自幼从军做伙夫,长成身高六尺,猿臂善射,勇力过人,积军功升至行军总管,封平西侯。我父亲,也就是你的老太爷,名讳叫薛平贵,身长八尺,有力如虎,官拜镇国大将军,因功封平西公。至于我,身高九尺,武力才能又在祖父之上,积祖宗之余荫,你看我该做个什么? 

红线:依奴婢之见,你该做皇上啦。 

薛嵩:咄!蛮婆不知高低!这等无君无父,犯上作乱的语言,岂是说得的呢?好在没人听见,你也不必告罪啦。我一长大成人,就发誓非要建功立业,名盖祖宗不可。可惜遇上开元盛世,歌舞升平。杨贵妃领导长安新潮流,空有一身文才武艺,竟无卖处! 

接下来红线就说,她不知开元盛世是怎么回事。薛嵩解释说,那年头长安城里彩帛缠树,锦花缀枝。满街嗡嗡不绝,市人尽歌:“阳春白雪”。虽小户人家,门前亦陈四时之花草,坊间市井,只闻箜篌琵琶之声。市上男子衣冠贱如粪土,时新妇女服装,并脂粉、奇花、异香之类,贵得要了命,而且抢到打破头。那年头与长安子弟游,说到文章武事,大伙儿都用白眼看你,直把你看成了不懂时髦的书呆子,吃生肉喝生鸡子的野蛮人。非要说歌舞弦管,饮酒狎妓之类的勾当,才有人理你。那年头妇女气焰万丈,尤其是漂亮的,夏日穿着超薄超透的衣服招摇过市,那是杨贵妃跳羽衣霓裳之舞时的制式。或着三点式室内服上街,那是贵妃娘娘发明的。她和安禄山通奸抓破了胸口,弄两块劳什子布遮在胸前,皇帝说美得不得了,也不知道自己当了王八。那年头儿杨贵妃就是一切。谁不知杨家一门一贵妃二公主三郡主三夫人?杨国忠做相国,领四十使:你就是要当个县尉也要走杨府的门子啦!弄不来这一套的,纵使文如李太白,武如郭子仪,也只好到饭馆去端盘子。贵妃娘娘的肉体美,是天下少女的楷模。她胸围臀围极大而腰围极细,这种纺锤式的体型就是惟一的美人模式。薛嵩的妹妹眉眼很好看,全家都把希望寄托在她身上。督着她束紧了腰猛练负重深蹲和仰卧推举,结果练出一个贵妃综合症来,柬着腰看,人还可以;等到把紧身衣一解,胸上的肉往下坠,臀上的肉往上涌,顿时不似纺锤,倒似个油锤。如此时局,清高点的人也就叹口气,绝了仕途之念。奈何薛嵩非要衣紫带玉不可。妹妹没指望,他就亲自出马:从李龟年习吹笛,随张野狐习弹筝,拜谢阿蛮为师习舞,拜王大娘为师习走绳。剃须描眉,节食束腰。三年之后诸般艺成,薛嵩变为一个身长九尺,面如美玉,弱不禁风,一步三摇之美丈夫,合乎魏国夫人(杨贵妃三妹,唐高宗之姨)面首的条件,乃投身虢门。看眼色,食唾余,受尽那臭娘们的窝囊气。那娘们还有点虐待狂哩,看薛嵩为其倒马桶,洗内裤,稍不如意便大肆鞭挞。总之,在虢府三年,过的都是非人生活。好容易讨得她欢心,要在圣上面前为他提一句啦,又出了安史之乱,杨氏一族灰飞烟灭。天下刀兵汹汹,世风为之一变。薛嵩又去投军,身经百战,屡建奇勋,在阵前斩将夺旗。按功劳该封七个公八个侯。奈何三司老记着他给虢国夫人当面首的事,说他“虢国男妾,杨门遗丑,有勇无品,不堪重任”,到郭子仪收复两都,天下已定,他才混到龙武军副使,三流的品级,四流的职事。此时宦官专权,世风又为之一变。公公们就认得孔方兄、阿堵物,也就是钱啦。薛嵩一看勤劳工事,克尽职守没出路,就弃官不做。变卖家中田产力资本,往来于江淮之间,操陶朱之业,省吃俭用。积十年,得钱亿万。回京一看,朝廷新主,沅西镇节度使一职有缺。薛嵩乃孤注一掷,把毕生积蓄都拿出来,买得此职。总算做了二军七州八县的节度使啦,到此一看,操他娘,是这么一种地面! 

红线说,故事讲到这一节,她就有点儿知道了。五年前一队唐军到山前下寨,她那时还是个毛丫头哩,领一帮孩子去看热闹。彼时朝霞初现,万籁无声。她们躲在树林里,看见老爷独自在溪中洗浴。在苗山从没见过老爷这么美的男人:身长九尺,长发美髯,肩阔腰细,目似朗星。胸前一溜金色的软毛直生到脐窝,再往下奴婢不敢说,怕老爷说奴是淫奔不才之流,老爷那两条腿,哇!又长又直。奴婢当时想,谁长这么两条腿,穿裤子就是造孽!当时奴婢就对那帮丫头说:我现在还小,再过几年,要不把这鸟汉子勾到手,我就不是人!当然,奴婢这么说,是罪该万死的啦! 

红线讲到这里,天已经亮了。太阳虽未出山,但东边天上一抹玫瑰色。那天正是万里无云的天气,半边天都做蓝白色。早上有点儿冷,她朝薛嵩身上偎过来。薛嵩却想:我虽落难,到底还是朝廷的一品大员,山顶上亮,可别叫别人看见。他就伸出一个指头把红线推开。 

那天早上从将破晓到日头出来,薛嵩都在教训红线。说的是他一生的教训,全是金玉良言,皆切中时弊,本当照录,叫那些在小胡同里楼搂抱抱的青年引以为戒。奈何事干薛氏著作之权,未敢全盘照抄,只能简单说个大概。薛嵩说,男欢女爱,原本人之大欲,绝然无伤,但是一不可过,二不可乱。过则为淫,乱则成奸。淫近败,奸近杀,此乃千古不易之理。君淫则倾国,如玄宗迷恋杨贵妃,把这锦绣山河败得一一塌糊涂;臣淫则败家,如薛嵩倒霉,完全是因为他给虢国夫人洗内裤。所以人办这男女之事,必须要心存警惕,如履薄冰,如临深渊,一失足则成千古恨。先贤曰一日三省吾身,要到这种事儿,三省都不为过。比方说现在,你往我身上凑,我就要自省:一、尔乃何人?余与尔押,名分得无过乎?当然你是我的妾,名分上是没问题啦。二、此乃何时?所行何事?古人云,暮前晓后,夫妇不同床。当然,你也不是要干那种事,不过是身上冷,要我搂着你。第三条最难,要顾及人言可畏。如今天已经大亮,我在山头上搂着你,别人看了,岂有不说闲话的?这比张敞画眉性质要严重多了!我是在男女关系上犯过错误的人,所以要特别警惕。 

红线说:禀老爷,奴婢知过了。又说:每回老爷为这种事教训版婢,奴婢心里就怒得很,真恨不得一刀把老爷杀了扔到山沟里去。所以下回老爷再遇到这种事儿,还是免开尊口,径直来动家法吧,打多少都没关系。别像个没牙老婆子啰嗦起来就没完。红线说到此处,眉毛扬起来,鼻孔鼓得溜圆,咬牙切齿,怒目圆睁。薛嵩想:这小蛮婆说得出做得出,还是别招惹她。另一方面,圣人曰:水至清则无鱼,人至察则无徒。如今我身边只剩一个蛮婆,还是要善加笼络。正好此时大雾起来,薛嵩就说,小贱人,现在没人能看见,你过来吧,老爷我暖着你。小子阅《薛氏宗谱》至此,曾掩卷长叹曰:薛嵩真不愧是名门之后,唐之良臣也!且不论其武功心计,单那早上对红线之态度,已见高明。正如武侯词上楹联所说: 

“不审势则宽严皆误,能攻心则反复自消!” 

余效得此法对付余妻小胡,把她治得服服贴贴,发誓说只要王二爷还有一口气,世上的男子她连看都不看一眼。就是高仓健跪在她面前,也只好叫他等到王二死了再来接班。闲话免谈,单说那早上薛嵩把红线搂在怀里。红线感泣曰: 

“老爷,你对我真好。有什么忧心的事儿,都对贱妾讲了吧,天大的事儿,奴给你担起一半。” 

薛嵩说,眼下的事儿连老爷都没主意,你能有什么办法?红线说,老爷休得小看了奴婢!这二年给老爷当侍妾,我老实多啦。前几年贱妾还是这一方苗山瑶寨的孩子王哩。登高凫水,无一不会。弯箭吹简,无一不精,刀枪剑戟都是小菜。就连下毒放蛊,祈鬼魔神那些深山里生番的诸般促狭法门,也要得比巫师神汉一点不差。当然啦,奴婢的本领没法儿和老爷比,老爷是人中之龙,名门之后,大唐之良将,还给虢国夫人当过面首的;不过小本领有时能派大用场。老爷读经史,岂不闻曹沫要离之事乎? 

薛蒿听了这种话,也不敢大当真。他接着讲他的倒霉事。这就要从沅西节度使这个名目说起。正德初年,有几个苗人到长安去,自称湘西大苗国的使臣,又说是大苗国领二军七州八县,户口三十万,丁口百万余。国王自愧德薄,情愿把这一方土地让与大唐皇帝治理,自己得为天朝之民,沾教化之恩足矣。当时朝廷中有些议论,说这大苗国不见经传,这几个苗使又鬼头蛤模眼。所贡之方物,多属不值一文。所以这八成是个骗局,是一帮青皮土棍榨取天朝回赐之物。要按这些大臣的意见,就要把这几名使臣下到刑部大牢里。可是当时是宦官专权,公公们要这大苗国。所以持此议的大臣们倒先进了刑部大牢啦,宦官们把持着皇上,开了御库,回赐苗使黄金千两,金银牌各千面,丝帛之类,难以尽述。这些东西,苗使带回去多少是很难说的。这种事儿总要给公公们上上供。然后就有沅西一镇,节度使一职索价干万缗,可以说便宜无比。不过别人都知道底细,谁也不来上这个当。偏巧薛嵩当时在江南经商,回京一看,居然有节度使出卖,只要这么点儿钱,就买了下来。办好手续,领到关防印信,拿到沅西镇版图,又花了比买官多十倍的钱。薛家的老少从原来的大宅子搬到一个小院里。薛嵩把部曲家丁改编成沅西镇标营。按图索膜到湘西一看——不必说了,什么都不必说了。慢说是二军七州八县,连一片下寨的地方都没有。这山苗洞瑶勇悍得很,你占一寸地他都要和你玩命。好不容易寻到凤凰寨这片无主之地,才有了落脚的地方。 

红线说,好教老爷得知,这凤凰寨也是有主的地方,归我爹爹管理。当年老爷在此下寨,爹爹要集合三十七寨上万名苗丁下山来打老爷。小贱人在爹爹面前打滚撒娇,说爹爹把老爷撵去,奴就要吞钉子。爹爹说,你既如此,就把这片地给你。将来我死后,三十七寨你都无份。后来下山来跟老爷,每回挨了家法,心里都有些罪该万死的气话。老爷不赦罪,奴一辈子也不敢说。薛嵩说,赦尔无罪,你且说来。红线说,奴婢想:小王八羔子占了老娘这么多便宜,还敢打老娘,而且打得这么痛!现在不理你,等半夜我把你切成八大块扔猪圈里去。等老爷睡了,奴又下不得手。薛嵩一听,吓出一头冷汗,连忙说:老爷打你都是一时气恼,你不要记恨。再往下有些话迹近狠亵,小子未敢尽录。总之是关于家法的事,红线表示想开了也没什么不可接受的,薛嵩对她的教化程度表示嘉许。然后又提到原来的话题上去,红线问薛嵩,既然知道沅西镇是个骗局,何不回京去,向中宫们索回买官之价。薛嵩说,买官之价既付出,已不能全部索回。老爷我不回长安,又和我平生所好有关。 

薛嵩对红线讲他平生所好时,正如那李后主词云:红日已高三丈透。彼时雾气散尽,绿草地青翠可爱,草上露珠融融欲滴。薛嵩的心情,却如陆游所发的牢骚:错、错、错!他觉得这一辈子都不对头,细究起来,他这人只有一个毛病:好名。其余酒色财气,有也可无也可,他不大在乎。再看他一生所遇,全是倒着来,什么都弄着过,就是没有好名声。开元时他年方弱冠,与一帮长安子弟在酒楼上畅饮,酒酣耳热之时,吟成一长短句。寄托着他今生抱负,调寄:嘣嘣嚓嚓(此乃唐代词牌,正如广陵散,已成千古绝响),词曰: 

乘白马,持银戟,啸西风!丈夫不惧阮囊羞,只恐功不成。祖辈功名粪土矣。还看今生。秩千石何足道,当取万户封! 

当时薛嵩乘酒高歌此曲,博得满堂倒彩。有人学驴叫,说薛嵩把D调唱成了E调,真叫难听。像这种歌喉,就该戴上嚼口。还有人说,薛嵩真会吹牛皮。他还要当万户侯哩,也不看看啥年月!舞刀弄棍吃不开啦!这可不比太宗时,凭你祖父一个伙头军,也能混上平西侯。又有人说令祖一顿要吃两条牛腿,而且瞎字不识。这等粗鄙之徒,令祖母不知怎么忍受的,薛嵩闻言大怒,说:你们睁开眼睛等着看吧,不出十年薛某人混不出个模样,当输东道。一晃十年,那帮长安旧友找上门来。这个说:薛嵩,你可是抱上虢国夫人的大粗腿啦。万户封在哪里?拿给我看看。那个说:咱们到酒楼上去,听薛嵩讲讲虢国夫人的裤衩是什么样子的。这种话真听不得。薛嵩在酒楼上说,再过十年做不成万户侯,还输东道。又过了十年,在长安市上又碰上旧友。人家这么说:“嗨,薛嵩!怎么着,听说在江南跑单帮哪?”薛嵩头一低,送给他一张银票说:“今秋东道,劳兄主持。寄语诸友,请宽限十年。不获万户封,当割首级!” 

那人说:“得啦老薛,千万别介。大伙都是好朋友,玩笑旧玩笑。你要真赌,我包你死为无头鬼!” 

他妈的,这不是咒人吗?转眼十年之期将至,就这么回乡去,别人的嗤笑难当。薛嵩决意死守在此,除了要逃人耻笑,还有两件事儿可干。第一,凭沅西节度府斗大一颗官印,派军需官到巴东江淮贩运盐铁,与苗人贸易。这么干到年终多少能有些钱物汇到家里去,要不只好喝西北风。第二,他还要等继任官来哩,叫他也尝尝这个上吊找不着绳的滋味。所以他今手下人对外只说沅西镇真个有七州八县。谁知这田承嗣也以为他有七州八县,来借一片山。如今弄得他上无片瓦、下无立锥之地。有家难回,有国难投。兽有林乌有巢,薛嵩竟无安身之处。雷呀,你响吧!电呀,你闪吧!…… 

小子录到此处,觉得这薛嵩秘籍有点儿不伦不类。晴空万里,何来雷电?倒像近代电影中男主人公失恋的俗套。余妻小胡以为此段乃绝妙好辞,千古文章,文盖上影厂,气夺好莱坞。但小子不以为然,遂将此段删去不载。却说日上了三竿,薛嵩看着脚下的凤凰寨,由于衣冠不整,下不去。红线说:“老爷,奴婢又有一个主意。咱们俩从林子里摸回去。你在草丛里躲着,我去找你的副将,借他的衣甲,就说昨晚家中失火,你老人家去得急啦,失了袍服,然后咱们扯块白布赶制袍服,拿红豆染染,也能穿。至于那外宅男,我来给你对付。小贱人在家里还是大小姐啦,上山去借百把苗丁总借得来。那些人在平地打仗不中用,要讲在林子里动手,比那外宅男强了百倍不止。逮着活的都阉了放回去。看他们下回还敢来不?” 

薛嵩一听,觉得这主意还可以,只要外宅男不来行刺,这片地方他还能守得住。他手下拨拉拨拉还有千把人,多数久经沙场。薛嵩本人又有万夫不当之勇。兵法云:山战不在众而在勇。田承嗣若从大路来进攻,薛嵩倒不怕他。于是他解开包印的包袱,把那方黄缎子当遮羞布围在腰间,和红线走草丛里的小路下山去。一直摸到寨中的竹林里,从草丛里探头出去,一个人也看不见,却听见寨前空场上人声鼎沸,有个驴叫天的嗓门儿在念文书: 

“领户部尚书、上柱国、镇国大将军衔,两湖节度使田,准沅源县文字:‘查沅西节度使薛嵩,家宅不慎,灯火有失,酿成火灾,一门良贱,葬身火窟,夫地方不可一日无主,薛镇所遗凤凰镇,及二军六州八县地面,仰请田镇暂为管辖,以待朝廷命令。正德十年,六月二十五日,沅源县令余。’诸位,这下面有田节度使的大印和沅源县印,你们都看明白啦。小的们,把它贴起来!还有一通文书。 

“沪部尚书上柱国镇国大将军,两湖节度使田,谕沅西镇军民人等文事:‘倾悉沅西节度使薛使相嵩,家宅不幸,火灾丧生,不胜悲悼之至。薛使君是咱老田的亲家啦。英年早丧,国家失去一位良将,地方上失去一位青天父母官,薛家嫂子中年丧夫,我田某焉得不伤心?日某当至凤凰寨抚慰军民,车骑在途。薛氏部属,愿去者给资遣散,愿留者帐下为军。滋事者立地格杀。切切此谕!’” 

此文书念毕,场上好一阵鸦雀无声。薛嵩只觉得当头一棒,手脚冰凉。他可没想到田承嗣的手脚有这么快,昨晚上派人行刺,今早上就派人到寨来接收人马。忽然会场上有人大喊一声: 

“弟兄们!咱们老爷死得不明白!多半是田承嗣捣的鬼呀!” 

一人呼百人应,会场上乱成一咽。红线连忙用手肘拱薛嵩: 

“老爷,咱们俩杀出去吧。场上都是你的人,咱们先把田家这几个小崽于摆平了再说!” 

谁知薛嵩长叹一声,面如灰土:“噫!余今赤身裸体,汝又不着一丝,乳阴毕露。纵事胜,亦将遗为千秋话柄。夫子云:土虽死而缨不绝,况不着一丝乎?不如走休。” 

这会场上那驴嗓子在吼:“诸位,想明白了啊!管他明白不明白,薛嵩是死了,是明白事儿的赶紧回家去,我们田大人来了有赏。不怕死的就留在这儿起哄!” 

于是场上的人声渐息。红线急得用双手来推薛嵩,叫道:“老爷你他妈的怎么了,再不动手下人就要散光了!” 

薛嵩回过头来,这张脸红线都不认识了。简言之,是张死人的脸。他呻吟着说话,其声甚惨:“此乃天亡我薛氏,非田氏之能也。余不合力虢国之男妾,遂遭此报!夫天生德于予,田承嗣奈我何?而天不降德于予,也不怪姓田的骑在我头上屙 届扈。红线,自古以来,就没人当过我这样的节度使,也没听说过哪个节度使曾叫人撵得光屁股跑。这种事非偶然也,都是我不守士德的报应,现在我觉得四肢无力,心中甚乱,想来命不长矣。你搀我一把,咱们走吧。” 

红线把薛嵩架到林里,扶他坐下。她叉着腰在薛嵩面前一站,气势汹汹,再没一点恭敬的样子,说出的话也都可圈可点:“老爷,我不喜欢你了!你怎么这么个窝囊的样子?老娘跟你,图的是你是条汉子!谁知你像条死蛇,软不出溜。我跟你干什么?” 

薛嵩呻吟一声说:“事非汝能知者,红线,笔墨侍候!老爷要写遗书。” 

“呸!别做梦啦。上哪儿找笔墨?” 

薛嵩一听,哇地一声吐出一口血来,他想起三国时的袁公路来,当年关东二十七路诸侯讨董或袁家兄弟为盟主,那时中兴得很。曾几何时,袁公路兵败如山倒,逃到破庙里,管手下要一碗蜜水喝。手下说:只有血水,哪有蜜水?袁公路听了呕血而死,为后世所耻笑。如今他临终,索笔墨不可得,和袁公路差不多了。红线见他可怜,就扯一片芭蕉叶,削个竹签来说:“行啦,您别急,在这上面写吧。” 

薛嵩要写遗书,怎奈手抖握不住竹签,只得把这蕉叶竹签都递给红线。然后又说:“红线你还是跪下来。不是我要拿架子,而是这种时候一定要郑重。” 

红线撅着小嘴下了跪,心里想:狗娘养的,反正就跪最后一回。她现在对薛嵩是一肚子气。那种不遵王化的人,也不懂什么夫妻情分。一觉得薛嵩可恶,就巴不得他早死。薛嵩先时—句:“红线,后园里埋的金银,你要多少?” 

“我要它没用处,随你怎么分派吧。” 

“好。我死以后,劳你把这封书信和那些金子送往长安东三坊薛宅。交薛湃收。这信这么写——说与湃儿知道:汝父流年不利,丧命荒郊,今将毕生所贮,及先祖所传之弓,付汝收持。汝母面前可以说知。汝少年有为,勿以父为念,努力上进,好自为之。又:持书之蛮女,乃父之侍妾红线。临终之时,多蒙彼服侍,吾死后,彼愿再醮,愿守节,悉从波便。汝终生当以母事之,不得有违,切切。父字,正德十年六月二十五日。” 

红线写完了见薛嵩画押,气得要发疯,心说我还年轻漂亮得很哩,你叫一个二十多岁的大老爷们儿管我叫娘,这不是要害死我?可是薛嵩又要她再写一封信,全文如下: 

“李二瓜并长安诸友钧鉴:仆薛嵩流年不利丧在荒郊,十年之约,死不敢忘。今将首级交余妾红线持去,你们好好照顾她吧。我这一辈子,全是被你们这批乌鸦咒坏了!今后梦中见无头之鬼,那就是我来问候诸位。红线是我的大令,对我很好;她到长安,吃喝玩乐,多烦各位招待。她要金子,你们不得给银子,要星星,你们不得给月亮。要有一桩不应,薛大爷的脾气你们是知道的,各位家里不免要闹宅,友薛嵩百拜无首,年月日。” 

然后他说:“红线,我知道你这个人不遵王化,无男女之礼法。尔见老爷英雄就走了来,却不意要守很多规矩,这在我们天朝女子,原是天经地义;对蛮婆来说,可是难为你啦。老爷平生受人滴水之恩,必当报以涌泉,岂有辜负你这蛮婆的道理。现下有个主意在此:我死之后,你把我的头切下来,身子就埋了吧。这颗头,你按腊猪头做法,先腌后熏。制好了拿到长安去,先给我的狐朋狗友看这封信。等念到一半,你啪地一声把我的头摔出来——有皮无毛,呲牙咧嘴,在案上一滚,吓他们个半死。这帮家伙都是迷信的。见了这种景象,日后难免见神见鬼。一者我报过他们平生相讥之仇,二者你管他们要什么,自无不应者。他们又有钱又有势,你不是要去长安看看花花世界吗?有那帮孙子做护花使者、送钱大爷,包你玩得痛快。” 

说完这些话,薛嵩从壶里抽出一支箭,双手持立,照心窝里就捅。小子阅至此处,不禁掩卷长叹日:薛嵩割首酬蛮婆,真英雄好汉也!大丈夫来去分明,相随之恩,虽死不忘,相诮之恨,虽死必报。就如吴起抱尸,死有余智。小子赞叹已毕,开卷再览——糟了,薛嵩没有死!千古佳话,登时吹灯拔蜡。原来是红线见薛嵩如此气概,就有点舍不得。薛嵩一箭桶下去,她却扑上去握着箭头往下扳,只听“啪”地一声箭杆折为两段。不仅大煞风景,而且可惜了一支好箭。薛嵩就叫“小贱人,你又来做什么!” 

红线说:“禀老爷,奴婢见老爷吩咐后事,英雄侠气,不减当年,对奴家又是非常之好。小贱人不禁喜欢得紧啦,不想让老爷死。您老人家不就是丢了寨子,活不下去了吗?这件事包在奴身上。不出旬日,我给你夺回来。” 

薛嵩说:“呸!吹什么牛皮,这一阵只听寨中人喊马嘶,田承嗣率千军万马已然进寨。我的部属,非降即丧。山川之险已去,身边羽翼已失。只剩你我主仆二人,还都光着身子。拿什么去夺回寨子?就算你上山求动了你爹爹,田承嗣的人马甚多,他也撵不走他。” 

红线说:“大人久经沙场,听见人马进寨就知道田承嗣来了,这大概不会有错。田老头不来还不好办,既来了,明天就要他把寨子交还,不然让他烂成一摊水。俗话说,强龙不压地头蛇,小奴家正是这一方的地头蛇!”说完,她请薛嵩稍安勿躁,自己就钻草棵走了。 

薛嵩在林子里等着,不到顿饭时,就有几名苗女瑶童到来,奉上酒饭。斩草为窝,编竹为墙,一会儿搭起个绳床叫薛嵩安歇。然后半桩小子、黄毛丫头陆陆续续到这片林子来,有携刀带杖的,有舞蛇弄蝎的。将近黄昏,这种人物到了有二三百之多。薛嵩想:要凭这种队伍去收复凤凰寨,还是门都没有。不过要是去捣乱破坏,倒是够人喝一壶。原来这帮孩子携来的蛇蝎,均系骇人听闻者。什么五步蛇、眼镜蛇、青竹标、过树榕,尚属平常。又有金头蜈蚣、火尾蝎子、斗大的蟾蛛等等,及苗人下蛊诸般毒虫。要是把这些东西都扔到凤凰寨里,那儿马上就成了爬虫馆。天刚半黑,只听顽童百口相传曰:“大家姐来!”薛嵩张目一视,真红线也!那一身装束,《甘泽谣》载之分明,想系诸君耳熟能详者:梳乌蛮舍,攒金风钦;衣紫绣短袍,系青丝轻履;胸前佩龙文匕首,额上书太乙神名,脖子上围一条金鳞大蟒蛇,气派非常。满山童子皆拜日:见过阿姐。红线又指嵩云:此乃姐夫。童子又拜日:见过姐夫。红线乃除蟒堆置嵩身云:给我拿着点儿。那东西在薛嵩身上蠕蠕爬动,朝他脸上吐信子。它要是个母的,还可以说是在表示好感;要是公的,多半就是尝尝味道,准备吞了。不消说薛嵩吓得要死。红线登高发令。指派各重各处做乱去了。然后对薛嵩说:“田承嗣处,非我亲自去不可。”于是把那条大蟒抓过来挂树上,要薛嵩写了一封致田承嗣的短简,拿着就走啦。 

这故事的余下部分,薛氏秘籍所载与《甘泽谣》没啥不同,都是说红线夜入辕门虎帐,从田承嗣枕下偷出一个金盒来,里面盛着田的生辰八字。还把他剥得精光,把衣服都拿走。惟一不同之处就是,薛本说,红线盗盒时见田承嗣在梦中犹呼热,心中有所不忍,在他胸前扔了几条眼镜蛇给他抱着取凉。是夜三更,田军忽然炸了营,都说见到猛蛇恶蝎,并有十余人中毒死亡。田承嗣从梦中惊醒,只见七八条眼镜蛇在胸口筑了窝,几乎吓断了气。等到把蛇撵走,又发现枕下失了金盒,被上有薛嵩的书信,当时还以为见了鬼哩。第二天早上薛嵩派人把金盒送回,田承嗣这才大惊大怒,以为薛嵩有什么驱蛇驭鬼的邪法,连忙夹屁而逃。不单不要薛嵩的寨子,还把山边的地盘割了若干县送给薛家。《甘泽谣》所载“明日遣使赠帛三万尺,名马二百匹,他物称是,以献于嵩”,漏了最重要的东西。薛氏秘籍上写的是:赠帛三万尺,名马二百匹,并割湖西郡县,以献于嵩。”又《甘泽谣》载红线盗盒时“拔其簪铒,脱其儒裳”,把田承嗣剥成了猪猡。为什么这么干却无解释,好像红线是个好贪小便宜的。要按薛本就好解释:她老公在山上光着屁股哩,田承嗣是一品大员,薛嵩也是一品大员,所以田的衣服薛可以穿。及至薛嵩平安度过危机,红线辞去;《甘泽谣》所载的理由均属迷信,完全不可信。薛本所载则详实可信。原来薛嵩得了山下的郡县,要下山去做有模有样的节度使,忽得长安书信,其妻安国夫人常氏已去世。薛嵩与其妻感情不好,所以也不大伤心。当时就要册封红线为正妻。红线踌躇三日,最后对薛嵩这么说: 

“老爷,你真是一条好汉,奴婢也确实爱你。不过当你太太的事,我想来想去,还是算了吧。下了山,我也算朝廷命妇啦,要是不遵妇道呢,别人要说闲话,我对不住你。要是克守妇道,好!三绺梳头两截穿衣,关在家里不准出来。这都不要紧,谁让我爱老爷呢?还得裹小脚!好好一双脚,捆得像猪蹄子,这我实在受不了!如今这事,只好这么计较:你到山下去做老爷,我在山上称老娘,这凤凰寨原本是我的,还归我管。我也学你的天朝礼仪,养一帮奴才,叫他们跪拜我。拗了我的意思,也如老爷对我似的,动动家法。总之,不负老爷平生教化之功。老爷还是我的大爷,要是想我了呢,就上山来看我。总之,拜拜了您哪。” 这番话是在半山上说的,说完红线就泣别薛嵩上山去了。薛氏秘籍中薛嵩红线事到此终。
