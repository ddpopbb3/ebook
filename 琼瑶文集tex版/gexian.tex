\chapter{歌仙}

  有一个地方,那里的天总是蓝澄澄,和暖的太阳总是在上面微笑着看着下面。 
 
 有一条江,江水永远是那么蓝,那么清澄,透明得好像清晨的空气。江岸的山就像路边的挺拔的白杨树,不高,但是秀丽,上面没有高大的森林,但永远是郁郁葱葱;山并不是绵延一串,而是一座座、独立的、陡峭的,立在那里,用幽暗的阴影俯视着江水,好像是和这条江结下了不解之缘的亲密伴侣。 
 
 你若是有幸坐在江边的沙滩上,你就会看见:江水怎样从陡峭的石峰后面涌出来,浩浩荡荡地朝你奔过来。你会看见,远处的山峰怎样在波浪上向你微笑。它的微笑在水面留下了很多黑白交映的笑纹。你会看见,不知名的白鸟在山后阴凉的江面上,静静地翱翔,美妙的倒影在江上掠过,让你羡慕不止,后悔没有生而为一只这样的白鸟。你在江边上静静地坐久了,习惯了江水拍击的沙沙声,你又会听见,山水之间,听得见隐隐的歌声:如丝如缕、若有若无、奇妙异常的歌声。这不像人的歌喉发出的,也听不出歌词,但好像是有歌词,又好像是有人唱。这个好地方的名字和这地方一样的美妙:阳朔。这条江的名字也和这条江一样可爱:漓江。 
 
 人们说,这地方有过一位歌声极为美妙的人。从她之后,江面上就永远留下了隐约可闻的歌声。可是关于这位歌仙的事迹,就只留下了和这歌声一样靠不住的传说。我知道,这全是扯淡。因为它们全是一些皆大欢喜的胡说。一切喜欢都不可能长久,只有不堪回首的记忆,才被人屡屡提起,难于忘怀。如果说,这歌声在江上久久不去,那么它一定因为含有莫大的辛酸。我知道,这位歌仙的一切事迹,孩子们,为了你们,我一切都知道。 
 
 人们说,这位歌仙叫刘三姐,我对这一点没有什么不同意见。大概五百年前,她就住在阳朔白沙镇东头的小土楼里。那时的白沙镇和现在没什么大两样:满镇的垂柳在街道到处洒下绿荫。刘三姐十八岁之后,远近的人们才开始知道她,那么我们的故事就从她十八岁谈起。 
 
 我们的刘三姐长得可怕万分,远远看去,她的身形粗笨得像个乌龟立了起来,等你一走近,就发现她的脸皮黑里透紫,眼角朝下搭拉着,露着血红的结膜。脸很圆,头很大,脸皮打着皱,像个干了一半的大西瓜。嘴很大,嘴唇很厚。最后,我就是铁石心肠,也不忍在这一副肖像上再添上这么一笔:不过添不添也无所谓了,她的额头正中,因为溃烂凹下去一大块,大小和形状都像一只立着的眼睛。尽管三姐爱干净,一天要用冷开水洗上十来次,那里总是有残留的黄脓。 
 
 刘三姐容貌就是专门这么可怕,但是心地又是特别善良,乐于助人,慷慨,温存,而且勤劳。镇上无论哪个青年穿着脏衣服,破鞋子,她看见都要难受:为什么人们这么褴褛呢!她会把衣服要来给你洗好、补好的。不然她就不是刘三姐了。她总是忙忙碌碌,心情爽朗,无论谁有求于她,总是尽力为之。一点不小心眼,给人家办事从来没忘记过。她也愿意把饭让给饿肚子的人吃:如果有人肯吃她的饭的话;不过没有一个要饭的接过她的饭,原因不必再说。 
 
 刘三姐有一个优美的歌喉,又响亮又圆润。她最爱唱给她弟弟听,哪怕一天唱一万遍也很高兴。她弟弟是个漂亮的小伙子,小的时候那么依恋她。刘三姐以弟弟为自豪,简直愿意为他死一万次(如果可能的话),不过她弟弟刘老四渐渐地长大了,越来越发现刘三姐像鬼怪一样丑陋。居然有一天发生了这样的事情,吃饭的时候,刘三姐照例把盘子里的几块腊肉夹到刘老四的碗里,而刘老四像发现几只癞蛤蟆蹲在碗里一样,皱着眉头,敏捷、快速地夹起来掷回三姐碗里。三姐儿眼里含着泪水把饭吃下去,跑到江边坐了半天。 
 
 她们家还有刘大姐、刘二姐、刘老头、刘老婆几名成员。大姐二姐也是属于丑陋一类的女人,不过不像三姐那么恶心。大姐二姐好像因为长得比三姐强些吧,总是装神弄鬼地做些小动作,好像三姐是一条蛇一样。刘老头刘老婆昏聩得要命,哪里知道儿女们搞什么鬼。 
 
 过了不久,刘三姐发现大姐二姐比往日勤快多了,每顿饭后总是抢着洗碗。当时刘三姐并没有怀疑到那方面去。又过了不久,她又发现,她们刷碗时总把她的碗拣出来等她自己刷,并且顿顿饭都让她用那个碗。刘三姐暗暗落泪,但也无可奈何。后来,从大姐开始,都不大和她说话了,和她说话时也半闭着眼睛,捂着鼻子。二姐和刘老四也慢慢这样做了。再后来,刘家的儿女们和三姐一起呆在家里的时间越来越少了。不是三姐回家他们躲出去,就是三姐在家不回来。 
 
 夏天到了,天气天天热起来。年轻人们晚上在家的时候越来越少了。附近的山上,越来越多地响起了歌声。终于到了那一天,传说中牛郎织女要在天上相会的日子;那天下午,地里一个未婚的年轻人都没有了,只剩下了老人和小孩,而年轻人都在家里睡大觉。 
 
 到傍晚时分,大群青年男女们站在村西头,眼巴巴地看见太阳下山,渐渐地沉入山后了。等到最后一小块光辉夺目的发光体也在天际消失,他们就发出一声狂喜的欢呼,然后四散回家吃饭。 
 
 刘老头家里,四个儿女都在狼吞虎咽地把米饭吞下去。不等到屋里完全暗下去,他们就一齐把碗扔下,出了大门。刘老头把大门当一声关死,落了闸,和老太婆一起回屋睡了。 
 
 刘三姐出门就和姐姐弟弟分开了,她沿着大路出村,这时天已经完全黑了。等到她摸着黑沿着一条熟悉的小道朝山上爬时,暗蓝色天空上已经布满了群星,密密麻麻的好像比平时多了五六倍。就在头顶上,一条浩浩的白气,正蜿蜒地朝远方流去。刘三姐爬上山顶,看看四周,几个高大的黑影,好像是神话里的独眼巨人。可是无需害怕,那不过是些山而已。这里的山晚上都是这个样子。 
 
 你也许要问,镇上的男女晚上到野外来干什么呢?原来照例有这么个风俗,每年的七月七的晚上,青年男女们都到野外来对歌。其实是为了谈恋爱,并不是对缪司女神的盛大祭祀。 
 
 好了,刘三姐在山顶上,稍稍平一平胸中的喘息,侧耳一听,远处到处响起了歌声。难道这里就没有人吗?不对。对面山上明明有两个男人在说话。刘三姐吸了一口气,准备唱了。可是唱不出来。四下里太静了,风儿吹得树叶沙沙响,小河里水声好像有人在趟河似的。真见鬼,好像到处都有人!弄得人心烦意乱,不知准备唱给谁听的。 
 
 刘三姐又吸了一口气,甚至闭上了眼睛。猛然她的歌冲出了喉咙;那么响,好像五脏六腑都在唱,连刘三姐自己都吓了一跳。 
 
 刘三姐唱毕一曲,听一听四周,鸦雀无声。怎么了?对面山上没有人吗?还说自己唱得太糟? 
 
 过了一会,对面山上飞起一个歌声:好一个热情奔放的男高音。不过,尽管歌儿听起来很美,歌词可是很伧俗,大意无非是:对面山上的姑娘,我看不到你的容貌,想来一定很好看,因为你的歌儿唱得太好了。 
 
 刘三姐脸红了,原来她参加这种活动还是第一次。但是四外黑古隆冬,很是能帮助撕破脸皮。她马上又回了一首,大意是我很高兴你的称赞,但是当不起你那些颂词。如果你愿意,我可以和你交个朋友。 
 
 对面静了一会,忽然唱起了求婚之歌:“七七之夕上山游,无意之间遇良友。小弟家里虽然穷,三十亩地一头牛。三间瓦房门南开,门前江水迎客来。屋后有座大青山,不缺米来不缺柴。对面大姐你是谁,请你报个姓名来。” 
 
 刘三姐心里怦怦直跳。她听着对面热情奔放的歌声,心里早已倾慕上了。她生来就不愿意挑挑拣拣,无论吃饭、穿衣,还是眼前这件事情。于是马上作歌答之曰:“我是白沙刘三姐……”才唱了一句,就被对面一声鬼叫打断了:“哎呀,我的妈也!饶命吧!” 
 
 这一夜,刘三姐再没有找到对歌的人,开了一夜独唱音乐会。 
 
 天亮之后,刘三姐回家吃早饭,看见大姐二姐在饭桌上那副得意洋洋的样子,心里更觉得酸楚无比。 
 
 从此之后,刘三姐越来越觉得在家里呆着没意思,终于搬到镇东面一个没人家的土楼上去了。在那里,她白天在下面种种菜园,天还没黑就关门上楼,绝少见人,心情也宁静了许多。不知不觉额头上数年不愈的脓疮也好了。当然,她决不是陶渊明,所以有时她在楼上看见远处来来往往的行人,心里还说免不了愁闷一番。她喜欢和人们往来,甚至可以说她喜欢每一个人。无论老人小孩,她都觉得有可爱之处。可是她再不愿出去和别人见面了,尤其一想到别人见到她那副惊恐万状的样子,她就难受。一方面是自疚,觉得惹得别人讨厌,另一方面就不消说了。 
 
 就这样,她就自愿地关在这活棺材里,就是真正厌世的人恐怕也有心烦的时候,何况刘三姐!到了明月临窗,独坐许久又不思睡的时候,不免就要唱上几段。当然了,刘三姐不是李青莲,尽管唱得好,歌词也免不了俗套,唱来唱去,免不了唱到自吹自擂的地方:那些词儿就是海伦、克利奥佩屈拉之流也担当不起。 
 
 有一天半夜,刘三姐又被无名的烦闷从梦里唤醒,自知再也睡不成了,就爬起来坐着。土楼四面全是板窗,黑得不亚于大柜中间,也懒得去开窗,就那么坐着唱起来。哪知道声音忒大了点,五里之外也听得见。正好那天白沙是集,天还不亮就有赶集的从镇东头过。先是有几个挑柴的站住走不动了,然后又是一帮赶骡子的,到了那里,骡子也停住脚,鞭子也赶不动。后来,路上足足聚了四百多人,顺着声音摸去,把刘三姐的土楼围了个水泄不通。谁也不敢咳嗽一声,连驴都竖着耳朵听着。刘三姐直唱到天明,露水把听众的头发都湿透了。 
 
 那一夜,刘三姐觉得自己从来也没有唱得那么好。她越唱越高,听的人只觉得耳朵里有根银丝在抖动,好像把一切都为忘了。直到她兴尽之后,人们才开始回味歌词,都觉得楼上住的一定是仙女无疑,于是又鸦雀无声的等着一睹为快。谁知一头毛驴听了这美妙的歌喉之后,自己也想一试,于是也高叫起来:“欧啊!欧欧啊……”马上就挨了旁边一头骡子几蹄子,嘴也被一条大汉捏住了。可是已经迟了,歌仙已经被惊动了,板窗后响起了启梢的声音,说时迟那时快!五六百双眼睛(骡马的在内)一齐盯住窗口…… 
 
 砰的一声,窗子开了。下面猛地爆发出一声呐喊:“妖怪来了!”人们转头就跑,骡马溜缰撞倒人不计其数,刹时间跑了个精光。只剩一头毛驴拴在树上,主人跑了,它在那里没命地四下乱踢,弄得尘土飞扬。 
 
 刘三姐楞在那儿了。她不知道下面怎么聚了那么多人,可是有一点很清楚,他们一定是被她那副尊容吓跑了的。她伏在窗口,哭了个心碎肠断。猛然间听见下面一个声音在叫她:“三姐儿!三姐儿!” 
 
 刘三姐抬起头,擦擦眼里的泪,只看见下面一个人扶着柳树站着,头顶上斑秃得一块一块的,脸好像一个葫芦,下面肥上面瘦。一个酒糟鼻子,少说也有二斤,比鸡冠子还红。短短的黄眉毛,一双小眼睛。唱得东歪西倒,衣服照得见人,口齿不清地对她喊:“三,三姐儿!他们嫌你丑,我我我不怕!咱们丑丑丑对丑,倒是一对!你别不乐意,等我酒醒了,恐怕我也看不上你了!” 
 
 刘三姐认出此人名叫陆癞子,是一个不可救药的酒鬼兼无赖,听他这一说,心里更酸。砰地关上窗子,倒在床上哭了个够。 
 
 从此之后,刘三姐在这个土楼上也呆不住了。她从家里逃到这个土楼上,可是无端的羞辱也从家里追了来。可是她有什么过错呢?就是因为生得丑吗?可是不管怎么说,人总不能给自己选择一种面容吧!再说刘三姐也没有邀请人们到土楼底下来看她呀! 
 
 刘三姐现在每天清晨就爬起来,到江边的石山上找一个树丛遮蔽的地方坐起来,看着早晨的浓雾怎样慢慢地从江面上浮起来,露出下面暗蓝色的江水。直到太阳出来,人们回家吃饭的时候再沿着小路回去。到下午,三姐干完了园子里的活,又来到老地方,看着夕阳的光辉怎样在天边创造辉煌的奇迹。等到西天只剩下一点暗紫色的光辉,江面只剩下幢幢的黑影的时候,打渔人划着小竹筏从江上掠过,都在筏子上点起了灯笼。江面上映出了粼粼的灯影,映出了筏边上蹲着的一排排渔鹰,好像是披着蓑衣的小个子渔夫。 
 
 打渔的人们有福了,因为他们早晚间从白沙东山边过的时候,都能听见刘三姐美妙的歌声。说来也怪,三姐的歌里永远不含有太多的悲哀。她总是在歌唱桂林的青山绿水,漓江的茫茫江天,好像要超然出世一样。 
 
 下游三十里的地方有一个兴坪镇,有一个兴坪的青年渔夫阿牛有次来到这里,马上就被三姐的歌声迷住了。以后每天早上,三姐都能看见阿牛驾着他的小竹筏在下面江上梭巡。阿牛的竹筏是三根竹子扎成的,窄得吓死人,逆着激流而上时,轻巧得像根羽毛。他最喜欢从江心浪花飞溅的暗礁上冲下去,小小的竹排一下子沉到水里,八只渔鹰一下子都不见了。等到竹筏子浮出水面,它们就在下面老远的地方浮出来,嘴里常叼着大鱼。这时候阿牛就哈哈大笑,强盗似的打一声唿哨,可是刘三姐在山上直出冷汗,心里咚咚直跳,好像死了一次才活过来一样。 
 
 每当刘三姐唱起歌来的时候,阿牛就仰起头来静听,手里的长桨左一下右一下轻轻地划着,筏头顶着激流,可是竹筏一动不动就好像下了锚一样。 
 
 有时阿牛也划到山底下,仰着头对着上面唱上一段。这时刘三姐就能清楚地看见他乌黑的头发,热情的面容。高高的鼻梁上,长着一个嘻嘻哈哈的大嘴,好像从来也没有过伤心的事情,不管什么事情他都耍笑一番。刘三姐心里觉得很奇怪:世界上竟有这样的小伙子,简直是神仙!只要阿牛把脸转向她这边,她就立刻把头缩到树丛里,隔着枝叶偷看。不管阿牛多么热情地唱着邀请她出来对歌的歌曲,她从来不敢答一个字。直到阿牛看看没有希望,耸耸肩膀,打着桨顺流而下时,她才敢探出头来看看他的背影。这时她的吊眼角上,往往挂着眼泪。 
 
 自从阿牛常到白沙之后,刘三姐的日子就更不好过了。每天从江边回来,刘三姐心里都难过得要命,更可怕的是阿牛打着桨在山下的时候,刘三姐提心吊胆往树丛后面缩,弄得大汗淋漓。最让人伤心的是阿牛唱的山歌,没有一次不是从赞美刘三姐的歌声唱到赞美她的容貌,那些话听起来就像刀子一样往心里扎。 
 
 可是刘三姐又没法不到江边去,到了江边又没法不唱歌。有次刘三姐决心不唱了,免得再受那份洋罪,于是阿牛以为刘三姐没来,心神恍惚地差点撞在石头上,把刘三姐吓出了一头冷汗。再说她也很愿意听阿牛豪放、热情的歌声。更何况刘三姐的境况又是那么可怜,从来也没有人把她看成过一个人。阿牛现在又是那么仰慕她,用世界一切称颂妇女最高级形容词来呼唤她。可是他哪里知道这些话都是刘三姐最难下咽的苦酒。 
 
 又有一天,那是个令人愉快的美好的晴天:金光闪耀在江面上,黑绿的山峰上,漓江水对着天空露出了蔚蓝的笑脸。刘三姐又坐在老地方,听着阿牛的歌声,心里绝顶辛酸。 
 
 “对面山上的姑娘,你为何不出来见面?你看看老实的阿牛,为了你流连难返。如果你永远不出来,我也情愿在这里。我是阿牛、阿牛、阿牛,为了你流连难返。” 
 
 刘三姐再也听不下去了,用手捂着耳朵;可是她仍然听见阿牛叹了一口气,看见他懒洋洋地抄起长桨,将要顺流而下。她心里怦怦乱跳,觉得泪水在吊眼角里发烫。猛然间,她的歌声冲出了喉咙,好像完全不由自主一样:“我是兴坪刘三姐,长得好像大妖怪。哥哥见了刘三姐,今后再也不会来,阿牛哥,阿牛哥,”……刘三姐忽然发现她泣不成声了。 
 
 阿牛沉默了。他低着头用长桨轻轻地拨着水面。刘三姐感到胸中有什么东西破裂了,一阵剧疼之后,忽然感到莫名其妙的快慰。原来阿牛也害怕她。 
 
 大概阿牛也曾对刘三姐其人有些耳闻吧!可是他沉思之后,毅然地抬起头来说:“我不怕!我阿牛不比他们,慢说你还不是妖怪,就是真妖怪,我也要把你接到家里来!现在你站出来吧!” 
 
 现在轮到刘三姐踌躇不定了,她决不愿把那面丑脸给任何人看!可是阿牛斩钉截铁的要求又是不可抗拒的,于是刘三姐觉得心好像被两头牛撕开了;她既不敢探出头去,又不敢拒绝阿牛,心里直想拖下去,可是最后一幕的开场锣鼓已经敲响,她还要躲到哪去!啊,但愿她这辈子没活过! 
 
 最后,阿牛听见刘三姐用微弱的声音哀求:“阿牛哥,明天吧!” 
 
 阿牛坐在竹筏上,任凭江水把他送到下游去。他不能相信,那么美妙的声音会从一张丑脸下发出来!可是就算她丑又怎么样?他无限地神往江上那个美妙的声音,就是那声音,好像命运的绳索一样把他往那座山峰边上拉。不管怎么样,她也不会把他吓倒。对不对,渔鹰们? 
 
 渔鹰们在细长脖子上会意地转转脑袋,好像在回答阿牛:它们并不反对!她一定是个好人,不会饿着它们的。阿牛哥,你下决心吧! 
 
 夕阳的金光沿着江面射来,在阿牛身上画出了很多细微的涟漪。对!他做得对!刘三姐是个悲伤的好人,她一定会是阿牛的好妻子!再说,怎见得人家就像传闻的那么丑?阿牛难道没见过那些好事之徒,怎么糟蹋人吗?怎么能想象,一个恶心的丑八怪能有一个美妙的歌喉?最可能的是,刘三姐有一点丑,但是决不会恶心人,更不是像人们说得那么伧俗不堪!他阿牛才不相信那些人们的审美能力呢!对了,也许干脆刘三姐根本不丑?或者更干脆一点,甚至很漂亮?可能!阿牛曾经见过一个受人称赞的美人,长了一个恬不知耻的大脸,脸蛋肥嘟嘟的,站着就要像个蛆一样乱扭,表情呆滞,像头猪!他们那些人哪,不可信! 
 
 阿牛信心百倍地站起来,把筏子划得像飞一样从江上掠过。 
 
 刘三姐直等到阿牛去远才想到要离开。两腿发软,要用手扶着石头才能站起来。她看看四周,真想干嚎一通,然后一头撞在石头上。啊呀天哪,你干吗这么作弄人!阿牛看见我一定也会吓个半死,然后逃走!老天爷,你为什么要我碰上好人?跟坏人在一起要好得多!明天哪里还敢上这儿来?我要永远看不见阿牛了,这个罪让我怎么受哇! 
 
 刘三姐走下山岗,心里叫失望咬啮得很难过。她才有了一点快慰,不不,审美快慰,简直是受苦!可是以后连这种苦也吃不上了。也许该找把刀把脸皮削下来?不成,要得脓毒败血症的。怎么办? 
 
 刘三姐猛的站住了。现在,附近的竹林,村庄都沉入淡墨一样的幽暗中了,可是金光还在那边山顶上朝上空放射着。一切都已沉寂,夜晚尚未到来。头顶的天空上,还飘着几片白云。可是好像云朵也比白天升高了,朝着高不可攀的天空,几颗亮星已经在那里闪亮。高不可攀的天空,好像深不可测,直通向渺渺的,更伟大的太空,但是被落日的金光仰射着,明亮而辉煌。在那里,最高、最远的地方,目力不可及的地方,是什么? 
 
 刘三姐忽然跪下了。她不信鬼神,但是这时也觉得,人生一定是有主宰的,一切人类的悲切,真正内在的悲切,都应该朝它诉说。 
 
 刘三姐不信上帝。她心里想到人们说的长胡子的玉皇大帝,就觉得可笑,以为不可能有。但是现在她相信,她的一切不为人信的悲切会有什么伟大的、超自然的东西知道。会有这种东西,否则世界与个蚁窝有什么两样! 
 
 她静静地跪着,内心无言朝上苍呼吁。可是时间静静地过去,四周黑下来了。什么事情也没发生。刘三姐站起来,默默朝家走去。说也奇怪,她的内心现在宁静得像一潭死水一样。 
 
 她走着,四周又黑又静,心里渐渐开始喜悦地觉得到,身上有点异样了。胸口在发热!一股热气慢慢地朝脸上升来,脸马上烫得炙手。上帝!上帝!刘三姐走回土楼躺在床上,浑身发烫,好像发了热病一样。 
 
 她偷偷伸出手来,摸摸自己的脸,好像细腻多了。似乎吊眼角也比原先小了。粗糙的头发也比较滋润了。刘三姐躺了半夜,不断有新的发现,直到她昏然睡去。 
 
 第二天刘三姐醒来的时候,天已经大亮了。刘三姐爬起来洗脸,很想找个镜子照照自己,但是找不到。原来倒是有两个镜子,可是早被她摔碎了,连破片也找不到。 
 
 她朝江走去,心里感到很轻快。但是过了一小会,心里又开始狐疑了。凭良心说,她根本不相信世界会出现奇迹,因为她从来也没有看见过奇迹。但是她现在宁可相信有这种可能。“有这种可能吗?有的,但是为什么以前没有听说过这种事情?而且以前也没有想到过有这种可能?咳,因为以前没有想到过应该向上苍请求啊!我多傻!” 
 
 刘三姐坚决地把以前的自己当成傻瓜,把今天的自己当成聪明人。于是感到信心百倍。为了免得再犯狐疑,索性加快脚步,心里什么也不想了。 
 
 等她爬上小山,从树丛后面朝江上一看,阿牛已经等在下面了。 
 
 阿牛早就听见了山上的脚步声,抬起头来大声说:“刘三姐,早上好哇!” 
 
 山上也传来刘三姐的回答:“你好,阿牛哥!” 
 
 这是又一个美好的晴天,江上的薄雾正在散去。太阳的光芒温暖地照在阿牛的身上,江水在山边拍溅。四下没有一个人,江上没有一只船。只有阿牛的小竹排,顶着江水飘着。阿牛抬起头,八只渔鹰也侧着脑袋,十只眼睛朝山上望去。 
 
 阿牛等待着,就要看见一个什么样的人呢?脸一定比较的黑,嘴也许相当大。但是一定充满生气,清秀,但是不会妖艳。当然也许不算漂亮,但是绝对不可能那么恶心人。 
 
 阿牛正在心里描绘刘三姐的容貌,猛然,在金光闪耀的山顶,一丛小树后面,伸出一张破烂茄子似的鬼脸来,而且因为内心紧张显得分外可怕:嘴唇拱出,嘴角朝上翘起,吊眼角都碰上嘴了!马上,江上响起了落水声,八只渔鹰全都跳下水去了。阿牛瞠目结舌,一屁股坐在竹排上,被江水带向下游。 
 
 中午时分,阿牛在白沙附近被人找到了。他坐在竹排上,眼睛直勾勾的,不住地摇头,已经不会说话了。在他身边站着八只渔鹰,也在不住地摇头。以后,他的摇头疯再也没有好。二十年后,人们还能看见他带着八只也有摇头疯的渔鹰在江上打渔。那时候,阳朔比现在要多上一景:薄暮时分,江面上几个摇摇晃晃的黑影,煞是好看。当时这景叫白沙摇头,最有名不过了。可惜现在已经绝了此景。 此后,人们再也没看见刘三姐。最初,人们在江面上能听见令人绝倒的悲泣,久后声音渐渐小了,变得隐约可闻,也不再像悲泣,只像游丝一缕的歌声,一直响了三百年!其间也有好事之徒,想要去寻找那失去踪迹的歌仙。他们爬上江两岸的山顶,只看见群山如林,漓江像一条白色的长缨从无际云边来,又到无际云边去。顶上蓝天如海,四下白云如壁。  
