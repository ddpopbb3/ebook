\chapter{似水流年}

王二年表: 

一九五○年出生。 

一九六六至一九六八年,“文化革命”。住在矿院,是一名中学生,目睹了贺先生跳楼自杀和李先生龟头血肿。 
一九六八年,和许由在地下室造炸药玩,出了事故,大倒其霉。先被专政,后被捕,挨了很多揍。 

一九六九至一九七二年,被释放。到云南插队。认识陈清杨。 

一九七二年至一九七七年、在京郊插队。与小转铃交好。与刘先生结识,刘老先生死。后来上调回城,在街道厂当工人。 

一九七七至一九八一年,上大学。 

一九八一至一九八四年,毕业,三十而立。与二妞子结婚。 

一九八五至一九九○年,与旧情人线条重逢,很惊讶地发现她己嫁了李先生。出国读学位。丧父。离婚。回国。 

一九九○年,四十岁。

\section{一}

岁月如流,如今已到了不惑之年。我现在离了婚,和我母亲住在一起。小转铃有时来看我,有时怄了气,十几天都不露面。如今我基本上算是一个单身汉。 

我住的是我父亲的房子,而我父亲已经不在了。我终于调进矿院来,在我父亲生前任教的学校教书。住在我家对面的是我的顶头上司李先生。李先生的夫人,是我的老同学,当年叫线条。线条在“文化革命”里很疯,很早就跑出来,和男孩子玩。现在提这些 事不大应该,但是我想,线条不会见我的怪。因为她就是和我玩的。也可以说,我们俩是老情人。 

至于李先生,更不会见怪,因为他不在乎这些事。除此之外,他和我的交情非常好。他从海外回大陆,第一个能叫上名字的人就是我。他还是个不善交际的人,直到现在,除了夫人之外,也就是和我能聊聊。我不知他在国外的情况,反正在中国,能说说心事的,也就是一个线条,一个王二。这实在不算多。用李先生的话说,别人和他没有缘。我也把李先生当个朋友。我向来不怕得罪朋友,因为既是朋友。就不怕得罪,不能得罪的就不是朋友,这是我的一贯作风。由这一点你也可猜出,我的朋友为什么这么少。 

我现在没有几个朋友了。许由找了个出国劳务的话,到中东去修公路。陈清扬见不着。小转铃说,我对线条旧情不断,还说我是癞蛤蟆想吃天鹅肉。她简直是个醋葫芦。我爱上了李先生老婆。李先生不知道,还说我和他有缘。该着做朋友。 


李先生说,和我有缘,这种缘分起源于二十三年前一个冬日的早上。那时我是十七岁一个中学生,个子像现在一样高,比今天瘦很多,像竹竿一样。头上戴狗皮帽,身穿蓝制服罩棉袄,脚下穿大头皮鞋,这身打扮在当时很一般。我身上的衣服不大干净,这在当时也很一般。我那顶帽子是朋友送的,而他也不是好来的,不是偷来就是抢来的,这在当时也很一般。当年的中学生,只要不是身体单薄性情懦弱,有谁没干过几件坏事,抢几顶帽子实在一般——我就这个样子走到矿院的大操场上去看大字报。在六七年大字报已没有了轰动效应,但是还有不少东西可看。某先生早年留学日本时去嫖妓,想赖嫖资;某教授三年困难下矿山,吃招待饭时偷了馒头藏在怀里;某书记当年贪污了党的经费,给自己打了一个银烟盒等,颇为有趣。看这种东西很容易入迷。不知不觉自己也变成了坏蛋。假如再有“文化大革命”,这种东西我绝不看了。在当年我有一个习惯,就是每天要把全院的大字报看一追。矿院很大,大字报很多,所以不能全看完。有些我只看看标题,有些览其大略,有些有趣的我仔细看。就是这样,还得起旱贪黑。一大早我就到了大操场上,而大操场早被席棚隔成了九宫八卦之型。我在八卦之中走动,起得早了,没碰见人。转了几个圈后遇上了第一个人,他躺在地上像条死鱼。这就是李先生。 

把时间推到二十三年前,李先生刚从香港回内地,过冬的衣服都是临时置办的。他身穿一件蓝色带风帽的棉大衣,北京人叫棉猴的那种东西,又小又旧,也不知是谁给他的。李先生个子小,那棉猴比他还小。可见是小孩子穿过的东西。棉猴下是粗呢裤管,这是他从海外带回来的东西。粗呢裤下是一双又肥又大的塑料底棉鞋,这是他在北京买的。李先生胡子拉碴,戴一副瓶子底也似的眼镜。我见时他就是这副样子倒在地上,半闭着眼睛,不见黑眼珠,浑身打着哆嗦,很像前几天跳楼的贺先生刚着地时的样子。但是仔细看时颇有不同,贺先生的脑子当时是洒出来的,而李先生的脑子还在脑壳里面,这是最主要的不同之点。贺先生从楼上跳下来时,我不在现场,是后来得到消息赶去的。虽然去得很快,也错过了不少场面。据说贺先生刚落地时,还在满地打滚,这场面我就没看见。据说贺先生的手还抓了两把,我也没看见。贺先生死时的景象,我几乎都没看见,只看见他最后抽抽了两下。这使我很没有面子。所以看见李先生倒在地下,我大为兴奋。虽然我拿不准他死了没有。 

假如我知道李先生没死,只不过是晕了过去,那么我肯定会去救他。虽然我当时很瘦,但是“文革”前的孩子重视体育,所以都有一把力气,李先生又不重,我把他扛走没什么问题。但是当时我以为他有可能已经没救了,在这种情况下,就该保护现场,等待警察。既然我拿不准他死没死,还有第三种办法:我去喊几个人来,看看他死没死。这个办法我最不乐意。设想李先生已死,我又离开了现场,别人再撞上了,那时我再说我是第一个到达现场之人,谁还肯信?就算信了,对我更不好,他们会说,王二叫死人吓跑了。如今到了不惑之年,我不怕人家说我胆小了。经过了插队,当工人,数十年的时间,所到之处人都说我胆子非常大,胆大心黑,色胆包天,胆大妄为等等。偶尔有人说一句王二胆小,我也不觉得有什么。可是在当时,我就怕人说我这个。因此我采取了第四个办法,站在当地不动,看李先生是越抽越厉害还是越抽越硬邦。假如是后者,我就嚷嚷起来。假如是前者,我就过去扛他。谁知他很快就睁开了眼睛,坐起身来,这叫我大失所望。我转过身去,准备走了。 

在李先生看来,那天早上的事就没这么轻松。当时他从香港赶来参加“文化革命”(后来他说,这是他这辈子犯的最大的错误),头天晚上刚到矿院,早上就来贴大字报。谁知和别人起了争执,遭人一脚踢成了重伤,晕倒在地。醒来一看,大出意料:原来没躺在医院里,也没人围着他。踢他的人也不见了。只有一个半桩孩子在一边看着,而且那孩子有姗姗离去之势。所以他急忙叫我回去搀他一把。李先生说,当时他伤处极疼,没人架一把一步也走不动。而我却摇头晃脑,好半天才走过去,可把他急坏了。所以等他能够上,就一把搂住了我的脖子,再也不敢放,生伯我也跑了。结果到了医院,我脖子上被箍出了一溜紫印。在这种情况下,我当然不肯再搀他回去,抽个冷子就跑掉了。这下又苦了李先生,他根本不认识回去的路,花了几倍的工夫才回到了矿院。 

对于这件事我还有些补充。当时我不认识李先生,不知他是矿院的人。假如认识,抢救的态度会积极一点。我也不知他是被人摆平的,还以为他是在抽羊角疯。假如知道,抢救的态度也会积极一点。做了这两点辩护之后我也承认,当时我对死人特别有兴趣,对活人不感兴趣。李先生说,他对我当时的心情能够理解。有件事他不能理解,就是那一脚踢得委实利害。只要再踢重一点,他就会变成我感兴趣的人。 

李先生挨那一脚的事是这样的:六七年大家都想写些大字报贴出去,然后看见别人在自己写的东西面前交头按耳,议论纷纷,这和我今天想发表作品的心情是一样的。顶叫人愤怒的是,自己辛辛苦苦写了一夜,才贴出去就被人盖掉。所以都在大字报上写着:保留五日,保留十日,无奈根本没人给你保留。那年头为这种事吵嘴、动手的事也不知有多少。李先生的大宇报正贴在司机班一伙冒失鬼好不容易诌出的大字报上,而且被本主当场逮到。叉住了脖子和他理论,和他又理论不清。因此照他档下踢了一脚,人家怎么也想不到他会让人踢个正着。当时我们院谁不知道司机班那伙人?只有李先生不知道。所以连挨揍的准备都没有。这一脚踢出麻烦来了,眼见得李先生脸色也变了,眼珠子也翻了,软软地挂在人家手上。人家也怕吃人命官司,赶紧把他放在地上跑掉了。谁又能想到他还有救呢?假如送他上医院,万一他又没救了呢? 

现在我们院的人都在背后叫李先生龟头血肿,包括那些没结婚的小姑娘。她们说,李先生原是日本人,姓龟头,名血肿。这是不对的。李先生从未到过日本。他叫这个名字,是因为他挨了一脚后,十分气愤,就把医院的诊断书抄出来寻求公道,那诊断中有这样的字句:“阴囊挫伤,龟头血肿”。他寻到的公道就是从此被叫作龟头血肿,一肿二十三年,至今还没消。


\section{二}

十几年后,我到当年李先生拿博士的学校里读书。李先生毕业后还在这儿任了两年教,所以不少人还记着他。人家对他的评价是:性情火爆,顽固到底,才华横溢。乍一听只觉得自己的英文出了问题:李先生性情火爆?他是最不火爆的呀!

李先生的才华横溢我倒是见过,那是在他被人龟头血肿了之后。他连篇累牍地写出了长篇大字报,论证龟头血肿的问题。第一篇大字报开头是这样的:李某不幸,惨遭小人毒手,业已将经过及医院诊断,披露于大字报。怎知末获矿院君子同情,反遭物议;兄弟不得不再将龟头血肿之事,告白于诸君子云云。

这篇大字报的背景是这样的:他把医院的诊断画成大字报贴出来,就有些道学的人在上面批:这种东西也贴出来,下流!无耻!至于他怎么挨了人踢,却没人理会。所以李先生在大字报里强调:李某人的龟头,并非先天血肿,而是被人踢的。 

李先生在大字报里说,他绝不是因为吃了亏,想要对方怎样赔罪才写大字报。他要说的是:龟头血肿很不好,龟头血肿很疼。龟头血肿应该否定,绝不要再有人龟头血肿。他这些话都被人看成了奇谈怪论。到这时,他回来有段日子了,大家也都认识他。在食堂里大师傅劝他;小李呀,拉倒吧。瞧瞧你被人踢的那个地方,不好张扬。李先生果然顽固,高声说:师傅,这话不对。人家踢我,可不是我伸出龟头让他踢的!踢到这里就拉倒,以后都往这里踢! 


虽然没有人同意李先生的意见,但是李先生的大字报可有人看。他就一论龟头血肿,二论龟头血肿,三论四论地往外贴。在三论里他谈到以下问题:近来我们讨论了龟头血肿,很多人不了解问题的严重,不肯认真对待,反而一味噎笑。须知但凡男人都生有龟头,这是不争的事实。龟头挨踢,就会血肿,而且很疼,这也是不争的事实。不争的事实,何可笑之有?不争的事实,又岂可不认真对待之?他这么论来论去,直把别人的肚子都要笑破。依我看,这龟头血肿之名,纯粹是他自己挣出来的。 

李先生论来论去,终于有人贴出一张大字报讨论龟头血肿问题,算是有了回应。那大字报的题目却是;龟头血肿可以休矣。其论点是:龟头血肿本是小事一件,犯不上这么喋喋不休。在伟大的“文化革命”里,大道理管小道理,大问题管小问题。小小一个龟头,它血肿也好,不血肿也罢,能有什么重要性?不要被它干扰了运动的大方向。一百个龟头之肿,也比不上揭批查。这篇大字报贴出来,也叫人批得麻麻扎扎:说作者纯属无聊。既知揭批查之重要,你何不去揭批查,来掺和这龟头血肿干嘛。照批者的意见,这李先生是无聊之辈,你何必理他?既然理他,你也是无聊之辈。但是李先生对这大字报倒是认真答辩了。他认为大道理管小道理,其实是不讲理。大问题管小问题,实则混淆命题。就算揭批查重要,也不能叫人龟头血肿呀?只论大小重要不重要,不论是非真伪,是混蛋逻辑。他只顾论着高兴,却不知这大小之说大有来头。所以就有人找上门,把他教训了一顿。总算念他是国外回来的左派,不知不罪,没大难为他。要不办起大不敬罪来,总比龟头血肿还难受。李先生也知道利害,从此不再言语。这龟头血肿之事,就算告一段落。 

流年似水,转眼就到了不惑之年。好多事情起了变化。如今司机班的风师傅绝不敢再朝李先生裤挡里飞起一脚弹踢,可是当年,他连我们都敢打。院里的哥们儿,不少人吃过他的亏。弟兄们合计过好几回,打算等他一个人出来时,大家蜂拥而上,先请他吃几十斤煤块,然后再动拳脚。听说他会武功,我们倒想知道挨了一顿煤雨后,他的武功还剩多少。为了收拾这姓风的,我们还成立了一个“杀鸡”战斗队,本人就是该战斗队的头。我曾经三次带人在黑夹道里埋伏短他,都没短到。风师傅干过侦察兵,相当机警,看见黑地里有人影就不过来。第四次我们用弹弓把他家的玻璃打坏了几块,黑更半夜的他也没敢追出来。经过此事,司机班的人再不敢揍矿院的孩子。 

关于龟头血肿,我们矿院的孩子也讨论过,得到的结论是,李先生所论,完全不对。我们的看法是:世界上的人分两种,龟头血肿之人和龟头不肿之人。你要龟头不肿的人理解血肿之痛,那是完全不可能的。惟一的办法是照他裆下猛踢一脚,让他也肿起来。 

有关李先生龟头血肿的事还可以补充如下:那些日子里北京上空充满了阴霾,像一口陈结了的粘痰,终日不散。矿院死了好几个人,除贺先生跳楼,还有上吊的,服毒的,拿剪子把自己扎死的,叫人目不暇接。李先生的事,只是好笑而已,算不了大事情。

\section{三}

流年似水,有的事情一下子过去了,有的事情很久也过不去。除了李先生龟头血肿,还有贺先生跳楼而死的事。其实贺先生是贺先生,和我毫无关系。但是他死掉的事嵌在我脑子里,不把这事情搞个明白,我的生活也理不出个头绪。 

贺先生死之前,被关在实验楼里。据我爸爸说,贺先生虽然不显老,却是个前辈。就是在我爸的老师面前,也是个前辈。到“文化革命”前,他虽还没退休,却已不管事了。用他自己的话来说就是:“我一辈子的事都已做完,剩下的事就是再活几年。”我爸爸还说,贺先生虽然是前辈,却一点不显老,尤其是他的脑子。偶尔问他点事,说得头头是道,而且说完了就是说完了,一句多余的话也没有。据此我爸爸曾预言他能活到很多当时五十岁的人后面。他被捉进去,是因为当过很大的官。然后他就从五楼上跳下来了。 

贺先生从楼上跳下时,许由正好从楼下经过。贺先生还和许由说了几句话,所以他不是一下就跳下来的。后来我盘问了许由不下十次,问贺先生说了什么,怎么说的等等。许由这笨蛋只记得贺先生说了:“小孩,走开!” 


“然后呢?” 

“然后就是砰地一下,好像摔了个西瓜!” 

再问十遍,也是小孩走开和摔了西瓜,我真想揍他一顿。 

在我年轻时,死亡是我思考的主题。贺先生是我见过的第一个死人。我想在他身上了解什么是死亡,就如后来想在陈清扬身上了解什么是女人一样。不幸的是,这两个目标选得都不那么好。就以贺先生来说,在他死掉之前,我就没和他说过话。而许由这家伙又被吓坏了,什么都忘记了。你怎能相信,一个存心要死的人,给世界留下最后的话仅仅是“小孩走开”呢? 

贺先生后来的事我都看见了。他脑袋撞在水泥地上,脑浆子洒了一世界,以他头颅着地点为轴,五米半径内到处是一堆堆一撮撮活像新鲜猪肺的物质。不但地上有,还有一些溅到了墙上和一楼的窗上。这种死法强烈无比,所以我不信他除小孩走开之外没说别的。 

贺先生死后好久,他坠楼的地方还留下了一摊滩的污迹。原来人脑中有大量的油脂。贺先生是个算无遗策的人(我和他下过棋,对此深有体会),他一定料到了死后会出这样的事。一个人宁可叫自己思想的器官混入别人鞋底的微尘,这种气魄实出我想象之外。 

虽然贺先生死时还蒙有不白之冤,但在他生前死后,我从没对他有过不敬之心。相反,我对他无限祟拜,无限热爱。不管别人怎么说他(反动学术权威、国民党官僚等等),都不能动摇我的敬爱之心。在我心中,他永远是那个造成了万人空巷争睹围观的伟大场面的人。

\section{四}

前面提到李先生说过,取道香港来参加革命工作是个错误,这可不是因为后来龟头血肿起了后悔。起码他没对我说过不革命的话。他说的是不该走香港。在港时他遇上了一伙托派,在一起混了一些时,后来还通信。到了后来清理阶级队伍,把他揭了出来。 

李先生的托派嘴脸暴露后,我和线条在小礼堂见过他挨打。那一回人家把他的头发剃光,在他头上举行了打大包的比赛,打到兴浓时还说,龟头血肿这回可叫名符其实。线条就在那回爱上了他。二十三年前,线条是个黄毛丫头,连睫毛都发黄,身材很单薄,腰细得几乎可以一把抓,两个小小的乳房,就如花蕾,在胸前时隐时现。现在基本还是这样,所不同的是显得憔悴疲惫。她是我所认识的最疯最胆大的女人,尽管如此,我也没料到她会嫁龟头血肿。 


现在应该说到李先生挨打的情形。那个小礼堂可容四五百人,摆满了板条钉成的持子,我们数十名旁观者,都爬在椅子上看。李先生和参赛选手数人在舞台上,还有人把大灯打开了,说是要造造气氛。李先生刮了个大秃瓢,才显出他的头型古怪:顶上有尖,脑后有反骨,反骨下那条沟相当之深。这种头剃头师傅也不一定能剃好,何况在场的没有一个是剃头出身,所以也就是剃个大概,到处是青黑的头发茬。我在乡下,有一回和几个知青偷宰了一口猪,最后就是弄成了这个样子。我和线条赶到时,他头上的包已经不少了,有的青,有的紫,有的破了皮,流出少许血来。但是还没赛出头绪,因为他们不是赛谁打的包大,而是赛谁打出的包圆。李先生头上的包有些是条状,有些是阿米巴状,最好也是椭圆,离决出胜负还差得远。李先生伸着脖子,皱着眉,脸上的表情半似哭,半似笑,半闭着眼,就如老僧入定。好几个人上去试过,他都似浑然不觉。直到那位曾令他龟头血肿的风师傅出场,他才睁开眼来。只见风师傅屈右手中指如风眼状,照他的秃头上就凿,剥剥剥,若干又圆又亮的疙瘩应声而起。李先生不禁朗声赞道:还是这个拳厉害! 

线条后来对我说:那回李先生在台上挨打,那副无可奈何的样子真可爱!对此我倒不意外。李先生那样子,和E.T.差不多。既然有人说E.T.可爱,龟头血肿可爱也不足怪。线条还说,有一种感觉钻进心里来,几乎令她疯狂。她很想奔上前去,把他抱在怀里,用纤纤小手把那些大包抚平。这我也不意外,她经常是疯狂的。真正使人意外的是她居然真的嫁给了龟头血肿。 

我也爱过李先生。在我看来,一个人任凭老大凿栗在头上剥剥地敲,脸不变色眉不皱,乃是英雄行为。何况在此之前,他曾不顾恶名,愤起为自己的龟头论战。虽然想法有点迂,倒也不失为一条好汉。所以当他被关在小黑屋里时,我曾飞檐走壁给他送去了馒头。线条说,要给李先生以鼓励,我也不反对。她给他的条子,都是我送去的。那上面写着:龟头血肿,坚持住!我爱你!我想,哥们儿,你活着不容易。让我婆子爱爱你也无所谓。谁知到后来弄假成真。线条真成了龟头夫人!

\section{五}

那年贺先生从楼上跳下来,在地上抽了几下就不动了。然后不久,警察来验尸,把贺先生就地剥光。那时我站在人群的前列,脚下如穿了钉鞋,结结实实扎下了根,谁也挤不动。因此我就近目睹了验尸的全过程。等把贺先生验完,他已经硬了,因此剥下的衣服也穿不回去。警察同志们把裤子草草给他套到屁股上,把衣服盖在他身上,就把他搭上了车运走了。验尸中也没发现什么,只发现他屁股上有一片紫印。有位年轻的警察顺嘴说:他死!当时我觉得简直废话。“他”当然死了,你没看见他脑子都出来了吗?然后马上想到这可能是术语。回去一查辞书,果然是的。那位小警察也没什么证据说是他死,只不过那么多人瞪着眼看着,屁股上那么一大片淤伤,又黑又紫,不说点啥不好。最后结论当然是自杀。其实打在屁股上,不伤筋骨不害命,还是相当人道的。后来和贺先生关在一起的刘老先生出来,别人问他是准打的,他也说不太清楚,因为谁想起来都去打两下,只单单把风师傅点了出来,倒不说他打得狠,只说他带黑皮手套,拎根橡皮管子,一边打一边摸,弄得人怪不好意思。 

后来家属据此要告凤师傅,但是刘老先生已经中风死掉了,死无对证。贺先生死的情形就是这样。对此我有一个结论,觉得犯不上和风师傅为难,因为不管怎么说,他也不是个大坏蛋。闹了一回红卫兵,他干这点坏事,不算多。闹纳粹时,德国人杀得犹太几乎灭了种。要照这么算,风师傅只打屁股,还该得颗人道主义的奖章。问题不在这里。问题也不在贺家大多数人身上。贺老妈妈七十多,又是小脚,只想到告状,不能怪她缺少想象力。贺家大公子五十多岁,也不能怪他没想象力。贺家小公子,和我同年,叫做贺旗。原来在院里生龙活虎,也是一条好汉。我真不知他是怎么了。

\section{六}

下乡时,线条没跟我去云南插队。她跟父母下了干校,其实是瞄着李先生而去。当然他们的情形不一样,下干校时,线条是家属,爱干不干,十分轻松。而李先生是托派分子,什么活都得干。后来不说他是托派了,干校是工人师傅主事,又觉得这龟头血肿不顺眼,继续修理。当地农村之活计有所谓四大累之说,乃是:打井,脱坯,拔麦子,操屄。 

除了最后一项,他哪一样都干过。再加上挑屎挑尿,开挖土方,泥瓦匠,木匠小工;初春挖河,盛夏看青。晚上守夜,被偷东西的老农民揍得不善。幸亏是吃牛肉长大的,身体底子好,加之年龄尚轻,不到三十岁;要不线条准是望门寡。 

现在系里的人说起李先生,对他下干校时的表现都十分佩服。说他一个海外长大的知识分子,能受得了这些真不容易。更难得的是任劳任怨,对国家,对党毫无怨言,真是好同志,应该发展他入党。但是李先生说,他背着龟头血肿的恶名,恐怕给党抹黑—一还是等等吧。 

线条说,李先生那时的表现真是有趣极了。叫他干啥就干啥,脸上还老带着被人打包时的傻笑。她觉得龟头血肿这大E.T.简直是好玩死了。要不是干校里耳目众多,她早就和他搞起来了。 


后来李先生自己对我说,老弟,我们是校友,同行,又是同事,当年你还给我送过馒头,这关系非比寻常。所以,告诉你实话不妨。在干校的时候,我正在发俗懂,觉得自己着了别人的道儿。像我这样学科学方法的人,也有这种念头,实在叫人难以置信。但是想到我在大陆遇到的这些事,又是血肿,又是托派,又是满头大包,实在比迷信还古怪。还有一件更古怪的事:每天下工以后,床上必有一张纸条。所以我宁愿相信自己是得罪了人,正在受捉弄。第一个可疑分子就是我大学时同宿舍的印度师兄。有一回我嫌他在房间里点神香,就钻到厕所里弄点声音给他听,一连扳了七八下抽水马桶。这下把他得罪了,他就叫我做起噩梦来,一梦三年不得醒转。既然碰上了这样的非自然力,还是乖乖屈服为好,免得吃更大的苦头。李先生在干校里的事就是这样。 

李先生在下干校时,我在云南插队,认识了陈清扬,不再把线条放在心上,但是有时还想到贺先生的事。我想出了贺先生为什么临死时要叫小孩走开,这是因为在他死时,不喜欢有人看。 

“文化革命”前,矿院有个俱乐部,夏天的晚上,从八点到十一点,一直亮着灯,备有扑克象棋等等。那里有吊扇,沙发上还铺了花边,既凉快,又宽敞。每天晚上我部到那里去下棋。有一天人家告诉贺先生说,王二的棋非常厉害。贺先生头发油黑(是染的),指甲修过,声音浑厚,非常体面。他的棋也好,却下不过我。但是他常来找我下棋,输了也不以为羞。 

贺先生死时,头发半截黑半截白,非常难看。两只手别在后面,脖子窝着,姿势不自然。总的来说,他死时像个土拨鼠。贺先生肯定预见到自己死后的样子不好,所以不想让人看见。 

贺先生的尸体被收走后,脑子还在地下。警察对矿院的人说,这些东西你们自己来处理。矿院的入想了想说:那就让家属来处理好啦,留下几个人看尸体,别人一哄而散。等到天色昏暗,家属还不来,那几个人就发了火,说道:爱来不来,咱们也走,留下这些东西喂乌鸦。天将黑时起了风,冷得很。 

在云南时,我又想起了贺先生的另一件事。验尸时看见,贺先生那杆大枪又粗又长,完全竖起来了。假如在做爱前想起这件事,就会欲念全消,一点不想干。

\section{七}

我在美国时,常见到李先生的印度师兄。他是我的系主任,又是我的导师。所以严格的讲,他既是我师父,李先生就是我师叔,线条就是我师婶。我和李先生称兄道弟,已是乱了辈分,何况我还对李先生说:线条原该是我老婆。不过在美国可不讲究这个。我早把导师的名字忘了,而且从来就没记住。他的名字着实难念,第一次去见他,我在他办公室外看了半天牌子,然后进去说:老师,您的名字我会拼了,能教教我怎么念吗?每回去见他,都要请他教我念名字,到现在也不会念。好在我根本不认他是我师父——这样线条也不是我的师婶。 

我不认这位印度师父,还因为他实在古怪,和你说着话,忽然就会入定,叫也叫不醒。上课时讲科学,下了课聚一帮老美念喇嘛教的经,还老让别人摸他的脑袋,因为达赖喇嘛给他摸过项。虽然这么胡闹,学校还是拿他当宝贝。这是因为人家出过有名的书。照我看他书出得越多,就越可疑。李先生疑他和龟头血肿有关系,不是没有道理。 

李先生告诉我说,他在大陆的遭遇,最叫人大惑不解的是在干校挨老农民的打。当时人家叫他去守夜,特别关照说,附近的农民老来偷粪,如果遇上了,一定要扭住,看看谁在干这不屙而获的事。李先生坚决执行,结果在腰上挨了一扁担,几乎打瘫痪了。事后想起来,这件事好不古怪。堂堂一个doctor,居然会为了争东西和人打起来,而这些东西居然是些屎,shit!回到大陆来,保卫东,保卫西,最后保卫大粪。“如果这不是做噩梦,那我一定是屎壳郎转世了!”

\section{八}

后来我离开了云南,到京郊插队,这时还是经常想起贺先生。他刚死的时候,我们一帮孩子在食堂背后煤堆上聚了几回,讨论贺先生直了的事。有人认为,贺先生是直了以后跳下来的。有人认为,他是在半空中直的。还有人认为,他是脑袋撞地撞直了的。我持第二种意见。 

我以为贺先生在半空中,一定感到自己像一颗飞机上落下来的炸弹。耳畔风声呼呼,地面逐渐接近,心脏狂跳不止,那落地的“砰”的一声,已经在心里响过了。贺先生既然要死,那么他一定把一切都想过了。他一定能体会到死亡的惨烈,也一定能体会死去时那种空前绝后的快感。 

我在京郊插队时,我们家从干校回来过一次。和贺先生关过一个小屋的刘老先生也从干校回来,住在我家隔壁。我问刘老先生,贺先生有何遗言,刘老先生说,贺先生死时我不在呀,上厕所去了。要是在,还不拉住他?到了贺先生跳下去以后,脑子都撞了出来,当然也不可能有任何遗言。故尔贺先生死前在想些什么后来就无法考证,也就设法知道,他为什么直了。 

贺先生死那天晚上,半夜两点钟,我又从床上起来,到贺先生死掉的地方去。我知道我们院里有很多野猫,常在夏夜里叫春,老松树上还常落着些乌鸦,常在黄昏时哇哇地叫;所以我想,这时肯定有些动物在享用贺先生的脑子。想到这些事我就睡不着,睡不着就要手淫,手淫伤身体。所以我走了出去。转过了一个楼角,到了那个地方,看到一副景象几乎把我的苦胆吓破。只见地上星星点点,点了几十支蜡烛。蜡烛光摇摇晃晃,照着几十个粉笔圈,粉笔圈里是那些脑子,也摇摇晃晃的,好像要跑出来。在烛光一侧,蹲着一个巨大的身影,这整个场面好像是有人在行巫术,要把贺先生救活,后来别人说王二胆子大,都是二三十岁以后的事。十七岁时胆力未坚,遭这一吓,差点转身就跑。 


我之所以没有跑掉,是因为听见有人说:小同学,你要过路吗?过来吧。小心一点,别踩了。我仔细一看,蜡烛光摇晃,是风吹的;对面的人影大,是烛光从底下照的。粉笔圈是白天警察照相时画的。贺先生的脑子一点也没动。因此我胆子也大了,慢慢走过去。对面的人有四十多岁,是贺先生的大儿子。他不住院里,有点面生,但是认识。他披了一件棉大衣,脚下放了一只手提包,敞着拉锁。包里全是蜡烛。我问他:白天怎么没看见你?他不说话,掏出烟来吸。手哆里哆嗦,点不着火。我接过火柴,给他点上了烟。然后在他身边蹲下,说:我和贺先生下过棋。他还是不说话。后来我说:已经验过尸啦。他忽然说道:小同学,你不知道。根本投验过。根本没仔细验过。说着说着忽然噎住。然后他说:小同学,你走吧。 

我慢慢走回家去,那天夜里没有月亮,但有星光。对于我这样在那些年里走惯夜路的人来说,这点亮足够了。我在想,贺先生家里的人到底想怎样?反正贺先生死了,再也活不了。但是想到贺先生家里那些人,我就觉得很伤心。 

贺先生的儿女们在寒风里看守着那些脑浆,没有人搭理他们,那些脑浆逐渐干瘪下去。到后来收拾的时候,有一些已经板结了。所以后来贺先生的脑子有很大一部分永久地附着在水泥地上了。告诉我贺先生遗言的刘老先生也死了。在刘老先生生前,我对他没有一点好印象。这老头子在棋盘上老悔棋,明明下不过,却死不认输。我不乐意说死人坏话,但我不说出来,别人怎能知道呢?他嘴极臭,正对着人说话时,谁也受不了。 

有关贺先生直了的事,我还有一点补充。不管他是在什么时候直了的,都只说明一件事:在贺先生身上,还有很多的生命力。别的什么都不说明。

\section{九}

流年似水,转眼到了不惑之年。我和大家一样,对周围的事逐渐司空见惯。过去的事过去了,未过去的事也不能叫我惊讶。只有李先生龟头血肿和贺先生的事,至今不能忘。 


那一年冬天,北京没一个好天,看不见太阳。那时候矿院是个一公里见方的大院子,其中三分之二的地方是松树林。那时候有好多人(革命师生,革命职工)从四面八方来到矿院,吃了窝窝头找不到厕所,在松林里屙野屎,屙出的屎撅子粗得吓死人。那时候,矿院的路上大字报层层板结,贴到一只厚,然后轰地一声巨响,塌下一层来。许由的奶奶括了七十八岁,碰上脑后塌大字报,被这种声音吓死啦。那时矿院里有好多高音喇叭,日日夜夜响个不停。后来我们的同龄人都学不好英文:耳朵不好,听不见清辅音。那时候烂纸特多,有很多捡烂纸的孩子,驾着自制的小车,在马路上作优美之滑行。那时有很多疯子被放出来,并且受到祟拜。那时我刚过了有志之年,瞪大了眼睛,把一切都看在眼里。 

如果我要把这一切写出来,就要用史笔。我现在还没有这种笔。所以我叙述我的似水流年,就只能谈谈龟头血肿和贺先生跳楼,这两件事都没在我身上发生(真是万幸),但也和我大有关系。 

在结束这个话题之前,谈一点别的事情。我和许由造炸药,落到了保卫组手里,当时我身上有一篇小说的手稿,是我和我们院里的小秀才鸡头合著。王二署名不执笔,执笔的是鸡头。他犯了大错误,写小说用了真名,里面谈到了矿院诸好汉的名次,还提到了我们的各种丰功伟绩,飞檐走壁,抛砖打瓦之类。最不该的是把我砸凤师傅窗子的事都写上了,而后来我正是落到了凤师傅的手里,他把我的腰都打坏了。这件事情告诉我们:写小说不可以用真名,尤其是小说里的正面人物。所以在本书里,没有一个名字是真的。小转铃可能不是小转铃,她是永乐大钟。王二不是王二,他是李麻子。矿院不是矿院,它是中山医学院。线条也不是线条,她是大麻包。李先生后来去的地方,也可能不是安阳,而是中国的另一个地方。人名不真,地点不真,惟一真实的是我写到的事。不管是龟头血肿还是贺先生跳楼,都是真的,我编这种事干什么?

\section{十}

七二年底李先生被发到河南安阳小煤窑当会计。河南的冬天漫天的风沙,水沟里流着黑色的水,水边结着白色的冰。往沟里看时,会发现沟底灰色的沙砾中混有黑色的小方决。这些小方块就是煤。水是从地下流出来的,地下有煤,所以带出了这种东西。一阵狂风过去之后,背风的地方积下了尘埃。在尘埃的面上,罩着黑色的细粉。这件事也合乎道理,因为风从铁路边上煤场吹过来,就会把粉煤吹起来。早上他从宿舍到会计室去,路上见到了这些,觉得一切井然有序,不像在梦里。 

李先生那个时候对一切都持将信将疑的态度。 

李先生到会计室上班时,头上总裁一顶软塌塌的毡帽。这种帽子的帽边可以放下来,罩住整个面部,使头部完全暖和起来。这种感觉是好的。李先生喜欢,乐意,并且渴望一天到晚用毡帽罩住头部。因为河南冬天太冷,煤矿又在山上。虽然有煤烧,但是房子盖得不好,漏风,所以屋里也冷。但是科长看见他在屋里戴着毡帽,就会勃然大怒:你别弄这个鬼样子吓我好不好?说着就会把他头上的帽子一把揪下来。这件事完全不合道理。 

李先生去上班,身上穿蓝色大衣。这衣服非常大,不花钱就拿到了。这件事非常之好,虽然不合道理。给他这件大衣的是矿上的劳资料长,一个广东人。李先生见了他倍感亲切,这是因为李先生所会的三种语言中,广东话仅次于英语。他就想和他讲粤语。劳资科长说:你这个“同机”不要和我讲广东话啦别人会以为我们在骂他啦。这非常合理,在美国也是这样子的。不能在老美面前讲中国话。广东科长给了他这件大衣,说是劳保。李先生问,何谓劳保,广东科长说:劳保就系国家对你的关怀啦。这个话不大明白,李先生也不深问。劳保里还有些怪东西,橡胶雨衣,半胶手套,防尘口罩等等。李先生问了一句:我不下井,发我这些干什么。旁边有个人就猛翻白眼说:想下井?容易!李先生赶紧不言语了。在干校学习了两年,到底学会了一点东西。 


李先生上班时也穿着这件大衣不脱。科长苦着脸看他,直到李先生被看毛了才来:很冷吗,你这么捂着?真的很冷?遇到这种情形,李先生也不答话,只是走到窗前,仔细看看温度表。看完后心里有了底,就走回来坐下来。科长也跟着走过去,看看温度表,说道:十五度。我还以为咱们屋是冷库呢! 

李先生知道,放蔬菜的冷库就是十五度,谁说不冷?但是他不说。在噩梦里,说什么就有什么。假如把这话说了出来,周围马上变成冷库,自己马上变成一棵洋葱也不一定。在干校里已经学会了很多,比如上厕所捏着鼻子,下午一定会被派挑屎,臭到半死,科长说十五度不冷,李先生已有十分的把握一—假如一时不察,顺嘴说出不满的话,大祸必随之而至。李先生暗想:“这肯定是我的印度师兄想把我变成洋葱!” 

在一九七三年,李先生对他的印度师兄的把戏已谙然于胸,那就是说什么来什么,灵验无比。这个游戏的基本规则就是人家叫你干啥,不要拒绝;遇上不舒服不好受的事应该忍受,不要抱怨。只要严守这两条,师兄也莫奈他何。 

李先生上班时脚上穿双大毛窝。他不适应北方气候,年年长冻疮。以前在美国,天也有冷的时候,那时不长冻疮。毫无疑间,这必是印度师兄搞的鬼。李先生认为,印度师兄这一手不漂亮。别的事印度人搞得很漂亮。比方说,龟头血肿,一个极可笑的恶作剧。满头起大包也想得好。有些地方师兄的想象力叫人叹为观止,包括叫他流落到河南安阳,中国肯定没有这么个地方。但是地名想得好:安阳。多像中国的地名啊!我要是个印度人,准想不出这么个地名来。但是长冻疮不好,一点不像真的。将来见了我也不好解释。别的事都是开玩笑,出于幽默感,冻疮里没有幽默感,只有恶意。 

李先生并不是死心塌地的相信眼前是一个噩梦或是印度人的骗局。那天早上到会计室上班,顶着很大的风。风里夹着沙粒,带来粗砾的感觉。说印度人能想出这样的感觉,实在叫人难以置信。风从电线,树枝,草丛上刮过,发出不同的声音。如果说,这声音是印度人想出的,也叫人不敢信。人类在一个时间只能想一件事,不可能同时造出好几钟声音。如果说,这一切都是印度人的安排,那么也是借助了自然的力量。这就是说,眼前的一切,既有真实的成分,也有虚构的成分。困难的是如何辨认,哪一些是虚构,哪一些是真实。 

那天早上李先生到会计室上班,科长不在,他有如释重负之感。那个科长非常古板,一天到晚的找麻烦。李先生不会打算盘,要算时总是心算。他的心算速度非常之快,而且从不出错。但是科长不但强迫他把算盘放在桌上,而且强迫他在算帐时不停地拨算盘珠。所以他见到科长不在,就赶快把算盘收起来,他一见到这东西就要发疯。 

如果算盘放在他面前,李先生就忍不住琢磨,这个东西到底有什么用处。在他看来,那东西好像是佛珠一类的东西,算帐时要不停地捻动,以示郑重。但是这佛珠的样子,真是太他妈的复杂了,简直不是入想出来的。然后他把脚翘在桌上,舒舒服服地坐着,把今天早上的所见仔细盘算一番。他觉得只要科长不在,别的人也不在,只有他—个入的时候,一切都比较贴近于自然。而当他们出现时,一切都好像出于印度师兄的安排。这种安排只有一个目的,就是要把他逼疯。其实他也没干什么坏事,不过是多扳了几下抽水马桶而已。为了这点小事把他灭掉,这印度人也太黑了! 

李先生后来说,他觉得那时候自己快发疯了。一方面,他不脱科学方法论的积习,努力辨认眼前的事,前因如何,后果如何,如何发生,如何结束,尽量给出一个与印度师兄无关的解释。另一方面,不管他怎么努力,最后总要想到印度人身上去。到了这时,就觉得要发疯:想想看,我们俩同窗数年,感情不错,他竟如此害我!惟一能防止他疯掉的,是他经常在心里长叹一声说:唉!姑妄听之吧。然后就什么也不想了。 

那天早上有人到会计室来,告诉李先生,山下有人找。李先生锁上门,往山下走,老远看见矿机关那片白房子。当时他精神比较好,又恢复了格物致知的它毛病,想道:这片房子在山的阳面,气候较好。比较干燥,冬天也暖和。而且是在山下,从外面回来不必爬山。把全矿的党,政,工,团放在那里,十分适宜。而全矿的大部分房子都在上面一条山沟里,又黑又潮,这也合乎道理,因为坑口在山沟里。你总不能让工人爬四百级台阶上来上班,这样到了工作现场(掌子面),累得上气不接下气,就不能干活了。所以这一个矿分了两个地方,是合乎情理的,并不可疑。 

山下的房子雪白的墙面,灰色的瓦面,很好看,这也合乎道理。因为那是全矿的门面嘛。但是走近了一看,就不是那么好。雪白的只是面上的一层灰。灰面剥落之处,裸露出墙的本体,是黄泥的大块(土坯——王二注)。仰头一看,屋格下的椽子都没上漆,因为风化之故,木头发黑。窗上玻璃有些是两片乃至三片拼出来的,门窗上涂的漆很薄,连木纹都遮不住。这也不难解释,矿上的经济状况不是太好。 

有关矿的经济情况,矿长知道的应该是最多。他说:同志们,要注意勤俭节约。我们是地方国营嘛。地方国营是什么,相当难猜,但也不是毫无头绪。在一些香烟和火柴盒上,常见这宇样。凡有了这四个字的,质量就不好,价格也不贵。在美国也是这样,大的有名的公司,商品品质好,卖的也贵。小的没名的公司,东西便宜,货也不好。在超级市场里有些货是白牌,大概也是地方国营。可以想见地方国营的煤矿,经济上不会宽裕,办公的房子也就很平常。 

就是不知道地方国营是什么意思,李先生也能猜出矿的经济状况。井下还是打钎子放炮,有两辆电瓶车,三天两头坏。坏时李先生就不当会计,去帮着修电瓶车。李先生说,我可不会修电瓶。可是人家说:管你会不会,反正你是矿院下来的,没吃过猪肉,总见过猪跑吧。在一边蹲着,出出主意。这是因为电瓶车坏了,井下的煤就得用人力推出来。要是大电机坏了,连医务室的大夫也得到一边蹲着去。她百无聊赖,就给大家听听肺。试想一个矿,雇不起工程师,把会计和医生拉去修电机,这是何等的因境。矿里还有三台汽车,有一台肯定在美国的工业博物馆里见过。这件事想不得,一想就想到印度师兄身上去。 

李先生走到矿上会议室门前时,精神相当稳定,这是因为早上格物致知大获成功。像这样下去,他的心理很快就会正常,不再是傻头傻脑的样子。假如是这样,线条见他不像E.T.,也许就不会喜欢他。不喜欢就不会嫁,这样现在我可能还有机会娶她为妻。然而岁月如流,一切都已发生过了。发生过的事再也没有改变的余地。

\section{十一}

李先生走进会议室,这是一间大房子,里面有好大一个方桌。桌边上坐着两个人。一位是副矿长。另一位是个女孩子,穿件军大衣,敞着衣扣;里面穿着蓝制服,领口露出一截鲜红的毛衣。她的皮肤很白,桃形脸,眼睛水汪汪;嘴巴很小,嘴唇很红,长得很漂亮。这件事不难理解:矿上来了个漂亮女孩子,说是来找人,副矿长出来陪着坐坐,有什么不合理的?但是她来找我干嘛?仔细一看,这姑娘是认识的。在矿院,在干校都见过。但是不知她叫什么名字。那女孩抬头看见李先生,就清脆地叫了一声:舅舅!李先生就犯起晕来:怎么?我是她舅舅?我没有姐妹,甥从何来7副矿长说:你们舅甥见面,我就不打搅了。李先生心想:你也说我是她舅舅?线条(这女孩就是线条。这两人以舅娶甥,真禽兽也!——王二注)说:叔叔再见。等他出了门,李先生就间:我真是你舅舅?线条出手如电,在他臂上狠拧了一把,说:我操你妈!你充什么大辈呀你!我是线条呀!李先生想:外甥女操舅舅的妈,岂不是要冒犯祖母吗?姑妄听之吧。 

然而线条这个名字却不陌生。在干校时,每天收工回来,枕头下面都有一张署名线条的纸条子。这是线条趁大家出工时溜进去塞的,以表示她对李先生的爱慕之心。有的写得很一般:龟头血肿,我爱称!——线条有的写得很正规:亲爱的龟头血肿:你好! 


我爱你。 

此致无产阶级文化大革命的敬礼! 

线条有的写得很缠绵:我亲爱的大龟头:我很想你。你也想我吗?——线条有的写得极简约,几乎不可解:龟,血:爱。条。 

李先生见了这些条子,更觉得自己在做梦。 

对于线条的为人,除了前面的叙述,还有一点补充。此人什么话都敢说,“文化革命”里,除了操,还常说一个字,与逼迫的逼宇同音不同形。当了教授太太后,脏字没有了,也只是不说中文脏字。现在在我院英语教研室工作。有一回给部里办的出国速成班上课,管学生(其实是个挺大的官)叫silly cunt(傻×)。那一回院里给她记了一过,还叫她写检查。她检讨道:我是怕他出国后吃亏,故此先教他记着。该同志出国后,准有人叫他silly cunt,因为他的确是个silly cunt!院长看了这份检查,也没说什么。大概也是想:姑妄听之吧。 

线条说,在干校时她已爱上了李先生。但是没有机会和他接近。后来李先生被分配到了河南,她就尾随而去。当然,这么做并不容易,但正如她自己所说,有志者事竞成。她靠她爸爸的老关系到安阳当了护士,然后打听到龟头血肿的所在地,然后把自己送上门去。这一切她都做了周密的计划,包括管李先生叫舅舅。最后他们俩终于到了一个没人的地方,这是在矿上的小山沟里。这也是计划中的事。她突然对准龟头血肿说:我要和你好!这是计划中关键的一步。说完了她拍起头来,看李先生的脸。这时她发现李先生的表现完全在意料之外:他把眼闭上了。这时她开始忐忑不安:龟头血肿这家伙,他不至于不要我吧? 

李先生说,他琢磨了好半天,觉得此事是个圈套。这十之八九是印度师兄的安排。怎么忽然跳出个漂亮女孩子来,说她要跟我好?他琢磨了好半天,决定还是问问明白。于是他睁开眼睛,说道:什么意思?问得线条很不好意思,很难受。她发了半天的窘才说:什么意思?做你老婆呗。 

不少人听说我会写小说,就找上门来,述说自己的爱情故事。在他们看来,自己的爱情可以写入小说,甚至载入史册。对此我是来者不拒。不过当我把这些故事写入小说时,全是用男性第一人称。一方面驾轻就熟,另一方面我也过过干瘾。但是写李先生的爱情故事我不用第一人称,因为它是我的伤心之事。线条原该是我老婆的,可她成了龟头血肿夫人! 

线条说了“做你老婆呗”,心里忽然一动。说实在的,以前她可没想过要做龟头血肿夫人。她想的不过是要和李先生玩一玩,甚至是要耍耍李先生。可是李先生说你可要慎重时,她就动了火,说:就是要做你老婆!你以为我不敢吗!因此悲剧就发生了。李先生又说:这事可不是开玩笑;线条就说:我真想抽你一嘴巴。李先生就想:姑且由之吧。 

后来李先生说,在我这一方面,当然不会发生问题;别的没有说。线条则凶巴巴地说,我这一方面更不会发生问题。忽然她惊叫起来:不得了,十一点半了。我得去赶汽车。原来从安阳来的就是这一班车,早上开过去,中午十二点开回来。如果误了,等两天才有下一班,她赶紧告诉李先生怎么去找她,还告诉他去时别忘了说,他是她舅舅。说完了这些话,就跑步去赶车。为了跑得快一点,还把大衣脱下来,叫李先生拿着。线条就这么跑掉了。如果不是这件大衣,什么事都不会发生。因为李先生觉得忽然跳出一个大姑娘要做他老婆,恐伯是个白日梦。他对世界上是否存在线条都有怀疑。在这种情况下,他不敢冒险跑到安阳去。假如坐了三个多钟头的长途车到了安阳,结果发现是印度师兄的恶作剧,他就难免要撒瘾症。有了这大衣就有了某种保证,使他敢到安阳去。找到线条固然好,找不到线条也不坏,可以把大衣据为已有。 

李先生说到当日的情形时指出,那个自称要做他老婆的小姑娘,和他说了没几句,就忽然不见了。等他跑出山沟,只见一个人影正以极快的速度向公路绝尘而去,而远处的公路上一辆客车正在开来。过了一秒钟,就起了一阵风沙,什么都看不见(李先生高度近视,带两个瓶子底——王二注);再过一秒钟,风沙散去,连人带车什么都没了。这些事活脱脱像白日见鬼。那时他不知道线条是四八百、一千五的好手,而且她还有骤然开始飞奔的暴走症。关于前一点,不但有她过去历年在中学生运动会上的成绩为证,而且可以从体形上看出来。她的体形不像黄人,也不像白人,甚至不像黑人,只像电视里体育节目中奔在长跑跑道前面的那种人。假如晚生二十年,人家绝不会容她跑到河南去胡闹,而是把她撵到运动场上去,让她拿金牌升国旗——这些事比龟头血肿重要。 

关于后一点,虽然暴走症是我杜撰的,但线条的确因为在我们院里滥用轻功,引起了很大议论。现在她已经是四十岁的女人,正是老来俏的时候,她却不穿高跟鞋。夏天她穿不住运动鞋,就穿软底的凉鞋。头发剪得不能再短,不戴任何首饰(首饰不但影响速度,而且容易跑丢了,造成损失——王二注);在学校的草坪上和人聊天,忽然发现上课的时间已到,于是她把绸上衣的下襟系在腰间,把西装裙反卷上来,露出黑色真丝三角裤,还有又细又长肌肉坚实决不似半老徐娘所有的两条腿,开始狂奔。中国教员见了这副景象,个个脸色苍白。那些西装革履手提皮箱的外籍教员见了,却高叫道:李太大——!fucking!——good!一个个把领带往后一掉,好像要上吊似的,就跟在后面跑出来。 

在这一节里,我们说到了线条对李先生初吐情愫的情形,谈到了她把大衣放在李先生手里,跑步去追汽车。由此又谈到线条有暴走的毛病。夏天她暴走之时,两条玉腿完全出笼。这还不能完全说明问题,最能说明问题的是我俩一块去游泳。在这里要做些说明。她从水池里爬上来——在池沿上用双臂支撑——然后爬上岸。真正说明问题的是支撑那一瞬间。那一瞬间我看见的是由上到下流畅的线条,这些线条从十七岁以来就没有变。如果仔细分辨,可以看出乳房大了一点。但这也是往好里变。线条那两个乳房,原来不够大。考虑到她是属于苗条快速的类型,还是嫌小;现在则无可挑剔了。我不能相信像她这样的女人会一辈子忠于龟头血肿,而且我们俩从十七岁就相爱,居然没做过爱,这事实在不对。所以我就说:假如你想红杏出墙的话,可别忘了我呀。

\section{十二}

线条听了这话,愣了一下才说:假如你的话只是称赞我美,那我很高兴,一定要请你吃一顿。到了四十还能得到这样的赞美,真是过瘾。假如还有别的意思的话,我要抽你一个嘴巴。当然,假如你不在意的话。要是你在意就不抽。二十多年的老友,可别为一个嘴巴翻脸。你到底是哪种意思?我当然不想挨耳光,就说:当然是头一种意思啰。不过我也想知道这是为什么。她说不为什么,只不过是因为早就下了决心,除龟头血肿,一辈子不和别的男人睡觉。 

线条这家伙就是这样,干的事又疯又傻。她自己也知道自己的所作所为是发疯,但是依然要发疯。这是因为地觉得疯一点过瘾。这种借酒撒疯的事别人也描写过,比如老萧(萧伯纳——王二注)就写过这么一出,参见《卖花女》(又名《匹克梅梁》——玉二注);卖花女伊丽沙白去找息金斯教授,求他收她为学生一场。在场人物除上述二人,还有一个老妈子别斯太大,一个辟克林上校。别斯太大心里明白,一个大学教授,收个没文化的卖花女当学生是发疯,而且是借酒撒疯。因为那姑娘虽然很脏,洗干净了准相当水灵。所以她对上校说:先生,您别唆着我们东家借酒撒疯! 

息金斯听了说道:人生是做嘛?!可不就是借酒撒疯嘛。想撒疯还撒不起来哪!借酒撒疯,别斯大太,你可真哏! 


编辑先生会党得这段话里错字待多。其实不然,那息金斯的特长是会讲各路乡谈,一高兴就讲起了天津话。题外的话说得太远了。我说的是线条的事,她一辈子都在借酒撒疯。 

以下的事主要是线条告诉我的。她从煤矿回来,只过了两天,龟头血肿就跟踪而至,送还大衣。那天线条的同宿舍的舍友也在。不但在。而且那女孩还歇班。外面刮着极大的黄风,天地之间好似煮沸了的一锅小米粥一样。这种天气不好打发别人出去。何况已经说了,龟头血肿是她舅舅,来了舅舅就撵人出去,没这个道理。线条只好装成个甜甜的外甥女,给龟头血肿削苹果。然后带他去吃饭,到处对人介绍说:我舅舅!别人说:不像。线条就说:我也不像我妈。别人说:太年轻。线条说:这是我小舅舅。别人又说:你怎么对舅舅一点不尊重?线条说:我小舅在我家长大,小时候一块玩的。到了没人的地方就对李先生瞪眼,说:你刚才臭美什么?你以为我真是你外甥? 

到了下午李先生回矿,线条送他出来时才有机会单独说话,线条叫他下礼拜天黑以后来,那一天同屋的上夜班。来的时候千万别叫人看见。然后她就回去等下星期天。李先生着实犹豫去不去,因为要想在晚上到安阳,只能坐火车,下车九点了。鬼才知道线条留不留他住。没有出差证明,住不上旅馆,在候车室蹲一夜可就糟了。李先生南国所生,最怕挨冻,要他在没生火的房子里待一夜,他宁可在盛暑时分跳一天大粪,而且他对这件事还是将信将疑。但是李先生还是来了。线条说起这件事,就扁扁她那张小嘴:我们龟头对人可好啦。 

线条说,李先生和她好之前,保持了完全的童贞。男人的这种话,他一说你就一听,反正没有处女膜那回事。但是线条对此深信不疑。据李先生自己说,在和线条好之前,只和高一年的一位女同学date了几次,而且始终是规规矩矩的。这件事我在美国调查过,完全属实。我的这位师姑和我的老师不是本科的同学,也不是硕士班的同学。当时是七十年代以前,试想一个美国女孩,假如不是长得设法看,怎么当上了理科的博士生?她又矮又肥,两人并肩坐时,还会放出肥人的屁来,可以结结实实臭死人。李先生说:我也嫌她难看。但我怎么也不忍伤了一个女孩子的心,所以不能拒绝她。 

其实李先生是个情种,他对线条的忠诚是实,我不便加以诋毁。但是别的女人要是作出可怜的样子来勾引他,他就靠不住了。我知道他教的研究生班里,有个女孩子漂亮得出奇,也笨得出奇。考试不及格时哭得如雨打梨花。等到补考时,李先生对我说,你给她辅导一下。然后假装不经意,把题全告诉了我。我自己把它们做了出来,把答案给了那女孩,说:背下来。假如再不及格,你就死吧。她就这样考了六十分。根据这个事实可以推导出,假如有个女人对李先生说,你不和我性交我就死!他一定把持不住。 

李先生成为革命者也是因为他心软,不但见不得女人的眼泪,而且见不得别人的苦难。他老念格瓦拉的一句话:我怎能在别人的苦难面前转过脸去?他就这样上了师姑的钩。后来该师姑又哭着说,你就是个黑人,我也不跟你吹。怎奈黄的和白的配出来,真是大难看!其实黄白混血,只是很小时不好看。大了以后,个顶个的好看,就如皮光缩肚的西瓜,个个黑籽红瓤。师姑的说法以偏概全,强词夺理,李先生居然就信了,白闻了不少臭屁。现在该师站在母校任教,嫁了个血统极杂的拉美人。生了一些孩子,全都奇形怪状。 

现在要谈到线条与李先生幽会的事。为了保持故事的完整,本节的下余部分将完全是第三人称,没有任何插话。 

李先生第二次到线条那里的日子,不但是星期天,而且是12月31日。那天刮起了大风。风把天吹黄了,屋里的灯光蓝荧荧。线条住的房子是一座石板顶的二层洋楼,原来相当体面,现在住得乱七八糟,有七八家人,还有女单身宿舍,所以就把房子改造了一下,除原有的大门外,又开了一个门,直通线条一楼住的房间,那房子相当大,窑洞式的窗子,在大风的冲击下,玻璃乒乓响。和她同屋的人上夜班,黄昏时分走了。 

如前所述,线条住的房子很大,有三米来宽,八九米长。这大概是原来房主打台球的地方。整个安阳大概也只有这么一座够体面的洋房,但是原来的房主早就不在了。后来的房主也不知到了哪里。但是这间房子里堆着他们的东西,箱子柜子穿衣镜等等,占去了三分之二以上的地方,要不偌大的房子不会只住两个姑娘。屋子正中挂了一盏水银灯,就是城市里用来做路灯的那种东西。一般很少安在家里。这种灯太费电,而且太耀眼。但是在这里没有这些问题。因为这里是单身宿舍,烧的是公家的电;这里住了两个未婚姑娘,电工肯给她们安任何灯;丫头片子不怕晃眼,除了这些东西,就是两张铁管单人床。 

傍晚时分线条就活跃起来。她打了两捅水放在角落里,又把床上的干净床单收起来,铺上一张待洗的床单。这是因为上次李先生来,在雪白的床单上一坐。就是一幅水墨荷叶。线条倒不在乎洗被单,主要的是,不能让人看出这房里来过人。故此她不但换了被单,而且换了枕巾。别人的床上也盖了一张脏被头。除此之外,她还换了一件脏上衣。这样布置,堪称万全。做完了这些事,她就坐下等待。天光刚刚完全消失(这间房子朝西,看得很清楚),大概是晚上八点。现在李先生刚下火车,正顶着大风朝这里行进。这段路平常要走四十分,今天要一小时以上。线条站起来,走到窗前往外看。什么也看不见。她把窗帘仔细拉上了。 

线条又回来,坐在床上等李先生。听着窗外的风声,她想到,李先生来一趟太不容易了。下回我到矿上去找他。但是这一回也不能让她安心。于是她在床下待洗的衣服堆里捡丁一件脏衬衣,走到穿衣镜面前,透过上面的积尘,久久地看着自己。她拣了一块布,把镜子擦了擦,就在镜前脱起衣服来。在把那件脏衬衣穿上之前,她看着镜子说了一句话:这么好的身体交给龟头血肿去玩,我是不是发了疯? 

晚上李先生走到线条门前时,他比她预见的要黑得多。这是因为李先生到火车站去,经过了煤场。当时正好有一阵旋风在那里肆虐。走过去以后,李先生的模样就和从井下刚出来时差不多了。然后他又从火车上下来,走了很远的路,几乎被冷风把耳朵割去。虽然人皆有好色之心,但是被冷风一吹,李先生的这种心就没了。他想的只是:我要是不去,那女孩子会伤心。 

李先生当时不但黑,而且因得要死。时近年底,矿上挖出的煤却不多,还不到任务的三分之一。所以矿上组织了会战,把所有的人都撵下井去,一定要在新年到来之前多挖些煤出来。开头是八小时一班,后来变了十二小时一班,然后变成十六小时一班,最后没班没点,都不放上井来,饭在下面吃,因极了就在下面打个盹。如此熬了三十六小时(本来想熬到新年的,那样可以打破会战纪录)之后,因为工人太累,精力不集中,出了事故,死了一个人。矿领导有点泄气,把人都放上来。李先生推了三十小时的矿车,刚上来洗了澡,天就到了下午。他在火车上打了一会盹,完全不够。所以他站在线条门前时,睡眼惺忪。 

晚上李先生到来之前,线条坐在床上想:龟头血肿虽然好玩,这一回可别玩得太过分。虽然她说过,要做龟头血肿的老婆,但是要是能不做当然好啦。这种心理和任何女人逛商店时的心理是一样的:又想少花钱,又想多买东西。更好的比方是说,像那些天生丽质的少女:又想体会恋爱的快乐,又不想结婚。然而她的心理和上述两种女人心理都不完全一样,龟头血肿之于线条,既不是商店里的商品,也不是可供体会快乐的恋人,而是介乎两者之间的东西。 

李先生进了线条的门,迷迷糊糊说了声:你这里真暖和。然后他打了个大呵欠,又说:你好,线条。圣诞快乐,新年快乐,上帝保佑你。他实在是困糊涂了,说话全不经过大脑。假如经过了大脑,就会想到:我们这里是无产阶级革命派的天地。假如有上帝,他老人家也不管这一方的事,正如他老人家管不了舀梅尼。

\section{十三}

晚上李先生到来之后。线条让他洗了脸,又叫他刷牙。李先生带着姑且由之的态度,照做了。此时她看着李先生那张毛扎扎的嘴,心里想:万一他要和我接吻,我就拒绝好啦。不必叫他刷牙。后来听见外面风响,又想到他今天来是多么的不容易。所以他要接吻也不好拒绝的,让他刷刷吧。现在李先生连牙缝里部是煤,被他亲上几下就成了扎染布啦。 

线条的这些想法,都以“够意思”为准则。“文化革命”里我们都以“够意思”为准则,这话就如美国人常说的“be reasonable……”但是意思稍有区别。美国人说的是,要像一位诚实的商人一样,而我们说的是:要像一个好样的土匪。具体到线条这个例子,就是她要像一位好样的女土匪对男土匪那样对待李先生。 

对于线条的够意思,还有如下补充。六八年夏天,正兴换纪念章(纪念章三个字怪得很。当时还没死嘛,何来纪念?——王二注),海淀一带,有几处人群聚集,好像跳蚤市场。线条常到那些地方去。除了换纪念章,那儿也是拍婆子的地方。有人对线条有了拍拖之心,就上前纠缠。线条嫣然一笑,展开手中的折扇。扇面上有极好的两个隶字(我写的——王二注),“有主”!那时是二十二年前,线条是个清丽脱俗的小姑娘,笑起来很好看。 


假如对方继续纠缠,线条就变了脸,娇斥一声:“王二,打丫的!”王二立刻跳出来,揪任对方就打。假如对方有伙伴,王二也有伙伴,那就是许由。许由一出场,就是流血事件。他是海淀有名的凶神。然后我们送打伤的人上医院,如果伤得厉害,以后还要请吃饭。这就是够意思了。 

李先生刷牙时,线条正在想,自己要够意思。但是她也想到了,够意思也要有止境。这个止境是个含混的概念。假如他想动手动脚,一般是不答应。但是也有答应的可能,所以线条做了这种准备。假如李先生想要她的贞节,那就决无可能。他敢在这事上多废话,就打丫的。当时线条决定和男人玩,但要做一辈子处女。她以为这样最为过瘾。 

李先生洗漱完了,他们到床上坐下。原来线条坐着自己的床,李先生坐别人的床,后来她叫李先生过来,坐在她身边。这是因为她看出李先生很疲惫。那被头只能垫住李先生的屁股,万一他往后一倒,就全完了。然后她就研究起李先生来。第一个研究成果是:李先生是招风耳。第二个研究成果是,李先生的毛孔里都是煤。她正要告诉李先生这些事,李先生却说:我想躺下睡一会。说着他就朝一边歪去,还没躺倒就睡着了。线条后来说:“当时我真想宰了他(谋杀亲夫!——王二注)!” 

李先生倒下后,打起呼噜来。线条简直想哭。可是她马上就镇定下来:妈的,你睡吧。老娘先来玩玩你!她给他脱了鞋,把他平放在床上,解开他胸前的衣扣和腰带,把手伸了进去,摸着了一大堆破布片(单身汉的衬衣——王二注)。后来她这样形容自己初次爱抚情人的感觉道:把龟头血肿捆在一根木棍上,就是一个墩布。 

然而龟头血肿不完全是墩布。把手伸得更深,就摸到了李先生的胸膛。那一瞬间线条几乎叫出来。当然,摸久了也稀松平常,但是第一次摸感觉不一样。李先生的胸上有疏琉落落的毛,又粗又硬,顺胸骨往下,奸像摸猪脊梁。这还得是中国猪,外国猪的鬃毛不够硬,不能做刷子。不管李先生的胸毛能不能做刷子,反正线条摸着心花怒放。她一路摸下去,最后摸到了一样东西,好像个大海参。这一下她停下来,想了好半天,终于想到李先生的外号上去。于是她咬着自己的手指说:乖乖。这哪里是器官,分明是杀人的凶器。 

一摸到这个地方,李先生就醒了。刚才他在做梦,梦见在矿上,从矿并里出来去洗澡,澡堂里一锅黑泥汤。好多工人光着屁股跳到泥塘里去,其实他梦的全是真实所见的事,只是他当时不敢相信自己的眼睛,到现在还不敢相信自己的眼睛。怎么能在一个房顶下,看见了那么多男性生殖器。所以他怀疑自己在做梦,而且怀疑自己是同性恋者。只有满足上述两个条件,才会看见这种东西。 

李先生说,他从睡梦中醒来,感到线条在模他,倒吓了一跳。那时他看到线条小脸通红,脸上笑盈盈。他刚从梦中醒来,所以觉得,眼前的事不是梦,而且他也不希望是梦。这是他的似水流年,不是我的。岁月如流,就如月在当空,照着我们每一个人,但是每个人的生活都不一样。 

后来线条叫李先生做了庄严保证:保证不做进一步的非分之想,保证在线条叫他停的时候停下来等等,线条就准许他的手从衣襟底下伸进去。这已经是第二次幽会时的事,和上次隔了一星期。线条说,李先生的手极粗。好像有鳞甲一样,但是透过他的手,还是感到自己的腰很纫,乳房很圆,肚皮很平坦。她对这些深为满意。除此之外,感觉也很舒服(但是有些惊恐),这比在班上聊大天好玩多了。 

与此同时,我在云南偷农场的菠萝。半夜三更一声不响地摸进去,砍下一个,先放到鼻子下同闻香不香。要是香的,就放到身后麻袋里;不香就扔掉。我们俩如出一辙,都不走正路。走正路的人在那年月里,连做梦都想着天下三分之二的受苦人。可是我说:这些受苦人我认得他们是谁吗?再说了,他们受苦,我不受苦?那晚上我一脚跺进了蚂蚁窝,而且我两只脚都得了水田脚气,趾缝里烂得没了皮。那些蚂蚁一齐咬我,像乱箭穿心一样疼。 

我们三人里,李先生感觉最好,可是他却想入非非,觉得眼前的感觉不可靠。人要是长T这个心服,就有点不可救药。当他的手掌从线条乳房上掠过时,感到乳头有点凉冰冰,于是他又动了格物致知的心思:这东西是凉的,对头吗? 

李先生迷迷糊糊,手往下边伸去。线条动作奇快,一下子挣脱出来,还推了李先生一把,说道:你好大胆!李先生说:对不起对不起!我不是这个意思。线条却说:管你什么意思,反正人家(同宿舍的河南小姑娘)快下班,你该走了。

\section{十四}

“文化革命”来到之时,有些人高兴,有些人不高兴。刘老先生对我说过,一开头他就想自杀。因为他见那势头,总觉得躲不过去。但是他想到在峨嵋酒家还能吃到东坡肘子,又觉得死了太亏。他属于不高兴者。线条属于高兴者,因为那一年我们上初三,她各科全不及格。她爸爸说:考不上高中,你给我到南口林场挖坑去。当时就是这么安置考不上高中者。她妈则说:这院里全是书香门第,还没人去挖坑呢。她叫老头到附中讲讲去。老头则说,我是党委书记,怎能干这种事?那年头天下三分之二的受苦人和党性原则部和真的似的。老妈妈实在怕丢人,就找我给线条补功课。实在补不动,差得太远。我王二不但是坏蛋,而且有怜香惜玉之心,所以订下计划,要点如下:一、线条要考的高中,不是外面的学校,只须参加毕业考,及格就能上。 

二、毕业考试上厕所的次数不限。 

三、男厕和女厕之间,我已打了一个小洞。 

虽然有此万全安排,线条仍然吓到要死。到临考前一星期,她告诉我,已经把月经吓了回去。到临考前三天又告诉我,开始掉头发。但是临考前一天,她把我从床上叫起,口唱革命战歌。原来根据革命需要,中学停课不考试了。 

我怎么也想不到线条后来不但考上了大学,而且上了研究生。我们学校要是来了有大学问的洋人做讲座,翻译非她不成。那些老外开头只以为她不过是个漂亮女人罢了,聊起来才发现,不管是集合论,递归论,控制沦,相对论,新三论老三论,线条无不精通。不但精通,而且著作等身(和李先生联名发表)。那些洋人只好摇头说道:我们国家像李太太这样有才的女人也有,但是长得都不像女人。 


现在我们院里的人都说:这有什么奇怪?她是龟头血肿夫人嘛。好像在李先生的精液里,含有无数智力因素,灌溉了线条的智力之花,此说是不对的。有三天前她和小转铃的话为证,地点是在我家的客厅里:线:铃子,你们还有吗? 

钟:什么东西? 

线:什么东西,老公干老婆用的东西嘛。橡皮的condom(套套)!我的妈,得了失语症了!(这是英文好的人才得的毛病,不是谁想得就得得了的。——王二注) 

铃:(不好意思)有是有,全是特号的。 

线:那才好哪。我们龟头那玩意可大了!肯定不比你们王二小。 

铃:他不是“我们”。他对我不好! 

线:那你制制他,买小号的,两次他就老实了。 

由上述对话可知,他们是用避孕套的,智力传染之说可以体矣。我讲这事的目的是要说明,线条原是个性早熟、智力晚熟的家伙,嫁给龟头血肿之前的线条,和以后的线条不一样。 

撵走了李先生,线条还有很多事要干,首先是要把床上的脏床单换下来,然后是刷洗李先生喝水的杯子,藏起李先生用过的牙刷和毛巾,因为上面都有煤。然后从隐秘的地方拿出一块很大的白毛巾。她把所有的衣服全脱光。站到镜子前面去。镜子里站着一位白皙、纤细的少女(有关这个概念,我和线条有过争论。我说她当时已经二十一岁,不算少女,她却说,当时她看起来完全是少女。如果不承认这一点,她毋宁死。我只好这样写了——王二注)。该少女眼睛水汪汪,皮肤洁白,双腿又直又长。腰非常细,保证玛丽莲·梦露看了都要羡幕。在小腹上,有很小一撮阴毛。虽然面积很小,但是很黑很亮。线条对此非常自豪。她说这一点非常重要,假如没有的话,就不好看,太多大乱,也是不好。她后来和李先生出国时,租了很多录像带,在录像机上定格比较,发现很多大名鼎鼎的脱星,在这一点上还不如她远甚。只有一位克瑞斯透,在十九岁拍的片子里,曾有过如此美丽的腹部(我没看见,不能为她作证——王二注)。 

线条还说,在这个美丽的躯体上,有极美的装饰,就是一道道黑色。这位美丽的少女,有绝美的黑色嘴唇。乳房上有黑色的斑纹,小腹上有几条细的条。初看似信手拈来,细看才发现那种惊人的美,要问此美从何而来?这是龟头血肿涂上的煤黑。线条用毛巾蘸了凉水,把黑印一一试去。然后她洗了脸,漱了口,刷了牙,穿上衣服,出了门,要把脏水倒掉。这个走道黑糊糊的,线条又不像王二那么胆大。所以当她听见呼呼的声音时,着实吓得够呛。 

线条说,那个走廊里没有灯,可是也没什么地方可以藏人。听见这声音可把她吓坏了。于是她放下了水桶,悄悄溜了回去,拿了一个大电简出来。这东西不但可以照亮,还可用来打架,她拿这个东西循声而去。结果找到一段楼梯下,有一块小得不得了的空间。在那块空间里,李先生正以娘胎里的姿式睡觉呢。他那件劳保大衣放在外面,没带进去,这是因为里边塞不下了。线条一看,登时勃然大怒,想道:龟头血肿:不是叫他找大车店睡觉去吗?她想立时把李先生叫起来暴打一顿,然后叫他滚蛋,再也别来。假如这样做了,不但太快人心,而且我现在还有机会。 

但是线条没有这么做。她做了另外的决定,所以现在她的户口本上户主一栏上写着李先生的名字,线条那一栏里写着,李某某之妻。这十足肉麻,做了这个决定之后,她就完全堕落了。 

在似水流年里线条做了这样的决定,要作龟头血肿之妻,永不反悔。对此我完全不能理解。但是,只要李先生不死,这事不会改变。虽然岁月如流,什么都会过去,但总有些东西发生了就不能抹煞。

\section{十五}

李先生听见线条说:你对我干什么都行,他就想起我那位胖师姑来,师姑过去老和他说这话,他只是不借。到吹了以后,师妨告诉他,那话的意思就是:make love to me!后来他想,幸亏没听懂。听懂了还能不答应?答应了还能不兑现?每回一想到兑现,就会眼前发黑,要晕死过去。 

因为有过上述经历,那天李先生听了这话,马上就反应过来了。他直言不讳地说:咱们做爱吧。线条一听,小脸挣得通红,厉声说你倒真不傻!然后想了想,又说:那就做吧。 

李先生和线条后来约定了在煤矿附近山上的庙里做爱。时间就定在春天停暖气的那一天。 

李先生决定相信线条,把自己理智的命运押在她身上。七三年的三月十五日中午十二时,他就到那破庙里去。为了验证一切,他非常仔细地记下了所有的细节。他受的是英式教育,故此像英国人那样一丝不苟,像英国人一样长于分析,像英国人一样难交往,交上以后像英国人一样,是生死朋友。 

李先生说:那个破庙在山顶上,只有十平米的正殿。围墙里的草有齐腰深,房顶上的草像瀑布一样泻下来。庙里的门柜,窗框,供桌等等一切可搬可卸的木头,都被人搬走了。正殿里有一小堆碎砖瓦,还有一个砖砌的供台,神像早没了。他想过,这会是个什么庙,照道理,山顶上的应该是玉皇庙,这是因为山离天较近,虽然是近乎其微的一点。作为中国人,他在海外读过有关民间风俗的书。但是在这座庙里,得不到一点迹象来验证这是玉皇庙的说法。而且也得不到一点验证它不是玉皇庙的说法。在这里,什么验证都得不到。因为没有神像,没有字迹,什么都没有。正因为如此,李先生对这庙的存在才坚信不移。 


李先生还说:那个庙里的墙该是白的,但是当时很多地方是黑的。房顶露洞的地方,下面就是一片黑。这是因为年复一年漏进来的雨水,把墙上的雨水都冲走了。墙皮剥落的地方也是一片黑。墙上有的地方长起了育苔,有的地方发了霉。地上是很厚的泥。泥从房顶上塌下来,堆在地上。在房顶露洞的地方,椽子砒牙咧嘴地露出来。那些椽子朽烂得像腐尸的肢体一样,要不也会被人拆光。地上的泥里还混有石子,石子的周围,长着小草,小草也是黑色的。院子里长着去年的蒿子,它们是黄色的。房上泻下的草也是黄色的。风从门口吹进来,从房顶的窟窿吹出去,所有的草都在括,映在房子里的光也在摇。但是线条没有来。李先生爬到香台上往外看,透过原来是窗子的洞,穿过路上的窟窿,可以看到很多地方,但是看不见线条。他又退回院子里,从门口往外看,只看见光秃秃的石山和疏疏落落的枯草,还是见不到线条。但是线条一定在这里,李先生刚决定要找一找,线条就像奇迹一样出现了。她从庙后走出来,把大衣拿在手里,小脸上毫无血色,身上甚至有点发抖,怯生生地说:龟头,你不会整死我吧。 

线条则说:当时确实害怕了。虽然从来不知什么叫害怕,以后也不知什么叫害怕。当时害伯的滋味现在也说不出来,只觉得心里很慌,这感觉有点像六七年我带她爬实验楼,从五楼的一个窗口爬出来,脚踏半尺宽的水泥棱,爬到另一个窗口去。但是爬窗口比这回的感觉好多了。 

李先生说:线条把大衣铺在平台上,自己坐上去,说道:你什么话也别说,也别动我,一切让我自己来。好吗?说完了这些话,就坐在那里,半天没有动。 

线条说:李先生果然什么都没说。 

李先生说:后来线条抬起头来,想朝他做个鬼脸,但是鬼脸僵死在脸上了,好像要哭的样子。她哆嗦着解开制服的扣子,然后把红毛衣从头顶上拽下去。那一刻弄乱了头发,就用手指抚了好半天。她穿了一件格子布衬衣,肩头开了线。然后她就像吃橄揽一样,一个一个地把扣子解开。那时的时间好像会随时停止一样。然后她又把乳罩解下来。那东西是细白布做的,边上缀着花边。然后她把裤子(包括罩裤、毛裤和线裤)一下都脱下来,钻到大衣里,坐在供台上发呆。 

线条说:那一回好像我把自己宰了。 

线条说:李先生露出那秆大枪来,真是吓死人。 

线条还说:最可怕的是第一次,只觉得小肚子上一热把下身弄得很脏。后来知道,所谓的做爱,原来还没有完。然后只好像要生孩子一样,拼命用手把腿分开。经过了这些事以后,就再也不想爱别人。

\section{十六}

在似水流年里,有件事叫我日夜不安。在此之前首先要解释一下什么叫似水流年。普鲁斯特写了一本书,谈到自己身上发生过的来。这些事看起来就如一个人中了邪躺在河底,眼看潺潺流水,粼粼流光,落叶,浮木,空玻璃瓶,一样一样从身上流过去。这个书名怎么译,翻译家大费周章。最近的译法是追忆似水年华。听上去普鲁斯特写书时已经死了多时,又诈了尸。而且这也不好念。 

照我看普鲁斯持的书,译作似水流年就对了。这是个好名字。现在这名字没主,我先要了,将来普鲁斯特来要,我再还给他,我尊敬死掉的老前辈。 

似水流年是一个人所有的一切,只有这个东西,才真正归你所有。其余的一切,都是片刻的欢娱和不幸,转眼间就已跑到那似水流年里去了。我所认识的人,都不珍视自己的似水流年。他们甚至不知道,自己还有这么一件东西,所以一个个像丢了魂一样。 

现在该谈谈刘老先生的事。要说这事,还有很多背景要谈,首先要谈刘老先生的模样。当时,他还没死,住在我家隔壁。那时他一头白发,红扑扑的脸,满脸傻笑。手持一根藤拐棍,奔走如飞,但是脚下没根,脚腕子是软的,所以有点连滚带爬的意思,如果不在我家吃饭,就上熟人家打秋风,吃到了好菜回来还要吹。他还是—个废话篓子,说起来没完,晚上总要和我爸爸下棋到十二点。照我看是臭棋,要不一晚怎能摆二十盘。 


刘老先生内急时,就向厕所狂奔,一边跑一边疯狂地解裤腰带有一次,一位中年妇女刚从女厕出来,误以为刘老先生是奔她去的就尖叫了一声,晕了过去。 

其次要谈谈地点——矿院。当然,它也可能不是矿院。那时矿院迁到了四川山沟里接着办(毛主席说了,大学还要办),可是矿院的人说,那山沟里有克山病,得了以后心室肥大。主事的军宣队说,你们有思想病,所以心室肥大;我没有思想病,所以不肥大。刚说完这话,他也肥大了。于是大家拔腿跑回了北京,原来的校舍被人占了,大家挤在后面平房里,热热闹闹。我爸我妈也跑回来,我正在京郊插队,也跑了回来,带着小转铃。一家人聚在一起,共享天伦之乐。 

谁知乐极生悲,上面派来了一批不肥大的军宣队。通知留守处,所有回京人员,必须回四川上班,不回者停发工资。只有肥大到三期或者老迈无能者例外。后来又来了一条规定,三期和老迈者只发将够糊口的工资,省得你们借钱给投病的人。出这主意的那位首长,后来生了个孩子没屁眼,是我妈动手术给孩子做了个人工肛门。这个故事告诉我们,随着医学的发展,干点缺德事不要紧,生孩子没屁眼可以做人工肛门,怕什么? 

然后就该谈时间,那是在不肥大的军宣队来了之后,矿院的人逐渐回到四川去。我爹我娘也回去了。我爸我妈走后两天,刘老先生就死了。在他死之前,矿院后面的小平房里只剩下三个人,其中包括我,小转铃,刘老先生。这对我没什么不好,因为我爸爸妈妈在时不自由,他们不准我和小转铃睡一个床。

\section{十七}

我始终记着矿院那片平房。那儿原不是住人的地方。一片大楼遮在前面,平房里终日不见阳光。盖那片平房时就没想让里面有阳光,因为它原来是放化学药品的库房。那里没有水,水要到老远的地方去打;也没有电,电也是从很远的地方接来;也没有厕所,拉屎撒尿要去很远的地方,这个地方就是远处的一个公共厕所。曾经有一个时候,矿院的几百号人,就靠一个厕所生活。就因为这个原因,这个厕所非常之脏,完全由屎和尿组成,没有人打扫,因为打扫不过来。 

库房里的情况也很坏。这房子隔成了很多间,所有房间的门全朝里,换言之,有一条走廊通向每一个房间。这房子完全不通风。夏天使在里面的人全都顾不上体面。所以,我整天都看见下垂的乳房和大肚皮,走了形的大腿,肿泡眼。当然,库房里也有人身上长得好看点的东西,可是都藏着不让人看见。 

除此之外,还有走廊里晾的东西全是女人的小衣服。这种东西不好晾到外面,只好晾在走廊里自家附近,好像要开展览会。我倒乐意看见年轻姑娘的乳罩裤权,怎奈不是这种东西。走廊里有床单布的大筒子,还有几条带子连起来的面口袋。假如要猜那是什么东西,十足令人恶心,可又禁不住要猜。最难看的是一种毡鞋垫式的东西,上面还有屎嘎巴似的痕迹。所以我认为一次性的月经棉是很伟大的发明,有时它可以救男人的命。中年妇女在中国是一种自然灾害,这倒不是因为她们不好看(我去过外国,中国的中年妇女比外国中年妇女长得好看——王二注),而是因为她们故意要恶心人! 


我听说有人做了个研究,发现大杂院里的孩子学习成绩差,容易学坏,都是因为看见了这些东西,对生活失去了信心。我没有因此学坏,这是因为我已经很坏,我只是因此不太想活了。 

在我看来,与其在这种环境里活着,还不如光荣地死去。像贺先生那样跳楼,造成万众瞩目的场面,或者在大家围观中从容就义。每天晚上睡觉之前我都给自己安排一种死法,每种死法都充满了诗意。想到这些死法,我的小和尚就直挺挺。 

临刑前的示众场面,血迹斑斑酷烈无比的执行,白马银车的送葬行列,都能引起我的性冲动。在酷刑中勃起,在屠刀下性交,在临终时咒骂和射精,就是我从小盼望的事。这可能是因为小时候,这样的电影看多了(电影里没有性,只有意识形态,性是自己长出来的——王二注)。我爸爸早就发现我有种寻死倾向,他对我很有意见。照他的说法就是:你自己要寻死我不管,可不要连累全家。照我看,这是十足恶心的说法。要是他怕连累,就来谋杀我好啦。 

我爸我妈对小转铃没有意见。首先,她是书香门第的女孩子(我爸有门第观念)。其次,她长得很好看。最后,她嘴甜,爸爸妈妈叫个不停。弄得我妈老说:我们真不争气,没生出个好点的孩子给你作女婿(这是挑拨离间——王二注)。小转铃就说:爸爸妈妈,够好的啦。这话像儿媳对婆婆说的吗?可是你见过婆婆非要和媳妇睡一个房间的吗?我爸和我睡在一起,他打呼噜。我提出过这样的意见:你们两位都不老,人说二十如狼四十如虎,五十赛过金钱豹。现在妈是虎,爸爸是金钱豹,你们俩不敦伦,光盯着我们怎么成。最好换换,你们睡一间,我们睡一间。我妈听了笑,我爸要揍我。不管怎么说,他们只管盯死了我们,不让我们干婚前性交的坏事。直到他们回四川,还把我们交给刘老先生看管。

\section{十八}

刘老先生我早就认识,早到他和贺先生关在一个屋里时,我就见过他。那时我和线条谈恋爱,专拣没人的地方钻,一钻钻上了实验楼的天花板,在顶棚和天花板的空里看见他在下面,和贺先生面对面坐着。贺先生黑着脸坐着,而刘老先生一脸痴笑,侧着脸。口水从另一边谈落下去,他也浑然不知,有时举起手来,用男童声清脆地说:报告!我要上厕所!人家要打他,他就脱下裤子,露出雪白的屁股。爬上桌子,高高地撅起来。刘老先生就是这么个人,似乎不值得认真对待。我爸爸和刘老先生攀交情,我很怀疑是为了借钱。 

我爸爸走时已是冬天,别人都回四川去了。他们不仅是因为没有钱,还因为留守处的同志天天来动员。但是谁也不敢到我家里来动员,因为他们都怕我。这班家伙都和我有私仇,我既然还活着,他们就得小心点。我爸爸能坚持到最后,都是因为我的关系。但是我们也有山穷水尽的时候,不但把一切都吃光当净,还卖掉了手表和大衣,甚至卖光了报纸。能借钱的全搬走了,不能撤走的全没有钱。库房里空空荡荡,到了好住的时候,可是我们二老没福消受了! 

我爸爸虽然一直看不起我,但是那时多少有点舐犊之情;到了那般年纪,眼看又没什么机会搞事业了(后来他觉得可以搞事业,就重新看不起我甚至嫉妒我——王二注),看见眼前有个一米九的儿子,一个漂亮儿媳——一双壁人,有点告不得离开,这可以理解。但我心里有点犯嘀咕:你们这么吃光当净,连刘老头的钱也借得净光净,走了以后叫我们怎么过嘛。当然,这话我也没说出来。 


我爸爸临走时,要我管刘老先生叫刘爷爷。操他妈,我可折了辈了。他还朝刘老头作揖说:刘老,我儿子交给你,请多多管教。这畜生不学好不要紧,不要把小转铃带坏,人家可是好女孩。刘老先生满口答应。我爸还对小转铃说:铃子,把刘爷爷照顾好。小转铃也满口答应(我爸爸向刘老先生借过不少钱,有拿我们俩抵债的意思)。临了对我说:小子,注意一点,可别再进(监狱——王二注)去。说完这些话他们就走了。矿院派了一辆大卡车,把他们拉到火车站,不让人去送。我的二者一走我就对刘老先生说:老头,你真要管我?老先生说:哪能呢,咱们骗他们的。王二呀,咱们下盘棋,听贺先生说,你下一手好棋! 

刘老先生要和我摆棋,我心里好不腻歪。你替我想想看:我和小转铃有好几个月没亲热了。好不容易我爸走了,我妈也定了,你再走出去,我一插门,就是我的天下。虽然大白天里她不会答应干脱裤子的事,起码摸一把是可以的吧。可恨刘老头没这眼力价,我也不好明说,恨死我啦。 

我恨刘老先生,不光是因为他延误了我的好事,而且因为他是贪生怕死之辈。他经常找我量血压,一面看着水银柱上下,一面问:高压多少? 

没多少,一百八。 

可怕可怕。铃子,给我拿药。高压一百八!低压多少? 

没多少,一百六。 

低压高!不行我得去睡觉。醒了以后再量。 

拿到一纸动脉硬化的诊断,就如接到死刑通知书一样。听说吃酸的软化血管,就像孕妇忌口一样。买杏都挑青的。吃酸把胃吃坏了,要不嘴不会臭很像粪缸一样。其实死是那么可怕吗?古今中外的名著中,对死都有达观的论述:吕布匹夫!死则死矣,何惧也?——三国演义,张辽。 

死是什么?不就是去和拿破仑、凯撤等大人物共聚一堂吗?——大伟人江奈江·魏尔德。 

弟兄们,我认为我死得很痛快。砍死了七个,用长矛刺穿了九个。马蹄踩死了很多人,我也记不清用枪弹打死了多少人。——果戈里,塔拉斯·布尔巴。 

(以上引自果氏在该书中描写哥萨克与波兰人交战一场。所有的哥萨克临死都有此壮语,所以波兰人之壮语当为:我被七个人砍死,被九个人刺穿,也不知多少人用枪弹打死了我,否则波兰人不敷分配也!王二注) 

怕死?怕死就不革命!怕死?怕死还叫什么共产党员!——样板戏,英雄人物。 

死啦死啦的有!——样板戏,反面人物。 

像这类的话过去我抄了两大本。还有好多人在死之前喊出了时代的最强音。“文革”中形式主义流行,只重最后一声,活着喊万岁的太一般,都不算。我在云南住医院,邻床是一个肺癌。他老婆早就关照上啦:他爹,要觉得不行,就喊一声,对我对孩子都好哇。结果那人像抽了疯,整夜不停地减:毛主席万岁!闹得大家都没法睡。直到把院长喊来了,当面说:你已经死了,刚才那一声就算!他才咽了气。想想这些人对死亡的态度,刘老先生真是怕死鬼! 

我和刘老先生摆起棋来,说实在的,我看他不起,走了个后手大列手炮局。看来刘老先生打过谱,认得,说一声,呀!你跟我走这样的棋!我轻声说:走走看,你赢了再说不迟。听我这么说,他就慌了。大列手炮就得动硬的,软一点都不成。他一怯,登时稀里哗啦,二十合就被杀死了。他赞一声,好厉害!再摆,摆出来又是大列手,一下午五个大列手,把刘老先生的脑门子都杀紫了! 

刘老先生吃了很多大列手炮局。打过谱的都知道,这是杀屎棋的着法。到晚上他又来和我下,真可恨。我早想睡啦,但也不好明说。我当然走列手炮!他一看我又走列手炮,就说:王二,你还会不会别的?我说:什么别的?他说:比方说,屏风马。我说:好说,什么都会。不过你先赢我这列手炮再说。他说:你老走这个棋不好。我说:怪,你还管我走什么棋?刘老先生委委屈屈地走下去,不到十五回合又输了。老头长叹一声道:看来我得拜你为师了。我说:我哪敢教您老人家。刘老先生气跑了。 

时隔二十年后,我也到了不惑之年。对刘老先生的棋力我有这样的看法:他的棋并不坏。和我爸下,一晚能下二十盘,那是因为我爸的棋太臭。而和我下时,假如我告诉他:他输棋是因为走了怯着,他可以多支持些时候。我当时能知道这些道理,但是我一心要和小转铃做爱,所以想快点打发他走。假如我能知道他第二天就要死了,真该把做爱的事缓缓,在棋盘上给他点机会。 

刘老先生经常拄着拐棍坐在椅子上打瞌睡,口水流在前襟上。

\section{十九}

我所认识的人里,就数刘老先生馋。当时他和我们搭伙,我们俩也很馋。像这种问题很容易解决(可以多买些肉来煮),但是我们没有钱,刘老先生也只领四十块钱生活费,除了吃还有其它花费,所以这问题也就不好解决了。如前所述,我爸爸他们没走时,就把一切吃光当净,连废报纸都卖了,所以我们除了白菜,也就是一点广东香肠。小转铃想,王二一米九的个子,在性生活里又会有些支出,和我吃的一样多恐怕不够。所以她尽量少吃。但是头天晚上,刘老先生到了餐桌上状加疯魔,运筷如飞,把香肠全夹走了。虽然我从小没受礼教的影响,但是和老头抢东西吃的事还干不出来,所以我只好瘪着半截肚子和小转铃做爱,对刘老先生深为不满。 

我现在知道了,刘老先生当时已到了非肉不饱之年,而且他前半生都在吃牛排。清水煮白菜吃下去完全不消化,机米饭吃下去也毫无用处,这样的饭荣是对他肠胃的欺骗。在他生命的最后时刻,他无时无刻不在饥饿中。从另一方面看,刘老先生打了一辈子光棍,也末听说他有任何风流韵事。到了那个年头,他也不搞什么学问了,一切一切都在嘴上。但当时我对此尚不能体会。我觉得糟老头贪吃简直该死。 

现在我还知道刘老先生晚饭吃了一顿熬白菜,到口不到肚,后半夜生生饿醒了。他在家里翻箱倒柜,只找到一块榨菜,就坐在那里以榨菜磨牙,直到天明。天一亮他就奔到菜市场买菜:我们的菜金全在他手里,他买菜我们做,就是这么分工。 


那晚上刘老先生走了后,我隔着场叫小转铃过来,她不肯。我就说:我生气了,我不理你了,我不跟你好了。说到最后一句,她过来了。我和她亲热了一番,她就要走。我让她别走。她说:你妈再三嘱咐,叫我别跟你睡。我都答应了。我知道小转铃答应人的事死也要坚持的,但是还是不死心。劝说了一番,她居然同意不走,和我做爱。那时我好不得意:连小转铃都为我破了诺言,可见我的魅力!心里一美,小和尚挺得像铁一样,可是过一会就不美了。小转铃坚持要给我套避孕套,还说:这是你妈嘱咐的!原来我妈让小转铃答应了不和我睡还不放心,她说:少男少女的事我还不知道吗?现在答应,未必能坚持住。记住,一定要套套子,别的措施全靠不住!王二粗心,这事你来做。你可一定要答应我!小转铃最后答应的是给我套套子,不是不和我睡。她要是答应了不和我睡,那晚上只好手淫了。 

这件事使我对我的爹娘怀恨在心。什么都管,管到了套套子!我最恨我爸爸,因为肯定是他的主意。我也恨小转铃,因为她不听我的,听我妈的。所以我最后没跟她结婚。 

我现在明白了我爸我妈为什么对我的性生活这么操心。当时我是二十三岁,小转铃还未成年。万一走了火,她怀了孕要做人流,还得开介绍信。别的地方开不出来,只有我们公社能开。你替我想想吧,假如发生这样的事,我会怎样。我爸爸妈妈死命看住我,心还不够狠,心狠就该把我阉掉。我现在明白小转铃最爱我,想和她结婚,她却不干了。 

那晚上的事我还有些补充,干之前,我编了个小故事,说到我将拉砍头。窗外正给我搭断头台,刽子手在门外磨刀,我脖子上已被面上了红线,脑后的头发已经剃光了。人们把小转铃叫来,给她一个框,让她在里面垫上干草:“别把脸磕坏了,这可是你的未婚夫!”准备接我的脑袋。而她终于说动了狱卒,让我们在临刑前半小时呆在一起。小转铃哭起来:那你就快点干吧,套子套好了。每听到一种新死法,她就哭起来。当我用到第二个避孕套时(说我将被绞死——王二注),就听见隔壁刘老先生闹,一直闹到第四个避孕套(那回是我被开膛挖心——王二注)。第六个避孕套时他出去了,当时已经天明。那夜一共就是六个,因为刘老先生骚扰,所以那一夜不是很开心。 

第二天早上他从外面跑回来敲我的门时,我们俩还没起床。当时我正以极大的兴趣抚摩小转铃的乳房。而小转铃的乳房乃是我一生所见乳房里最好的一对:形状是最完备的半球形,皮肤最洁白,乳头又小又好看。假如世界上有乳房大赛,她绝对有参赛的资格,小转铃对性生活的其它方面毫无兴趣,只对此事有兴趣。通过胸前的爱抚达到高潮,是她享受性乐趣的惟一途径。这种事情不容易搞成,可遇不可求的,那天她兴趣极大(戒欲两个月,贞女如小转铃都会有变化),头枕双臂,双眼紧闭,脸色潮红,马上就要来了。就在这时刘老先生来砸门,乓乓乓,所以去开门时我说了:这老鸡巴头子真该死啦。 

打开门以后的第一观感是:这老头像喝了子母河的水,怀孕了。他的肚子上圆下尖,秃顶周围的白毛全竖了起来,脸上露出了蒙娜丽莎似的微笑。然后他就像分娩一样艰难地从肚皮下拉出一只填鸭来。看到他这样做作,我也不禁惊喜道:这是你愉的吗?他听了大惊道:偷?怎么能偷?偷东西是要判刑的嘛,是买的。我也顾不上向他解释知青的理论“偷吃的不是偷”。也顾不上问他为什么要把鸭子藏在衣服底下,这些都顾不上问。我只问他花了多少钱。他说很便宜,五块钱。我说混帐,像你这么花,下半月只好吃屎啦。他听了这话,也觉得不好意思。这时小转铃跑出来说:王二,怎能对刘爷爷这样,快道歉。其实我也不是在乎这五块钱,我只嫌刘老头没出息。你猜他为什么把鸭子藏在怀里?是怕留守处那几个把大门的说他贪嘴。他是回城治病的,怕人家说他没病,一天吃一只大肥鸭。说到底,是“文化革命”里挨了几下打,把胆子打破了。 

如果说到挨打,刘老先生简直不能和我相提并论,虽然当时我是那样年轻,而他已经老了。他一生所挨的打,也就是实验楼里那儿下,数都能数出来。而我挨的打,绝不可能数清楚。我校专政时,风师傅把我叫到地下室,屋顶亮着灯,四周站了很多人。他说道:你看好了,我们不打你。工宣队部进校了,我们不打人。然后灯就黑了。等灯再亮时,我从地下爬起来,满头部是血。凤师傅笑着说:我们没打你,对吧。你能说出淮打你了吗?当然我说不出。我说的是:操你妈!然后灯又黑了,在黑暗里挨打,数都没法数。打我的就是留守处那班家伙,和打刘老先生的相同。可是我一点也不怕他们,连姓风的都管我叫爷爷,我还伯准? 

现在到了不惑之年,我明白了,我挨的打,的确不能和刘老先生相提并论。因为我是那样的人,所以挨的揍里面,有很大自找的成分。刘老先生挨的打,没有一点自找的成分。我还年轻,还有机会讨回帐来,可是刘老先生已经到了垂暮之年,再不能翻本,每一下都是白挨。因此刘老先生当然怕得厉害。 

刘老先生给自行车打气,对不准气嘴,打不进气,就气急败坏,把自行车推倒。

\section{二十}

早上刘老先生对我说:昨晚上一宿没睡,就想两件事。一是要吃一只鸭,二是要向王二学棋,搞清楚为什么他的大列手炮我就是下不过。我告诉他说:这路列手炮,乃是一路新变化。公元一九六六年,天下著名的中国象棋名手,包括广东杨官麟,上海何顺安,湖北柳大华,黑龙江王嘉良等等十五人,齐集杭州城。大家说:上海胡荣华太厉害,一连得了好几届冠军,可恶!咱们得算计他一回。都说大列手是臭棋,就从这里编出变化来,让他一辈子也想不到,要他的命!于是想了七七四十九天,编出十五着来,邪门得厉害!刘老先生听得眉飞色舞,嘴里啧啧咂出声来。小转铃就笑,说,您别听王二臭编。刘老先生说:铃子,你不懂棋,别打岔!有这么回事!接着说,后来怎么了?我说:当时大伙约定,一人记一路变化,这接着说,后来怎么了?我说:当时大伙约定,一人记一路变化,这路变化只有对胡荣华才能用,自己人之间不能用——铃子,你去收拾鸭子,你听不懂——但是后来谁也没用。胡荣华还是冠军!刘老,你懂棋,猜猜为什么? 

刘老先生想了半天,才迟迟疑疑地说:刚才你说,何顺安? 

我说:着哇!到底是老前辈!那厮是胡荣华同乡,专做奸细(要不是刘老先生一提,我还编不下去了呢——王二注)!比赛头一天,参加杭州棋会的每个棋手,都收到一封信,就写了一句话:车八平五。下署:知名不具!刘老,再猜猜,怎么回事?他拍案叫道:好个胡荣华!真真厉害!何顺安只会一变,其它十四种变化肯定记不全。老胡见不能取胜,就把大列手第一步写下,给人家寄去,人家一看,你知道我们要走列手炮,就不敢走了。这是死诸葛惊走活仲达之计!你一定会这十五路变化,难怪下不过你。这大列手好大的来历,教给我罢。我说,教也可,一路一块钱。他说,便宜! 


人老了就像小孩一样,此话不虚,刘老先生搬来棋盘,裁好了纸,好了铅笔准备记谱,圆睁怪眼,上下打量我。我心里痒痒,真想在他头上打一下。才走了一步,刘老先生就高声唱道:车八平——五!举手就记谱。把我笑得打跌,连棋盘都打翻了。 

后来我告诉他,没有这路变化,是我编着骗他的,他很不高兴。转眼之间又高兴了,因为想起了鸭子。人老了就这么天真,事事都在别人意料中。刘老先生对着那可怜的鸭子的尸体,出了很多主意要把它分成几部分。一部分香酥,一部分清蒸,一部分煮汤,一部分干炸,那鸭子假如死而有灵,定然要问刘老先生这是为什么。假如我死了,有人拿我的四分之一火葬,四分之一土葬,四分之一天葬,四分之一做木乃伊,我也有此疑问。但是我们的厨房里只有酱油膏,所以只能红烧。刘老先生说,红烧鸭要烧到稀烂才好吃,要烧到天黑。刘老先生把莱金花了个精光,只买了一只鸭。所以中午只好挨饿了。刘老先生说,好饭不怕晚,但是他老去揭那炖鸭子的锅,说是看了也解馋;他那副馋相叫人不敢看。炖鸭的香味飘到屋里,刘老先生坐不住,走来走去,状如疯魔。到晚上还有一白天,他血压又高,肯定挨不过。所以小转铃把我叫出去,给了我一点钱,叫我带他去吃午饭。她还说,她不饿。于是我对刘老先生说:老头,陪我去逛逛。我骑一辆男车,他骑一辆女车,出了矿院的门。然后我对刘老先生说:我还有一点钱,够咱俩去新街口吃一顿羊肉泡馍。只听垮的一声,刘老先生连人带车倒在地上。我连忙停车回头,只见刘老先生从地上爬起来,口角流涎,说道:羊~~肉泡馍!! 

我请刘老先生吃了泡馍。因为早上我骂了他,有点内疚。后来他就死掉了。他到底没吃到那只鸭。当天晚上我吃那只鸭,第一口就吐了,小转铃也吃不下,最后倒掉了。鸭子的肉又钻又滑,吃时的感觉实在可怕,我到现在也不爱吃鸭子。 

和刘老先生吃泡馍时,我和他谈起了贺先生。老头的脸色登时大变,说道:吃饭,吃饭,别谈这些事,怪害怕的。我说:谈谈何妨,老头,你怕什么。他说:别提死人。我说,真笑话,你这么一大把年纪,还怕死吗?老头很天真地说:谁不怕。我说:怕就能不死吗?老头,你看看你吃的东西,乃是羊杂碎。全是胆固醇。吃下去动脉硬化,离死就不远了。那老头的样子真好看,手都抖起来了。 

后来刘老先生大起胆子(他说,回家喝点醋,能解——王二注),告诉我贺先生死之前的事,都不大有趣。贺先生跳楼前只说,告诉我家里人,别太伤心了。没有说过像二十年后又是一条好汉之类的话,甚至也没说:让我儿子给我报仇。那时我想,像刘老先生这种没劲的人,说出的事都没劲。 

吃完饭,我叫刘老先生回家,自己在外面遛到天黑方回。我活得很没劲,好像一个没用的人。人到了这步田地,反而会满脑子伟大的想法。那时我想:假如发生了战争就好了。 

活得没劲的人希望发生战争,那是很自然的想法。我们那一代人,都是在对战争的期待中长大的。以我为例,虽然一不怕疼,二不怕死,但是在和平年月里只能挖挖坑,而中国并不够少挖坑的人。 

在和平年月里,生活只是挖坑种粮的竞争。虽然生得人高马大,我却比不过别人。这是因为第一,我不是从小于惯了这种活计,第二,我有腰疼病,干农活没有腰不成。所以我盼望另一种竞争。在战场上,我的英勇会超过一切人。假如做了俘虏,我会偷偷捡块玻璃,把肚子划破,掏出肠子挂到敌人脖子上去。像我这样的兵员一定大为有用。但是不发生战争,我就像刘老先生一样没用。 

到现在我明白了,掏出肠子挂到别人脖子上,那是很糟糕的想法。自己活得不痛快,就想和别人打仗。假如大家都这么想,谁也别想过好日子了。而且我也明白,刘老先生伯死,那是再自然也没有的事,他在世上什么都没有了,只有最后的日子。 

刘老先生在厕所里撤尿,经常尿到自己裤子上。

\section{二十一}

刘老先生死了以后我常想,我老了以后,可能和刘老先生一样。 

刘老先生活着时,我老在背后说,没骨气的人就是活得长。贺先生和刘老先生比,一个在天上,一个在地下,贺先生大义凛然,从楼上跳下去,刘老先生挨了两下打就把胆子吓破了。但他死时我还是着了急。我从外面回来时,小转铃对我说:去看看刘老先生怎么了,躺在那里打呼噜,叫也不答应。我到他房里一看,他流了很多哈喇子,翻开眼皮一看,眼珠子不动。我转过身来就打小转铃一凿栗:你是死人吗?快找车,送老头上医院! 

据小转铃说,刘老先生回来时,骑车骑得飞快,头上见了汗,回来就看鸭子,看到鸭子已经烂了,摩拳擦掌,口水直流。后来说,感到不舒服,要回去睡,告诉王二,回来给我量血压。王二回来,不量血压,先打小转铃一凿栗:老头都这样了,还等我回来吗? 

小转铃也不是省油的灯。我蹬干板三轮送刘老先生上医院,她坐在后面胡搅蛮缠:好哇,你敢打我!我非打回来不可。我说:刘老先生中风了。以后好了,也是歪嘴耷拉眼,你看看他嘴歪了没有。我这么说是要分散她的注意力。到了医院里,把刘老先生推进急诊室。过了一会儿就遮着白布推出来。有个大夫对我说:老先生已经逝世了。我说:你别逗了。我们送来那会儿,刚才还打呼噜呢,你跟别人说去。 


可是那大夫说:请您节哀,总共就送进去一个。我登时瞪起眼来,说:胡扯!刚送进去,你还没给他看!他就说:令尊来的时候,呼吸已经停止了。你别揪我领子好不好!快来人!救命哪! 

这时来了一群白大褂,可是我只对那个急诊大夫紧追不舍。后来出来一个穿制服的,喝道:不准乱闹!你是哪单位的?我找你们领导:我说:你们他妈的找去!老子是知青!那人一听又缩了回去,知道全是亡命之徒,谁也不敢惹。 

刘老先生的事是这么结束的:最后医院的院长出来,请我和小转铃到办公室坐。他说:人总是要死的,这是不可避免的现象。所以有些危重病人,我们救不活,既然对我们的抢救措施有怀疑,做个尸检好吗?我们不但要对病人负责,也要对我们的大夫负责。那时我已经清醒了,说道:我和这死人没关系,你等矿院留守处来找你们吧。说完就和小转铃回家了,路上我和小转铃说,他是叫鸭子馋死的。 

当晚我和小转铃在一起,谈到刘老先生的好多事,均属鸡毛蒜皮。比方说:走廊里黑,又堆了很多东西。刘老先生定进来时看不见,就拿藤棍乱打,打得那报像狗咬过一样。刘老先生贪嘴,拿香肠在煤护上烤着吃,叫我们碰上啦。他怕我们说他,老脸臊得通红,圆睁怪眼立在那里说:你们谁敢说我一句,我就自杀!不活了!他怎么忽然死了呢?这事真逗哇。我们应该干一回纪念他。 

我们想起刘老先生好多事,都很逗,除了一件。有一回我爸爸告诉我:刘老先生并不笨,矿院的老人都知道,此人绝顶的聪明。他是故意装出一副傻样,久而久之弄假成真。所以我就去向他:老头,干嘛不要脸面?他马上回答:顾不上了! 

后来我下了床,走到窗口去,看见外面黑夜漫漫,星海茫茫。一切和昨夜一样,只是少了一个刘老先生。忽然之间我想到,虽然刘老先生很讨厌,嘴也很臭,但是我一点也不希望他死,我希望他能继续活在世界上。 

流年似水,日月如梭。很多事情已经过去了。在七三年元旦回首六七年底,很多事情已经发生,还有一些事将要发生。无论未发生和己发生的事,我都没有说得很清楚。这是因为,在前面的叙述中,略去一条重要线索。这就是在我身上发生了很多变化。有些变化已经完成,有些变化正在发生。前面说过,刘老先生告诉我贺先生的遗言,我听了当时很不以为然。但那天夜里我和小转铃干到一半停下来,走到窗前,想起这话来,觉得很惨。看到外面的星光,想起他脑子前面的烛火,也觉得很惨。刘老先生死了,也很惨。对这些很惨的事,我一点办法也没有,所以觉得很惨。和小转铃说起这些事,她哭了,我也想哭。这是因为,在横死面前无动于衷,不是我的本性。 

我说过,在似水流年里,有一些事叫我日夜不安。就是这些事:贺先生死了,死时直挺挺。刘老先生死了,死前想吃一只鸭。我在美国时,我爸爸也死了,死在了书桌上,当时他在写一封信,要和我讨论相对论。虽然死法各异,但每个人身上都有足以让他们再活下去的能量。我真希望他们得到延长生命的机会,继续活下去。我自己也再不想掏出肠子挂在别人脖子上。

\section{二十二}

流年似水,转眼到了不感之年。我觉得心情烦闷,因为没碰上顺心的事。而且在我看来,所有的人都在和我装丫挺的。 

线条在装丫挺的,每天早上上班之前,必然要在楼道里大呼小叫:“龟头,别把房子点着!按时吃药!” 

回来时又在楼下大叫:“大龟头!快下来接我,看我拿了多少东西!” 

李先生也装丫挺的,推开门轰隆轰隆冲下去。这简直是做戏给人看。要不是和他们是朋友,我准推门出去,给他们一个大难堪:李教授、李夫人:你们两口子加起来够九十岁了,还在楼道里过家家,肉麻不肉麻? 

我和线条,交情极为深厚。上初二时,到了夏天,我常和线条到玉渊潭去游泳。那时她诧异道:王二,你怎么了?裤衩里藏着擀面杖,不硌吗? 

我说:你不但,因为你不读书。我有本好书,叫《十日谈》,回去借给你看看。重要的地方我都夹了条子。你只看“送魔鬼下地狱”和“装马尾巴”两篇就够了。 

她说,这些话越听越不明白,最好找个没人的地方脱下来给我看看。于是找到了没人的地方,脱了给她看。线条见了惊道:王二,你病啦!小鸡鸡肿到这个样子,快上医院看看吧! 

当然,我没去医院。晚上把书借给她。线条还书时,满面通红地说:王二,你该不是现在就要把那魔鬼送给我吧? 


怎么?你反对? 

不是反对。我是说,就是要把它送给我,也得等我大。现在硬要送给我,我可能就会死掉啦! 

自从我把小和尚给她看过之后,线条的成绩就一落千丈,中英文数理化没一门及格的。因为给别的女孩讲过马尾巴,被老师知道了,操行评语也是极差。要不是我给她打小抄。她早就完蛋了。这线条原是绝顶聪明一个女孩,小学的老师曾预言她要当居里夫人的。他们可没想到,该居里夫人险些连高中也考不上。 

线条自己说,上初二韧三时,她被一个噩梦魇住了,所以连音乐都考不及格。那时候她觉得除了嫁给王二别无出路,可王二那杆大枪……噩梦醒了以后,嗓子眼都痒痒。 

如今我与线条话旧,提起这件事,她就不高兴。说道:王二,你也老大不小的啦,还老提这件事!不怕你不高兴,你那杆枪和我老公的比,只好算个秫秸杆啦。 

我马上想到,女人家就是不能做朋友。不说小时候我给她打过多少小抄、考试时作过多少弊,只说后来我在京郊插队,忽然收到一封电报:“需要钱,线条”,我就把我的奥米伽手表卖了,换了二百块钱,给她寄去了。 

我自己会修表,知道手表的价值。那块奥米伽样子虽老,却是正装货。所有的机件都镀了金,透过镜子一看,满目黄澄澄。全部钻石都是天然的,无一粒人造的。后来到美国,邻居是个修表的老头,懂得机械表,我对他说有过一块这样的表,他就说:你要真有,就给我拿来,五百一千好商量。要是没有,就别胡扯吊我胃口。我血压高,受不了刺激。那块表除了是机械工艺的结晶和收藏的上品,还是我爸爸给我的纪念品。我妈认识联合国救济署的人,所以家里不缺吃的。这块表是我爹拿一袋洋面换的。要是寻常年景,他也买不起这样的表。只为线条一句话,我就把这表卖了,二十年来未曾后悔过,直到她说我是秫秸杆才后悔了! 

我对线条说,这辈子再也不交朋友,免得伤心。线条就说:至于的吗?好吧好吧,秫秸杆的话收回了。可是你也太腻歪了。我老公和你是何等的交情,我和小转铃又是好朋友。你迫我干嘛?小转铃不是挺好的吗? 

李先生和我交情好,我也不想甩了小转铃,这些我全知道。怎奈我就是想抱她一抱,难道她不该让我抱一抱。所以我说她装丫挺的。 

小转铃也和我装丫挺。每次我要和她做爱,她就拿个中号避孕套给我套上。我的小和尚因此口眼歪斜,面目全非,好像电影上脸套丝裤。)去行劫的强盗。于是我就应了那些野药的招贴:“(专治)举而不坚、坚而不久!”这也很容易理解。假如一位一米九的宇航员,被套入一米六的宇航服,他也会很快瘫软下去。为此我向小转铃交涉:“铃子,这套子太小了。” 

“没办法。全城药房只有这一种号。” 

这医药公司也装丫挺的。我们这个年龄的人都会背这两句诗:“太平世界,环球同此凉热。”可也投听说环球同此长短的。我知道计生委发放避孕药具,各种尺寸全有。小转铃说:“王二,咱们将就一点吧。你知道不知道,我已经离了婚,是个单身女人?” 

其实真去要,也能要来。可是小转铃说:她单位正要评职称。假如人家知道她在和一个尺寸三十七毫米的家伙睡觉,会影响她升副编审。为了副编审,就给男人套中号,是不是装丫挺的? 

其实我自己也可以去要,我们单位也在评职称,而且我也是个离了婚的单身男人。我去要三十七毫米的套子,势必影响到我升副教授。所以我也得装丫挺的,连我妈也在装丫挺的。我让她去搞一些特号,她说:王二呀,我丧了偶,也是单身女人! 

我说:妈,您快七十岁了,谁会疑到您。再说,你教授已经到手了,还怕什么,不好意思说是给儿子要,就说要了回家当气球吹。 

“呸!实话跟你说,能要来,就是不去要。你还欠我个孙子呢!” 

我的生活就是这样,到了四十岁,还得装丫挺的。我就像我的小和尚,被装进了中号,头也伸不直,小的时候,我头发有三个旋(三旋打架不要命——王二注),现在只剩了一个,其它的两个谢掉了。往日的勇气,和那两个旋儿一道谢光。反正去日无多,我就和别人一样,凑合着过吧。 

我现在给本科生上数学分析课。早几年用不了一秒钟的积分题,现在要五分钟才能反应上来,上课时我常常犯木,前言不搭后语,我也知道有学生在背后笑我。有个狂妄的研究生当面对我说:听说您是软件机器,我看您不像嘛。 

我答道:机器?机器头顶上有掉毛的吗? 

还有个更狂的研究生说我:老师,我觉得您讲话它犯重复。 

我说:是吗?一张唱片用的时候久了,也会跑针的。 

还有一个女研究生对我说:老师,听说您是有名的王铁嘴,是名不虚传。 

这话我倒是爱听。但她在背地里说:这家伙老了以后一定得吧得吧得,讨厌得要命。 

我妈跟我说的却是:人就是四十岁时最难过。那时候脑子很清楚,可以发现自己在变老。以后就糊里糊涂,不知老之将至。 

叔本华说:人在四十岁之前,过得很慢,过了四十岁,过得就快了。 

咱们孔夫子说的是:四十而不惑,五十知天命,六十耳顺,七十从心所欲不逾矩。好像越活越有劲,真美妙呀!可不逾矩以后又是什么?所以我恐怕他是傻高兴了一场。 

除了别人说我和说四十岁的话,我还发现自己找不着东西;刚看过一本书,击节赞赏,并推荐给别人看,可是过了几天,忽然发现内容一个字也记不起来了。而过去我是出了名的一目十行、过目不忘。这对我倒是一件好事:以前只根书不够读,现在倒有无穷阅读的快乐。因为以上种种,在这不惑之年,我却惶惶不可终日,对什么都失去了兴趣,成天想的是要和线条搞婚外恋。更具体地说,是想和她干,当然,也不想干太多。我的身体状况是这样的:一局一次有余,二次勉强。所以干一两次就够了。 

我和线条谈这件事,是在矿院学生办的咖啡馆里,说着说着情绪激动,嚷嚷了两次。一次是因为说到秫秸扦,还有一次是谈到李先生和小转铃。我说他们知道了又有什么呢?小转铃爱我,李先生爱你,一定会原谅我们。现在一想到你,我就会直。所以有一件事可以肯定:假如现在不干,到直不起来时一定会后悔。有海涅的悲歌为证:在我的记忆之中,有一朵紫罗兰熠熠生辉。 

这轻狂的姑娘!我竟未染指!! 

妈的,我好不后悔!! 

我读过的诗里,以此节为最惨。线条说:这儿有我的学生,就站在吧台后面。你要是一定要嚷嚷,咱们到外面去。 

我和线条出了咖啡馆,在外面漫步。外面漫天星斗。我马上想起了二十三年前,也是仲夏时节,我和线条半夜里爬到实验楼顶上,看到漫天星斗,不禁口出狂言:假如有一百个王二和一百个线条联手,一定可以震惊世界! 

时至今日,我仍不以为这是狂言。两百个一模一样的怪东西聚在一起,在热力学上就是奇迹,震惊世界不足为奇,不震惊世界反而不对头。比方说,二百名歌星联袂义演,一定会震惊世界。一百个左独眼和一百个右独眼一齐出现,也会震惊世界。一百个十七岁的王二和一百个十七岁的线条联手,那就是二百名男女亡命徒,世界安得不惊耶?! 

那天晚上在实验楼顶,除了口出狂言,我还干了点别的事,对女人的内衣有了初步的了解。我的手从她上衣下伸了进去,解开了背后乳罩的挂钩,然后那东西就如护胸甲,松松散散挂在外衣和皮肤之间,以后探手到她胸前,就如轻骑入阵,十分方便。我发觉女人的乳房比其它部分温度要低,摸起来就如两个小苹果一样。除此之外,还说了些疯话:我们生在这亡命的时代,作为两个亡命之徒,是何等的幸福!真应该联手做一番事业! 

那天夜里我说道:在这世界上要想成一番事业,非(做)亡命徒不可。比如布鲁诺这厮,在宗教法庭肆虐之时提倡日心说,就是十足的不想活了。他被烧死了。作为一个男人,被烧死不足为奇,但他还熬丁无数的酷刑,实在可钦可佩。教廷说,只要你承认曾受魔鬼之诱惑,可以免遭刑罚。砍头、上吊、喝毒药,可随便你挑。临死前还可玩个妓女,嫖资教廷报销。但他选择了一条光荣的荆棘之路,被吊上拷问架去。两根绳子,一根捆手,一根捆脚,咯咯一叫劲,把他活活地拉长,原本一米六十的身高,放下来时被拉到三米七八。火刑处死之时,刽于手用杈子把他挑到柴堆上,盘成一堆(像蛇一样——王二注),放火烧掉。布鲁诺真好汉也!还有圣女贞德,被捕后,只消承认与魔鬼同谋,就可先吊死再烧。但她不认,选择了被活着烧。年轻姑娘的皮嫩,烧起来最难煞。根据史籍记载,那一天贞德身着亵衣,腰束草绳,被引到火刑柱旁,铁链拦腰束定。这时她发现,柴堆上面还铺了一层油松松针。这种搞法缺德得很。贞德见此,只微微皱眉,对刽子手说:愿上帝宽恕你。这贞德真是个好样的娘们!一点火时,松针上火苗猛窜上去,把头发眉毛亵衣一燎而光。还烧了一身燎浆大泡!把个挺漂亮的姑娘烧得像癞蛤蟆,还要忍受慢火的烘烤。人家在她对面放了镜子,让她看着自己发泡。只见那泡泡一个个烤到迸裂,浆水飞溅,而贞德在火焰中,双手合十,口中只颂圣母之名,直到烤成北京烤鸭的模祥,一句脏话也投骂。烤成烤鸭的模祥,她就热啦,圣母之名也念不出来了。在我看来,贞德比布鲁诺伟大。因为王二可以做布鲁诺,做不了贞德。我要被烤急了,一定要骂操你妈。圣女要是骂出这话,一切就都完了。 

我对线条说:老天爷会垂青我们,给咱们安排一场酷刑,到那时你我可要挺住,像个好样的爷们和好样的娘们! 

而线条则说:她希望酷刑之前给五分钟上厕所,见到血淋淋的场面她就尿频。 

二十三年之后,线条对我说:现在机会到了:我们正可以联手做一番事业。摆在我们面前的正是一场酷刑。我会秃顶,性欲减迟,老花眼,胃疼,前列腺肿大尿不出尿来,腿痛,折磨了我一辈子的腰痛变成截瘫,驼背,体重减轻,头脑昏聩,然后死去。而她会乳房下垂,月经停止,因阴道萎缩而受欲火的煎熬。皱纹满脸,头发脱落,成为丑八怪,逐渐死于衰竭。这是老天爷安排的衰老之刑,这也是你一生惟一的机会,挺起腰杆来,证明你是个好样的! 

线条所建议的是:在衰老到来之时,做一件值得一做的事,正如布鲁诺提倡日心说,贞德捍卫奥尔良一样。我们要在未来的痛苦面前,毫不畏缩,坚持到神志丧失的时刻:正如布鲁诺被拉成面条之前还在坚持日心说,贞德被烤熟之前口诵圣母之名一样。我们做这件事不是为了别的,只是为了证明自已是好样的! 

线条建议的事情相当值得一做。起码找还没想出有什么事比这还值得做。她还说,挑选我来做这件事,不是因为我有做成这事的能力与资格,只是因为少年时期我们是同伴,曾经发誓要联手证明白已是英雄(雄)好汉(娘们)! 

线条说,王二年轻时虽像一条好汉,但是到了四十岁,却只想苟安偷欢,不似一条好汉。况且他还没经过任何考验,不能证明他是好汉。而王二则说:他出过斗争差,被人打背了过去。和刘二师傅偷过泔水(偷泔水比偷汽车更需要勇气——三二注),怎么还不算条好汉?如果王二不是条好汉,线条又有什么事情能够证明她是个好汉(娘们)? 

线条说道:她爱上了龟头血肿。只此一条就能证明她是个好娘们。如果需要细节的话,那就是:她曾在河南安阳某地的一个破庙里,在寒冷和恐惧中,赤裸裸躺在砖砌的供台上,尽全力分开双腿,把贞操献给了李先生而不要任何保证。她还决定要在一生中倾全力去爱龟头血肿,其实李先生就像任何男人一样毫无可爱之处。只此一条她就可算通过了考验。 

线条的这些鬼话,不过是强词夺理罢了,不值得深论。但是这些说法倒可以说明,她为什么到河南去跟了李先生。她说,她是按自己的方式,在光荣的荆棘路上走到如今(参见安徒生《光荣的荆棘路》——王二注)。现在她还提供机会,让我们联手去搏取光荣。这个光荣就是把我们的似水流年记叙下来,传传后世,不论它有多么悲惨,不论这会得罪什么人。 

我一直在干这件事,可是线条说,我写的小说中只有好的事,回避了坏的事,不是似水流年的全貌,算不得直笔。如果真的去写似水流年,就必须把一切事都写出来,包括乍看不可置信的事,不敢写出这样的事情,就是媚俗。比如不敢写这样的事,就是媚俗:现在矿院门口正在建房子,有些地方盖起半截来,有些地方正在挖地基。结果挖出几方黑土来。别的地方是黄土,就那几块是黑的。年轻的工人不能辨认,有人说是煤,有人说是沥青,有入说是窖藏炭化的粮食。为了考据到底是什么,有人还抉了一块,放在嘴里尝尝,到底也没尝出个味道来。这件事情我们就知道:既非煤,也非粮食,是人屙的屎。 

在我们的似水流年里见过这样的事:我八岁那年,正逢大跃进,人们打算在一亩地里种出十万斤粮食,这就要用很多肥料。新鲜的粪便不是肥料,而是毒药,会把庄稼活活烧死,所以他们就在操场上挖了很多极深的坑,一个个像井一样,把新鲜大粪倒了进去。因为土壤里有甲烷菌存在,那些粪就发起酵来,嘟嘟地冒泡。我小的时候,曾立在坑旁,划着火柴扔进去,粪面上就泛起了蓝幽幽的火光。 

在我小时,觉得这蓝幽幽的火十分神秘。在漫漫黑夜里,几乎对之顶礼膜拜,完全忘记了它是从大便中冒出来。 

不幸的是,这挖坑倒粪的事难以为继,因为当粪发酵之后,人们才发现很难把它弄出来:舀之太稠,挖之太稀,从坑边去掏又难以下手,完全不似倒下去时那么容易。何况那些坑深不可测,万一失足掉下去,很少有生还的机会。所以那些坑,连同宝贵的屎,就一齐被放弃。 

过了一些时候,坑面上罩上了浮土,长起了青草,与地面齐,就成了极可怕的陷井。我的一个同伴踩了上去,惨遭灭顶之灾。这就是似水流年中的一件事。 

线条说,此事还不算稀奇,下干校时所说过另一件事。在同一个时期,当地的干部认为,挖坑发酵太慢了。为了让大粪快速成熟,他们让家家户户在开饭前,先用自家的锅煮一锅屎(参见北京大学社会学系沈关宝博士论文—一王二注)。一边煮,一边用勺子搅匀,和煮肉的做法是一样的。还要把柴灰撒进锅里,好像加入作科一样。煮到后来,厨房里完全是这种味儿。有些人被熏糊涂了,以为这种东西可以吃,就把它盛进碗里,吃了下去。 

这个故事是线条讲的,我听出前面是实(有沈博士论文为证——王二注),后面两句是胡扯,这种浪漫主义要不得。但是煮尿的事则绝不可少,因为它是似水流年中的一条线索。它说明有过一个时候,所有的人都要当傻×(线条所谓silly cunt——王二注),除此之外,别无选挥。当时我们还小,未到能作出选择的年纪。 

而当我们长大之时,就有了两种选择:当傻×或是当亡命之徒。我们的选择是不当傻×,要做亡命之徒。 

要记做亡命之徒的事,那就太多了。我们的很多同伴死了。死得连个屁都不值。比方说,在云南时,有些朋友想着要解救天下三分之二的受苦人,越境去当游击队,结果被人打死了。这种死法真叫惨不忍睹。想想吧:一、天下三分之二的受苦人,你知道他们是谁吗? 

二、天下三分之二的受苦人,你知道他们受的什么苦吗? 

三、正如毛主席所说,世上没有无缘无故的爱,也没有无缘无故的恨。你什么都不知道就为他们而死,不觉得有点肉麻吗? 

死掉的人里有我的朋友。他们的本意是要做亡命徒,结果做成了傻×。这样的故事太悲惨了,我不忍心写出来。假如要求直笔来写似水流年,我就已经犯了矫饰之罪。 

我还知道很多更悲惨的事——在我看来,人生最大的悲哀,在于受愚弄。这些悲惨的故事还写得完吗? 

线条说:就凭你这平凡、没长性、已经谢顶的脑袋瓜,还想在其它方面给人类提供一点什么智慧吗?假如你写了矿院的黑土之来历,别人就会知道它是屎,不会吃进嘴里,这不是一点切实的贡献吗?难道你不该感谢上帝赐给了你一点语言才能,使你能够写出一点真实,而不完全是傻屄话吗? 

如果决定这样去写似水流年,倒不患没得写,只怕写不过来。这需要一支博大精深的史笔,或者很多支笔。我上哪儿找这么一支笔?上哪儿去找这么多人?就算找到了很多同伴,我也必须全身心投入,在衰老之下死亡之前不停地写。这样我就有机会在上天所赐的衰老之刑面前,挺起腰杆,证明我是个好样的,但要作这个决定,我还需要一点时间。
