\chapter{地久天长}

十七岁那年,我去了云南。我去的那地方是一个群山环绕的小平原,有翠绿的竹林和清澈的小河。旱季里,天空湛蓝湛蓝的,真是美极了。我是兵团战士,穿着洗白了的军衣,自以为很神气,胸前口袋里装着红宝书,在地头休息时给老乡们念报纸。我从不和女同学谈话,以免动摇自己的革命意志。除此之外,那几年我干的事情就像水漏过筛子一样,全从记忆里漏出去啦。但后来发生的一些事情却使我终生难忘,印象是那么鲜明,一切宛如昨日。 

事情发生在那年春天。队里有个惯例,农忙时一天要给牛喂两顿红糖稀饭,要不牛就会累垮。那一天,教导员从营部来,正好看见我的朋友大许提了桶稀饭去喂牛。他一见瞪起眼来就喊:“给牛喝稀饭!哪个公子哥儿干的事儿!” 

他等着大许跑到他面前来认罪。可是大许偏不理他。教导员喊一声没人理,又直着脖子吼起来:“谁干的?” 

大许走过去说:“我提来的稀饭。耕牛都要喂稀饭,不然牛要垮的。” 

教导员斜着眼打量了他一番,冲他大喝一声:“牛吃稀饭!人吃什么?你给我哪儿来的送哪儿去!” 

大许被他溅了一脸唾沫星子,不由地发怒:“哪儿来的?那边大锅熬的,一头牛一桶。” 

教导员大怒:“你放屁!拿粮食喂牛就是要改!把桶提到伙房去!给人喝!” 

大许冷笑一声:“人不能喝啦,教导员。桶里我撒了尿啦。” 

大许没撒谎。牛就是爱喝人尿。我猜这是为了补充盐分,另外据说尿素牛可以吸收。因此,我们在没人的地方常常撒尿给牛喝,有时就撒到牛食桶里。教导员以为大许是拿他开心,伸手就揪大许的领子,要把他提溜走。大许当然要挣扎,两人撕扯起来。教导员大骂:“你这流氓!二流子!”大许回嘴:“你知道个屁!你就会瞎喳喳!” 

后来,别人把他们劝开了。教导员怒气不息,坚持要开大许的批判会,队长百般解释,他执意不听。直到队长急了,冲着他大叫:“教导员同志!你这么搞我们怎么做工作!我要向团党委汇报。”教导员这才软下来。可是晚点名时他又说:“你们队,拿大米喂牛!我批评以后还有人和我顶起来,好嘛!有两下子嘛!这叫什么?这叫无政府主义!”老职工在下边直嗤他:“他是怎么搞的,喂牛的饲料粮是上面发下来的嘛!”“咱们的牛都瘦成一把骨头了,还要犁地,他娘的不犁地的还要吃四十二斤大米哩。” 

从此以后,教导员见了大许总斜着眼。他知道大许出身不好,背地里常骂他狗崽子。后来就三天两头往我们队里跑,想找大许的碴儿。我发现他来意不善,常在背地里关照大许:“教导员要整你啦。”大许并不害怕,说:“我干我的工作,他整得着吗?” 

碴儿到底还是给教导员找着了。那年秋收时,大许的脚扎伤了,雨后地里潮湿,队里照顾他在场上干活。几千斤稻谷上了场,需要留人翻晒,于是又派了我和一个女同学邢红。 

早上雾气消了以后,我们打开麻袋,把半湿的稻谷倒出来,摊在场上,这活儿直到中午才干完。下午我们到场上时,她已经在那儿了。她洗了头,长发披在肩上,在树荫底下盘腿坐着,笑嘻嘻地看着小鸟飞,好像很感兴趣。我去拿耙子,想把稻谷翻一遍,可是她对我说:“别翻了!五分钟以前我刚翻过一遍。” 

于是我们俩也到树荫里坐下。我对大许说:“我看你什么时候还是去找教导员谈谈,他可能对你有误解,谈了就解开了。” 

大许回答得很干脆:“我不去!” 

我说:“还是去谈谈好。我可以替你先去说说。”这时我听见哧哧的响,原来是她在鼻子里哼哼。她说:“没意思。干吗让大许去讨饶?” 

我白了她一眼,觉得她瞎搭碴儿。她觉察出来,就笑了笑,走开了。 

大许低着头半天不说话,忽然,他抬起头来大叫一声:“不好!来雨了!” 

我一看,果然,乌云已经起来半天高了。我们赶紧去收稻谷。她不见了。我就喊:“邢红!邢红!来了雨了!” 

她在远处答应:“知道了!我在拉牛。” 

她从河边拉来一头牛。我们给牛架上个刮板,用牛拉着把稻谷堆起来果然快得多,一会儿就把谷堆撮起来一多半。 

风来了,雨马上就到,偏巧这会儿牛一撅尾巴。她赶快把牛尾巴按住说:“这个该死的!”她笑起来了。我连忙把牛赶到一边去,让它拉了一脬牛粪。这一弄实在耽误工夫。等我们堆好谷堆,雨点子已经劈里啪啦地打了下来。当时有一块盖谷堆的席子不合适,反正那席子已经烂了半边,大许就拿镰刀削下一块来,然后盖上防水布。刚弄完雨就下大了。 

我们跑到凉棚里躲雨,大许还拿着那块席片呢。我说:“扔了吧。”他说:“留着可以补箩筐。”忽然邢红弯下腰去看那席片,然后直起腰来在大许肩上拍了一下说:“你看这儿!” 

我们一看,席子上粘着一角人像。坏了,那会儿根本没有别人的像。大许吓得手直哆嗦,悄悄地把一角画像揭下来捧在手里看。 

这块席原来一定是草屋里打隔断的。我说:“怎么办?另一半在谷堆里呢。天晴以后打开就该被别人看见了。大许,你快报告去吧。” 

她说:“报告说是谁搞坏的呢?” 

我没吭声。大许说:“当然是我。” 

邢红说:“你瞎说,不是你。教导员正要整你呢,说是我好啦。” 

大许不干,他是个诚实的人。我忽然想出一条妙计来:“要是人家看见了,问是谁弄的,就说不记得有这么回事,不知道谁干的,这样就谁也不用承认了。” 

大家都同意了。可是傍晚收工时,那片席子就被上场摊稻谷的人发现了,而且教导员马上就知道了。他急如星火地赶了来,逼问我们这是谁弄的。我们当然说记不得了。可是他怎肯善罢甘休!他把我们挨个逼问了一通,让我们仔细讲一遍当天下午的活动,一个细节一个细节地讲,尤其是盖席子的过程,要一个动作一个动作地讲。不知他们感觉怎么样,反正在教导员逼我的时候,我觉得手心出冷汗,舌根发硬,说起话来结结巴巴。我讲完了以后他盯住我说:“你热爱毛主席吗?” 

我说:“热爱。” 

“好。你再讲一遍,是谁用刀削下席子的那个角的?” 

“记不清了。真的记不清,也许席子本来就缺一角。”他瞪起眼来说:“真的?有人反映,那些席子本来是不缺角的,一个缺角的也没有。你再想想。” 

我流着冷汗说:“我不记得有谁拿过刀。也许是折了以后撕的?” 

他眼睛发出亮光:“对,对,是谁?” 

“不记得是谁,我没看见。” 

他冷笑着看着我。 

他走了,我一个人坐在屋里,忽然心狂跳起来。也许这真是犯罪行为?我的做法是革命的吗?我对得起毛主席吗?一想到这个,我的心脏都要冻结了。 

正在这时,我又听到教导员在隔壁房间里咆哮:“就是你干的!你这个小狗崽子!我一猜就是你!你坦白吧,坦白了宽大你。不然要判刑的!” 

啊呀,原来是在审问大许! 

教导员吼了半天,大许没理他。他把大许轰走了,又把邢红叫了去,对她也像对我一样说了一气。邢红回答得很干脆:“我记不清是谁撕的席子了,很可能就是我。” 

教导员说:“你再想想。” 

她说:“实在想不起来。要是你一定要找个承担责任的人,就说是我撕的好啦。” 

教导员吓唬她:“这是个政治事件!撕毁宝像是反革命行为!” 

“我们是无意的。” 

“谁知有意无意。你知道犯这个罪要怎么处理吗?” 

“不知道。” 

教导员气得直咬牙:“你这种态度……哼,不用上纲,本身就在纲上!你回去考虑吧!” 

第二天,教导员宣布我们三个人停工,在家写交代。让我在宿舍里写,大许在办公室,邢红在会计室。还好,没派人看着我们。 

我坐在宿舍里,心里好不凄凉。说实在的,让我停工交待可把我吓坏啦。我倒不是热爱劳动到了这个份上,实在是吓的。要是教导员背地里骂我,说我是流氓、坏分子,我也顶多是害怕一阵。这一不让我下地,可就和群众隔离开了。我只要能和一般人一样吃饭睡觉干活,就会觉得心安理得。这一分开,我,我,我成了什么啦?我为什么一下子就成了这么一个需要隔离的人?想着想着我就没出息地哭了起来,就着这股心酸劲就写起来了。啊呀,提起这份检查我要臊一辈子。我写“敬爱的教导员”,还说我出身工人家庭,对毛主席是忠的,对领导是热爱的。又说自己工作一贯还好,受过教导员表扬等等,写了一大堆摇尾乞怜的话。后面说自己在宝像这个问题上粗心大意,一时疏忽,没有看清谁撕的,心里很难过,“心如刀绞,泪如泉涌”。最后是说要在今后的工作中将功补过,等等。还算好,我没把大许给卖了,可是也够糟的了,我说“没看清谁撕的宝像”,言下之意就是不是我撕的。我都奇怪,当时我怎么能干这种事? 

写完以后,我正坐在窗前发愣,忽然听见有人在我脑门前边说话:“哎呀,你都写完了?快拿来我看看。” 

我一看,原来是她站在窗外,笑嘻嘻的。她说:“怎么?你哭了!” 

我羞得满脸通红,把头转到一边去。忽然我想也跑出来是不许可的,尤其是不能来和我说话,就瞪着她说:“你怎么出来了?” 

她一迈腿坐在窗台上说:“为什么不能出来?” 

“哎呀,不是让咱们老老实实坐在各人屋里写检讨吗?” 

她撅起嘴来哼了一声:“听他的。又没人看着。出来玩玩有什么不可以?” 

我说:“呀。这可不成!要是叫教导员知道了事情就更大了。你快回去吧。!” 

她吃惊地挑起眉毛来:“怎么啦?教导员有什么了不起,我看他能不能把咱们怎么办。当然了,也不能和他顶僵了,这个检查还是要写。可我还真不会写这玩意呢,你写的检查让我参考参考好不好?” 

我不想给她。可是她真漂亮……于是我勉强答应了。她伸手去抓我的检查,我说:“你别拿走。”她嗯了一声,坐在窗台上看。我又说:“你下来吧,来个人看见就要命了!”她就下来坐在床上看。我的检查有五张纸,着实不短呢。她看着看着就笑了,还说:“好玩!小王,你这‘心如刀绞,泪如泉涌’可写得真棒!哈哈,你可真会装哭丧脸儿。”原来她把我的种种沉痛之词当成了讽刺!当然她不能体会我失魂落魄的心情。看完了以后她把它还给我,想了想,皱起眉毛来说:“可是你这检查整个看起来还像是告饶。当然了,告饶就告饶,没什么。可是你怎么写了个没看清谁撕了宝像?这点儿你得改改,要不然教导员会认定是大许撕的,他就更不肯甘休了。” 

我的脸马上红了,连忙拿笔把“看”字划了,换了个“记”字。她笑了笑说:“这就对了。看来你这篇我不能参考,写的全是你的话。我去看看大许写的什么。”她跳出窗户,又回过头来说:“喂!下午到河边去游泳啊?” 

我一听头都大了。去游泳!这是犯了错误反省的态度吗?我要是不去,她和大许去了,就我一个人在家,又显得太那个,何况大许又是我的朋友。我要去呢,一下午三个人都不在,万一教导员知道呢?再说我很害怕和个女孩子去游泳。不过我又很有点向往。结果我说:“不去好吧?万一有人看见?” 

她说:“不怕!中午最热的时候去。中午谁会出来走动?回来的时候从菜地边上的小树林里出来,那才叫万无一失呢。你放心吧!队里人都去山边挖渠了,剩下几个喂猪做饭的老太婆,她们才不来看你呢。” 

“可是教导员要是突然回来呢?” 

她笑了:“他呀,中午他肯定不回来!这太阳要把他鼻子晒脱皮。好啦,我来叫你。再见!” 

中午吃完了饭,我躺在床上想心事。忽然听见窗前有人叫:“小王,快出来。”我一看是她,就从窗口爬出去。我们两个叫上大许,她领着我们从菜地后面的树林往河边走。我问她:“怎么不走大路?”她说:“小河边有人洗衣服。好家伙,真不怕热!” 

我们从树林里出来,果然看见小河边上有个人在洗衣服,把小桥堵上了。于是我们绕到小河拐弯的地方,从老乡垒的拦鱼小坝上过了河,又在路边的沟里走了好长一段到了大河边上,头都晒晕了。 

大河里的水在旱季是很清的,就是太浅,最深的地方才不过齐胸深,又太急。邢红穿了一件绿色的游泳衣,在水里又踢又打,连水里的沙子都溅了出来。大许下了水,他情绪很阴沉,涮了涮又到岸上去坐着。我在水最深流最急的地方站定,让流水猛烈地冲着胸口,心里倒轻松了一点。我看着她在浅水处疯,心里有点高兴。我想过去,但是又不好意思。直到她叫我们:“大许,小王,你们都过来!” 

我们膛水过了河,到她身边去。她指着清清的河水里一些闪光的小片说:“这是什么?”河水中有一些闪光的小薄片,被水流冲得旋转着,在阳光下闪着金光。她跪在沙滩上,用手掬起一捧水,端到眼前,那些小薄片沉下去了。我告诉她这是云母,她有点失望地把水放了,说:“我还当是金子呢。” 

这一回就连大许都笑了一声。她让我们坐在她身边。这个地方很隐蔽:河在这里转了个大弯,河岸上长着很高的茅草,从哪儿都看不到。她说:“我有一件红游泳衣,可是我拿了明明的绿游泳衣。怎么样,我想的不错吧?” 

我说:“什么不错?” 

“嗐!红的暴露目标呀!” 

我们又忍不住笑了一笑。我说:“要是被人发现我们不在,你穿隐身衣也没用了。我看我们还是早点回去为妙。”大许默默地点点头。她说:“忙什么?先到对面树荫下坐一会。” 

到了那儿,她把一件洗白了的破军装披在肩上,从衣服兜里掏出两张纸说:“这是我的检查,你们看看。” 

她的检查就是一个最缺乏幽默感的人看了也要笑出声来。开头说的是:“敬爱的教导员:祖国山河红旗飘,六亿神州尽舜尧。在一片革命歌声中,我们迎来了七十年代第一春!”结尾是:“我的水平不高,毛著活学活用得不好,检查之中如有不符合毛泽东思想之处,请教导员指正。”中间尽是一片胡说八道,好像是篇批判稿,说什么,宝像的被毁坏,是由于国际帝修反的破坏。说到事情的过程,只有一行字,“可能是我们三人中任何一个弄坏的,斗私批修地说,尤其可能是我。”总之,你看了她的检讨,猜不出她说的是什么。她说:“我把会计室的报纸全翻遍啦。”她又要大许拿他写的来看看,大许不给她。原来邢红上午去找他,他还没有写。我说:“要是写了就拿来看看,别怕,我写的也给她看过。你还信不过我们?” 

大许低着头说:“我怎么会?你们对我太好了。你们要看就看吧。”他掏出来递给她。那纸上总共三行字,写的有核桃大小:“割破宝像的就是我,我是在盖谷子时用刀子裁席子裁破的,是无意的,请领导上批判教育。检讨人:许得明。” 

邢红抬起头微微一笑,说:“我早就知道你要这么写!”她把这张纸哧地撕了,扔到河里。她冷笑着说:“你为什么要这么写?以为这么写了我们就不受连累?傻!我们都说没记清,你要咬我们一口?还是怕我们以后说出来?你听着,我以后要是告诉除咱们三个人之外的任何人,就是王八!” 

我俩都笑了。这么一个女孩子一本正经地赌咒可真好玩。我说:“我也是。绝不告诉别人。” 

大许皱着眉说:“可是我确实撕了宝像。不说,对吗?” 

听了这种话,我感到沉重。不管怎么说,我们在向组织隐瞒一个重大问题,这是不可宽恕的。可是邢红说:“你多笨哪!明摆着教导员要整你,你还要自己送上门去。” 

他听了她的话,低下头去。忽然又抬起头来说:“可是你们这么包庇我,是对的吗?” 

邢红猛然一伸胳膊,把上衣扬到地上,她站起来,把她苗条的身体投到阳光里去。她扬起头,把披散的头发垂到脑后,眯起子眼睛,双手交叉在胸前说:“当然我们是对的。不管怎么说,我相信自己是个好人。你也是个好人,小王也是。至于其他的,我都随他去,要批斗就批斗好了,有什么了不起。”她忽然转过身来说:“我衣兜里有一份检查,是给你写的,我书包里有纸笔,你抄——份吧。你不要这么提心吊胆的,没什么了不起。我要下水去啦,小王,你去吗?” 

我点点头,于是我们下河去了,大许在岸上呆子一会儿,就心安理得去抄检查了。我和邢红一起在浅水处奔跑,又到深水处去掏老乡下的鱼篓,看看他们捉了几条鱼,不过我们没拿他们的。我有点迷上邢红了,她显得矫健又玲珑。她真美啊。我开始对她有了一点不寻常的感情。后来我们上了岸,大许已经抄好了他的检查。我们就一起溜回去,谁也没看见我们。等挖渠的人回来,我正手托着头冥思苦想哩。可是我想的是邢红这么帮大许的忙,莫不是爱上他了?这时,教导员来要检查,我就给了他。 

教导员把我们的检查看了一遍,勃然大怒。他立刻决定批判我们。吃完了晚饭,他把一些人叫去开预备会,其中有好几个是活学活用的积极分子。开完会回来,他们都绷起脸来不理我们,和别的同学说话也背着我们。有人小声告诉我:要批判你们啦。我心里慌了一下,后来一想,慌什么呢,反正到了这步田地,豁出去了。顶多是“站起来”,“到前边站着”,去听批判。 

谁知到了晚上,教导员派了两个人来跟着我,连我上厕所也跟着。平时我跟他们都住一个屋,这会儿耷拉着脸也不理我了。我觉得有点不妙,脑袋后面直发凉。到晚上有人吹哨,叫大家去开会,我看见大许背后也跟着两条大汉。啊哈,会场上点着四盏大汽灯,可真舍得油啊。教导员站到桌前,说:“今天这个会,是批判破坏宝像的许得明、王小力和邢红的大会。把许得明和王小力带上来!邢红在下面接受批判。”我后面的两个人就来推我。我站起来走上去,可是感觉有点腿软。大许也走到前边来。邢红也跟上来了。教导员对她了瞪眼说:“谁让你上来的?”她说:“批判我们三个人嘛,我当然上来。”教导员冷笑一声:“好啊!”他大喝一声:“你们面向群众,低头!” 

面向群众倒不怕,低头可是低不下去。教导员大吼一声:“把许王捆起来!”跟着我的两个人立刻就来扭我的胳膊,我拼命挣扎。真想给那两个家伙一人一拳,还是同学呢。可是我不敢打人,只把双手捏在一起,不让他们把我的手扭到背后。我听见大许使劲地喊:“啊……!!”底下老职工乱起来,有人叫:“是些小娃娃嘛,捆起来干哪样?”折腾了半天,教导员扑过去帮着捆大许,结果把大许捆起来了,我呢,还没捆上。我也不知哪儿来的劲,简直邪性,双手握在一起,三四个人都弄不开。教导员来看了看,说一声“算了”,于是就开会。可是邢红站到他面前说:“你也把我捆起来!你捆!”我们那儿批判会常常捆人,可还没捆过女的呢。教导员不敢动手,就叫女知青来“押住”邢红,果然就有两个积极分子上来扭住了她的胳膊。教导员回头来看我,我冲他瞪大眼睛,他又叫人来捆我,这回我让他们捆了。那硬邦邦的竹壳子捆住手腕疼得要命,绳子往脖子上一扣马上就透不过气来。这会儿下面的人走散了一半,我们队长也不见了。发言的人一个接着一个,说我们是“知识青年的败类”等等。正在批判,队长跑来说:“团部指示,这个会不能开,尤其不准捆人;叫先把人放了。”教导员刚要瞪眼,队长说:“政委说了,这个事你要负责任。”教导员立刻软了下来,不得不宣布散会。 

根据团里的意见,毁坏宝像的事情是无意的,不予追究。捆打知识青年一事教导员要道歉,受害者也不要上告,事情就这样两拉倒。 

当晚,我和大许坐在床上根本不想睡,气得脑门子发涨。细细一想,斗我们捆我们的全是自己的同学,为了什么呀,不过是为了给教导员留个好印象,以后能在讲用会上说说他们怎样站稳了立场,然后到团里当个文书、干事之类,写些狗屁不通的报告。为了这个背叛我们,值得吗? 

熄灯时,我们屋那两个家伙回来了,怯生生地轻手轻脚地溜进门来,悄悄地坐在床上。我一下子站起来,大喝一声:“你们两个搬出去!别跟反革命住在一块!”有一个小声说:“王哥,别赖我们。我们也没法子。”我的野性发作起来,大吼一声:“滚出去!快滚!”接着把他们的东西全都扔了出去,他们两个不敢再说什么,忍气吞声地捡起东西走了。 

邢红也不和同屋的女生说话了,还拌了两句嘴。我和大许知道以后,第二天上工的路上毫不留情地骂那个女生。我们简直丧失理性了。我们两个叉着腰骂她是“走狗”,是“马屁精”、“缺德鬼”,骂得她捂着脸哭了一整天。其实我们本不至于骂出这样的话,可是我们一想起那天晚上她在会场上撅邢红的胳膊,还揪她的头发,就气得要命。她要是个男的非挨我一顿打不可。大许不会打人,他只会在别人打他的时候还手,可是我那些天像个野人一样,邢红说我在地里干活时都斜着眼看人,一副恶相。 

这事过去之后,有些家伙开始在背后给我们造起种种谣言来。队里风言风语地传说我们有什么生活问题。这种话使邢红很伤心,可是她从来也没对我们提起过。我们也不好和她说这个,只是以后我们益发形影不离,就连吃饭她都要端着碗到我们屋里来吃。在地里干活休息时,不论时间多短,她也要来和我们一起坐一会儿。和我们在一起时她显得迷人,她对我俩都好。她箱子里有很多书,晚上我们就读书,哪儿也不去,就是连里开批判会我们也只当不知道。后来她索性把脸盆漱口杯都拿过来了,弄得我们的懒觉再也睡不成,因为天一亮她就来敲门,说:“快起来!我要进来啦。”中午我们睡午觉的时候,她就在我们屋洗头,洗好头以后就静静地坐下来看书。只有晚上睡觉才回她屋去。 

我和大许都爱她,可是我们都不想剥夺了她给别人的一份爱,因为她似乎同样地喜欢我们两个人。 

我到现在还记得我们三个人在一起度过的愉快时光。我们那里的旱季天特别长,由于是农闲,收工又早,我们回来时天还很亮呢。大许去水井打水,我把我俩的脸盆和毛巾拿到走廊上来。他把水打回来了,我们在门前脱成赤膊,洗去身上的泥巴,这时我们可以听见屋里的溅水声。我们洗完以后就坐在门前的小板凳上。这时她就在屋里说:“大许,小王,你们洗好啦?”“啊。”“你们别进来,我还没好呢。”她从来不插门。等到她说“好啦”,我们就走进去。她坐在窗前的床上,嘴里咬着发卡。我说:“我们干什么?” 

“看书吧。把我的书箱子打开。” 

她有好多书,有她带来的,还有她借来的,还有人家送给她的。她穿着我的拖鞋走过去把门打开,让黄昏的阳光照进屋来。她喜欢躺在床上看书,用一块塑料布垫在枕头上,免得湿头发把枕头弄湿。她还有很多孩子气的小毛病,看书的时候会用脚趾弹出“橐橐”的声响。开饭钟打响的时候,她有时会发起懒来,当我们收拾起饭盒,对她说:“小红,起来!去吃饭。”这时候她会轻轻地一笑:“我不想起来。你们给我打来吧。”我们说:“你太懒了。我们今天不想侍候你。”她会说:“那我还给你补袜子了呢!我还给你洗衣服了呢!”我们就说:“我们这是为你好,你要得懒病啦。”她慢慢坐起来,然后又躺下去。“不会的,少打一次饭得不了懒病。再说我比你们都小,你们应该让着我。”于是我们就让着她了。 

吃完饭,天开始暗下来,她还是躺在床上看书,过一会儿她会忽然欠起身来问:“大许,你看什么书呢?”大许告诉她,她说:“噢。”然后躺下去,再过一会儿她又来问我,我也告诉她。她也许会高兴地继续说下去:“噢,是肖。你喜欢他吗?”我说:“挺细腻的,不过还是不喜欢。”“哎呀,我可喜欢他呢,那老头可精啦。”要不然就会莫名其妙地说:“喂,喂喂!你们俩都别看书啦。问你们,喜欢杰克·伦敦吗?”我们这样的毛头小伙子哪会说不喜欢。她说:“他太野蛮啦。人应该会爱,像好人一样。对!我不喜欢。”我反唇相讥:“你是小姑娘。你别傻啦。”她会高高兴兴地说:“对啦,我是小姑娘。”说完了就不作声了。 

天黑到在屋里不能看书时,我们就都到门外去坐。有时候一声不响,看着天边一点点暗下去,对面傣寨里的竹梢背后泛出最后一点红色。有时候她会给我们讲小时候的一些琐事,她讲得特别有意思。她讲她有一次和哥哥爬上屋顶去摘桑葚,那是一座西式的房子,尖尖的洋铁皮顶,哥哥上树去了。让她坐在屋顶上等着,可是她往下一看,高极了,足有七层楼高——那是两层楼,不过她才四五岁,当然觉得高。于是她反过身来往上爬,越爬就越打滑,一直滑到离房檐不远的地方,吓得她一动也不敢动,大哭起来。晚上回家以后,衣服上剐破的窟窿叫妈妈看见丁。不管妈妈怎么问,她也没说出哥哥来。她骄傲地说:从那时我就感到,大人的话有时可以不听,应该正直,不出卖人,这比听话重要得多。她还讲过别的一些小事儿,我们都很爱听。她说困难时期,她的同桌家里孩子多,总是吃不饱。她每天给他带一个窝头。可是后来上中学以后他就忘了她,见了面也不理了。我们都知道这是为什么。嘻,我们上中学时也不敢和女同学来往,为了做个正派人。总之,我们渐渐发现她是个特别好的女孩子,她什么也不怕。她本能地憎恶任何虚伪,赞美光明,在我们困惑的地方,她可以毫不费力地指出什么是对的。我觉得她比我们俩加起来还聪明得多。 

因为我们三个人形影不离,大家渐渐把我们看成怪人。他们看见我们一起走过来都带着宽容的微笑。他们还是喜欢我们的。有一次我远远听见几个老职工说:“三个挺好的孩子,都是教导员给害的。”原来他们认为我们得了某种神经病。后来我告诉大许和小红,他们都觉得好笑。不管怎么说,我们愿意在一起,让他们去说吧。 

后来队长派活也把我们三个派到一块,通常都是三个人单独在一块干活。可是有某种默契,就是我们必须不挑活。开头是让我们三个去田里把稻草拉回来。我们赶着三辆牛车。一般女同志不适合赶牛车,因为牛有时候会调皮。可是邢红赶得很好。我们赶上车到地里去。旱季的天空是青白色的,地平线上白茫茫,田野里光秃秃。太阳从天上恶狠狠地晒下来,连一片云也没有。稻草干得发脆,好像鸡蛋壳一样。我们往车上扔稻草的时候,邢红站在车顶上接着。她穿着我们的破衣服,衣服显得又大又肥,她的样子好玩极了。我们把稻草捆拼命地往上扔,一直扔到她抱怨起来:“慢一点啊!”等我们停下手来,她就趴在稻草上笑着说:“你们真伟大,不过还是慢一点。”如果我们再快扔,她就躺下不动,直到我们扔上去的草把她埋起来,她才从草里钻出来,飞快地把草码好,还高兴地喊:“来吧,我不怕。我比你们快!”然后我们就拉着三个稻草垛回去。我们运的稻草比六辆车运的都多。 

后来草运完了,队长很满意,说:“如果知青都和你们一样,我们可以多种一千亩地。”可是他又让我们去出牛圈,他说:“你们可以慢慢干,让邢红在外边干点杂活。牛圈离家近,你们可以自己安排时间,什么时候干都可以。” 

我们队的牛圈有好几年不出了。那是一间大草棚,有一个篮球场那么大。因为从来不出粪,也不垫草,简直成了个稀屎塘,大牛下去淹到肚子,小牛下去可以淹死,真够呛。我们去看了一下,我说:“邢红别下去了,留在外边吧。” 

她说:“我不在外边,我要和你们在一起。” 

我进去探探深浅,牛粪一直淹到我大腿上半截。我们拉来一头顶壮的水牛,驾上一套拖板,邢红在前边拉牛,我们两个在后面压住板梢,把那些牛粪从圈里拖出来晒。哎呀,那些粪真是骇人听闻,说起来你都不信。那头该死的牛拼命地甩尾巴,溅起来的粪总打到人脸上。每当我们从牛圈里推出一大堆粪来都要到水沟里洗洗脸,邢红的头发里也溅上了。这里太脏了,我们连话都顾不上说。连那条该死的牛出来以后都不肯再进圈,总要做一些古怪花样才肯进去。我们连中午饭也没吃,弄到下午三点钟,那条牛一下跪下不起来了。邢红大叫一声:“我也受够了!”她骑到牛背上说:“走,牛,咱们到河边游泳去。”那牛腾的一声跳起来,飞快地朝河边跑去了,快得让我们两个死追也追不上。我在后边一边追一边喊:“小红!你勒着点鼻绳呀,别摔下来!”她在牛背上说:“你别怕,我摔不下来。”她哈哈地疯笑起来。水牛背又宽又滑比马难骑多了,那牛跑得比马还快,可是她居然没有摔下来。到了河边,那牛一头蹿下水去,她也从牛背上翻下来摔到水里了。可是她马上又跳起来,在齐腰深的水里朝上游跑过去,最后弯腰一头扎到水里。等我们跳到水里去的时候,她在上边大叫:“我已经洗干净了,你们快好好洗洗。” 

后来我们在沙洲上坐在一块儿,她全身水淋淋的,衣服都贴到身上,头发披在肩上。她哈哈笑着说:“多棒啊!我觉得妙得很。” 

那地方河水分成两股,围绕着一个小岛,牛跑到岛上吃草去了,小红很高兴,她喘过气来以后又到水里去,还和我们打水仗,后来就坐在沙滩上让太阳把衣服晒干。坐了一会儿,她躺在沙滩上,两眼看着天空,说:“天多蓝啊。我有时觉得它莫名其妙。我觉得,我是从那里宋的,将来还要消失在那里。”她有点伤感。我们也伤感起来。我们想到,总有一天,我们也会消失在自然的怀抱里,那个时候我们注定要失去小红了。还有,也许我们注定永远在这里生活了。哎,这世界上我们不知道的事情太多了。可是她悄悄地坐起来说:“不管到哪里,我只要做一个好人,只要能够做好事,只要我能爱别人并且被别人爱,我就满足了。大许,小王,你们都喜欢我吗?” 

我们都说:“喜欢。”我们目不转睛地注视着她。斜射的夕阳把她飘扬的头发、把她的脸、把她的睫毛、把她美丽的胸和修长的身体都镀上了一层金。她很美地笑了。她说:“我喜欢你们。我爱你们。”我们静了一会,她忽然高兴地笑了:“好啦,我教你们唱一支歌吧。一个好歌,古老的苏格兰民歌。” 

她教我们唱了《友谊地久天长》。以后我们常在一起唱这支歌。她后来又教给我们好多歌,但是都没有这支歌好。我和大许都是音盲,除她教给我们的歌就不能把任何歌唱好。 

后来我们都觉得饿了,就把牛找回来,赶着它回家了。 

第二天我们又去出牛圈,这一回牛粪浅了。我们三个驾起三套拖板一齐把牛粪推出去。牛还是甩尾巴,甩得粪点子横飞。三条牛尾巴弄得人走投无路。后来小红用一根绳子把牛尾巴拴起来,它就再也不能甩了。可是牛被拴住了尾巴觉得很不受用,走起路来大大地叉开后腿,怪模怪样的。被拴住的尾巴拼命扭动着,好像一条被钉住的蛇。我们大笑起来,也把我们的牛这么拴住。于是三头牛跨着不稳定的舞步走来走去,我们都觉得很好玩。邢红还温存地对它们说:“牛,对不起你们。牛,等一会带你去游水。” 

到下午我们三个就骑上牛到河里去玩。邢红还带了米和锅,我们在河边做饭吃。吃完了饭,我们坐着看傍晚的云彩,刊天黑才赶牛回去,为的是让它们多吃点草。可是第二天我们去拉牛,那三条牛都惶恐万状地躲开我们。小红很伤心,以后她就不拴牛尾巴,我们也不拴了。后来牛又和她好了。牛会悄悄走到她面前来,她就轻轻地摸摸它们的鼻子。她对我们说她很喜欢水牛,喜欢它们弯弯的角、大大的眼睛,还喜欢凉荫荫的牛鼻子。她说牛的傻样很可爱,可是我就看不出来。 

我们把牛圈出好,队长又派我们到镇上去拉米,后来又让我们三个去放牛。从来也没见过让女孩子放牛的,不过因为可以和我们在一块,她便毫不犹豫地答应了。 

我们一起去放牛。早晨的雾气刚刚散去我们就赶着牛到山上去,带着斗笠和防雨的棕衣,还带着米和菜。我们跟在牛后面走着,小红倒骑在最后一头牛背上。我们商量把这些牛赶到哪儿去。小红忽然高兴地挺直身子,拍打着牛背说:“到山里边小树林去,那儿可好啦。”牛向前一蹿,把她扔下来了。我们赶紧搀住她。她和我们一起笑了,然后说:“到小树林去,到小树林去!那儿有好几个水特别清的水塘,我顶喜欢那儿啦!那儿草也好,去吗?” 

她这么说好,我们怎好说不去。到了山底下,牛群争先恐后地往陡陡的山坡上爬,简直比打着走得还快。爬上第一个山坡,我们并肩站住往山下看:整个坝子笼罩在淡淡的白色雾气中,四外是收割后的黄色田野,只有村寨里长满了大树和竹子,好像一座座绿色的城堡。起伏的山丘到了·远处就忽然陡立起来,上面长满了树,黑森森的,神秘莫测。在寂静的小山谷中,有一片密密的小树林,那就是小红要去的地方。这里的天空多么蓝啊,好像北方的初秋一样。小红往我们脸上看了看,笑了一下说:“嘿,走吧!” 

牛群早就冲到山谷里去了,我们追上去。接着,我们必须分开了。我到左边的山坡上去,大许到右边的山坡上去,小红留在后面,为的是不让牛群走得太散。其实牛只要看见这边山—上有人,自然就不会过来,把小红留在后面也是多余的,因为没有一头牛会掉头回去的。牛都散开了,一心一意地吃草,慢慢地朝前去。我坐在一棵孤零零的小树下,我也是孤零零的一个人。大许隔得很远,小红也隔得很远,他们看起来都不过一粒豆子那么大。我倚着小树,铺开我的棕衣坐着,面对着蓝蓝的天空和白白的、丝一样的游云,翠绿的山峦,还有草地和牛,天地是那么开阔。 

我半躺着,好像在想什么,又好像什么也没有想,我忽然觉得有一重束缚打开了:天空的蓝色,还有上面的游云,都滔滔不绝地流进我的胸怀……我开始倾诉:我爱开阔的天地,爱像光明一样美好的小红,还爱人类美好的感情,还爱我们三个人的友谊。我要生活下去,将来我要把我们的生活告诉别人。我心里在说:我喜欢今天,但愿今天别过去。 

这时我听见小红在叫我,我看见她跑过来,披散的头发在身后飘扬。她穿着我们的旧衣服,可是她还是那么可爱,好像羚羊那么矫健。她一个鱼跃扑在我身边,然后又翻身坐起来。她喘吁吁地说:“哎呀,好累。往山上跑真要命。” 

我笑着说:“小红,出了什么事?” 

“没事,来看你。”她转过脸来,慢慢地说:“你一点也不需要人来看吗?” 

她蜷起腿来坐着,说:“我一个人坐着有点闷呢,你就不闷口马?” 

我说:“不闷,我很喜欢这么坐着。我喜欢。你看,从天上到地下都多么可爱呀。”我转过身来,看见她正笑着看着我,她说:“你越来越可爱啦。” 

我有点不好意思地低下头去,可是她满不在乎地哼起一支歌,接着就躺在我身边了。 

我觉得紧张,就往前看。后来听见她叫我,我转过身去,看见她躺在草地上,头发散在草上,她很高兴。她的眼睛映着远处的蓝天。她说:“你和大许怎么啦?” 

我说:“我们怎么啦?” 

她笑了。她在草地上笑好看极了。她说:“你们两个好像互相牵制呢。不管谁和我好都要回头看看另一个跟上来没有。是不是怕我会跟谁特别好,疏远另一个呢?” 

我辩白:“没有。”其实是有这么回事的。 

她一本正经地说:“你们别这样了。我不会喜欢这一个就忘了另一个的。你们两个我都喜欢。你们都来爱我吧,我要人爱。” 

我也很高兴。她又说:“将来咱们都不结婚,永远生活在一起。” 

我也像应声虫一样地说:“不结婚,永远在一起。” 

她又规规矩矩地坐好,用双手抱着膝头,无忧无虑地说:“多好呀,和人在一起。”一转眼她就站起来跑开了,跑出了树荫,她的头发在阳光下闪着光。我对她喊:“你去哪儿?” 

她高高兴兴地回答:“我去看大许!” 

她像一只小鹿一样穿过牛群,一直跑上对面的山坡,头发飞扬。她真可爱,她说的一切都会实现的,我想。 

到中午牛都吃饱了,甩着尾巴朝前走起来,越走越快,渐渐地汇成群。我们三个人又走到一块来啦。我们跟着牛走,小红还嫌牛走得太慢,拾起土块去打牛。我们唱起歌来。后来就走到小树林了,牛开始往前疯跑,大概是闻见水味了。我们怕它们跑远了,也加快脚步抢到前边去,大许向左我向右。小红跑了一上午,再也跑不动了,她在后边喊:“小王,大许,去给咱们占个好地儿啊!别叫这些该死的把水塘全占了!”我冲进小树林,找着一个又深又清的水塘守住,把来的牛一律打开,轰到小水塘和泥坑里去。过一会小红和大许都来了。小红笑着说:“这些该死的全下了塘啦。咱们没事儿了。乌拉!我们来做饭!” 

我们来到的地方真好,草地上疏疏落落地长着小树,上游下来的小溪在树林中间汇成一个又一个池塘,我挑中的这一个简直可以叫做小湖呢。我们在树荫下边的一个小干沟里支起锅来,把我们的棕衣在一边铺好。小红从书包里拿出一块腊肉,她笑着对我们说:“上回赶街子我买的。我们今天来吃吧。”我们三个人的工资都交给她管,我和大许就真正不问阿堵物了。可是钱一给了她我们就老有钱,再也不会捉襟见肘了,这真是一件奇怪的事情。吃完了饭,我和大许就跳下水去游泳,小红跑到树丛里换衣服。她在树林里大喊大叫:“喂,水好吗?水里好吗?”水特别凉,可真是从森林里流出来的。我们说:“好,好极啦!你快来吧!”一会儿她蹦蹦跳跳地走出来,穿着她的红色游泳衣,嘴里喊:“我来啦!我来了!”她一下跳到水里,马上又探出头来说:“嘿!可真要命,这水可真凉。”她高兴地仰泳起来,中间的水清得发黑。她游到中间时我们可以看见她发白的小脚掌在一蹬一蹬的,她喊:“你们游泳没我游得好!不信你们就追过来,比比看。” 

我们迅速地游近她,她一下子潜到水下去了,我也潜下去、啊呀,这个塘底下准有泉眼,寒气刺人。我简直就下不去。我在水里睁开眼睛,看见她在我下面游,可是我捉不住她,我就回到水面上来,我和大许焦急地往水下看。后来看见一个人影飞快地浮上来,我们就游过去,等她一蹿出水面就从前边捉住她。她的身上像鱼一样凉。她噗噗地出着气,在水里跳了几下说:“嘿,底下可真凉,我身上都起鸡皮疙瘩了。我还给你们捧了一捧底下的水来,叫你们一捉全洒了。你们怎么不下去玩?”我说:“水太凉,冷得死人。你也别下去了,会抽筋的。”她撅起小嘴说:“你又来吓唬人,抽筋我也淹不死。”她又往下潜,出来的时候神秘地对我们说:“喂,底下有大鱼呢!就是滑溜溜的,不好捉。你们等着,我捉条鱼晚上吃。”我说:“你得了!水里的鱼手可捉不住,滑着呢。”她歪起头来一笑,说:“真的吗?我偏要试试。”她在水里穿着小小的红游泳衣,好像水仙女一样。我和大许游开去上岸晒太阳了,她还在水中间潜水,她真是疯得没底啦。一会儿说:“差一点没捉住!”一会儿说:“这次没碰上!”我和大许对着她笑,因为她那么高兴。后来她下去好长时间才上来,她还在水下我们就发现她上来得慢,动作不正常,我看大许,他也变了脸色,我们赶快下水朝她游去。果然她一露出水面就用手乱打着水说:“我抽筋啦!你们快来救我呀!”我们吓得眼睛都要瞪出来了,只恨爹妈没多生出几条腿来打水。可是她还笑:“你们吓得龇牙咧嘴啦!别害怕,我不会立刻就沉下去的!”可是我们紧张得心都跳坏了。等我们游到跟前,她蹿起来,用双手勾住我们的脖子,她又笑又咧嘴,一会儿说:“你们拖我上岸吧。”一会儿说:“啊呀,腿痛死啦厂我们可一点开玩笑的心情也没有,转过身去就朝岸上游。她架在我们脖子上,一点也不介意地把高耸的胸脯倚在我们肩上,还说笑话:“哎呀,这可真像拉封丹的寓言!两只天鹅用一根棍把个蛤蟆带上天……不对,你们在游蛙泳,蛤蟆是你们!” 

我们可一点开玩笑的心思也没有。我们拖着她一点也游不快!为了抵消她浮在水上的上半身的重量,我们几乎是在踩水,哪能游得快呢。她仍是高兴地说个不停,急得我喝了好儿口水呢。等到我的腿一够到水底,我就在她背上啪啪地打了两下,说:“你这坏蛋!大坏蛋!”大许伸手给她理头发,也说地:“你吓死我了!”她撅起嘴来。我们俩把她从水里抬上来,收到棕衣上。这时我们的腿都软了,百分之九十都是吓的。他喊“抽筋了”时我们离她还有七八十米呢,我都不知怎么游过去的。在把她拖上水来之前我心里一直是慌的。我真想多打她几下,让她再也不敢。我去给她捏腿,她不高兴地说:“你们对我太凶了!”我抬起头来一看,她噙着泪。她又说:“你骂我坏蛋时,哑着嗓子野喊。我怎么啦?”她小声抽泣起来。 

我们都低下头去。后来我抬起头来,小声说:“你不知道吗?我们太怕你淹死了。我看见你出了危险,吓得手都抖起来了。” 

她撅着小嘴看我们,眼睛里有好多怨艾。看看我,又看看大许,后来眼睛里的怨艾一点一点退去了,再后来她阴沉的小脸又开朗起来。她忽然笑了,伸手揩去眼泪,眼睛里全是温情她说:“你们,你们这是太爱我呀。”我们俩点头。她顽皮地笑着说:“你们过来。”等我们蹲到她身边时,她猛地坐起来,用双臂勾着我们的脖子,她的额头和我们的额头碰在一起,她的眼睛闪闪发亮,说:“我也爱你们。你们对我太好啦!”她把我们放开,说:“我以后听你们的话,好吧?快去看看牛吧。” 

我们赶快穿上凉鞋去找牛,牛已经走得很散了,好不容易才把它们赶回来。我们赶着牛回来时她已经站起来了,一瘸一拐地要来帮忙。我冲她喊:“你别来啦,我们两个人够了。” 

她就拿起衣服一瘸一拐走到树林里去换。后来她出来,我们拉来一条牛让她骑,大许把东西收拾起来,我赶着牛慢慢地朝回走。牛吃得肚皮滚圆,一出树林就呼呼呼地冲下山去,直奔我们队,也不用赶了。就这样到家天也快黑了。队长在路口迎着我们,他笑嘻嘻地说:“辛苦了!牛肚子吃得挺大。你们把牛赶到晒场上圈起来吧,牛圈叫营部牛帮占了。” 

我们就把牛赶到晒场上去。晒场有围墙,进口处还有拦牛门,是为了防牛吃稻谷的。晒场北面是凉棚,头上有一间小屋,原是保管室,后来收拾出来,供教导员来队住。我们把牛赶进晒场,忽然发现北边空场上有汽灯光,还有一个公鸭嗓在大声大气地说话。教导员来啦。我们站在空凉棚里,不由地勾起旧恨:这就是我们当初挨斗的地方!我和大许走到教导员住的屋门前,一推,门呀的一声开了。划根火柴一看,哼,他的床铺好干净。我知道有几个女生专门到他屋里做好事,每天他回来时屋里都收拾得干干净净。现在就是,床铺收拾好了,洗脸水也打来了,毛巾泡在水里,牙膏也挤在牙刷上了。我和大许笑着跑出来。小红走过来问:“怎么啦?”我们告诉她,她也笑起来。忽然她心生一计:“我们也对教导员表示一下敬意,对!我们拣两头肚子吃得最大的牛赶到他屋里去。” 

我们俩一听,憋不住地笑。可真是好主意,他的门又没插,牛进去就是自己走进去的。我们找了两头吃得最饱的牛。啊,这两个家伙吃的肚子都要爆炸了,那里边装的屎可真不少啊!可以断定两个小时之内它们会把这些全排泄出来,我猜有两大桶,一百多斤。我们把它们轰起来,一直轰到小屋里。不一会儿,我们就听见屋里稀里哗啦地乱响起来,简直是房倒屋塌!后来就不响了。我猜它们在那么窄的房子里不太好掉头,它们也未必肯自己走出来。我们都走了,回去弄饭吃。吃完了饭我们坐下来聊天,还泡了茶喝,就等着听招呼。可是教导员老说个不停,我们都挤到窗口看他。会场就在我们门前。我们数着人。—会溜了一个,一会又溜了一个,一个又一个溜了一半啦。教导员宣布散会,他也打了个大呵欠。我们看见他转过屋角回去了。大许说:“好呀,这会儿牛把屎也拉完了。”我们就坐下等着。过了一会儿,就听见远远的教导员一声喊叫。他叫得好响,隔这么老远都能听见。我们三个全站起来听,憋不住笑。后来就听见他一路叫骂着跑到这边来,他说:“谁放的牛?谁放的牛?怎么牛都关在场上?” 

我们三个推开门跑出来站在走廊上,小红说:“我们放的牛怎么啦?教导员。” 

他一跳三尺高,大叫起来:“牛都跑到我屋里来了!谁叫你们把牛关在场上的?” 

我们七嘴八舌地说:“牛进屋了?那可好玩啦!”“你怎么没把门锁上呢?”“牛是冯队长叫关在场上的。牛圈叫营部牛帮占了!”后来我们仔细一看,教导员的额头上还有一条牛粪印,就哈哈大笑起来。教导员大骂着找队长去了。小红大叫一声:“去看看!”她撒腿就跑,大许也跟去了。我把我们的马灯点上,也跟着去了。 

啊哈,教导员屋里多么好看哪!简直是牛屎的世界!那两个宝贝把地上全拉满了,连个落脚的地方也没有。牛尾巴把粪都甩上墙了!桌子也撞倒了。煤油灯摔了个粉碎,淹没在稀屎里,脸盆里的水全溢出来啦,代之以牛屎,毛巾泡在里面多么可笑啊!教导员挂在墙上的衣服、雨衣、斗笠全被蹭下来了,惨遭蹂躏,斗笠也踏破了。我们站在那儿笑得肚子痛,小红还跳起来拍手。一会儿教导员拉着队长来了,他一路走一路说:“你来看看!你来看看!我进屋黑咕隆咚,脸上先挨了一下,毛扎扎的,是他娘的牛尾巴!我还不知是什么东西,吓得我往旁边一躲,脚下就踏上了,稀糊糊、热呼呼的,这还不够吓人!屋里有两个东西喘粗气!我吓得大喊一声:谁!!这两个东西就一头撞过来,还亏我躲得快,没撞上。冯队长,这全要怪你,你怎么搞的!” 

队长一路赔情,到屋里来一看,嘻!他也憋不住要笑。他说:“小王、小许、小邢,快帮教导员收拾一下嘛!”我们不去收拾,反而笑个不住。小红说:“队长,又要派我们出牛圈哪!我们干够了!”于是我们笑着跑开了。 

唉,这都是好多年以前的恶作剧了,可是我记得那么清楚。我常常一个细节一个细节地回忆,一切都那么清晰。我那时是二十一岁,大许和我同岁,小红才二十岁。人可以在那么年轻时就那么美,那么成熟,那么可爱。她常说她喜欢一切好人。她还说她根本分不清友谊和爱的界限在哪里。她给我们的是友爱:那么纯洁、那么热烈的友爱。她和我们那么好,根本就不避讳她是女的、我们是男的。我们对她也没有过别的什么念头。可是她给我们的还不止这些。我回想起来,她绝对温存,绝对可爱,生机勃勃,全无畏惧而且自信。我从她身上感到一种永存的精神,超过平庸生活里的一切。 

我们都学会了她的口头禅:管牛叫该死的,管去游泳叫去玩呀,她还会说:嘿,真要命。或者干脆就说:要命。她的记性好极了,看书也很快。有时候她和我们讨论一些有关艺术哲学的问题。我发觉她想问题很深入,她的见解都很站得住。她爱艺术。她说:“有一天我会把我的见解整理出来的。”可惜她没有来得及做这件事。她病了。 

有一天中午,我们在屋里看书,看着看着她把书盖在脸上。我们以为她睡了,于是蹑手蹑脚地走出去。过了半个小时,上工哨响了,我们回来。她把书从脸上拿起来,我发现她脸色不好看,而且眼睛里一点睡意也没有。我问她:“小红,你怎么啦?你气色不好。” 

她说:“我看着看着突然眼花起来,觉得脑后有点儿凉。大概是这几天睡得少了吧。” 

我说:“那你不要去了,倒半天休吧。”她说:“好”,就让我去和队长说。下午我们回来的时候看见她高高兴兴地坐在走廊上给我们洗衣服,还说:“你们到屋里去看看。” 

我们进屋一看,她把屋里的布置改了,还把我们的一切破鞋烂袜子全找了出来,可以利用的全洗干净补好了。屋里也干净得出奇。她悄悄地跟了进来,像小孩子一样欢喜地说:“我干得棒吧?” 

我说:“很棒!你睡了没有?” 

她笑着说:“睡了一个小时。然后我起来干活。” 

大许说:“你该多睡会儿,等我们回来一块动手那要快多啦!你好了没有?” 

她说:“我全好啦,我要起来干活。我是劳动妇女。” 

我们觉得“劳动妇女”这个词很好玩,就笑了半天,以后有时就叫她劳动妇女。可是当天晚上她又不好,说是“眼花,头痛”。我一问她,原来这毛病早就有了,只是很少犯。于是我们叫她去看病。星期天我们陪她到医院去,医生看了半天也说不出个名堂来,给了她一瓶谷维素,还说:“这药可好啦,可以健脑,简直什么病都治!”我们买了一些东西回来,走到大河边上,她看见河水就高兴了,她说:“我们膛过去!”我说:“你得了!好好养着吧!”她笑了。于是我们走桥过去。那座桥是竹板架在木桩上搭成的,走—亡去“吱啦吱啦”响,桥下边河水猛烈地冲击桥桩,溅起的水花有时能打上桥来。我走在前面,她在中间,她一边走一边笑嘻嘻地说:“我需要养着啦,都要我养着啦。水真急……”忽然她站住了,说:“小王,你走慢一点!”我站住了。她橐橐地走了几步,一把抓住我肩头的衣服,抓得紧极了,我感觉她的手在抖。我觉得不妙,赶快转过身来扶住她。我看见她闭着眼睛,脸上的神情又痛苦又恐慌。我吓坏了,对她说:“你怎么啦!是不是晕水了?你睁开眼往远处看!”人走在急流的桥上或者蹚很急的水,如果你死盯住下面的浪花有时会晕水,这时你就会觉得你在慢慢地朝水里倒去。这个桥很窄,桥上也没有扶手,有时可以看见在桥头上的人晕水趴下爬过去。我才来时也晕过一次,所以我问她是不是晕水了。这时大许也从后边赶上来,我们俩扶住她,她像一片树叶一样嗦嗦地抖,她说:“我头疼,我一点也看不见了……你们快带我离开这桥,我害怕呀!我怕……”她流了眼泪。我们赶紧把她抬起来,她用双手抱住头哭起来。过了河,我们把她放下,她躺在草地上抱着头小声哭着说:“我头痛得凶。刚才过河的时候突然眼就花了,眼前成了一大片白茫茫的雾,接着就头痛……你们快带我回家,我在这儿害怕,我心里慌。” 

我赶快抱起她往家里跑,她一路上抱着头,有时她又紧抱住我,把头紧贴在我胸前,她不仅痛苦,而且恐惧。看见她跟痛苦与恐惧搏斗,我们都吓坏了。半路上大许替换了我,她一察觉换了人就恐慌地叫起来:“你是谁?你说一句话。”大许说:“是我,小红,是我。”她就放了心,又把头贴在大许胸前。 

我们急如风火地奔回家,把她放在床上,我奔出去找卫生员。我一拉门她就恐慌地叫:“你们别都走了呀!”大许说:“我在呢,我在呢。”他握住她的手,她才安静下来。 

我把卫生员找来,她根本就没问是什么病,就给她打了一针止痛针,小红一会儿就不太痛了。后来她睡了。我们给她打来了饭,可是我们自己却没有吃什么。天很快就黑了。我们给她把蚊帐放—F来,在窗上点起了煤油灯。我们又害怕空气太坏,把前后窗户全打开了。我和大许蜷坐在床上,谁也没有睡。这真是凄惨的一夜!我们谁也没说话。窗前经常有黑影晃动,我也没去管它。后来才知道和邢红住在一起的女生发现她没回去睡,就悄悄地叫起几个人准备捉奸。她们准备灯一灭就冲进来,可是灯一直没灭,她们也就没敢来。谢天谢地她们没来,她们要是闯进来,很难想像我和大许会做出什么举动。我们的窗台上放了一把平时用来杀鸡、切菜的杀猪刀,当时我们肯定会想起来用它。要是出了这种事,后果对大家都是不可想像的。 

到天快亮的时候小红醒了。她在蚊帐里说,“小王、大许,你们都没睡呀?” 

我们走过去问她:“你好一点没有?” 

她笑着说:“好一点?我简直是全好了。我要回去睡了。” 

我们说:“你别走了,就在这儿好好睡吧,天马上就要亮了。你到底是怎么了?” 

她说:“嘻,过河的时候头猛然疼起来了。我猜这是一种神经性的毛病。没什么大不了,你们别怕!” 

我不信,说:“恐怕没你说的那么轻巧。你说害怕,那是怎么啦?” 

她好半天不说话,后来说:“头疼的时候我心里特别慌,也不知为什么。”她不好意思地笑了一声,然后说:“我有一种不好的感觉……不说啦,不说啦!” 

我说:“为什么不说?你的病可能很重。告诉我们,到底是怎么回事?” 

她接下去说,说着说着声音忧郁起来:“我感到疼痛不是从外边来的,是从里边来的。也可能是遗传的吧?你别吓唬我了,人家自己就够害怕的啦!” 

我们都不作声了。后来大许说:“你应该去看病,要争取到外边去看。一定要把病根弄明白,一定要。” 

她说:“没那么厉害,也许是小毛病。干吗兴师动众?我要去看病你们要陪着我。我不去。” 

我们说非去不可,不然我们不放心。后来她就答应了,不过说她不要我们陪着去。第二天我们下地,中午回来时她还没去医院,反而起来给我们弄了一顿饭,做得香极了。她拍着手叫我们来尝。可是我们板着脸上伙房打了饭来,不和她说话,低头吃起来。她不高兴了,说:“你们不吃我做的饭呀?” 

我白了她一眼说:“叫你去看病,谁叫你做饭?说好的事情你不干。” 

她愣了一会儿,就哭了:“你们怎么啦?这么对付我?人家下午去看病就不行吗?我比你们小,我是女孩子,你们就这么对付我呀……” 

我们赶快把饭盆放下过去哄她,后来她不哭了,后来又笑了。她噙着眼泪说:“我一定去看病,可是你们一定要吃我做的饭。我做得得意极啦!你们要是不吃我就不去看病,就不去!” 

于是我们坐下一起吃她做的饭,她又说:“以后不带这样的啦,两个人合伙给一个人脸色看。” 

我说:“为了你好还不成吗?” 

“不成,就不成。你不知道吗?你不管叫别人做什么事,不光是为了他好,还要让他乐意。这是爱的艺术。要让人做起事情来心里快乐,只有让人家快乐才是爱人家,知道吗?” 

我们俩直点头。我们把她做的饭大大夸奖了一番,而且是由衷的夸赞,她高兴了。下午上工前我们把她送到桥边。收工的时候她已经回来了,坐在走廊上,刚洗了头,看样子很高兴。 

我们问她:“查出什么病了吗?” 

她说:“可以说查出来了。俞大夫给我看的,她说很可能是青光眼,让我去眼科看。眼科张大夫出差了,家里只有个转业大夫,我听人说他在部队是个兽医。他给我看了半天,什么毛病也没看出来,给了我一大堆治青光眼的药。我就先用这些药吧。”我们以为这就是正确的诊断,就放心了。 

大夫给她开了假,她就在家里休息。我们去干活,她在家里给我们做家务事。可是她的头痛病用了青光眼的药一点不见好,反而常犯,她渐渐的也不太害怕了。等张大夫出差回来我们又陪她去看,张大夫马上就把她的青光眼否定了,又转回内科。内科看不出毛病来,就让她住院观察,她简直是绝对不考虑。我们说破了嘴皮,举出一千条论据也说服不了她。最后我们提出威胁:如果她回去,我们谁也不理她;又许下大愿:如果她留下,我们每天都来看她。经过威胁利诱,她终于招架不住了,答应住院,不过要我们“常来看她,但是不要每天都来”。我们留下她,回去了。每天下工以后我们收拾一下,就到医院去看她。我们那儿到医院有八里路,四十分钟可以走到。她看见我们很高兴,有时候还到路上迎接我们。有时候下午她就溜回来在家里等我们,做好了饭,躺在我床上看书。她老说她不愿意住院,她想回来就不走了,可是我们当晚就把她押送回去。星期天她是一定要溜回来的。不过她的病可越来越坏,她的头痛发作得越来越频繁,面色越来越苍白,人也瘦了。她还是那么活蹦乱跳,可是体力差多了。我们心里焦虑极了,我们俩全得了神经衰弱,一晚上睡不了几个小时。我们什么书也不看了,只看医书。医院的大夫始终说不清她是什么病。 

有一天我看到她呕吐,我马上想到,她患的是脑瘤。我问她吐丁多久了,她说:吐过两三次。我马上带她去找俞大夫,说:“她最近开始呕吐,会不会是脑瘤?”俞大夫说:“不会吧,她这么年轻。”我说:“大夫,她老不好,这儿又查不出来,好不好转到昆明去看看?”俞大夫假作认真地说:“我也在这么考虑。” 

小红这次没有闹脾气,她服从了理智。也许她也感到她的病不轻。我和大许到处催人给她办转院手续,很快就办好了。大许去县城给她买汽车票,我和她回队去收拾东西。她打开箱子把换洗的衣服拿出来放到手提包里,有点忧伤地说:“我这次去的时间会长吗?” 

我说:“也许会长的。小红,你病好以后争取转到北京去吧!你以后身体不会像以前那么好丁。你应该回家。” 

她一把抓住我的手,双眼紧张地看着我说:“你们不喜欢我了么?为什么这么说?为什么要我离开?”她眼睛里迅速地泛起泪水。我轻轻拍拍她的肩膀说:“你别紧张呀,别紧张。我们也会回去的,我们会找到你。我们三个人会永远在一起生活。” 

她想了一会儿,自言自语地说:“真的,我病了,我想家。家里有妈妈,有哥哥,他们知道了会想我。这儿有你们。我能离开家,可是离不开你们。你们应该和我一起回我家去。没有你们我不走!”忽然她伏到我肩上痛哭起来:“我觉得病重了!也许不会好,也许我会变成个大傻子。”我心里十分酸楚,可是我尽量克制地说:“不会,不会。小红在瞎想,小姑娘瞎想,我求她别乱想了,我求她别哭了!”可是她伏在我肩上,纵情地说出好多可怕的想法:“我得的很可能是脑瘤。他们要给我开刀,把我头盖骨掀开,我害怕!”她蜷缩在我怀里小声说:“他们要动我的脑子,可是我就在那儿思想呀,他们要在我脑子上摸来摸去。弄不好我就要傻了!再也不会爱,也说不出有条理的话,也许,连你们都认不出来。我可真怕……”我听得心惊肉跳,好像这一切我都看见了。我叫她别说了,我说这都不可能,可是泪水在我脸上滚,滴到她耳朵上。她觉察了,跳开来看我。她掏出一块手绢擦掉眼泪,又来给我擦眼泪,她慢慢地笑了,先是勉强地笑,后来是真心地笑。她说:“我高兴啦!你也高兴吧。什么事也没有。我有预感,什么事也不会有。我会好好的。高兴吧!”她开始活泼起来,快手快脚地收拾东西,然后快活地说:“我刚才冒傻气了,我冒傻气。你什么也别跟大许说。” 

后来大许回来,她始终很高兴。第二天我们送她上公路。她高高兴兴地跳上汽车,在里面笑着对我们挥手,还临时编出个谎来,对我们说:“大哥、二哥,我很快会回来的!” 

我说:“治好病回来。” 

她说:“当然,当然,治好病回来。”汽车开动了,她又探出头宋喊:“我好了咱们玩去啊!” 

我们挥着手追着汽车跑,喊着:“再见,小红!” 

她也喊:“再见!再见!” 

我们在家里等她来信。我们焦虑不安地等着她的来信。我和大许话都少了。每天我们去干活都感到很不自然,好像少了一只手,或者少丁一半脑子。每次回到家里,我都产生一种冲动,要到病房去问候小红,或者茫然地收拾起东西来想到那儿去看她。晚上坐在屋里,我们不看书,连灯也不点。我们在黑暗中直挺挺地坐着,想着小红。后来她来信了,她——到昆明就写了信,可是信在路上走了五天。她说她一到昆明就住进了医院,医院里条件很好。她高高兴兴地把大夫和护士一个一个形容了一遍,然后说,马上要给她做血管造影了,是不是脑瘤做了以后就可以知道。到后来她的字迹潦草起来。她说:“我一个人很寂寞。我很想你们,很想很想很想。有时候我想溜回去,不治病了,又怕你们骂我。要是有可能的话,你们来看我吧!哥哥们,来吧!”她哭了,哭得信纸上泪迹斑斑。最后她又高兴起来,不过可以看出是装的,她说昆明这地方很好玩,医院里也很好玩,让我们别为她担心,她很高兴,病好了就回来。最后她很高兴地写上了“再见”。 

我们把信看了又看,忽然我想到我们都有两年没探亲了,可以请探亲假。对了,太棒了!这回教导员也捣不了鬼,探亲假是有条例规定的。我们两个飞奔到连部去请假,队长马上就批了我们俩假。我们马上到营部去办手续,结果碰上了教导员。他拿过队长的条子,阴阳怪气地说:“你们都是连里的壮劳动力呀。一下走两个是不是太多?一个一个走吧!回来一个再走一个。”这家伙多缺德!咳呀,去你的教导员!我们一个一个走好了。重要的是要有一个人去安慰我们的小红。我先走,一个月以后回来,大许再去。我们谁也不打算回家,就想到昆明去陪着她。我就要走了,又接到她的信。她抱怨说:血管造影好难受啊,然后说脑瘤已经确诊了,只是长的位置不好,昆明的医院不敢动,所以给她转到北京的医院,她已经买好车票,就要走了。她让我们想办法到北京来,她也想到我们可以请探亲假。她说:“我想起来啦,你们可以请探亲假!我一想到这个心里就安静多啦。我们一起回家去。” 

我赶紧动身。大许写了信交给我。我乘汽车走了。分手的时候关照大许要经常写信。 

在路上我遇上一些不顺利:在保山等了两天车,在昆明又买不到直达的火车票。结果用了半个月才到北京。北京当时寒风刺骨。我下了车就直奔小红家:他爸爸、妈妈,还有哥哥都在。他们家看来是个高级知识分子家庭,家里书很多,她爸爸是个秃顶的小老头,人很开通,妈妈也很好。她哥哥挺像她,我一见了就喜欢。我一下闯进去,他们都吃了—惊,问:“你是谁?你找谁?”  

我说:“我是邢红的同学,我姓王,从云南来……她现在在哪儿?” 

他们马上就知道了:“噢!你是小王。她常念叨你。小红在医院里,她才动了手术。手术很顺利,瘤子在做切片。请坐吧!我们正要去看她。” 

我也没有坐,立即同他们一起到医院去看小红。她脸色苍白,瘦多了,可是一看见我就猛坐起来,高兴地大叫:“小王,你来啦!我等你等坏了。我接到大许的信了,我一直在等你。我动了手术了,我就要好了!” 

后来我就天天陪着她,那会儿医院也乱,什么探视不探视的,我每天都很早就来,很晚才走。她的身体渐渐好起来,常常要我陪着她到院子里走动。才来的时候我特别迂,连给她剪趾甲都不好意思,后来我也不怕了。我常常给她裹好大衣,搀着她到院子里去。护士们有时瞎说,说这小两口多好,我们也不理她们。 

我走的时候天气开始暖和了,小红的身体也更好了。可是我发现她爸爸和妈妈神色都不正常。但没有放在心上。我懂的事情太少,一点也不知道切片有什么重要性,我只看见她好了。大许又偷偷来信催我回去,他要来。于是我就回去了。小红的哥哥送我上火车,他心情不好。我问他怎么啦,他说是他自己的事儿。我开头一点儿也没疑心,可是火车开走的时候他忽然扶住柱子痛哭起来。这不由我不起疑。 

果然,我回到云南以后,大许正准备动身,我们忽然收到小红一封信。她说她的病重了。病得很厉害,也许不会好了。她说,她感到出了大变故,很可能瘤子是恶性的,它还在脑子里。这真是当头一盆凉水!我们全都呆若木鸡。小红叫大许快点去。我们拿出全部积蓄,还借了一些钱,央求团里开了一张坐飞机的证明,让大许飞到她那儿去。我让大许到了北京马上打个电报来。大许慌慌张张地走了。 

大许走后有七八天音信全无!我急得走投无路。晚上睡不着觉,用手抓墙皮,把墙掏破了一大块。第八天大许来了一个电报:已到京小红尚好信随后到。我心里稍稍安定。 

后来大许来了信,他说小红开始经常头痛,痛得让人害怕。她已不能吃饭,全靠打点滴维持。有时候眼睛看不见。大许痛心地描写她一看见他怎么像往常一样笑了,高兴地抱住他脖子。她让大许告诉我,她想我想得要命。她说她在昏睡的时候可以听见我的声音。她说她很想很想让我们三个在一起,三个人在一起她死也不怕了。她还说她虽然可以笑,可以说话,可是意识深处已经有点昏乱。她说她怕这种死,从内部来扼杀她。我看了这信差一点疯了。我写信让她、求她、命令她坚强起来,坚持住一点也不退让。我求她拼命去和疾病争夺,为我们三个争夺,一定要保住什么。我说:“千万千万别失望,还有希望。你还年轻,你的活力比十个人的都多。你能胜利,我知道你能胜利。想一想我们还可以永远在一起生活!” 

我不记得那些天是怎么过的了。后来大许又来一封信,说大夫试了一种新药,小红好多了,眼睛也可以看清了。她看了我的信,很高兴。她成天和大许说话,说她头疼比以前好了,头脑也清楚了。还说他们两人成天谈论我,小红说我是个最好的人。小红不住地说起我的细节,我是怎么笑的,她说我有一种笑很有趣:先是要生气,嘴角往下一耷拉,然后慢慢地笑起来。她还说我有二-种阴沉的气质,又有一种浪漫的气质,结合起来可好了,她特别喜欢。她说我可以做个艺术家。 

信的末尾小红写了几个字:“王,我爱你。你的信我很喜欢。我要为咱们三个人争夺。一直要到很久很久以后,你还会叫我小姑娘。”她能写信了!尽管字迹歪歪斜斜,可是很清楚。 

我看了信高兴极了。 

后来又来了一封信。大许说:小红的病情急转直下,忽然开始昏迷,要输氧气。他日夜陪伴着她。他说他都快傻了,他的字迹行不成行字不成字,有几个地方我看不懂。最后他说:还有希望,只要她活着就有希望,希望很微弱,可是会大起来。医生说没希望,可他们是瞎说。 

过了一天大许又来一封信,他说:“昨天她清醒了一会儿,可是什么也看不见,眼前漆黑。我把你的信念给她听,后来她把信拿过来贴在胸前。她说,我要去了。我只为你们担心。要去的人只为留下的人担心,她是什么也不怕了。我求她别说下去,她的声音就低微下去。昨天夜里她很不好,可是她挺过来了。小王,还有希望吗?还有希望吗?” 

我简直狂乱了,后来我接到一封信。信里封了一张电报纸,大许写道:“小红已去世。她的最后一句话是让我们节哀。我即回来和你在一起。许。” 

我看了这些话发出一声长嚎,双手乱抓了一阵。我感到脑后一阵冰凉。我坐了很久,天黑下来,又亮起来。我机械地去吃饭,又机械地去干活,机械地回家来。我很孤独,真正的哀痛被我封闭起来了,我什么也不想。直到有一天下午大许推开我们的屋门,把夕阳和他长长的身影投进来。 

我站起来,我看见大许的头发白了不少,他黑色的头发上好像罩了一层白霜。我扑过去拥抱他。一个阀门打开了。一切都涌上来。我们大哭,然后我们并排坐下来哭泣,小声地啜泣。大许挂着黑纱,他瘦了。他站起来从提包里拿出一个黑漆的小盒子放在我床上。我用眼光问他,他艰难地说:“小红留下遗言,她把骨灰分留给家里和我们。这就是她。” 

我感到颈后好像挨了重重一击。我跪倒下来,用痉挛的手指抓住盒子,抚摸盒子。我在哭吗?没有声也没有泪,只有无穷的惨痛从粗重的喘气里呼出来,无穷无尽。 后来我和大许在一起过了两年,就分开了。我们把小红最后几封信分了。他要走了小红的遗骨,把她的箱子和衣物留给我。我们把小红留下的书分开,一人拿了—半,然后收拾好行装,反锁上房门。我们离开那里,走向新的生活。
