\chapter{夜行记}

  玄宗在世最后几年,行路不太平。那年头出门在外的人无不在身上怀有兵刃。虽然如此,见到路边躺着喂乌鸦的死人,还是免不了害怕。一般人没有要紧的大事,谁也不出门,大路上因此空空荡荡。有一天,一个书生骑着骏马,押着车仗,在关中的大道上行走。那时候正值夏日,在马上极目四望,来路上没有行人,去路上也没有行人,田野上看不到农夫,只有远处地平线上空气翻滚,好像无色的火焰。车轮吱吱响,好像在脑子里碾过。书生在马背上颠簸,只觉得热汗淋漓,昏昏沉沉。旅行真是乏味的事,如果有个人聊聊就好了。书生不想和车夫谈话,因为他们言语粗鄙,也不想和轿车里的女人谈话,因为她们太蠢了。因此他就盼着遇上个行人,哪怕是游方的郎中,走方的小炉匠也好。可是从上午一直走到下午,谁也没遇上。直到夕阳西下,天气转凉时,才遇上一个和尚。 

  和尚骑着骡子,护送着一队车仗。轿车里传出女人的笑语,板车上满载箱笼。虽然书生盼望一个谈伴,这一位他可不喜欢。第一,和尚太无耻,居然和女人同行。第二,和尚太肥,连脑后都堆满了一颤一颤的肥肉。因为和尚不留头发,这一点看得十分清楚。等了一天,等来这么一个人,不是晦气么?等到彼此通过姓名,书生就出言相讥,存心要和尚难堪: 

  “大师,经过十年战乱,不仅是中原残破十室九空,而且人心不古世道浇漓。我听说有些尼姑招赘男人过活,还听说有些和尚和女人同居。生下一批小娃娃,弄得佛门清净地里晾满了尿布,真不成体统!” 

  和尚虽然肥胖,但却一点也不喘,说起话来底气充足,声如驴鸣:“相公说的是!现在的僧寺尼庵,算什么佛门清静?那班小和尚看起女人来,直勾勾地目不转睛。老衲要出门云游,家眷放在寺里就不能放心,只得带了同行。这世道真没了体统!” 

  书生想:这和尚恁地没廉耻!我不要他同行。此时太阳已经落山,前面是个市镇。书生说:“大师要住宿吗?这里有好大客栈,正好住宿!” 

  “依相公说,我们就住宿。” 

  “大师宿下,我们乘晚凉再行一程。” 

  “那就依相公说,我们再行一程!” 

  “大师要宿,我们便行。大师要行时,我们就宿。” 

  “相公,正好要说话,怎么撇了开?相公要宿,我们也宿,相公要行,我们也行!” 

  书生听了又好气又好笑,真想骂他一声。但是没有骂,只是想:和尚要同行,也由他。车马行过市集,走上山道,太阳已经落山,一轮满月升起来,又大又圆,又黄又荒唐。月下的景物也显得荒唐。山坡上一株枯树,好像是黑纸剪成。西边天上一抹微光中的云,好像是翻肚皮的死鱼。马蹄声在黑暗中响着,一声声都很清楚。和尚的大秃头白森森,看上去令人心中发痒。书生真想扑过去在上面咬一口。当然,这种事干不得。和尚要问:好好地走路,你啃我干什么?书生又想:捡块石头开了他的瓢儿也能止痒。这种事也干不得。和尚在喋喋不休,听了他的话,书生心里痒得更厉害。和尚在谈女人,谁能想象佛门子弟会说出这种话来? 

  和尚说:安南的女子娇小玲珑,性情温柔,拥在膝上别有一番情趣;鲜卑女子高大白净,秀颈修长,最适于在榻上玉体横陈;东瀛的少女深谙礼节,举止得体,用做侍婢再合适也没有;西域的蛮女热情如火,性欲旺盛,家里有一个就够,万不能有两个。谈到中国女人,和尚认为三湘女子温柔,巴蜀女子多才,陇西的女子忠诚,关中的女子适合当老婆。天下只有燕赵的老婆最要不得,因为完全是母老虎。听到最后一句话,书生有点上火,因为他老婆是河北人。于是他接口说道,现在的女人都不成体统,遇上谁就和谁过,也不管他是和尚道士,头上有毛没毛。关于这一点,和尚说不能怪女人。这些年来先是安史之乱,后来又边乱纷纷。天下男了去了十之八九,女孩子却还得嫁人。所以,嫁个和尚也不错。听了这种话,书生差点笑出来,这个和尚有趣得紧啦! 

  和尚说,谈女人无趣,不如来谈骑射。书生听了心里又发痒——出家人谈谈击鼓撞钟、敲木鱼念经也罢,他偏要谈跑马射箭!不过这是书生心爱的话题,虽然对着一个和尚,他也禁不住发言道:习射的人多数都以为骑烈马,挽强弓,用长箭,百步穿杨,这就是射得好啦。其实这样的射艺连品都没有。真正会射的人,把射箭当一种艺术来享受。三秋到湖沼中去射雁,拿拓木的长弓,巴蜀的长箭,乘桦木的轻舟,携善凫的黄犬,虽然是去射雁,但不是志在得雁,意在领略秋日的高天,天顶的劲风,满弓欲发时志在万里的一点情趣。隆冬到大漠上射雕,要用强劲的角弓、北地的鸣镝,乘口外的良马,携鲜卑家奴,体会怒马强弓射猛禽时一股冲天的怒意。春日到岭上射鸟雉,用白木的软弓,芦苇的轻箭,射来挥洒自如,不用一点力气,浑如吟诗作赋,体会春日远足的野趣。夏天在林间射鸟雀,用桑木的小弓小箭,带一个垂发的小童提盒相随。在林间射小鸟儿是一桩精细的工作,需要耳目并用,射时又要全神贯注,不得有丝毫的偏差,困倦时在林间小酌。这样射法才叫做射呢。 

  和尚说,看来相公对于射艺很有心得,可称是一位行家。不过在老僧看来,依照天时地利的不同,选择弓矢去射,不免沾上一点雕琢的痕迹。莫如就地取材信手拈来。比如老僧在静室里参禅,飞蝇扰人,就随手取绿豆为丸弹之,百不失一,这就略得射艺的意思。夏夜蚊声可厌,信手撅下竹帘一条,绷上头发以松针射之,只听嗡嗡声一一终止,这就算稍窥射艺之奥妙。跳蚤扰人时,老僧以席蔑为弓,以蚕丝为弦,用胡子茬把公跳蚤全部射杀,母跳蚤渴望爱情,就从静室里搬出去。贫僧的射法还不能说是精妙,射艺极善者以气息吹动豹尾上的秋毫,去射击阳光中飞舞的微尘,到了这一步,才能叫炉火纯青。 

  书生听了这些话,把脸都憋紫了。他想:幸亏是在深山里说话,没人听见,否则有人听了去,一定要说这是两个牛皮精在比着吹牛皮。倘若如此,那可冤哉枉也!我那射雁、射雕、射雉、射雀,全是真事儿,不比这秃驴射苍蝇、射蚊子、射跳蚤,纯是信口胡吹。别的不要说,捉个跳蚤来,怎么分辨它的牝牡?除非跳蚤会说话,自称它是生某某或者妾某某。纵然如此,你还是不知道它是不是说了实话,因此你只能去查它的户籍——这又是糟糕,跳蚤的户口本人怎能看见?就算能看见;人也不识跳蚤文。所以只好再提一个跳蚤当翻译。你怎么能相信这样的翻译?跳蚤这种东西专吸人血,完全不可信。因此分辨跳蚤的牝壮,根本就不可能。和尚吹这样的牛皮,也不怕闪了舌头!想到这些事,书生心里更是奇痒难熬。他真想在和尚的大秃头上开两个黑窟窿,但是他又想,这种事儿可干不得。和尚的老婆在一边看见,难免要责怪于我。 

  书生抬头一看,发现已经走到深山里。和尚哈哈大笑,说走夜路有人谈话,真真是有趣。我们不如叫家眷车仗先行,自己在后面深谈。书生点点头,心里说:这样好多啦!我要是憋不住了,没人看见正好揍你。于是他们站在路边,让车辆到前面去。 

  此时月亮已经升到中天,山里一片银色世界。坡上吹着轻轻的风,又干净,又明亮,好像瓦面上的琉璃。月光下满山的树叶都在闪亮,在某些地方晃动。在另一些地方不晃动。书生想,这真是个漂亮的世界。老天保佑,我可别干什么不雅的事情。等到心里的奇痒平息,他就随和尚走去,继续谈到很多事情。 

  和尚说,谈过了骑射,我们来谈剑术。这也是书生心爱的话题,所以他就抢先发言道:百炼的精钢,最后化为缠指之柔。他有柄这种钢打制的宝剑,薄如蝉翼,劈风无声。不用时,这剑可以束在腰里为带,用时拿在手里,剑刃摇曳不定,就如一道光华。挥起来如一匹白练,刺去时变幻不定。倘若此时此剑在我手里,我只消轻轻一挥,不知不觉之间上人的脑袋就滚到地上啃泥巴,那时您老人家只觉得天旋地转,脸皮在地上蹭得生痛,还想不到是自己的脑袋掉下地了呢。书生说完这些话纵声大笑,心里可有点不踏实。确实有这么一把剑,不过不全是他的。这是他家的传世之宝,他爸爸还没死,这剑不能说是他的。这回出山,身边也没有这柄剑,如若和尚要看,他又拿不出来,这就有吹牛皮之嫌。不过这不要紧,可以请和尚到家里去看。倘若他不肯去,非说书生是吹牛皮不可,正好借这个碴儿和他打一架,不敲出他一头青疙瘩不算完。 

  书生盘算了好多,可是和尚却不来质疑。他说像这样的剑只能说是凡品,虽然在凡品中又算是最上等。如果以剃刀在青竹面上剥下一缕竹皮,提在指间就是一柄好剑。拿它朝水上的蜉蝣一挥,那虫子犹不知死,还在飞。飞出一丈多远,忽然分成两半掉下来。倘若老僧手中有这么一柄剑,只消轻轻一挥、相公不知不觉之中就着了和尚的道儿。你还不知道,高高兴兴走回家去。到晚间更衣,要与夫人同入罗绍帐时,才发现已被老僧去了势。说完了和尚哈哈大笑,书生却气坏了,心说: 

  “你这老贼秃!我不来杀你,已经是十分好了,你倒来取笑我,可是活得不耐烦了?”可是那和尚又说下去: 

  “当然,相公是老僧的好友,和尚绝不会阉了你。老僧这等剑术,在剑客里也只算一般。有一位大盗以北海的云母为刀,那东西不在正午阳光下谁也看不见,砍起人来,就如人头自己往地下滚,真是好看!还有一位剑客以极细的银丝为剑,剑既无形,剑客的手法又快到无影。不知不觉一剑刺在你左胸,别住了心脏不能跳动。登时你胸闷气短,又请郎中,又灌汤药,越治越不灵。此时剑客先生站在一边看热闹,要是他老人家心情好,上前把剑拔去,你还能活。万一他输了钱,你就死吧,到死还以为是自己得了心绞痛!” 

  书生听了这番话,心里又是一片麻痒。这贼秃吹得真是没谱了。试问云母极脆,何以为刀?银丝极柔,又何以为剑?倘若云母、银丝都杀得了人,用一根头发就能把人脑袋勒了去。试问人身子是豆腐做的吗?原来女蜗造人是这么一个过程:她老人家补天之余,在海边煮了一大锅豆浆,用海水一点,点出一锅豆腐来,这就是咱们的老祖宗。女娲娘娘不简单,一只锅里能煮出男豆腐和女豆腐,两块豆腐一就合,就生下一个小豆腐?真他妈岂有此理。玉皇大帝坐在九天之上,阎罗大帝坐在冥罗地府,主管人的福禄生死,原来是两家合资开了个豆腐坊。好,太好了!书生悄悄落到后面去,偷手取出弹弓,照和尚脑后一弹弹去。 

  书生的弹弓铁胎裹漆,要是没学过射箭,任凭你有多大蛮力也拉不开。他的弹丸是安南铜铸成,拿在手里不小心掉下去,能把脚砸肿。这一弹要是打在和尚的脑袋上,势必贯脑而出。书生想到和尚正在夸夸其谈,冷不防嘴里钻出个大铜丸,势必要大吃一惊。要是弹丸从眼眶里钻出去,和尚觉得脸上掉下东西,随手一接,接到自己的眼珠子。这种事儿只要没落到自己身上,谁都觉得有趣。书生觉得自己有幽默感,就大笑起来。 

  谁知那和尚吹得高兴,摇头晃脑,那一弹就从他耳边偏过去。书生一看没打中,不禁暗暗心惊。他的准头可以打中三十丈外一个小酒盅,如今打这么大一颗秃头,怎么会打不中?那和尚怎么早不晃头,晚不晃头,偏等他发弹时晃头?莫非这秃头不是吹牛,而是有些真实本领?书生收起弓,赶上去探探和尚的口风: 

  “上人,可听见什么声音?” 

  “噢,一个大屎克螂飞过去,嗡的一声!” 

  书生想:这和尚的耳朵不知是怎么长的,弹丸飞过是什么声音,屎克螂飞过是什么声音?他又觉得这和尚怪可怜的,嘴里谈着出神入化的武功,背后有人暗算,却都不知道。催命的小鬼儿擦耳根子过去,他还以为是屎克螂!让他想去吧,不值当为他说嘴就把他打死。两人又并肩而行,谈到各种武功,说到拳脚棍棒,和尚又有很多说法,就如骑射剑术,都是书生见所未见,闻所未闻,根本无法想象的事。而且他胖乎乎。傻呵呵,月光下一颗大秃头白森森、亮灼灼,让人看了一发忍不住要朝上面下手。 

  此时的月亮比刚才又亮了些。书生心里在大笑,满山的玉树银花仿佛在他身边飞舞。心里想笑,嘴上却不能笑,这可不好受。他想:我要和这位秃大爷谈些悲哀的题目,免得他招得我要打他的秃脑壳。于是他说: 

  “上人,你可知如今路上不太平?现在山有山贼,水有水寇。有些贼杀了人往道边上一扔,那是积德的。有的贼杀法新奇,伤天害理。昨天我们过汉水,车夫见水色青青,就下去凫水。一个猛子扎下去,见到水底下一大群人,一个个翻着白眼儿,脚下坠着大铁球,鼻子嘴唇都被鱼啃了去,那模样真是吓死人!我还听说温州有个土贼专门要把人按在酱缸里淹死,日后挖出来,腌得像酱黄瓜,浑身都是皱。还有人把活人挂到熏坊里熏死,尸首和腊肉一般无二,差点儿当猪卖了出去。现在的人哪,杀人都杀出幽默感来了!” 

  和尚说:“这些小贼的行径,有什么幽默感?我知道洞庭湖上有几位水寇,夜里把客商用迷香熏过去,灌上一肚子铅沙,再把肚皮缝上。第二天早上那人起床,只觉得身躯沉重,拼老命才站得住。在舱里走两步,只听肚子里稀里哗啦,就惊惶失措地跑出去,失足落水,立刻就沉底儿啦。还有几位山贼,捉到客人就分筋错骨大动手术,把双手拧成麻花别在脑后,再把两条腿拧得一条朝前一条朝后。然后把人放出去,那人在山道上颠三倒四行不直,最后摔到山涧里。像这样杀人,才叫有幽默感。” 

  书生想:这和尚有痰气。和你说正经事儿,你只当是胡扯。看来有必要深谈下去,才能激发你的危机感。于是他说:“如今敢出门走路的人也都不简单。这年头儿,出远门儿就如爬刀山下火海,没个三头六臂谁敢出来?所以你看到个走乡的货郎,他可能在腰里挂着铁流星。看到个挑脚的力夫,他袖里可能有袖箭。就是个卖笑的娼妓,怀里还可能有短剑哪!人身上有了家伙,胆就粗,气就壮,在酒楼和陌生人饮酒,一语不合就互挥老拳,手上还戴着带刺的手扣子。在山道上与人争路,气不愤时就抡起檀木棍,打出脑子来就往山洞一扔。只要你敢用白眼瞪我,老子就用八斤重的铁蒺藜拽你,躲得过躲不过是你自己的事,所以如今走路可是要小心。说话要小心,做事也要小心。招得别人发了火,你的脑袋就不安稳。” 

  和尚说:“这样的行路人也只算些胆小鬼,见到发狠的主儿,只能夹屁而逃,只恨爹娘少生了两条腿。你看和尚我,手无寸铁,坦荡荡走遍天下,随身只有一根撒尿的肉棍儿,谁敢来动老子一根毫毛?老和尚吼一声,能震得别人耳朵里流汤。跺跺脚,对面的人就立脚不稳。山贼水寇、见了我都叫爷爷;响马强盗在我面前,连咳嗽都不敢高声。所以我走起路来,兴高采烈,这样出门才有兴致。小心?小心干什么?”   书生一听,心里更麻痒难忍。强盗响马见了你不咳嗽,你是止咳丸吗?我读遍了药书没见有这么一条,秃和尚,性寒平,镇咳平喘,止痰生津,不须炮制,效力如神。是药王爷爷写漏了,还是你来冒充?就算你是止咳九,吃了才能生效,怎么看一眼也管用?你不如去开诊所,让普天下的三期肺痨,哮喘症,气管炎,肺气肿的病号排着队去看你的秃脑袋。吹牛皮不上税,生怕稍有疏漏,吃了小贼的亏,就凭你一个吹牛皮的和尚,走起路来这么舒心。强盗大约是觉得抢和尚晦气,所以放过了你,不过我却放你不过! 

  书生又偷偷落后,拿出弓来。他心里暗暗祷告说:“和尚和尚,你到阴间别怪我。不是我心狠,是你招得我忍不住,我这一弹就把你脑袋打开花,不痛不痒!让你猛一睁眼就换了世界,这也就对得起你啦!”祝祷完毕,他咬紧牙一弹朝和尚打去,这就如案头上砍西瓜,绝无砍不着的道理。 

  书生发弹的时候,和尚刚好走到阴影里。转眼之间他又从阴影里走出来,闪光的秃头还是安然无恙。书生这一惊非同小可,因为他放这一弹时格外的小心手稳,绝无脱靶的可能。看来这和尚不是吹牛皮,而是真有本领。他把弓收起来,打马追上。去,心想不得了,和尚说的全是实话,射蚊子射跳蚤实有其事,云母刀、银丝剑也是真的。和尚确实是止咳丸,也确实有人认识跳蚤文。女蜗娘娘确实在海边点了一锅豆腐,药书上也确实写着秃和尚寒平。这都是从和尚不吹牛推出的必然结论!书生这么一想心里马上乱糟糟。抬头一看前面,书生又禁不住惊叫一声: 

  “大师,我们走迷了!” 

  “迷什么?没有迷!” 

  书生想:这不对。要是不迷路,早该走出山区。可是前面山势更险峻!何况车辆也不见了,这要不是走错路,除非我真的长了一脑子豆腐渣!他说: 

  “大师,我们的车辆也不见了!” 

  “相公,这是去我家的路,老僧一世也没见过比你更有趣的人。所以要请相公到寒寺盘桓几天,宝眷和行李走了近路,现在已经到家了,我和相公走一条远路,意在聆听高论。” 

  书生想,这更是岂有此理!谁要到你家去?我的家眷和行李怎么会到了你家?你请我到你家去做客,我答应了吗?这个秃驴我还是要打死他?女蜗娘娘点豆腐我死活也不信。 

  虽然书生不信和尚的牛皮,他也怕和尚的本领。忽然天上飞过一片黑云,把月亮遮了个严丝合缝。周围伸手不见五指,两个人都勒马不行。和尚还在喋喋不休。书生拿出弓来,朝黑地里发声的地方打一串连环弹,这回就是神出鬼没的黄鼠狼,也逃不开黑暗中袭来的弹雨。最后一弹刚出手,书生就鼓掌大笑起来。 

  忽然和尚一声暴喝:“深山无人,相公这么一惊一乍,可是要吓死老僧?”书生大吃一惊,连忙把弓收起。过了一会,乌云过去,书生看到和尚安全无恙,两个人重新上路。 

  书生心里还在发痒,他真不乐意世界上有和尚这个人。如果世界上存在这和尚,就得相信跳蚤有户口本,人是豆腐做的。这些事一想痒得受不住,所以根本没法相信。但是同样没法相信的事儿已经发生了。今晚用弹子打斗大一个秃脑袋,三番五次打不中。他只顾想这些心事,忽听和尚说: 

  “相公,你的马瘸了,看看它是不是漏了蹄?” 

  书生想:真糟糕,心不在焉,马瘸了都不知道。于是他下马去,把四个蹄子全看遍,蹄铁全是好好的。这却怪,蹄不漏,马怎会瘸?牵着马走几步,发现它根本不瘸。马既然不瘸,和尚怎么说它瘸?再抬头一看,和尚也不见了,书生真的大吃一惊,觉得是遇上了鬼。他上马向前追去,大呼:“上人!上人!等一等!“

  追了十里路,总算追上了和尚。书生长出一口气,两个人并缰行起来,他可没看见和尚瞪起三角眼,面上罩起了乌云。两人各自想心事,再也不交谈。 

  书生忽然想到:和尚没说过跳蚤有户口本,也没说过人是豆腐做的。他只说能识别跳蚤的牝牡,云母银丝也能杀人。既然他没有这么说,我怎么会这么想:这件事细究起来可有趣啦!原来是我非要这么想,好有理由打死他。现在和尚打不死,我可怎么办好?相信跳蚤有户口本,还是相信自己一脑子豆腐渣?他只顾想心事,就没看到月儿西坠,东方破晓,林间展鸟瞅瞅,山谷里起了雾气。他也没看到这条路走也走不完,原来是和尚领着他在兜圈子。忽然和尚把他领进一个山凹,这里有一辆轿车,车夫在辕上打瞌睡。 

  车夫听见马蹄响抬头一看,见到这一增一儒,吓得直翻白眼,这一夜他经过不少惊吓,吓得再不敢说话。和尚说:“相公,宝眷都在这里,我到家去吩咐酒宴,一会儿就回来接你。” 

  书生到轿车前撩开帘子一看,老婆丫环在里面正在熟睡。这些人可享福啦,车一进山就睡着,到现在还没有醒。回头再看和尚,他已经去远了,书生又纵马追上去,这回和尚十分不耐烦。 

  “相公,家眷已经还给你,你还跟着我待怎地!” 

  书生说:“大师,我们还是同行。书生在想些心事,想明了要向大师一诉心曲。” 

  于是这两人又在山路上同行,渐渐走到山顶上去。终于旭日东升,阳光普照,书生勒住马长出一口气说: 

  “大师,我想明白了!” 

  和尚也在想心事,他也勒住马,长出一口气说:“相公,我也想明白了。” 

  书生说:‘大师,小生自幼习武,会些弹术剑法。别人说话不合我心意,我就把他脑袋打开花,叫他说不下去。现在我明白了,这种做法非常之不好。小时候下棋,每到要输时我就把刀拔出来往棋盘上一插,于是长胜不败,结果到现在还是一把屎棋。听人说话也如此,倘若大师说得不对我胃口就把您打杀,怎能够增加见识。比方说,大师若说生姜是树生的果子,我只能说,您说得不对,却不能把大师打死。因为打不死时,我就太难堪了。大师现在活着站在我面前,难道我就因此相信生姜是树上生的?所以杀人不是好游戏,无论如何,不要杀人。” 

  和尚说:“相公,老僧自小习些武艺,专在山道上干没本的生意。和尚虽然抢劫,却不杀人,我专拣相公这样的人同行。你说东,我说西,你说鸡生蛋,我说蛋生鸡。说急了你打我我就露几手把你吓跑,家眷行李就都归我了。现在我想明白了,这种做法非常之不好。就以今晚来说。你打我一弹打不着,两弹打不着,最后打我一串连环弹,你还是不逃走,此时我就太难堪了。你现在站在我面前,难道我就因此一巴掌把你脑袋拍到腔子里?这不好,因为我已经抢了你的行李,又把你打死,实在太凶残。难道我就因此把行李还你?这也不好,因为你已经打了我十七八弹,还是我招着你打的。不抢你的东西,我来挨你打,那不成了受虐狂?所以,抢劫不是好游戏,无论如何,不要抢劫。” 

  这一僧一儒互诉心曲以后,就一起到和尚家里去。和尚要招待书生,把他当成最好的朋友。 
