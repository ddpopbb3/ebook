\chapter{万寿寺(前面部分)}

\section{第一章}

第一节 

莫迪阿诺在《暗店街》里写道:“我的过去一片朦胧……”。这本书就放在窗台上,是本小册子,黑黄两色的封面,纸很糙,清晨微红色的阳光正照在它身上。病房里住了很多病人,不知它是谁的。我观察了很久,觉得它像是件无主之物,把它拿到手里来看;但心中惕惕,随时准备把它还回去。过了很久也没人来要,我就把它据为己有。过了一会儿,我才骤然领悟到:这本书原来是我的。这世界上原来还有属于我的东西──说起来平淡无奇,但我确实没想到。病房里弥漫着水果味、米饭味、汗臭味,还有煮熟的芹菜味。在这个拥挤、闭塞、气味很坏的地方,我迎来了黎明。我的过去一片朦胧…… 

病房里有一面很大的玻璃窗。每天早上,阳光穿过不平整的窗玻璃,在对面墙上留下火红的水平条纹;躺在这样的光线里,有如漂浮在溶岩之中。本来,我躺在这张红彤彤的床上,看那本书,感到心满意足。事情忽然急转而下,大夫找我去,说道,你可以出院了。医院缺少床位,多少病人该住院却进不来──听他的意思,好像我该为此负责似的。我想要告诉他,我是出于无奈(别人用汽车撞了我的头)才住到这里的,但他不像要听我说话的样子,所以只好就这样了。 

此后,我来到大街上,推着一辆崭新的自行车,不知该到哪里去。一种巨大的恐慌,就如一团灰雾,笼罩着我──这团雾像个巨大的灰毛老鼠,骑在我头上,早晨城里也有一层雾,空气很坏,我自己也带着医院里的馊味。我总觉得空气应该是清新的,弥漫着苦涩的花香──如此看来,《暗店街》还在我脑中作祟…… 

莫迪阿诺的主人公失去了记忆。毫无疑问,我现在就是失去了记忆。和他不同的是,我有张工作证,上面有工作单位的地址。循着这个线索,我来到了“西郊万寿寺”的门前。门洞上方有“敕造万寿寺”的字样,而我又不是和尚……这座寺院已经彻底破旧了,房檐下的檩条百孔千疮,成了雨燕筑巢的地方,燕子屎把房前屋后都变成了白色的地带,只在门前留下了黑色的通道。这个地带对人来说是个禁区。不管谁走到里面,所有的燕巢边上都会出现燕子的屁股,然后他就在缤纷的燕粪里,变成一个面粉工人,燕子粪的样子和挤出的儿童牙膏类似。院子里有几棵白皮松,还有几棵老得不成样子的柏树。这一切似曾相识……我总觉得上班的地点不该这样的老旧。顺便说一句,工作证上并无家庭住址,假如有的话,我会回家去的,我对家更感兴趣……万寿寺门前的泥地里混杂着砖石,掘地三尺也未必能挖干净。我在寺门前巡逡了很久,心里忐忑不安,进退两难。直到有一个胖胖的女人经过。她从我身边走过时抛下了一句:进来呀,愣着干啥。这几天我总在愣着,没觉得有什么不对。但既然别人这么说,愣着显然是不对的。于是我就进去了。 

出院以前,我把《暗店街》放在厕所的抽水马桶边上。根据我的狭隘经验,人坐在这个地方才有最强的阅读欲望。现在我后悔了,想要回医院去取。但转念一想,又打消了这个主意。把一本读过的书留给别人,本是做了一件善事;但我很怀疑自己真有这么善良。本来我在医院里住得好好的,就是因为看了这本书,才遇到现在的灾难。我对别的丧失记忆的人有种强烈的愿望,想让他们也倒点霉──丧失了记忆又不自知,那才是人生最快乐的时光…… 

对于眼前这座灰蒙蒙的城市,我的看法是:我既可以生活在这里,也可以生活在别处;可以生活在眼前这座水泥城里,走在水泥的大道上,呼吸着尘雾;也可以生活在一座石头城市里,走在一条龟背似的石头大街上,呼吸着路边的紫丁香。在我眼前的,既可以是这层白内障似的、磨砂灯泡似的空气,也可以是黑色透明的、像鬼火一样流动着的空气。人可以迈开腿走路,也可以乘风而去。也许你觉得这样想是没有道理的,但你不曾失去过记忆──在我衣服口袋里,有一张工作证,棕色的塑料皮上烙着一层布纹。里面有个男人在黑白相片里往外看着。说实在的,我不知道他是谁。但是,既然出现在我口袋里,除我之外,大概也不会是别人了。也许,就是这张证件注定了我必须生活在此时此地。 

2 

早上,我从医院出来,进了万寿寺,踏着满地枯黄的松针,走进了配殿。我真想把鞋脱下来,用赤脚亲近这些松针。古老的榆树,矮小的冬青丛,都让我感到似曾相识;令人遗憾的是,这里有股可疑的气味,于茅厕相似,让人不想多闻。配殿里有个隔出来的小房间,房间里有张桌子,桌子上堆着写在旧稿纸上的手稿。这些东西带着熟悉的气息迎面而来──过去的我带着重重叠叠的身影,飘扬在空中。用不着别人告诉我,我就知道,这是我的房间、我的桌子、我的手稿。这是因为,除了穿在身上的灰色衣服,这世界上总该有些属于我的东西──除了有些东西,还要有地方吃饭,有地方睡觉,这些在目前都不要紧。目前最要紧的是,有个容身的地方。坐在桌子后面,我心里安定多了。我面前还放了一个故事。除了开始阅读,我别无选择了。 

“晚唐时,薛嵩在湘西当节度使。前往驻地时,带去了他的铁枪”。故事就这样开始了。这个故事用黑墨水写在我面前的稿纸上,笔迹坚挺有力。着种纸是稻草做的,呈棕黄色,稍稍一折就会断裂,散发着轻微的霉味。我面前的桌子上有不少这样的纸,卷成一捆捆的,用橡皮筋扎住。随手打开一卷,恰恰是故事的开始。走进万寿寺之前,我没想到会有这么多故事。可以写几个字来对照一下,然后就可认定是不是我写了这些故事。但我觉得没有必要。在医院里醒来时,我左手的食指和中指上,都有黑色的墨迹。这说明我一直用黑墨水来写字。在我桌子上,有一个笔筒,里面放满了蘸水钢笔,笔尖朝上,像一丛龙舌兰的样子;笔筒边上放着一瓶中华牌绘图墨水。坐在这个桌子面前,我想道:假如我不是这个故事的作者,也不会有别人了;虽然我一点不记得这个故事。这些稿子放在这里,就如医院窗台上的《暗店街》。假如我不来认领,就永无人来认领。这世界之所以会有无主的东西,就因为有人失去了记忆。 

手稿上写道:盛夏时节,在湘西的红土丘陵上,是一片萧杀景象;草木凋零,不是因为秋风的摧残,却是因为酷暑。此时山坡上的野草是一片黄色,就连水边的野芋头的三片叶子,都分向三个方向倒下来;空气好像热水迎面浇来。山坡上还刮着干热的风。把一只杀好去毛的鸡皮上涂上盐,用竹杆挑到风里去吹上半天,晚上再在牛粪火里烤烤,就可以吃了。这种鸡有一种臭烘烘的香气。除了风,吃腐肉的鸟也在天上飞,因为死尸的臭味在酷热中上升,在高空可以闻到。除了鸟,还有吃大粪的蜣螂,它们一反常态,嗡嗡地飞了起来,在山坡上寻找臭味。除了蜣螂,还有薛嵩,他手持铁枪,出来挑柴禾。其它的生灵都躲在树林里纳凉。远远看去,被烤热的空气在翻腾,好像一锅透明的粥,这片山坡就在粥里煮着──这故事开始时就是这样。 

在医院里,我那张床就很热,我一天到晚都在锅里煮着,但我什么都不记得,也就什么都不抱怨,连个热字都说不出,只觉得很快乐。我不明白,热有什么可抱怨的呢。这篇稿子带有异己的气味。今天早上我遇到了很多东西:北京城、万寿寺、工作证、办公室,我都接受下来了。现在是这篇手稿──我很坚决地想要拒绝它。是我写的才能要,不是我写的──要它干啥? 

手稿上继续写道:薛嵩穿着竹笋壳做的凉鞋,披散着头发,把铁枪扛在肩上,用一把新鲜的竹篾条拴在腰上,把龟头吊起来,除此之外,身上一无所有。现在正是盛夏时节。假如是严冬,景象就有所不同:此时湘西的草坡上一片白色的霜,直到中午时节,霜才开始融化,到下午四点以后,又开始结冻,这样就把整个山坡冻成了一片冰,绿色的草都被冻在冰下,好像被罩在透明的薄膜里──原稿就是这样的,但我总怀疑亚热带地方会有这样冷──薛嵩穿着棉袍子出来,肩上扛着缠了草绳的铁枪──如果不缠草绳子,就会粘手。他还是出来挑柴火。春秋两季他也要出来挑柴火──因为要吃饭就得挑柴火──并且总是扛着他的大铁枪。 

我依稀记得,自己写到过薛嵩,每次总是从红土丘陵的正午写起,因为红土丘陵和正午有一种上古的气氛,这种气氛让我入了迷。此处地形崎岖,空旷无人,独自外出时会感到寂寞:在山坡上走着走着,忽然觉得天低了下来,连蓝天带白云都从天顶扣下来,天地之间因而变得扁平。再过一会,天地就会变成一口大碗,薛嵩独自一人走在碗底。他觉得自己就如一只倒臼里的蚂蚁,马上就会被粉碎,情不自禁地丢掉了柴捆,倒在地上打起滚来。滚完以后,再挑起柴来走路,走进草木茂盛的寨子,钻进空无一人、黑暗的竹楼。此时寂寞不再像一种暧昧的癫狂,而是变成了体内的刺痛。后来,薛嵩难于忍受,就去抢了红线为妻。这样他就不会被寂寞穿透,也不会被寂寞粉碎。如果感到寂寞,就把红线抱在怀里,就如胃疼的人需要一个暖水袋。如果这样解释薛嵩,一切都进行得很快。但这样的写法太过直接,红线在此时出现也为时过早。这就是只写红土丘陵和薛嵩的不利之处。所以这个故事到这里截止,从下一页开始,又换了一种写法。 

读到薛嵩走在红土丘陵上,我似乎看到他站在苍穹之下,蓝天、白云在他四周低垂下来,好似一粒凸起的大眼球。这个景象使我感到亲切,仿佛我也见到过。只可惜由此再想不到别的了。因此,薛嵩就担着柴禾很快地走了过去,正如枪尖刺在一块坚硬的石头上,轻飘飘地滑过了……如你所见,这种模糊的记忆和手稿合拍。看来这稿子是我写的。 

既然已经有了一个属于我的故事,把《暗店街》送给别人也不可惜。但我不知道谁是薛嵩,也不知道谁是红线;正如我不知道谁是莫迪阿诺,谁是居伊·罗朗。我更不知道自己是谁。 

3 

正午时分的山坡上,罩着一层蓝黝黝的烟雾。走在这种烟雾里,就是皮肤白皙的人也会立刻变得黝黑,就是牙色焦黄的人也会立刻牙齿洁白,头发笔直的人也会变得有点鬈发──手稿上这样写,仿佛嫌天还不够热──薛嵩在山坡上走,渐渐感到肩上的铁枪变得滚烫,好像是刚从溶炉里取出来。这根铁棍他是准备作扁担来用的,除了烫手之外,它还有一种不便之处──那东西有三十多斤重,用来作扁担很不适用。但是他决不肯把任何扁担扛在肩上。在铁枪的顶端,有个不大锋利的枪头,还有一把染红了的麻絮。如果你不知道这是枪缨,一定会把这条枪的性质看错,以为它不是一件兵器,而是一根墩布。在他的肚脐前面,一根竹篾条,好像吊了个大蘑菇。他就这样走下山坡,去找他的柴捆。 

薛嵩的身体颀长、健壮,把它裸露出来时,他缺少平常心。当他赤身裸体走在原野上时,那个把把总是有点肿胀,不是平常的模样;所以他小心翼翼地避开一切低洼的地方。低洼的地方会有水塘,里面满是浓绿色的水。一边被各种各样的脚印搅成黑色的污泥,另一边长满了水芋头、野慈菇,张开了肥厚的绿叶,开着七零八落的白花。只听哗啦一声水响,叶子中间冒出一个女孩的头来。她直截了当地往薛嵩胯下看来,然后哈哈笑着说:瞧你那个模样!要不要帮帮你的忙?成熟男性的这种羞辱,总是薛嵩的恶梦。等他谢绝了帮忙之后,那女孩就沉下水去。在混浊的水面上,只剩下一根掏空的芦苇竖着,还有一缕黑色的头发。在亚热带的旱季,最混的水里也是凉快的。薛嵩发了一会儿愣,又到山脊上走着,找到了自己的柴禾捆,用长枪把它们串成一串,挑回家来,蜣螂也是这样把粪球滚回家。此时他被夹在一串柴捆中间,像一只蜈蚣在爬。他被柴禾挤得迈不开步子,只能小步走着,好像一个穿筒裙的女人。假如有一阵狂风吹来,他就和柴捆一起在山坡上滚起来。故事虽然发生在中古,但因为地方偏僻,有些上古的景象。 

我对这个故事有种特殊的感应,仿佛我就是薛嵩,赤身裸体走进湘西的炎热,就如走入一座灼热的砖窑;铁枪太过沉重,嵌进了肩上的肉。至于腰间的篾条,它太过紧迫,带着粗糙勒进了阴茎的两侧──这好像很有趣。更有趣的是有个苗族小姑娘从水里钻出来要帮我的忙。但作者对这故事不是全然满意,他说,这是因为薛嵩是孤零零的一个人。孤零零一个人的故事必定殊为无趣,所以这个故事又重新开始道:晚唐时节,薛嵩曾住在长安城里。 

长安城是一座大得不得了的城市,周围围着灰色的砖墙。墙上有一些圆顶的城门洞,经常有一群群灰色的驴驮着粮食和柴草走进城里来。一早一晚,城市上空笼罩着灰色的雾,在这个地方买不到漂白布,最白的布买到手里,凑到眼前一看,就会发现它是灰的。这种景象使薛嵩感到郁闷,久而久之,他变得嗓音低沉。在冷天里他呵出一口白气,定眼一看,发现它也是灰的。这样,这个故事就有了一个灰色的开始,这种色调和中古这个时代一致。在中古时,人们用灶灰来染布,妇女用草灰当粉来用,所以到处都是灰色的。薛嵩总想做点不同凡响的事情。比方说,写些道德文章,以便成为圣人;发表些政治上的宏论,以便成为名臣;为大唐朝开辟疆土,成为一代名将。他总觉得后一件事情比较容易,自己也比较在行。这当然是毫无根据的狂想…… 

后来,薛嵩买到了一纸任命,到湘西来作节度使。节度使是晚唐时最大的官职,有些节度使比皇帝还要大。薛嵩觉得自己中了头彩,就变卖了自己的万贯家财,买了仪仗、马匹和兵器,雇佣了一批士兵,离开了那座灰砖砌成的大城,到这红土山坡上建功立业。后来,他在这片红土山坡上栽了树,种了竹子,建立了寨子,为了纪念自己在长安城里那座豪华住宅,他把自己的竹楼盖成了三重檐的式样,这个式样的特点是雨季一来就漏得厉害。他还给自己造了一座后园,在园里挖了一个池塘,就这样住下去;遇到了旱季里的好天气,就把长了绿霉的衣甲拿出来晒。过了一些年,薛嵩和他的兵都老了。薛嵩开始怀念那座灰色的长安城,但他总也不会忘记建功立业的雄心。 

与此同时,我坐在万寿寺的配殿里,头顶上还有一块豆腐干大小的伤疤。这块疤正在收缩,使我的头皮紧绷绷。我和薛嵩之间有千年之隔,又有千里之隔。如果硬要说我们之间有什么关系,实在难以想象。但我总要把自己往薛嵩身上想──除了他,我不知还有什么可供我来想象:过去我可能到过热带地方,见过三重檐的竹楼,还给自己挖过一个池塘;我在那里怀念眼前这座灰色的北京城,并且总不能忘记自己建功立业的决心──这样想并非无理。但假如我真的这样想过,就是个蠢东西。 

过去某个时候,薛嵩的故事是在长安城里开始的,到了湘西的红土山坡上,才和现在的开始汇合。这就使现在的薛嵩多了一个灰色的回忆,除此之外,还多了一些雇佣兵。我觉得这样很好,人多一点热闹。 

薛嵩部下的雇佣兵在找到雇主之前是一伙无赖,坐在长安城外晒太阳──从早上起来,就坐在城门口,要等很久才能等到太阳。这样看来,太阳好像很宝贵,但现在去晒,肯定要起痱子。长安城门口有一排排的长条凳,上面坐满了这种人,脚下放着一块牌子,写着:愿去南方当兵、愿去北方当兵、或者是愿去任何地方当兵;在这行字下面是索要的安家费。薛嵩既然付得起买官的钱,也就付得起雇佣兵的安家费。当然,这些钱不能白给,当场就要请刺字匠在这些兵脸上刺字,在左颊上刺下“凤凰军”,在右颊上刺下“亲军营”。这些刺下的字就是薛嵩和他们的契约。有了这六个字的保证,薛嵩觉得有了一批自己人,再不是孤零零的。不幸的是这个刺字匠和这些兵认识,所以把字迹刺得很浅,还没等走到湘西,那些字迹就都不见了,于是薛嵩又觉得自己还是孤零零的一个人。 

在这种情况下,薛嵩当然觉得自己钱花得不值,想要请人来在士兵脸上补刺,但那些兵都不干,并且以哗变相威胁。此时薛嵩干出了一件不雅的事情:他把裤子脱了下来,请他们看他的屁股。薛嵩为了和士兵同甘共苦,并且表示扎根湘西的决心,也请刺字匠刺了两行字,左边的是“凤凰军”,右边的是“节度使”。但他以为自己是朝廷大员,这些字不能刺在脸上,所以刺在了屁股上。不幸的是,屁股上的字也不能打动那些雇佣兵。而且这两行字刺得非常之深,一辈子都掉不了。所以,这会是薛嵩的终身笑柄。那些兵看了这些字就往上面吐唾沫。我觉得自己能够看到那两行字,是扁扁的隶书,就像刻在象棋上的字。而且我有一种难以抑制的冲动,想要脱下裤子,看看自己的屁股。之所以没有这样办,是因为这间房子里没有镜子。另外,这间房子也不够僻静。假如有人撞见我做这个举动,我就不好解释自己的行为…… 

4 

有一段时节,薛嵩的屁股甚为白皙,那些黑字嵌在肉里,好像是黑芝麻摆成的。现在薛嵩虽然已经晒黑,但那些字还是很清楚。他只好拿墨把屁股上的字涂掉。在那个赤裸裸的红土山坡上,一切都一览无遗,长着一个黑屁股,看上去的确可笑;但总比当个屁股上有字的节度使要好些。薛嵩还给每个兵都出了甲仗钱,足够他们买副铁甲,但是他们买的全是假货,是木片涂墨做成的,穿在身上既轻便,又凉快。可惜的是路上淋了几场雨,就流起了黑汤,还露出了白色木头底。薛嵩说:穿木甲去打仗,你们可是拿自己的生命去开玩笑哪;但那些兵脸上露出了蒙娜·丽莎般的微笑。等薛嵩转过头去,那些兵就纵声大笑,拍着肚子说:打仗!谁说我们要去打仗!那些兵一听说打仗,就好像听到了天大的笑话。这说明,虽然他们是士兵,但不准备打仗。他们给自己盖房子、抢老婆却很在行。 

雇佣兵最擅长的不是打仗,也不是盖房子和抢老婆,而是出卖;但薛嵩不知道这一点。统帅手下有了雇佣兵,就如一般人手里有了伪钞,最大的难题是把它打发掉。想要使这些人在战场上死掉,需要最高超的指挥艺术。很显然,这种艺术薛嵩并不具备。我听说有些节度使用骑兵押雇佣兵去打仗,但是不管用,那些人在战场上跑得比骑兵还快,还有些节度使用雇佣兵守寨子,把他们锁在栅栏上,但也不管用。敌方来打寨时,一个雇佣兵也见不到。因为他们像土拨鼠一样在脚下打了洞,一有危险就钻进洞里藏起来。所以最好把地面也夯实、灌上水泥,让他们打不成洞,但这样做太费工了。我还听说有些最精明的节度使手下有“长杆队”这样的兵种,由可靠的基干士兵组成,手持坚硬的木杆,杆端有铁索,锁住雇佣兵的脖子,用这种方式把雇佣兵推向阵前。只有在这种情况下,雇佣兵才会进入交战。长杆队的士兵还必须非常机警,因为稍不小心,就会变成自己被锁上长杆,被雇佣兵推向敌阵。除了不肯打仗,雇佣兵还很喜欢闹事:闹军饷、闹伙食、闹女人,等等。薛嵩率领着这支队伍刚刚到了湘西,就被人闹了一次,打出了满头的青紫块,具体地说,是一些圆圆的大包,全是中指的指节打出来的。被人敲了这么多的包,薛嵩会不会很疼,我不知道。因为我把自己视为薛嵩,我很不喜欢这个情节。我还觉得让那些兵这样猖狂很不好。 

薛嵩手下这伙雇佣兵从长安城跟薛嵩跋山涉水,到凤凰寨来。当时薛嵩骑在马上,手里拿着一张上面发下来的地图,注明了他管辖的疆域。结果他发现这片疆域是一片荒凉的红土山坡,至于凤凰寨的所在,竟是一个红土山包。总而言之,这是一片一文不值的荒地,犯不上倾家荡产去买。那些雇佣兵见了这片山坡,鼓噪一声,就把薛嵩从马上拉了下来,拔掉他的头盔,在他的头上大打凿栗。打完以后却都发起愣来,因为四方都是旷野──如前所述,这些人擅长出卖,但现在竟不知把薛嵩出卖给谁。因为没有买主,他们又给薛嵩戴上了头盔,把他扶上马去,听他的命令。薛嵩命令说:住下来,他们就住了下来,当然心里不是很开心,因为要开河挖渠,栽种树木,还要在山凹里种田。那些二流子从来没做过如此辛苦的工作,加之水土不服,到现在已经死了一半,还剩一半。我已经说过,让手下的雇佣兵死掉,是让所有节度使头疼的难题,所以薛嵩的这种成绩让大家都羡慕。正因为有了这种成绩,薛嵩不大受手下将士的尊重。假如没有这些成绩,也不可能受到他们的尊重。这样,这个故事从灰色开始,现在又变成红色的了。 

第二节 

我在万寿寺里努力回忆,有关自己,所能想起的只是如下这些:我头上裹着绷带,在病房里乐呵呵地躺着时,有个护士告诉我说,我骑了一辆自行车,被一辆面包车撞倒了,这辆面包车在我头盖骨上撞了一个坑,使我昏迷不醒;我就乐呵呵地相信了。现在我才知道:这是别人告诉我的事,我自己并不记得;而且我不能人家说什么就听什么,最起码得问问那开车的为什么要撞我──所以,必须要自己有主见。有一段时间我怀疑自己是薛嵩,但眼前无疑是二十世纪。此时我在万寿寺里,火红的阳光正把对面的屋影压低,投在我面前的窗户纸上。我不该无缘无故来到这里,总得有个前因才对。 

有关万寿寺,我的看法是:这地方不坏。院子古朴、宽敞,长满了我所喜欢的古树,院子打扫得很干净,但有一股令人疑惑的臭味,刺鼻子、刺眼睛。房子上装着古老的窗棂,上面糊着窗户纸,像这样的窗子,冬天恐怕难以防寒,但那是冬天的事情。眼下的问题是:这是个什么地方,我到这里来干什么。虽然这是一座寺院,但没有僧人出现,我自己也不是和尚。这一切都漫无头绪,唯一的头绪是我被一辆面包车撞了。还有一个问题是:那个开面包车的人和我到底有何仇恨,要这样来害我…… 

据说,对方出了我的医药费,赔了我一辆崭新的自行车,还赔了一套新衣服,这件事就算了结了。出院之前,我对大夫说,我好像还失掉了记忆。他笑了一笑,说道:适可而止吧;然后毅然决然地给我开了半个月的病假条。这个大夫又白又胖,长着很长的鼻毛……我对他说的话、做的事一点都不懂。但我还是觉得,他不信任我。可能他受了开车的什么好处──想到了此处,我露出了微笑,觉得自己已经很奸诈了。 

现在我猛然领悟,医生怀疑我之所以假称丧失记忆,是想让对方赔偿更多的东西。其实我没有这样想。我不想对方赔偿什么,不过是想打听一下我该做什么,到哪里去。为了证明我的诚意,我把病假条拿了出来,撕得粉碎。我想给自己倒点水喝,却发现暖瓶盛了一些污浊的冷水。然后,我坐了下来,疑虑重重地看着那个暖瓶,终于想到,这里既有暖瓶,肯定有地方能打到开水,于是起身拿了暖瓶出去,终于在角落里找到了那个小锅炉──取得了一个小小的胜利,感到很快乐──所以,失掉记忆也不全然是坏事。总想着自己丧失了记忆,才全然是坏事。 

现在,在万寿寺里,我读到这样的故事:过去有一天,薛嵩到山坡上去担柴,回寨的道路却不止一条。他的寨子是一片亚热带的林薮,盘踞在红土山坡上,如果从高空看去,这地方像个大旋涡,一圈圈长着大青树、木菠萝、山梨树,这些树呈现出成熟的紫色;在竹丛之间长满了仙人掌、霸王鞭、龙舌兰,这些林荫中的植物呈现出蓝色。在仙人掌之间长满了茅草,在茅草下面是青色的苔藓,在苔藓下面是霉菌生长的所在。至于还有什么在霉菌下面生长,它们是什么颜色,我就看不到了。在林带里,盘旋着可供大队人马通行的红土大路,上面铺着米黄色的砂石。在大路两边,岔出无数单人行走的小路,这些小路跨沟越坎,穿进了林荫。小路两面有猪崽子走的路,有时是一道印满了蹄印的泥沟,有时是灌木丛上的缺口。在猪崽子走的路边,有蛇行的小道──在压弯的茅草上面蜿蜒的痕迹。在蛇行的小道边上,有蚂蚁的小道──蚁道绕开了绵密的草根。在蚁道的两侧,理当还有更细微的小道,但不是人眼可以看到的。薛嵩像一串活动的柴捆一样从大路上走过,越走近旋涡的中心,道路就越窄,两边的林荫也越逼近。最后出现在他面前的,是一道真正的壕沟,沟壁有卵石砌的护坡。在壕沟对面,有一道真正的营栅,是一排无头树组成的,树干上长出了密密层层的嫩枝条。壕沟正面是一道吊桥。这道吊桥是十六根梨树扎成的木排做成,由碗口粗的青藤吊着。不幸的是它吊不起来,因为梨树在壕沟两端都生了根。这些树还结了一些梨,但都结在了桥下面,不下到沟里就摘不到。 

我也不记得这片亚热带的林薮。但这不是别人告诉我的事情。这是我自己告诉我的事情。比之别样的事情,这件事更可相信,所以,我宁可相信以前有一个薛嵩担着柴捆从两面生根的吊桥上走过,也不相信我骑在自行车上被汽车撞倒了──虽然我头上有个很大的伤疤,但它也可以是被人打出来的──假如大夫受了打人凶手的好处,就会这样来骗我,帮他开脱罪责。这样一想,我有觉得自己还不够奸诈。奸诈这件事,只要开了头,就不会有够。 

薛嵩挑着柴捆从吊桥上走了过去,在大青树的环抱之下,眼前是个小小的圆形广场。在阴暗的光线下,有座草棚,草棚下面,有个黑色大漆的案子,两端木架上放着薛嵩的铠甲、弓箭、仪仗等等破烂发霉的东西。这里是薛嵩心中的圣地。广场的侧面有夯土而成的台子,台上有木板房,这是薛嵩心目中的另一个圣地。这两个地方都是军队凝聚力的源泉,是凤凰寨的中枢。 

他把柴捆卸在木板房的屋檐下,拉开纸糊的拉门,走了进去,坐在木头地板上,解开拴住龟头的竹篾,等了一会儿,不见有人来,就用手掌拍击起地板来了。假如我的故事如此开始,那天下午薛嵩没有回到自己家里,而是走到寨心去了。需要说明的是,这座木板房住了一个营妓。看到此处,我也恍然大悟:原来,薛嵩手下是一帮无赖。没有女人的地方,无赖们怎么肯来呢。 

薛嵩坐在寨中心的木板房子里,用手叩着地板,从屏风后面跑出一个女人来。她描眉画目,头上有一个歪歪倒倒的发髻,身上穿着紫花的麻纱褂子,匆匆忙忙束着腰带,脚下踏着木屐,跑到薛嵩面前匍匐在地,细声叫道:“大人。”她愿意给薛嵩用黄泥的小炉子烧一点茶,但他拒绝了。她还愿意为薛嵩打扇,陪他坐一会儿,他也拒绝了。如前所述,薛嵩赤身裸体,像个野蛮人──虽然他已经把龟头从竹篾条上解下来了。这种装束使他决定使事情简单一些,所以他做了一个坚决的手势:左掌举平,掌心向下,朝前平伸着。这个女人平躺下来,岔开两腿,两手平摊,躺成一个大字形。于是薛嵩膝行前进,进到那女人的两腿之间,帮她除去脚上的木屐和袜子──她的脚因为总穿木屐,所以足趾变成了蟹爪形──并且解开她的腰带,让她身体的前半面袒露出来。她的身体当然像粉雕玉琢一样的白。至于模样,可能是这样:大腿有点过粗,腹部的皮有点松懈,乳头尖尖的,整个胸部是个W形,但也可能不是这样。薛嵩憋住一口气,插了进去,这仿佛是打开了语言的禁忌。那个女人开始和他聊起来:你怎么老不来呀?这么热的天,怎么还出来?等等。但薛嵩憋着气,一声都不吭。 

这位妓女十分白皙:不但脸色白,连嘴唇都白。眉毛几近透明,只带有一点点淡黄色,浑身上下到处可以见到蓝色的血管。只是这些血管全都很粗,全都曲张着,好像打着滚。她好像笼罩在一团白雾里,显得比较年轻,实际上是个老太太。在凤凰寨的中心,一切都是绿色的:首先,一切都笼罩在一片绿荫之下;其次,到处长满了绿色的青苔;就是呆在白色的纸门后面,浓绿的光线还是透过了窗纸,沁到房子里来。在这间房子里,薛嵩黝黑的身体变成了青铜色,而妓女苍白的身体上好像布满了细碎的绿点,好像某一种磁砖──当然,这只是一种错觉,假如凑近了去看,却看不到任何的绿点。除此之外,空气也潮湿得像油一样,这使薛嵩感觉自己悬浮在绿油当中,一切都变得缓慢,甚至就要停止了。在这绿色的一团里,有一股浓郁的水草气。一切都归于沉寂,但真正沉寂下来时,又听到远处水牛在“哞哞”地叫,那种声音很沉重,很拖沓;近处的青蛙在“哇哇”地叫,这种声音很明亮,很紧凑。而那女人确一声不吭了。她还闭上了眼睛,好像一个死人。 

整个凤凰寨泡在一片绿荫里,此地又是绿荫的中心。就是呆在屋里,也感到了绿色的逼迫。薛嵩鹰勾鼻子斗鸡眼,披着一头长发,正在奋发有为的年纪。在做爱时他也想要有所作为──他在努力做着,想给对方一点好的感觉。所谓努力,就是忘掉了自己在干什么,只顾去做;与此同时,听着青蛙叫和水牛叫;但对方感觉任何,他一点都不知道。这就使他感觉自己像个奸尸犯。那女人长了一张刀一样的长脸,闭上眼以后,连一根睫毛都不动,我想,这应该可以叫做冷漠了。后来,她在铺板上挪动了一下头,整个发髻就一下滚落下来。原来这是个假头套。在假发下面她把头发剃光,留下了一头乌青的发茬。她急忙睁开眼睛,等到她从薛嵩的眼色里看出发髻掉了,这件事已经不可挽救。她伸出手去,把头套抓在手里,对薛嵩负疚地说道:没办法,天气热嘛。这话大有道理,在旱季里,气温总在三十七八度以上,总顶着个大发髻是要长痱子的。头套的好处是有人时戴上,没人的时候可以摘下来。薛嵩看到了一个既青又亮的和尚头,这种头有凉爽的好处。除此之外,他还发现她的小腿和身上的肤色不同,是古铜色的,而且有光泽。这说明她经常跑出去,光着腿在草丛里走过。这两件事使薛嵩感到沮丧,这样一个女人叫他感觉不习惯。他很快地疲软下来。那个老娼妓用粗哑的嗓子讲起话来:弄完了吗?快点起来吧,热死了!于是薛嵩说道:我就不热吗?然后就爬到一边去,傻愣愣地不知道自己干了些什么。与此同时,他感到心底在刺痛。 

2 

如果用灰色的眼光来看凤凰寨,它应该是座死气沉沉的兵营。在寨栅后面,是死气沉沉的寨墙,在寨墙后面,是棋盘似的道路和四四方方的帐篷,里面住着雇佣兵。在营盘的正中,住着那个老妓女,她像一个纸糊没胎的人形,既白,又干瘪。在她脸上,有两道牦牛尾巴做的假眉毛,尾梢从两鬓垂了下来。一开始,凤凰寨就是这样的,像一张灰色的棋盘上有一个孤零零的白色棋子。只可惜那些雇佣兵不满意,一切就发生了变化;这个故事除了红色,又带上了灰色以外的色彩。手稿的作者就这样横生起枝节来…… 

那个老营妓当初和这些雇佣兵一起来到凤凰寨,在前往湘西的行列里,她横骑在一匹瘦驴身上,头上束了一条三角巾,戴了一顶斗笠,脚下穿着束着裤脚的裤子,脸上敷了很厚的粉,一声不吭,也毫无表情。这女人长了一个尖下巴,眉心还有一颗痣。在行军的道路上,那些士兵轮流出列,跑到队尾去看她,然后就哈哈大笑,对她出言不逊,但她始终一声也不吭,保持了尊严。据说,薛嵩买下了湘西节度使的差事之后,也动了一番脑子,还向内行请教过。所有当过节度使的人一致认为,在边远地方统率雇佣军,必需有个好的营妓,她会是最重要的助手。为此薛嵩花重金礼聘了最有经验营妓,就是这个老婆子。当然,走到路上听到那些雇佣兵起哄,薛嵩又怀疑自己被人骗了,钱花得不值。但那个女人什么都不说,她对自己很有信心。任凭尘土在她周围飞扬──假如有只苍蝇飞过来要落在她脸上,她才抬起一只手去撵它;一直来到红土山坡底下,她才从驴背上下来,坐在自己的行李上,看男人工作,自己一把手都不帮。顺便说一句,她做生意,也就是和男人干事时,也是这样:不该帮忙时绝不帮忙,需要帮忙时才帮忙。 

后来,薛嵩率领着手下的士兵修好了寨子,也给她修好了房子,这女人就开始工作:按照营规,她要和节度使做爱,并且要接待全寨每一个出得起十文铜钱的人,不管他是官佐还是士兵,是癞痢还是秃子,都不能拒绝。一开始那帮无赖都不肯到她那里去,还都说自己不愿冒犯老太太。但后来发现再无别处可去,也就去了,这个女人埋头苦干,恪守营规,赢得了大家的尊敬。开头她每五天就要和全寨所有的人性交一次,这是十分繁重的工作,但她也赚了不少铜钱。顺便说一句,这种工作的繁重是文化意义上的,从身体意义上说就满不是这样,因为干那事时,她只是用头枕着双手躺着。虽然她也要用这些铜钱向士兵们买柴买米,但总是赚得多,花得少。后来事情就到了这种地步,全寨子里的铜钱全被她赚了来,堆在自己的厢房里,这寨子里的铜钱又没有新的来源,所以她就过得十足舒服:白天她躺在家里睡大觉,到了傍晚,她数出十文铜钱,找出寨里最强壮、最英俊的士兵,朝他买些柴或米;当夜就可以和他同床共枕,像神仙一样快活,并且把那十文钱又赚了回来。就如邱吉尔①所说,这是她最美好的时刻,并且整个凤凰寨也因此变得井然有序。这位营妓从来不剪头发,也不到外面去。不管天气是多么炎热,屋里是多么乏味。由于她的努力,整个凤凰寨变成了长安城一样的灰色。 

薛嵩和他的人在凤凰寨里住了好几年了,所以这里什么都有,有树木和荒草、竹林、水渠等等,有男人和女人,到处游逛的猪崽子、老水牛,还有一座座彼此远离的竹楼,这一点和一座苗寨没有什么区别;还有节度使、士兵、营妓,这一点又像座大军的营寨,或者说保留了一点营寨的残余。这就是说,老妓女营造的灰色已经散去,秩序已经荡然无存了。 

在这个时刻,凤凰寨是一个树木、竹林、茅草组成的大旋涡,在它的中心,有座唐式的木板房子,里面住了一个妓女──这是合乎道理的:大军常驻的地方就该有妓女。在木板房子的周围,有营栅、吊桥等等。所以,只有在这个妓女身上时,薛嵩才觉得自己是大唐的节度使,这种感觉在别的地方是体会不到的。而这个妓女,如我所说,是个奶子尖尖的半老徐娘,假如真是这样的话,等到薛嵩坐起来时,她也坐了起来,戴好了假头套,拉拢了衣襟,就走到薛嵩身边坐下,帮他揉肩膀、擦汗,然后取过那根竹篾条,拴在他腰上,并且把他的龟头吊了起来;然后把纸拉门拉开,跪在门边,低下头去。薛嵩从屋子里走出去,默不作声地担起了柴担走开了。此时他的柴担已经轻了不少──有半数柴捆放在妓女的屋檐下了。 

我写过,这个女人很可能不是半老徐娘。她是一个双腿修长、腰身纤细、乳房高耸的年轻姑娘,在这种情况下,她会不戴假发、穿上衣服,更不会给薛嵩揉肩膀。用她自己的话来说:我这么年轻漂亮,何必要拍男人的马屁?她站起身来,遛遛达达地走到门口,从桑皮纸破了的地方往外看,与此同时,她还光着身子、秃着头;这颗头虽然剃出了青色,但在耳畔和脑后的发际,还留了好几缕长长的头发。这就使她看起来像个孩子……后来她猛地转过身来,用双手捧住自己的乳房,对薛嵩没头没脑地说,还能风流好几年,不是吗?然后就自顾自地走到屏风后面去了。与此同时,那件麻纱的褂子、假发、袜子和木屐等等,都委顿在地上,像是蛇蜕下的皮。薛嵩自己拴好了竹篾条,心中充满了愤懑,恶狠狠地走出去,把那担柴全部挑走了。这个妓女的年龄不同,故事后来的发展也不同。在后一种情况下,薛嵩深恨这个妓女,老想找机会整她一顿;在前一个故事里就不是这样。如果打个比方的话,前一个故事就像一张或是一叠白纸,像纸一样单调、肃穆,了无生气;而后一个故事就像一个半生不熟的桃子。在世间各种水果中,我只对桃子有兴趣。而桃子的样子我还记得,那是一种颜色鲜艳的心形水果…… 

① 邱吉尔的战时演说。 

3 

必须说明,“邱吉尔的战时演说”是原稿上的注。我现在不记得谁是邱吉尔,并且并不感到羞愧,我也不知道该不该为此感到羞愧──凤凰寨里原来只有一个奶子尖尖的老妓女。现在多出一个年轻姑娘,这说明情况有了一些变化。现在凤凰寨里不但有一个老营妓,又来了一个新营妓。理由很简单,那些二流子兵对薛嵩说:老和一个老太太做爱没说明味道。薛嵩觉得这些兵说得对,就掏出最后的积蓄,又去请了一个妓女。这样一来,就背叛了原来的营妓,也背叛了自己。因为这个新来的女孩一下就摧毁了老妓女建立的经济学秩序。除此之外,她还常在日暮时分坐在走廊下面,左边乳房在一个士兵手里,右边乳房在另一个士兵手里,自己左右开弓吻着两个不同的男人,完全不守营规。这样一来,寨子里就变得乱糟糟。那些二流子常为了她争风吃醋打架,纪律荡然无存。就连薛嵩自己,也按捺不住要去找这个年轻的姑娘。因为在做爱时,她总是津津有味地吃着野李子,有时会猛然抱住他,用舌头把一粒李子送到她嘴里,然后又躺下来,小声说:“吃吧,甜的!”当然,这粒李子她已吃掉一半了。总之,这女孩很可爱。但薛嵩觉得找她对自己的道德修养有害。每次走过那里,他都有一种内疚、自责的心情。这就是他要揍她的原因。 

在后一个故事里,那天晚上薛嵩击鼓招集他的士兵,在寨子中心升起一堆火来,把一个烧黑了的锅子吊到火焰上。秩序兵披散着头发,是一些高高矮矮的汉子,有的腿短、有的头大、有的脸上有刀疤、有的上腹部高高地凸起来,聚在一起喝了一点淡淡的米酒,就借酒撒疯,把木板房里的姑娘拖出来,绑在大树上,轮流抽她的背,据说是惩罚她未经许可就剃去了头发。揍完以后又把她解下来,让她在火堆边上坐下,用新鲜的芭蕉树芯敷她的背,还骗她说:揍她是为她好。这个姑娘在火边坐得笔直──这是因为如果躬着身子,背上的伤口就会更疼──小声啜泣着,用手里攥着的麻纱手绢,轮流揩去左眼或右眼的泪。这块手绢她早就攥在手心里,这说明她早就知道用得着它。这个女孩跪在一捆干茅草上,雪白的脚掌朝外,足趾向前伸着,触到了地面,背上一条红、一条绿。红就无须解释,绿是因为他们用嫩树条来抽她的脊梁,有些树条上的叶子没有摘去。如前所述,她身子挺得笔直,头顶一片乌青,但是发际的软发很难剃掉,所以就一缕缕地留在那里,好像一种特别的发式。从身后看去,除了臀部稍过丰满之外,她像个男孩子,当然,从身前看来,就大不一样。最主要的区别有两个,其一是她胯下没有用竹篾条拧起来的一束茅草、嫩树条,如薛嵩所说,用“就便器材”吊起来的龟头,其二就是她胸前长了两个饱满的乳房,在心情紧张时,它们在胸前并紧,好像并排的两个拳头,就是现在这个样子。在疲惫或者精神涣散时,就向两侧散开;就如别人的眉头会在紧张时紧皱,在涣散时松开。这个女孩除了擦眼泪,还不时瞪薛嵩一眼,这说明她知道挨揍是因为薛嵩,更说明她一点也不相信挨揍是为了自己好。而薛嵩回避着她的目光,就像小孩子做错了事情后回避父母。后来,小妓女从别人手里接过那个小漆碗,喝了碗里的茶──茶水里有火味,碗底还有茶叶,连叶带梗,像个表示和平的橄榄枝。喝下了这碗水,她的心情平静一点了。 

到目前为止,我的故事里有一个长安来的纨绔子弟,有一伙雇佣兵,有一个老妓女,有一个小妓女,还有一个叫作红线的女孩,但她还没有出现。我隐约感到这个故事开头拖沓、线索纷乱,很难说它隐喻着些什么。这个故事就这样放在这里吧。 

第三节 

我终于走出房子,站在院子中央,和进来的人打招呼。有很多人进来,我都不认识──我总得认识一些别人才对。在医院里,常从电视上看到有人这样做:站在大厅的门口,微笑着和进来的人握手──但病友们说这个样子是傻帽,所以我控制了自己,没把手伸出去,而是把它夹在腋下,就这样和别人打招呼,有点像在电视上见过的希特勒。不用别人说,我自己也觉得这样子有点怪。 

现在似乎是上班的时节,每隔几分钟就有一个人进来。我没有手表,不知道是几点。但从太阳的高度来看,大概是十点钟。看来我是来得太早了。我对他们说:你早。他们也说:你早。多数人显得很冷淡,但不是对我有什么恶意,是因为这院子里的臭气。假如你正用手绢捂住口鼻,或者正屏住呼吸,大概也难以对别人表示好意。最后进来一个穿黄色连衣裙的女孩。她一见到我,就把白纱手绢从嘴上拿了下来,瞪大了眼睛说:你怎么出来了,你?这使我觉得自己是个炸尸的死人。这个姑娘圆脸,眼睛不瞪就很大,瞪了以后,连眼眶都快没有了。我觉得她很漂亮,又这样关心我,所以全部内脏都蠢蠢欲动。但她马上又转身朝门口看去,然后又回过头来说:她到医院去看你了,一会儿就来。我不禁问道:谁?她娇嗔地看了我一眼说:小黄嘛,还有谁。我谨慎地答道:是嘛……但是,小黄是谁?她马上答道:讨厌,又来这一套了;然后用手绢罩住鼻子,从我身边走开。 

我也转过身去,背对着恶臭,带着很多不解之谜走回自己屋里。有一位小黄就要来看我,这使我深为感动。遗憾的是,我不知道她是谁。那位黄衣姑娘说我“讨厌,又来这一套”,不知是什么意思。这是不是说,我经常失去记忆?如果真是这样,那就是说,那辆面包车老来撞我的脑袋──不知它和我有何仇恨。这只能说那辆车讨厌,怎么能说是我讨厌呢? 

坐在凳子上,我又开始读旧日的手稿,同时把我的处境往好处想。在《暗店街》里,主人公费尽一生的精力来找自己的故事,这是多么不幸的遭遇。而我不费吹灰之力就找到了,这是多么幸运的遭遇。从已经读过的部分判断,我是个不坏的作者,我很能读得进去。但我也希望小黄早点到来……虽然我还不知小黄是谁,是男还是女。 

在凤凰寨里,这个小妓女经常挨揍,因为此地是一所军营,驻了一些雇佣兵。为此应该经常惩办一些人,来建立节度使的权威。他对别人进行过一些尝试,但总是不成功。比方说,薛嵩在红土山坡上扎寨,虽然开了一小片荒,但还是难以保障大家的口粮。好在大唐朝实行盐铁专卖,这样他就有了一些办法。每个月初,他都要开箱取出官印,写一纸公文,然后打发一个军吏、一个士兵,到山下的盐铁专卖点领军用盐,然后再用盐来和苗人换粮食。等到这两个人回来,薛嵩马上就击鼓升帐,亲自给食盐过磅,检查他们带回来的收据,然后就会发现军吏贪污。顺便说一句,军吏就是现在的司务长,由有威信的年长士兵担任。在理论上,他该是薛嵩的助手,实际上远不是这样。 

等到查实了军吏贪污有据,薛嵩感到很兴奋,因为他总算有了机会去处置一个人。他跳了起来,大叫道:来人啊!给我把这贪污犯推出去,斩首示众!然后帐上帐下的士兵就哄堂大笑起来。薛嵩面红耳赤地说:你们笑什么?难道贪污犯不该杀头吗?那些人还接着笑。那个军吏本人说:节度使大人,我来告诉你吧。军吏不贪污,还叫作军吏吗。那些士兵随声附和道:是啊,是啊。薛嵩没有办法,只好说:不杀头,打五十军棍吧。那个军吏问:打谁?薛嵩答道:打你。军吏斩钉截铁地说:放屁!说完自顾自地走开了。薛嵩只好不打那个军吏,转过头去要打那个同去的士兵。那个兵也斩钉截铁地回答道:放屁!说完也转身走了。这使薛嵩很是痛哭,他只好问手下的士兵:现在打谁?那些兵一齐指向小妓女的房子,说道:打她!那个小妓女坐在自己家里,隔着纸拉门听外面升帐,听到这里,就连忙抓住麻纱手绢,嘴里嘟囔道:又要打我,真他妈的倒霉!后来她就被拖出去,扔在寨心的地下,然后又坐起来,从嘴里吐出个野李子的核来,问道:打几下?别人说,要打她五十军棍。她就高叫了起来:太多了!士兵们安慰她道:没关系,反正不真打;说完就把她拖翻在满是青苔的地面上,用藤棍打起来了。虽然薛嵩很重视礼仪,但他总是中途退场,因为他看不下去。这已经不是惩罚人的仪式,成了某种嬉戏。总而言之,自从到了凤凰寨,薛嵩没有杀过一个手下人,他只杀了一个刺客。他也没打过一个手下的人,除了那个小妓女,她每隔一段时间就要被从草房里拖出去打一顿,虽然不是真打。这使薛嵩感到自己的军务活动成了一种有组织的虐待狂,而且每次都是针对同一个对象。这让他自己都觉得不好意思了。 

后来,有一些人在我门前探头探脑,问我怎么出院了;说完这些话,就一个个地走了。最后,有一个穿蓝布制服、戴蓝布制帽的人走到我房子里来,回避着我的注视,把一份白纸表格放在我桌子上,说道:小王,有空时把这表格再填一填;然后他就溜走了。这个人有点娘娘腔,长了一脸白胡子茬,有点面熟……稍一回忆,就想到今天早上在院子里见过他三四次。他总是溜着墙根走路。但根据我的经验,墙角比院子中间臭得更厉害。所以这个人大概嗅觉不灵敏。虽然刚刚认识,但我觉得他是我们的领导。我的记忆没有了,直觉却很强烈。由这次直觉的爆发,我还知道了有领导这种角色。你看,我还不知道自己是谁,就知道了领导;不管多么苛刻的领导,对此也该满意了…… 

这份表格已经填过了,是用黑墨水填的,是我的笔迹。但不知为什么还有再填。经过仔细判读,我发现了他们为什么要把这表格给我送回来。在某一栏里,我写下了今年计划完成的三部书稿。其一是《中华冷兵器考》,有人在书名背后用红墨水打了一个问号;其二是《中华男子性器考》,后面有两个红墨水打上的问号;其三是《红线盗盒》(小说),下面被红墨水打了双线,后面还有四个字的评语:“岂有此理!”这说明这样写报告是很不像话的,所以需要重写。但到底为什么这是很不像话的,我还有点不明白。这当然要加重我的焦虑…… 

有关我的办公室,需要仔细说明一下:这间房子用方砖漫地,但这些砖磨损得很厉害,露出了砖芯里粗糙的土块。我的办公桌是个古老的香案,由四叠方砖支撑着。案面上漆皮剥落之处露出了麻絮──在案子正中有一块裁得四四方方黑胶垫。案上还有一瓶中华牌的绘图墨水,是黑色的。旁边的笔筒里插了一大把蘸水笔;还有个四四方方、笨头笨脑的木凳子放在案前,凳子上放了一个草编的垫子。桌上堆了很多旧稿纸,有些写满了字,有些还是空白。虽然有这些零乱之处,但这间房子尚称整洁,因为每件家具都放得甚正,地面也清扫得甚为干净。可以看出使用这间房子的人有点古板、有点过于勤俭,又有点怪癖。此人填了一份很不像话的报告,这份报告又回到了我手里。我该怎么办,是个大问题。我急切地需要有个人来商量一下,所以就盼着小黄快来。我不知小黄是谁,所以又不知能和他(或她)商量些什么。 

2 

我忽然发现,我对自己所修的专业不是一无所知,这就是说,记忆没有完全失去──我所在的地方,是在长河边上。这条河是联系颐和园和北京内城的水道,老佛爷常常乘着画舫到颐和园去消夏。所谓老佛爷,不过是个黄脸老婆子。她之所以尊贵,是因为过去有一天有个男人,也就是皇帝本人,拖着一条射过精,疲软的鸡巴从她身上爬开。我们所说的就是历史,这根疲软的鸡巴,就是历史的脐带。皇帝在操老佛爷时和老佛爷在挨操时,肯定都没有平常心:这不是男女做爱,而是在创造历史。我对这件事很有兴趣,有机会要好好论它一论……因为那个老婆子需要有条河载她到颐和园游玩,在中途又要有个寺院歇脚,因此就有了这条河、这个寺院。在一百年后,这座寺院作为古建筑,归文物部门管理;而我们作为文史单位,凭了一点老关系,借了这个院子,赖在里面。这一切都和那根疲软了的鸡巴有某种关系。老佛爷对那根鸡巴,有过一种使之疲软的贡献,故而名垂青史。作为一个学历史的人,这条处处壅塞的黑水河,河上漂着的垃圾,寺院门上那暗淡、釉面剥落的黄琉璃瓦,那屋檐上垂落的荒草,都叫我想起了老佛爷,想到了历史那条疲软了的脐带。诚然,这条河有过刚刚疏浚完毕的时刻;这座寺院有过焕然一新的时刻;老佛爷也有过青春年少的时刻;那根脐带有过直愣愣、紧绷绷的时刻。但这些时刻都不是历史。历史疲惫、瘫软,而且面色焦黄,黄得就像那些陈旧的纸张一样。很显然,我现在说到的这些,绝不是今天才有的想法,但现在想起来依旧感到新奇。 

现在总算说到了凤凰寨的男人为什么要把龟头吊起来:这是一种礼节,就如十七世纪那些帆缆战舰鸣礼炮。一条船向另一条船表示友好,把装好的炮都放掉,含义是:我不会用这些炮来打你。红土山坡上的男人把自己的龟头吊了起来,意在向对方表示,我不会用这东西来侵犯你。当然,放掉的炮可以再装上,吊起的龟头业可以放下来,但总是在表示了礼节之后。因为此地有一种上古的气氛,所以男人们对自己的龟头也是潦草行事,随便的一吊;它也就死气沉沉地呆在那里,像一条死掉多年、泡在福尔马林里的老鲇鱼。 

因为是大地方来的人,薛嵩对“就便器材”甚是考究,每天晚上都要砍一节嫩竹,把它破成一束竹条浸到水塘里,使之更加柔软。这东西是一次性使用,撒尿或做爱时解下来,就要换一根新的。在家里时,薛嵩总是拿着那捆竹条,行坐皆不离手。出门时,他把它挂在铁枪上。用这种篾条吊着,它显得多少有点生气,虽然依然像条老鲇鱼,但死后的时间短了一些。后来他就用这束竹条抽了那小妓女的脊背。经过漫长的一天,竹条只剩了三四根,抽起人来特别疼。那女孩挨了一下,抽搐着从树干上扬起头来,说道:薛嵩!真狠哪你。这使薛嵩感到不好意思,差点把竹条扔掉,去拣根别人用过的柳条。但转念一想:我是为了她好,就继续用竹条抽下去。又抽了三四下,才走到一旁,把她让给别人。 

这个女孩子面朝大树站着,双臂环抱着大树,手腕用就便器材捆在一起。这个就便器材是一把青芦苇,拧成绳子状;捆妇女儿童可以,捆男人就把不牢。在大树底下,有裸出地面的树根,还有青苔细泥。那女孩在树根和青苔上踱步,装似在健身自行车上或跑步机上锻炼身体。薛嵩看着这一切,沉思着,忽然用竹条在自己腿上抽了一下──这种疼痛虽然厉害,但还不是无法忍受。然后他放了心,觉得自己还不算过分。如果我说,薛嵩在构思一篇名为“以就便器材刑责违纪人员的若干体会”的军事论文,就未免过分;但他的确是在想着一些什么;这如我也在考虑《中华男子性器考》应该怎么写…… 

后来有个兵报告说:打完了!还干点啥?薛嵩说:放了她!人们把她放开,她的手腕上有两条绿色的环形。她想到山涧里洗去,但别人劝止到:别去。着了水露,伤口要化脓。其实也没有什么伤口,但总要这么一说来表示关心。所以她就用麻纱手绢蘸了树叶上的露水,揩去了手腕上的绿印。此时她的大腿、腹部还有乳房上满是青苔和树皮;有个兵从地下拔了一把羊胡子草,帮她把这些擦去。她很快接过了那把草,说道:谢谢。自己来。总而言之,在她走到火堆边上自己座位上之前,很是忙碌了一阵,这个女孩是忙碌的中心。这种忙碌带有一点驾轻就熟的意味。此时薛嵩孤零零地坐在火堆边上,体会到了作为将帅和领袖的寂寞,心里默默地想道:我又把她揍了一顿。这样,这一章就有了一个灰色的开始。接下去她还要灰得更厉害。那天晚上,薛嵩揍着小妓女,心里却在想着老妓女。每抽一下,他都把头转向老妓女的木板房,想要看出她是否坐在纸门后面,透过门缝看这件事;单因为天色已暗,那房子里又没有点灯,所以他瞪得眼睛都要瞎了,还是什么都没看见。 

3 

如前所述,在凤凰寨的中心,有座夯土而成的平台。需要说明的是,这座高台的四周有卵石砌成的护坡,以防它被雨水淋垮;台上有座木板房,用树皮做房顶。树皮上早已生了青苔,正在长出青草来,在木板房子里住了一个妓女,或年老或年轻,或敬业或不敬业,或把男人叫作“官人”、“大人”,或叫作“喂,你!”。这是个矛盾,所以在凤凰寨里,实际上有两个妓女──这么大的寨子,只有一个营妓是不够的。这就是说,寨里有两座木板房子、两个夯土的平台,并肩而立。这样解决矛盾,可称为高明。在这两座房子后面,有两个不同的花园,前一个妓女的园子里,有碎石铺成的小路,有一座小小的圆形水池,里面栽了一蓬印度睡莲。在长安城里,可以买到印度睡莲的种子,但要把它遥迢地带来。除了小径和水池,所有的地面都铺上了砂子,以抑制杂草。特别要指出的是,花园的一角有一口深不可测的枯井,为了防止井壁坍塌,还用石块砌住了,枯井上铺了一块有洞的厚木板,厚木板四面是个薄板钉成的小亭子。 

你可能已经想到,这是一种卫生设备,直言不讳地说,这是一个厕所。那位老妓女在其中便溺之时,可以听到地下遥远的回声。花园里当然还种了些花草,但已经不重要,总之,那老妓女得暇时,就收拾这座花园。而那位年轻姑娘的后园里长满了野芭蕉、高过头顶的茅草、乱麻杆、旱芦苇等等,有时她兴之所至,就拿刀来砍一砍,砍得东一片、西一片,乱七八糟。更可怕的是她在这后园乱草里屙野屎。离后园较远处,有一棵笔直的木菠萝树,看来有三五十岁,长得非常之高。有一根藤子,或者是树皮绳,横跨荒园,一头拴在树干分岔处,另一头拴在屋柱上。树上有个藤兜,只要没有人来,那女孩就顺着藤子爬到藤兜里睡懒觉。 

对于这种区别,手稿里有种合理的解释:老妓女是先来的,在她到来之前,寨中并无妓女。薛嵩督率手下人等修好了房子,并且认真建了一座花园,迎接她的到来。小妓女是后来的,此时薛嵩等人已修了一座花园,有点怠倦。除此之外,他们是在老妓女的监视之下修筑房舍,太用心会有喜新厌旧的罪名。总而言之,先到或后到凤凰寨,待遇就会有些区别。当然,你若说我在影射先到或后到人世上,待遇会有区别,我也没有意见,因为一部小说在影射什么,作者并不知道。那天晚上因为不敬业而受责的是小妓女,但是薛嵩执意要把她绑到老妓女门前的树上抽。这说明,薛嵩还有更深的用意。 

手稿中说,薛嵩他们打那女孩子的原因是:她剃了头,装了假头套。在这座寨子里,随便剃头是犯了营规。但那个老妓女也剃了头,就没人打她。他们打过了那女孩,又把她放开,让她坐在火堆边上。过了一些时候,她疼也疼过了、哭也哭过了,心情有所好转,就说:喂,你们!谁想玩玩?在座的有不少人有这种心情,就把目光投向薛嵩。薛嵩想,我没有理由反对,就点了点头。于是一个大兵转过身来,把后腰上竹篾条的扣对准她,说道:“解开!”那女孩伸手去解,忽而又把手撤回来,在她背上猛击一下道:你刚还打过我哪!我干嘛要给你“解开”!薛嵩暗暗摇头,从火堆边上走开,心里想着:这女孩被打得还远远不够;但他对打她已经厌烦了。 

不久之前,我在医院里从电视上看到一部旧纪录片。里面演到二战结束后。法国人怎么惩办和德国兵来往的法国姑娘──你可能已经知道了,他们把她们的头发剃光──在屋檐下有一把椅子,那些女孩子轮流坐上去,低下头来。坐上去之前是一些少女,站起来时就变成了成年的妇人。刮得发青的头皮比如云的乌发显得更成熟,带有更深的淫荡之意──那些女孩子全都很沉着地面对理发师的推子和摄影机,那样子仿佛是说:既然需要剃我们的头发,那就剃吧。 

那个小妓女对受鞭责也是这样一种态度:既然需要打我的脊梁,那就打吧。她自己面对着一棵长满了青苔的树,那棵树又冷又滑,因为天气太热,却不讨厌。有些人打起来并不疼,只是麻酥酥的,很煽情。这时她把背伸向那鞭打者。有些人打起火辣辣地疼,此时她抱紧这棵清凉的树……她喜欢这种区别。假如没有区别,生活也就没意思。虽然如此,被打时她还是要哭。这主要是因为她觉得,被打时不哭,是不对的。我很欣赏她的达观态度。但要问我什么叫做“对”,什么叫“不对”,我就一点也答不上来了。 

我的故事又重新开始道:晚唐时节,薛嵩是个纨绔子弟,住在灰色、窒息的长安城里。后来,他听了一个老娼妇的蛊惑,到湘西去当节度使,打算在当地建立自己的绝对权威。但是权威这种东西,花钱是买不到的。薛嵩虽然花钱雇了很多兵,但他自己也知道,这些兵都不能指望。他觉得那个老妓女是可以指望的,但对这个看法的信心又不足。说来说去,他只能指望那个小妓女。这位小妓女提供了屁股和脊背,让他可以在上面抽打,同时自欺欺人地想着:这就是建功立业了。 

我该讲一讲那位老娼妇的事。她曾经漂泊四海,最后在长安城里定居,住在一座四方形的砖亭子里。那座亭子虽然庞大,但只有四个小小的拱门,而且都像狗洞那样大小。人们说:她并不是出卖肉体,而是供给男人一种文化享受。因为不管谁进到那个亭子里,都会受到最隆重的接待、最恭敬的跪拜,她总要说嫖客不是寻常人,可以建功立业。至于她自己,也有一番建功立业的决心。所有跟着薛嵩来到了这不毛之地。打算在凤凰寨里做一番前无古人的事业。但是薛嵩什么功业也没有建立,只是经常在她门前鞭打一位小妓女。这个老女人坐在纸门后面听着,心里恨的痒痒的,磨着牙齿小声唠叨着:姓薛的混蛋!我知道你想打谁!早晚要叫你知道我的厉害……这就是说,老妓女提供高档次的文化服务,这种服务不包括挨打。薛嵩敢对她作这种档次很低的暗示,自然要招致愤怒。 

4 

现在我又回到生活里。我在一座寺院里,更准确地说,是在这座寺院的东厢房里,面前是一座被砖头垫高了的香案。在香案底下是一捆捆黄色的纸。时逢盛夏,可以闻到霉味、碱味,还有稻草味;而稻草正是发黄的纸的主要成分。透过打开的窗子,可以看到院子里的白皮松。当你走进这所院子,会看到青色的砖墙,墙上长满了青苔;油灰开裂的庭住、肥大无比的白皮松──总而言之,是一座古老的庭院。相信你可以从中感觉到一种文化气氛。这就如在一千多年前,你走进那位老娼妇在长安城里的四角亭子。不管你从哪面进去,都要穿过一个又矮又长的门洞,然后直起身,仰望头顶深不可测的砖砌的穹顶。此时整个世界都压在你的头上,所以你也感到了这种文化气氛。在这个四方形的房间里,一共有四股低矮的自然光,照着人的下半截。后来,那个老娼妇匍匐着出现在光线里──她有一张涂得雪白的脸,脸上还有两条牦牛尾巴做的眉毛──声音低沉地说道:官人。不知你感觉怎样,反正薛嵩很感动。他到那个亭子里去过,感到自己变成了一个庄严肃穆的死人。我也不知那个老娼妇对他做了什么,反正从那亭子里出来,他就鬼迷心窍地想要建功立业,到荒蛮地方去做节度使,为大唐朝开辟疆土。考虑到当时薛嵩尚未长大成人,情况可能是这样的:那个老娼妇把他那个童稚型的男根握在手里,轻声说道:官人,你不是个等闲之人……等等。因为我从没有被感动过,可能想得不对。但我以为,从来就不会感动。是我的一项大资本。不管什么样的老娼妇拿着我的男根说我不同凡响,我都不会相信:但我也承认。有很多人确实需要有个老娼妇拿着他的男根说这些话。这也是薛嵩迷恋她的原因。我影影绰绰记得有一回领导忘了史料的出处,偏巧我记得,顺嘴提示了一下。他很高兴,说道:小王是人才嘛。我也振奋乐一小下,但马上就蔫掉了。 

对于薛嵩被拿住男根的事,需要详加解释:当时他躺在了亭子的中心,此地阴暗、潮湿,与亭子这个名称不符。薛嵩摊开双手呈十字形,躺在亭子的中央,头、脚和两臂的方向,都通向有个门洞,薛嵩好像躺在了十字路口。你也可以说,他自己就是那个十字路口。而这个路口所连接的四条路都很长,那些路的顶端,各有有个泄入天光的门洞,好像针孔一样,仿佛通往无尽的天涯。无论他往哪边看,都能看到遥远的天光,而且听到水滴单调地从穹顶滴落,有一些滴到了远处,还有一些滴到了他身上。假如他往天顶上看,在一片黑暗之中,可以看到几只大得骇人的壁虎在顶上爬动,并能听到遥远的风声和车马声。就在这一片黑暗和寂静中,出现了那老娼妇的脸,那张脸像墙皮一样刷得雪白,上面有漆黑的两道扫帚眉。她用像墓穴一样冰凉的手拿住了薛嵩的男根,开始说话(“官人,你不是个等闲之人”,等等)。薛嵩不禁勃起如坚铁,并在那一瞬间长大成人了。我读着自己旧日的手稿,同时在脑子里进行批判。做这件事有何意义,我自己都不明白。我很不喜欢现在这个写法,主要是因为,我很不喜欢有个老妓女用冷冰冰的手来拿我的男根,这地方不是谁都能来碰的──虽然在这种情况下,我也会勃起如坚铁,但我还是不喜欢。真不知以前那个我是怎么想的。

\section{第二章}

第一节 

我的故事还有一种开始,这个开始写在另一叠稿纸上。如前所述,香案上下堆了不少稿纸,假如写的都是开始,就会把我彻底搞糊涂──晚唐时,薛嵩在湘西的山坡上安营扎寨。起初,他在山坡上挖掘壕沟,立起了栅栏,但是只过了一个雨季,壕沟就被泥沙淤平,变成了一道环形的洼地,栅栏也被白蚁吃掉了。那些栽在山坡上的树干乍看起来,除了被雨水淋得死气沉沉,还是老样子;仔细一看,就看出它半是树,半是泥。碗口粗细的木头用手一推就会折断,和军事上用的障碍相差很远。因为白蚁藏在土里看不见,所以薛嵩认定,这山坡上最可恨的东西是雨水。 

旱季里,薛嵩从远处砍来竹子,要在壕沟上面搭棚子,让它免遭雨水的袭击,来解决壕沟淤平的问题。等他把架子搭好,去搜集芭蕉叶子,要给棚子上顶时,白蚁又把竹子吃掉了。薛嵩这才想到,山坡上最可恶的原来是白蚁。于是,他就扛起了锄头,要把山坡上所有上午白蚁窝都刨掉。这是个大受欢迎的决定,因为白蚁可以吃:成虫可以吃,蛹可以吃,卵也可以吃。特别是白蚁的蚁后,是一种十全大补的东西,但是白蚁的窝却被一层厚厚的硬土壳包着,很需要有人出力把它刨开。所以薛嵩扛着锄头在前面走,方圆三十里之内的苗族小孩全赶来跟在他身后,准备拣洋落──他们都知道,汉族人不知道怎样吃白蚁。而白蚁也动员起来,和薛嵩作斗争,斗争的武器是唾液。一分白蚁的唾液和十分土掺起来,就是很硬的土,一分唾液和三分土掺起来,就像是水泥,一分唾液掺一分土,就如钢铁一样坚不可摧。自然,假如纯用唾液来筑巢,那就像金刚石一样的硬,薛嵩连皮都刨不动。但是这样筑巢,白蚁的哈喇子就不够用了。 

薛嵩用锄头刨蚁巢的外壁,白蚁在巢里听得清清楚楚,就拼命吐唾沫筑墙;薛嵩的锄头声越近,它们就越拼命地吐,简直要把血都吐出来。所以薛嵩越刨,土就越硬;满手都起了血泡。最后他自己住手不刨了。白蚁用自己的意志和唾液保住了蚁巢,而那些苗族孩子看到薛嵩是这样的有始无终,都拣起地上的碎土块来打他,打得他落荒而逃。等到第二天早上,薛嵩又出现在红土坡上,扛着锄头,而那些苗族孩子又跟在他身后准备拣洋落。这件事周而复始,好像永无休止。这件事的要点是:一个黑黝黝的人,扛着锄头在红土山坡上奔走,搞不清他是被太阳晒黑的,还是被热风吹黑的。他想把所有的白蚁巢都刨掉,但是一个都没刨掉;还锛坏了很多锄头,打了很多血泡。事情为什么会是这样,薛嵩自己都不知道。 

我清楚地记得那片亚热带的红土山坡,盛夏时节,土里的砂砾闪着白光──其中有像粗盐一样的石英颗粒,也有像蝉翼碎片般的云母。这种土壤像砂轮一样,把锄头磨得雪亮。新锄头分量很重,很难使,越用越锋利,分量也就越轻。它变得越来越小,越来越薄,最后在锄头把的顶端消失了。在烈日下挥锄时,汗水腌着脖子,脖子像火鸡一样变得通红。着是否说明我就是薛嵩? 

在这个故事里,薛嵩在山坡上年复一年地忙碌,只留下了一些浅浅的土坑,还有一些被白蚁吃剩的半截柱子,雨季一到,这些柱子上长起了狗屎苔,越长越多,好像一些陆生的珊瑚。到雨季到来时,薛嵩急急忙忙地给自己搭了个小棚子来住,这种小棚子挡不住瓢泼大雨,所以里面总是湿漉漉的,而且雨下得丝毫不比外面小。久而久之,他脸上长了青苔,身上长满了霉斑,腿上得了风湿病,好像一棵沉在水底的死树。旱季一到,这个地方没有一棵树,又热得很,棚子里比外面似乎一点都不见凉快;薛嵩呆在棚子里,两眼通红,心情很坏。一阵风吹来,棚子立刻塌掉,因为支棚子的竹子已经被白蚁吃了,只剩下一层皮来冒充竹子。此时我们才知道,棚子里比烈日下还是凉快一些。像这样下去,薛嵩要么在雨季里霉掉,要么在旱季里被晒爆,这个故事就讲不下去了。 

后来有人告诉薛嵩,白蚁什么都吃,就是不吃活的草木,所以他就在壕沟边上种了一些带刺的植物,比方说,仙人章、霸王鞭之类,在栅栏所在之处载了几棵母竹,引山上下来的水一灌,很快就是葱茏一片──寨里寨外,到处是竹丛、灌木丛,底下沟渠纵横。从此,薛嵩被解脱了在山坡上刨蚁巢的苦刑。他就这样扎下了寨子,但他不像是大军的营寨,倒像一片亚热带的迷宫。从实用的角度来看,它的防御力量并不弱,因为在草丛和灌木丛里,有无数不请自来的蚂蚁窝和土蜂窝,还有数目不详的眼镜蛇在其中出没。除了猪崽子,谁也不敢钻灌木丛。但是薛嵩有一颗装满军事学术的脑袋,因为在“野战筑城”这一条目之下,出现了蚂蚁、土蜂、甚至猪崽子这样的字眼,薛嵩觉得自己彻底堕落了。既然已经堕落,再堕落一点也没有关系。所以他准许自己抢苗女为妻。 

在我的手稿中,薛嵩抢老婆的始末记载得异常的简单明快:薛嵩身强力壮,胆大妄为;他在树林里遇上了红线,后者正在射小鸟。他喜欢这个脖子上系着红丝带的小姑娘,马上就把她抢走了。至于抢法,也是非常简单:一手抓脖子,一手钳腿,把她扛上了肩头,就这样扛走了。红线尽力挣扎了一下,感觉好像是撞上了一堵墙:薛嵩的力气大极了。红线想道:既然落到了这样的手里,那就算了罢。她伏在薛嵩的肩头不动;在林间阴冷的潮气中,想着自己会遇到什么样的对待。这个讲法太过简单,这就是我不喜欢它的原因。 

2 

上古单调的色彩使我入迷。然而循这条道路,也就没有什么故事可写。在我的调色板上,总要加入一些近代人情的灰色──以上所述,是我现在对旧稿的一些观感──所以薛嵩抢红线的事,也不能那么简单:晚唐时,薛嵩到湘西做节度使,骑来了一匹白马,还带来了一伙雇佣兵。后来,他的马老了,这些士兵也想起家来。那匹马长了胡子,那些兵也经常哗变;薛嵩只好把缰绳从马嘴上解下来,放它到树林里自由走动,同时也放松了军纪,让那些雇佣兵去抢山上的苗女为妻。但他自己却洁身自好,继续用军纪约束自己。那些苗女的肤色像红土一样红,头发和眉毛因而特别黑。我好像也见过这样的苗女,并对她们怦然心动。 

此后薛嵩在寨子里踱步,走在篱笆间的小路上,忽然就会发现某家竹楼前面出现一个没见过的女人,正在劈柴或是捣米。这些篱笆是或粗或细的柴棒栽在地下,顶端长出了绿芽;那片红土的院子铺上了黄砂;那个陌生的女人肢体壮硕,穿着短短的蓑草裙子。见到薛嵩过来,站直了以后,转过身子,用手梳理头发。她把头发分作两下,从脸旁垂下来,遮住了乳房,转向薛嵩,和他搭话。苗女的眉毛像柳叶一样的宽,下颚宽广,嗓音浑厚有力──薛嵩也会讲些苗语,他们聊了起来。但就在这时,竹楼上响起了一声咳嗽,围廊上出现了一个男人。他是一个雇佣兵,是薛嵩的手下。他用敌意的眼神看着他们,那苗女就扔下薛嵩,去做她的工作。此时薛嵩只好像个穿了帮的贼那样走开,同时心里感到阵阵刺痛──要知道,他是节度使,在巡视自己的寨子啊。他继续向前走,浏览着各家的院子和里面的苗女,就像一个流浪汉看街边上的橱窗;同时也在回顾那个女人健壮的身体、浑厚的声音。最后他终于想到:别人都去抢老婆,假如自己不去抢一个,未免吃了亏。作为读者,我觉得这是个大快人心的决定。 

有关薛嵩那匹长胡子的马,可以事先提到,这匹马原来是白色的,后来逐渐变绿。这是因为它总在树林里吃草,身上长满了青苔。后来,马儿紧不住蚊虫的叮咬,常到泥坑里打滚,又变得灰溜溜的。它既吃草,也吃树叶子,吃出了一个滚圆的大肚子,像产卵前的母蝈蝈,不像一匹马。因为总在潮湿的地面上行走,它的蹄子也裂开了。总在丛林中行走,需要有东西把眼前的枝条拨开,所以它也长出了犄角。你当然知道我说的是什么:这匹马逐渐变成了一头老水牛,而且也学会了“哞哞”地叫。在湘西,到处都是水牛,只要你看到一蓬茂盛的草木,里面准有几头老水牛在吃草,其中有一头是马变的。这匹马就此失踪了。据说它原是一匹西域来的宝马良驹,在马市上值很多钱。薛嵩的情形也可以事先提到:他原是长安城里的富户,擅长跑马,斗蛐蛐,长着雪白的肉体;后来被晒得鬼一样黑,擅长担柴挑水,因为嚼起了槟榔,把满嘴的牙弄成像焦炭一样黑。凤凰寨里有不少这样的人物,其中有一个是薛嵩变的。但这是后来发生的事。当初发生的事是:薛嵩对凤凰寨里发生的变化──这变化之一就是他也要去抢一个老婆──虽然心生厌恶,但也无可奈何。 

薛嵩准许自己的部下抢苗女为妻,后来他想到,假如他自己不也去抢上一个就算是吃了亏。这件事非常的重要,因为它标志着薛嵩长大成人。在此之前,他是个纨绔子弟,不懂吃亏是件坏事。在此之后,他既然已经抢了一个女人,尝到了甜头,就不能再这样说。事先他做了不少筹划和准备工作,但是对这种强盗行径还是觉得很不好意思,所以是一个人去的。对这件事,我感到激动,怀着一颗贼心, 走进一片荒山,去猎取女人。这样的故事怎不叫人心花怒放……我可以看见那座荒山,土色有如铁矿石。也可以看到那些绿叶,鲜翠欲滴,就如蜡纸所做。我也可以听见自己的心在怦怦乱跳。我也可以看到那些女人,肤色暗红,长着圆滚滚的小肚子,小肚子下面是漆黑的毛……但是别的就一点也想不出,还得看看以前是怎么写的。 

3 

过去有一天,薛嵩赤身裸体地骑在那匹长胡子的光背马上,肩上扛着那条浑铁大枪,沿着红土小路,走进山上的树林。他在枪缨里藏了一把竹篾条,准备用它来捆抢到的女人,藏的很是牢靠,谁也看不出来。遇上了苗族的男人,他就红着脸对人家打招呼,此时他又觉得自己不是强盗,是个小偷。进山的道路不止一条,他走的是预先选好的一条,因为不少部落的人不分男女都有纹身,有些纹成蓝荧荧,有些纹得黑糊糊,除此之外,有些寨子里的小姑娘从小就嚼槟榔,把牙齿嚼得像木炭一样。总而言之,这条选好的路避开了这些姑娘,因为假如是这样的姑娘,就不如不抢。进山的路他倒是满熟的。每次寨里没有粮食,他就带人到寨里来,用盐巴换军粮。以免别人贪污;但在路上常被人一棍子打晕,醒来以后只好独自灰溜溜地回来。身为朝廷命官被人打了闷棍不甚光彩,只好不声张;听任手下人贪污。但若我是他,就一定会戴顶钢盔。 

走在这条路上,薛嵩遇到了不少苗族女人,有些太老,有些背着小孩子,都不是合适的赃物。一直走到苗寨边上,他才遇到了红线,这个女孩穿着一件蓑草的裙子,拿了一个弹弓在打小鸟。他打量了她半天,觉得这女孩长得满漂亮,尤其喜欢她那两条橄榄色的长腿,就决定了要抢她。薛嵩以前见过红线,只觉得她是个寻常的小姑娘;这是因为当时他没动抢的心。动了抢的心以后,看起人来就不一样。 

薛嵩从马背上下来,鬼鬼祟祟地走到她身边,把长枪插在地下,假装看林间的小鸟,还用半生不熟的苗话和她瞎扯了几句。忽然间,他一把抓住她的脖子,并且从枪缨里抽出一根竹篾条来。这时薛嵩心情激动,已经达到了极点。当时雨季刚过,旱季刚到,树叶子上都是水,林子里闷得很。薛嵩的胸口也很闷。他还觉得自己没有平时有劲。在恐惧中,他一把捂住了红线的嘴,怕她叫出声来──这个地方离寨子里太近了。与此同时,他也丧失了平常心,竹篾条拴着的东西胀得很大。奇怪得是,红线站在那里没有动,也没有使劲挣扎,只是脸和脖子都涨得通红。后来她猛地一扭脸说:你再这样捂着,我就要闷死了。薛嵩感到意外,就说:我是强盗、是色狼,还管你的死活吗?然后他又一把捂住红线的嘴。但是红线又挣开,说:这事你一点都不在行。捂嘴别捂鼻子──色狼也不是这种捂法!薛嵩说:对不起。就用正确──也就是色狼的方式捂住了她的嘴。他用两只手抓着她,就腾不出手来捆她,就这样僵持住了。实际上,薛嵩此时把红线搂在了怀里。但是天气热得很,不是热烈拥抱的恰当时刻。所以过了一会儿,红线就挣脱出来,说道:大热天的,你真讨厌!她上下打量了薛嵩一阵,就转过身去,先用手抿抿头发,然后把双手背过去说:捆吧。于是薛嵩把她捆了起来:用竹篾条绕在她的手腕上,再把竹篾条的两端拧在一起。据我所知,青竹篾条的性质和金属丝很近似。 

因为当地盛行抢婚,所以红线对自己被抢一事相当镇定。不过,她总是第一次被抢,心情也相当激动,禁不住唠唠叨叨,首先她对薛嵩用篾条来捆她就相当不满,说道:你难道连条正经绳子都没有吗?这使薛嵩惭愧地说:我什么都学得会,就是学不会打绳子。红线评论道:你真笨蛋──还敢吹牛说自己是色狼呢。她还说:下次上山来抢老婆,你不如带个麻袋,把她盛在里面。过了一会儿,她又补充说:当然,我也不希望你再有下一次。此时薛嵩从枪缨里抽出第二根篾条,蹲下身去,红线又把双脚并在一起,让他把脚捆在一起。薛嵩说:我没有麻袋,只有蒲包,蒲包不结实,会把你掉出来。就这样,薛嵩把红线完全捆好了。后者打量着拴在脚上的竹篾条,跳了一下说:他妈的,怎么能这样对待我!此时发生了一件更糟的事:薛嵩要去牵马,想把红线放到马背上驮走,但是那马很不像话,自己跑掉了。薛嵩只好自己驮着红线在山路上跋涉,汗下如雨,还要忍受红线的唠叨:连匹马都没有?就这么扛着我?我的上帝啊,你算个什么男人!直到薛嵩威胁说要把她送回去,她才感到恐惧,把嘴闭上了。 

后来,薛嵩就这样把红线扛进寨子,招来很多人看,都说他抢女人都抢不利索。薛嵩觉得自己很丢面子,闷闷不乐,性格发生了很大变化。他想让红线回到山上去,自己备好了麻袋、绳子,给马匹配好缰绳,再上山去抢一次。但红线不答应,她说自己是不小心才被抢来的,这样才有面子。假如第二次再被同一个男人抢到,那就太没面子了。她是酋长的女儿,面子是很重要的──甚至比命都重要。后来薛嵩让她学习汉族的礼节,自称小奴家、小贱人,把薛嵩叫作大老爷、大人之类,她都不大乐意,不过慢慢地也答应了。薛嵩在家里板起脸来,作威作福──这说明他当了一回抢女人的强盗以后,又想假装正经了。 

4 

有关薛嵩抢到红线的事,还有另一种说法是这样的:他不是在山上,而是在水边逮住了她。这地方离凤凰寨很近,就在薛嵩家后面的小溪边上。红线在河里摸鱼,身上一丝不挂,只有拦腰一根绳子,拴着一个小小的渔篓,就这样被薛嵩看到了。他很喜欢她的样子──她既没有纹身,也不嚼槟榔──就从树丛里跳出来,大叫一声:抢婚!红线端详了他一阵,叹了一口气,爬上岸来,从腰间解下鱼篓,转过身去,低下头来说:抢吧。按照抢婚的礼仪,薛嵩应该在她脑后打上一棍,把她打晕、抢走。但是薛嵩并没有预备棍子。他连忙跑到树林里去,想找一根粗一点的树枝,但一时也找不到。可以想见,假如薛嵩总是找不到棍子,红线就会被别的带了棍子的人抢走,这就使薛嵩很着急。后来从树林里跑了出来,用拳头在红线的脑后敲了一下,红线就晕了过去。然后薛嵩把她扛到了肩上,此时她又醒了过来,叫薛嵩别忘了她的鱼篓。直到看见薛嵩拾起了鱼篓,并且看清了鱼篓里的黄鳝没有趁机逃掉,她才呻吟一声,重新晕了过去。此后薛嵩就把她扛回了家去。 

自然,还有第三种可能,那就是薛嵩在树林里遇上了红线,大喝一声:抢婚!红线就晕了过去,听凭薛嵩把她抢走。但在这种说法中,红线的尊严得不到尊重,所以,我不准备相信这第三种说法。按照第二种说法,红线在薛嵩的竹楼里醒来,问他用什么棍子把她打晕的,薛嵩只好承认没有棍子,用的是拳头。此后红线就大为不满,认为应该用裹了牛皮的棒棰、裹了棉絮的顶门杠,最起码也要用根裹布条的擀面棍。棍棒说明了抢婚的决心,包裹物说明新郎对新娘的关心。用拳头把她打晕,就说明很随便。虽然有种种不满,但也后悔莫及。红线只好和薛嵩过下去──实际上,第二种说法和第一种说法是殊途同归。 

还有一件事,也相当重要:薛嵩把红线抢来以后好久,那件事还没有搞成。这是因为薛嵩有包皮过长的毛病。有一天,红线把他仔细考察了一番,按照他所教的礼节说道:启禀大老爷,恐怕要把前面的半截切掉;说着就割了薛嵩一刀,疼得他满地打滚,破口大骂道:贱人!竟敢伤犯老爷!但是过了几天,伤口就好了。然后他对红线大做那件事,十分疯狂,使她嘟嘟囔囔地说:妈的,我这不是自己害自己吗?经过了这个小手术,薛嵩的把把很快长到又粗又大,并且时常自行直立起来。这时他很是得意,叫红线来看。起初红线还按礼节拜伏在地板上说:老爷!可喜可贺!后来就懒得理他,顶多耸耸肩说:看到了──你自己就不嫌难看吗?但不管怎么说,这总是薛嵩长大成人的第一步。在此之后,薛嵩在寨子里也有了点威信。因为他的把把已经又粗又大,别人也都看见了。 

有关薛嵩抢到红线的经过,有各种各样的说法,这是最繁复的一种。假如说,这种说法还不够繁复,也就是说,它还不够让人头晕。在这个故事里,有薛嵩、有红线,还影影绰绰的出现了一些雇佣兵。这个故事暂时也这样放着吧。这样我就有了两个开始,这两个开头互相补充,并不矛盾。在这个故事里,男根、勃起,长大成人,都有特殊的含义。薛嵩在一个老娼妇面前长大成人,又在一个苗族女孩面前长大成人,这两件事当然很是不同。因此就可以说薛嵩不是一个人,是两个人。假如这样分下去,薛嵩还可以是三个人,四个人;生出无数的支节来。所以,还是不分为好。我很不喜欢过去的我这种颠三倒四的作风。但是,这一切都是过去做下的事,能由得了现在的我吗? 

第二节 

一切变得越来越不明白了。因为我的故事又有了另一个开始:作了湘西节度使以后,每天早上醒来时,薛嵩都要使劲捏自己的鼻子,因为他怀疑自己没有睡醒,才会看到对面的竹排墙。他觉得这墙很不像样,说白了,不过是个编的紧密的篱笆而已。在那面墙上,有一扇竹编的窗子,把它支起来,就会看到一棵木瓜树,树上有个灯笼大小的马蜂窝,上面聚了成千上万只马蜂,样子极难看,像一颗活的马粪蛋。就是不支开窗户,也能听见马蜂在嗡嗡叫。作为一个中原人,让一个马蜂窝如此临近自己的窗子,是一种很不容易适应的心情。他还容易想到要找几把稻草来,放火熏熏这些马蜂。这在温带地方是个行得通的主意,但在此地肯定行不通:熏掉了一个马蜂窝,会把全寨的马蜂都招来,绕着房子飞舞,好像一阵黄色的旋风,不但螫人、螫猪、螫狗,连耗子都难逃毒手。这说明马蜂在此地势力很大。当然,假如你不去熏它们,它们也绝不来螫你,甚至能给你看守菜园,马蜂认识和自己和睦相处的人。薛嵩没有去熏马蜂,他也不敢。但他不喜欢让马蜂住进自己的后院,这好像和马蜂签了城下之盟。 

他还不喜欢自己醒来的方式,在醒来之前,有个女孩子在耳畔叫道:喂喂!该起了!醒来以后,看到自己的把把被抓在一只小手里。这时他就用将帅冷峻的声音喝道:放开!那女孩被语调的严厉所激怒,狠狠一摔道:讨厌!发什么威呀!被摔的人当然觉得很疼,他就骂骂咧咧地爬起来,到园子里去找早饭吃。薛嵩和一切住在亚热带丛林里的人一样,有自己的园子。这座园子笼罩在一片紫色的雾里,还有一股浓郁的香气,就如盛开的夹竹桃,在芳香里带有苦味。那个摔了他一把的女孩也跟他来到这座紫色的花园里,她脖子上系了一条红丝带,赤裸赭橄榄色的身躯──她就是红线。红线跟在薛嵩后面,用一种滴滴达达的快节奏说:我怎么了──我哪儿不对了──你为什么要发火──为什么不告诉我──好像在说一种快速的外语。薛嵩站住了,不耐烦地说:你不能这样叫我起床!你要说:启禀老爷,天明了。红线愣了一下,吐吐舌头,说道:我的妈呀,好肉麻!薛嵩脸色阴沉,说道:你要是不乐意就算了。谁知红线瞪圆了眼睛,鼓起了鼻翼,猛然笑了出来:谁说我不乐意?我乐意。启禀老爷,我要去劈柴。老爷要是没事,最好帮我来劈。要劈的柴可不少啊。说完后她就转身大摇大摆地走开,到门口去劈柴。这回轮到薛嵩愣了一下,他觉得红线有点怪怪的。但我总觉得,古怪的是他。 

薛嵩后园里的紫色来自篱笆上的藤萝,这种藤萝开着一种紫色的花,每个花蕾都有小孩子的拳头那么大,一旦开放,花蕊却是另一个花蕾。这样开来开去,开出一个豹子尾巴那样的东西。香气就是从这种花里来。而这个篱笆却是一溜硬杆野菊花,它们长到了一丈多高,在顶端可以见到阳光处开出一种小黄花,但这种花在地面上差不多是看不到的,能看到的只是野菊花紫色的叶子,这种叶子和茄子叶有某种相似之处。在园子里,有四棵无花果树,长着蓝色的叶子,果实已经成熟,但薛嵩对无花果毫无兴趣。蓝色无花果挂了好久,没有人来摘,就从树上掉下去,被猪崽子吃掉。在园子里,还长了一些龙舌兰,一些仙人掌,暗紫的底色上有些绿色的条纹,而且在藤萝花香的刺激下,都开出了紫色的花朵。薛嵩认为,这些花不但诡异,而且淫荡,所以他从这些花旁边走了过去,想去摘个木瓜吃。木瓜的花朴实,果实也朴实。于是他就看到了那个马蜂窝。这东西像个悬在半空的水雷,因为现在是早晨,它吸收了雾气里的水,所以变得很重,把碗口粗细的木瓜枝压弯了。大树朝一边弯去。到中午时,那棵树又会正过来。这个马蜂窝有多大,也就不难想象。但这个马蜂窝还不够大。更大的马蜂窝挂在别的树上,从早上到中午,那树正不过来,总是那么歪。t靮 

马蜂窝是各种纤维材料做的,除了枯枝败叶,还有各种破纸片、破布头,所以马蜂窝是个不折不扣的垃圾堆。天一黑,它就会发出一种馊味,能把周围的荧火虫全招来。这时马蜂都回巢睡觉了,荧火虫就把马蜂窝的表面完全占据,使它变成一个硕大无朋的冷光灯笼;而且散发着酿醋厂的味道。众所周知,荧火虫聚在一起,就会按同一个节拍明灭。亮起来时,好像薛嵩的后院里落进了一颗流星,或者是升起了一个麻扎扎的月亮;灭下去时,那些荧火虫好像一下都不见了,只听见一片不祥的嗡嗡声。假如此时薛嵩正和红线做爱,不知不觉会和上荧火虫的节拍。此时他觉得自己变成了一只绿壳甲虫,在屁股后面一明一灭。荧火虫的光还会从竹楼的缝隙里漏进来,照着红线那张小脸,还有她脖子上束着的红丝带,她把上半身从地板上翘起来,很专注地看着薛嵩。──我说过,感到寂寞时,薛嵩就把红线抱在怀里。但他总觉得她是个小孩子,很陌生──在这光线之下,红丝带会变成黑色。她的上半身光溜溜、紧绷绷的,不像个女人,只像个女孩。她那双眼睛很专注地看着薛嵩,好像不知道自己在干什么。过了好久,她好像是看明白了,大声说道:启禀老爷,你是对眼啊,然后放松了身体,仰倒在竹地板上,大声呻吟起来。不知为什么,这使薛嵩感觉很坏,也许是因为知道了自己是对眼。红线的乳房紧绷绷、圆滚滚,这也让薛嵩不能适应;在这种时刻,他常常想到那个老妓女那口袋似的乳房──老妓女又从不说他是对眼。等到面对老妓女那口袋似的乳房,他又不能适应,回过头来想到红线那对圆滚滚的乳房,还觉得老妓女总是那几句套话,实在没意思。如此颠来倒去,他总是不能适应。不管怎么说,让我们暂且把薛嵩感觉很坏的事情放一放。那天早上,薛嵩到园子里摘木瓜,忽然遭人暗算,被砍了一刀,失掉了半个耳朵──不仅血流满面,而且永久地破了相。假设这才是故事真正的开始,则在此以前的文字都可以删去。 

2 

现在来说说薛嵩怎样被砍去了半个耳朵。那天早上他到树上去摘个木瓜,路过水塘边。这园子里还有甜得发腻的无花果,有奶油味的木菠萝,但是薛嵩不想吃这种东西,觉得吃这种果子于道德修养有害。红线喜欢吃半生不熟的野李子,黄里透青的楂子。这些果实酸得叫人发狂,薛嵩也不肯吃。说来说去,他就喜欢吃木瓜。这东西假如没熟透,简直一点味都没有,就算熟透了,也只有一股生白薯味;吃过以后,嘴里还会有一股麻木的感觉。这就是中庸的味道。我总不明白薛嵩怎么会爱吃这种东西──也许他是假装爱吃。不管怎么说,他是个节度使,总是假装正经才行。 

这水搪是薛嵩和红线的沐浴之所,塘里还有一大片水葫芦,是喂猪的,开着黄蕊的白花。除了水葫芦,还漂着一大蓬垃圾──枯枝败叶、烂布头一类的东西。这个水塘通着寨里的水渠,垃圾可以从别处漂过来。薛嵩觉得恶心,用随身带着的铁枪想把它挑出去。也不知是为什么,那东西好像在水里有根,挑不起来。他就把它拨到塘边来,俯下身去,准备用手把它揪出来;就在这时,他看到垃圾中间竖着一节通气的竹管,还看到昏昏糊糊的水下好像有个人的身体──那池里的水是绿色的,大概其中有不少单细胞藻类──他先是一愣,然后猛醒,伸手去拔插在身后地上的铁枪。但已经迟了,眼前水花飞溅,水里钻出一个人来,满脸的水都在往下流,好像琉璃做成,双腮鼓起,显得很是肥胖。那刺客先喷了他一脸水,然后“飕”地给了他一刀。水迷了薛嵩的眼,在这种情况下挨刀砍,实在危险得很。好在对方刚从水里钻出来,眼睛里全是水,也看不大清,没把他的脑袋认准,只把半个耳朵砍了下来;假如认准了,砍下的准不止是这些。因为耳朵里软骨,所以薛嵩感到哗啦的一下,以后薛嵩往后一滚,拿了铁枪、抹掉脸上的水,要和这个刺客算帐,已经来不及了。那人一半滚一半爬、一半水一半陆,到了树篱边上,钻到一个洞里去,不见了。想要到树林去追敌人显然是徒劳的,那里面密密麻麻,连三尺都看不出去。此时薛嵩端平了大枪,满脸流着血和水,心情很是激动。 

这种激动无处发泄,薛嵩就大吼起来了。而红线正在竹楼前面劈柴,听到后院里有薛嵩的吼声,急忙丢下了柴火,手舞长刀赶来,嘴里也发出一阵呐喊来呼应薛嵩。这一对男女就在后园里连喊带舞,很忙了一阵子。最后红线问薛嵩:人呢?薛嵩才傻愣愣地说:什么人?红线说:砍你那个人──你要砍的人。薛嵩说:跑了。红线说:跑了还喊啥,快来包包伤口吧。于是薛嵩就和红线回到竹楼里去,让她包扎伤口;此时才发现左耳朵的很大一部分已经不见了。在这种情况下,当然会很疼,但薛嵩首先感到的是震惊──不管怎么说,他总是朝廷任命的节度使,是此地的官老爷。连他都敢砍,这不是造反吗? 

红线给薛嵩包扎伤口,发现耳朵残缺不全,也很激动。这是因为薛嵩是她的男人,有人把该男人的一部分砍掉,此事当然不能善了。所以她不停地说:好啊,砍成这个样子。太好了。这话乍听起来不合逻辑,但你必须考虑到,红线原来是山上的一个野姑娘,她很喜欢打仗。既然薛嵩被砍成了这样,就必须打仗,所以她连声叫好,表示她不怕流血,也不怕战争。假如说,砍成这个样子,太惨了,那就是害怕流血,害怕战争,这种话勇敢的人绝不会说。只可惜薛嵩不懂这些,他听到红线这样叫好,觉得她狼心狗肺,心里很不高兴。 

3 

薛嵩家的后园里有一个池塘,塘边的泥岸上长满了青苔。那一池水是绿油油的颜色,里面漂着搅碎了的水葫芦,还有一个惨白的碎片,好像一个空蛋壳,仔细辨认后才发现它原是薛嵩的半个耳朵。薛嵩把它从水里捞了出来,拿在手里看了很久,才相信自己身体的这一部分已经永远失去了。古人曾说:身体发肤,受之父母,不能轻易放弃,所以薛嵩就该把这块耳朵吃下去,但他觉得有点恶心,还觉得自己已经沦落到了食人生番的地步──所以他又把耳朵吐了出来。后来他用铁枪掘了一个坑,把耳朵葬了进去,还是觉得气愤难平,就平端着长枪,像一头河马一样吼叫着。假如此时红线按照他要求的礼节说道:启禀老爷,贼人去远了,请保重贵体。那还好些。偏巧这个小蛮婆心情也很激动,满腹全是战斗的激情,就大咧咧地说:人家都跑没影了,还瞎嚷嚷什么?还不想想怎么去捉他?这使薛嵩很是恼火,顺口骂道:贱婢!全没有个上下。没准这贼和你是串通一气的。红线不懂得玩笑,把刀往地下一摔,说:混帐!怪到我身上来了!这就使薛嵩更加气愤:有把老爷叫混帐的吗?忽然他又想到影影绰绰看到那个刺客身上有纹身,像个苗人的样子,就脱口而出道:可不是!那个刺客正是个苗子!十之八九和你是一路。你要谋杀亲夫!顺便说一句,苗子是对苗人的蔑称,平时薛嵩绝不会当着红线这么说,这回顺嘴带出来了。更不幸的是它和后一句串在了一起,这使红线更加气愤,从地下捡起刀来,对准薛嵩劈面砍去道:好哇!要和我们开仗了!老娘就是要谋杀你这狗屁亲夫!当然,这一刀瞄得不准,砍得也不快,留给薛嵩躲开的时间──红线并不想当寡妇。但她的战斗激情也需要发泄,所以就这么砍了。需要指出的是,红线和薛嵩学了一些汉族礼节,薛嵩也知道了一些红线的脾气。双方互相有了了解,打起架来结果才会好。假如没有这样的前提,这一刀起码会把他的另一只耳朵砍掉。这样薛嵩就没有耳朵了。 

后来,薛嵩向后退去,一步步退出了院门,终于大吼一声:小贱人!说是苗子砍我你不信,你就是个苗子,现在正在砍我!说着他就转身跑掉了。假如不跑的话,红线就会真的砍他的脑袋,而且她就会真的当寡妇了。对此必须补充说:薛嵩当时二十三岁,红线只有十七岁。这两个人合起来才四十岁,在一起生活,当然要吵吵闹闹,把一切搞得一团糟。 

有关薛嵩被刺的经过,还有一种说法是这样的:薛嵩家的后院里,有一个水池,是他和红线戏水之所。这座池子清可见底,连水底铺着的鹅卵石都清晰可见,因为水清的缘故,这水池显得很浅,水面上的涟漪映在水底,好像水底紧贴在水面上。清晨时分,薛嵩从水边经过,看到水里躺着一个女人,像雪一样白,像月亮一样发亮,这一池水就因此像蚌壳的内侧,有一种伸手可及的亮丽。后来,她从池底开始往上浮──必须说明,这池子其实很深,只是看不出来罢了。薛嵩看到她左手曲在身前,右手背在身后,眼睛紧闭着;而两腿却岔开着,呈人字形,细细的水纹从她身上滑过。必须承认,她是一位赤身裸体的绝代佳人,但是生死未卜,因为在她的口鼻里没有冒出一个气泡。薛嵩当然愣住了,看着这个女人,在寂静中,她浮上来,离薛嵩越来越近。在她的小腹上,有一撮茵茵的短毛,显得很俏皮,也离薛嵩越来越近;薛嵩也就入了迷,只是她眼睛紧闭,好像熟睡着。她醒来以后会是怎样,这是一个谜。 

后来,她嘴上出现了一缕微笑,好像一滴血落在水里,马上散成缕缕血丝。猛然间她睁开了眼睛,眼睛又大又圆,这使薛嵩为之一愣。然后她就突出水面,挥起藏在身后的右手,那手里握了一把锋利的刀,白若霜雪,朝薛嵩的头上挥来,所幸他还有几分明白,及时地躲了一下,只把半只耳朵砍掉了。假如不躲,后果也是不堪想象。然后,这个女刺客就逃掉了,仿佛消失在白色的晨雾里。只剩下薛嵩,呆站在水边发愣:他觉得,总有什么事情搞错了。像这样一个女人,根本不该来刺杀我,而是该去刺杀别人。至于搞错了是好是坏,他还有点搞不清楚。这种说法太过亮丽,和上一种说法也是大同小异。总而言之,那个刺客跑掉以后,薛嵩和红线起了争执。薛嵩非要说砍他一刀的是个苗子,红线不喜欢他这么说,两人就打了起来,但也不是真打。然后薛嵩就出去招集他的军队,要征讨那些苗人──假如苗女真是这么漂亮,的确需要征讨。 

在万寿寺里,面对着那份待填的表格,我终于想了起来,我们是社会科学院的历史研究所,在万寿寺里借住。这份表格是我们在年初交的工作报告。年底时还要交一份考绩报告──好在现在距年底还有一段时间。这是因为我们是国家级的研究单位,制度严明,还因为我们的领导──也就是那个穿蓝制服的人──很是古板。他总让我们做重大的、有现实意义的题目。什么叫作重大,我不知道。现实意义我倒是懂的。那就是不要考证历史,要从现代考起。举例来说,我不该去考据历史上的男子性器,而是应该直接从他的性器考起……但我今年的题目改成《本所领导性器考》,显然不够恰当。假如我真做这个题目,他可能会来砍我一刀。 

顺便说一句,我影影绰绰记得《冷兵器考》的一些内容。上古时,人们伐巨木为兵,到了中古才用大刀长矛。宋元时人们爱用刀剑,到了明清以降,最长的家伙不过是短刀。根据史书记载,清末的人好用暗器,什么铁莲子、铁菩提,还有人发射绣花针。根据这种趋势,未来的人假如还用冷兵器,必然是发射铁原子组成的微粒,透过敌人的眼底,去轰击他的神经中枢──我总觉得这是中规中式的一篇历史论文,不知为什么要给我打问号……说实在的,我有点想去砍他一刀。这不是因为我脾气坏,而是因为连《性器考》这样的题目,我现在都想不出来了。 

除此之外,我再想不起别的。由此可见,丧失记忆这种游戏有这样的规则:没有适当的提示,我什么都想不起来。有了适当的启示,最好是确凿的证据,我就会什么都想起来。举例来说,我原本不知自己在什么地方,还不知道自己是干什么的。但当一位领导带着指示出现在我屋里时,这些问题就迎刃而解了……最好这位领导能告诉我,我该去考些什么。受此启示,我又到院子里走动,太阳越升越高,直射着地面,院子里的臭味也越来越犀利:它带有琉黄气、腐尸气,近似于新鲜的人屁,又像飞扬的石灰粉,刺激着我的鼻孔。和屋顶琉璃瓦的金色反光混为一体。 

我并不喜欢闻这种臭味──不管琉黄、腐尸还是人屁,都不是我喜欢嗅到的东西。我也不喜欢有人往我鼻子里洒石灰。但我总觉得这种臭气里包含着某种信息,催我想起些什么来。 

第三节 

对于我的过去,现在我有了一种猜测:我好像是个玩世不恭的家伙,或者说,是个操蛋鬼。没人告诉我这件事,是我自己猜出来的。虽然说起来不够好听,但我对此深感欣慰。这种猜测是从阅读这篇手稿得来的;作者信口开河,自相矛盾,前面这样写,后面又那样写,好像不是个负责的人;既然我是这样的人,就不必去理睬重填表格的要求。说实在的,我也不知该填点什么才好。再说,倘若我过去是个严肃认真的老学究,按我现在的情形,想当个学究,还真做不来哩。 

过去有一天,薛嵩被人砍了一刀以后,流着血跑到那个老妓女家里去要他的武装,准备征讨山上的苗人──这样一来,就续上了第一章的线索。按照大唐的军事惯例,营妓要给将帅保管东西,就如今天的人,有钱不放在家里,而是放在小蜜的手里。薛嵩一切重要的东西都放在那个老妓女(她该叫作老蜜)的房子里,包括他的铠甲、弓箭和印鉴。那女人把它重重包裹,放在了箱子里。为了让自己良心得到安宁,他也给了小妓女一把没鞘的旧宝剑,她就用它在后园里挖蚯蚓来钓鱼。这把剑用来劈柴太钝,也太轻,所以只能挖蚯蚓。后来它就生了锈,变成了红色,好像一条赤练蛇。他还送给过她一把折扇,她用它来打蚊子,很快把扇骨打断,变了乱糟糟的一堆破烂。他急匆匆地跑来要武装,就如一个人清早起来跑到银行门口等待,想要取出自己的存款,有急用。有一些银行会因为门口等了这种顾客而急于开门,这就是那个小妓女。她慌慌张张地赶来,拿来了薛嵩的旧宝剑。那把剑的样子很不怎么样,而且也没有鞘。说实在的,薛嵩把它交给小妓女来保管,就是不准备要了。他把那剑拿了一会,就把它扔在屋檐下边了。还有些银行却因为这种顾客而不急于开门,她就是那个老妓女,她的动作慢慢吞吞;慢慢地找钥匙,又慢慢地开箱子,并且时时回顾薛嵩。薛嵩头上馋了白布,好像一个阿拉伯人,但他光着屁股,这一点有不像了。那个小妓女心情激动,围着他团团打转,因为紧张,她的乳房又在胸前并拢,好像一对拳头。 

与此同时,薛嵩还在大吼大叫,好像一个火车头;终于招来一些雇佣兵。他告诉他们,有个苗子躲在他家的后院里,砍了他一刀,砍掉了他的耳朵;他要上山去征讨,那些兵就胡乱起哄道:好啊,好。太好了。这些人说太好了,而且不是说要打仗好,而是说薛嵩掉了耳朵好。但他一点不发火。薛嵩就像他的把把,见了女人才发威。他一叠声地催促老妓女把真正的武装拿出来,那些东西是:贴身穿的麂皮衣服,麂皮外面穿的锁子甲,锁子甲外穿的皮甲,皮甲外面穿的铁叶穿成的重铠甲,还有头盔、面甲,脚下穿的镶铁片的靴子,重磅的弓、箭等等。他准备把这些东西都穿戴到身上,骑上白马到山上去,除了要给苗人一些厉害,还要给他们一次威武的时装表演──他简直急不可耐──我想这是因为他曾在一个苗族女孩面前长大成人,耀武扬威。总而言之,薛嵩的这些毛病,全都是红线惯出来的。 

那个老妓女最后终于开了箱子把那些东西拿了出来。出乎薛嵩的意外,这些武器的状况很糟糕。实际上,无论是兵器还是甲胄,都需要养护;而那个老妓女什么都没干。仅举一件东西为例,锁子甲锈得粘在了一起,像一块砖头,至于那些皮衣,上面的绿霉层层隆起,简直像些蘑菇。还有一个最严重的问题,就是薛嵩的战马很难找到。从理论上说,它还在寨里,假如它没有被偶尔来闲逛的豹子吃掉。但也不知到哪里去找。有一件事必须预先提到:任何一件会走的东西迷失在寨子里以后,假如它不想出来,都很难找到,因为这寨子是大得不得了的一片林薮;不管他是一个人,或是一匹马,或者别的什么东西。都在这个故事里很重要。还没有出征就遇到了这些困难,这使薛嵩更加愤怒,恶狠狠地瞪了那老妓女一眼,该女人有点畏缩,躲到后面去了。现在薛嵩面临着一个问题:怎么把这块红砖和蘑菇穿上身去。鉴于盔甲的现状,有人建议薛嵩别穿它了,手里拿一个藤牌遮挡一下就可以。在这种情况下,当然就不能使长枪。提这个建议的人说,薛嵩不必用枪,可以拿把单手用的长刀。这主意也被否定了。虽然它有显而易见的好处,既轻便,又凉快。后来他们把锁子甲挂在树上用棍子打,打落了一大堆红锈,勉强可以穿,但穿上还是很不舒服。薛嵩还需要一匹坐骑,假如那匹马还是找不到,那就只好骑水牛,一位重装武士骑在牛背上,那样子简直是无法想象。在这种情况下,薛嵩还会不会上山征讨苗人还是一个谜。所幸出现了一个奇迹:这个畜牲自己出现了在大路上,而且基本上还像匹马,不像牛。于是它就被逮住,套上了缰绳。现在薛嵩松了一口气,拿眼光去搜索那个老妓女。假如他今天不能出征,就不能不办那老妓女玩忽职守,没有养护军械的最。按照军纪,这就不但要打那老妓女四十军棍,还要用箭扎穿她的耳朵,押着她游营。薛嵩很不想这样办这个女人──这是因为,他曾在这女人面前长大成人。以前我写过薛嵩是在红线面前长大成人,但现在薛嵩和红线打翻了,他就不承认有这回事。好在薛嵩已经长大成人,过程也就无关紧要。 

如前所述,这个老妓女想要在凤凰寨里作一番事业,在她的事业里,薛嵩有很重要的地位,但这毕竟是她的事业,不是薛嵩的事业。所以她就没有好好保管薛嵩的武装,假如他再迟一段时间来要,这些东西通通要报废。虽然有种种不愉快,但结果还算好。薛嵩终于穿戴整齐,骑上了他那匹捣蛋的马(它很不想让薛嵩骑上),这时他的兵也武装了起来,但武装得不十分彻底──兵器多数人是有的, 穿甲的人却很少,把甲穿全了的一个也没有, 因为天气实在热──就这样到了出征的时刻。不言而喻,到山上去征讨苗人,才是真正难办的事情。苗人武勇善战,人数又多。但薛嵩觉得自己可以打胜──看来红线惯出的毛病可真不小啊。 

随着薛嵩的口令,那些兵站起队来,队形像一条蚯蚓。因为盔甲里太热,薛嵩无心把队伍整理好,想早点走──真要去整也未必整得动。那个年老的妓女浓妆艳抹,站在马前,用扇子着脸,拖着长声吟道:早早得胜归来。这既不是军规,也不是礼仪,而是营妓的传统。薛嵩很感动,同时把戴着头盔的头转到年轻的营妓所居的房子,看到她在门廊上,倚着柱子站着,什么都没有穿,也没戴假发;既裸露着整个身体,又裸露着娃娃式的头,表情专注。发现薛嵩在看她,她就挺直了身子,朝他飞了一吻。薛嵩不懂她是什么意思,或者因为他已准备出征,不便懂得,所以装作不懂。这种表示远不能令人振奋。后来他们就出发了。 

当这队人马从寨子中间通过时,有一粒石头子打在薛嵩的头盔上。他朝石头来的方向转过头去,看到红线站在路边。她做着一个奇怪的姿势:右手横擎着一把长刀,刀口朝外;左手掌向下按着,正好在自己阴毛的高度上,与此同时,她横向跳动着,嘴里“嘟嘟”地叫。这是苗族人挑战的姿势──如果你是个苗族人,见到这个姿势不上前应战,就是承认失败──但薛嵩不知道这些,他径直走开了。红线也不知道薛嵩不知道这些,她收起了长刀回家去。她甚至还觉得薛嵩很大度,有点感动了。 

2 

看来,我的故事写了很多年还没有写完,我找来找去,找到的都是开始,并无结束。我猜是因为有很多谜一样的细节困惑着我。比方说,这个故事为什么要发生在亚热带的红土山坡上。那里有一种强迫人赤身裸体的酷暑,红土也有一种令人触目惊心的颜色。这是一种跨越时空的诱惑,使我想要脱掉衣服,混迹于这团暑热之中。但真的混迹其中,我又会怀疑是否真的有好感觉。我虽然瘦,但也很怕热。还有红线,她的皮肤是古铜色或者是橄榄色的。当她呆在凤凰寨的绿荫里时,就和背景混为一体。因为这个缘故,她在脖子上系了一条红丝带。我很喜欢这女孩,但我也怕人拿刀坎我,所以假如她对我嘟嘟叫,我马上就缴械投降。还有那个小妓女,她的眼睛很大,虽然是长脸,但有一个浑圆的下巴,站在一个男人面前时,不会用手掌去抚摸他的胸膛,却会用手背去触他;但面对勃起的男性生殖器时,却毫不犹豫地伸手去拿。我也喜欢她。我决不会打她。还有内心阴暗的老妓女,时而暴躁、时而压抑的薛嵩──这两个人我一点都不喜欢,尤其是后者。要是我,就决不把他们写成这样。你大概从这个故事里看出了一点推理小说的痕迹。这种小说总有一个谜,而这个谜就是我自己。这个故事会把我带到一个地方,但我还不知道那是哪里。 

在我的故事里,薛嵩出发去打苗寨,出了寨子,他发现身后跟了几十个人,他可没指望会来这么多。所以他很是感动,觉得这些兵还不坏。当然,这些兵不像他那样武装整齐,谁也没穿铠甲,有些人拿了藤牌,有些人拿了根棍子,有人拿了把长刀。还有人什么都没有拿,他们的队伍在路上漓漓拉拉拖了很长,根本就不像要打仗的样子。薛嵩问那个赤手空拳的人为什么空着手,那人笑了一声,答道:空着手逃起来快些。这种答案能把任何统帅气死,但薛嵩对这种事已经习惯了,一点都不生气,他还说:带什么无关紧要,来了就好。但他可没想到这些兵都在背地里合计好了,只要苗人一出来应战,就把薛嵩押到前面和苗人拼命。等到苗人把薛嵩杀死,他们马上就和苗人讲和──这件事并不困难,他们和苗人是姻亲嘛。此后这寨子就是他们的了。从这个情况看来,薛嵩不大可能从山上活着回来。但事有凑巧,出了寨子不过五里地,他就从马上一头栽了下来。这原因很简单──中了暑。当时气温有四十度,穿上好几重铁皮,跑到太阳下去晒,不可能不中暑。这就打破了雇佣兵们的计划,他们只好把他扶在马上驮了回来。在此之前,他们也合计了好久,讨论要不要把薛嵩丢在那里,结论是:不把他弄回来不好交待──当然是不好向红线交待。红线是酋长的女儿,最好别得最。他们把晕倒的薛嵩载回家里,扔到竹楼门口,喊了红线一声,就分头回家去了。现在薛嵩和红线在一起,整个故事当然就按红线的线索来进行了。 

如前所述,红线一听说薛嵩嘴里说出“苗子”,就和他翻了脸,用刀来劈他,而且还舞着刀追赶薛嵩,但是追到院门口,看到有些木柴没有劈好,就劈起柴来;劈了一会柴,又想起薛嵩要去打她的寨子,就赶出了向他挑战,见他不应,又回家去劈柴。就这样往返奔走着。这说明她年纪虽小,但还是个居家过日子的人,心里是有活儿的;还说明她没把薛嵩和他那几个兵看在眼里──苗寨里人很多,而且人人都能打仗,他们去了以后,很快就都会被打翻在地。我们说过,红线是酋长的女儿,地位尊贵。她觉得因为她,也没人敢杀薛嵩,就是揍他也会有分寸;所以她既不为苗寨、也不为薛嵩操心,她可没想到薛嵩会在路上中暑。 

3 

家里有一件事,薛嵩和红线都没有想到:早上向薛嵩行刺的刺客并没有跑掉,他就躲在附近的树丛里,等到家里没有人了,他就溜了出来,打算潜进竹楼,找个地方躲起来,以便再次行刺,但刺客也有没想到的事,就是后园里木瓜树上的马蜂窝。那些马蜂早上就发现园里进来了生人,但因为露水打湿了翅膀飞不起来,就没有管这件事。到了将近正午时分,它们的翅膀早就干了,此人又从木瓜树下经过,那些有刺的昆虫就一轰而起,把他团团围住。那位刺客想到了跳进水塘去躲避,水塘又近在咫尺,但已经来不及了,这种热带的野蜂螫人实在厉害。总之,红线回家时,看到野蜂在飞舞,木瓜树下倒了一个人,已经休克了。从他携带的利刃来看,正是早上那位刺客。红线就取来薛嵩吊龟头的就便器材,把他捆了起来,然后把他拖到竹楼底下,用芭蕉叶子把他遮住,不让马蜂再螫他。然后她跑上竹楼,给自己弄了点饭吃;又跑下来,撩起芭蕉叶子,看那个昏倒的人。那人没有要醒的意思,只是像水发的海参那样在胀大。红线觉得这是个好现象,人被螫以后,长久的晕迷不是件坏事。倘若立刻醒来,倒可能是回光返照。当然,他也可能醒过来,但装作没有醒,在转逃走的主意。这也不成问题。因为他被螫得很重,已经跑不了啦。红线看清了这一点,又爬上竹楼去玩羊拐。但马上又跑回来,撩开芭蕉叶子,跨在那男人身上,用热辣辣的尿浇他,并且说道:“大叔,你别见怪,尿可以治虫伤啊。”这句话用汉语和苗语说了两遍,谅他一定可以听懂。然后她把此人盖好,又回楼上去玩。过一会她又回来,喝斥那些飞舞的马蜂说:去!去!回窝里去!又过了一会,因为天气热,浇上去的尿很快发了酵,刺客身上骚味很大, 马蜂都被熏跑了。看到这个情景, 红线又放了心,回到竹楼上,但一会儿又要跑下来……总而言之,红线心情激动,一刻也不能安宁。她当然是盼着薛嵩早点回来, 看看这个刺客。显而易见,刺客不是苗族人,而是汉族人,有眼睛的都能看见,此人身上的纹身是画出来的。她觉得这可以使薛嵩消除对苗人的偏见──她当然不能体会薛嵩要教化她和她的同族的好心。 

最后,薛嵩终于回来了。但他人事不知,从甲缝里流着馊汤,像一只漏了的醋桶。直到卸去衣甲、身上被泼了好几桶水,才醒过来。在醒来之前,薛嵩身上起了无数鲜红色的小颗粒,是痱子。因为他的样子很是狼狈,那些士兵帮了几把手就溜掉了,把他交给红线去弄──主要是怕他醒来老羞成怒,找他们的毛病。红线把他弄醒以后,又用腌菜的酸水灌他,灌过以后,在屋里来回跑动,坐卧不安,终于引起了薛嵩的注意。他支起身子来说:你怎么了?幸灾乐祸吗?红线说:你这样想也可以;就领他下楼去,请他看那个芭蕉叶遮着的人。虽然他肿得像一匹河马,但薛嵩还能认出就是早上那位刺客。这使薛嵩也很兴奋,这是因为在战场上俘获了敌方将士,除了劝其投降,就只能砍头示众。出于对军人这一职业的敬重,绝不能滥用刑法。但对于潜入己方营寨的奸细、刺客,就不受这种限制。所以这个人是个难得的机会,可以用酷刑来拷问。不管是在战场上还是营寨里,薛嵩都没俘获过敌人,这是第一回。说实在的,这个敌人也不是他俘获的,但他把这件事忘了。薛嵩从芭蕉树上扯下一片叶子,让红线以竹签为笔,口授了一个清单,都是准备对此奸细施用的刑罚: 

一:用皮绳把他仔细地反绑起来,同时鞭大起码一百下; 

二:用竹签刺他的手心和足心,肘关节和膝关节内侧,各扎一百下,每一下都以见血为度;然后敷上辣椒和盐的混合物; 

三:用打结的线把他的整个屁股和嘴巴都缝起来,并把他的包皮牢牢地缝在龟头上…… 

那个刺客听着听着,猛地翻了一个身,说道:不要折磨爷爷!我招供了。红线听了,觉得不过瘾,就劝他道:大叔!你这样很没有意思。别招供嘛。但他不肯听,执意要招供。红线对此很不满,后来她和那位小妓女聊天时说:你们汉族人真没劲。在杀掉那个刺客时,她和这位小妓女都在圈外看着。人是她逮来的,杀人时却不让她插手,这让她很不满意。 

她还说,在苗族人那里,假如有人去刺杀首领,失手被擒,为了表示对勇士的敬意,就要给他安排一场虐杀。所有的刺客被擒后,最关心的就是这个。倘若得到一种万刃穿身的死法,就会感到很幸福,要是一刀杀掉,死都没意思。照她看来,薛嵩所列的单子,不过是刚刚开始有点意思,那刺客就支持不住了。她这样地攻击汉族人,那个小妓女还是无动于衷,仿佛她不是汉族人。红线说起这件事,两眼瞪得圆滚滚,看上去虎头虎脑,这女孩觉得她很有趣,就伸手去搂她──妓女都有点同性恋倾向。出于礼貌,红线让她抱了一会儿,然后从她腋下挣脱了──写来写去,写出了女同性恋,我还不知道自己是这么爱赶时髦。 

4 

如前所述,这个刺客还有可能是个亮丽的女人。在薛嵩去征讨苗寨时,她又潜入薛嵩的竹楼,被红线逮住了。因此而发生的一切就很不同。等到薛嵩醒来之后,红线请他下楼去,就看到这名女刺客站在院子里,面朝着树篱,背朝着薛嵩,浑身上下毫发未损,只是双手被一根竹篾条拴住了。这回是红线向薛嵩建议用酷刑逼供,但他只顾呆呆地看着这个女人的背影。红线见他心不在焉,就用指甲去抓他,在他背后抓出了很多血道子。等到红线抓累了,停下手来时,他却转过身来说:你抓我干嘛? 

后来,那个女刺客侧过头来说:还是把我杀掉吧──声音异常柔和浑厚。薛嵩愣了一下,然后说:好罢。请跟我来。他转身朝外走去,那个女刺客跟在后面,头发垂在肩膀的一侧。她比红线要高,也要丰满一些,而且像雪一样白,因此是个女人,而不是女孩。在这个行列的最后走着红线,手里拿了一把无鞘的长刀,追赶着那女人的脚步,告诉她说:行刺失手者死,这是天经地义的事。而那个女人轻声答道:我知道。她的态度几乎可以说是温柔的。红线又说,你既然来行刺,还是受些酷刑再死的好,那女人就微笑不答了。他们走到了寨子的中心,薛嵩转过身来站定,而那女刺客继续向他走去,几乎要站到他的怀里。薛嵩把双手放在她的肩上,状似拥抱,但是把她轻轻往下按。于是那女人就跪了下来,在地下把腿岔开了一些,这样重心就比较稳定。在这种姿势下,薛嵩用就便器材吊起的东西就正对着她的脸,使她不禁轻声嗤笑了一声,然后马上恢复了镇定。此时天光暗淡,那女人白皙的身体在黑暗里,好像在散发着白色的荧光。于是薛嵩俯下身去,在她脑后搜索,终于把所有的头发都拢了起来,在手中握成一束,就这样提起她的头说:准备好了吗?那女人闭上了眼睛。于是薛嵩把她的头向前引去,与此同时,红线一刀砍掉了她的脑袋。这时,薛嵩急忙闪开她倒下来的身体和喷出的血。他把头提了起来,转向阴暗的天光。那女人的头骤然睁开了眼睛,并且对他无声地说道:谢谢。薛嵩想把这女人的头拿近,凑近自己的嘴唇,但是她闭上眼睛,作出了拒绝的神色;而且红线也在看着。他只好把它提开了。 

那个没有头的身体依旧美丽,在好看的乳房下面,还可以看到心在跳动;至于那个没有身体的头,虽然迅速地失去了血色(这主要表现在嘴唇的颜色上),但依旧神彩飞扬,脸色也就更加洁白。在这两样东西中间,有一滩血迹。漂亮女人的血很稀,所以飞快地渗进了地里。这就使人感到,这是一桩很大的暴行,残暴的意味昭然若揭。后来,他们把那个身子埋掉了,把污黑的泥土倒在那个洁白的身体上,状似亵渎;这个景象使薛嵩又一次失掉了平常心,变得直橛橛的,红线看了很是气愤。后来,他们把那个人头高高地吊了起来,这个女人就被杀完了。 

薛嵩用竹篾绳拴住了她的头发,把绳子抛过了一根树枝,然后就拽绳索。对于那颗人头来说,这是它一生未有的奇妙体验,因为薛嵩每拽一把,她就长高了几尺(它还把自己当个完整的人看待),这个动作如此真实地作用在自己身上,连做爱也不能相比;它微笑了一下,想到:我成了长颈鹿了。只可惜拽了没有几把,它就升到了树端。然后薛嵩把绳子拴在了树上,这件事也做完了。然后就没了下文。我无法抑制自己的失望心情:如此的有头无尾,乱七八糟。这就是我吗?

\section{第三章}

第一节 

我还在前述的寺院里,时间已经接近正午。天气比上午更热、更湿,天上似乎有一层薄雾,阳光也因此略呈昏黄之色;院里的白皮松把这种颜色的阳光零零碎碎地漏在地面上。有一个身着白色衣裙的女人从寺外急匆匆走进来,走进了阳光的迷彩……她走进我房间里来,带着一点匆忙带来的喘息,极力抑制着自己,也就是说,把喘息闷在身体里……这间房子的墙处处开裂,墙上到处是尘土,但只有一个地方例外,那就是门口。门口边上有人糊了一整张白纸,纸背后干涸的浆糊在墙上刷出了条纹,我以为这种条纹和木纹有点像。这个女人朝我张张嘴,似是想要说什么,但又没有说。她笑了一笑,搬过一张凳子──它四四方方,凳面处处开裂,边上贴了一个标签,上面写着“文物”二字──放到墙边上,然后坐上去,把背倚着墙,翘起了二郎腿。在这种姿势之下,可以看到她膝盖下方的衬裙。她把阳光晒红的脸朝我转了过来,脸上带了一点笑容。就这样呆住不动了。 

我记得她到医院里来看过我,只要同病房的人不注意,就来碰碰我的手──这使我浮想连翩。当时我还不知道自己失去了记忆。现在知道了,就不是浮想连翩,而是满怀希望。也许,我们是情人?也许刚刚是女朋友?还有可能刚刚相识,才有一点好感……我真想马上搞清楚,但又想,这件事急不得,等她先做出表示更好一点──理由很简单:我不知道该怎么称呼她。不幸的是,她就这么坐着,脸上带着笑容;直到中午,才站起来说:走吧,去吃饭。我就和她吃饭去了。 

走出这座寺院,门前有棵很大的槐树。我想这棵树足有四五百年。槐树后面有一排高大的平房,门边有个牌子,写着:国营粮店。又有一个牌子: 平价超市。这就让我犯上了糊涂,不知它到底是“国营粮店”,还是“平价超市”。树下有几张桌子,油漆剥落,桌上有几个玻璃瓶,瓶里放了些油辣子。苍蝇在飞舞……我一面觉得这地方很脏,一面犹犹豫豫地坐了下来,吃了一碗刀削面。我以为她会和我说点什么。但她什么都没说。这就使我很疑惑:难道我们之间的关系就是在一起吃面? 

饭后,我回到自己屋子里,她没有跟来。这个女人对我来说是个谜:她是谁?为什么要朝我微笑?那碗刀削面有何寓意?也许,她就是那个小黄?她为什么不给我些提示,让我把她想起来?一想到她,我就激动不已……因为她的出现,我把失掉记忆的痛苦全都忘掉了。我焦急地等着她再到我房间里来,但她总是不来。也许,我该去找她──但我又不知到哪里去找。这座寺院里跨院很多,贸然走出去,很可能回不来;再说,我也不爱闻院子里的味儿。我总得有个办法渡过焦急,所以就回到薛嵩。但是,如你所知,我已经不大喜欢他了。 

如前所述,薛嵩杀了一个刺客。这刺客也可能是个男的,这件事就将循男人的线索来进行,和女人没有什么关系。薛嵩把他押到寨子中心,大喊大叫,招来了他的雇佣兵;然后就升帐问案,所提的问题十分简单,你是什么人?从哪里来?为什么要刺杀本官?等等。那个刺客说,他不记得自己是什么人,从哪里来。他没有刺杀薛嵩。至于薛嵩的耳朵,他说是自己掉下来的。如你所知,这完全不合情理,他还不停地傻笑,假装是个疯子。假如想从他口中得到有用的信息,必须要对他严刑逼供──否则就是说双口相声,这种表演对薛嵩的威信有害。但是那些雇佣兵却对这些回答鼓掌叫好。薛嵩自己也陷入了内心的矛盾之中,他确实很想知道这个刺客是谁派来的,那人为什么要杀他,以后还会不会再派刺客来,等等。但另一方面,他又佩服这刺客的倔强,觉得他是个男子汉大丈夫。对一个男子汉大丈夫,就该让他从容就义,壮烈成仁,折磨人家显得很卑鄙。因为那些雇佣兵在场,薛嵩不得不装点假正经──就这样马马虎虎地把他砍了。要是不升帐问案倒会好些,在自己家里,有红线作帮手,想怎么打就怎么打,不容这小子不说实话。薛嵩已经想到了这些,但后悔已经晚了。 

砍头的情形是这样的:那个刺客跪在地上,有一个兵站在他的腿上,按住了他的肩膀,薛嵩站在他对面,手里握着他的头发,尽力往上拉,使他的脖子伸长;还有一个兵准备从中间去砍。在砍之前,刺客不停地叫疼,而薛嵩则安慰他道:忍一忍,一会儿就完了。这是薛嵩第一次参加杀人,心情激动,使的劲很大,把那个刺客的脖子拽得像鹅脖子一样长,但是持刀的兵总是不砍。薛嵩问他为什么不下刀子,那人却笑着说道:启禀老爷,你再使点劲就能把他脑袋揪下来,用不着我砍了──这是嘲笑薛嵩在杀人时过于激动。当然,最后那个兵还是砍了一刀,此后薛嵩和那颗人头一起跳了起来,等到落在地下时,已经被溅了一身血。不知为什么,那颗刺客的人头下端拖着长长的食道和气管,像两条尾巴,很不好看。薛嵩要过杀人的刀,帮他修理了一下,还要来水,自己冲洗了一下,也洗掉了人头上的血迹。此时那颗人头脸上露出了微笑,并且无声地说道:谢谢。此后那颗人头就混迹于一群人之中,被大家传递和端详。有人说:被砍下的人头正如剪下来的鲜花,最好把伤口用热蜡封住,或是用火烧一下,这样可以避免腐烂,长久地保持鲜活。那颗人头听到以后皱起眉来,薛嵩也坚决地表示反对。然后他们用绳子拴住它的头发,把它像一面旗子一样在一棵树上升起来,薛嵩率领全体士兵在人头对面立正,对它行举手礼,直到人头升到了最高点才礼毕。此时薛嵩感到很满意,因为他已经杀了一个人,死者的尊严也得到了保证。美中不足的是,薛嵩还是没有得到所需的信息,但是这件事已经无法挽回了。所以,他隐隐地感到这件事进行得太快了。但不是他在控制此事的节奏,是那些雇佣兵在控制此事的节奏,他们哄着快点把刺客杀掉,绝不是为薛嵩的利益着想。薛嵩已经想到了这些,但又想到:这些兵是自己的战友,胡乱猜疑是不对的。所以,他赶紧把这些想法忘掉了。 

假如那个刺客是女的,杀她时也会有雇佣兵在场。杀人的地方在寨心的火堆旁,那帮家伙不请自来,躲在黑暗里,怪声怪气地叫着,要对这女人严刑逼供,还提出一些下流、残忍的建议,在此不便转述。那女人很害怕,情不自禁地倚到了薛嵩身上。这是因为薛嵩允诺了结束她的生命,所以薛嵩就是死亡。而死亡是干净的。薛嵩一手搂着她的肩,一手挥动着大铁枪,不让那些家伙靠近。当时红线也在场,手里舞着一把长刀,谁敢从黑暗中走出来,她就砍他一刀。小妓女也在场,她高声尖叫着:大叔!大叔们!你们就积点德吧!老妓女也在场,她躲在屋檐下一声不吭。我比较喜欢这个场景,也喜欢这个薛嵩。然后,薛嵩和红线把这女人杀掉──这正是被杀者的愿望。但不管怎么说,我不喜欢杀人。 

2 

如前所述,那颗被砍下的人头里隐藏了一个秘密:谁指使她或他杀掉薛嵩。这个秘密薛嵩急于知道。对此我有一个古怪的主意:让薛嵩把那颗脑袋劈开,把脑浆子吃掉,然后凝神思索片刻,也许就能想出是谁要杀他。但是这个主意不可行:假如那脑袋属于亮丽的女人,想必会是种美味,但薛嵩会觉得不忍去吃;假如那脑袋属于威武的男人,薛嵩吃了又会恶心。既然这主意不可行,这个秘密就揭不开了。 

按照侦探小说的说法,这秘密要在最后揭开,因为它是全书的基点,很是重要。在我看来,凤凰寨建在一座红土山坡上,是一座由热带林薮组成的迷宫,这在这个故事里有更加重要的意义。这座寨子的中央,住了一个浪浮的小妓女,还有一个古板的老妓女。这个小妓女经常呆在树上,这是一个防范措施,因为她怕那个老妓女暗算她。随后就可以看出,这种防范是有道理的。至于那个老妓女,她有一个没胎人形似的身体,假如这个身体会被男人看到,她会先用白纸贴住下垂的乳头,再把阴毛刮掉,在私处扑上粉。这样她的身体就像刷过的墙一样白。就是她要杀掉薛嵩,然后还要杀掉小妓女。天黑以后,她从房子里出来,看看树上挂着的人头,啐了它一口,小声骂道:笨蛋!废物!就回到屋里去。又过了一会儿,她再次出来,放飞了一只白鸽,鸽脚上拴了一封信,告诉她的同谋说,第一位刺客已经失败,脑袋吊到树上了,请求再派新的刺客来。她还提醒那些人说:要提防薛嵩后园里的马蜂。如此说来,是老妓女要杀薛嵩。但我怀疑这种说法是不是过分了──我不喜欢让相识的人互相乱杀。入暮时分,一只鸽子在天上扑啦啦地飞,看着就怪可疑。此时红线在附近的河沟里摸黄鳝,看见以后,急忙到岸上拿弩箭,要把它射下来。但是来不及了,鸽子已经飞走了。 

在凤凰寨里的沟渠边上,密密麻麻长着一种红色的篦麻,叶子比蒲叶要大,果实有拳头大,种子有栗子大。剥掉篦麻子的硬皮,种肉油性很大,但是不能吃,吃了要泻肚子。唯一的用处就是当灯来点。红线剥了很多篦麻子,用竹签拴成一串,点着以后,照着捉黄鳝,并把捉到的黄鳝用篾条穿成一串。她当然知道,一个寨子里来了刺客,说明寨内有奸细,所以她保持了警惕。她更知道信鸽是奸细和同党联系的手段,所以就想把信鸽射下来,但是晚了一步没有射到。然后她就犹豫起来:是赶回家去,把这件事告诉薛嵩呢,还是接着摸黄鳝。就在这时,她发现自己大腿上有一条蚂蟥在吸血。她把蚂蟥揪了下来,放在火上烧死,然后就只记得一件事:要下水去摸黄鳝。她倒是有点纳闷,自己刚才在犹豫些什么,想来想去没想起来。假如她立刻跑回家告诉薛嵩,薛嵩就能知道,寨子中间住了一个奸细。可以肯定,这奸细就是两个妓女之一。以薛嵩的聪明才智,马上就能找到一种方法,判断出这奸细是谁:那颗刺客的人头高高地挂在天上,肯定看见了是谁放了那只鸽子,可以把它放下来问问,它只要努努嘴,或是闭上一只眼,就指出谁是奸细。这颗刺客的头也一定喜欢有另一颗人头和自己并排挂着──这样不寂寞。何况假如它不说的话,还可以把它放到火上烤,放到水里去煮。有一些头颅常遭到这样的待遇,所以能够安之若素。但闹事猪头,不是人头──人头受不了这种待遇,会招供的。但是红线想去摸黄鳝,把这件事忘掉了。 

薛嵩因此错过了逮住奸细的机会。但红线也没有下水去摸黄鳝,蹋低下头去看自己腿上被蚂蟥叮破的伤口,又发现自己的臀位很高──换句话说,就是腿长。翻过来掉过去看了一会儿之后,她决定去找那个小妓女,表面上是要送几条黄鳝给她,实际上是请她对自己的腿发表些意见。小妓女本不肯说她腿长,但又很喜欢吃黄鳝,就说了违心的话;然后她们炒鳝鱼片吃。这样一来,红线很晚才回家。那只信鸽则带着情报飞远了。入夜以后,就会有大批的刺客到来。这对薛嵩是件很糟糕的事。但这又要怪薛嵩自己。假如在家里时,他没有忽略红线的两条腿──举例来说,当他倒在地板上要睡觉,红线从他前面走过时,他从底下看到了这双长腿,就该坐起半身,高叫一声:哇!腿很长嘛!红线就会感到幸福。对女孩来说,得到男性的赞誉,肯定是更大的满足──她就不会老往小妓女那里跑,还会把摸到的黄鳝带回家来。但他总端着老爷架子,什么都不肯说。端这个架子的结果是,有大批刺客前来杀他,他还蒙在鼓里。我完全同意作者的意见:这是他自作自受。 

3 

在我心目中,凤凰寨是一幅巨大的三维图像,一圈圈盘旋着的林木、道路、荒草,都被寨心那个黑咚咚的土场吸引过去了。天黑以后,在这个黑里透灰的大大旋涡里亮起了星星点点的灯光,每一盏灯都非常的孤独──偌大的寨子里根本就没有几户人。等到红线回家时,这些灯火大多熄灭了。薛嵩在灯下作愤怒状,他说红线回来晚了,要用家法来打红线;所谓家法是一根光溜溜的竹板子,他要红线把这根板子拿过来,递到他手上,然后在地板上伏下,让他打自己的屁股。这个要求颇有些古怪之处,假如我是红线,就会觉得薛嵩的心理阴暗。所以红线就大吵大闹,说她今天还抓到了刺客,为什么要挨打。薛嵩沉下脸来说:你不乐意就算了。红线忽然笑了起来,说:谁说我不乐意?她把板子递给薛嵩以后,说道:不准真打啊,就在地板上趴下了。薛嵩原是长安城里一位富家子弟,经常用板子、鞭子、藤棍等等,敲打婢女、丫鬟们的手心、屁股或者脊背,这本是他生活中的一种乐趣。但是这些女人在挨打之前总是像杀猪一样的嚎叫,从没说过:“不准真打啊”,虽然薛嵩也没有真打──薛嵩饱读诗书,可不是野蛮人啊。女孩这样说了之后,再敲打这个伏在竹地板上的、橄榄色的、紧凑的臀部就不再有乐趣──不再是种文化享受。所以,薛嵩把那根竹板扔掉了。 

现在可以说说薛嵩的竹楼内部是怎样的。这座房子相当的宽敞,而且一览无遗,没有屏风,也没有挂着的帘子,只有一片亮晶晶的金竹地板。还有两三个蒲团。薛嵩就坐在其中一个的上面,想着久别了的故乡,还想到有人来刺杀他的事,心情坏得很。此时红线趴在他的脚下,等了好久不见动静,就说:启禀老爷,小奴家罪该万死,请动家法。就在这时,薛嵩把手里的竹板扔掉,说道:起来说话。红线就爬起来,坐在竹地板上说,那我还是不是罪该万死了?但薛嵩愁眉苦脸地说:你听着,我觉得心惊肉跳,感觉很不好。红线就松了一口气说:噢,原来是这样。那就没有我的事了。于是她就地转了一个身,头枕着蒲团,开始打瞌睡,还睡意惺忪地说了一句:什么时候想动家法就再叫我啊。这个女孩睡着以后有一点声音,但还不能叫作鼾声。 

午夜时分,红线被薛嵩推醒,听见他说:小贱人!醒醒,小贱人!她半睡半醒地答道:谁是小贱人?薛嵩说:你啊!你是小贱人。红线就说:妈的,原来我是小贱人。你要干什么?薛嵩答道:老爷我要和你敦伦。红线迷迷糊糊地说:妈的,什么叫作敦伦?这时她已经完全醒了,就翻身爬起,说道:明白了。回老爷,小奴家真的罪该万死──这回我说对了吧。由此可见,薛嵩常给红线讲的那些男尊女卑的大道理,她都理解到性的方面去了。我也不知怎么理解更对,但薛嵩总觉得那个老娼妇说话更为得体。在这种时刻,那个老女人总是从容答道:老爷是天,奴是地。于是薛嵩就和她共享云雨之欢,心里想着阴阳调合的大道理,感觉甚是庄严肃穆。红线在躺下之前,还去抓了一大把瓜子来。那种瓜子是用蛇胆和甘草炮制的,吃起来甜里透苦。她一边磕,一边说,既然干好事,就不妨多干一些:既“罪该万死”,又磕瓜子。你要不要也吃一点?薛嵩被这种鬼话气昏了头,不知怎样回答。 

我又涉入了老妓女的线索,现在只好按这个线索进行。夜里,老妓女迎来了所雇的刺客。那是一批精壮大汉,赤裸着身体,有几个臀部很美。她叫他们去把小妓女抓来,马上就抓到了。他们把小妓女绑了起来,嘴里塞上了臭袜子。她让他们去杀薛嵩,他们就把刀擦亮。那间小小的房间里有好几十把明晃晃的刀,好像又点亮了十几支蜡烛。用这些人可以做她的事业。为此要杀掉那个小妓女,而她就躺在她身边,被绑得紧紧的,下巴上拖着半截袜子,像牛舌头一样。于是那个老娼妇想道,今天夜里,一切都能如愿以偿。这是多么美好啊! 

午夜时分,凤凰寨里有两个女孩受到罪该万死的待遇,她们是红线和小妓女。实施者分别是薛嵩和老妓女,单老妓女是当真的,薛嵩却不当真。我基本同意作者的意见:不把这件事当真,说明薛嵩是个好人。但不做这件事,或者在做这件事时,不说红线罪该万死,他就更是好人了。 

午夜时分,那个老妓女送走了刺客们,就在门外用黄泥炉子烧水,沏茶,准备在他们凯旋而归时用茶水招待。她还有件小事要麻烦他们,就是把那个小妓女杀掉。这件事她现在自己就能干,但是她觉得别人逮来的人,还是由别人来杀的好。水开了以后,她沏好了茶,放在漆盘里,把它端到屋子里。如前所述,那个女孩被捆倒在这间房子里,嘴里塞了一只臭袜子。那个老娼妇站了很久,终于下定了决心,俯下身来,把茶水放在地板上,然后取下了女孩嘴上的臭袜子,搂住她的肩,把她扶了起来。那女孩在地板上跪着,好像一条美人鱼,表情木讷,两只乳房紧紧的并在一起,乳头附近起了很多小米粒一样的疙瘩,这说明她既紧张,又害怕。老娼妇在漆碗里盛了一点茶水,递到女孩嘴边轻轻地说:喝点水。女孩没有反应。那个老娼妇就把浅碗的边插到她嘴唇之间,碰碰她的牙,又说:喝点水。这回带了一点命令的口气。那女孩俯下头去,把碗里的水都喝干,然后就哭了起来,她手里还攥着一条麻纱手绢,本该在这种时候派用场。但因为被绑着,也用不上。于是她的胸部很快就被泪水完全打湿。过了一会儿,她朝老娼妇转过头来,这使那老女人有点紧张,攥紧了那只臭袜子,随时准备塞到对方嘴里去──她怕她会骂她,或者啐她一口。但是那女孩没有这样做。她只是问道:你要拿我怎么办?杀了我吗?这老娼妇饱经沧桑,心像铁一样硬。她耸了一下肩说:我不得不这么办──很遗憾。那个女孩又哭了一会儿,就躺下去。说道:塞上吧。就张开嘴,让老娼妇把袜子塞进去;她的乳房朝两边涣散着,鸡皮疙瘩也没有了。现在她不再有疑问,也就不再有恐惧,躺在地下,含着臭袜子,准备死了。 

而那个老娼妇在她身边盘腿坐下,等待着进一步的消息。后来,薛嵩家的方向起了一把冲天大火,把纸拉门都映得通红。老娼妇跪了起来,激动地握紧了双拳。随着呼吸,鼻子里发出响亮的声音,好像在吹洋铁喇叭。后来,这个老娼妇掀开了一块地板,从里面拿出一把青铜匕首,那个东西做工精巧,把手上镌了一条蛇。她把这东西握在手里,手心感觉凉飕飕,心里很激动,好像感觉到多年不见的性高潮。她常拿着这把匕首,在夜里潜进隔壁的房子去杀小妓女,但因为她在树上睡觉,而那个老女人又爬不上去,所以总是杀不到。现在她紧握匕首,浮想连翩。而那个女孩则侧过头来,看她的样子。那个老娼妇赤裸着上身,乳房好像两个长把茄子。时间仿佛是停住了。 

在薛嵩家的竹楼里,红线在和薛嵩作爱。她像一匹仰卧着的马,也就是说,把四肢都举了起来,拥住薛嵩,兴高采烈,就在这一瞬间,忽然把表情在脸上凝住,侧耳到地板上去听。薛嵩也凝神去听,白天被人砍了一刀,傻子才会没有警惕性,但除了耳朵里的血管跳动,什么也没有听见。他知道红线的耳朵比他好──用他自己的话来说:该小贱人口不读圣贤书,所以口齿清楚。耳不闻圣人言,所以听得甚远。目不识丁,所以能看到三里路外的蚊子屁股。结论当然是:中华士人不能和蛮夷之人比耳聪目明,所以有时要求教于蛮夷之人。薛嵩说:有动静吗?红线说:不要紧,还远。但薛嵩还是不放心,开始变得软塌塌的。红线又说:启禀老爷,天下太平;这都是老爷治理之功,小贱人佩服得紧!听了这样的赞誉,薛嵩精神抖擞,又变得很硬…… 

4 

红线很想像那个亮丽的女人一样生活一次,被反拴着双手,立在院子里,肩上笼罩着白色的雾气。此时马蜂在身边飞舞,嗡嗡声就如尖厉的针,在洁白的皮肤上一次次划过。因为时间过得很慢,她只好低下头去,凝视自己形状完美无缺的乳房。因为园里的花,她身体上曲线凸起之处总带有一抹紫色;在曲线凹下之处则发射出惨白的光。后来,她就被带出去杀掉;这是这种生活的不利之处。在被杀的时候,薛嵩握住了那一大把丝一样的头发往前引,她自己则往后坐,红线居中砍去。在苗寨里,红线常替别人分牛肉,两个人各持牛肉的一端,把它拉长,红线居中坎去。假如牛肉里没有骨头,它就韧韧地分成两片。这种感觉在刀把上可以体验到,但在自己的脖子上体验到,就一定更为有趣。然后就会身首异处,这种感觉也异常奇妙。按照红线的想象,这女人的血应该是淡紫色的,散发着藤萝花的香气。然后,她就像一盏晃来晃去的探照灯,被薛嵩提在手里。红线的确是非常地爱薛嵩,否则不会想到这些。她还想象一颗砍掉的人头那样,被安座在薛嵩赤裸的胸膛上。这时薛嵩的心,热哄哄地就在被砍断的脖端跳动,带来了巨大的轰鸣声。此时,她会嫣然一笑,无声地告诉他说:嗓子痒痒,简直要笑出来。但是,她喜欢嗓子痒痒。此时寨子里很安静──这就是说,红线的听觉好像留在了很远的地方。 

而那个老妓女,则在一次次地把小妓女杀死。但是每一次她自己都没有动手。起初,她想让那些刺客把这女孩拖出去一刀砍掉。后来她又觉得这样太残忍。她决定请那些刺客在地下挖一个坑,把那个小妓女头朝下的栽进去,然后填上土,但不把她全部埋起来,这样也太残忍。要把她的脚留在地面上。这个女孩的脚很小,也很白,只是后脚跟上有一点红,是自己踩的,留在地面上,像两株马蹄莲。老妓女决定每天早上都要去看看那双脚,用竹签子在她脚心搔上一搔。直到有一天,足趾不动了,那就是她死掉了。此时就可以把她完全埋起来,堆出一个坟包。老妓女还决定给她立一个墓碑,并且时常祭奠。这是因为她们曾萍水相逢,在一座寨子里共事,有这样一种社会关系。那个老妓女正想告诉她这个消息,忽然又有了更好的主意。如前所述,这位老太太有座不错的园子,她又喜欢园艺;所以她就决定剖开一棵软木树,取出树心,把那个女孩填进去,在树皮上挖出一个圆形的洞,套住她的脖子,然后把树皮合上,用泥土封住切口,根据她对这种树的了解,不出三天,这棵树就能完全长好。以后这个人树嫁接的怪物就可以活下去:起初,在树皮上有个女孩的脸,后来这张脸就逐渐消失在树皮里;但整棵树会发生一些变化,树皮逐渐变得光滑,树干也逐渐带上了少女的风姿。将来男人走到这棵树前,也能够辨认出哪里是圆润的乳房,哪里是纤细的腰肢。也许他兴之所致,抚摸树干,这棵树的每一片叶子都会为之战栗,树枝也为之骚动。但是她说不出话,也不能和男人做爱。只能够体味男人的爱抚带来的战栗。 

作为一个老娼妓,她认为像这样的女人树不妨再多一些。因为她们没有任何害处,假如缺少燃料,还可以砍了当柴烧。除了这个小妓女,这寨子里的女人还不少(她指的是大家的苗族妻子),所以绝不会缺少嫁接的材料。总而言之,这个老女人自以为想出了一种处置年轻女人的绝妙方法,所以她取下了小妓女嘴上的袜子,把它放到一边,告诉她这些,以为对方必定会欢欣鼓舞,迫不急待地要投身于树干之中。但那个小妓女发了一会儿愣,然后断然答道:你快杀了我!说完侧过头去,叼起那只臭袜子,把它衔在嘴里──片刻之后,又把它吐了出来,补充说道:怎么杀都可以。然后,她又咬住袜子,把它强行吞掉,直到嘴唇之间只剩了袜子的一角──这就是说,她不准备把它再吐出来了。她就这样怒目圆睁地躺在地板上,准备死掉。老娼妇在她腿上拧了一把,说道:小婊子,你就等着罢;然后到走廊上去,等着刺客们归来,带来薛嵩的首级。而那个小妓女则闭上了眼睛,忘掉了满嘴的臭袜子味,在冥冥中和红线做爱。她很喜欢这小蛮婆橄榄色的身体──不言而喻,她把自己当成了薛嵩。在她们的头顶上、在一团黑暗之中,那颗亮丽的人头在凝视着一切。 

按照通俗小说的写法,现在正是写到那小妓女的恰当时机。我们可以提到她姓甚名谁,生在什么地方,如何成长、又是如何来到这个寨子里来;她为什么宁愿被头朝下栽在冷冰冰的潮湿的泥土之中,长时间忍受窒息以及得不到任何信息的寂寞──可以想见,在这种情况下,她一定巴不得老娼妇来搔她的脚心,虽然奇痒难熬,但也可因此知道又过了一天──也不愿变成一棵树。在后一种处置之下,她可以享受到新鲜空气、露水,还可以看到日出日落,好处是不言而喻的。一个人自愿放弃显而易见的好处,其中必有些可写的东西。但作者没有这样写。他只是简单地说道:对那小妓女来说,只要不看到老妓女,被倒放进油锅里炸都行。 

第二节 

夜里,薛嵩的竹楼里点着灯,光线从墙壁的缝隙里漏了出去,整座房子变成了一盏灯笼。因为那墙是编成的,所以很像竹帘子。假如帘子外亮,帘子里暗,它就是一道可靠的、不可透视屏障;假如里面亮,外面暗,就变得完全透明,还有放大的作用。走进他家的院子,就可以看到墙上有大大的身影──乍看起来是一个人,实际上是两个人,分别是卧姿的红线和跪姿的薛嵩──换句话说,整个院子像座电影院。在竹楼的中央有一根柱子,柱上斜插了一串燃烧中的蓖麻子。对此还可以进一步描写道:雪白的籽肉上拖着宽条的火焰,“噼噼”地爆出火星,火星是一小团爆炸中的火焰,环抱着一个滚烫的油珠。它向地下落去,忽然又熄掉,变成了一小片烟炱,朝上升去了。换句话说,在宁静中又有点火爆的气氛。薛嵩正和红线做爱,与此同时,刺杀他的刺客正从外面走进来。所以,此处说的火爆绝不只是两人之间的事。 

后来,红线对薛嵩说:启禀老爷,恐怕你要停一停了。但薛嵩正沉溺在某种气氛之中,不明白她的意思,还傻呵呵地说:贱人!你刚才还说佩服老爷,怎么又不佩服了?后来红线又说:喂!你快起开!薛嵩也不肯起开,反而觉得红线有点不敬。最后红线伸出了手,在薛嵩的胸前猛地一推──这是因为有人蹑手蹑脚地走进了这个电影院,然后又顺着梯子爬进了这个灯笼;红线先从寨里零星的狗叫声里听到了这些人,后从院里马蜂窝上的嗡嗡声里感到了这些人,然后又听到楼梯上的脚步声。最后,她在薛嵩背后的灯影里看到了这个人:乌黑的宽脸膛(可能抹了黑泥),一张血盆大口,手里拿了一把刀,正从下面爬上来。此时她就顾不上什么老爷不老爷,赶紧把薛嵩推开,就地一滚,摸到了一块磨刀石扔了出去,把那个人从楼梯上打了下去。对此薛嵩倒没有什么可惭愧的:女人的听力总比男人要好些,丛林里长大的女孩比都市里长大的男人听力好得更多;后者的耳朵从小就泡在噪声里,简直就是半聋。总的来说,这属动物本能的领域,能力差不是坏事。但是薛嵩还沉溺在刚才的文化气氛里,虽然红线已经停止了拍他的马屁,也无法立刻进入战斗的气氛。就这样,红线在保卫薛嵩,薛嵩却在瞎比划,其状可耻…… 

薛嵩眼睁睁地看着红线抢了一把长刀,扑到楼口和人交了手,他还没明白过来,而第二个冲上来的刺客看到薛嵩直愣愣地跪在那里,也觉得可笑,刚“嗤”了一声,就被红线在头上砍了一刀,鲜血淋漓地滚了下去。对这件事还有补充的必要:薛嵩跪在那里,向一片虚空做爱,这景象的确不多见;难怪会使人发呆。薛嵩也很想参战,但是找不着打仗的感觉,满心都是作老爷的感觉。这就如他念书,既已念出了“子曰”,不把一章念完就不能闭嘴。但是,老爷可不是作给男人看的,那个被红线砍伤的刺客滚下楼去,一路滚一路还在傻笑着说:臭比划些什么呀…… 

但刺客还在不断地冲上来,红线在阻拦他们,虽然地形有利,也觉得寡不敌众。她就放声大叫:老爷!老爷!快来帮把手!薛嵩还是找不到感觉。后来她又喊:都是来杀你的!再不来我也不管了啊!但薛嵩还是挣不出来。直到红线喊:兔崽子!别作老爷梦了!你想死吗!他才明白过来,到处找他的枪,但那枪放在院子里了。于是他大吼了一声,撞破了竹板墙,从二楼上跳了出去,去拿他的铁枪,以便参加战斗。这是个迎战的姿态,但看上去和逃跑没什么两样。 

我越来越不喜欢这故事的男主人公──想必你也有同感。因为你是读者,可以把这本书丢开。但我是作者,就有一些困难。我可以认为这不是我写的书,于是我就没有写过书;一点成就都没有──这让我感到难堪。假如我认为自己写了这本书,这个虚伪、做作的薛嵩和我就有说不清楚的关系。现在我搞不清,到底哪一种处境更让我难堪…… 

在上述叙述之中,有一个谜:为什么红线能马上从做爱的状态进入交战,而薛嵩就不能。对此,我的解释是,在红线看来,做爱和作战是同一类的事,感觉是同样的火爆,适应起来没有困难。薛嵩则是从暧昧的文化气氛进入火爆的战斗气氛,需要一点时间来适应。当然,假如没有红线在场,薛嵩就会被人当场杀掉。马上就会出现一个更大的问题:在顷刻之间,薛嵩会从一个正在做爱的整人变成一颗人头,这样他就必须适应从暧昧到悲惨的转变,恐怕更加困难。但总的来说,人可以适应任何一种气氛。虽然这需要一点时间。 

薛嵩从竹楼里撞了出去,跳到园子里,就着塌了墙的房间里透出的灯光,马上就找到了他的铁枪,然后他就被十几个刺客围住了。这些刺客擎着火把,手里拿着飞快的刀子,想要杀他。薛嵩把那根大铁枪舞得呼呼作响,自己也在团团旋转,好像一架就要起飞的直升飞机,那几十个人都近他不得,靠得近的还被他打倒了几个。这样他就暂时得到了安全。但也有一件对他不利的事情:这样耍着一根大铁棍是很累的。这一点那些刺客也看出来了。他们围住了他,却不向他进攻,反而站直了身子说:让他多耍一会儿;并且给他数起了圈数,互相打赌,赌薛嵩还能转几圈。薛嵩还没有累,但感到有点头晕,于是放声大叫道:来人!来人!这是在喊他手下的士兵。但是喊破了嗓子也不来一个人。后来他又喊红线:小贱人!小贱人!但是红线也自顾不暇。她和三条大汉对峙着,如果说她能打得过,未免是神话;但对方想要活捉她,她只要保住自己不被抓住就可以。就是这样,也很困难。所以她就答道:老爷,请你再坚持一下。后来他又指望树上的马蜂窝,就大叫道:马蜂!马蜂!但那些昆虫只是嗡嗡地扇动翅膀,一只也不飞起来。这是因为所有的马蜂,不管是温带的马蜂还是热带的马蜂,都不喜欢在天黑以后起飞螫人,它们都患着夜盲症。这些刺客也知道这一点,所以他们虽然在数量上有很大的优势,还是等到天黑了才进攻,以防被螫到。还有一个指望就是逃走,但薛嵩在团团的旋涡中,早已不辨东西南北,所以无法逃走。假如硬要跑的话,很可能掉进水塘里,那就更不好了。那些刺客们一致认为,这小子再转一百圈准会倒,但没有人下注说他能转一百圈以上;这也不是赌了。薛嵩觉得自己要不了一百圈就会倒。他陷入孤立无援的境地,被困住了。 

最后薛嵩总算是逃脱了。后来他说,自己经过力战打出了一条血路。但一面这样说,一面偷偷看红线。此种情形说明他知道自己在说谎, 事实是红线帮他逃了出来。但红线也不来拆穿他。 久而久之,他也相信自己从大群刺客的包围中凭掌中枪杀出了一条血路──这样他就把事实给忘了。所有的刺客都去看薛嵩转圈,没有人注意红线,她就溜掉了。溜到竹楼下面,捡到了一个火把,一把火点着了自家的竹楼,一阵夜风吹来,火头烤到了树上的马蜂窝。马蜂被激怒了,同时院子里亮如白昼,它们也能看见了,就像一阵黄色的旋风,朝闯入者扑去,螫得他们落荒而逃。红线趁势喝住了薛嵩(他还在转圈子),钻水沟逃掉了。这一逃的时机掌握得非常好,因为被烧了窝的马蜂已经不辨敌我,逢人就螫。红线还干了件值得赞美的事,她退出战场时,还带走了薛嵩的弓箭。这就大大增强了他们的力量。现在,在他们手里,有一条铁枪、一口长刀,还有了一张强弓。而且他们藏身的地方谁也找不到。那地方草木茂盛,哪怕派几千人去搜,也照样找不到。更何况刺客先生们已经被螫了一通,根本不想去找。 

2 

凤凰寨里林木茂盛,夜里,这地方黑洞洞的。也许,只有大路上可以看到一点星光,所以,这条路就是灰蒙蒙的,有如夜色中的海滩。至于其它地方,好像都笼罩在层层黑雾里。这些黑雾可以是树林,也可以是竹林,还可能是没人的荒草,但在夜里看不出有什么区别。那天夜里,有一瞬间与众不同,因为薛嵩的竹楼着了火。作为燃料,那座竹楼很干燥,又是枝枝岔岔地架在空中,所以在十几分钟之内都烧光了;然后就只剩了个木头架子,在夜空里闪烁着红色的炭火。在它熄灭之前,火光把整个寨子全映红了;然后整个寨子又骤然沉没在黑暗之中。这火光使老妓女很是振奋,她在自己的门前点亮了一盏纸灯笼,并且把它挑得甚高,以此来迎接那些刺客。而那些刺客来到时,有半数左右脸都肿着,除此之外,他们的表情也不大轻松。这就使那老女人问道:杀掉了吗?对方答道:杀个屁,差点把我们都螫死!她又问:薛嵩呢?对方答道:谁知道。谁知道薛嵩。谁知道谁叫薛嵩。那个老女人说:我是付了钱的,叫你们杀掉薛嵩。对方则说:那我们也挨了螫。这些话很不讲理;刺客们虽然打了败仗,但他们人多势大,还有讲这些话的资格。 

那个老女人把嘴瘪了起来,呈鲇鱼之态,准备唠叨一阵,但又发现对方是一大伙人,个个手里拿着刀杖,而且都不是善良之辈,随时准备和她翻脸;所以就变了态度,低声下气地问他们薛嵩到底在哪里。有人说,好像看见他们钻了树棵。于是她说,她愿再出一份钱,请他们把薛嵩搜出来杀掉。于是他们就商量起来。商量的结果是拒绝这个建议,因为这个寨子太大,一年也搜不过来。于是他们转身就走。顺便说一句,这些人为了不招人耳目,全都是苗人装束:披散着头发,赤裸着身体,挎着长刀。当他们转过身去时,就着昏暗的灯光,那个老女人发现,有好几个男人有很美的臀部。对于这些臀部,她心里有了一丝留恋之情。但是那些男人迈开腿就走。假如不是寨里住的那些雇佣兵,他们就会走掉了。 

现在我们要谈到的事情叫作忠诚,每个人对此都有不同的理解。当那些刺客在寨子里走动,引起了狗叫,这些雇佣兵就起来了,躲在自家屋檐下面的黑暗里朝路上窥视。等刺客走过之后,又三三五五地串连起来,拿着武器,鬼鬼祟祟地跟在后面,但为了怕刺客看见,引起误会,这些家伙小心翼翼地走在路边的水沟里。如前所述,薛嵩在受刺客围攻时,曾经大叫“来人”,那些兵倒是听到了。他们出来是看出了什么事,手里都拿了武器,只是要防个万一;所以谁也不去救薛嵩。相反,倒盼着他被刺客杀死。红线放火,马蜂把刺客螫走,他们都看到了,单都一声不吭。薛嵩他们不怕,但不想招惹红线。然后这些刺客到寨中间去找那个老妓女,他们也跟在后面,始终一声不吭。等到这些刺客要走时,他们才从路边的浅沟里爬出来,把路截住,表现出雇佣兵的忠诚。这种忠诚总是要使人大吃一惊。 

如前所述,雇佣兵的忠诚曾使薛嵩震惊。当他上山去打面寨时,后面跟了几十个兵,他觉得太多了,多得让他不好意思。现在这种忠诚又使那个老妓女吃了一惊,她原以为在盘算刺杀薛嵩时,可以不把雇佣兵考虑在内的,现在觉得自己错了。当然,最吃惊的是那些刺客,雇佣兵来了黑压压的一片,总有好几百人,手里还拿了明晃晃的刀,这使刺客们觉得脖子后面有点发凉,不由自主地往后退。薛嵩不在这里,要是在这里,必然要跳出去大叫:你们怎么才来?噢,说错了。来了就好。假如事情是这样,薛嵩马上就需要适应悲惨的气氛;因为这些雇佣兵站了出来,可不一定是站在他这一方。总而言之,那些刺客见到他们人多,就很害怕,就想找别的路走。这寨子里路很多,有人行的路、牛行的路、猪崽子行的路。不管他们走哪条路,最后总是发现被雇佣兵们截在了前头。好像这寨子里不是只有一百来个雇佣兵,而是有成千上万个雇佣兵,把到处都布满了。 

最后,这些刺客也发现了这一事实:雇佣兵比他们熟悉这个地方。于是,刺客群里站出一个人(他就是刺客的头子),审慎地向拦路的雇佣兵发问道:好啦,哥儿们。你们要干什么?对方一声不吭。他只好继续说道:我知道你们人多路熟……这句话刚出口,马上就被对方截断道:知道这个就好。别的不必说了。他们就这样栏住了外来的刺客,不让他们走。至于他们要做些什么,没有人能够知道。好在这一夜还没有过完,天上还有星星。 

3 

我的故事又到了重新开始的时刻,面对着一件不愿想到的事,那就是黎明。薛嵩和红线坐在凤凰寨深处的树丛里,这时候黎明就来到了。红线是个孩子,折腾了一夜,困得要命,就睡着了;在黎明前的寒冷之中,她往薛嵩怀里钻来。黎明前的寒冷是一层淡蓝色稀薄的雾。薛嵩有时也喜欢抱住红线,但那是在夜里,现在是黎明,在淡蓝色的黎明里,他觉得搂搂抱抱的不成个样子。打他想到红线又困又冷,也就无法拒绝红线的拥抱。在睡梦之中,红线感到前面够暖和了,就翻了一个身,躺到了薛嵩怀里。薛嵩此时盘腿坐在地下,背倚着一棵树,旁边放着他的铁枪;而红线则横躺着睡了,这样子叫薛嵩实在开心不起来。假如他也能睡着,那倒会好些。但是蚊子叮得太凶,他睡不着。他只好睁大眼睛,看每一只飞来的蚊子,看它要落在谁的身上。很不幸的是,每个蚊子都绕过了红线,朝他大腿上落过来,这使他满心委屈和愤恨。他不敢把蚊子打死,恐怕会把红线惊醒,就任凭蚊子吸饱了血游飞走。更使他愤恨的是红线睡得并不死,每十分钟必醒来一次,咂着嘴说道:好舒服呀,然后往四下看看;最后盯住薛嵩,含混不清地说:启禀老爷,小奴家罪该万死──你对我真好。然后马上又睡着了。 

黎明可能是这样的:红线倒在薛嵩怀里时,周围是一片淡淡的紫色。睡着以后,她那张紧绷绷的小脸松懈下来。然后,淡紫色就消散了。一片透明的浅蓝色融入了一切,也融入红线小小的身体。此时红线觉得有一点冷,就抬起一只手放在自己的乳房上。在天真无邪的人看来,这没有什么。但在薛嵩看来,这景象甚是扎眼。有一个字眼从他心底冒起,就是“淫荡”。后来,一切颜色都褪净了,只剩下灰白色。不知不觉之中,周围已经很亮。熟睡中的红线把双臂朝上伸,好像在伸个懒腰。她在薛嵩的膝上弯成个弧度很大的拱形──这女孩没有生过孩子,也没有干过重活,腰软得很。这个慵懒的姿势使薛嵩失掉了平常心。作为对淫荡的反应,他的把把又长又硬,抵在红线的后腰上。 

在不知不觉之中,我把自己当作了红线,在一片淡蓝色之中伸展开身体,躺在又冷又湿的空气里。与此同时,有个热烘烘硬邦邦的东西抵在我的后腰上。这个场景使我感到真切,但又毫无道理。我现在是个男人,而红线是女的。假如说过去某个时刻我曾经是女人,总是不大对…… 

第三节 

“早晨,薛嵩醒来时,看到一片白色的雾”,我的故事又一次的开始了。醒来的时候,薛嵩抱着自己的膝盖,蜷着身体坐在一棵大树下,屁股下面是隆起的树根;耳畔是密密麻麻的鸟鸣声。有一个压低的嗓音说:启禀大老爷,天明了。薛嵩抬头看去,看见一个橄榄色的女孩子倚着树站着,脖子上系了一条红色的丝带,她又把刚才的话重说了一遍。薛嵩不禁问道:谁是大老爷?红线答道:是你。你是大老爷。薛嵩又问道:我是大老爷,你是谁?红线答道:你是小贱人。薛嵩说:原来是这样,全明白了。虽然说是明白了,他还是不明白自己为什么会醒在这里。他也不明白红线为什么老憋不住要笑。这地方四周是密密麻麻的野菊花和茅草,中间只有很小的一片空地,这就是说,他们被灌木紧紧地包围着。后来,红线叫他拿起自己的弓箭,出去看看──她自己当先在前面引路,小心地在草丛里穿行,尽量不发出响声。薛嵩模仿着她的动作,但不知为什么要这样做,也不知要到哪里去;但他紧紧地跟住了红线,他怕前面那个橄榄色的身体消失在深草里。 

黎明对我来说,也是个艰涩的时刻。自从我被车撞了以后,早上都要冥思苦想,自以为可以想起些什么,实际上则什么都想不起──这是一种痛苦的强迫症。克治这种毛病的办法就是去想薛嵩。早上起雾时,红线和薛嵩在林子潜行。红线还不断提醒道:启禀老爷,这里有个坑。或者是:老爷,请您迈大步,草底下是沟啊。所到之处,草木越来越密,地形越来越崎岖,一会儿爬上一道坎,一会下到一条沟里。薛嵩觉得这里很陌生,好像到了另一个星球。转了几个弯,薛嵩觉得迷迷糊糊的,头也晕起来了──人迷路后就有这种感觉,而薛嵩此时又何止是迷路。红线忽然站住了脚,拨开草丛。顺着她指的方向看去,里面躺着一条死水牛,已经死得扁扁的了,草从皮破的地方穿了出来。牛头上站了一只翠羽红冠的鸟,脚爪瘦长,有点像鹭鹚。这种鸟大概是很难看到的,薛嵩就说:小贱人,你带我来看鸟吗?红线说不是;然后又捂着嘴笑起来,说道:老爷,您真逗。薛嵩有一点恼怒,小声喝道:什么叫真逗?红线就收起笑容,往后退了半步,福了一福道:是。小贱人罪该万死。然后她继续引路,但是肩头乱抖,好像在狂笑。薛嵩跟着她走去,心里在想:今天早上的事我怎么一点都不懂了? 

我说过,薛嵩在一个老娼妇的把握下长大成人,然后就出发去建功立业。这件事他记得很清楚,以后的事就有点不清不楚。比方说,他怎样来到这片红土山坡,又怎样被手下的兵揪下马来大打凿栗等等。他还影影绰绰记得自己昨天被人砍了一刀,然后就中了暑。夜里又被二十个人围攻,差点死掉了。今天早上又在草丛里醒来,在灌木丛里跋涉。鼻子里吸进了冰冷的雾气,马上就不通气了。这些事和建功立业有什么关系,叫人殊难领会。他也搞不清现在是要去哪里。后来他着了凉,开始打喷嚏。好像就说:请老爷悄声。后来又说:启禀老爷,请不要打喷嚏,别人也有耳朵。最后她干脆转过身来,一把捂住了薛嵩的嘴,对着他的耳朵喝道:兔崽子!打喷嚏时捂着嘴,转过身去!你要害死我们吗?薛嵩觉得眼前这个小贱人真是古怪死了。 

早上,那颗挂起来的人头从梦中醒来,骤然发现自己高高跃起在高空,下面是一片白茫茫的雾气。它感到惊恐万状,觉得自己正在落下去。如前所述,它被吊在了树枝上,是掉不下去的。所以它马上又觉得自己从脑后被揪住,悬在空中了。这一瞬间,它觉得整个头皮都在麻酥酥的疼痛。与此同时,它也发现自己自脖子往下是空空荡荡。一团团的雾气北难以察觉的微风推动,穿过它原来身体的所在,引起强烈的恐惧。醒来时失掉了身体和醒来时失掉了记忆相比,哪种更令人恐惧,我还没有想清楚,总而言之,那颗人头在回忆起自己那个亮丽的身体,觉得它是红蓝两色组成的。有一种可能是这样的:这个身体发着浅蓝色的光,只在乳头、指甲等部位留有暗红色的阴影。另一种可能是身体发着粉红色的光,阴影是青紫色。这两种回忆哪种更真实它已经搞不清楚了。 

与此同时,那个小妓女也从梦里醒来,发现自己被捆得紧绷绷,嘴里还塞了一条臭袜子,也觉得难以适应。然后她就低下头去,看自己身上那些触目惊心的绳索。总而言之,黎明是个恐怖的时分,除非彻夜未眠,你可能发现自己此时失掉了过去,失掉了身体,或者发现自己像一条跳上了案板等待宰割的鱼。 

早上,那个老娼妇坐在木板房的走廊下,身上穿着麻纱褂子。她觉得很困,但又不能去睡,所以就把一把铜夜壶拿了出来,练习往里投石子,那个夜壶也发出叮叮咚咚的声音;同时,她斜眼看那些刺客和雇佣兵在壕沟边上拉锯。她的处境不妙:她请人杀薛嵩,但薛嵩并没有死;所以她已经完全败露了。但她也一点都不着急。虽然她的命运难以预测,但既然已经完全败露,也就不用急了。有一些人很急,他们是被围困的刺客。雇佣兵和刺客在寨中心对峙着。这些兵是一些披头散发、赤身裸体的彪形大汉,站在壕沟边上,挺着胸膛,腆着大肚子,脸上带着蒙娜丽莎似的微笑;双手环抱于胸,把长刀夹在腋下。有一点必须说明,在他们挺出的肚子上,肚脐眼边上凹下去,而是凸出来的。这说明不是脂肪丰厚的肚子,而是惯吃粗食、大肠粗大的肚子;这些人的脑袋又圆又大,都长着络腮胡子。而那些刺客也是同样的一批彪形大汉,退到了壕沟的里面,神情紧张,把刀拿到手里。就这样,黎明在他们头上出现了。开头,最初的阳光在林梢上闪耀,再过一会儿就起雾了。就在起雾时,那些雇佣兵退走了。但他们不是各回各家,而是退到寨外去把守路口;走的时候还说:既然来杀薛嵩,就把薛嵩杀掉;杀不掉别想走。现在这些兵的态度总算是明朗了:他们希望薛嵩死掉,但不肯自己动手去杀。所以,假如有人来杀薛嵩,他们是不管的。那些人杀死了薛嵩退走时,他们也不管。并且仅当那些人没有杀掉薛嵩就想走时,他们才出来挡道。因为有了这些兵,这座寨子成了个捕鼠笼,进来时容易,出去就有点困难了。 

2 

晨雾正在消散时,那颗挂着的人头看到它的刺客兄弟们在用刀把敲打那个老妓女的头,逼问她薛嵩在哪里。它觉得这件事很怪:她怎么会知道薛嵩在哪里?但它不明白,那些人被困在凤凰寨里,心情很坏,总要找个借口来揍人。如前所述,她把头发剃掉了,秃头缺少保护,一敲一个包。在这种情况下,她很想说出薛嵩在哪里,但说不出来。于是她心生一计,说那小妓女和薛嵩比较要好,肯定知道薛嵩在哪里。对此需要解释一下,这个老妓女就喜欢把一切不愉快的事都推到小妓女身上。这个局面有一定的复杂性:刺客揍老妓女,让她说薛嵩在哪里;老妓女就让他们去揍小妓女,并且说她知道薛嵩在哪里;其实大家都知道,无论是老妓女还是小妓女,都不知道薛嵩在哪里。所以,实际上是刺客想要揍人,所以找上了老妓女。老妓女想不挨揍,就说出了小妓女,根据经验她知道,男人一定对揍后者有更大的兴趣。当然,假如谁也不揍谁,那就更好了。 

于是,刺客们回到了屋里,把小妓女抬了出来,拔去她嘴里的臭袜子,恢复了她说话的能力。那女孩先呼吸了几口新鲜空气,然后开始和刺客打招呼:各位大叔,早上好。你们是要活埋我,还是把我填在树心里?因为被捆在了房子里,外面发生的很多事她都不知道。刺客说:都不是的。想请你带我们去找薛嵩。小妓女看到人群里的老娼妓,发现她已头破血流,就笑了起来,朝她努嘴说道:我不知道。她(即那个老妓女)才知道。老妓女听见她这样说,很生气,就说道:你怎能这样说话?咱们是邻居呀。那个小妓女则说:噢!我们是邻居!我还不知道呢。又过了一会儿,那些刺客也会意到了这其中的可笑之处,也跟着笑了起来。那个老娼妓在大家的耻笑之中面红耳赤,马上就提议对小妓女用严刑来逼供;她觉得这帮刺客急了只会用刀把子敲人,在这方面没有想象力;就出了一个主意:把那个小妓女倒吊起来,用青蒿烧烟来熏她的口鼻。假如这招不灵,还有别的招数。严刑拷问有两种不同的效果:一种是让意志坚定的人招出真话,还有一种是让意志不坚定的人招出假话。不管得到哪一种结果,她都能满意。刺客的头子听了以后,抹了抹鼻子,说道:很好。你来做这件事。说完他笑了笑,就和手下的人向后退去,围成一个圆,把这两个女人围在里面。过了一会儿,他又催促道:快动手!我们没时间等你! 

此时这个老妓女只好动手去搬小妓女,准备把她倒吊起来。搬了两下,发现她很重。假如有滑轮组、钢丝绳、手推车等机械,还有可能作成此事。现在的问题是没有这些东西。老妓女说:哪位大爷来帮把手?但没人理她。只有刺客头子咳嗽了一声说:别磨蹭了,快点动手吧。她又和小妓女商量道:我把你扶起来,你自己跳到树边上,然后我把你吊起来──这样可好?小妓女冷冷地答道:你搞清楚些,是你要熏我,不是我要熏你。我为什么要跳到树边上?难道因为我们是邻居?围观的刺客对她的回答报以哄笑和掌声。现在这个老妓女真正感到了孤立无援,四周都是催促之意。 

3 

天明时分,凤凰寨里满是冷牛奶般的雾。这种东西有霜血的颜色,但没有霜雪那样冷。在清晨,雾带来光线──雾里有很多细小的水点,每一粒都发着白光,合起来就是白茫茫的一片。在这白茫茫的一片里,那个老妓女拖着地上一个捆成一束的女孩子,要把她吊到树上去。那地上长满了青苔,相当滑,但那老女人还觉得女孩像是陆地上的一条船,太沉、拖不动。虽然天凉,但空气潮湿,所以那老妓女汗下如雨,像狗一样喘了起来。从吊在树上的人头看来,脚下的空场上虽然留下了一条弯弯扭扭的拖出的痕迹,但这痕迹还不够长,不足以和任何一棵树联系起来。最糟的是那老女人总在改变主意,一会儿想把女孩拖向这棵树,一会儿想把她拖向另一棵树,结果是哪棵也没有拖到;最后她自己也歪歪倒倒地站不直,而且像一座活火山一样呼出很多烟雾。后来,她把女孩撇下,走近刺客头子说:我看不用把她吊起来用烟熏,就放在地下揍一顿也可以。刺客头子想了一想,说道:很好。那个老妓女也觉得很好,就停下来歇口气。过了一会儿,那个刺客头子看到没人动弹,就对老娼妓说:你去揍。那个老妓女也愣了一阵,也很想对那小妓女说:你去揍,但又觉得让人家自己揍自己是不合适的。她只好转头去找可以用来揍人的东西,找来找去找不到。最后,她居然跑到了屋侧,用双手在拔一棵箭竹。别人都觉得她有毛病:谁要是能把一棵活竹子从土里拔出来,那他就不是人,而是一个神。最后她总算是想出了办法:她找一个刺客借了一把刀,砍下了一根箭竹,并把枝岔都用刀修掉。这样她手里就有了一根足以揍人的东西。她决定用这根青竹来揍女孩的屁股。她拿着这根竹子走过去时,那个女孩自动地翻滚过来,露出了身体背面的绿泥。因为她总在挨揍,所以有些习惯成自然的举动。 

后来,老妓女就动手揍她,一连抽了十下,打得非常之疼。那个老妓女当然还想多打几下,但是她用力过猛,手上抽了筋,只好停下来歇歇气,而那个小妓女则伏在地下,嘴里啃着青苔。就在此时,那伙刺客从她身后走过来,揪住她的耳朵,把她按在地下说:好了。你也该歇歇了;同时把那个小妓女从地上放了起来,解开了她的手臂,把竹子放到她手里,说:好了,现在轮到你了。她接过这根竹子,呆愣愣地看到那群刺客把老妓女捆住,撩起了她的麻纱裙子,露出了屁股,然后那些刺客就退后,并且催促道:快开始吧。小妓女问:快开始干什么?那些人说:快开始打她。小妓女问:我为什么要打她?那些人解释道:她先打了你嘛。于是她欢呼了一声,把那根竹子舞得呼呼作响,并且说道:太好了!现在就能打了吗?那个老妓女被捆倒在地下,听见这种声音,连脊梁带屁股一阵阵地发凉──这是因为她不知道这女孩要打哪里。她在恐惧之中一口咬住了一根裸露在地面上的树根。但是那个女孩子并没有打下来,她停下手来问道:我能打她几下?刺客头子说:她打你几下,你就打她几下。那女孩就说:大叔,你把我的脚解开了吧。捆着腿使不上劲啊。这些话使老妓女一下感到了心脏的重压:这是因为,她可没有习惯挨打呀。 

4 

黎明时分,薛嵩和红线走到了寨心附近的草丛里。隔着野草,可以看见寨子里发生的一切。早上空气潮,声音传得远,所以又能听见一切对话。所以,他们对寨子里发生的一切都清楚了。红线说:启禀老爷,该动手了。薛嵩糊里糊涂地问:谁是老爷?动什么手?红线无心和他扯淡,就拿过了他手上的弓箭,拽了两下,说:兔崽子!用这么重的弓,存心要人拉不动……此时薛嵩有点明白,就把弓箭接了过来。很显然,这种东西是用来射人之用的。他搭上一支箭,拉弓瞄向站得最近的一个刺客。此时红线在他耳畔说道:你可想明白了,这一箭射出去,他们会来追我们──只能射一箭,擒贼擒王,明白吗?薛嵩觉得此事很明白,他就把箭头对准了刺客头子。红线又说:笨蛋!先除内奸!亏你还当节度使哪,连我都不如!他把箭头对准了手下的兵。红线冷冷地说:这么多人,射得过来吗?现在一切都明白了,薛嵩别无选择,只好把箭头对准了老妓女……与此同时,他的心在刺痛……原稿就到这里为止。 

我觉得自己对过去的手稿已经心领神会。那个小妓女是个女性的卡夫卡:卡夫卡曾说,美一个障碍都能克服我。那个小妓女也说:这寨子里不管谁犯了错误,都是我挨打。相信你能从这两句话里看出近似之处。薛嵩就是鲁滨逊,红线就是星期五。至于那位老妓女,绝非外国的人物可比,她是个中国土产的大怪物。但她和薛嵩多少有点近似之处,难怪薛嵩要射死她时心会刺痛。手头的稿子没说她是不是被射死了,但我希望她被射死。这整个故事既是《鲁滨逊飘流记》,又是卡夫卡的《变形记》,还有些段落隐隐有福尔斯《石屋藏娇》的意味。只有一点不明白:我为什么要写下这个故事?我既不可能是笛福,又不可能是卡夫卡,更不可能是福尔斯。我和谁都不像。最不像我的,就是那个写下了这些文字的家伙──我到底是谁呢? 

5 

下午,我一直在读桌子上的稿子。这些手稿不像看起来那样多,因为它不断地重复,周而复始,我渐渐感到疲惫。后来发生了一件很不应该的事情:在丧失记忆的焦虑之中,我竟沉沉睡去;而后,带着满脸的压痕和扭歪的脖子,在桌子上醒来;想到自己要弄清的事很多,可不能睡觉啊──这样想过以后,又睡着了…… 

傍晚,我推了一辆自行车从万寿寺里出来,跟随着一件白色的衣裙。这件衣裙把我引到一座灰色的楼房面前,下了自行车。它又把我引入三楼的一套房子里。这个房门口有个纸箱子,上面放了一捆葱。这捆葱外面裹着黄色的老皮,里面早就糠掉了,就如老了的茭白,至于它的味道,完全无法恭唯;所以它就被放在这里,等着完全干掉、发霉,然后就可以被丢进垃圾堆。我在门口等了很久,才进到屋里,然后那件白连衣裙就挂上了墙壁。她很热烈地拥抱我,说:才出院就跑来了……这让我有点吃惊,不知如何反应──才出了医院就跑来了,这有何不对?好在她自己揭开了谜底:“想我了吧。”这就是说,她以为我很想她,所以一出了医院就跑到单位去看她。我连忙答道:是啊,是啊。其实我根本就没有想过她。我谁都没想过──都忘记了。她的热烈似乎暗示着谜底,但我不愿把它揭开──然后,在一起吃饭、脱掉最后一件内衣,到卫生间里冲澡。最后,在床上,那件事发生了。就在此时此地,我不得不想了起来,她是我老婆。我是在自己的家里……恐怕我要承认,这使我有点泄气。我跟着她来时,总希望这是一场罗曼史。说实在的,我什么都想到了,就是没想到我已经结了婚……老婆这个字眼实在庸俗。好在我还记得怎么做爱。其实,也是假装记得。她说了一句:别乱来啊,我就没有乱来。当然,最后的结果我还是满意的──我有家,又有太太,这不是很好嘛。 

我对她的身体也深感满意,她的皮肤上洋溢着一种健康的红色。我也欣赏她对性那种不卑不亢的态度。但她若不是我老婆,是个别的什么人的话,那就更好了。我头疼得厉害。这是因为我不管怎么努力,也想不起她的名字来。户口本上一定有答案,要是我知道它在哪里就好了……这套房子里满满当当塞满了家具,想在这里找到一个小本子也非易事……她温婉而顺从,直到午夜时分。此时她猛地爬了起来,恶很很地说道:我要咬你!任何一个男人到了这时,都会感到诧异,并且急于声明自己和食品不是一类东西。但是我没有。我只是坐了起来,诧异地问道:为什么?她很凶暴地说:因为你拿着脑袋往汽车上撞,想让我当寡妇。我想了想,觉得罪名成立──寡妇这个名称太难听了,难怪人家不想当;就转身躺下。如你所知,男人的背比较结实,也比较耐咬。但她推推我的肩膀说,翻过来。我翻过身来,暴露出一切怕咬的部位,在恐惧中紧闭眼睛──但她只是轻轻地咬我的肚子,温柔的发丝拂着侧腹部,还响着带着笑意的鼻息。感觉是相当好的。因为这些事件,我对自己又满意起来了…… 

此事发生以后,她问我:上次玩是什么时候了?我假装回忆了一阵,然后说:记不得了。她说:混帐!这种事你都记不得,还记得什么。我坦白道:说老实话,我什么都记不得。她嗤地笑了一声道:又是老一套。你脑袋上有个疤,可别吓唬我。我说,好吧,不吓唬你──我桌上那篇稿子到底是谁写的?如你所知,这是我最想知道的问题──我很希望它是别人写的,因为我对它不满意。但她忽然说:讨厌,我不理你了,睡觉。说着她拉过被单,转过身去睡了。我想了想,觉得我“记不得”了的事目前不宜谈得太多,免得她被吓着。所以,就到此为止罢。 

尽管心事重重,我又有点择席,但我还是睡着了。顺便说一句,那天夜里起夜,我在黑暗中碰破了脑袋。这说明我虽能想起自己的老婆,还是想不起自己的房子,很有把握地走着,一头撞在墙上了。失掉记忆这件事,很不容易掩饰,正如撞破了的眼眶也很不容易掩饰。

\section{第四章}

第一节 

清晨,在床上醒来时,我撩开被单,看到有个身体躺在我的身边──虽然我知道她是我老婆,但因为我什么都不记得,只能把她看作是一个身体──作为一个身体,她十分美丽,躺在微红色的阳光里──这间卧室挂着塑料百页窗帘,挡得住视线,挡不住阳光;所以这个身体呈玫瑰红色。我怀着虔诚之意朝她俯过身去,把我的嘴唇对准她身体的中线,从喉头开始,直到乳房中间,一路亲近下来,直到耻骨隆起的地方──她的皮肤除了柔顺,还带一种沙沙的感觉,真是好极了。此时我发现这身体已经醒来了;此后我就不能把她看作一个身体。此时我抬起头来,看到她的眼睛,她眼睛里流露出的,与其说是新奇,倒不如说满是惊恐之意。她翻过身去,趴在床单上。我又把嘴唇贴在她的脊梁骨上,从发际直到臀部……她低声说道:不要这样,还得上班呢,语气温柔;再后来,她匆匆地用床单裹起身体,从我视野里逃开了。对那个身体的迷恋马上融进我的记忆里。 

早上,我来上班,坐在高高的山墙之下自己的椅子上,重读自己的手稿时,马上看出,在这个故事里,有一个人物是我自身的写照。他当然不是红线,也不是老妓女或者小妓女,所以只能是薛嵩,换言之,薛嵩就是我。我不应该如前面写到的那样心理阴暗。我应该是个快乐的青年,内心压抑、心理阴暗对我绝无好处。所以我的故事必须增加一些线索──既然已经确知这稿子是我写的,我也不必对作者客气──人和自己客气未免太虚伪──可以径直改写。 

一切如前所述,晚唐时节,薛嵩在湘西做节度使,在红土山坡上安营扎寨。这座寨子和一座苗寨相邻,在旷野上有如双子星座。有一天,薛嵩出去挑柴,看到了红线,他很喜欢她,决定要抢她为妻。他像我一样,是天生的能工巧匠,也不喜欢草草行事。所以他要打造一座囚车,用牛拉着,一起出发去抢红线,抓住她之后,把她关在车里,拉回寨来。如前所述,凤凰寨里的人都抢苗女为妻,把她们打晕后放在牛背上扛回来。那些男人不过是些小兵,而薛嵩却是节度使;那些女人不过是普通的女人,红线却是酋长的女儿。让她被关在囚车里运进凤凰寨,才符合双方的身分。 

我的故事重新开始的时候,薛嵩已经不是个纨绔子弟,成了一位能工巧匠。这就意味着他到湘西来做节度使,只是为了施展他的才华。所以,他先在红土山坡上造好了草木茂盛的寨子,就进一步忙了起来,给每个人造房子,打造家具;而且从中得到极大的乐趣。等到房子和家具都造好以后,他又忙于改良旧有的用具,发明新的用具,建造便利公众的设施。直到有一天,他到外面去担柴,准备烧一批自来水用的陶管子,忽然看到了红线,一切才发生了改变。此后,他就抛下一切工作不做,去建造囚禁红线的囚车──虽然凤凰寨里有很多工作等着他做。 

冒着雨季将至时的阵雨,薛嵩带着斧子出发,到山上去伐木做这个囚车。如果用山梨一类的木料,寨子里也有。但他已经决定,这座囚车要用柚木来建造。就我所知,不足三十岁的柚树只是些普通的木料,三十岁以上的柚木才是硬木,可以抛出光泽。高龄的柚木抛光之后,色泽与青铜相仿,但又不像青铜那么冷,正是做囚车的合适材料。薛嵩到山上去,找最粗的柚树下手,斧子只会锛口,一点都砍不进去──这是因为树太老,木料太硬,应该用电锯锯,但薛嵩又没有这种东西;细的柚树虽比较嫩,能够砍动,他又看不上眼。最后他终于伐倒了一棵适中的柚树,用水牛拖回家里,此时他已疲惫不堪,还打了满手的血泡。此后他把树放在院内的棚子里,等待木材干燥。雨季到来时,天气潮湿,木头干得很慢,他就在那座棚子里生起了牛粪火,来驱赶潮气。与此同时,他开始画图,设计那座关红线的囚车……我喜欢这样来写。 

今天上午,有一个男人到寺院里来找我。他的额头有点秃,身材有点肥胖,左手的无名指上戴着很宽的金戒指,穿着绿色的西服……他说他是我表弟,在泰国做木材生意。虽然明知无望,我还是回忆了一番;但我想不起有过任何表弟。这说明我远远还没恢复记忆。然后他递给我一张名片,这张名片比扑克牌略厚,是柚木做成的。上面有镌出的绿字,陈某某,某某木材出口公司总经理。这张名片在手里沉甸甸的,带有一点檀香气,嗅起来像一块肥皂。我把它放到鼻子下面嗅着,还是记不起有这样一个表弟。于是他就责备道:表哥,你怎么了,真把什么都忘了?小时候咱俩净在一块玩。我说道:是呀,是呀;但口气却没有什么把握。这个自称是我表弟的人拿出皮夹来,里面有一张相片。这是我们小时的合影──一张五寸的黑白相纸,已经有点发黄了,上面有两个男孩子,这张相片引起了我极大的兴趣。 

现在我又取出了那张柚木名片,把它夹在指缝中。它好像一块铁板,但比铁要温柔。正是因为这个缘故,薛嵩决定要用它做成一个囚笼,把红线装在里面,运进凤凰寨。这座笼子相当宽敞,有六尺见方,五尺高,截面是四叶的花朵形;上下两面是厚重的木板,抛光,去角;中间用粗大的圆柱支撑。薛嵩还想在笼子里装了一张凳子──更准确地说,是一块架在空中的木板;在木板上放了一块棕织的座垫。众所周知,在硬木上可以雕花。薛嵩给囚笼的框子设计了一种花饰,是由葡萄藤叶组成。但他有很久没有见过葡萄,画出的葡萄叶和篦麻叶相似。这样一座笼子可以体现薛嵩的赤诚,也可以体现他的温柔。用笼子的厚重、坚固体现他的赤诚,用柚木的质地和光泽来体现他的温柔……而红线坐在赤诚和温柔中间,双手和双脚各由一块木枷锁住,显得既孤独,又高傲。整个雨季里,薛嵩都坐在那间新建的草房里,在柚树的旁边,烤着牛粪火画图。从柚树砍断的一端不断地流出绿水,不顾外面降落的雨水,草房里温暖如春。有好几个月就这样过去了。 

在我表弟拿出的相片上,两个男孩子都穿着蓝布学生制服。我还有点记得那种衣服,它有一个较小的直领,左胸上有一个暗兜;好处是式样简朴,年轻人穿上后,形象清纯一些;坏处是兜太少。两个孩子都留着平头,其中一个站在画面的中央,脸迎着阳光,一副虎头虎脑的模样,体质比较强壮。另一个站在画面右侧,略微低着头,把阴影留在了脸上。瘦长脸,体质也比较瘦弱。我把手指放在中间那个孩子的下巴上说:啊,原来我小时候是这样的。此时我表弟略呈尴尬之色,说道:表哥,你认错了。中间这个是我。后来,我又仔细看了看右面那个孩子,脸像和我有点近似。但我还是觉得,中央那个才是我。他(或者说,是过去的我)神情专注,好像很固执。他的皮肤也比较黑。在我的想象中,就是这个男孩子躲在雨季的屋顶下,在牛粪火边蜷着褚石色的身体,在画着一幅囚车的图样,想把他爱的女孩装进去。 

2 

薛嵩决定要抢红线为妻,为此他要做一辆囚车,把红线装在里面运进凤凰寨。他把砍到的木材焙干,又找人帮忙把木头解成板材──因为木头太硬,这件事可不容易。这时候别人都以为他想要打家具,都劝他别用这样硬的木头,但他不听。他还想做两块枷,分头枷住红线的手和脚。后来他又决定从手枷做起,以此来练习他的木匠手艺。这是因为做手枷用的木料有限,做坏了也不可惜;除此之外,还可以让大块的木板继续干一干。这个东西可以分成两半,也可以借助一些卡榫严丝合缝地合为一体。当然,分成两半时,木板上应该和红线的手腕相吻合。做到这里时,薛嵩就开始冥思苦想,因为他不知道红线手腕的尺寸。后来他觉得不妨实际看一看,就丢下木匠活,出发去找红线。 

此时雨季已过,原野上到处是泛滥的痕迹──窄窄的小河沟两边,有很宽的、茵茵的绿草带──再过一些时候,烈日才会使草枯萎,绿色才会向河里收缩。此时草甚至从河岸上低垂下来,把土岸包得像个草包。渠平沟满,但水总算是退回了河里。红线就在小河里摸鱼。踏站在水里,双手在河岸下摸索,因为鱼总呆在岸边的泥窝里──水面平静,好像是一层油;河也不像在流动。这是因为雨季里落下的水太多,只能慢慢地流走。我总觉得自己在热带的荒野地方呆过,否则,这个景象也不会如此逼真地出现在我眼前。这片荒原色彩斑斓,到处是被陆地分割后的静止水面,天上有很多云,太阳也看不见。 

薛嵩就在这个景象面前,但他全神贯注地看着红线。看了好半天,只看到一个圆滚滚的小屁股;还看见一个脊背,上面有一串脊梁骨。薛嵩把每一块脊梁骨的位置和形状通通记住了,但他还是不知红线的手腕有多粗。这是因为他站在红线的背后,离得还比较远。而红线则躬下身去,闭着眼睛,双手在淤泥中摸索──这些泥是这个雨季里刚刚淤下来的,还没有变成土,所以细腻到几乎温柔,而且是暖洋洋的。有时候,她的指端遇上一股冷流,那就是淤泥下的一下股泉水。有时候她的指端遇上了一股温暖,那就是摸到了自己的脚趾。有时候手指遇上了蠕动中的黄鳝,因为现在天气暖,再加上是在软泥里,就很难把它捉住──这种东西滑得很。红线期待着手忽然伸到一个空腔里,这里有很多尖刺来刺她的手──这就是她要找的鱼窝。那里面有很多高原上的胡子鲇鱼,密密层层地挤在一起,发现有人把手伸起来,就一齐去啄那只手──其实不啄还好些,这一啄把自己完全暴露。假如发现了这种鱼窝,红线就会不动声色地把手抽回去,做好准备,再把它们一举捉光。我不记得自己什么时候在河沟里摸过鱼,但是这个过程我感到十分亲切。红线全神贯注地做这些事,但也感到有一股冷流,就如一股泉水,阴阴地从背后袭来。作为一个小姑娘,她很知道这是有一个臭男人在打她的主意。所以,后来她只是假装在摸鱼,实际上却在听背后的声音:有无压抑的鼻息、蹑手蹑脚的脚步声──她准备等他走近,然后猛一转身,用膝盖朝他胯下一顶──此后的情景也不难想象:那个男人蹲在水里,翻着白眼,嘴里欧吼欧吼地乱喊一通。说实在的,我很希望薛嵩被红线一膝盖顶在小命根上,疼得七死八活。但是这件事并未发生。 

实际发生的事情是这样的:后来,红线站起身来,用手往前顶了盯自己的腰,就转过身来;发现身后空无一人,只是在小河对面老远的地方,薛嵩坐在草地上。她眯起眼来说:噢!是薛嵩!如前所述,此时雨季刚过,天上布满了密密层层的云朵,好像一窝发亮的白羽毛,天地之间也充满了白云反射的光线。红线发现了薛嵩,就涉过了小河,水淋淋地坐在薛嵩身边,告诉他一些鸡毛蒜皮的事情:比方说,现在雨季刚过,不冷不热,是一年里最好的时节。过一些日子,天气要转为湿热。再过一些日子,天气还会转为干热。这是因为她觉得薛嵩是个新来的人,不知道此地的情况,需要她来介绍一番;还因为她对薛嵩有好感。薛嵩一声不吭地听着,猛地一伸手,捉住了她的左手,用一根棉线量了她的手腕;然后又捉住她右手,量了右手的手腕。本来量一个手腕就够了,但薛嵩害怕红线两只手的腕子不一样粗,就多量了一只。假如你是一位能工巧匠,就会知道,小心永远不会是多余的。作好了这两件事,薛嵩满脸通红,起身拔脚就走,对自己的所作所为未加解释。他也觉得自己的行径太过突兀。但不管怎么说,红线手腕的尺寸他已知道了。剩下红线一人坐在草地上,她觉得薛嵩的举动像一个谜。她想了一会儿,没想出他要干什么,就起身下河去,继续摸鱼。据我所知,那一天她找到了好几个鱼窝,不但满载而归,还有几个鱼窝原封未动地留着,只是在岸上做了标记。这种标记是一根竹篾条,上面用她的牙咬过。以后别人在河里摸到了这个鱼窝,看到了岸上有这种标记,就知道这是红线先发现的,是她的财产,就不摸坑里的鱼。而红线原准备第二天来摸这些鱼,但第二天她把这些鱼窝通通忘记了,总也不来摸,这些泥坑里的鱼因而长命百岁;比那些被捉住的鱼幸福得多。据我所知,后者被逮到了篓子里还继续活着,直到红线烧熟了一锅粥,把那些鱼倒进去,才被活生生地烫死了。据说这种粥很是鲜美,而且是补的。但那些被烫死的鱼不见得会喜欢这样的粥。 

等到天气热了起来,红线每天早上到草地上去捉蝗虫,用细竹签把它们穿起来。那些蝗虫被扎穿以后,还在空中猛烈地蹬着腿,嘴里吐出褐色的粘液。每捉到三五串,她就在草地上生一堆火,把蝗虫放上去烤,那些虫子猛蹬了几下腿,就僵住不动了;但它们的复眼还瞪着,直到被火烤爆为止。红线继续烤着蝗虫,直到它们通体焦黄而且滋滋地冒油,就把它们当羊肉串吃掉。蝗虫又香又脆,但这些蝗虫对自己是如何又香又脆这一点,肯定缺少理解。然后这个小女孩就到干涸的水田里去挖黄鳝;挖到以后放到干草里烧。黄鳝在被烤着以后会往地下钻去,但是遇上了一片硬地,变成罗旋状,就被烧死在那里。此后红线把它的尸体拿起来,吹掉上面的灰,然后吃掉。假如她逮住了一条蛇,就把它的皮扒掉,扔到滚开的水里;蛇的身体就在锅里翻翻滚滚。总而言之,她是这片荒原上的一个女凶手。而薛嵩却躲在家里,给这个凶手制造枷锁。 

3 

知道了红线手腕的尺寸,薛嵩很快把手枷造成了。那东西的形状像一条鲤鱼,不仅有头、有身子、有尾,嘴上还有须。但是它身上有两个洞,这一点与鱼不同。薛嵩以为,红线把它戴在手上时,会欣赏到他的雕刻手艺。他还想把红线的脚也枷住,并且要把足枷做成圆形,像莲花的模样。但他又不知道红线脚腕的尺寸,所以又出发去找红线。这一回他看到红线在对付白蚁,把耳朵贴在蚁冢上听里面的动静。她告诉薛嵩,假如蚁窝里闹哄哄的,就是到了繁殖的时刻。当晚会有无数春情萌动的繁殖蚁飞出来,互相追逐、交配。配好以后落在地下,咬掉翅膀,钻到地下去,就形成一窝新的白蚁。不幸的是,当他们飞出蚁巢时,红线会在外面等着,用一个大纱袋把它们全部兜住;等他们在里面交配完毕,咬掉了翅膀,就把他们放到锅里去炒。据说这种白蚁比花生米还要香;要用干锅去爆炒,以后还能出半锅油。她还说,假如今晚薛嵩也来帮助捉白蚁,她就把炒白蚁分他一半。可是薛嵩另有主意,他猛地蹲下身来,用棉线量了她脚腕的尺寸,然后又跑掉了。虽然红线不知道薛嵩的种种设计,但也隐隐猜到了他要干什么──就像一个人想到自己早晚会死掉一样。对此她有点忧伤。此后红线继续在山坡上嬉戏,但心里已经有了一点隐患。因为她已知道,薛嵩早晚要抢她为妻。 

我表弟说,小时候我的手很巧,喜欢做航模、半导体收音机一类的东西。我的手很嫩,只有左手中指上有点茧子;这说明起码有十年我没做过手工活。从这点茧子上可以看出我原是左撇子,用左手执笔。但我现在不受这种限制,想用哪只手就用那只手:一般情况下我尽量用右手,急了用左手,因为左手毕竟灵活些。不管怎么说罢,我喜欢知道自己小时候手巧。我表弟还说,我从小性情阴沉,寡言少语,总是躲人,好像有些不可告人的秘密;这个消息我就不大喜欢。我想象中的薛嵩有一双巧夺天工的手,用一把雕刻刀把一块木头雕成一只木枷,然后先用粗砂打、后用细砂抛光,又用河床里淘出的白膏泥精抛光,这时候那个木枷已被抛得很明亮。最后一道工序是用他自己的手来抛光──薛嵩的皮肤是棕色的,但手心的皮肤和任何人一样是白的──说来也怪,经手心的摩娑,那枷就失去了明亮的光泽,变得乌溜溜的,发着一种黑光;但也因此变得更温和。就这样,他把手枷和足枷都做好了,挂在墙上。有了这两件成品,薛嵩的信心倍增。开始做囚笼的零件──首先从圆笼柱做起。但无论用斧用刨,都做不出好的圆形,为此薛嵩费煞苦心,终于决定要做一架旋床。他先设计出了图样,又砍了一棵野梨树,把它做成了。但是这旋床上第一件成品却不是柱子,而是一个棒棰形的东西,是用柚木枝杈车成的,沉甸甸的很有点分量。 

薛嵩在棒端包好了软木,在自己头上试了一下,只在脑后轻轻一碰,就觉得天旋地转,一头栽倒在地上;过了一小时才爬起来。拿这么重的一根棍子去打个小姑娘,薛嵩自己也觉得不好意思。他只好另做了一根,这回又太轻,打在后脑勺上毫无感觉。后来他又做了很多棍子,终于做出了最合适的木棍。这棍子既不重,又不轻,敲在脑袋上晕晕糊糊的挺舒服;晕倒的时间正好是十五分钟。薛嵩在这根棍子上拴了一根红丝线作为标记。这使别人猜到了他的目标是红线。于是就有人去通知她说:大事不好了,我们那位薛节度使造了十几根棍子,要打你的后脑勺!红线此时正手执弹弓看树上的鸟儿,背朝着传话的人。她也不转过身来,就这么说道:是嘛──口气有点随意。但传话的人知道,她不是漠不关心;于是就加上了一句:他要来抢你!红线耸耸肩说:抢就抢吧。等到那人要走时,她才加上一句:劳你问他一句,什么时候来抢我。传话的人没想到她会是这样,简直气坏了;所以不肯替她去问薛嵩。红线那天射下了好几只翠羽的鹦鹉,活生生地拔掉了它们的毛,放在火上烤得半生不熟,然后全都吃下去了。然后她就回家去,在草地上剩下一堆黑色的灰烬,还有一堆根上连着血肉的绿色羽毛。 

后来,薛嵩把放柚木的草棚改成了工作间。这是因为他不想让别人看见他在做什么。他用竹片编了四面墙,把它悬挂在四根柱子上,棚子就变成了房子。他用搀了牛粪的泥把墙里抹过,再用石灰粉刷一遍,里面就亮了很多;对于外墙,他什么都没有做。这间房子的可疑之处在于既没有门,也没有窗子,要顺着梯子爬到墙上面,再从草顶和墙的接缝处钻进去──当然,里面也有一把梯子,这样他就避免了跳墙。他在地上生了两堆伙,一堆是牛粪火,用来熬胶。在牛粪火里,放了好多瓦罐,熬着牛皮膘、猪皮膘、鱼鳔膘、骨膘,这些胶各自有不同的用处,但我没作过木匠,不太清楚。另外一堆是炭火,用来制作铁工具。薛嵩没有风箱,用个皮老虎来代替。在牛粪火边上是木匠的工作台,在炭火边上是铁钻子。薛嵩在这两个地点之间来回奔走,到处忙碌。虽然忙,但他绝不想请帮手,他在享受独自工作的狂喜。像这样的心境,我也仿佛有过。寨子里的人只听到铁锤打铁,斧子砍木头,却见不到薛嵩。因此就有种传闻,说他已经疯了。直到有一天,他把工作间的墙推倒,人们才知道他做了一个木笼子,有八尺见方,一丈来高。到了此时,他也不讳言自己的打算:他想把红线逮住关在里面。别人说,要关一个小女孩,用不着把笼子做那么高。薛嵩只简单地回答说:高了好看。我以为他的看法是对的。 

4 

有人跑去告诉红线薛嵩造了个笼子,还补充道:看样子他想把你关在里面,一辈子都不放出来。红线有点紧张,脸色发白,小声地说道:他敢!告诉她这件事的人说:有什么他不敢干的事?你还是快点跑了吧。然后,这个人看到红线表现出犹豫的神情,感到很满意。这是早上发生的事。到了中午,红线就潜入薛嵩的后院,看他做的活。结果发现那座笼子比她预料的还要大,立在草棚里,像一个高档家具。在笼子的四周还搭了架子,薛嵩在架子上忙上忙下,做着最后的抛光工作。在笼子后面,还残留着最后一堵墙,上面挂着好几具木枷,还有数不清的棍棒。红线大声说道:好哇!你居然这样的算计我!薛嵩略感羞愧,但还可以用勤奋工作来掩饰。此时还有两根笼柱没有装上,红线就从空档中钻进笼子里。如前所述,笼子里有一条长凳,这凳子异常的宽,所以说是张床也可以,上面铺着棕织的毯子。红线就躺到长凳上,双手向后攀住柱子,说道:这里面不坏呀。好吧,你就把我关起来吧。但上厕所时你可要放我出来呀。薛嵩听了倒是一愣,他根本就没打算把红线常关在笼子里。他把墙打掉,是想给这笼子装车轮。总而言之,这囚笼只是囚车的一部分,不是永久的居室。 

愣过乐以后,薛嵩想到:既然人家提了出来,就得加以考虑,给这笼子装个活门。但到底装在哪里,只有在笼里面能看清。所以他叫红线出来,自己钻到笼里,上下左右的张望。而红线在外面溜溜答答,抄起一具木枷,往自己身上比划了一下说,好哇薛嵩,这种东西你也好意思做。薛嵩的脸又红了一下。他没有回答。后来红线就帮薛嵩干活──帮他造那些打自己、关自己、约束自己的东西。孩子毕竟是孩子,就是贪玩,也不看看玩的是什么。有了两个人,工程的进度就加快了。但直到故事开始的时候,这囚车还没有完工,但已在安装抽水马桶。薛嵩给红线做了一张很大的梳妆台,台上装了一面镀银的铜镜,引得全凤凰寨的人都来看。有人说,薛嵩对红线真好。也有人说,薛嵩太过奢华,要遭报应。 

第二节 

在故事开始时,我提到有个刺客(一个亮丽的女人)来刺杀薛嵩。据说此人在设计狙杀计划、设伏、潜入等等方面,常有极出色的构思,只是在砍那一刀时有点笨手笨脚;所以没有杀死过一个人。她也没能杀死薛嵩,只砍掉了他半个耳朵。还有一种说法是,这个女人的目标根本就不是薛嵩,而是红线。只是因为被薛嵩看到,才不得不砍了他一刀。后来她再次潜入薛嵩的竹楼,这回不够幸运,被红线放倒了。这件事很简单:红线悄悄跟在她身后,拿起敲脑袋的棍子(这种东西这里多得很)给了她一下,就把她打晕了。等到醒来时,她发现自己的手脚都被木头枷住,躺倒在地上,身前坐了一个橄榄色的女孩子,脖子上系着一条红带子,坐在绿色的芭蕉叶上。这女孩吃着青里透黄的野樱桃,把核到处乱吐,甚至吐到了她身上;并且说:我是红线,薛嵩是我男人。那女刺客蜷起身子,摇摇脑袋,说道:糟糕。她记得自己挨了一闷棍,觉得自己应该感到头晕,后脑也该感到疼痛,但实际上却不是,因为那个棍子做得很好──这个故事因此又要重新开始了。但在开始之前,应该谈谈这囚车为什么没完工。照薛嵩原来的构思,完成了囚笼就算完成了囚车的主体部分。但后来发现不是这样,主体部分是那对车轮。笼子这样大,车轮也不能小。按薛嵩的意见,车轮该用柚木制造;但木材不够了,又要上山砍树。但红线以为铁制的车轮更好。经过争论,红线的意见占了上风,于是他们就打造轮辐、车轴,还有其它的零件。做到一半,忽然想到连轮带笼,这车已是个庞然大物,有两层楼高,用水牛来拖恐怕拖不动。于是又想到,由此向南不过数百里,山里就有野象出没。在打造车轮的同时,他们又在讨论捕、训、喂养大象的事。他们做事的方式有点乱糟糟,就像我这个故事。但是可以像这样乱糟糟的做事,又是多么好啊。 

在这个乱糟糟的故事里,我又看到了我自己。我行动迟缓,头脑混乱,做事没有次序。有时候没开锁就想拉开抽屉,有时没揭锅盖就往里倒米。但那个自称是我妻子的女人并不因此而嫌弃我。现在就是这样,我乱拔了一阵抽屉,感到精疲力尽,就坐下来,指着它说:抽屉打不开。她走过来,拧动钥匙,然后说,拉吧──抽屉应手而开。我只好说:谢谢。你帮我大忙了。这是由衷的,因为刚才我已经想到了斧子。她从我身边走开,说:你这都是故意的。我问:为什么呢?她说:你想试试我到底是不是你老婆。这就是说,我故意颠三倒四。假如她不是我老婆,就会感到不耐烦;假如是我老婆,就不会这样。所以,结论是:她是我老婆,虽然我自己想不起来了……她想得是有道理的。我说:原来是这样,我明白了。她又折了回来,一把搂住我的头,把它压在自己的乳房上,说道:你真逗……我爱你。然后把我放开,一本正经地走开。这件事的含义我是明白的:不是我老婆的女人,不会把我的头压在自己乳房上。所以,结论还是:她是我老婆。不会有别的结论了。白天的结论总是这样。晚上则相反。按夫妻应有的方式亲近过之后,我虔诚地问:我没有弄疼你吧?你还没有讨厌我吧?回答是:讨厌!你闭嘴!这不像是夫妻相处的方式。因为有晚上,我已经彻底糊涂了。我的故事又可以从新开始道:某年某月某日,在凤凰寨、薛嵩家的后院里,那个亮丽的女刺客坐在一捆稻草上,手脚各有一道木枷锁住。她的身体白皙,透着一点淡紫色。红线站在她面前,觉得这个身体好看,就凝视着她。这使她感到羞涩,就把手枷架在膝盖上,稍微遮住一点;环顾四周,所见到的都是庄严厚重的刑具,密密麻麻。身为刺客,失手被擒后总会来到某个可怕的地方,她有这种思想准备。但她依然不知人间何世。同时,因为这个刺客的到来,红线和薛嵩生活的进程也中断了……我真的不知道,这个故事会把我引向何处。 

2 

我的故事从红线面对那个女刺客时重新开始。她对她有乐好感,就说:来,我带你看看我们的房子。世界上任何地方的人招待客人,都从领他看房子开始。那个女刺客艰难地站了起来,看着自己脚上的木枷,说道:我走不动呀。红线却说:走走试试。然后女刺客就发现,那个木枷看似一体,实际上分成左右两个部分,而且这两部分之间可以滑动,互相可以错开达四分之三左右……总而言之,带着它可以走,只是跑不掉。那刺客不禁赞美道:很巧妙。红线很喜欢听到这样的话,她又说:你还不知道,手也可以动的。于是刺客就发现,手上的枷也是两部分合成,中间用轴连接,可以转动,戴着它可以掏耳朵、擤鼻子,甚至可以搔首弄姿。这些东西和别的刑具颇有不同,其中不仅包含了严酷,还有温柔。刺客因此而诧异。这使红线大为得意,就加上一句:这可是我的东西。借给你戴戴。那刺客明白这是小孩心性,所以笑笑说:是。是。我知道。这使红线更加喜欢她了。她引她在四处走了一遭,看了竹楼,但更多的是在看她和薛嵩共同制造的东西。特别是看那座未完工的囚车。在那个深棕色的庞然大物衬托下,那个女人显得更加出色。看完了这些东西,她回到那堆稻草上,跪坐在自己的腿上,出了一阵神,才对红线说:你们两个真了不起。说实话,真了不起。红线听了以后,从芭蕉叶上跳了起来,说道:我去烧点茶给你──估计得到晚上才能杀你。然后她就跑了。只剩女刺客一个人时,她不像和红线在一起时那么镇定。这是因为红线刚才说了一个“杀”字,用在了她身上;而她只有二十二岁,听了大受刺激。 

后来发生的事是这样的:红线提了一铜壶茶水回来,还带来了一些菠萝干、芒果干。她把这些东西放在地下,拿起一把厚木的大枷说:对不起啊……我总不能把滚烫的茶水交在你手里,让你用它来泼我。那女人跪了起来,把脖子伸直,说道:能理解,能理解。红线把大木枷扣在她脖子上,把茶碗和果盘放在枷面上,用一把亮银的勺子舀起茶水,自己把它吹凉,再喂到她嘴里。如此摆布一个成年美女,使红线觉得很愉快。而那个刺客就不感到愉快。她想:一个孩子就这样狡猾,不给人任何机会……然而我的心思已经不在事件的进程之中。在那个枷面上,只有一颗亮丽的人头,还有一双性感的红唇。当银勺移来时,人头微微转动,迎向那个方向……这个场景把我的心思吃掉了。 

那个女人在院子里度过了整个白天。早上还好,时近中午,她感觉有点冷,然后就打起了哆嗦。后来她对红线说:喂,我能叫你名字吗?红线说:怎么不可以,大家是朋友嘛;她就说:红线,劳驾你给我生个火。我要冷死了。红线斜眼看看她,就拿来一个瓦盆,在里面放了两块干牛粪,点起火来。那女人烤起火来。当时的气温怕总有三十八九度,这时候烤火……红线问道:你是不是打摆子?女人答道:我没有这种病。红线接着说下去:那你就是怕死;同时用怜悯的目光看她。那女人马上否认道:岂有此理!我也是有尊严的人,哪能怕死?来杀好了……她滔滔不绝地说了起来,但红线继续用怜悯的眼光看她,她就住了嘴。过了一会儿,她又承认道:是。你说得对。我是怕死了;说着她又大抖起来。后来她又说:红线,劳驾给我暖暖背。火烤不到背上啊。红线搂住她的双肩,把橄榄色的身体贴在她背上。如此凑近,红线嗅到了她身上的香气,与力士香皂的气味想仿,但却是天生的。虽然刚刚相识,她们已是很亲近的朋友。但在这两个朋友里,有一个将继续活着,另一个就要死了。 

3 

有一件必须说明的事,就是对于杀人,红线有一点平常心。这是因为原来她住的寨子里,虽不是总杀人,偶尔也要杀上个把。举例来说,她有一个邻居,是三十来岁一个独身男子,喜欢偷别人家的小牛,在山凹里杀了吃掉。这件事败露之后,他被带到酋长面前;因为证据确凿,它也无从辩解,就被判了分尸之刑。于是大家就一道出发,找到林间一片僻静之地。受刑人知道了这是自己的毙命之所,并且再无疑问之后,就进入角色,猛烈地挣扎起来。别人也随之进入角色,一齐动手,把他按倒在地,四肢分别拴到四棵拉弯的龙竹上,再把手一松,他就被弹向空中,被绷成一个平面,与一只飞行中的鼯鼠相似。此时已经杀完了,大家也要各自回家。但这个人还没死,总要留几个人来陪他。红线因为是近邻,也在被留的人之中。这些被留的人因为百无聊赖,又发现那个绷在空中的人是一张良好的桌子,就决定在他身上打扑克牌。经过受刑者同意,他们就搬来树桩作为凳子,在他身边坐下来。为了对他表示尊敬,四家的牌都让他看,他也很自觉地闭着嘴,什么都不说。但是这里并不安静,因为受刑人的四肢在强力牵引之下,身体正在逐步解体,他也在可怕的疼痛之中,所以时而响起“剥地”一声。这可能是他的某个骨节被拉脱臼,也可能是他咬碎了一颗牙。不管是什么,大家都不闻不问。红线坐在他右腿的上方,右肋之下。伸手拿牌时,右手碰到一个直撅撅、圆滚滚、热烘烘的东西。她赶紧道歉道:对不起,不是有意挑逗你!对方则在牙缝里冷静地答道:没关系!我都无所谓!严格地说,那东西并不直,而是弧线形的,头上翘着;也不太圆,是扁的。红线问道:平时你也这样吗?回答是:平时不这样,是抻的──这就是说,假如一个人在猛烈的拉伸中,他的那话儿也会因此变扁。在牌局进行之中,大家往后挪了几次位子,因为他正变扁平,而且慢慢向四周伸展开来。后来他猛然喝道:把牌拿开!快!然后,他肚皮裂开、内脏迸出、血和体液飞溅;幸亏大家听了招呼,否则那副纸牌就不能要了。 

后来,那位偷牛贼说:现在我活不了啦。你们放心了吧?可以走了。此时大家冷静地判断了形势,发现对方已被拉成了个四方框子。肠子、血管和神经在框内悬空交织,和一张绷床相似。像这个样子想再要活下去,当然多有不便。所以大家同意了他的意见,离开了这个地方。走时砍倒了几棵树,封锁了道路;这个地方和这个人一样,永远从大家的视野中消失了。由此,对杀人这件事,可以有一个定义:在杀之前,杀人者要紧紧地盯住被杀者,不给他任何活下去的机会;在杀之后,要忍心地离去,毫不留恋。在之前之后中间,要有一个使对方无法存活的事件。对于这位偷牛贼来说,这事件就是被拉成床框。在这个杀法里,事件发生得很快。别的杀法就不是这样。举例来说,有一种杀法是把被杀者的屁股割开,让他坐在一棵竹笋上。此时你就要耐心等待竹笋的顶端从他嘴里长出来。此后,他就大张着嘴,环绕着这棵竹子,再也挣不脱……对于这位女刺客,则是把她的脖子砍断。要如此对待一个朋友,对红线是很大的考验。越是杀朋友,越是要有平常心。身为苗女,她就是这样想问题。她没觉得有什么不对。 

还有一件需要补充的事,就是对于让自己被杀掉一事,那个女刺客没有平常心。她对红线抱怨道:你看,我活着活着,怎么就要死了呢。此时红线趴在她的背上,双手抱着她的肩膀,用舌头去舔她的发际,所答非所问地说道:你是甜的哎。然后又鼓励她道:就这么甜甜的死掉,有什么不好。那个女人因此说道:我倒宁愿苦上一些。红线又把鼻子伸到她的背上,就如把鼻子伸进了一个熟透的木瓜,或是波萝蜜的深处。她不禁赞叹道:很好闻。那个女刺客说:她倒宁愿难闻一些。最后,女刺客终于转过半个身子,朝红线抱怨道:你干吗要杀掉我!红线皱皱鼻子,冷静地答道:谁让你来行刺──这怪不得我。那女人因此低下头来,她也觉得这话不该说。 

4 

在这个女刺客被红线逮住的事情上,我恐怕没有穷尽一切可能性。这个女人的身体的质地像是一种水果。也许可以说,她像一个白兰瓜,但这种甜瓜在白里透一点绿,或是一点黄色;但她的身体如前所述,是在白色里面透一点玫瑰色。找不出一种瓜果来和她配对──应该承认自己在农业方面的浅薄。红线看着她的身体,总觉得把她一刀杀掉之后不会流出血来,只会流出一种香喷喷的、无色透明的液体。因此她对杀掉这位朋友感到无限的快意。顺便说一句,那个女刺客觉得大家既然是朋友,就没有什么不该说的话,所以总在转弯抹角地求红线放了她。后来,红线觉得不好意思直接推托,就找了个借口道:这家里我作不了主。这样吧,等会儿薛嵩回来你去求他。我也可以帮你说说……那女人听后几乎跳了起来,带着深恶痛绝的态度说:求他?求一个男人?那还不如死了的好!这个腔调像个女权主义者。在唐朝,每个女人都是女权主义者。不但这位女刺客是女权主义者,红线也是女权主义者,她对这位被擒的刺客抱着一种姐妹情谊。但她还是觉得刺客应该被杀掉,不该被饶恕。她还觉得杀掉刺客,免得她再去杀人,也是为她好。 

第三节 

傍晚,薛嵩回家时,看到那个女刺客心定气闲的等待死亡,她真是惊人的美。此时只有一件事可干,就是把她带出去杀掉;薛嵩也这样做了。那女人在引颈就戮时,处处表现了尊严与优美。这使薛嵩赞叹不已。虽然她砍掉了他半个耳朵,但他决定不抱怨什么。但是薛嵩看到的事件是片面的,还有很多内情他没看见。红线看见了那些内情,但她决定忘掉这些事──记住朋友的短处是不好的。比方说,下午时那个女人曾喋喋不休地说道:她觉得自己有种冲动,一见到薛嵩就要朝他跪拜,苦苦哀求他饶她一命。当然,她也明白向男人跪拜、哀求饶命都是不可能的事情,但她真不知怎样才能抑制这种冲动。而红线把头从她肩后探出来,注视着那女人的胸前。她觉得她的乳房好看,就指着它们说:能让我摸摸吗?刺客答道:怎么不可以,反正我要死了……总而言之,那女人在为死而焦虑着,红线却一点都不焦虑。那女人发现红线心不在焉,就说:你怎么搞的!一点忙都不帮吗?红线把手从她胸前撤了回来,说道:我能做点什么?噢!我去给你烧点姜汤水。说着就要离去。这使刺客发起了漂亮女人的小脾气:喂!你一点主意都不出吗?根据我近日的观察,越漂亮的女人越会朝别人要主意,而我在出主意方面是很糟糕的。红线听了这句抱怨,转过身来,吐吐舌头说:没有办法,我岁数小嘛。然后她就去烧姜汤了。 

就我所知,红线不是那种对朋友漠不关心的人。在烧水时,她替刺客认真的考虑了一阵,就带着主意回来了,这主意是这样的:你可以在笼子里住上一段时间,等到不怕了再杀你──不过不能长了,这笼子是我有用的……那女人看了看身后那具棕绿色的囚笼,又看看红线那张嘻笑的小脸,明白了这是对她怯懦的迁就,除了拒绝别无出路了。这就是说,除死之外,别无出路……于是,她跪了起来,摆正了姿式,坐在自己腿上,把手枷放在大腿上挺直了身体,说道:我明白了。就在今天晚上杀吧。不过,这两块木板可真够讨厌的,杀的时候可得解下来。红线马上答道:没有问题。没有问题。她为她高兴,因为她决定了从容赴死,所以恢复了尊严。 

如前所述,那女人被杀时没有披枷带锁,只是被反拴着双手。这是她自己的选择。红线说,等薛嵩回来,我们就是两个人。两个对一个,谅你跑不掉。可以不捆你的手。那女人想了一下说:捆着吧。不然有点滑稽。她是被一刀杀掉的,红线建议用酷刑虐杀她,还觉得这样会有意思,但她皱了皱眉头说:我不喜欢。这主意又被否定了。当晚薛嵩揪着她的头发,红线砍掉了她的头。这也是她自己的选择。红线自己对揪头发有兴趣,想让薛嵩来砍头,但那女人说:我喜欢你来砍;这件事就这样定下来了。红线不想把她的头吊上树梢;但那女人说:别人都要枭首示众,我也不想例外。一切事情都是这样定的,因为那女人对一切问题都有了自己的主意。最后,红线建议她在脖子上戴个花环,园里有很好的花,那女人说:不戴,砍头时戴花,太庸俗,这件事就这样定下来了。 

晚上,薄雾降临时,听到有人从寨外归来,她对红线说:拿篾条来捆手吧──可不要薛嵩用过的。红线就奔去找篾条。回来的时候,红线有点伤感地说:才认识了,又要分手……要不过上一夜,明早上杀你?早上空气好啊。对于这个提议,她倒是没有简单的拒绝,而是从眼睛里浮起了笑意:来摸摸我的腿。红线在她美丽的大腿上摸了一把,发现温凉如玉──换言之,她体温很低。那女人解释道:我已经准备好了。不想重新准备。于是,红线给她卸开手上的木枷,她闭上了眼睛;坦然承认道:整整一天,她都在研究怎样开这个木枷,但没有研究出来;现在看到怎么开,就会心生懊悔。然后她睁开眼睛,对红线说:我很喜欢你。红线说:我能抱抱你吗?那女人狡黠地一笑,说:别抱,你要倒霉的;就转过身去,让红线拴住她的手。就在薛嵩走进院子时,她让红线打开了她的足枷。就这样,除了杀死她之外,什么都没给薛嵩剩下。 

很可惜,这两个朋友走向刑场时,却不是并肩走着。红线走在后面,右手擎着刀,刀头放在肩上;左手推着那女人的肩膀──左肩或右肩──给她指引方向。因为友谊,她没有用手掌去推,觉得那样不礼貌。她只是用指尖轻轻一触。红线说:别想跑啊,这地方我比你熟──这意思是说,她跑不掉。那女人侧着头,躲开自己的散发说:怎么会?我不想失掉你的友谊。她还说,你还保持着警惕,我很喜欢这一点。除了是朋友,她们还是敌人,在这些小事上露出蛛丝马迹。到了地方以后,刺客往地上看了看。这是一片长着青苔的泥地。红线猛然觉得不妥,想去找个垫子来。那女人却说:没有关系,就跪在地下。一般来说,跪着有损尊严,但杀头时例外。这时是为了杀着方便。倘若硬撑着不跪,反倒没有尊严了。 

在死之将至时,刺客和红线还谈了点别的。有关男人,刺客是这样说的:男人热烘烘的,有点臭味。有时候喜欢,有时候不喜欢。后来红线时常想起这句话来,觉得很精辟。有关性,前者的评论是:简单的好,花哨的不好,这和死是一样的。这使红线的观念受到了冲击,想到自己期待着被薛嵩打晕,坐在高楼一样的囚车里驶入凤凰寨,也有花哨的嫌疑。有关女同性恋,刺客说:有点感觉,但我不是。红线马上觉得自己也不是同性恋者。有关薛嵩,她说:看上去还可以。红线对这个评价很满意。有关谁派她来杀薛嵩,刺客说:这不能说。红线想,她答得对,当然不能说。总而言之,这都是红线关心的问题,她一一做了解答。她还说:同样一件事,在我看来叫作死,在你看来叫作杀,很有意思。很高兴和你是朋友。杀吧。此时她跪在地下,伸长了脖子,红线擎着刀。红线虽然觉得还没有聊够,但只好杀。杀过之后,自然就没有可聊的了。 

2 

对以上故事,又可以重述如下:那个女人,也就是那个刺客,潜入凤凰寨里要杀薛嵩,被红线打晕逮住了。刺客被擒之后,总是要被杀掉的。对于这件事,开始她很害怕,后来又不怕了。怕的时候她想:我才二十二岁,就要死掉了。后来她又想:这是别人要杀我呀;所以就不怕。但她依旧要为此事张罗,出主意,做决定。举例来说,她背过身去,让红线用竹篾条拴她的手,此时红线曾有片刻的犹豫,不知怎样拴更好。那女人的身体表面,有一种新鲜瓜果般的光滑,红线不知怎样把竹篾条勒上去。她就出主意道:先在腰上勒一道,然后把手拴在上面;来,我做给你看。说着她就转过身去,但红线异常灵活地退后了很远,摆了个姿式,像一只警惕的猫;紧张得透不过气来,小声说道:别骗我呀──假如红线不退后,她就要把红线拴住了。 

那女人的计谋没有成功。后来,她只好惨然一笑,又转了回来,背着手说:好吧,不骗你。来捆吧。于是红线回来,把她捆住。就按她说的那种捆法,只是捆得异常仔细:不但把两只手腕捆在一起,还把两个大拇指捆在一起。她还想把每对手指都捆在一起,但那女人苦笑着说:这样就可以了吧?再仔细就不像朋友了。红线觉得她说得对,就仔细打了个扣,结束了这项工作。然后她退后了几步,看到细篾条正陷入刺客的腰际,就说:你现在像个男人了。这意思是说,从侧后看,她像个用篾条吊起龟头的男人。那女人明白了这个意思,侧过头来惨然说道:不要拿我开玩笑啊,这样不好。想到这女人就要被杀掉,红线也惨然了一阵,然后又高兴起来──她毕竟是个孩子嘛。 

后来,红线转到那女人身前,端详着她浅玫瑰色的身体。在这个身体上,红线最喜欢腹部,因为小腹是平坦的,肚脐眼是纵的椭圆,其中坦坦荡荡地凸起了一些,像小孩子的肚脐。红线走上前去,把手放在上面,然后又谨慎地退开,说道:好看。那女人说:也就是现在好看。再过一些年就不会好看。然后她又补充道:当然,我也不能再过一些年了。此时她神色黯然。但在黯然的神色下面,她还在寻找红线的破绽。红线忽然说道:你跪下好不好?我也安全些。那女人往后挪了几下,向前跪下来;然后勉强笑笑说:呆会儿你可得扶我起来啊──其实她在跪下之前就知道这是个狡猾的陷阱。因为脚上有一具木枷并被反拴着手,跪下就难以重新站起来,因而再没有逃走的机会。其实,红线也没有给过她这种机会,不然她已经跑了。有一瞬间,她感到很悲惨,几乎想向红线抱怨。但她最终决定了不抱怨。红线说,她要找几个熟透的樱桃给她吃,就离去了。她独自在院子里,坐在自己腿上,开始感觉到绝望。然而她最终却发现,绝望其实是无限的美好。 

“绝望是无限的美好”,这句话引起我的深思。我可能会懂得这句话──如你所知,我失去了记忆,正处于绝望的境界;所以我可能会懂,但还没有懂……红线带着樱桃回来,一粒粒摘去了果梗,放进那个女人嘴里。每一粒她都没有拒绝,然后想把果核吐掉。但红线伸出手来,说:吐在这里。她就把果核吐进红线的掌心。红线把果核丢掉。吃过樱桃以后,这女人又坐在自己的腿上,微微有点心不在焉。而红线在一阵冲动中,在她对面跪下,说道:我想吻吻你。出于旧日的积习,那女人皱了皱眉,感觉自己不喜欢此事。转瞬又发现自己其实是喜欢的。于是她挺直了身体,抿抿嘴唇。红线用双手勾住她的脖子,端详了她一阵,然后把她拉近,开始热吻。此时她们的乳房紧贴在一起,红线发现对方的乳房比自己要坚实,感到很受刺激;但那女人的双唇柔顺,这又让她感到满意。那女人的头微微侧着,起初,目光越过了红线,看着远处。这使红线感到不满意。后来,她的目光又专注于红线,并且露出了笑意。最终红线想道:有满意,有不满意,其实这是最好的;就把她放开。此后那女人甩甩自己的头发,又坐了回去。你可能已经注意到了,她不想说什么。这一点和我是一样的。红线几次想要和她交谈,都碰了壁。后来,她总算给自己找了件事干:磨起刀来。 

新刀的样子是这样的:长方形,见棱见角,装着木制的把,带着锻打时留下的黑色,刀口笔直。但这一把的样子颇为不同,它有一点浑圆,像调色板一类的东西,刀口向下凹去,与新月相似。这是一把旧刀,总在石头上磨,变得像纸一样薄,也没剩什么钢火。它有好处,也有不好处。好处是只要在砂石上蹭几下,就变得飞快。不好处是锋锐难以持久。红线磨刀时,那女人看了她一眼。她就比划了一下说:只砍一下,没有问题。那女人点点头说:噢;就把头转回去。红线觉得她心神恍惚,并没有明白。但她还要磨这把刀:用砂蹭出的刀口有点粗糙,割起来恐怕要疼的。她又用细磨石来磨,直到刀口平滑无损;然后,红线仔细端详着几乎看不到的刀口,想着:用这把刀杀人,对方感到的不是疼痛,而是一片凉爽;就像洒在皮肤上的酒精,或者以太──以太就是 ether,红线要是知道这个名词可就怪了──感到的只是快意。她拿了这把刀走过来,平放在那女人赤裸的肩上,并让烂银似的光芒反射在她脸上,给她带去一缕寒意,然后问道:喜欢吗?这是一个明确无误的表示,说明这就是杀她的刀。红线注意到那女人的目光曾有瞬时的暗淡,但马上又明亮了过来。她也明确无误地答道:喜欢。 

红线在苗寨里住着时,那里杀人。被杀者神情激动,面红耳赤,肢体僵硬,每根神经和肌肉都已绷紧。每个人都大声说话,虽然说的是什么难以听懂;他们都又撑又拒,有人是和别人撑拒,有人是和自己撑拒。假如是杀头的话,让他们跪下来可不容易,而且每个人都要站着撒一泡热辣辣的尿,在这方面男人和女人颇有不同,但总能看出是做了同一件事。按这个标准来衡量,眼前这个女人颇有差距。她坐在那里,面带微笑,心神恍惚,就像一个人要哼歌时的样子。红线恐怕她已误入歧途,对自己行将被杀一事缺少了解;总想帮她回到正道上来,单没有成功。按照现在的讲法,那刺客没有请红线来摸她的腿,展示她的体温。她什么都没做。直到薛嵩回来,好把她杀掉。死掉之前,她也没有和红线闲聊。因此,这是另一个故事了。在此后的日子里,红线经常怀念这个女人:她在她手里时,起初是个被俘的敌人,也是朋友。那时她不能接受被杀一事,总想逃掉。后来她接受了这件事,就既不是朋友,也不是敌人,也不想逃掉,变成了一个陌生人。而一想起这个陌生人,红线就感到热辣辣的性欲,而且想撒尿。 

3 

现在我想到,不提那刺客被杀的经过总是一种缺失,虽然这件事没有什么可讲的。在林荫里,那个陌生的女人跪在地下,伸直了脖子,颈椎的骨节清晰可见。红线一刀砍了下去,那把薄薄的旧刀不负红线的厚望,切过了骨节中的缝隙,把人头和身体分开。此后,人头拎在薛嵩的手上,身体则向前扑倒,变成了两样东西。身体的目标较大,吸引了红线的注意。它俯卧在地下,双肩上耸,被反绑着的双手攥成拳头,猛烈地下撑,把那根竹篾条拉得像紧绷的弓弦也似。与此同时,一股玫瑰色的液体,带着心脏的搏动从腔子里冲了出来,周围充满了柚子花的香味。当然,也有点辛辣的气味,因为这毕竟是血。这是血带有稀油般的渗性,流到地上马上就消失了,只留下几乎看不出的痕迹,等到血流完以后,那个身体(更准确地说,是脊背和背着的双手)好像叹了一口气一样,松弛了下来;双肩下颓,手也收回,交叉作X形,手指也向后张开。它微微屈起一条腿,就这样静止住。红线立刻上前,解开了竹篾条,因为人既死了,就用不着约束。而在此之前,她的这位朋友一直在她巧妙的约束之中。在这一瞬间,红线回想起她在她手里吃樱桃,觉得这件事非常之好──我很怀疑这样写有滥情的嫌疑,但既然已经写出来,也无从反悔──然后,死者的双手就滑落到身体的两侧,并半握成拳。她把这身体翻了过来。这身体的正面异常安详,似有一股温和的气氛扑面而来。这身体好像有呼吸,但其实是没有的。只是凸起的肚脐以自动武器连发的速度在跳动。红线觉得它以这种方式来承认自己已经死去,于是,就像台湾人说的那样,觉得“它好乖呀”。 

然后,红线把那身体扶坐起来,感到它很柔软,关节也很灵活,简直是在追随她的动作。她又扶它站了起来,搀着它走向一个早已掘好的坑。这时红线觉得有人在身后叫她,回头一看,只见那颗人头提在薛嵩手里,瞪大了双眼,正专注地看着她们(含无头身体)。红线忍心地回过头去,搀着身体继续走,并不无道理地想:我也不能两头都顾啊。她把身体扶到坑底坐下,然后又让它躺好,然后捧起又湿又糯的黑色泥土,要把它埋葬。才埋了脚,她就觉得不妥,顺手抓住了一只草蜢,用草叶绑住,丢在坑里给身体陪葬。才埋住这只草蜢,她又觉得不妥当,就从坑里爬了出来,去找她的另一个朋友,也就是前面提到的小妓女要一张蒲草的席子,想给尸体盖在身上。所以她要从薛嵩身边经过,而那个人头始终在专注地看着她。红线想假作不知地走过,但第三次觉得不妥当。于是她转过身,看那颗人头。那人头朝她一笑,很俏皮,还皱了皱鼻子,伸出舌头舔舔嘴唇。红线知道它在招她过去。她有点不乐意。Anyway,这人可是她杀的呀。 

我像一支破枪一样走了火,冒出一个“Anyway”来。现在只好扔下笔,到字典上查它的意思。查到以后才知道,这个词我早就认识。我越来越像破枪,走火也成了常事。红线站在人头面前,看到它把湿润的双唇耸起,就知道它想让她吻它。这一回她有点不喜欢:不管怎么说,你可是死了的呀。但这念头一出现,人头就往下撇嘴,露出了要哭的意思。这使红线别无选择(毕竟是朋友嘛),把泥手往自己背上擦了擦,捧住它的后脑(这时她发现,这位朋友变得轻飘飘的了),吻它的双唇。这样做其实并无不适之处,因为这双唇比以前还温柔了很多。那双眼睛就在面前,它先往下看,看清了红线的面颊,又和红线短暂的对视,然后往上看,看红线的眉毛。最后转回来,满眼都是笑意;既快乐,又顽皮;但红线觉得很要命。她支持了一会儿,才把人头放开:先把她推开,然后放下去。这两个动作都是小心翼翼的,尽量轻柔、准确,把它放置在头发的悬挂之下;然后放开手,人头没有丝毫的摇晃。对方舔了舔嘴,笑了一笑,又眨眨眼。红线明白她在表示感谢。红线不禁想到:这颗人头与它被杀下来前相比,更性感、更甜蜜;其实她更加喜欢它;然后就赶紧不想──但已经想过了。其实红线还有正事要做──埋掉那个身体。但在人头的依依不舍面前,总是犹豫不定。最后她终于下定了决心,留下来陪它──我指的是人头,不是身体。这个故事的寓意是:不要杀朋友,杀成两块你忙不过来。但这故事本身并无寓意。 

在那女人被杀时,薛嵩表现得木木痴痴,他只顾偷看人家的身体,特别是羞处,还很不要脸地勃起过几次。这使红线觉得很是丢脸,好在被杀的人并不在意。然后,这个男人用绳子拴住了人头的头发,要把它升起来,它却目不转睛地注视着红线,露出了乞求的神色。红线明白她的意思,她想让红线带着它,和它朝夕相处,起卧相随。事情是这样的:那位女刺客在被红线杀掉之前,只把红线当做朋友。到了被杀之后,就真正爱上她了。 

红线实在不喜欢这个主意,也不喜欢被人头爱上,就假装不明白,把这个想法拒之门外。当那颗人头升起来时,满脸都是凄婉的神色。红线硬下心来,举手行礼,目送它升入高空。然后就跑回那个土坑里。就是这短短的几分钟,死尸的脖子上已经爬了一圈蚂蚁。她赶紧把它埋掉,顾不上找草席来盖了。然后她又回来,站在树下看那颗人头。此时林间已经相当幽暗,但树顶上还比较亮,那人头用期待的目光看着她。而红线硬下心来想到:我今天逮住了她,看守了她,把她杀掉,又埋了。而我只是个小孩子,总得干点别的事,比方说,去玩……所以她觉得自己此时没有爬上树梢去陪这位朋友,也满说得过去。但红线毕竟是善良的,她决定另找时间来陪这个朋友。但后来发生的事情很多,把她绊住了。 

顺便说说,上次杀掉自己的邻居之后,红线也曾回去过,发现在闷热的林子里,那个人的一切都变成了深棕色,除了那对哆出来的眼珠子。那两个东西离开了眼眶,东歪西倒地挂着,依然是黑白分明的样子。其它的东西,包括原来鲜红的肠子,都变得像土一样,悬在空中,显得很不结实。几棵新竹穿过他的肚子,朝天上长着;还有几只捕鸟的大蜘蛛,在他的框架之内结了网。那地方有股很难闻的味儿。红线闭着气,在那里呆了一会儿。后来,她觉得自己要憋死了,对自己表现出的善良感到满意,就转身离开了那地方。 

4 

现在我发现,这个故事最大的缺失是没有提到那女人的内心。我总觉得这是不言自明的,其实却远不是这样。被反绑着跪在地下时,她终于明白自己这回是死定了。至此,她一生的斗争都已结束,只剩下死。她也可以喜欢这件事,也可以不喜欢这件事。她决定喜欢这件事:对于无法逃避的事,喜欢总比不喜欢要好一些。 

此后她就变得轻松,甚至是快乐起来。站在行将死去的人面前,会感到一团好意迎面而来。红线常参加杀人,对这种感觉很熟悉。比方说,上次那个邻居被拉成一张牌桌时,就说:红线,我家里有一张角弓,要就拿去。红线很高兴,说道:谢谢!我会怀念你!打掉一张红心A。等他被拉成一张床框时,红线又到了他面前。这时他嘴里爬了好多蚂蚁,正在吃他的舌头,所以他含混不清地说:我有一把铜鞘的小刀,要就拿去。红线也说:谢谢。随着时间的推移,好意和臭味日重。最后一次他说:想要什么只管拿,别来了,会得病的。但红线毕竟是善良的,还常去看他,直到他变成土为止。这个女刺客也是这样的,漂亮的乳房也好,好看的肚脐也罢,要什么只管拿去。可惜的是,这些东西都拿不走,只能摸摸弄弄。这就是问题的所在。红线摸过了那个美丽的身体,咂咂嘴,就满意了;一刀把她的头颅砍了下来。而薛嵩没有触及这个身体,只是看到她的身体和眉梢眼旁的笑意,感到了她的好意,就受到很大的触动。作为一个思路慎密的人,他马上就想到自己所做的一切都错了。与其用枷锁去控制人的身体,不如去控制她的内心。这才是问题之真正所在。 

如前所述,红线和那小妓女是朋友。所以,杀掉了另一个朋友之后,她来到小妓女的家里,并排躺在地板上,抽随手采来、在枕头下风干的大麻烟,并且胡聊一通。此时红线总要说到那辆柚木囚车,谈到里面状似残酷,实则温柔的陈设;还谈到那些巧夺天工的枷锁。当然,谈得最多的是,在未来的某一天,她会被套上这些枷锁,关进囚笼,成为永远的囚徒和家庭主妇,终身和那些柚木为伍,就再也出不来了。在此之前,她要做的是监督薛嵩把周到、细致、温柔和严酷都做到极致,在此之后,她就要享受这些周到、细致、温柔和严酷。 

举例来说,身为家庭主妇,要管理果园和菜地,所以那辆囚车就有一套自动机构,可以越野行驶。红线在笼子里,透过栅栏,操作着一根长杆,杆顶有一个小小的锄头,可以除去采地里的一棵野草,但不致伤到一棵邻近的采苗。考虑到距离很远,红线手上有枷,不那么灵便,这条长杆自然是装在一个灵巧的支架上。听她说的意思,我觉得这好像是雅马哈公司出品的某种钓鱼杆。但她又说,另一根长杆可以装上一个小纱网和一把小剪子,伸到树上,剪下一个熟透的芒果。总而言之,红线把自己形容成一个斯诺克台球的高手。另一方面,你当然也想到了,这座囚车又是一辆旅行车。它可以准确地行驶在采畦里,把车下废水箱里的东西(也就是红线自己的屎和尿)施到地里做肥料。红线还说,这些都不是这辆囚车的主题。主题是只有薛嵩可以进那辆车,带去周到、细致、温柔和残酷的性爱。所以,薛嵩的性爱才是这辆车的主题。因为薛嵩是如此慎密、苦心孤诣,红线才会住进这辆车。那个小妓女对这个故事不大喜欢,想要给红线泼点凉水,就说:恐怕那车没有你说的那么好。而红线吐了一个烟圈,很潇洒地说道:放心吧,不好我就不进去。我的后脑勺也不是那么容易打的──此时杀人时的感觉还没从红线身上退去。红线隐隐地感到,她对那个女刺客所做的一切,远远不能说把周到、细致、温柔和残酷都做到了极致。但她把这归咎于已死的女刺客;仿佛是说:谁让你被我打晕了。 

现在轮到小妓女来炫耀自己,她只能把寨子里的男人说一说:某某和我好;我和某某做爱,快乐极了;等等。在这些男人里,她特别提到了薛嵩,一面说,一面偷看红线的脸色。但红线无动于衷。时至今日,红线还没和薛嵩做过爱,这使小妓女感到特别得意。但她也知道,一大筐烂桃也敌不上一个好桃。没有人对她这样慎密、这样苦心孤诣,大家都是玩玩,玩过就算了。她因此而骄傲,甚至仇恨;但还不至于找人来把薛嵩杀掉。这是因为她很年轻,保持着善良的天性。假如年龄再大一些就难保了。然后,这两个朋友有一些亲热的举动,在此不便描写。 

红线对小妓女说,遇上薛嵩,我已经死定了。说这话时,她已经坐了起来,抽着另一支大麻烟。此时她眉梢眼尾都是笑意,就和那被砍头的女刺客相似。那个小妓女说:我真不明白,死定了有什么好。也许红线应该解释说:虽然已经死定了,但不会马上死;或者解释说:这种死和那种死不同;或者解释说:这是个比方嘛。但她什么都不解释,手指一弹,把烟蒂弹到了门外;然后自己也走了出去;只是在出门时轻描淡写地说了一句:这个你不懂。于是那小妓女嫉妒得要发狂,因为自己没有死定。这个小小的例子使我想到,穷尽一切可能性和一种可能都没有一样,都会使你落个一头雾水。 

后来,那女刺客的头就像一朵被剪下的睡莲花那样,在树端逐渐枯萎。莲花枯萎时,花瓣的边缘首先变成褐色,人头也是那样。她的面颊上起了很多黄褐色的斑点,很像是老年斑。当然,假如把斑点扣除在外,还是满好看。说实在的,她正在腐烂,发出烂水果那种甜得发腥的味道。但为了不让朋友伤心,红线照常吻她。人头每次见到红线,总要皱皱眉头,咪起嘴来说一个字,从口形来看,是个“埋”字。红线知道她的意思,她要红线把她埋掉。在这方面,红线实在是爱莫能助。因为只有薛嵩是此地的主人,他说了才能算。于是她硬起心来,假装没有听明白,爬下树去了。这是因为薛嵩在树下练习箭法,红线要去陪他。 

现在,薛嵩丢下了手上的木工活,在那棵挂着人头的树上刻了一颗红心,每天用长箭去射它。在红线看来,这应该是一个象征。但她怎么也想不出这象征的是什么。也许,这颗心象征着自己,箭象征着薛嵩的爱情。也许,这颗心象征着自己的那话儿,箭则象征着薛嵩的那话儿。不管象征着什么,反正红线被他的举动给迷住了。她站在薛嵩身边,从箭壶里取箭给他,态度越来越恭敬。起初是用一只手递箭给他,后来用两只手递箭给他。再后来,她屈下一条腿,把双手捧过头顶。在这个故事里,薛嵩没有用繁文絮节去约束红线。他用双手把她魇住了。这也是我的选择。拿枷锁和一种没落的文化相比,我更喜欢枷锁。而那位白衣女人读完了这个故事,怒目圆睁,朝我怒吼一声:瞎编什么呀你!

